\documentclass[11pt,a4paper]{article}
\usepackage[T1]{fontenc}
\usepackage{isabelle,isabellesym}

% further packages required for unusual symbols (see also
% isabellesym.sty), use only when needed

%\usepackage{amssymb}
  %for \<leadsto>, \<box>, \<diamond>, \<sqsupset>, \<mho>, \<Join>,
  %\<lhd>, \<lesssim>, \<greatersim>, \<lessapprox>, \<greaterapprox>,
  %\<triangleq>, \<yen>, \<lozenge>

%\usepackage{eurosym}
  %for \<euro>

%\usepackage[only,bigsqcap,bigparallel,fatsemi,interleave,sslash]{stmaryrd}
  %for \<Sqinter>, \<Parallel>, \<Zsemi>, \<Parallel>, \<sslash>

%\usepackage{eufrak}
  %for \<AA> ... \<ZZ>, \<aa> ... \<zz> (also included in amssymb)

%\usepackage{textcomp}
  %for \<onequarter>, \<onehalf>, \<threequarters>, \<degree>, \<cent>,
  %\<currency>

\usepackage{amsmath}
\usepackage{float}

% this should be the last package used
\usepackage{pdfsetup}

% urls in roman style, theory text in math-similar italics
\urlstyle{rm}
\isabellestyle{it}

% for uniform font size
%\renewcommand{\isastyle}{\isastyleminor}

\begin{document}

\title{Formalization of "Knight's Tour Revisited"}
\author{Lukas Koller}
\maketitle

\begin{abstract}
This is a formalization of \cite{cull_decurtins_1987}. In \cite{cull_decurtins_1987}
the existence of Knight's paths and Knight's circuits are proved for arbitrary $n\times m$-boards with 
$\operatorname{min}(n,m) \geq 5$ and for the Knight's circuit $n\cdot m$ is even.

A Knight's path is a sequence of squares on a chessboard s.t. every step in sequence is a valid move for 
a Knight. A Knight is a chess figure that is only able to move two squares vertically and one square 
horizontally or two squares horizontally and one square vertically. Finding a Knight's path is an 
instance of the Hamiltonian Path Problem. A Knight's circuit is a Knight's path, where additionally 
the Knight can move from the last square to the first square of the path, forming a loop.

The main idea for the proof of the existence of a Knight's path on a $n\times m$-board, is to inductivly 
construct a Knight's paths for the $n\times m$-board from a few pre-computed Knight's paths for small 
boards, i.e. $5\times 5$, $5\times 6$, ..., $8\times 9$. The paths for small boards are transformed 
(i.e. transpose, mirror, translate) and concatenated to create paths for larger boards.

While formalizing the proofs I have noticed two mistakes in the original proof in \cite{cull_decurtins_1987}: 
(i) the pre-computed path for the $6\times 6$-board that ends in the upper-left (in Figure 2) and (ii) 
the pre-computed path for the $8\times 8$-board that ends in the upper-left (in Figure 5) are incorrect. 
I.e. on the $6\times 6$-board the Knight cannot step from square 26 to square 27; in the $8\times 8$-board 
the Knight cannot step from square 27 to square 28. In this formalization I have replaced the two incorrect 
paths with correct paths.
\end{abstract}

\tableofcontents

% sane default for proof documents
\parindent 0pt\parskip 0.5ex

% generated text of all theories
%
\begin{isabellebody}%
\setisabellecontext{KnightsTour}%
%
\isadelimtheory
\isanewline
\isanewline
%
\endisadelimtheory
%
\isatagtheory
\isacommand{theory}\isamarkupfalse%
\ KnightsTour\isanewline
\ \ \isakeyword{imports}\ Main\isanewline
\isakeyword{begin}%
\endisatagtheory
{\isafoldtheory}%
%
\isadelimtheory
%
\endisadelimtheory
%
\isadelimdocument
%
\endisadelimdocument
%
\isatagdocument
%
\isamarkupsection{Introduction and Definitions%
}
\isamarkuptrue%
%
\endisatagdocument
{\isafolddocument}%
%
\isadelimdocument
%
\endisadelimdocument
%
\begin{isamarkuptext}%
A Knight's path is a sequence of moves on a chessboard s.t. every step in sequence is a 
valid move for a Knight and that the Knight visits every square on the boards exactly once. 
A Knight is a chess figure that is only able to move two squares vertically and one square 
horizontally or two squares horizontally and one square vertically. Finding a Knight's path is an 
instance of the Hamiltonian Path Problem. A Knight's circuit is a Knight's path, where additionally 
the Knight can move from the last square to the first square of the path, forming a loop.

\cite{cull_decurtins_1987} proves the existence of a Knight's path on a \isa{n{\isasymtimes}m}-board for
sufficiently large \isa{n} and \isa{m}. The main idea for the proof is to inductivly construct a Knight's 
path for the \isa{n{\isasymtimes}m}-board from a few pre-computed Knight's paths for small boards, i.e. \isa{{\isadigit{5}}{\isasymtimes}{\isadigit{5}}}, 
\isa{{\isadigit{8}}{\isasymtimes}{\isadigit{6}}}, ..., \isa{{\isadigit{8}}{\isasymtimes}{\isadigit{9}}}. The paths for small boards are transformed (i.e. transpose, mirror, translate) 
and concatenated to create paths for larger boards.

While formalizing the proofs I discovered two mistakes in the original proof in 
\cite{cull_decurtins_1987}: (i) the pre-computed path for the \isa{{\isadigit{6}}{\isasymtimes}{\isadigit{6}}}-board that ends in 
the upper-left (in Figure 2) and (ii) the pre-computed path for the \isa{{\isadigit{8}}{\isasymtimes}{\isadigit{8}}}-board that ends in 
the upper-left (in Figure 5) are incorrect. I.e. on the \isa{{\isadigit{6}}{\isasymtimes}{\isadigit{6}}}-board the Knight cannot step 
from square 26 to square 27; in the \isa{{\isadigit{8}}{\isasymtimes}{\isadigit{8}}}-board the Knight cannot step from square 27 to 
square 28. In this formalization I have replaced the two incorrect paths with correct paths.%
\end{isamarkuptext}\isamarkuptrue%
%
\begin{isamarkuptext}%
A square on a board is identified by its coordinates.%
\end{isamarkuptext}\isamarkuptrue%
\isacommand{type{\isacharunderscore}{\kern0pt}synonym}\isamarkupfalse%
\ square\ {\isacharequal}{\kern0pt}\ {\isachardoublequoteopen}int\ {\isasymtimes}\ int{\isachardoublequoteclose}%
\begin{isamarkuptext}%
A board is represented as a set of squares. Note, that this allows boards to have an 
arbitrary shape and do not necessarily need to be rectangular.%
\end{isamarkuptext}\isamarkuptrue%
\isacommand{type{\isacharunderscore}{\kern0pt}synonym}\isamarkupfalse%
\ board\ {\isacharequal}{\kern0pt}\ {\isachardoublequoteopen}square\ set{\isachardoublequoteclose}%
\begin{isamarkuptext}%
A (rectangular) \isa{{\isacharparenleft}{\kern0pt}n{\isasymtimes}m{\isacharparenright}{\kern0pt}}-board is the set of all squares \isa{{\isacharparenleft}{\kern0pt}i{\isacharcomma}{\kern0pt}j{\isacharparenright}{\kern0pt}} where \isa{{\isadigit{1}}\ {\isasymle}\ i\ {\isasymle}\ n} 
and \isa{{\isadigit{1}}\ {\isasymle}\ j\ {\isasymle}\ m}. \isa{{\isacharparenleft}{\kern0pt}{\isadigit{1}}{\isacharcomma}{\kern0pt}{\isadigit{1}}{\isacharparenright}{\kern0pt}} is the lower-left corner, and \isa{{\isacharparenleft}{\kern0pt}n{\isacharcomma}{\kern0pt}m{\isacharparenright}{\kern0pt}} is the upper-right corner.%
\end{isamarkuptext}\isamarkuptrue%
\isacommand{definition}\isamarkupfalse%
\ board\ {\isacharcolon}{\kern0pt}{\isacharcolon}{\kern0pt}\ {\isachardoublequoteopen}nat\ {\isasymRightarrow}\ nat\ {\isasymRightarrow}\ board{\isachardoublequoteclose}\ \isakeyword{where}\isanewline
\ \ {\isachardoublequoteopen}board\ n\ m\ {\isacharequal}{\kern0pt}\ {\isacharbraceleft}{\kern0pt}{\isacharparenleft}{\kern0pt}i{\isacharcomma}{\kern0pt}j{\isacharparenright}{\kern0pt}\ {\isacharbar}{\kern0pt}i\ j{\isachardot}{\kern0pt}\ {\isadigit{1}}\ {\isasymle}\ i\ {\isasymand}\ i\ {\isasymle}\ int\ n\ {\isasymand}\ {\isadigit{1}}\ {\isasymle}\ j\ {\isasymand}\ j\ {\isasymle}\ int\ m{\isacharbraceright}{\kern0pt}{\isachardoublequoteclose}%
\begin{isamarkuptext}%
A path is a sequence of steps on a board. A path is represented by the list of visited 
squares on the board. Each square on the \isa{{\isacharparenleft}{\kern0pt}n{\isasymtimes}m{\isacharparenright}{\kern0pt}}-board is identified by its coordinates \isa{{\isacharparenleft}{\kern0pt}i{\isacharcomma}{\kern0pt}j{\isacharparenright}{\kern0pt}}.%
\end{isamarkuptext}\isamarkuptrue%
\isacommand{type{\isacharunderscore}{\kern0pt}synonym}\isamarkupfalse%
\ path\ {\isacharequal}{\kern0pt}\ {\isachardoublequoteopen}square\ list{\isachardoublequoteclose}%
\begin{isamarkuptext}%
A Knight can only move two squares vertically and one square horizontally or two squares 
horizontally and one square vertically. Thus, a knight at position \isa{{\isacharparenleft}{\kern0pt}i{\isacharcomma}{\kern0pt}j{\isacharparenright}{\kern0pt}} can only move 
to \isa{{\isacharparenleft}{\kern0pt}i{\isasymplusminus}{\isadigit{1}}{\isacharcomma}{\kern0pt}j{\isasymplusminus}{\isadigit{2}}{\isacharparenright}{\kern0pt}} or \isa{{\isacharparenleft}{\kern0pt}i{\isasymplusminus}{\isadigit{2}}{\isacharcomma}{\kern0pt}j{\isasymplusminus}{\isadigit{1}}{\isacharparenright}{\kern0pt}}.%
\end{isamarkuptext}\isamarkuptrue%
\isacommand{definition}\isamarkupfalse%
\ valid{\isacharunderscore}{\kern0pt}step\ {\isacharcolon}{\kern0pt}{\isacharcolon}{\kern0pt}\ {\isachardoublequoteopen}square\ {\isasymRightarrow}\ square\ {\isasymRightarrow}\ bool{\isachardoublequoteclose}\ \isakeyword{where}\isanewline
\ \ {\isachardoublequoteopen}valid{\isacharunderscore}{\kern0pt}step\ s\isactrlsub i\ s\isactrlsub j\ {\isasymequiv}\ {\isacharparenleft}{\kern0pt}case\ s\isactrlsub i\ of\ {\isacharparenleft}{\kern0pt}i{\isacharcomma}{\kern0pt}j{\isacharparenright}{\kern0pt}\ {\isasymRightarrow}\ s\isactrlsub j\ {\isasymin}\ {\isacharbraceleft}{\kern0pt}{\isacharparenleft}{\kern0pt}i{\isacharplus}{\kern0pt}{\isadigit{1}}{\isacharcomma}{\kern0pt}j{\isacharplus}{\kern0pt}{\isadigit{2}}{\isacharparenright}{\kern0pt}{\isacharcomma}{\kern0pt}{\isacharparenleft}{\kern0pt}i{\isacharminus}{\kern0pt}{\isadigit{1}}{\isacharcomma}{\kern0pt}j{\isacharplus}{\kern0pt}{\isadigit{2}}{\isacharparenright}{\kern0pt}{\isacharcomma}{\kern0pt}{\isacharparenleft}{\kern0pt}i{\isacharplus}{\kern0pt}{\isadigit{1}}{\isacharcomma}{\kern0pt}j{\isacharminus}{\kern0pt}{\isadigit{2}}{\isacharparenright}{\kern0pt}{\isacharcomma}{\kern0pt}{\isacharparenleft}{\kern0pt}i{\isacharminus}{\kern0pt}{\isadigit{1}}{\isacharcomma}{\kern0pt}j{\isacharminus}{\kern0pt}{\isadigit{2}}{\isacharparenright}{\kern0pt}{\isacharcomma}{\kern0pt}\isanewline
\ \ \ \ \ \ \ \ \ \ \ \ \ \ \ \ \ \ \ \ \ \ \ \ \ \ \ \ \ \ \ \ \ \ \ \ \ \ \ \ \ \ \ \ \ \ \ \ {\isacharparenleft}{\kern0pt}i{\isacharplus}{\kern0pt}{\isadigit{2}}{\isacharcomma}{\kern0pt}j{\isacharplus}{\kern0pt}{\isadigit{1}}{\isacharparenright}{\kern0pt}{\isacharcomma}{\kern0pt}{\isacharparenleft}{\kern0pt}i{\isacharminus}{\kern0pt}{\isadigit{2}}{\isacharcomma}{\kern0pt}j{\isacharplus}{\kern0pt}{\isadigit{1}}{\isacharparenright}{\kern0pt}{\isacharcomma}{\kern0pt}{\isacharparenleft}{\kern0pt}i{\isacharplus}{\kern0pt}{\isadigit{2}}{\isacharcomma}{\kern0pt}j{\isacharminus}{\kern0pt}{\isadigit{1}}{\isacharparenright}{\kern0pt}{\isacharcomma}{\kern0pt}{\isacharparenleft}{\kern0pt}i{\isacharminus}{\kern0pt}{\isadigit{2}}{\isacharcomma}{\kern0pt}j{\isacharminus}{\kern0pt}{\isadigit{1}}{\isacharparenright}{\kern0pt}{\isacharbraceright}{\kern0pt}{\isacharparenright}{\kern0pt}{\isachardoublequoteclose}%
\begin{isamarkuptext}%
Now we define an inductive predicate that characterizes a Knight's path. A square \isa{s\isactrlsub i} can be
pre-pended to a current Knight's path \isa{s\isactrlsub j{\isacharhash}{\kern0pt}ps} if (i) there is a valid step from the square \isa{s\isactrlsub i} to 
the first square \isa{s\isactrlsub j} of the current path and (ii) the square \isa{s\isactrlsub i} has not been visited yet.%
\end{isamarkuptext}\isamarkuptrue%
\isacommand{inductive}\isamarkupfalse%
\ knights{\isacharunderscore}{\kern0pt}path\ {\isacharcolon}{\kern0pt}{\isacharcolon}{\kern0pt}\ {\isachardoublequoteopen}board\ {\isasymRightarrow}\ path\ {\isasymRightarrow}\ bool{\isachardoublequoteclose}\ \isakeyword{where}\isanewline
\ \ {\isachardoublequoteopen}knights{\isacharunderscore}{\kern0pt}path\ {\isacharbraceleft}{\kern0pt}s\isactrlsub i{\isacharbraceright}{\kern0pt}\ {\isacharbrackleft}{\kern0pt}s\isactrlsub i{\isacharbrackright}{\kern0pt}{\isachardoublequoteclose}\isanewline
{\isacharbar}{\kern0pt}\ {\isachardoublequoteopen}s\isactrlsub i\ {\isasymnotin}\ b\ {\isasymLongrightarrow}\ valid{\isacharunderscore}{\kern0pt}step\ s\isactrlsub i\ s\isactrlsub j\ {\isasymLongrightarrow}\ knights{\isacharunderscore}{\kern0pt}path\ b\ {\isacharparenleft}{\kern0pt}s\isactrlsub j{\isacharhash}{\kern0pt}ps{\isacharparenright}{\kern0pt}\ {\isasymLongrightarrow}\ knights{\isacharunderscore}{\kern0pt}path\ {\isacharparenleft}{\kern0pt}b\ {\isasymunion}\ {\isacharbraceleft}{\kern0pt}s\isactrlsub i{\isacharbraceright}{\kern0pt}{\isacharparenright}{\kern0pt}\ {\isacharparenleft}{\kern0pt}s\isactrlsub i{\isacharhash}{\kern0pt}s\isactrlsub j{\isacharhash}{\kern0pt}ps{\isacharparenright}{\kern0pt}{\isachardoublequoteclose}\isanewline
\isanewline
\isacommand{code{\isacharunderscore}{\kern0pt}pred}\isamarkupfalse%
\ knights{\isacharunderscore}{\kern0pt}path%
\isadelimproof
\ %
\endisadelimproof
%
\isatagproof
\isacommand{{\isachardot}{\kern0pt}}\isamarkupfalse%
%
\endisatagproof
{\isafoldproof}%
%
\isadelimproof
%
\endisadelimproof
%
\begin{isamarkuptext}%
A sequence is a Knight's circuit iff the sequence if a Knight's path and there is a valid 
step from the last square to the first square.%
\end{isamarkuptext}\isamarkuptrue%
\isacommand{definition}\isamarkupfalse%
\ {\isachardoublequoteopen}knights{\isacharunderscore}{\kern0pt}circuit\ b\ ps\ {\isasymequiv}\ {\isacharparenleft}{\kern0pt}knights{\isacharunderscore}{\kern0pt}path\ b\ ps\ {\isasymand}\ valid{\isacharunderscore}{\kern0pt}step\ {\isacharparenleft}{\kern0pt}last\ ps{\isacharparenright}{\kern0pt}\ {\isacharparenleft}{\kern0pt}hd\ ps{\isacharparenright}{\kern0pt}{\isacharparenright}{\kern0pt}{\isachardoublequoteclose}%
\isadelimdocument
%
\endisadelimdocument
%
\isatagdocument
%
\isamarkupsection{Executable Checker for a Knight's Path%
}
\isamarkuptrue%
%
\endisatagdocument
{\isafolddocument}%
%
\isadelimdocument
%
\endisadelimdocument
%
\begin{isamarkuptext}%
This section gives the implementation and correctness-proof for an executable checker for a
knights-path wrt. the definition \isa{knights{\isacharunderscore}{\kern0pt}path}.%
\end{isamarkuptext}\isamarkuptrue%
%
\isadelimdocument
%
\endisadelimdocument
%
\isatagdocument
%
\isamarkupsubsection{Implementation of an Executable Checker%
}
\isamarkuptrue%
%
\endisatagdocument
{\isafolddocument}%
%
\isadelimdocument
%
\endisadelimdocument
\isacommand{fun}\isamarkupfalse%
\ row{\isacharunderscore}{\kern0pt}exec\ {\isacharcolon}{\kern0pt}{\isacharcolon}{\kern0pt}\ {\isachardoublequoteopen}nat\ {\isasymRightarrow}\ int\ set{\isachardoublequoteclose}\ \isakeyword{where}\isanewline
\ \ {\isachardoublequoteopen}row{\isacharunderscore}{\kern0pt}exec\ {\isadigit{0}}\ {\isacharequal}{\kern0pt}\ {\isacharbraceleft}{\kern0pt}{\isacharbraceright}{\kern0pt}{\isachardoublequoteclose}\isanewline
{\isacharbar}{\kern0pt}\ {\isachardoublequoteopen}row{\isacharunderscore}{\kern0pt}exec\ m\ {\isacharequal}{\kern0pt}\ insert\ {\isacharparenleft}{\kern0pt}int\ m{\isacharparenright}{\kern0pt}\ {\isacharparenleft}{\kern0pt}row{\isacharunderscore}{\kern0pt}exec\ {\isacharparenleft}{\kern0pt}m{\isacharminus}{\kern0pt}{\isadigit{1}}{\isacharparenright}{\kern0pt}{\isacharparenright}{\kern0pt}{\isachardoublequoteclose}\isanewline
\isanewline
\isacommand{fun}\isamarkupfalse%
\ board{\isacharunderscore}{\kern0pt}exec{\isacharunderscore}{\kern0pt}aux\ {\isacharcolon}{\kern0pt}{\isacharcolon}{\kern0pt}\ {\isachardoublequoteopen}nat\ {\isasymRightarrow}\ int\ set\ {\isasymRightarrow}\ board{\isachardoublequoteclose}\ \isakeyword{where}\isanewline
\ \ {\isachardoublequoteopen}board{\isacharunderscore}{\kern0pt}exec{\isacharunderscore}{\kern0pt}aux\ {\isadigit{0}}\ M\ {\isacharequal}{\kern0pt}\ {\isacharbraceleft}{\kern0pt}{\isacharbraceright}{\kern0pt}{\isachardoublequoteclose}\ \ \isanewline
{\isacharbar}{\kern0pt}\ {\isachardoublequoteopen}board{\isacharunderscore}{\kern0pt}exec{\isacharunderscore}{\kern0pt}aux\ k\ M\ {\isacharequal}{\kern0pt}\ {\isacharbraceleft}{\kern0pt}{\isacharparenleft}{\kern0pt}int\ k{\isacharcomma}{\kern0pt}j{\isacharparenright}{\kern0pt}\ {\isacharbar}{\kern0pt}j{\isachardot}{\kern0pt}\ j\ {\isasymin}\ M{\isacharbraceright}{\kern0pt}\ {\isasymunion}\ board{\isacharunderscore}{\kern0pt}exec{\isacharunderscore}{\kern0pt}aux\ {\isacharparenleft}{\kern0pt}k{\isacharminus}{\kern0pt}{\isadigit{1}}{\isacharparenright}{\kern0pt}\ M{\isachardoublequoteclose}%
\begin{isamarkuptext}%
Compute a board.%
\end{isamarkuptext}\isamarkuptrue%
\isacommand{fun}\isamarkupfalse%
\ board{\isacharunderscore}{\kern0pt}exec\ {\isacharcolon}{\kern0pt}{\isacharcolon}{\kern0pt}\ {\isachardoublequoteopen}nat\ {\isasymRightarrow}\ nat\ {\isasymRightarrow}\ board{\isachardoublequoteclose}\ \isakeyword{where}\isanewline
\ \ {\isachardoublequoteopen}board{\isacharunderscore}{\kern0pt}exec\ n\ m\ {\isacharequal}{\kern0pt}\ board{\isacharunderscore}{\kern0pt}exec{\isacharunderscore}{\kern0pt}aux\ n\ {\isacharparenleft}{\kern0pt}row{\isacharunderscore}{\kern0pt}exec\ m{\isacharparenright}{\kern0pt}{\isachardoublequoteclose}\isanewline
\isanewline
\isacommand{fun}\isamarkupfalse%
\ step{\isacharunderscore}{\kern0pt}checker\ {\isacharcolon}{\kern0pt}{\isacharcolon}{\kern0pt}\ {\isachardoublequoteopen}square\ {\isasymRightarrow}\ square\ {\isasymRightarrow}\ bool{\isachardoublequoteclose}\ \isakeyword{where}\isanewline
\ \ {\isachardoublequoteopen}step{\isacharunderscore}{\kern0pt}checker\ {\isacharparenleft}{\kern0pt}i{\isacharcomma}{\kern0pt}j{\isacharparenright}{\kern0pt}\ {\isacharparenleft}{\kern0pt}i{\isacharprime}{\kern0pt}{\isacharcomma}{\kern0pt}j{\isacharprime}{\kern0pt}{\isacharparenright}{\kern0pt}\ {\isacharequal}{\kern0pt}\ \isanewline
\ \ \ \ {\isacharparenleft}{\kern0pt}{\isacharparenleft}{\kern0pt}i{\isacharplus}{\kern0pt}{\isadigit{1}}{\isacharcomma}{\kern0pt}j{\isacharplus}{\kern0pt}{\isadigit{2}}{\isacharparenright}{\kern0pt}\ {\isacharequal}{\kern0pt}\ {\isacharparenleft}{\kern0pt}i{\isacharprime}{\kern0pt}{\isacharcomma}{\kern0pt}j{\isacharprime}{\kern0pt}{\isacharparenright}{\kern0pt}\ {\isasymor}\ {\isacharparenleft}{\kern0pt}i{\isacharminus}{\kern0pt}{\isadigit{1}}{\isacharcomma}{\kern0pt}j{\isacharplus}{\kern0pt}{\isadigit{2}}{\isacharparenright}{\kern0pt}\ {\isacharequal}{\kern0pt}\ {\isacharparenleft}{\kern0pt}i{\isacharprime}{\kern0pt}{\isacharcomma}{\kern0pt}j{\isacharprime}{\kern0pt}{\isacharparenright}{\kern0pt}\ {\isasymor}\ {\isacharparenleft}{\kern0pt}i{\isacharplus}{\kern0pt}{\isadigit{1}}{\isacharcomma}{\kern0pt}j{\isacharminus}{\kern0pt}{\isadigit{2}}{\isacharparenright}{\kern0pt}\ {\isacharequal}{\kern0pt}\ {\isacharparenleft}{\kern0pt}i{\isacharprime}{\kern0pt}{\isacharcomma}{\kern0pt}j{\isacharprime}{\kern0pt}{\isacharparenright}{\kern0pt}\ {\isasymor}\ {\isacharparenleft}{\kern0pt}i{\isacharminus}{\kern0pt}{\isadigit{1}}{\isacharcomma}{\kern0pt}j{\isacharminus}{\kern0pt}{\isadigit{2}}{\isacharparenright}{\kern0pt}\ {\isacharequal}{\kern0pt}\ {\isacharparenleft}{\kern0pt}i{\isacharprime}{\kern0pt}{\isacharcomma}{\kern0pt}j{\isacharprime}{\kern0pt}{\isacharparenright}{\kern0pt}\ \isanewline
\ \ \ \ \ {\isasymor}\ {\isacharparenleft}{\kern0pt}i{\isacharplus}{\kern0pt}{\isadigit{2}}{\isacharcomma}{\kern0pt}j{\isacharplus}{\kern0pt}{\isadigit{1}}{\isacharparenright}{\kern0pt}\ {\isacharequal}{\kern0pt}\ {\isacharparenleft}{\kern0pt}i{\isacharprime}{\kern0pt}{\isacharcomma}{\kern0pt}j{\isacharprime}{\kern0pt}{\isacharparenright}{\kern0pt}\ {\isasymor}\ {\isacharparenleft}{\kern0pt}i{\isacharminus}{\kern0pt}{\isadigit{2}}{\isacharcomma}{\kern0pt}j{\isacharplus}{\kern0pt}{\isadigit{1}}{\isacharparenright}{\kern0pt}\ {\isacharequal}{\kern0pt}\ {\isacharparenleft}{\kern0pt}i{\isacharprime}{\kern0pt}{\isacharcomma}{\kern0pt}j{\isacharprime}{\kern0pt}{\isacharparenright}{\kern0pt}\ {\isasymor}\ {\isacharparenleft}{\kern0pt}i{\isacharplus}{\kern0pt}{\isadigit{2}}{\isacharcomma}{\kern0pt}j{\isacharminus}{\kern0pt}{\isadigit{1}}{\isacharparenright}{\kern0pt}\ {\isacharequal}{\kern0pt}\ {\isacharparenleft}{\kern0pt}i{\isacharprime}{\kern0pt}{\isacharcomma}{\kern0pt}j{\isacharprime}{\kern0pt}{\isacharparenright}{\kern0pt}\ {\isasymor}\ {\isacharparenleft}{\kern0pt}i{\isacharminus}{\kern0pt}{\isadigit{2}}{\isacharcomma}{\kern0pt}j{\isacharminus}{\kern0pt}{\isadigit{1}}{\isacharparenright}{\kern0pt}\ {\isacharequal}{\kern0pt}\ {\isacharparenleft}{\kern0pt}i{\isacharprime}{\kern0pt}{\isacharcomma}{\kern0pt}j{\isacharprime}{\kern0pt}{\isacharparenright}{\kern0pt}{\isacharparenright}{\kern0pt}{\isachardoublequoteclose}\isanewline
\isanewline
\isacommand{fun}\isamarkupfalse%
\ path{\isacharunderscore}{\kern0pt}checker\ {\isacharcolon}{\kern0pt}{\isacharcolon}{\kern0pt}\ {\isachardoublequoteopen}board\ {\isasymRightarrow}\ path\ {\isasymRightarrow}\ bool{\isachardoublequoteclose}\ \isakeyword{where}\isanewline
\ \ {\isachardoublequoteopen}path{\isacharunderscore}{\kern0pt}checker\ b\ {\isacharbrackleft}{\kern0pt}{\isacharbrackright}{\kern0pt}\ {\isacharequal}{\kern0pt}\ False{\isachardoublequoteclose}\isanewline
{\isacharbar}{\kern0pt}\ {\isachardoublequoteopen}path{\isacharunderscore}{\kern0pt}checker\ b\ {\isacharbrackleft}{\kern0pt}s\isactrlsub i{\isacharbrackright}{\kern0pt}\ {\isacharequal}{\kern0pt}\ {\isacharparenleft}{\kern0pt}{\isacharbraceleft}{\kern0pt}s\isactrlsub i{\isacharbraceright}{\kern0pt}\ {\isacharequal}{\kern0pt}\ b{\isacharparenright}{\kern0pt}{\isachardoublequoteclose}\isanewline
{\isacharbar}{\kern0pt}\ {\isachardoublequoteopen}path{\isacharunderscore}{\kern0pt}checker\ b\ {\isacharparenleft}{\kern0pt}s\isactrlsub i{\isacharhash}{\kern0pt}s\isactrlsub j{\isacharhash}{\kern0pt}ps{\isacharparenright}{\kern0pt}\ {\isacharequal}{\kern0pt}\ {\isacharparenleft}{\kern0pt}s\isactrlsub i\ {\isasymin}\ b\ {\isasymand}\ step{\isacharunderscore}{\kern0pt}checker\ s\isactrlsub i\ s\isactrlsub j\ {\isasymand}\ path{\isacharunderscore}{\kern0pt}checker\ {\isacharparenleft}{\kern0pt}b\ {\isacharminus}{\kern0pt}\ {\isacharbraceleft}{\kern0pt}s\isactrlsub i{\isacharbraceright}{\kern0pt}{\isacharparenright}{\kern0pt}\ {\isacharparenleft}{\kern0pt}s\isactrlsub j{\isacharhash}{\kern0pt}ps{\isacharparenright}{\kern0pt}{\isacharparenright}{\kern0pt}{\isachardoublequoteclose}\isanewline
\isanewline
\isacommand{fun}\isamarkupfalse%
\ circuit{\isacharunderscore}{\kern0pt}checker\ {\isacharcolon}{\kern0pt}{\isacharcolon}{\kern0pt}\ {\isachardoublequoteopen}board\ {\isasymRightarrow}\ path\ {\isasymRightarrow}\ bool{\isachardoublequoteclose}\ \isakeyword{where}\isanewline
\ \ {\isachardoublequoteopen}circuit{\isacharunderscore}{\kern0pt}checker\ b\ ps\ {\isacharequal}{\kern0pt}\ {\isacharparenleft}{\kern0pt}path{\isacharunderscore}{\kern0pt}checker\ b\ ps\ {\isasymand}\ step{\isacharunderscore}{\kern0pt}checker\ {\isacharparenleft}{\kern0pt}last\ ps{\isacharparenright}{\kern0pt}\ {\isacharparenleft}{\kern0pt}hd\ ps{\isacharparenright}{\kern0pt}{\isacharparenright}{\kern0pt}{\isachardoublequoteclose}%
\isadelimdocument
%
\endisadelimdocument
%
\isatagdocument
%
\isamarkupsubsection{Correctness Proof of the Executable Checker%
}
\isamarkuptrue%
%
\endisatagdocument
{\isafolddocument}%
%
\isadelimdocument
%
\endisadelimdocument
\isacommand{lemma}\isamarkupfalse%
\ row{\isacharunderscore}{\kern0pt}exec{\isacharunderscore}{\kern0pt}leq{\isacharcolon}{\kern0pt}\ {\isachardoublequoteopen}j\ {\isasymin}\ row{\isacharunderscore}{\kern0pt}exec\ m\ {\isasymlongleftrightarrow}\ {\isadigit{1}}\ {\isasymle}\ j\ {\isasymand}\ j\ {\isasymle}\ int\ m{\isachardoublequoteclose}\isanewline
%
\isadelimproof
\ \ %
\endisadelimproof
%
\isatagproof
\isacommand{by}\isamarkupfalse%
\ {\isacharparenleft}{\kern0pt}induction\ m{\isacharparenright}{\kern0pt}\ auto%
\endisatagproof
{\isafoldproof}%
%
\isadelimproof
\isanewline
%
\endisadelimproof
\isanewline
\isacommand{lemma}\isamarkupfalse%
\ board{\isacharunderscore}{\kern0pt}exec{\isacharunderscore}{\kern0pt}aux{\isacharunderscore}{\kern0pt}leq{\isacharunderscore}{\kern0pt}mem{\isacharcolon}{\kern0pt}\ {\isachardoublequoteopen}{\isacharparenleft}{\kern0pt}i{\isacharcomma}{\kern0pt}j{\isacharparenright}{\kern0pt}\ {\isasymin}\ board{\isacharunderscore}{\kern0pt}exec{\isacharunderscore}{\kern0pt}aux\ k\ M\ {\isasymlongleftrightarrow}\ {\isadigit{1}}\ {\isasymle}\ i\ {\isasymand}\ i\ {\isasymle}\ int\ k\ {\isasymand}\ j\ {\isasymin}\ M{\isachardoublequoteclose}\isanewline
%
\isadelimproof
\ \ %
\endisadelimproof
%
\isatagproof
\isacommand{by}\isamarkupfalse%
\ {\isacharparenleft}{\kern0pt}induction\ k\ M\ rule{\isacharcolon}{\kern0pt}\ board{\isacharunderscore}{\kern0pt}exec{\isacharunderscore}{\kern0pt}aux{\isachardot}{\kern0pt}induct{\isacharparenright}{\kern0pt}\ auto%
\endisatagproof
{\isafoldproof}%
%
\isadelimproof
\isanewline
%
\endisadelimproof
\isanewline
\isacommand{lemma}\isamarkupfalse%
\ board{\isacharunderscore}{\kern0pt}exec{\isacharunderscore}{\kern0pt}leq{\isacharcolon}{\kern0pt}\ {\isachardoublequoteopen}{\isacharparenleft}{\kern0pt}i{\isacharcomma}{\kern0pt}j{\isacharparenright}{\kern0pt}\ {\isasymin}\ board{\isacharunderscore}{\kern0pt}exec\ n\ m\ {\isasymlongleftrightarrow}\ {\isadigit{1}}\ {\isasymle}\ i\ {\isasymand}\ i\ {\isasymle}\ int\ n\ {\isasymand}\ {\isadigit{1}}\ {\isasymle}\ j\ {\isasymand}\ j\ {\isasymle}\ int\ m{\isachardoublequoteclose}\isanewline
%
\isadelimproof
\ \ %
\endisadelimproof
%
\isatagproof
\isacommand{using}\isamarkupfalse%
\ board{\isacharunderscore}{\kern0pt}exec{\isacharunderscore}{\kern0pt}aux{\isacharunderscore}{\kern0pt}leq{\isacharunderscore}{\kern0pt}mem\ row{\isacharunderscore}{\kern0pt}exec{\isacharunderscore}{\kern0pt}leq\ \isacommand{by}\isamarkupfalse%
\ auto%
\endisatagproof
{\isafoldproof}%
%
\isadelimproof
\isanewline
%
\endisadelimproof
\isanewline
\isacommand{lemma}\isamarkupfalse%
\ board{\isacharunderscore}{\kern0pt}exec{\isacharunderscore}{\kern0pt}correct{\isacharcolon}{\kern0pt}\ {\isachardoublequoteopen}board\ n\ m\ {\isacharequal}{\kern0pt}\ board{\isacharunderscore}{\kern0pt}exec\ n\ m{\isachardoublequoteclose}\isanewline
%
\isadelimproof
\ \ %
\endisadelimproof
%
\isatagproof
\isacommand{unfolding}\isamarkupfalse%
\ board{\isacharunderscore}{\kern0pt}def\ \isacommand{using}\isamarkupfalse%
\ board{\isacharunderscore}{\kern0pt}exec{\isacharunderscore}{\kern0pt}leq\ \isacommand{by}\isamarkupfalse%
\ auto%
\endisatagproof
{\isafoldproof}%
%
\isadelimproof
\isanewline
%
\endisadelimproof
\isanewline
\isacommand{lemma}\isamarkupfalse%
\ step{\isacharunderscore}{\kern0pt}checker{\isacharunderscore}{\kern0pt}correct{\isacharcolon}{\kern0pt}\ {\isachardoublequoteopen}step{\isacharunderscore}{\kern0pt}checker\ s\isactrlsub i\ s\isactrlsub j\ {\isasymlongleftrightarrow}\ valid{\isacharunderscore}{\kern0pt}step\ s\isactrlsub i\ s\isactrlsub j{\isachardoublequoteclose}\isanewline
%
\isadelimproof
%
\endisadelimproof
%
\isatagproof
\isacommand{proof}\isamarkupfalse%
\isanewline
\ \ \isacommand{assume}\isamarkupfalse%
\ {\isachardoublequoteopen}step{\isacharunderscore}{\kern0pt}checker\ s\isactrlsub i\ s\isactrlsub j{\isachardoublequoteclose}\isanewline
\ \ \isacommand{then}\isamarkupfalse%
\ \isacommand{show}\isamarkupfalse%
\ {\isachardoublequoteopen}valid{\isacharunderscore}{\kern0pt}step\ s\isactrlsub i\ s\isactrlsub j{\isachardoublequoteclose}\isanewline
\ \ \ \ \isacommand{unfolding}\isamarkupfalse%
\ valid{\isacharunderscore}{\kern0pt}step{\isacharunderscore}{\kern0pt}def\ \isanewline
\ \ \ \ \isacommand{apply}\isamarkupfalse%
\ {\isacharparenleft}{\kern0pt}cases\ s\isactrlsub i{\isacharparenright}{\kern0pt}\isanewline
\ \ \ \ \isacommand{apply}\isamarkupfalse%
\ {\isacharparenleft}{\kern0pt}cases\ s\isactrlsub j{\isacharparenright}{\kern0pt}\isanewline
\ \ \ \ \isacommand{apply}\isamarkupfalse%
\ auto\isanewline
\ \ \ \ \isacommand{done}\isamarkupfalse%
\isanewline
\isacommand{next}\isamarkupfalse%
\isanewline
\ \ \isacommand{assume}\isamarkupfalse%
\ assms{\isacharcolon}{\kern0pt}\ {\isachardoublequoteopen}valid{\isacharunderscore}{\kern0pt}step\ s\isactrlsub i\ s\isactrlsub j{\isachardoublequoteclose}\isanewline
\ \ \isacommand{then}\isamarkupfalse%
\ \isacommand{show}\isamarkupfalse%
\ {\isachardoublequoteopen}step{\isacharunderscore}{\kern0pt}checker\ s\isactrlsub i\ s\isactrlsub j{\isachardoublequoteclose}\isanewline
\ \ \ \ \isacommand{unfolding}\isamarkupfalse%
\ valid{\isacharunderscore}{\kern0pt}step{\isacharunderscore}{\kern0pt}def\ \isacommand{by}\isamarkupfalse%
\ auto\isanewline
\isacommand{qed}\isamarkupfalse%
%
\endisatagproof
{\isafoldproof}%
%
\isadelimproof
\isanewline
%
\endisadelimproof
\isanewline
\isacommand{lemma}\isamarkupfalse%
\ step{\isacharunderscore}{\kern0pt}checker{\isacharunderscore}{\kern0pt}rev{\isacharcolon}{\kern0pt}\ {\isachardoublequoteopen}step{\isacharunderscore}{\kern0pt}checker\ {\isacharparenleft}{\kern0pt}i{\isacharcomma}{\kern0pt}j{\isacharparenright}{\kern0pt}\ {\isacharparenleft}{\kern0pt}i{\isacharprime}{\kern0pt}{\isacharcomma}{\kern0pt}j{\isacharprime}{\kern0pt}{\isacharparenright}{\kern0pt}\ {\isasymLongrightarrow}\ step{\isacharunderscore}{\kern0pt}checker\ {\isacharparenleft}{\kern0pt}i{\isacharprime}{\kern0pt}{\isacharcomma}{\kern0pt}j{\isacharprime}{\kern0pt}{\isacharparenright}{\kern0pt}\ {\isacharparenleft}{\kern0pt}i{\isacharcomma}{\kern0pt}j{\isacharparenright}{\kern0pt}{\isachardoublequoteclose}\isanewline
%
\isadelimproof
\ \ %
\endisadelimproof
%
\isatagproof
\isacommand{apply}\isamarkupfalse%
\ {\isacharparenleft}{\kern0pt}simp\ only{\isacharcolon}{\kern0pt}\ step{\isacharunderscore}{\kern0pt}checker{\isachardot}{\kern0pt}simps{\isacharparenright}{\kern0pt}\isanewline
\ \ \isacommand{by}\isamarkupfalse%
\ {\isacharparenleft}{\kern0pt}elim\ disjE{\isacharparenright}{\kern0pt}\ auto%
\endisatagproof
{\isafoldproof}%
%
\isadelimproof
\isanewline
%
\endisadelimproof
\isanewline
\isacommand{lemma}\isamarkupfalse%
\ knights{\isacharunderscore}{\kern0pt}path{\isacharunderscore}{\kern0pt}intro{\isacharunderscore}{\kern0pt}rev{\isacharcolon}{\kern0pt}\ \isanewline
\ \ \isakeyword{assumes}\ {\isachardoublequoteopen}s\isactrlsub i\ {\isasymin}\ b{\isachardoublequoteclose}\ {\isachardoublequoteopen}valid{\isacharunderscore}{\kern0pt}step\ s\isactrlsub i\ s\isactrlsub j{\isachardoublequoteclose}\ {\isachardoublequoteopen}knights{\isacharunderscore}{\kern0pt}path\ {\isacharparenleft}{\kern0pt}b\ {\isacharminus}{\kern0pt}\ {\isacharbraceleft}{\kern0pt}s\isactrlsub i{\isacharbraceright}{\kern0pt}{\isacharparenright}{\kern0pt}\ {\isacharparenleft}{\kern0pt}s\isactrlsub j{\isacharhash}{\kern0pt}ps{\isacharparenright}{\kern0pt}{\isachardoublequoteclose}\ \isanewline
\ \ \isakeyword{shows}\ {\isachardoublequoteopen}knights{\isacharunderscore}{\kern0pt}path\ b\ {\isacharparenleft}{\kern0pt}s\isactrlsub i{\isacharhash}{\kern0pt}s\isactrlsub j{\isacharhash}{\kern0pt}ps{\isacharparenright}{\kern0pt}{\isachardoublequoteclose}\isanewline
%
\isadelimproof
\ \ %
\endisadelimproof
%
\isatagproof
\isacommand{using}\isamarkupfalse%
\ assms\isanewline
\isacommand{proof}\isamarkupfalse%
\ {\isacharminus}{\kern0pt}\isanewline
\ \ \isacommand{assume}\isamarkupfalse%
\ assms{\isacharcolon}{\kern0pt}\ {\isachardoublequoteopen}s\isactrlsub i\ {\isasymin}\ b{\isachardoublequoteclose}\ {\isachardoublequoteopen}valid{\isacharunderscore}{\kern0pt}step\ s\isactrlsub i\ s\isactrlsub j{\isachardoublequoteclose}\ {\isachardoublequoteopen}knights{\isacharunderscore}{\kern0pt}path\ {\isacharparenleft}{\kern0pt}b\ {\isacharminus}{\kern0pt}\ {\isacharbraceleft}{\kern0pt}s\isactrlsub i{\isacharbraceright}{\kern0pt}{\isacharparenright}{\kern0pt}\ {\isacharparenleft}{\kern0pt}s\isactrlsub j{\isacharhash}{\kern0pt}ps{\isacharparenright}{\kern0pt}{\isachardoublequoteclose}\isanewline
\ \ \isacommand{then}\isamarkupfalse%
\ \isacommand{have}\isamarkupfalse%
\ {\isachardoublequoteopen}s\isactrlsub i\ {\isasymnotin}\ {\isacharparenleft}{\kern0pt}b\ {\isacharminus}{\kern0pt}\ {\isacharbraceleft}{\kern0pt}s\isactrlsub i{\isacharbraceright}{\kern0pt}{\isacharparenright}{\kern0pt}{\isachardoublequoteclose}\ {\isachardoublequoteopen}b\ {\isacharminus}{\kern0pt}\ {\isacharbraceleft}{\kern0pt}s\isactrlsub i{\isacharbraceright}{\kern0pt}\ {\isasymunion}\ {\isacharbraceleft}{\kern0pt}s\isactrlsub i{\isacharbraceright}{\kern0pt}\ {\isacharequal}{\kern0pt}\ b{\isachardoublequoteclose}\isanewline
\ \ \ \ \isacommand{by}\isamarkupfalse%
\ auto\isanewline
\ \ \isacommand{then}\isamarkupfalse%
\ \isacommand{show}\isamarkupfalse%
\ {\isacharquery}{\kern0pt}thesis\isanewline
\ \ \ \ \isacommand{using}\isamarkupfalse%
\ assms\ knights{\isacharunderscore}{\kern0pt}path{\isachardot}{\kern0pt}intros{\isacharparenleft}{\kern0pt}{\isadigit{2}}{\isacharparenright}{\kern0pt}{\isacharbrackleft}{\kern0pt}of\ s\isactrlsub i\ {\isachardoublequoteopen}b\ {\isacharminus}{\kern0pt}\ {\isacharbraceleft}{\kern0pt}s\isactrlsub i{\isacharbraceright}{\kern0pt}{\isachardoublequoteclose}{\isacharbrackright}{\kern0pt}\ \isacommand{by}\isamarkupfalse%
\ auto\isanewline
\isacommand{qed}\isamarkupfalse%
%
\endisatagproof
{\isafoldproof}%
%
\isadelimproof
%
\endisadelimproof
%
\begin{isamarkuptext}%
Final correctness corollary for the executable checker \isa{path{\isacharunderscore}{\kern0pt}checker}.%
\end{isamarkuptext}\isamarkuptrue%
\isacommand{lemma}\isamarkupfalse%
\ path{\isacharunderscore}{\kern0pt}checker{\isacharunderscore}{\kern0pt}correct{\isacharcolon}{\kern0pt}\ {\isachardoublequoteopen}path{\isacharunderscore}{\kern0pt}checker\ b\ ps\ {\isasymlongleftrightarrow}\ knights{\isacharunderscore}{\kern0pt}path\ b\ ps{\isachardoublequoteclose}\isanewline
%
\isadelimproof
%
\endisadelimproof
%
\isatagproof
\isacommand{proof}\isamarkupfalse%
\isanewline
\ \ \isacommand{assume}\isamarkupfalse%
\ {\isachardoublequoteopen}path{\isacharunderscore}{\kern0pt}checker\ b\ ps{\isachardoublequoteclose}\isanewline
\ \ \isacommand{then}\isamarkupfalse%
\ \isacommand{show}\isamarkupfalse%
\ {\isachardoublequoteopen}knights{\isacharunderscore}{\kern0pt}path\ b\ ps{\isachardoublequoteclose}\isanewline
\ \ \isacommand{proof}\isamarkupfalse%
\ {\isacharparenleft}{\kern0pt}induction\ rule{\isacharcolon}{\kern0pt}\ path{\isacharunderscore}{\kern0pt}checker{\isachardot}{\kern0pt}induct{\isacharparenright}{\kern0pt}\isanewline
\ \ \ \ \isacommand{case}\isamarkupfalse%
\ {\isacharparenleft}{\kern0pt}{\isadigit{3}}\ s\isactrlsub i\ s\isactrlsub j\ xs\ b{\isacharparenright}{\kern0pt}\isanewline
\ \ \ \ \isacommand{then}\isamarkupfalse%
\ \isacommand{show}\isamarkupfalse%
\ {\isacharquery}{\kern0pt}case\ \isacommand{using}\isamarkupfalse%
\ step{\isacharunderscore}{\kern0pt}checker{\isacharunderscore}{\kern0pt}correct\ knights{\isacharunderscore}{\kern0pt}path{\isacharunderscore}{\kern0pt}intro{\isacharunderscore}{\kern0pt}rev\ \isacommand{by}\isamarkupfalse%
\ auto\isanewline
\ \ \isacommand{qed}\isamarkupfalse%
\ {\isacharparenleft}{\kern0pt}auto\ intro{\isacharcolon}{\kern0pt}\ knights{\isacharunderscore}{\kern0pt}path{\isachardot}{\kern0pt}intros{\isacharparenright}{\kern0pt}\isanewline
\isacommand{next}\isamarkupfalse%
\isanewline
\ \ \isacommand{assume}\isamarkupfalse%
\ {\isachardoublequoteopen}knights{\isacharunderscore}{\kern0pt}path\ b\ ps{\isachardoublequoteclose}\isanewline
\ \ \isacommand{then}\isamarkupfalse%
\ \isacommand{show}\isamarkupfalse%
\ {\isachardoublequoteopen}path{\isacharunderscore}{\kern0pt}checker\ b\ ps{\isachardoublequoteclose}\ \ \ \ \ \ \ \ \ \ \ \ \ \ \ \ \ \ \isanewline
\ \ \ \ \isacommand{using}\isamarkupfalse%
\ step{\isacharunderscore}{\kern0pt}checker{\isacharunderscore}{\kern0pt}correct\ \isanewline
\ \ \ \ \isacommand{by}\isamarkupfalse%
\ {\isacharparenleft}{\kern0pt}induction\ rule{\isacharcolon}{\kern0pt}\ knights{\isacharunderscore}{\kern0pt}path{\isachardot}{\kern0pt}induct{\isacharparenright}{\kern0pt}\ {\isacharparenleft}{\kern0pt}auto\ elim{\isacharcolon}{\kern0pt}\ knights{\isacharunderscore}{\kern0pt}path{\isachardot}{\kern0pt}cases{\isacharparenright}{\kern0pt}\isanewline
\isacommand{qed}\isamarkupfalse%
%
\endisatagproof
{\isafoldproof}%
%
\isadelimproof
\isanewline
%
\endisadelimproof
\isanewline
\isacommand{corollary}\isamarkupfalse%
\ knights{\isacharunderscore}{\kern0pt}path{\isacharunderscore}{\kern0pt}exec{\isacharunderscore}{\kern0pt}simp{\isacharcolon}{\kern0pt}\ {\isachardoublequoteopen}knights{\isacharunderscore}{\kern0pt}path\ {\isacharparenleft}{\kern0pt}board\ n\ m{\isacharparenright}{\kern0pt}\ ps\ {\isasymlongleftrightarrow}\ path{\isacharunderscore}{\kern0pt}checker\ {\isacharparenleft}{\kern0pt}board{\isacharunderscore}{\kern0pt}exec\ n\ m{\isacharparenright}{\kern0pt}\ ps{\isachardoublequoteclose}\isanewline
%
\isadelimproof
\ \ %
\endisadelimproof
%
\isatagproof
\isacommand{using}\isamarkupfalse%
\ board{\isacharunderscore}{\kern0pt}exec{\isacharunderscore}{\kern0pt}correct\ path{\isacharunderscore}{\kern0pt}checker{\isacharunderscore}{\kern0pt}correct{\isacharbrackleft}{\kern0pt}symmetric{\isacharbrackright}{\kern0pt}\ \isacommand{by}\isamarkupfalse%
\ simp%
\endisatagproof
{\isafoldproof}%
%
\isadelimproof
\isanewline
%
\endisadelimproof
\isanewline
\isacommand{lemma}\isamarkupfalse%
\ circuit{\isacharunderscore}{\kern0pt}checker{\isacharunderscore}{\kern0pt}correct{\isacharcolon}{\kern0pt}\ {\isachardoublequoteopen}circuit{\isacharunderscore}{\kern0pt}checker\ b\ ps\ {\isasymlongleftrightarrow}\ knights{\isacharunderscore}{\kern0pt}circuit\ b\ ps{\isachardoublequoteclose}\isanewline
%
\isadelimproof
\ \ %
\endisadelimproof
%
\isatagproof
\isacommand{unfolding}\isamarkupfalse%
\ knights{\isacharunderscore}{\kern0pt}circuit{\isacharunderscore}{\kern0pt}def\ \isacommand{using}\isamarkupfalse%
\ path{\isacharunderscore}{\kern0pt}checker{\isacharunderscore}{\kern0pt}correct\ step{\isacharunderscore}{\kern0pt}checker{\isacharunderscore}{\kern0pt}correct\ \isacommand{by}\isamarkupfalse%
\ auto%
\endisatagproof
{\isafoldproof}%
%
\isadelimproof
\isanewline
%
\endisadelimproof
\isanewline
\isacommand{corollary}\isamarkupfalse%
\ knights{\isacharunderscore}{\kern0pt}circuit{\isacharunderscore}{\kern0pt}exec{\isacharunderscore}{\kern0pt}simp{\isacharcolon}{\kern0pt}\ \isanewline
\ \ {\isachardoublequoteopen}knights{\isacharunderscore}{\kern0pt}circuit\ {\isacharparenleft}{\kern0pt}board\ n\ m{\isacharparenright}{\kern0pt}\ ps\ {\isasymlongleftrightarrow}\ circuit{\isacharunderscore}{\kern0pt}checker\ {\isacharparenleft}{\kern0pt}board{\isacharunderscore}{\kern0pt}exec\ n\ m{\isacharparenright}{\kern0pt}\ ps{\isachardoublequoteclose}\isanewline
%
\isadelimproof
\ \ %
\endisadelimproof
%
\isatagproof
\isacommand{using}\isamarkupfalse%
\ board{\isacharunderscore}{\kern0pt}exec{\isacharunderscore}{\kern0pt}correct\ circuit{\isacharunderscore}{\kern0pt}checker{\isacharunderscore}{\kern0pt}correct{\isacharbrackleft}{\kern0pt}symmetric{\isacharbrackright}{\kern0pt}\ \isacommand{by}\isamarkupfalse%
\ simp%
\endisatagproof
{\isafoldproof}%
%
\isadelimproof
%
\endisadelimproof
%
\isadelimdocument
%
\endisadelimdocument
%
\isatagdocument
%
\isamarkupsection{Basic Properties of \isa{knights{\isacharunderscore}{\kern0pt}path} and \isa{knights{\isacharunderscore}{\kern0pt}circuit}%
}
\isamarkuptrue%
%
\endisatagdocument
{\isafolddocument}%
%
\isadelimdocument
%
\endisadelimdocument
\isacommand{lemma}\isamarkupfalse%
\ board{\isacharunderscore}{\kern0pt}leq{\isacharunderscore}{\kern0pt}subset{\isacharcolon}{\kern0pt}\ {\isachardoublequoteopen}n\isactrlsub {\isadigit{1}}\ {\isasymle}\ n\isactrlsub {\isadigit{2}}\ {\isasymand}\ m\isactrlsub {\isadigit{1}}\ {\isasymle}\ m\isactrlsub {\isadigit{2}}\ {\isasymLongrightarrow}\ board\ n\isactrlsub {\isadigit{1}}\ m\isactrlsub {\isadigit{1}}\ {\isasymsubseteq}\ board\ n\isactrlsub {\isadigit{2}}\ m\isactrlsub {\isadigit{2}}{\isachardoublequoteclose}\isanewline
%
\isadelimproof
\ \ %
\endisadelimproof
%
\isatagproof
\isacommand{unfolding}\isamarkupfalse%
\ board{\isacharunderscore}{\kern0pt}def\ \isacommand{by}\isamarkupfalse%
\ auto%
\endisatagproof
{\isafoldproof}%
%
\isadelimproof
\isanewline
%
\endisadelimproof
\isanewline
\isacommand{lemma}\isamarkupfalse%
\ finite{\isacharunderscore}{\kern0pt}row{\isacharunderscore}{\kern0pt}exec{\isacharcolon}{\kern0pt}\ {\isachardoublequoteopen}finite\ {\isacharparenleft}{\kern0pt}row{\isacharunderscore}{\kern0pt}exec\ m{\isacharparenright}{\kern0pt}{\isachardoublequoteclose}\isanewline
%
\isadelimproof
\ \ %
\endisadelimproof
%
\isatagproof
\isacommand{by}\isamarkupfalse%
\ {\isacharparenleft}{\kern0pt}induction\ m{\isacharparenright}{\kern0pt}\ auto%
\endisatagproof
{\isafoldproof}%
%
\isadelimproof
\isanewline
%
\endisadelimproof
\isanewline
\isacommand{lemma}\isamarkupfalse%
\ finite{\isacharunderscore}{\kern0pt}board{\isacharunderscore}{\kern0pt}exec{\isacharunderscore}{\kern0pt}aux{\isacharcolon}{\kern0pt}\ {\isachardoublequoteopen}finite\ M\ {\isasymLongrightarrow}\ finite\ {\isacharparenleft}{\kern0pt}board{\isacharunderscore}{\kern0pt}exec{\isacharunderscore}{\kern0pt}aux\ n\ M{\isacharparenright}{\kern0pt}{\isachardoublequoteclose}\isanewline
%
\isadelimproof
\ \ %
\endisadelimproof
%
\isatagproof
\isacommand{by}\isamarkupfalse%
\ {\isacharparenleft}{\kern0pt}induction\ n{\isacharparenright}{\kern0pt}\ auto%
\endisatagproof
{\isafoldproof}%
%
\isadelimproof
\isanewline
%
\endisadelimproof
\isanewline
\isacommand{lemma}\isamarkupfalse%
\ board{\isacharunderscore}{\kern0pt}finite{\isacharcolon}{\kern0pt}\ {\isachardoublequoteopen}finite\ {\isacharparenleft}{\kern0pt}board\ n\ m{\isacharparenright}{\kern0pt}{\isachardoublequoteclose}\isanewline
%
\isadelimproof
\ \ %
\endisadelimproof
%
\isatagproof
\isacommand{using}\isamarkupfalse%
\ finite{\isacharunderscore}{\kern0pt}board{\isacharunderscore}{\kern0pt}exec{\isacharunderscore}{\kern0pt}aux\ finite{\isacharunderscore}{\kern0pt}row{\isacharunderscore}{\kern0pt}exec\ \isacommand{by}\isamarkupfalse%
\ {\isacharparenleft}{\kern0pt}simp\ only{\isacharcolon}{\kern0pt}\ board{\isacharunderscore}{\kern0pt}exec{\isacharunderscore}{\kern0pt}correct{\isacharparenright}{\kern0pt}\ auto%
\endisatagproof
{\isafoldproof}%
%
\isadelimproof
\isanewline
%
\endisadelimproof
\isanewline
\isacommand{lemma}\isamarkupfalse%
\ card{\isacharunderscore}{\kern0pt}row{\isacharunderscore}{\kern0pt}exec{\isacharcolon}{\kern0pt}\ {\isachardoublequoteopen}card\ {\isacharparenleft}{\kern0pt}row{\isacharunderscore}{\kern0pt}exec\ m{\isacharparenright}{\kern0pt}\ {\isacharequal}{\kern0pt}\ m{\isachardoublequoteclose}\isanewline
%
\isadelimproof
%
\endisadelimproof
%
\isatagproof
\isacommand{proof}\isamarkupfalse%
\ {\isacharparenleft}{\kern0pt}induction\ m{\isacharparenright}{\kern0pt}\isanewline
\ \ \isacommand{case}\isamarkupfalse%
\ {\isacharparenleft}{\kern0pt}Suc\ m{\isacharparenright}{\kern0pt}\isanewline
\ \ \isacommand{have}\isamarkupfalse%
\ {\isachardoublequoteopen}int\ {\isacharparenleft}{\kern0pt}Suc\ m{\isacharparenright}{\kern0pt}\ {\isasymnotin}\ row{\isacharunderscore}{\kern0pt}exec\ m{\isachardoublequoteclose}\isanewline
\ \ \ \ \isacommand{using}\isamarkupfalse%
\ row{\isacharunderscore}{\kern0pt}exec{\isacharunderscore}{\kern0pt}leq\ \isacommand{by}\isamarkupfalse%
\ auto\isanewline
\ \ \isacommand{then}\isamarkupfalse%
\ \isacommand{have}\isamarkupfalse%
\ {\isachardoublequoteopen}card\ {\isacharparenleft}{\kern0pt}insert\ {\isacharparenleft}{\kern0pt}int\ {\isacharparenleft}{\kern0pt}Suc\ m{\isacharparenright}{\kern0pt}{\isacharparenright}{\kern0pt}\ {\isacharparenleft}{\kern0pt}row{\isacharunderscore}{\kern0pt}exec\ m{\isacharparenright}{\kern0pt}{\isacharparenright}{\kern0pt}\ {\isacharequal}{\kern0pt}\ {\isadigit{1}}\ {\isacharplus}{\kern0pt}\ card\ {\isacharparenleft}{\kern0pt}row{\isacharunderscore}{\kern0pt}exec\ m{\isacharparenright}{\kern0pt}{\isachardoublequoteclose}\isanewline
\ \ \ \ \isacommand{using}\isamarkupfalse%
\ card{\isacharunderscore}{\kern0pt}Suc{\isacharunderscore}{\kern0pt}eq\ \isacommand{by}\isamarkupfalse%
\ {\isacharparenleft}{\kern0pt}metis\ Suc\ plus{\isacharunderscore}{\kern0pt}{\isadigit{1}}{\isacharunderscore}{\kern0pt}eq{\isacharunderscore}{\kern0pt}Suc\ row{\isacharunderscore}{\kern0pt}exec{\isachardot}{\kern0pt}simps{\isacharparenleft}{\kern0pt}{\isadigit{1}}{\isacharparenright}{\kern0pt}{\isacharparenright}{\kern0pt}\isanewline
\ \ \isacommand{then}\isamarkupfalse%
\ \isacommand{have}\isamarkupfalse%
\ {\isachardoublequoteopen}card\ {\isacharparenleft}{\kern0pt}row{\isacharunderscore}{\kern0pt}exec\ {\isacharparenleft}{\kern0pt}Suc\ m{\isacharparenright}{\kern0pt}{\isacharparenright}{\kern0pt}\ {\isacharequal}{\kern0pt}\ {\isadigit{1}}\ {\isacharplus}{\kern0pt}\ card\ {\isacharparenleft}{\kern0pt}row{\isacharunderscore}{\kern0pt}exec\ m{\isacharparenright}{\kern0pt}{\isachardoublequoteclose}\isanewline
\ \ \ \ \isacommand{by}\isamarkupfalse%
\ auto\isanewline
\ \ \isacommand{then}\isamarkupfalse%
\ \isacommand{show}\isamarkupfalse%
\ {\isacharquery}{\kern0pt}case\ \isacommand{using}\isamarkupfalse%
\ Suc{\isachardot}{\kern0pt}IH\ \isacommand{by}\isamarkupfalse%
\ auto\ \isanewline
\isacommand{qed}\isamarkupfalse%
\ auto%
\endisatagproof
{\isafoldproof}%
%
\isadelimproof
\isanewline
%
\endisadelimproof
\isanewline
\isacommand{lemma}\isamarkupfalse%
\ set{\isacharunderscore}{\kern0pt}comp{\isacharunderscore}{\kern0pt}ins{\isacharcolon}{\kern0pt}\ \isanewline
\ \ {\isachardoublequoteopen}{\isacharbraceleft}{\kern0pt}{\isacharparenleft}{\kern0pt}k{\isacharcomma}{\kern0pt}j{\isacharparenright}{\kern0pt}\ {\isacharbar}{\kern0pt}j{\isachardot}{\kern0pt}\ j\ {\isasymin}\ insert\ x\ M{\isacharbraceright}{\kern0pt}\ {\isacharequal}{\kern0pt}\ insert\ {\isacharparenleft}{\kern0pt}k{\isacharcomma}{\kern0pt}x{\isacharparenright}{\kern0pt}\ {\isacharbraceleft}{\kern0pt}{\isacharparenleft}{\kern0pt}k{\isacharcomma}{\kern0pt}j{\isacharparenright}{\kern0pt}\ {\isacharbar}{\kern0pt}j{\isachardot}{\kern0pt}\ j\ {\isasymin}\ M{\isacharbraceright}{\kern0pt}{\isachardoublequoteclose}\ {\isacharparenleft}{\kern0pt}\isakeyword{is}\ {\isachardoublequoteopen}{\isacharquery}{\kern0pt}Mi\ {\isacharequal}{\kern0pt}\ {\isacharquery}{\kern0pt}iM{\isachardoublequoteclose}{\isacharparenright}{\kern0pt}\isanewline
%
\isadelimproof
%
\endisadelimproof
%
\isatagproof
\isacommand{proof}\isamarkupfalse%
\isanewline
\ \ \isacommand{show}\isamarkupfalse%
\ {\isachardoublequoteopen}{\isacharquery}{\kern0pt}Mi\ {\isasymsubseteq}\ {\isacharquery}{\kern0pt}iM{\isachardoublequoteclose}\isanewline
\ \ \isacommand{proof}\isamarkupfalse%
\isanewline
\ \ \ \ \isacommand{fix}\isamarkupfalse%
\ y\ \isacommand{assume}\isamarkupfalse%
\ {\isachardoublequoteopen}y\ {\isasymin}\ {\isacharquery}{\kern0pt}Mi{\isachardoublequoteclose}\isanewline
\ \ \ \ \isacommand{then}\isamarkupfalse%
\ \isacommand{obtain}\isamarkupfalse%
\ j\ \isakeyword{where}\ {\isacharbrackleft}{\kern0pt}simp{\isacharbrackright}{\kern0pt}{\isacharcolon}{\kern0pt}\ {\isachardoublequoteopen}y\ {\isacharequal}{\kern0pt}\ {\isacharparenleft}{\kern0pt}k{\isacharcomma}{\kern0pt}j{\isacharparenright}{\kern0pt}{\isachardoublequoteclose}\ \isakeyword{and}\ {\isachardoublequoteopen}j\ {\isasymin}\ insert\ x\ M{\isachardoublequoteclose}\ \isacommand{by}\isamarkupfalse%
\ blast\isanewline
\ \ \ \ \isacommand{then}\isamarkupfalse%
\ \isacommand{have}\isamarkupfalse%
\ {\isachardoublequoteopen}j\ {\isacharequal}{\kern0pt}\ x\ {\isasymor}\ j\ {\isasymin}\ M{\isachardoublequoteclose}\ \isacommand{by}\isamarkupfalse%
\ auto\isanewline
\ \ \ \ \isacommand{then}\isamarkupfalse%
\ \isacommand{show}\isamarkupfalse%
\ {\isachardoublequoteopen}y\ {\isasymin}\ {\isacharquery}{\kern0pt}iM{\isachardoublequoteclose}\ \isacommand{by}\isamarkupfalse%
\ {\isacharparenleft}{\kern0pt}elim\ disjE{\isacharparenright}{\kern0pt}\ auto\isanewline
\ \ \isacommand{qed}\isamarkupfalse%
\isanewline
\isacommand{next}\isamarkupfalse%
\isanewline
\ \ \isacommand{show}\isamarkupfalse%
\ {\isachardoublequoteopen}{\isacharquery}{\kern0pt}iM\ {\isasymsubseteq}\ {\isacharquery}{\kern0pt}Mi{\isachardoublequoteclose}\isanewline
\ \ \isacommand{proof}\isamarkupfalse%
\isanewline
\ \ \ \ \isacommand{fix}\isamarkupfalse%
\ y\ \isacommand{assume}\isamarkupfalse%
\ {\isachardoublequoteopen}y\ {\isasymin}\ {\isacharquery}{\kern0pt}iM{\isachardoublequoteclose}\isanewline
\ \ \ \ \isacommand{then}\isamarkupfalse%
\ \isacommand{obtain}\isamarkupfalse%
\ j\ \isakeyword{where}\ {\isacharbrackleft}{\kern0pt}simp{\isacharbrackright}{\kern0pt}{\isacharcolon}{\kern0pt}\ {\isachardoublequoteopen}y\ {\isacharequal}{\kern0pt}\ {\isacharparenleft}{\kern0pt}k{\isacharcomma}{\kern0pt}j{\isacharparenright}{\kern0pt}{\isachardoublequoteclose}\ \isakeyword{and}\ {\isachardoublequoteopen}j\ {\isasymin}\ insert\ x\ M{\isachardoublequoteclose}\ \isacommand{by}\isamarkupfalse%
\ blast\isanewline
\ \ \ \ \isacommand{then}\isamarkupfalse%
\ \isacommand{have}\isamarkupfalse%
\ {\isachardoublequoteopen}j\ {\isacharequal}{\kern0pt}\ x\ {\isasymor}\ j\ {\isasymin}\ M{\isachardoublequoteclose}\ \isacommand{by}\isamarkupfalse%
\ auto\isanewline
\ \ \ \ \isacommand{then}\isamarkupfalse%
\ \isacommand{show}\isamarkupfalse%
\ {\isachardoublequoteopen}y\ {\isasymin}\ {\isacharquery}{\kern0pt}Mi{\isachardoublequoteclose}\ \isacommand{by}\isamarkupfalse%
\ {\isacharparenleft}{\kern0pt}elim\ disjE{\isacharparenright}{\kern0pt}\ auto\isanewline
\ \ \isacommand{qed}\isamarkupfalse%
\isanewline
\isacommand{qed}\isamarkupfalse%
%
\endisatagproof
{\isafoldproof}%
%
\isadelimproof
\isanewline
%
\endisadelimproof
\isanewline
\isacommand{lemma}\isamarkupfalse%
\ finite{\isacharunderscore}{\kern0pt}card{\isacharunderscore}{\kern0pt}set{\isacharunderscore}{\kern0pt}comp{\isacharcolon}{\kern0pt}\ {\isachardoublequoteopen}finite\ M\ {\isasymLongrightarrow}\ card\ {\isacharbraceleft}{\kern0pt}{\isacharparenleft}{\kern0pt}k{\isacharcomma}{\kern0pt}j{\isacharparenright}{\kern0pt}\ {\isacharbar}{\kern0pt}j{\isachardot}{\kern0pt}\ j\ {\isasymin}\ M{\isacharbraceright}{\kern0pt}\ {\isacharequal}{\kern0pt}\ card\ M{\isachardoublequoteclose}\isanewline
%
\isadelimproof
%
\endisadelimproof
%
\isatagproof
\isacommand{proof}\isamarkupfalse%
\ {\isacharparenleft}{\kern0pt}induction\ M\ rule{\isacharcolon}{\kern0pt}\ finite{\isacharunderscore}{\kern0pt}induct{\isacharparenright}{\kern0pt}\isanewline
\ \ \isacommand{case}\isamarkupfalse%
\ {\isacharparenleft}{\kern0pt}insert\ x\ M{\isacharparenright}{\kern0pt}\isanewline
\ \ \isacommand{then}\isamarkupfalse%
\ \isacommand{show}\isamarkupfalse%
\ {\isacharquery}{\kern0pt}case\ \isacommand{using}\isamarkupfalse%
\ set{\isacharunderscore}{\kern0pt}comp{\isacharunderscore}{\kern0pt}ins{\isacharbrackleft}{\kern0pt}of\ k\ x\ M{\isacharbrackright}{\kern0pt}\ \isacommand{by}\isamarkupfalse%
\ auto\isanewline
\isacommand{qed}\isamarkupfalse%
\ auto%
\endisatagproof
{\isafoldproof}%
%
\isadelimproof
\isanewline
%
\endisadelimproof
\isanewline
\isacommand{lemma}\isamarkupfalse%
\ card{\isacharunderscore}{\kern0pt}board{\isacharunderscore}{\kern0pt}exec{\isacharunderscore}{\kern0pt}aux{\isacharcolon}{\kern0pt}\ {\isachardoublequoteopen}finite\ M\ {\isasymLongrightarrow}\ card\ {\isacharparenleft}{\kern0pt}board{\isacharunderscore}{\kern0pt}exec{\isacharunderscore}{\kern0pt}aux\ k\ M{\isacharparenright}{\kern0pt}\ {\isacharequal}{\kern0pt}\ k\ {\isacharasterisk}{\kern0pt}\ card\ M{\isachardoublequoteclose}\isanewline
%
\isadelimproof
%
\endisadelimproof
%
\isatagproof
\isacommand{proof}\isamarkupfalse%
\ {\isacharparenleft}{\kern0pt}induction\ k{\isacharparenright}{\kern0pt}\isanewline
\ \ \isacommand{case}\isamarkupfalse%
\ {\isacharparenleft}{\kern0pt}Suc\ k{\isacharparenright}{\kern0pt}\isanewline
\ \ \isacommand{let}\isamarkupfalse%
\ {\isacharquery}{\kern0pt}M{\isacharprime}{\kern0pt}{\isacharequal}{\kern0pt}{\isachardoublequoteopen}{\isacharbraceleft}{\kern0pt}{\isacharparenleft}{\kern0pt}int\ {\isacharparenleft}{\kern0pt}Suc\ k{\isacharparenright}{\kern0pt}{\isacharcomma}{\kern0pt}j{\isacharparenright}{\kern0pt}\ {\isacharbar}{\kern0pt}j{\isachardot}{\kern0pt}\ j\ {\isasymin}\ M{\isacharbraceright}{\kern0pt}{\isachardoublequoteclose}\isanewline
\ \ \isacommand{let}\isamarkupfalse%
\ {\isacharquery}{\kern0pt}rec{\isacharunderscore}{\kern0pt}k{\isacharequal}{\kern0pt}{\isachardoublequoteopen}board{\isacharunderscore}{\kern0pt}exec{\isacharunderscore}{\kern0pt}aux\ k\ M{\isachardoublequoteclose}\isanewline
\isanewline
\ \ \isacommand{have}\isamarkupfalse%
\ finite{\isacharcolon}{\kern0pt}\ {\isachardoublequoteopen}finite\ {\isacharquery}{\kern0pt}M{\isacharprime}{\kern0pt}{\isachardoublequoteclose}\ {\isachardoublequoteopen}finite\ {\isacharquery}{\kern0pt}rec{\isacharunderscore}{\kern0pt}k{\isachardoublequoteclose}\isanewline
\ \ \ \ \isacommand{using}\isamarkupfalse%
\ Suc\ finite{\isacharunderscore}{\kern0pt}board{\isacharunderscore}{\kern0pt}exec{\isacharunderscore}{\kern0pt}aux\ \isacommand{by}\isamarkupfalse%
\ auto\isanewline
\ \ \isacommand{then}\isamarkupfalse%
\ \isacommand{have}\isamarkupfalse%
\ card{\isacharunderscore}{\kern0pt}Un{\isacharunderscore}{\kern0pt}simp{\isacharcolon}{\kern0pt}\ {\isachardoublequoteopen}card\ {\isacharparenleft}{\kern0pt}{\isacharquery}{\kern0pt}M{\isacharprime}{\kern0pt}\ {\isasymunion}\ {\isacharquery}{\kern0pt}rec{\isacharunderscore}{\kern0pt}k{\isacharparenright}{\kern0pt}\ {\isacharequal}{\kern0pt}\ card\ {\isacharquery}{\kern0pt}M{\isacharprime}{\kern0pt}\ {\isacharplus}{\kern0pt}\ card\ {\isacharquery}{\kern0pt}rec{\isacharunderscore}{\kern0pt}k{\isachardoublequoteclose}\isanewline
\ \ \ \ \isacommand{using}\isamarkupfalse%
\ board{\isacharunderscore}{\kern0pt}exec{\isacharunderscore}{\kern0pt}aux{\isacharunderscore}{\kern0pt}leq{\isacharunderscore}{\kern0pt}mem\ card{\isacharunderscore}{\kern0pt}Un{\isacharunderscore}{\kern0pt}Int{\isacharbrackleft}{\kern0pt}of\ {\isacharquery}{\kern0pt}M{\isacharprime}{\kern0pt}\ {\isacharquery}{\kern0pt}rec{\isacharunderscore}{\kern0pt}k{\isacharbrackright}{\kern0pt}\ \isacommand{by}\isamarkupfalse%
\ auto\isanewline
\ \ \isanewline
\ \ \isacommand{have}\isamarkupfalse%
\ card{\isacharunderscore}{\kern0pt}M{\isacharcolon}{\kern0pt}\ {\isachardoublequoteopen}card\ {\isacharquery}{\kern0pt}M{\isacharprime}{\kern0pt}\ {\isacharequal}{\kern0pt}\ card\ M{\isachardoublequoteclose}\isanewline
\ \ \ \ \isacommand{using}\isamarkupfalse%
\ Suc\ finite{\isacharunderscore}{\kern0pt}card{\isacharunderscore}{\kern0pt}set{\isacharunderscore}{\kern0pt}comp\ \isacommand{by}\isamarkupfalse%
\ auto\isanewline
\ \ \isanewline
\ \ \isacommand{have}\isamarkupfalse%
\ {\isachardoublequoteopen}card\ {\isacharparenleft}{\kern0pt}board{\isacharunderscore}{\kern0pt}exec{\isacharunderscore}{\kern0pt}aux\ {\isacharparenleft}{\kern0pt}Suc\ k{\isacharparenright}{\kern0pt}\ M{\isacharparenright}{\kern0pt}\ {\isacharequal}{\kern0pt}\ card\ {\isacharquery}{\kern0pt}M{\isacharprime}{\kern0pt}\ {\isacharplus}{\kern0pt}\ card\ {\isacharquery}{\kern0pt}rec{\isacharunderscore}{\kern0pt}k{\isachardoublequoteclose}\isanewline
\ \ \ \ \isacommand{using}\isamarkupfalse%
\ card{\isacharunderscore}{\kern0pt}Un{\isacharunderscore}{\kern0pt}simp\ \isacommand{by}\isamarkupfalse%
\ auto\isanewline
\ \ \isacommand{also}\isamarkupfalse%
\ \isacommand{have}\isamarkupfalse%
\ {\isachardoublequoteopen}{\isachardot}{\kern0pt}{\isachardot}{\kern0pt}{\isachardot}{\kern0pt}\ {\isacharequal}{\kern0pt}\ card\ M\ {\isacharplus}{\kern0pt}\ k\ {\isacharasterisk}{\kern0pt}\ card\ M{\isachardoublequoteclose}\isanewline
\ \ \ \ \isacommand{using}\isamarkupfalse%
\ Suc\ card{\isacharunderscore}{\kern0pt}M\ \isacommand{by}\isamarkupfalse%
\ auto\isanewline
\ \ \isacommand{also}\isamarkupfalse%
\ \isacommand{have}\isamarkupfalse%
\ {\isachardoublequoteopen}{\isachardot}{\kern0pt}{\isachardot}{\kern0pt}{\isachardot}{\kern0pt}\ {\isacharequal}{\kern0pt}\ {\isacharparenleft}{\kern0pt}Suc\ k{\isacharparenright}{\kern0pt}\ {\isacharasterisk}{\kern0pt}\ card\ M{\isachardoublequoteclose}\isanewline
\ \ \ \ \isacommand{by}\isamarkupfalse%
\ auto\isanewline
\ \ \isacommand{finally}\isamarkupfalse%
\ \isacommand{show}\isamarkupfalse%
\ {\isacharquery}{\kern0pt}case\ \isacommand{{\isachardot}{\kern0pt}}\isamarkupfalse%
\isanewline
\isacommand{qed}\isamarkupfalse%
\ auto%
\endisatagproof
{\isafoldproof}%
%
\isadelimproof
\isanewline
%
\endisadelimproof
\isanewline
\isacommand{lemma}\isamarkupfalse%
\ card{\isacharunderscore}{\kern0pt}board{\isacharcolon}{\kern0pt}\ {\isachardoublequoteopen}card\ {\isacharparenleft}{\kern0pt}board\ n\ m{\isacharparenright}{\kern0pt}\ {\isacharequal}{\kern0pt}\ n\ {\isacharasterisk}{\kern0pt}\ m{\isachardoublequoteclose}\isanewline
%
\isadelimproof
%
\endisadelimproof
%
\isatagproof
\isacommand{proof}\isamarkupfalse%
\ {\isacharminus}{\kern0pt}\isanewline
\ \ \isacommand{have}\isamarkupfalse%
\ {\isachardoublequoteopen}card\ {\isacharparenleft}{\kern0pt}board\ n\ m{\isacharparenright}{\kern0pt}\ {\isacharequal}{\kern0pt}\ card\ {\isacharparenleft}{\kern0pt}board{\isacharunderscore}{\kern0pt}exec{\isacharunderscore}{\kern0pt}aux\ n\ {\isacharparenleft}{\kern0pt}row{\isacharunderscore}{\kern0pt}exec\ m{\isacharparenright}{\kern0pt}{\isacharparenright}{\kern0pt}{\isachardoublequoteclose}\isanewline
\ \ \ \ \isacommand{using}\isamarkupfalse%
\ board{\isacharunderscore}{\kern0pt}exec{\isacharunderscore}{\kern0pt}correct\ \isacommand{by}\isamarkupfalse%
\ auto\isanewline
\ \ \isacommand{also}\isamarkupfalse%
\ \isacommand{have}\isamarkupfalse%
\ {\isachardoublequoteopen}{\isachardot}{\kern0pt}{\isachardot}{\kern0pt}{\isachardot}{\kern0pt}\ {\isacharequal}{\kern0pt}\ n\ {\isacharasterisk}{\kern0pt}\ m{\isachardoublequoteclose}\isanewline
\ \ \ \ \isacommand{using}\isamarkupfalse%
\ card{\isacharunderscore}{\kern0pt}row{\isacharunderscore}{\kern0pt}exec\ card{\isacharunderscore}{\kern0pt}board{\isacharunderscore}{\kern0pt}exec{\isacharunderscore}{\kern0pt}aux\ finite{\isacharunderscore}{\kern0pt}row{\isacharunderscore}{\kern0pt}exec\ \isacommand{by}\isamarkupfalse%
\ auto\isanewline
\ \ \isacommand{finally}\isamarkupfalse%
\ \isacommand{show}\isamarkupfalse%
\ {\isacharquery}{\kern0pt}thesis\ \isacommand{{\isachardot}{\kern0pt}}\isamarkupfalse%
\isanewline
\isacommand{qed}\isamarkupfalse%
%
\endisatagproof
{\isafoldproof}%
%
\isadelimproof
\isanewline
%
\endisadelimproof
\isanewline
\isacommand{lemma}\isamarkupfalse%
\ knights{\isacharunderscore}{\kern0pt}path{\isacharunderscore}{\kern0pt}board{\isacharunderscore}{\kern0pt}non{\isacharunderscore}{\kern0pt}empty{\isacharcolon}{\kern0pt}\ {\isachardoublequoteopen}knights{\isacharunderscore}{\kern0pt}path\ b\ ps\ {\isasymLongrightarrow}\ b\ {\isasymnoteq}\ {\isacharbraceleft}{\kern0pt}{\isacharbraceright}{\kern0pt}{\isachardoublequoteclose}\isanewline
%
\isadelimproof
\ \ %
\endisadelimproof
%
\isatagproof
\isacommand{by}\isamarkupfalse%
\ {\isacharparenleft}{\kern0pt}induction\ arbitrary{\isacharcolon}{\kern0pt}\ ps\ rule{\isacharcolon}{\kern0pt}\ knights{\isacharunderscore}{\kern0pt}path{\isachardot}{\kern0pt}induct{\isacharparenright}{\kern0pt}\ auto%
\endisatagproof
{\isafoldproof}%
%
\isadelimproof
\isanewline
%
\endisadelimproof
\isanewline
\isacommand{lemma}\isamarkupfalse%
\ knights{\isacharunderscore}{\kern0pt}path{\isacharunderscore}{\kern0pt}board{\isacharunderscore}{\kern0pt}m{\isacharunderscore}{\kern0pt}n{\isacharunderscore}{\kern0pt}geq{\isacharunderscore}{\kern0pt}{\isadigit{1}}{\isacharcolon}{\kern0pt}\ {\isachardoublequoteopen}knights{\isacharunderscore}{\kern0pt}path\ {\isacharparenleft}{\kern0pt}board\ n\ m{\isacharparenright}{\kern0pt}\ ps\ {\isasymLongrightarrow}\ min\ n\ m\ {\isasymge}\ {\isadigit{1}}{\isachardoublequoteclose}\isanewline
%
\isadelimproof
\ \ %
\endisadelimproof
%
\isatagproof
\isacommand{unfolding}\isamarkupfalse%
\ board{\isacharunderscore}{\kern0pt}def\ \isacommand{using}\isamarkupfalse%
\ knights{\isacharunderscore}{\kern0pt}path{\isacharunderscore}{\kern0pt}board{\isacharunderscore}{\kern0pt}non{\isacharunderscore}{\kern0pt}empty\ \isacommand{by}\isamarkupfalse%
\ fastforce%
\endisatagproof
{\isafoldproof}%
%
\isadelimproof
\isanewline
%
\endisadelimproof
\isanewline
\isacommand{lemma}\isamarkupfalse%
\ knights{\isacharunderscore}{\kern0pt}path{\isacharunderscore}{\kern0pt}non{\isacharunderscore}{\kern0pt}nil{\isacharcolon}{\kern0pt}\ {\isachardoublequoteopen}knights{\isacharunderscore}{\kern0pt}path\ b\ ps\ {\isasymLongrightarrow}\ ps\ {\isasymnoteq}\ {\isacharbrackleft}{\kern0pt}{\isacharbrackright}{\kern0pt}{\isachardoublequoteclose}\isanewline
%
\isadelimproof
\ \ %
\endisadelimproof
%
\isatagproof
\isacommand{by}\isamarkupfalse%
\ {\isacharparenleft}{\kern0pt}induction\ arbitrary{\isacharcolon}{\kern0pt}\ b\ rule{\isacharcolon}{\kern0pt}\ knights{\isacharunderscore}{\kern0pt}path{\isachardot}{\kern0pt}induct{\isacharparenright}{\kern0pt}\ auto%
\endisatagproof
{\isafoldproof}%
%
\isadelimproof
\isanewline
%
\endisadelimproof
\isanewline
\isacommand{lemma}\isamarkupfalse%
\ knights{\isacharunderscore}{\kern0pt}path{\isacharunderscore}{\kern0pt}set{\isacharunderscore}{\kern0pt}eq{\isacharcolon}{\kern0pt}\ {\isachardoublequoteopen}knights{\isacharunderscore}{\kern0pt}path\ b\ ps\ {\isasymLongrightarrow}\ set\ ps\ {\isacharequal}{\kern0pt}\ b{\isachardoublequoteclose}\isanewline
%
\isadelimproof
\ \ %
\endisadelimproof
%
\isatagproof
\isacommand{by}\isamarkupfalse%
\ {\isacharparenleft}{\kern0pt}induction\ rule{\isacharcolon}{\kern0pt}\ knights{\isacharunderscore}{\kern0pt}path{\isachardot}{\kern0pt}induct{\isacharparenright}{\kern0pt}\ auto%
\endisatagproof
{\isafoldproof}%
%
\isadelimproof
\isanewline
%
\endisadelimproof
\isanewline
\isacommand{lemma}\isamarkupfalse%
\ knights{\isacharunderscore}{\kern0pt}path{\isacharunderscore}{\kern0pt}subset{\isacharcolon}{\kern0pt}\ \isanewline
\ \ {\isachardoublequoteopen}knights{\isacharunderscore}{\kern0pt}path\ b\isactrlsub {\isadigit{1}}\ ps\isactrlsub {\isadigit{1}}\ {\isasymLongrightarrow}\ knights{\isacharunderscore}{\kern0pt}path\ b\isactrlsub {\isadigit{2}}\ ps\isactrlsub {\isadigit{2}}\ {\isasymLongrightarrow}\ set\ ps\isactrlsub {\isadigit{1}}\ {\isasymsubseteq}\ set\ ps\isactrlsub {\isadigit{2}}\ {\isasymlongleftrightarrow}\ b\isactrlsub {\isadigit{1}}\ {\isasymsubseteq}\ b\isactrlsub {\isadigit{2}}{\isachardoublequoteclose}\isanewline
%
\isadelimproof
\ \ %
\endisadelimproof
%
\isatagproof
\isacommand{using}\isamarkupfalse%
\ knights{\isacharunderscore}{\kern0pt}path{\isacharunderscore}{\kern0pt}set{\isacharunderscore}{\kern0pt}eq\ \isacommand{by}\isamarkupfalse%
\ auto%
\endisatagproof
{\isafoldproof}%
%
\isadelimproof
\isanewline
%
\endisadelimproof
\isanewline
\isacommand{lemma}\isamarkupfalse%
\ knights{\isacharunderscore}{\kern0pt}path{\isacharunderscore}{\kern0pt}board{\isacharunderscore}{\kern0pt}unique{\isacharcolon}{\kern0pt}\ {\isachardoublequoteopen}knights{\isacharunderscore}{\kern0pt}path\ b\isactrlsub {\isadigit{1}}\ ps\ {\isasymLongrightarrow}\ knights{\isacharunderscore}{\kern0pt}path\ b\isactrlsub {\isadigit{2}}\ ps\ {\isasymLongrightarrow}\ b\isactrlsub {\isadigit{1}}\ {\isacharequal}{\kern0pt}\ b\isactrlsub {\isadigit{2}}{\isachardoublequoteclose}\isanewline
%
\isadelimproof
\ \ %
\endisadelimproof
%
\isatagproof
\isacommand{using}\isamarkupfalse%
\ knights{\isacharunderscore}{\kern0pt}path{\isacharunderscore}{\kern0pt}set{\isacharunderscore}{\kern0pt}eq\ \isacommand{by}\isamarkupfalse%
\ auto%
\endisatagproof
{\isafoldproof}%
%
\isadelimproof
\isanewline
%
\endisadelimproof
\isanewline
\isacommand{lemma}\isamarkupfalse%
\ valid{\isacharunderscore}{\kern0pt}step{\isacharunderscore}{\kern0pt}neq{\isacharcolon}{\kern0pt}\ {\isachardoublequoteopen}valid{\isacharunderscore}{\kern0pt}step\ s\isactrlsub i\ s\isactrlsub j\ {\isasymLongrightarrow}\ s\isactrlsub i\ {\isasymnoteq}\ s\isactrlsub j{\isachardoublequoteclose}\isanewline
%
\isadelimproof
\ \ %
\endisadelimproof
%
\isatagproof
\isacommand{unfolding}\isamarkupfalse%
\ valid{\isacharunderscore}{\kern0pt}step{\isacharunderscore}{\kern0pt}def\ \isacommand{by}\isamarkupfalse%
\ auto%
\endisatagproof
{\isafoldproof}%
%
\isadelimproof
\isanewline
%
\endisadelimproof
\isanewline
\isacommand{lemma}\isamarkupfalse%
\ valid{\isacharunderscore}{\kern0pt}step{\isacharunderscore}{\kern0pt}non{\isacharunderscore}{\kern0pt}transitive{\isacharcolon}{\kern0pt}\ {\isachardoublequoteopen}valid{\isacharunderscore}{\kern0pt}step\ s\isactrlsub i\ s\isactrlsub j\ {\isasymLongrightarrow}\ valid{\isacharunderscore}{\kern0pt}step\ s\isactrlsub j\ s\isactrlsub k\ {\isasymLongrightarrow}\ {\isasymnot}valid{\isacharunderscore}{\kern0pt}step\ s\isactrlsub i\ s\isactrlsub k{\isachardoublequoteclose}\isanewline
%
\isadelimproof
%
\endisadelimproof
%
\isatagproof
\isacommand{proof}\isamarkupfalse%
\ {\isacharminus}{\kern0pt}\isanewline
\ \ \isacommand{assume}\isamarkupfalse%
\ assms{\isacharcolon}{\kern0pt}\ {\isachardoublequoteopen}valid{\isacharunderscore}{\kern0pt}step\ s\isactrlsub i\ s\isactrlsub j{\isachardoublequoteclose}\ {\isachardoublequoteopen}valid{\isacharunderscore}{\kern0pt}step\ s\isactrlsub j\ s\isactrlsub k{\isachardoublequoteclose}\isanewline
\ \ \isacommand{obtain}\isamarkupfalse%
\ i\isactrlsub i\ j\isactrlsub i\ i\isactrlsub j\ j\isactrlsub j\ i\isactrlsub k\ j\isactrlsub k\ \isakeyword{where}\ {\isacharbrackleft}{\kern0pt}simp{\isacharbrackright}{\kern0pt}{\isacharcolon}{\kern0pt}\ {\isachardoublequoteopen}s\isactrlsub i\ {\isacharequal}{\kern0pt}\ {\isacharparenleft}{\kern0pt}i\isactrlsub i{\isacharcomma}{\kern0pt}j\isactrlsub i{\isacharparenright}{\kern0pt}{\isachardoublequoteclose}\ {\isachardoublequoteopen}s\isactrlsub j\ {\isacharequal}{\kern0pt}\ {\isacharparenleft}{\kern0pt}i\isactrlsub j{\isacharcomma}{\kern0pt}j\isactrlsub j{\isacharparenright}{\kern0pt}{\isachardoublequoteclose}\ {\isachardoublequoteopen}s\isactrlsub k\ {\isacharequal}{\kern0pt}\ {\isacharparenleft}{\kern0pt}i\isactrlsub k{\isacharcomma}{\kern0pt}j\isactrlsub k{\isacharparenright}{\kern0pt}{\isachardoublequoteclose}\ \isacommand{by}\isamarkupfalse%
\ force\isanewline
\ \ \isacommand{then}\isamarkupfalse%
\ \isacommand{have}\isamarkupfalse%
\ {\isachardoublequoteopen}step{\isacharunderscore}{\kern0pt}checker\ {\isacharparenleft}{\kern0pt}i\isactrlsub i{\isacharcomma}{\kern0pt}j\isactrlsub i{\isacharparenright}{\kern0pt}\ {\isacharparenleft}{\kern0pt}i\isactrlsub j{\isacharcomma}{\kern0pt}j\isactrlsub j{\isacharparenright}{\kern0pt}{\isachardoublequoteclose}\ {\isachardoublequoteopen}step{\isacharunderscore}{\kern0pt}checker\ {\isacharparenleft}{\kern0pt}i\isactrlsub j{\isacharcomma}{\kern0pt}j\isactrlsub j{\isacharparenright}{\kern0pt}\ {\isacharparenleft}{\kern0pt}i\isactrlsub k{\isacharcomma}{\kern0pt}j\isactrlsub k{\isacharparenright}{\kern0pt}{\isachardoublequoteclose}\ \isanewline
\ \ \ \ \isacommand{using}\isamarkupfalse%
\ assms\ step{\isacharunderscore}{\kern0pt}checker{\isacharunderscore}{\kern0pt}correct\ \isacommand{by}\isamarkupfalse%
\ auto\isanewline
\ \ \isacommand{then}\isamarkupfalse%
\ \isacommand{show}\isamarkupfalse%
\ {\isachardoublequoteopen}{\isasymnot}valid{\isacharunderscore}{\kern0pt}step\ s\isactrlsub i\ s\isactrlsub k{\isachardoublequoteclose}\isanewline
\ \ \ \ \isacommand{apply}\isamarkupfalse%
\ {\isacharparenleft}{\kern0pt}simp\ add{\isacharcolon}{\kern0pt}\ step{\isacharunderscore}{\kern0pt}checker{\isacharunderscore}{\kern0pt}correct{\isacharbrackleft}{\kern0pt}symmetric{\isacharbrackright}{\kern0pt}{\isacharparenright}{\kern0pt}\isanewline
\ \ \ \ \isacommand{apply}\isamarkupfalse%
\ {\isacharparenleft}{\kern0pt}elim\ disjE{\isacharparenright}{\kern0pt}\isanewline
\ \ \ \ \isacommand{apply}\isamarkupfalse%
\ auto\isanewline
\ \ \ \ \isacommand{done}\isamarkupfalse%
\isanewline
\isacommand{qed}\isamarkupfalse%
%
\endisatagproof
{\isafoldproof}%
%
\isadelimproof
\isanewline
%
\endisadelimproof
\isanewline
\isacommand{lemma}\isamarkupfalse%
\ knights{\isacharunderscore}{\kern0pt}path{\isacharunderscore}{\kern0pt}distinct{\isacharcolon}{\kern0pt}\ {\isachardoublequoteopen}knights{\isacharunderscore}{\kern0pt}path\ b\ ps\ {\isasymLongrightarrow}\ distinct\ ps{\isachardoublequoteclose}\isanewline
%
\isadelimproof
%
\endisadelimproof
%
\isatagproof
\isacommand{proof}\isamarkupfalse%
\ {\isacharparenleft}{\kern0pt}induction\ rule{\isacharcolon}{\kern0pt}\ knights{\isacharunderscore}{\kern0pt}path{\isachardot}{\kern0pt}induct{\isacharparenright}{\kern0pt}\isanewline
\ \ \isacommand{case}\isamarkupfalse%
\ {\isacharparenleft}{\kern0pt}{\isadigit{2}}\ s\isactrlsub i\ b\ s\isactrlsub j\ ps{\isacharparenright}{\kern0pt}\isanewline
\ \ \isacommand{then}\isamarkupfalse%
\ \isacommand{have}\isamarkupfalse%
\ {\isachardoublequoteopen}s\isactrlsub i\ {\isasymnotin}\ set\ {\isacharparenleft}{\kern0pt}s\isactrlsub j\ {\isacharhash}{\kern0pt}\ ps{\isacharparenright}{\kern0pt}{\isachardoublequoteclose}\isanewline
\ \ \ \ \isacommand{using}\isamarkupfalse%
\ knights{\isacharunderscore}{\kern0pt}path{\isacharunderscore}{\kern0pt}set{\isacharunderscore}{\kern0pt}eq\ valid{\isacharunderscore}{\kern0pt}step{\isacharunderscore}{\kern0pt}neq\ \isacommand{by}\isamarkupfalse%
\ blast\isanewline
\ \ \isacommand{then}\isamarkupfalse%
\ \isacommand{show}\isamarkupfalse%
\ {\isacharquery}{\kern0pt}case\ \isacommand{using}\isamarkupfalse%
\ {\isadigit{2}}\ \isacommand{by}\isamarkupfalse%
\ auto\isanewline
\isacommand{qed}\isamarkupfalse%
\ auto%
\endisatagproof
{\isafoldproof}%
%
\isadelimproof
\isanewline
%
\endisadelimproof
\isanewline
\isacommand{lemma}\isamarkupfalse%
\ knights{\isacharunderscore}{\kern0pt}path{\isacharunderscore}{\kern0pt}length{\isacharcolon}{\kern0pt}\ {\isachardoublequoteopen}knights{\isacharunderscore}{\kern0pt}path\ b\ ps\ {\isasymLongrightarrow}\ length\ ps\ {\isacharequal}{\kern0pt}\ card\ b{\isachardoublequoteclose}\isanewline
%
\isadelimproof
\ \ %
\endisadelimproof
%
\isatagproof
\isacommand{using}\isamarkupfalse%
\ knights{\isacharunderscore}{\kern0pt}path{\isacharunderscore}{\kern0pt}set{\isacharunderscore}{\kern0pt}eq\ knights{\isacharunderscore}{\kern0pt}path{\isacharunderscore}{\kern0pt}distinct\ \isacommand{by}\isamarkupfalse%
\ {\isacharparenleft}{\kern0pt}metis\ distinct{\isacharunderscore}{\kern0pt}card{\isacharparenright}{\kern0pt}%
\endisatagproof
{\isafoldproof}%
%
\isadelimproof
\isanewline
%
\endisadelimproof
\isanewline
\isacommand{lemma}\isamarkupfalse%
\ knights{\isacharunderscore}{\kern0pt}path{\isacharunderscore}{\kern0pt}take{\isacharcolon}{\kern0pt}\ \isanewline
\ \ \isakeyword{assumes}\ {\isachardoublequoteopen}knights{\isacharunderscore}{\kern0pt}path\ b\ ps{\isachardoublequoteclose}\ {\isachardoublequoteopen}{\isadigit{0}}\ {\isacharless}{\kern0pt}\ k{\isachardoublequoteclose}\ {\isachardoublequoteopen}k\ {\isacharless}{\kern0pt}\ length\ ps{\isachardoublequoteclose}\isanewline
\ \ \isakeyword{shows}\ {\isachardoublequoteopen}knights{\isacharunderscore}{\kern0pt}path\ {\isacharparenleft}{\kern0pt}set\ {\isacharparenleft}{\kern0pt}take\ k\ ps{\isacharparenright}{\kern0pt}{\isacharparenright}{\kern0pt}\ {\isacharparenleft}{\kern0pt}take\ k\ ps{\isacharparenright}{\kern0pt}{\isachardoublequoteclose}\isanewline
%
\isadelimproof
\ \ %
\endisadelimproof
%
\isatagproof
\isacommand{using}\isamarkupfalse%
\ assms\ \isanewline
\isacommand{proof}\isamarkupfalse%
\ {\isacharparenleft}{\kern0pt}induction\ arbitrary{\isacharcolon}{\kern0pt}\ k\ rule{\isacharcolon}{\kern0pt}\ knights{\isacharunderscore}{\kern0pt}path{\isachardot}{\kern0pt}induct{\isacharparenright}{\kern0pt}\isanewline
\ \ \isacommand{case}\isamarkupfalse%
\ {\isacharparenleft}{\kern0pt}{\isadigit{2}}\ s\isactrlsub i\ b\ s\isactrlsub j\ ps{\isacharparenright}{\kern0pt}\isanewline
\ \ \isacommand{then}\isamarkupfalse%
\ \isacommand{have}\isamarkupfalse%
\ {\isachardoublequoteopen}k\ {\isacharequal}{\kern0pt}\ {\isadigit{1}}\ {\isasymor}\ k\ {\isacharequal}{\kern0pt}\ {\isadigit{2}}\ {\isasymor}\ {\isadigit{2}}\ {\isacharless}{\kern0pt}\ k{\isachardoublequoteclose}\ \isacommand{by}\isamarkupfalse%
\ force\isanewline
\ \ \isacommand{then}\isamarkupfalse%
\ \isacommand{show}\isamarkupfalse%
\ {\isacharquery}{\kern0pt}case\isanewline
\ \ \ \ \isacommand{using}\isamarkupfalse%
\ {\isadigit{2}}\isanewline
\ \ \isacommand{proof}\isamarkupfalse%
\ {\isacharparenleft}{\kern0pt}elim\ disjE{\isacharparenright}{\kern0pt}\isanewline
\ \ \ \ \isacommand{assume}\isamarkupfalse%
\ {\isachardoublequoteopen}k\ {\isacharequal}{\kern0pt}\ {\isadigit{2}}{\isachardoublequoteclose}\isanewline
\ \ \ \ \isacommand{then}\isamarkupfalse%
\ \isacommand{have}\isamarkupfalse%
\ {\isachardoublequoteopen}take\ k\ {\isacharparenleft}{\kern0pt}s\isactrlsub i{\isacharhash}{\kern0pt}s\isactrlsub j{\isacharhash}{\kern0pt}ps{\isacharparenright}{\kern0pt}\ {\isacharequal}{\kern0pt}\ {\isacharbrackleft}{\kern0pt}s\isactrlsub i{\isacharcomma}{\kern0pt}s\isactrlsub j{\isacharbrackright}{\kern0pt}{\isachardoublequoteclose}\ {\isachardoublequoteopen}s\isactrlsub i\ {\isasymnotin}\ {\isacharbraceleft}{\kern0pt}s\isactrlsub j{\isacharbraceright}{\kern0pt}{\isachardoublequoteclose}\ \isacommand{using}\isamarkupfalse%
\ {\isadigit{2}}\ valid{\isacharunderscore}{\kern0pt}step{\isacharunderscore}{\kern0pt}neq\ \isacommand{by}\isamarkupfalse%
\ auto\isanewline
\ \ \ \ \isacommand{then}\isamarkupfalse%
\ \isacommand{show}\isamarkupfalse%
\ {\isacharquery}{\kern0pt}thesis\ \isacommand{using}\isamarkupfalse%
\ {\isadigit{2}}\ knights{\isacharunderscore}{\kern0pt}path{\isachardot}{\kern0pt}intros\ \isacommand{by}\isamarkupfalse%
\ auto\isanewline
\ \ \isacommand{next}\isamarkupfalse%
\isanewline
\ \ \ \ \isacommand{assume}\isamarkupfalse%
\ {\isachardoublequoteopen}{\isadigit{2}}\ {\isacharless}{\kern0pt}\ k{\isachardoublequoteclose}\isanewline
\ \ \ \ \isacommand{then}\isamarkupfalse%
\ \isacommand{have}\isamarkupfalse%
\ k{\isacharunderscore}{\kern0pt}simps{\isacharcolon}{\kern0pt}\ {\isachardoublequoteopen}k{\isacharminus}{\kern0pt}{\isadigit{2}}\ {\isacharequal}{\kern0pt}\ k{\isacharminus}{\kern0pt}{\isadigit{1}}{\isacharminus}{\kern0pt}{\isadigit{1}}{\isachardoublequoteclose}\ {\isachardoublequoteopen}{\isadigit{0}}\ {\isacharless}{\kern0pt}\ k{\isacharminus}{\kern0pt}{\isadigit{2}}{\isachardoublequoteclose}\ {\isachardoublequoteopen}k{\isacharminus}{\kern0pt}{\isadigit{2}}\ {\isacharless}{\kern0pt}\ length\ ps{\isachardoublequoteclose}\ \isakeyword{and}\isanewline
\ \ \ \ \ \ \ \ take{\isacharunderscore}{\kern0pt}simp{\isadigit{1}}{\isacharcolon}{\kern0pt}\ {\isachardoublequoteopen}take\ k\ {\isacharparenleft}{\kern0pt}s\isactrlsub i{\isacharhash}{\kern0pt}s\isactrlsub j{\isacharhash}{\kern0pt}ps{\isacharparenright}{\kern0pt}\ {\isacharequal}{\kern0pt}\ s\isactrlsub i{\isacharhash}{\kern0pt}take\ {\isacharparenleft}{\kern0pt}k{\isacharminus}{\kern0pt}{\isadigit{1}}{\isacharparenright}{\kern0pt}\ {\isacharparenleft}{\kern0pt}s\isactrlsub j{\isacharhash}{\kern0pt}ps{\isacharparenright}{\kern0pt}{\isachardoublequoteclose}\ \isakeyword{and}\isanewline
\ \ \ \ \ \ \ \ take{\isacharunderscore}{\kern0pt}simp{\isadigit{2}}{\isacharcolon}{\kern0pt}\ {\isachardoublequoteopen}take\ k\ {\isacharparenleft}{\kern0pt}s\isactrlsub i{\isacharhash}{\kern0pt}s\isactrlsub j{\isacharhash}{\kern0pt}ps{\isacharparenright}{\kern0pt}\ {\isacharequal}{\kern0pt}\ s\isactrlsub i{\isacharhash}{\kern0pt}s\isactrlsub j{\isacharhash}{\kern0pt}take\ {\isacharparenleft}{\kern0pt}k{\isacharminus}{\kern0pt}{\isadigit{1}}{\isacharminus}{\kern0pt}{\isadigit{1}}{\isacharparenright}{\kern0pt}\ ps{\isachardoublequoteclose}\isanewline
\ \ \ \ \ \ \isacommand{using}\isamarkupfalse%
\ assms\ {\isadigit{2}}\ take{\isacharunderscore}{\kern0pt}Cons{\isacharprime}{\kern0pt}{\isacharbrackleft}{\kern0pt}of\ k\ s\isactrlsub i\ {\isachardoublequoteopen}s\isactrlsub j{\isacharhash}{\kern0pt}ps{\isachardoublequoteclose}{\isacharbrackright}{\kern0pt}\ take{\isacharunderscore}{\kern0pt}Cons{\isacharprime}{\kern0pt}{\isacharbrackleft}{\kern0pt}of\ {\isachardoublequoteopen}k{\isacharminus}{\kern0pt}{\isadigit{1}}{\isachardoublequoteclose}\ s\isactrlsub j\ ps{\isacharbrackright}{\kern0pt}\ \isacommand{by}\isamarkupfalse%
\ auto\isanewline
\ \ \ \ \isacommand{then}\isamarkupfalse%
\ \isacommand{have}\isamarkupfalse%
\ {\isachardoublequoteopen}knights{\isacharunderscore}{\kern0pt}path\ {\isacharparenleft}{\kern0pt}set\ {\isacharparenleft}{\kern0pt}take\ {\isacharparenleft}{\kern0pt}k{\isacharminus}{\kern0pt}{\isadigit{1}}{\isacharparenright}{\kern0pt}\ {\isacharparenleft}{\kern0pt}s\isactrlsub j{\isacharhash}{\kern0pt}ps{\isacharparenright}{\kern0pt}{\isacharparenright}{\kern0pt}{\isacharparenright}{\kern0pt}\ {\isacharparenleft}{\kern0pt}take\ {\isacharparenleft}{\kern0pt}k{\isacharminus}{\kern0pt}{\isadigit{1}}{\isacharparenright}{\kern0pt}\ {\isacharparenleft}{\kern0pt}s\isactrlsub j{\isacharhash}{\kern0pt}ps{\isacharparenright}{\kern0pt}{\isacharparenright}{\kern0pt}{\isachardoublequoteclose}\isanewline
\ \ \ \ \ \ \isacommand{using}\isamarkupfalse%
\ {\isadigit{2}}\ k{\isacharunderscore}{\kern0pt}simps\ \isacommand{by}\isamarkupfalse%
\ auto\isanewline
\ \ \ \ \isacommand{then}\isamarkupfalse%
\ \isacommand{have}\isamarkupfalse%
\ kp{\isacharcolon}{\kern0pt}\ {\isachardoublequoteopen}knights{\isacharunderscore}{\kern0pt}path\ {\isacharparenleft}{\kern0pt}set\ {\isacharparenleft}{\kern0pt}take\ {\isacharparenleft}{\kern0pt}k{\isacharminus}{\kern0pt}{\isadigit{1}}{\isacharparenright}{\kern0pt}\ {\isacharparenleft}{\kern0pt}s\isactrlsub j{\isacharhash}{\kern0pt}ps{\isacharparenright}{\kern0pt}{\isacharparenright}{\kern0pt}{\isacharparenright}{\kern0pt}\ {\isacharparenleft}{\kern0pt}s\isactrlsub j{\isacharhash}{\kern0pt}take\ {\isacharparenleft}{\kern0pt}k{\isacharminus}{\kern0pt}{\isadigit{2}}{\isacharparenright}{\kern0pt}\ ps{\isacharparenright}{\kern0pt}{\isachardoublequoteclose}\isanewline
\ \ \ \ \ \ \isacommand{using}\isamarkupfalse%
\ take{\isacharunderscore}{\kern0pt}Cons{\isacharprime}{\kern0pt}{\isacharbrackleft}{\kern0pt}of\ {\isachardoublequoteopen}k{\isacharminus}{\kern0pt}{\isadigit{1}}{\isachardoublequoteclose}\ s\isactrlsub j\ ps{\isacharbrackright}{\kern0pt}\ \isacommand{by}\isamarkupfalse%
\ {\isacharparenleft}{\kern0pt}auto\ simp{\isacharcolon}{\kern0pt}\ k{\isacharunderscore}{\kern0pt}simps\ elim{\isacharcolon}{\kern0pt}\ knights{\isacharunderscore}{\kern0pt}path{\isachardot}{\kern0pt}cases{\isacharparenright}{\kern0pt}\isanewline
\isanewline
\ \ \ \ \isacommand{have}\isamarkupfalse%
\ no{\isacharunderscore}{\kern0pt}mem{\isacharcolon}{\kern0pt}\ {\isachardoublequoteopen}s\isactrlsub i\ {\isasymnotin}\ set\ {\isacharparenleft}{\kern0pt}take\ {\isacharparenleft}{\kern0pt}k{\isacharminus}{\kern0pt}{\isadigit{1}}{\isacharparenright}{\kern0pt}\ {\isacharparenleft}{\kern0pt}s\isactrlsub j{\isacharhash}{\kern0pt}ps{\isacharparenright}{\kern0pt}{\isacharparenright}{\kern0pt}{\isachardoublequoteclose}\isanewline
\ \ \ \ \ \ \isacommand{using}\isamarkupfalse%
\ {\isadigit{2}}\ set{\isacharunderscore}{\kern0pt}take{\isacharunderscore}{\kern0pt}subset{\isacharbrackleft}{\kern0pt}of\ {\isachardoublequoteopen}k{\isacharminus}{\kern0pt}{\isadigit{1}}{\isachardoublequoteclose}\ {\isachardoublequoteopen}s\isactrlsub j{\isacharhash}{\kern0pt}ps{\isachardoublequoteclose}{\isacharbrackright}{\kern0pt}\ knights{\isacharunderscore}{\kern0pt}path{\isacharunderscore}{\kern0pt}set{\isacharunderscore}{\kern0pt}eq\ \isacommand{by}\isamarkupfalse%
\ blast\isanewline
\ \ \ \ \isacommand{have}\isamarkupfalse%
\ {\isachardoublequoteopen}knights{\isacharunderscore}{\kern0pt}path\ {\isacharparenleft}{\kern0pt}set\ {\isacharparenleft}{\kern0pt}take\ {\isacharparenleft}{\kern0pt}k{\isacharminus}{\kern0pt}{\isadigit{1}}{\isacharparenright}{\kern0pt}\ {\isacharparenleft}{\kern0pt}s\isactrlsub j{\isacharhash}{\kern0pt}ps{\isacharparenright}{\kern0pt}{\isacharparenright}{\kern0pt}\ {\isasymunion}\ {\isacharbraceleft}{\kern0pt}s\isactrlsub i{\isacharbraceright}{\kern0pt}{\isacharparenright}{\kern0pt}\ {\isacharparenleft}{\kern0pt}s\isactrlsub i{\isacharhash}{\kern0pt}s\isactrlsub j{\isacharhash}{\kern0pt}take\ {\isacharparenleft}{\kern0pt}k{\isacharminus}{\kern0pt}{\isadigit{2}}{\isacharparenright}{\kern0pt}\ ps{\isacharparenright}{\kern0pt}{\isachardoublequoteclose}\isanewline
\ \ \ \ \ \ \isacommand{using}\isamarkupfalse%
\ knights{\isacharunderscore}{\kern0pt}path{\isachardot}{\kern0pt}intros{\isacharparenleft}{\kern0pt}{\isadigit{2}}{\isacharparenright}{\kern0pt}{\isacharbrackleft}{\kern0pt}OF\ no{\isacharunderscore}{\kern0pt}mem\ {\isacartoucheopen}valid{\isacharunderscore}{\kern0pt}step\ s\isactrlsub i\ s\isactrlsub j{\isacartoucheclose}\ kp{\isacharbrackright}{\kern0pt}\ \isacommand{by}\isamarkupfalse%
\ auto\isanewline
\ \ \ \ \isacommand{then}\isamarkupfalse%
\ \isacommand{show}\isamarkupfalse%
\ {\isacharquery}{\kern0pt}thesis\ \isacommand{using}\isamarkupfalse%
\ k{\isacharunderscore}{\kern0pt}simps\ take{\isacharunderscore}{\kern0pt}simp{\isadigit{2}}\ knights{\isacharunderscore}{\kern0pt}path{\isacharunderscore}{\kern0pt}set{\isacharunderscore}{\kern0pt}eq\ \isacommand{by}\isamarkupfalse%
\ metis\isanewline
\ \ \isacommand{qed}\isamarkupfalse%
\ {\isacharparenleft}{\kern0pt}auto\ intro{\isacharcolon}{\kern0pt}\ knights{\isacharunderscore}{\kern0pt}path{\isachardot}{\kern0pt}intros{\isacharparenright}{\kern0pt}\isanewline
\isacommand{qed}\isamarkupfalse%
\ auto%
\endisatagproof
{\isafoldproof}%
%
\isadelimproof
\isanewline
%
\endisadelimproof
\isanewline
\isacommand{lemma}\isamarkupfalse%
\ knights{\isacharunderscore}{\kern0pt}path{\isacharunderscore}{\kern0pt}drop{\isacharcolon}{\kern0pt}\ \isanewline
\ \ \isakeyword{assumes}\ {\isachardoublequoteopen}knights{\isacharunderscore}{\kern0pt}path\ b\ ps{\isachardoublequoteclose}\ {\isachardoublequoteopen}{\isadigit{0}}\ {\isacharless}{\kern0pt}\ k{\isachardoublequoteclose}\ {\isachardoublequoteopen}k\ {\isacharless}{\kern0pt}\ length\ ps{\isachardoublequoteclose}\isanewline
\ \ \isakeyword{shows}\ {\isachardoublequoteopen}knights{\isacharunderscore}{\kern0pt}path\ {\isacharparenleft}{\kern0pt}set\ {\isacharparenleft}{\kern0pt}drop\ k\ ps{\isacharparenright}{\kern0pt}{\isacharparenright}{\kern0pt}\ {\isacharparenleft}{\kern0pt}drop\ k\ ps{\isacharparenright}{\kern0pt}{\isachardoublequoteclose}\isanewline
%
\isadelimproof
\ \ %
\endisadelimproof
%
\isatagproof
\isacommand{using}\isamarkupfalse%
\ assms\isanewline
\isacommand{proof}\isamarkupfalse%
\ {\isacharparenleft}{\kern0pt}induction\ arbitrary{\isacharcolon}{\kern0pt}\ k\ rule{\isacharcolon}{\kern0pt}\ knights{\isacharunderscore}{\kern0pt}path{\isachardot}{\kern0pt}induct{\isacharparenright}{\kern0pt}\isanewline
\ \ \isacommand{case}\isamarkupfalse%
\ {\isacharparenleft}{\kern0pt}{\isadigit{2}}\ s\isactrlsub i\ b\ s\isactrlsub j\ ps{\isacharparenright}{\kern0pt}\isanewline
\ \ \isacommand{then}\isamarkupfalse%
\ \isacommand{have}\isamarkupfalse%
\ {\isachardoublequoteopen}{\isacharparenleft}{\kern0pt}k\ {\isacharequal}{\kern0pt}\ {\isadigit{1}}\ {\isasymand}\ ps\ {\isacharequal}{\kern0pt}\ {\isacharbrackleft}{\kern0pt}{\isacharbrackright}{\kern0pt}{\isacharparenright}{\kern0pt}\ {\isasymor}\ {\isacharparenleft}{\kern0pt}k\ {\isacharequal}{\kern0pt}\ {\isadigit{1}}\ {\isasymand}\ ps\ {\isasymnoteq}\ {\isacharbrackleft}{\kern0pt}{\isacharbrackright}{\kern0pt}{\isacharparenright}{\kern0pt}\ {\isasymor}\ {\isadigit{1}}\ {\isacharless}{\kern0pt}\ k{\isachardoublequoteclose}\ \isacommand{by}\isamarkupfalse%
\ force\isanewline
\ \ \isacommand{then}\isamarkupfalse%
\ \isacommand{show}\isamarkupfalse%
\ {\isacharquery}{\kern0pt}case\isanewline
\ \ \ \ \isacommand{using}\isamarkupfalse%
\ {\isadigit{2}}\isanewline
\ \ \isacommand{proof}\isamarkupfalse%
\ {\isacharparenleft}{\kern0pt}elim\ disjE{\isacharparenright}{\kern0pt}\isanewline
\ \ \ \ \isacommand{assume}\isamarkupfalse%
\ {\isachardoublequoteopen}k\ {\isacharequal}{\kern0pt}\ {\isadigit{1}}\ {\isasymand}\ ps\ {\isasymnoteq}\ {\isacharbrackleft}{\kern0pt}{\isacharbrackright}{\kern0pt}{\isachardoublequoteclose}\isanewline
\ \ \ \ \isacommand{then}\isamarkupfalse%
\ \isacommand{show}\isamarkupfalse%
\ {\isacharquery}{\kern0pt}thesis\ \isacommand{using}\isamarkupfalse%
\ {\isadigit{2}}\ knights{\isacharunderscore}{\kern0pt}path{\isacharunderscore}{\kern0pt}set{\isacharunderscore}{\kern0pt}eq\ \isacommand{by}\isamarkupfalse%
\ force\isanewline
\ \ \isacommand{next}\isamarkupfalse%
\isanewline
\ \ \ \ \isacommand{assume}\isamarkupfalse%
\ {\isachardoublequoteopen}{\isadigit{1}}\ {\isacharless}{\kern0pt}\ k{\isachardoublequoteclose}\isanewline
\ \ \ \ \isacommand{then}\isamarkupfalse%
\ \isacommand{have}\isamarkupfalse%
\ {\isachardoublequoteopen}{\isadigit{0}}\ {\isacharless}{\kern0pt}\ k{\isacharminus}{\kern0pt}{\isadigit{1}}{\isachardoublequoteclose}\ {\isachardoublequoteopen}k{\isacharminus}{\kern0pt}{\isadigit{1}}\ {\isacharless}{\kern0pt}\ length\ {\isacharparenleft}{\kern0pt}s\isactrlsub j{\isacharhash}{\kern0pt}ps{\isacharparenright}{\kern0pt}{\isachardoublequoteclose}\ {\isachardoublequoteopen}drop\ k\ {\isacharparenleft}{\kern0pt}s\isactrlsub i{\isacharhash}{\kern0pt}s\isactrlsub j{\isacharhash}{\kern0pt}ps{\isacharparenright}{\kern0pt}\ {\isacharequal}{\kern0pt}\ drop\ {\isacharparenleft}{\kern0pt}k{\isacharminus}{\kern0pt}{\isadigit{1}}{\isacharparenright}{\kern0pt}\ {\isacharparenleft}{\kern0pt}s\isactrlsub j{\isacharhash}{\kern0pt}ps{\isacharparenright}{\kern0pt}{\isachardoublequoteclose}\ \isanewline
\ \ \ \ \ \ \isacommand{using}\isamarkupfalse%
\ assms\ {\isadigit{2}}\ drop{\isacharunderscore}{\kern0pt}Cons{\isacharprime}{\kern0pt}{\isacharbrackleft}{\kern0pt}of\ k\ s\isactrlsub i\ {\isachardoublequoteopen}s\isactrlsub j{\isacharhash}{\kern0pt}ps{\isachardoublequoteclose}{\isacharbrackright}{\kern0pt}\ \isacommand{by}\isamarkupfalse%
\ auto\isanewline
\ \ \ \ \isacommand{then}\isamarkupfalse%
\ \isacommand{show}\isamarkupfalse%
\ {\isacharquery}{\kern0pt}thesis\isanewline
\ \ \ \ \ \ \isacommand{using}\isamarkupfalse%
\ {\isadigit{2}}\ \isacommand{by}\isamarkupfalse%
\ auto\isanewline
\ \ \isacommand{qed}\isamarkupfalse%
\ {\isacharparenleft}{\kern0pt}auto\ intro{\isacharcolon}{\kern0pt}\ knights{\isacharunderscore}{\kern0pt}path{\isachardot}{\kern0pt}intros{\isacharparenright}{\kern0pt}\isanewline
\isacommand{qed}\isamarkupfalse%
\ auto%
\endisatagproof
{\isafoldproof}%
%
\isadelimproof
%
\endisadelimproof
%
\begin{isamarkuptext}%
A Knight's path can be split to form two new disjoint Knight's paths.%
\end{isamarkuptext}\isamarkuptrue%
\isacommand{corollary}\isamarkupfalse%
\ knights{\isacharunderscore}{\kern0pt}path{\isacharunderscore}{\kern0pt}split{\isacharcolon}{\kern0pt}\ \ \ \ \ \ \ \ \ \ \ \ \ \isanewline
\ \ \isakeyword{assumes}\ {\isachardoublequoteopen}knights{\isacharunderscore}{\kern0pt}path\ b\ ps{\isachardoublequoteclose}\ {\isachardoublequoteopen}{\isadigit{0}}\ {\isacharless}{\kern0pt}\ k{\isachardoublequoteclose}\ {\isachardoublequoteopen}k\ {\isacharless}{\kern0pt}\ length\ ps{\isachardoublequoteclose}\isanewline
\ \ \isakeyword{shows}\ \isanewline
\ \ \ \ {\isachardoublequoteopen}{\isasymexists}b\isactrlsub {\isadigit{1}}\ b\isactrlsub {\isadigit{2}}{\isachardot}{\kern0pt}\ knights{\isacharunderscore}{\kern0pt}path\ b\isactrlsub {\isadigit{1}}\ {\isacharparenleft}{\kern0pt}take\ k\ ps{\isacharparenright}{\kern0pt}\ {\isasymand}\ knights{\isacharunderscore}{\kern0pt}path\ b\isactrlsub {\isadigit{2}}\ {\isacharparenleft}{\kern0pt}drop\ k\ ps{\isacharparenright}{\kern0pt}\ {\isasymand}\ b\isactrlsub {\isadigit{1}}\ {\isasymunion}\ b\isactrlsub {\isadigit{2}}\ {\isacharequal}{\kern0pt}\ b\ {\isasymand}\ b\isactrlsub {\isadigit{1}}\ {\isasyminter}\ b\isactrlsub {\isadigit{2}}\ {\isacharequal}{\kern0pt}\ {\isacharbraceleft}{\kern0pt}{\isacharbraceright}{\kern0pt}{\isachardoublequoteclose}\isanewline
%
\isadelimproof
\ \ %
\endisadelimproof
%
\isatagproof
\isacommand{using}\isamarkupfalse%
\ assms\isanewline
\isacommand{proof}\isamarkupfalse%
\ {\isacharminus}{\kern0pt}\isanewline
\ \ \isacommand{let}\isamarkupfalse%
\ {\isacharquery}{\kern0pt}b\isactrlsub {\isadigit{1}}{\isacharequal}{\kern0pt}{\isachardoublequoteopen}set\ {\isacharparenleft}{\kern0pt}take\ k\ ps{\isacharparenright}{\kern0pt}{\isachardoublequoteclose}\ \isanewline
\ \ \isacommand{let}\isamarkupfalse%
\ {\isacharquery}{\kern0pt}b\isactrlsub {\isadigit{2}}{\isacharequal}{\kern0pt}{\isachardoublequoteopen}set\ {\isacharparenleft}{\kern0pt}drop\ k\ ps{\isacharparenright}{\kern0pt}{\isachardoublequoteclose}\isanewline
\ \ \isacommand{have}\isamarkupfalse%
\ kp{\isadigit{1}}{\isacharcolon}{\kern0pt}\ {\isachardoublequoteopen}knights{\isacharunderscore}{\kern0pt}path\ {\isacharquery}{\kern0pt}b\isactrlsub {\isadigit{1}}\ {\isacharparenleft}{\kern0pt}take\ k\ ps{\isacharparenright}{\kern0pt}{\isachardoublequoteclose}\ \isakeyword{and}\ kp{\isadigit{2}}{\isacharcolon}{\kern0pt}\ {\isachardoublequoteopen}knights{\isacharunderscore}{\kern0pt}path\ {\isacharquery}{\kern0pt}b\isactrlsub {\isadigit{2}}\ {\isacharparenleft}{\kern0pt}drop\ k\ ps{\isacharparenright}{\kern0pt}{\isachardoublequoteclose}\isanewline
\ \ \ \ \isacommand{using}\isamarkupfalse%
\ assms\ knights{\isacharunderscore}{\kern0pt}path{\isacharunderscore}{\kern0pt}take\ knights{\isacharunderscore}{\kern0pt}path{\isacharunderscore}{\kern0pt}drop\ \isacommand{by}\isamarkupfalse%
\ auto\isanewline
\ \ \isacommand{have}\isamarkupfalse%
\ union{\isacharcolon}{\kern0pt}\ {\isachardoublequoteopen}{\isacharquery}{\kern0pt}b\isactrlsub {\isadigit{1}}\ {\isasymunion}\ {\isacharquery}{\kern0pt}b\isactrlsub {\isadigit{2}}\ {\isacharequal}{\kern0pt}\ b{\isachardoublequoteclose}\isanewline
\ \ \ \ \isacommand{using}\isamarkupfalse%
\ assms\ knights{\isacharunderscore}{\kern0pt}path{\isacharunderscore}{\kern0pt}set{\isacharunderscore}{\kern0pt}eq\ \isacommand{by}\isamarkupfalse%
\ {\isacharparenleft}{\kern0pt}metis\ append{\isacharunderscore}{\kern0pt}take{\isacharunderscore}{\kern0pt}drop{\isacharunderscore}{\kern0pt}id\ set{\isacharunderscore}{\kern0pt}append{\isacharparenright}{\kern0pt}\isanewline
\ \ \isacommand{have}\isamarkupfalse%
\ inter{\isacharcolon}{\kern0pt}\ {\isachardoublequoteopen}{\isacharquery}{\kern0pt}b\isactrlsub {\isadigit{1}}\ {\isasyminter}\ {\isacharquery}{\kern0pt}b\isactrlsub {\isadigit{2}}\ {\isacharequal}{\kern0pt}\ {\isacharbraceleft}{\kern0pt}{\isacharbraceright}{\kern0pt}{\isachardoublequoteclose}\isanewline
\ \ \ \ \isacommand{using}\isamarkupfalse%
\ assms\ knights{\isacharunderscore}{\kern0pt}path{\isacharunderscore}{\kern0pt}distinct\ \isacommand{by}\isamarkupfalse%
\ {\isacharparenleft}{\kern0pt}metis\ append{\isacharunderscore}{\kern0pt}take{\isacharunderscore}{\kern0pt}drop{\isacharunderscore}{\kern0pt}id\ distinct{\isacharunderscore}{\kern0pt}append{\isacharparenright}{\kern0pt}\isanewline
\ \ \isacommand{show}\isamarkupfalse%
\ {\isacharquery}{\kern0pt}thesis\ \isacommand{using}\isamarkupfalse%
\ kp{\isadigit{1}}\ kp{\isadigit{2}}\ union\ inter\ \isacommand{by}\isamarkupfalse%
\ auto\isanewline
\isacommand{qed}\isamarkupfalse%
%
\endisatagproof
{\isafoldproof}%
%
\isadelimproof
%
\endisadelimproof
%
\begin{isamarkuptext}%
Append two disjoint Knight's paths.%
\end{isamarkuptext}\isamarkuptrue%
\isacommand{corollary}\isamarkupfalse%
\ knights{\isacharunderscore}{\kern0pt}path{\isacharunderscore}{\kern0pt}append{\isacharcolon}{\kern0pt}\ \ \ \ \ \ \ \ \ \ \ \ \ \isanewline
\ \ \isakeyword{assumes}\ {\isachardoublequoteopen}knights{\isacharunderscore}{\kern0pt}path\ b\isactrlsub {\isadigit{1}}\ ps\isactrlsub {\isadigit{1}}{\isachardoublequoteclose}\ {\isachardoublequoteopen}knights{\isacharunderscore}{\kern0pt}path\ b\isactrlsub {\isadigit{2}}\ ps\isactrlsub {\isadigit{2}}{\isachardoublequoteclose}\ {\isachardoublequoteopen}b\isactrlsub {\isadigit{1}}\ {\isasyminter}\ b\isactrlsub {\isadigit{2}}\ {\isacharequal}{\kern0pt}\ {\isacharbraceleft}{\kern0pt}{\isacharbraceright}{\kern0pt}{\isachardoublequoteclose}\ {\isachardoublequoteopen}valid{\isacharunderscore}{\kern0pt}step\ {\isacharparenleft}{\kern0pt}last\ ps\isactrlsub {\isadigit{1}}{\isacharparenright}{\kern0pt}\ {\isacharparenleft}{\kern0pt}hd\ ps\isactrlsub {\isadigit{2}}{\isacharparenright}{\kern0pt}{\isachardoublequoteclose}\isanewline
\ \ \isakeyword{shows}\ {\isachardoublequoteopen}knights{\isacharunderscore}{\kern0pt}path\ {\isacharparenleft}{\kern0pt}b\isactrlsub {\isadigit{1}}\ {\isasymunion}\ b\isactrlsub {\isadigit{2}}{\isacharparenright}{\kern0pt}\ {\isacharparenleft}{\kern0pt}ps\isactrlsub {\isadigit{1}}\ {\isacharat}{\kern0pt}\ ps\isactrlsub {\isadigit{2}}{\isacharparenright}{\kern0pt}{\isachardoublequoteclose}\isanewline
%
\isadelimproof
\ \ %
\endisadelimproof
%
\isatagproof
\isacommand{using}\isamarkupfalse%
\ assms\isanewline
\isacommand{proof}\isamarkupfalse%
\ {\isacharparenleft}{\kern0pt}induction\ arbitrary{\isacharcolon}{\kern0pt}\ ps\isactrlsub {\isadigit{2}}\ b\isactrlsub {\isadigit{2}}\ rule{\isacharcolon}{\kern0pt}\ knights{\isacharunderscore}{\kern0pt}path{\isachardot}{\kern0pt}induct{\isacharparenright}{\kern0pt}\isanewline
\ \ \isacommand{case}\isamarkupfalse%
\ {\isacharparenleft}{\kern0pt}{\isadigit{1}}\ s\isactrlsub i{\isacharparenright}{\kern0pt}\isanewline
\ \ \isacommand{then}\isamarkupfalse%
\ \isacommand{have}\isamarkupfalse%
\ {\isachardoublequoteopen}s\isactrlsub i\ {\isasymnotin}\ b\isactrlsub {\isadigit{2}}{\isachardoublequoteclose}\ {\isachardoublequoteopen}ps\isactrlsub {\isadigit{2}}\ {\isasymnoteq}\ {\isacharbrackleft}{\kern0pt}{\isacharbrackright}{\kern0pt}{\isachardoublequoteclose}\ {\isachardoublequoteopen}valid{\isacharunderscore}{\kern0pt}step\ s\isactrlsub i\ {\isacharparenleft}{\kern0pt}hd\ ps\isactrlsub {\isadigit{2}}{\isacharparenright}{\kern0pt}{\isachardoublequoteclose}\ {\isachardoublequoteopen}knights{\isacharunderscore}{\kern0pt}path\ b\isactrlsub {\isadigit{2}}\ {\isacharparenleft}{\kern0pt}hd\ ps\isactrlsub {\isadigit{2}}{\isacharhash}{\kern0pt}tl\ ps\isactrlsub {\isadigit{2}}{\isacharparenright}{\kern0pt}{\isachardoublequoteclose}\ \isanewline
\ \ \ \ \isacommand{using}\isamarkupfalse%
\ knights{\isacharunderscore}{\kern0pt}path{\isacharunderscore}{\kern0pt}non{\isacharunderscore}{\kern0pt}nil\ \isacommand{by}\isamarkupfalse%
\ auto\isanewline
\ \ \isacommand{then}\isamarkupfalse%
\ \isacommand{have}\isamarkupfalse%
\ {\isachardoublequoteopen}knights{\isacharunderscore}{\kern0pt}path\ {\isacharparenleft}{\kern0pt}b\isactrlsub {\isadigit{2}}\ {\isasymunion}\ {\isacharbraceleft}{\kern0pt}s\isactrlsub i{\isacharbraceright}{\kern0pt}{\isacharparenright}{\kern0pt}\ {\isacharparenleft}{\kern0pt}s\isactrlsub i{\isacharhash}{\kern0pt}hd\ ps\isactrlsub {\isadigit{2}}{\isacharhash}{\kern0pt}tl\ ps\isactrlsub {\isadigit{2}}{\isacharparenright}{\kern0pt}{\isachardoublequoteclose}\isanewline
\ \ \ \ \isacommand{using}\isamarkupfalse%
\ knights{\isacharunderscore}{\kern0pt}path{\isachardot}{\kern0pt}intros\ \isacommand{by}\isamarkupfalse%
\ blast\isanewline
\ \ \isacommand{then}\isamarkupfalse%
\ \isacommand{show}\isamarkupfalse%
\ {\isacharquery}{\kern0pt}case\ \isacommand{using}\isamarkupfalse%
\ {\isacartoucheopen}ps\isactrlsub {\isadigit{2}}\ {\isasymnoteq}\ {\isacharbrackleft}{\kern0pt}{\isacharbrackright}{\kern0pt}{\isacartoucheclose}\ \isacommand{by}\isamarkupfalse%
\ auto\isanewline
\isacommand{next}\isamarkupfalse%
\isanewline
\ \ \isacommand{case}\isamarkupfalse%
\ {\isacharparenleft}{\kern0pt}{\isadigit{2}}\ s\isactrlsub i\ b\isactrlsub {\isadigit{1}}\ s\isactrlsub j\ ps\isactrlsub {\isadigit{1}}{\isacharparenright}{\kern0pt}\isanewline
\ \ \isacommand{then}\isamarkupfalse%
\ \isacommand{have}\isamarkupfalse%
\ {\isachardoublequoteopen}s\isactrlsub i\ {\isasymnotin}\ b\isactrlsub {\isadigit{1}}\ {\isasymunion}\ b\isactrlsub {\isadigit{2}}{\isachardoublequoteclose}\ {\isachardoublequoteopen}valid{\isacharunderscore}{\kern0pt}step\ s\isactrlsub i\ s\isactrlsub j{\isachardoublequoteclose}\ {\isachardoublequoteopen}knights{\isacharunderscore}{\kern0pt}path\ {\isacharparenleft}{\kern0pt}b\isactrlsub {\isadigit{1}}\ {\isasymunion}\ b\isactrlsub {\isadigit{2}}{\isacharparenright}{\kern0pt}\ {\isacharparenleft}{\kern0pt}s\isactrlsub j{\isacharhash}{\kern0pt}ps\isactrlsub {\isadigit{1}}{\isacharat}{\kern0pt}ps\isactrlsub {\isadigit{2}}{\isacharparenright}{\kern0pt}{\isachardoublequoteclose}\ \isacommand{by}\isamarkupfalse%
\ auto\isanewline
\ \ \isacommand{then}\isamarkupfalse%
\ \isacommand{have}\isamarkupfalse%
\ {\isachardoublequoteopen}knights{\isacharunderscore}{\kern0pt}path\ {\isacharparenleft}{\kern0pt}b\isactrlsub {\isadigit{1}}\ {\isasymunion}\ b\isactrlsub {\isadigit{2}}\ {\isasymunion}\ {\isacharbraceleft}{\kern0pt}s\isactrlsub i{\isacharbraceright}{\kern0pt}{\isacharparenright}{\kern0pt}\ {\isacharparenleft}{\kern0pt}s\isactrlsub i{\isacharhash}{\kern0pt}s\isactrlsub j{\isacharhash}{\kern0pt}ps\isactrlsub {\isadigit{1}}{\isacharat}{\kern0pt}ps\isactrlsub {\isadigit{2}}{\isacharparenright}{\kern0pt}{\isachardoublequoteclose}\isanewline
\ \ \ \ \isacommand{using}\isamarkupfalse%
\ knights{\isacharunderscore}{\kern0pt}path{\isachardot}{\kern0pt}intros\ \isacommand{by}\isamarkupfalse%
\ auto\isanewline
\ \ \isacommand{then}\isamarkupfalse%
\ \isacommand{show}\isamarkupfalse%
\ {\isacharquery}{\kern0pt}case\ \isacommand{by}\isamarkupfalse%
\ auto\isanewline
\isacommand{qed}\isamarkupfalse%
%
\endisatagproof
{\isafoldproof}%
%
\isadelimproof
\isanewline
%
\endisadelimproof
\isanewline
\isacommand{lemma}\isamarkupfalse%
\ valid{\isacharunderscore}{\kern0pt}step{\isacharunderscore}{\kern0pt}rev{\isacharcolon}{\kern0pt}\ {\isachardoublequoteopen}valid{\isacharunderscore}{\kern0pt}step\ s\isactrlsub i\ s\isactrlsub j\ {\isasymLongrightarrow}\ valid{\isacharunderscore}{\kern0pt}step\ s\isactrlsub j\ s\isactrlsub i{\isachardoublequoteclose}\isanewline
%
\isadelimproof
\ \ %
\endisadelimproof
%
\isatagproof
\isacommand{using}\isamarkupfalse%
\ step{\isacharunderscore}{\kern0pt}checker{\isacharunderscore}{\kern0pt}correct\ step{\isacharunderscore}{\kern0pt}checker{\isacharunderscore}{\kern0pt}rev\ \isacommand{by}\isamarkupfalse%
\ {\isacharparenleft}{\kern0pt}metis\ prod{\isachardot}{\kern0pt}exhaust{\isacharunderscore}{\kern0pt}sel{\isacharparenright}{\kern0pt}%
\endisatagproof
{\isafoldproof}%
%
\isadelimproof
%
\endisadelimproof
%
\begin{isamarkuptext}%
Reverse a Knight's path.%
\end{isamarkuptext}\isamarkuptrue%
\isacommand{corollary}\isamarkupfalse%
\ knights{\isacharunderscore}{\kern0pt}path{\isacharunderscore}{\kern0pt}rev{\isacharcolon}{\kern0pt}\isanewline
\ \ \isakeyword{assumes}\ {\isachardoublequoteopen}knights{\isacharunderscore}{\kern0pt}path\ b\ ps{\isachardoublequoteclose}\isanewline
\ \ \isakeyword{shows}\ {\isachardoublequoteopen}knights{\isacharunderscore}{\kern0pt}path\ b\ {\isacharparenleft}{\kern0pt}rev\ ps{\isacharparenright}{\kern0pt}{\isachardoublequoteclose}\isanewline
%
\isadelimproof
\ \ %
\endisadelimproof
%
\isatagproof
\isacommand{using}\isamarkupfalse%
\ assms\isanewline
\isacommand{proof}\isamarkupfalse%
\ {\isacharparenleft}{\kern0pt}induction\ rule{\isacharcolon}{\kern0pt}\ knights{\isacharunderscore}{\kern0pt}path{\isachardot}{\kern0pt}induct{\isacharparenright}{\kern0pt}\isanewline
\ \ \isacommand{case}\isamarkupfalse%
\ {\isacharparenleft}{\kern0pt}{\isadigit{2}}\ s\isactrlsub i\ b\ s\isactrlsub j\ ps{\isacharparenright}{\kern0pt}\isanewline
\ \ \isacommand{then}\isamarkupfalse%
\ \isacommand{have}\isamarkupfalse%
\ {\isachardoublequoteopen}knights{\isacharunderscore}{\kern0pt}path\ {\isacharbraceleft}{\kern0pt}s\isactrlsub i{\isacharbraceright}{\kern0pt}\ {\isacharbrackleft}{\kern0pt}s\isactrlsub i{\isacharbrackright}{\kern0pt}{\isachardoublequoteclose}\ {\isachardoublequoteopen}b\ {\isasyminter}\ {\isacharbraceleft}{\kern0pt}s\isactrlsub i{\isacharbraceright}{\kern0pt}\ {\isacharequal}{\kern0pt}\ {\isacharbraceleft}{\kern0pt}{\isacharbraceright}{\kern0pt}{\isachardoublequoteclose}\ {\isachardoublequoteopen}valid{\isacharunderscore}{\kern0pt}step\ {\isacharparenleft}{\kern0pt}last\ {\isacharparenleft}{\kern0pt}rev\ {\isacharparenleft}{\kern0pt}s\isactrlsub j\ {\isacharhash}{\kern0pt}\ ps{\isacharparenright}{\kern0pt}{\isacharparenright}{\kern0pt}{\isacharparenright}{\kern0pt}\ {\isacharparenleft}{\kern0pt}hd\ {\isacharbrackleft}{\kern0pt}s\isactrlsub i{\isacharbrackright}{\kern0pt}{\isacharparenright}{\kern0pt}{\isachardoublequoteclose}\isanewline
\ \ \ \ \isacommand{using}\isamarkupfalse%
\ valid{\isacharunderscore}{\kern0pt}step{\isacharunderscore}{\kern0pt}rev\ \isacommand{by}\isamarkupfalse%
\ {\isacharparenleft}{\kern0pt}auto\ intro{\isacharcolon}{\kern0pt}\ knights{\isacharunderscore}{\kern0pt}path{\isachardot}{\kern0pt}intros{\isacharparenright}{\kern0pt}\isanewline
\ \ \isacommand{then}\isamarkupfalse%
\ \isacommand{have}\isamarkupfalse%
\ {\isachardoublequoteopen}knights{\isacharunderscore}{\kern0pt}path\ {\isacharparenleft}{\kern0pt}b\ {\isasymunion}\ {\isacharbraceleft}{\kern0pt}s\isactrlsub i{\isacharbraceright}{\kern0pt}{\isacharparenright}{\kern0pt}\ {\isacharparenleft}{\kern0pt}{\isacharparenleft}{\kern0pt}rev\ {\isacharparenleft}{\kern0pt}s\isactrlsub j{\isacharhash}{\kern0pt}ps{\isacharparenright}{\kern0pt}{\isacharparenright}{\kern0pt}{\isacharat}{\kern0pt}{\isacharbrackleft}{\kern0pt}s\isactrlsub i{\isacharbrackright}{\kern0pt}{\isacharparenright}{\kern0pt}{\isachardoublequoteclose}\isanewline
\ \ \ \ \isacommand{using}\isamarkupfalse%
\ {\isadigit{2}}\ knights{\isacharunderscore}{\kern0pt}path{\isacharunderscore}{\kern0pt}append\ \isacommand{by}\isamarkupfalse%
\ blast\isanewline
\ \ \isacommand{then}\isamarkupfalse%
\ \isacommand{show}\isamarkupfalse%
\ {\isacharquery}{\kern0pt}case\ \isacommand{by}\isamarkupfalse%
\ auto\isanewline
\isacommand{qed}\isamarkupfalse%
\ {\isacharparenleft}{\kern0pt}auto\ intro{\isacharcolon}{\kern0pt}\ knights{\isacharunderscore}{\kern0pt}path{\isachardot}{\kern0pt}intros{\isacharparenright}{\kern0pt}%
\endisatagproof
{\isafoldproof}%
%
\isadelimproof
%
\endisadelimproof
%
\begin{isamarkuptext}%
Reverse a Knight's circuit.%
\end{isamarkuptext}\isamarkuptrue%
\isacommand{corollary}\isamarkupfalse%
\ knights{\isacharunderscore}{\kern0pt}circuit{\isacharunderscore}{\kern0pt}rev{\isacharcolon}{\kern0pt}\isanewline
\ \ \isakeyword{assumes}\ {\isachardoublequoteopen}knights{\isacharunderscore}{\kern0pt}circuit\ b\ ps{\isachardoublequoteclose}\isanewline
\ \ \isakeyword{shows}\ {\isachardoublequoteopen}knights{\isacharunderscore}{\kern0pt}circuit\ b\ {\isacharparenleft}{\kern0pt}rev\ ps{\isacharparenright}{\kern0pt}{\isachardoublequoteclose}\isanewline
%
\isadelimproof
\ \ %
\endisadelimproof
%
\isatagproof
\isacommand{using}\isamarkupfalse%
\ assms\ knights{\isacharunderscore}{\kern0pt}path{\isacharunderscore}{\kern0pt}rev\ valid{\isacharunderscore}{\kern0pt}step{\isacharunderscore}{\kern0pt}rev\isanewline
\ \ \isacommand{unfolding}\isamarkupfalse%
\ knights{\isacharunderscore}{\kern0pt}circuit{\isacharunderscore}{\kern0pt}def\ \isacommand{by}\isamarkupfalse%
\ {\isacharparenleft}{\kern0pt}auto\ simp{\isacharcolon}{\kern0pt}\ hd{\isacharunderscore}{\kern0pt}rev\ last{\isacharunderscore}{\kern0pt}rev{\isacharparenright}{\kern0pt}%
\endisatagproof
{\isafoldproof}%
%
\isadelimproof
\isanewline
%
\endisadelimproof
\isanewline
\isanewline
\isanewline
\isanewline
\isacommand{lemma}\isamarkupfalse%
\ knights{\isacharunderscore}{\kern0pt}circuit{\isacharunderscore}{\kern0pt}rotate{\isadigit{1}}{\isacharcolon}{\kern0pt}\isanewline
\ \ \isakeyword{assumes}\ {\isachardoublequoteopen}knights{\isacharunderscore}{\kern0pt}circuit\ b\ {\isacharparenleft}{\kern0pt}s\isactrlsub i{\isacharhash}{\kern0pt}ps{\isacharparenright}{\kern0pt}{\isachardoublequoteclose}\isanewline
\ \ \isakeyword{shows}\ {\isachardoublequoteopen}knights{\isacharunderscore}{\kern0pt}circuit\ b\ {\isacharparenleft}{\kern0pt}ps{\isacharat}{\kern0pt}{\isacharbrackleft}{\kern0pt}s\isactrlsub i{\isacharbrackright}{\kern0pt}{\isacharparenright}{\kern0pt}{\isachardoublequoteclose}\isanewline
%
\isadelimproof
%
\endisadelimproof
%
\isatagproof
\isacommand{proof}\isamarkupfalse%
\ {\isacharparenleft}{\kern0pt}cases\ {\isachardoublequoteopen}ps\ {\isacharequal}{\kern0pt}\ {\isacharbrackleft}{\kern0pt}{\isacharbrackright}{\kern0pt}{\isachardoublequoteclose}{\isacharparenright}{\kern0pt}\isanewline
\ \ \isacommand{case}\isamarkupfalse%
\ True\isanewline
\ \ \isacommand{then}\isamarkupfalse%
\ \isacommand{show}\isamarkupfalse%
\ {\isacharquery}{\kern0pt}thesis\ \isacommand{using}\isamarkupfalse%
\ assms\ \isacommand{by}\isamarkupfalse%
\ auto\isanewline
\isacommand{next}\isamarkupfalse%
\isanewline
\ \ \isacommand{case}\isamarkupfalse%
\ False\isanewline
\ \ \isacommand{have}\isamarkupfalse%
\ kp{\isadigit{1}}{\isacharcolon}{\kern0pt}\ {\isachardoublequoteopen}knights{\isacharunderscore}{\kern0pt}path\ b\ {\isacharparenleft}{\kern0pt}s\isactrlsub i{\isacharhash}{\kern0pt}ps{\isacharparenright}{\kern0pt}{\isachardoublequoteclose}\ {\isachardoublequoteopen}valid{\isacharunderscore}{\kern0pt}step\ {\isacharparenleft}{\kern0pt}last\ {\isacharparenleft}{\kern0pt}s\isactrlsub i{\isacharhash}{\kern0pt}ps{\isacharparenright}{\kern0pt}{\isacharparenright}{\kern0pt}\ {\isacharparenleft}{\kern0pt}hd\ {\isacharparenleft}{\kern0pt}s\isactrlsub i{\isacharhash}{\kern0pt}ps{\isacharparenright}{\kern0pt}{\isacharparenright}{\kern0pt}{\isachardoublequoteclose}\isanewline
\ \ \ \ \isacommand{using}\isamarkupfalse%
\ assms\ \isacommand{unfolding}\isamarkupfalse%
\ knights{\isacharunderscore}{\kern0pt}circuit{\isacharunderscore}{\kern0pt}def\ \isacommand{by}\isamarkupfalse%
\ auto\isanewline
\ \ \isacommand{then}\isamarkupfalse%
\ \isacommand{have}\isamarkupfalse%
\ kp{\isacharunderscore}{\kern0pt}elim{\isacharcolon}{\kern0pt}\ {\isachardoublequoteopen}s\isactrlsub i\ {\isasymnotin}\ {\isacharparenleft}{\kern0pt}b\ {\isacharminus}{\kern0pt}\ {\isacharbraceleft}{\kern0pt}s\isactrlsub i{\isacharbraceright}{\kern0pt}{\isacharparenright}{\kern0pt}{\isachardoublequoteclose}\ {\isachardoublequoteopen}valid{\isacharunderscore}{\kern0pt}step\ s\isactrlsub i\ {\isacharparenleft}{\kern0pt}hd\ ps{\isacharparenright}{\kern0pt}{\isachardoublequoteclose}\ {\isachardoublequoteopen}knights{\isacharunderscore}{\kern0pt}path\ {\isacharparenleft}{\kern0pt}b\ {\isacharminus}{\kern0pt}\ {\isacharbraceleft}{\kern0pt}s\isactrlsub i{\isacharbraceright}{\kern0pt}{\isacharparenright}{\kern0pt}\ ps{\isachardoublequoteclose}\isanewline
\ \ \ \ \isacommand{using}\isamarkupfalse%
\ {\isacartoucheopen}ps\ {\isasymnoteq}\ {\isacharbrackleft}{\kern0pt}{\isacharbrackright}{\kern0pt}{\isacartoucheclose}\ \isacommand{by}\isamarkupfalse%
\ {\isacharparenleft}{\kern0pt}auto\ elim{\isacharcolon}{\kern0pt}\ knights{\isacharunderscore}{\kern0pt}path{\isachardot}{\kern0pt}cases{\isacharparenright}{\kern0pt}\isanewline
\ \ \isacommand{then}\isamarkupfalse%
\ \isacommand{have}\isamarkupfalse%
\ vs{\isacharprime}{\kern0pt}{\isacharcolon}{\kern0pt}\ {\isachardoublequoteopen}valid{\isacharunderscore}{\kern0pt}step\ {\isacharparenleft}{\kern0pt}last\ {\isacharparenleft}{\kern0pt}ps{\isacharat}{\kern0pt}{\isacharbrackleft}{\kern0pt}s\isactrlsub i{\isacharbrackright}{\kern0pt}{\isacharparenright}{\kern0pt}{\isacharparenright}{\kern0pt}\ {\isacharparenleft}{\kern0pt}hd\ {\isacharparenleft}{\kern0pt}ps{\isacharat}{\kern0pt}{\isacharbrackleft}{\kern0pt}s\isactrlsub i{\isacharbrackright}{\kern0pt}{\isacharparenright}{\kern0pt}{\isacharparenright}{\kern0pt}{\isachardoublequoteclose}\isanewline
\ \ \ \ \isacommand{using}\isamarkupfalse%
\ {\isacartoucheopen}ps\ {\isasymnoteq}\ {\isacharbrackleft}{\kern0pt}{\isacharbrackright}{\kern0pt}{\isacartoucheclose}\ valid{\isacharunderscore}{\kern0pt}step{\isacharunderscore}{\kern0pt}rev\ \isacommand{by}\isamarkupfalse%
\ auto\isanewline
\isanewline
\ \ \isacommand{have}\isamarkupfalse%
\ kp{\isadigit{2}}{\isacharcolon}{\kern0pt}\ {\isachardoublequoteopen}knights{\isacharunderscore}{\kern0pt}path\ {\isacharbraceleft}{\kern0pt}s\isactrlsub i{\isacharbraceright}{\kern0pt}\ {\isacharbrackleft}{\kern0pt}s\isactrlsub i{\isacharbrackright}{\kern0pt}{\isachardoublequoteclose}\ {\isachardoublequoteopen}{\isacharparenleft}{\kern0pt}b\ {\isacharminus}{\kern0pt}\ {\isacharbraceleft}{\kern0pt}s\isactrlsub i{\isacharbraceright}{\kern0pt}{\isacharparenright}{\kern0pt}\ {\isasyminter}\ {\isacharbraceleft}{\kern0pt}s\isactrlsub i{\isacharbraceright}{\kern0pt}\ {\isacharequal}{\kern0pt}\ {\isacharbraceleft}{\kern0pt}{\isacharbraceright}{\kern0pt}{\isachardoublequoteclose}\isanewline
\ \ \ \ \isacommand{by}\isamarkupfalse%
\ {\isacharparenleft}{\kern0pt}auto\ intro{\isacharcolon}{\kern0pt}\ knights{\isacharunderscore}{\kern0pt}path{\isachardot}{\kern0pt}intros{\isacharparenright}{\kern0pt}\isanewline
\isanewline
\ \ \isacommand{have}\isamarkupfalse%
\ vs{\isacharcolon}{\kern0pt}\ {\isachardoublequoteopen}valid{\isacharunderscore}{\kern0pt}step\ {\isacharparenleft}{\kern0pt}last\ ps{\isacharparenright}{\kern0pt}\ {\isacharparenleft}{\kern0pt}hd\ {\isacharbrackleft}{\kern0pt}s\isactrlsub i{\isacharbrackright}{\kern0pt}{\isacharparenright}{\kern0pt}{\isachardoublequoteclose}\isanewline
\ \ \ \ \isacommand{using}\isamarkupfalse%
\ {\isacartoucheopen}ps\ {\isasymnoteq}\ {\isacharbrackleft}{\kern0pt}{\isacharbrackright}{\kern0pt}{\isacartoucheclose}\ {\isacartoucheopen}valid{\isacharunderscore}{\kern0pt}step\ {\isacharparenleft}{\kern0pt}last\ {\isacharparenleft}{\kern0pt}s\isactrlsub i{\isacharhash}{\kern0pt}ps{\isacharparenright}{\kern0pt}{\isacharparenright}{\kern0pt}\ {\isacharparenleft}{\kern0pt}hd\ {\isacharparenleft}{\kern0pt}s\isactrlsub i{\isacharhash}{\kern0pt}ps{\isacharparenright}{\kern0pt}{\isacharparenright}{\kern0pt}{\isacartoucheclose}\ \isacommand{by}\isamarkupfalse%
\ auto\isanewline
\isanewline
\ \ \isacommand{have}\isamarkupfalse%
\ {\isachardoublequoteopen}{\isacharparenleft}{\kern0pt}b\ {\isacharminus}{\kern0pt}\ {\isacharbraceleft}{\kern0pt}s\isactrlsub i{\isacharbraceright}{\kern0pt}{\isacharparenright}{\kern0pt}\ {\isasymunion}\ {\isacharbraceleft}{\kern0pt}s\isactrlsub i{\isacharbraceright}{\kern0pt}\ {\isacharequal}{\kern0pt}\ b{\isachardoublequoteclose}\isanewline
\ \ \ \ \isacommand{using}\isamarkupfalse%
\ kp{\isadigit{1}}\ kp{\isacharunderscore}{\kern0pt}elim\ knights{\isacharunderscore}{\kern0pt}path{\isacharunderscore}{\kern0pt}set{\isacharunderscore}{\kern0pt}eq\ \isacommand{by}\isamarkupfalse%
\ force\isanewline
\ \ \isacommand{then}\isamarkupfalse%
\ \isacommand{show}\isamarkupfalse%
\ {\isacharquery}{\kern0pt}thesis\isanewline
\ \ \ \ \isacommand{unfolding}\isamarkupfalse%
\ knights{\isacharunderscore}{\kern0pt}circuit{\isacharunderscore}{\kern0pt}def\isanewline
\ \ \ \ \isacommand{using}\isamarkupfalse%
\ vs\ knights{\isacharunderscore}{\kern0pt}path{\isacharunderscore}{\kern0pt}append{\isacharbrackleft}{\kern0pt}OF\ {\isacartoucheopen}knights{\isacharunderscore}{\kern0pt}path\ {\isacharparenleft}{\kern0pt}b\ {\isacharminus}{\kern0pt}\ {\isacharbraceleft}{\kern0pt}s\isactrlsub i{\isacharbraceright}{\kern0pt}{\isacharparenright}{\kern0pt}\ ps{\isacartoucheclose}\ kp{\isadigit{2}}{\isacharbrackright}{\kern0pt}\ vs{\isacharprime}{\kern0pt}\ \isacommand{by}\isamarkupfalse%
\ auto\isanewline
\isacommand{qed}\isamarkupfalse%
%
\endisatagproof
{\isafoldproof}%
%
\isadelimproof
%
\endisadelimproof
%
\begin{isamarkuptext}%
A Knight's circuit can be rotated to start at any square on the board.%
\end{isamarkuptext}\isamarkuptrue%
\isacommand{lemma}\isamarkupfalse%
\ knights{\isacharunderscore}{\kern0pt}circuit{\isacharunderscore}{\kern0pt}rotate{\isacharunderscore}{\kern0pt}to{\isacharcolon}{\kern0pt}\isanewline
\ \ \isakeyword{assumes}\ {\isachardoublequoteopen}knights{\isacharunderscore}{\kern0pt}circuit\ b\ ps{\isachardoublequoteclose}\ {\isachardoublequoteopen}hd\ {\isacharparenleft}{\kern0pt}drop\ k\ ps{\isacharparenright}{\kern0pt}\ {\isacharequal}{\kern0pt}\ s\isactrlsub i{\isachardoublequoteclose}\ {\isachardoublequoteopen}k\ {\isacharless}{\kern0pt}\ length\ ps{\isachardoublequoteclose}\isanewline
\ \ \isakeyword{shows}\ {\isachardoublequoteopen}{\isasymexists}ps{\isacharprime}{\kern0pt}{\isachardot}{\kern0pt}\ knights{\isacharunderscore}{\kern0pt}circuit\ b\ ps{\isacharprime}{\kern0pt}\ {\isasymand}\ hd\ ps{\isacharprime}{\kern0pt}\ {\isacharequal}{\kern0pt}\ s\isactrlsub i{\isachardoublequoteclose}\isanewline
%
\isadelimproof
\ \ %
\endisadelimproof
%
\isatagproof
\isacommand{using}\isamarkupfalse%
\ assms\isanewline
\isacommand{proof}\isamarkupfalse%
\ {\isacharparenleft}{\kern0pt}induction\ k\ arbitrary{\isacharcolon}{\kern0pt}\ b\ ps{\isacharparenright}{\kern0pt}\isanewline
\ \ \isacommand{case}\isamarkupfalse%
\ {\isacharparenleft}{\kern0pt}Suc\ k{\isacharparenright}{\kern0pt}\isanewline
\ \ \isacommand{let}\isamarkupfalse%
\ {\isacharquery}{\kern0pt}s\isactrlsub j{\isacharequal}{\kern0pt}{\isachardoublequoteopen}hd\ ps{\isachardoublequoteclose}\isanewline
\ \ \isacommand{let}\isamarkupfalse%
\ {\isacharquery}{\kern0pt}ps{\isacharprime}{\kern0pt}{\isacharequal}{\kern0pt}{\isachardoublequoteopen}tl\ ps{\isachardoublequoteclose}\ \ \ \ \ \ \ \ \ \ \ \isanewline
\ \ \isacommand{show}\isamarkupfalse%
\ {\isacharquery}{\kern0pt}case\isanewline
\ \ \isacommand{proof}\isamarkupfalse%
\ {\isacharparenleft}{\kern0pt}cases\ {\isachardoublequoteopen}s\isactrlsub i\ {\isacharequal}{\kern0pt}\ {\isacharquery}{\kern0pt}s\isactrlsub j{\isachardoublequoteclose}{\isacharparenright}{\kern0pt}\isanewline
\ \ \ \ \isacommand{case}\isamarkupfalse%
\ True\isanewline
\ \ \ \ \isacommand{then}\isamarkupfalse%
\ \isacommand{show}\isamarkupfalse%
\ {\isacharquery}{\kern0pt}thesis\ \isacommand{using}\isamarkupfalse%
\ Suc\ \isacommand{by}\isamarkupfalse%
\ auto\isanewline
\ \ \isacommand{next}\isamarkupfalse%
\isanewline
\ \ \ \ \isacommand{case}\isamarkupfalse%
\ False\isanewline
\ \ \ \ \isacommand{then}\isamarkupfalse%
\ \isacommand{have}\isamarkupfalse%
\ {\isachardoublequoteopen}{\isacharquery}{\kern0pt}ps{\isacharprime}{\kern0pt}\ {\isasymnoteq}\ {\isacharbrackleft}{\kern0pt}{\isacharbrackright}{\kern0pt}{\isachardoublequoteclose}\isanewline
\ \ \ \ \ \ \isacommand{using}\isamarkupfalse%
\ Suc\ \isacommand{by}\isamarkupfalse%
\ {\isacharparenleft}{\kern0pt}metis\ drop{\isacharunderscore}{\kern0pt}Nil\ drop{\isacharunderscore}{\kern0pt}Suc\ drop{\isacharunderscore}{\kern0pt}eq{\isacharunderscore}{\kern0pt}Nil{\isadigit{2}}\ le{\isacharunderscore}{\kern0pt}antisym\ nat{\isacharunderscore}{\kern0pt}less{\isacharunderscore}{\kern0pt}le{\isacharparenright}{\kern0pt}\isanewline
\ \ \ \ \isacommand{then}\isamarkupfalse%
\ \isacommand{have}\isamarkupfalse%
\ {\isachardoublequoteopen}knights{\isacharunderscore}{\kern0pt}circuit\ b\ {\isacharparenleft}{\kern0pt}{\isacharquery}{\kern0pt}s\isactrlsub j{\isacharhash}{\kern0pt}{\isacharquery}{\kern0pt}ps{\isacharprime}{\kern0pt}{\isacharparenright}{\kern0pt}{\isachardoublequoteclose}\isanewline
\ \ \ \ \ \ \isacommand{using}\isamarkupfalse%
\ Suc\ \isacommand{by}\isamarkupfalse%
\ {\isacharparenleft}{\kern0pt}metis\ list{\isachardot}{\kern0pt}exhaust{\isacharunderscore}{\kern0pt}sel\ tl{\isacharunderscore}{\kern0pt}Nil{\isacharparenright}{\kern0pt}\isanewline
\ \ \ \ \isacommand{then}\isamarkupfalse%
\ \isacommand{have}\isamarkupfalse%
\ {\isachardoublequoteopen}knights{\isacharunderscore}{\kern0pt}circuit\ b\ {\isacharparenleft}{\kern0pt}{\isacharquery}{\kern0pt}ps{\isacharprime}{\kern0pt}{\isacharat}{\kern0pt}{\isacharbrackleft}{\kern0pt}{\isacharquery}{\kern0pt}s\isactrlsub j{\isacharbrackright}{\kern0pt}{\isacharparenright}{\kern0pt}{\isachardoublequoteclose}\ {\isachardoublequoteopen}hd\ {\isacharparenleft}{\kern0pt}drop\ k\ {\isacharparenleft}{\kern0pt}{\isacharquery}{\kern0pt}ps{\isacharprime}{\kern0pt}{\isacharat}{\kern0pt}{\isacharbrackleft}{\kern0pt}{\isacharquery}{\kern0pt}s\isactrlsub j{\isacharbrackright}{\kern0pt}{\isacharparenright}{\kern0pt}{\isacharparenright}{\kern0pt}\ {\isacharequal}{\kern0pt}\ s\isactrlsub i{\isachardoublequoteclose}\ \isanewline
\ \ \ \ \ \ \isacommand{using}\isamarkupfalse%
\ Suc\ knights{\isacharunderscore}{\kern0pt}circuit{\isacharunderscore}{\kern0pt}rotate{\isadigit{1}}\ \isacommand{by}\isamarkupfalse%
\ {\isacharparenleft}{\kern0pt}auto\ simp{\isacharcolon}{\kern0pt}\ drop{\isacharunderscore}{\kern0pt}Suc{\isacharparenright}{\kern0pt}\isanewline
\ \ \ \ \isacommand{then}\isamarkupfalse%
\ \isacommand{show}\isamarkupfalse%
\ {\isacharquery}{\kern0pt}thesis\ \isacommand{using}\isamarkupfalse%
\ Suc\ \isacommand{by}\isamarkupfalse%
\ auto\isanewline
\ \ \isacommand{qed}\isamarkupfalse%
\isanewline
\isacommand{qed}\isamarkupfalse%
\ auto%
\endisatagproof
{\isafoldproof}%
%
\isadelimproof
%
\endisadelimproof
%
\begin{isamarkuptext}%
For positive boards (1,1) can only have (2,3) and (3,2) as a neighbour.%
\end{isamarkuptext}\isamarkuptrue%
\isacommand{lemma}\isamarkupfalse%
\ valid{\isacharunderscore}{\kern0pt}step{\isacharunderscore}{\kern0pt}{\isadigit{1}}{\isacharunderscore}{\kern0pt}{\isadigit{1}}{\isacharcolon}{\kern0pt}\isanewline
\ \ \isakeyword{assumes}\ {\isachardoublequoteopen}valid{\isacharunderscore}{\kern0pt}step\ {\isacharparenleft}{\kern0pt}{\isadigit{1}}{\isacharcomma}{\kern0pt}{\isadigit{1}}{\isacharparenright}{\kern0pt}\ {\isacharparenleft}{\kern0pt}i{\isacharcomma}{\kern0pt}j{\isacharparenright}{\kern0pt}{\isachardoublequoteclose}\ {\isachardoublequoteopen}i\ {\isachargreater}{\kern0pt}\ {\isadigit{0}}{\isachardoublequoteclose}\ {\isachardoublequoteopen}j\ {\isachargreater}{\kern0pt}\ {\isadigit{0}}{\isachardoublequoteclose}\isanewline
\ \ \isakeyword{shows}\ {\isachardoublequoteopen}{\isacharparenleft}{\kern0pt}i{\isacharcomma}{\kern0pt}j{\isacharparenright}{\kern0pt}\ {\isacharequal}{\kern0pt}\ {\isacharparenleft}{\kern0pt}{\isadigit{2}}{\isacharcomma}{\kern0pt}{\isadigit{3}}{\isacharparenright}{\kern0pt}\ {\isasymor}\ {\isacharparenleft}{\kern0pt}i{\isacharcomma}{\kern0pt}j{\isacharparenright}{\kern0pt}\ {\isacharequal}{\kern0pt}\ {\isacharparenleft}{\kern0pt}{\isadigit{3}}{\isacharcomma}{\kern0pt}{\isadigit{2}}{\isacharparenright}{\kern0pt}{\isachardoublequoteclose}\isanewline
%
\isadelimproof
\ \ %
\endisadelimproof
%
\isatagproof
\isacommand{using}\isamarkupfalse%
\ assms\ \isacommand{unfolding}\isamarkupfalse%
\ valid{\isacharunderscore}{\kern0pt}step{\isacharunderscore}{\kern0pt}def\ \isacommand{by}\isamarkupfalse%
\ auto%
\endisatagproof
{\isafoldproof}%
%
\isadelimproof
\isanewline
%
\endisadelimproof
\isanewline
\isacommand{lemma}\isamarkupfalse%
\ list{\isacharunderscore}{\kern0pt}len{\isacharunderscore}{\kern0pt}g{\isacharunderscore}{\kern0pt}{\isadigit{1}}{\isacharunderscore}{\kern0pt}split{\isacharcolon}{\kern0pt}\ {\isachardoublequoteopen}length\ xs\ {\isachargreater}{\kern0pt}\ {\isadigit{1}}\ {\isasymLongrightarrow}\ {\isasymexists}x\isactrlsub {\isadigit{1}}\ x\isactrlsub {\isadigit{2}}\ xs{\isacharprime}{\kern0pt}{\isachardot}{\kern0pt}\ xs\ {\isacharequal}{\kern0pt}\ x\isactrlsub {\isadigit{1}}{\isacharhash}{\kern0pt}x\isactrlsub {\isadigit{2}}{\isacharhash}{\kern0pt}xs{\isacharprime}{\kern0pt}{\isachardoublequoteclose}\isanewline
%
\isadelimproof
%
\endisadelimproof
%
\isatagproof
\isacommand{proof}\isamarkupfalse%
\ {\isacharparenleft}{\kern0pt}induction\ xs{\isacharparenright}{\kern0pt}\isanewline
\ \ \isacommand{case}\isamarkupfalse%
\ {\isacharparenleft}{\kern0pt}Cons\ x\ xs{\isacharparenright}{\kern0pt}\isanewline
\ \ \isacommand{then}\isamarkupfalse%
\ \isacommand{have}\isamarkupfalse%
\ {\isachardoublequoteopen}length\ xs\ {\isachargreater}{\kern0pt}\ {\isadigit{0}}{\isachardoublequoteclose}\ \isacommand{by}\isamarkupfalse%
\ auto\isanewline
\ \ \isacommand{then}\isamarkupfalse%
\ \isacommand{have}\isamarkupfalse%
\ {\isachardoublequoteopen}length\ xs\ {\isasymge}\ {\isadigit{1}}{\isachardoublequoteclose}\ \isacommand{by}\isamarkupfalse%
\ presburger\isanewline
\ \ \isacommand{then}\isamarkupfalse%
\ \isacommand{have}\isamarkupfalse%
\ {\isachardoublequoteopen}length\ xs\ {\isacharequal}{\kern0pt}\ {\isadigit{1}}\ {\isasymor}\ length\ xs\ {\isachargreater}{\kern0pt}\ {\isadigit{1}}{\isachardoublequoteclose}\ \isacommand{by}\isamarkupfalse%
\ auto\isanewline
\ \ \isacommand{then}\isamarkupfalse%
\ \isacommand{show}\isamarkupfalse%
\ {\isacharquery}{\kern0pt}case\ \isanewline
\ \ \isacommand{proof}\isamarkupfalse%
\ {\isacharparenleft}{\kern0pt}elim\ disjE{\isacharparenright}{\kern0pt}\isanewline
\ \ \ \ \isacommand{assume}\isamarkupfalse%
\ {\isachardoublequoteopen}length\ xs\ {\isacharequal}{\kern0pt}\ {\isadigit{1}}{\isachardoublequoteclose}\isanewline
\ \ \ \ \isacommand{then}\isamarkupfalse%
\ \isacommand{obtain}\isamarkupfalse%
\ x\isactrlsub {\isadigit{1}}\ \isakeyword{where}\ {\isacharbrackleft}{\kern0pt}simp{\isacharbrackright}{\kern0pt}{\isacharcolon}{\kern0pt}\ {\isachardoublequoteopen}xs\ {\isacharequal}{\kern0pt}\ {\isacharbrackleft}{\kern0pt}x\isactrlsub {\isadigit{1}}{\isacharbrackright}{\kern0pt}{\isachardoublequoteclose}\isanewline
\ \ \ \ \ \ \isacommand{using}\isamarkupfalse%
\ length{\isacharunderscore}{\kern0pt}Suc{\isacharunderscore}{\kern0pt}conv{\isacharbrackleft}{\kern0pt}of\ xs\ {\isadigit{0}}{\isacharbrackright}{\kern0pt}\ \isacommand{by}\isamarkupfalse%
\ auto\isanewline
\ \ \ \ \isacommand{then}\isamarkupfalse%
\ \isacommand{show}\isamarkupfalse%
\ {\isacharquery}{\kern0pt}thesis\ \isacommand{by}\isamarkupfalse%
\ auto\isanewline
\ \ \isacommand{next}\isamarkupfalse%
\isanewline
\ \ \ \ \isacommand{assume}\isamarkupfalse%
\ {\isachardoublequoteopen}{\isadigit{1}}\ {\isacharless}{\kern0pt}\ length\ xs{\isachardoublequoteclose}\isanewline
\ \ \ \ \isacommand{then}\isamarkupfalse%
\ \isacommand{show}\isamarkupfalse%
\ {\isacharquery}{\kern0pt}thesis\ \isacommand{using}\isamarkupfalse%
\ Cons\ \isacommand{by}\isamarkupfalse%
\ auto\isanewline
\ \ \isacommand{qed}\isamarkupfalse%
\isanewline
\isacommand{qed}\isamarkupfalse%
\ auto%
\endisatagproof
{\isafoldproof}%
%
\isadelimproof
\isanewline
%
\endisadelimproof
\isanewline
\isacommand{lemma}\isamarkupfalse%
\ list{\isacharunderscore}{\kern0pt}len{\isacharunderscore}{\kern0pt}g{\isacharunderscore}{\kern0pt}{\isadigit{3}}{\isacharunderscore}{\kern0pt}split{\isacharcolon}{\kern0pt}\ {\isachardoublequoteopen}length\ xs\ {\isachargreater}{\kern0pt}\ {\isadigit{3}}\ {\isasymLongrightarrow}\ {\isasymexists}x\isactrlsub {\isadigit{1}}\ x\isactrlsub {\isadigit{2}}\ xs{\isacharprime}{\kern0pt}\ x\isactrlsub {\isadigit{3}}{\isachardot}{\kern0pt}\ xs\ {\isacharequal}{\kern0pt}\ x\isactrlsub {\isadigit{1}}{\isacharhash}{\kern0pt}x\isactrlsub {\isadigit{2}}{\isacharhash}{\kern0pt}xs{\isacharprime}{\kern0pt}{\isacharat}{\kern0pt}{\isacharbrackleft}{\kern0pt}x\isactrlsub {\isadigit{3}}{\isacharbrackright}{\kern0pt}{\isachardoublequoteclose}\isanewline
%
\isadelimproof
%
\endisadelimproof
%
\isatagproof
\isacommand{proof}\isamarkupfalse%
\ {\isacharparenleft}{\kern0pt}induction\ xs{\isacharparenright}{\kern0pt}\isanewline
\ \ \isacommand{case}\isamarkupfalse%
\ {\isacharparenleft}{\kern0pt}Cons\ x\ xs{\isacharparenright}{\kern0pt}\isanewline
\ \ \isacommand{then}\isamarkupfalse%
\ \isacommand{have}\isamarkupfalse%
\ {\isachardoublequoteopen}length\ xs\ {\isacharequal}{\kern0pt}\ {\isadigit{3}}\ {\isasymor}\ length\ xs\ {\isachargreater}{\kern0pt}\ {\isadigit{3}}{\isachardoublequoteclose}\ \isacommand{by}\isamarkupfalse%
\ auto\isanewline
\ \ \isacommand{then}\isamarkupfalse%
\ \isacommand{show}\isamarkupfalse%
\ {\isacharquery}{\kern0pt}case\ \isanewline
\ \ \isacommand{proof}\isamarkupfalse%
\ {\isacharparenleft}{\kern0pt}elim\ disjE{\isacharparenright}{\kern0pt}\isanewline
\ \ \ \ \isacommand{assume}\isamarkupfalse%
\ {\isachardoublequoteopen}length\ xs\ {\isacharequal}{\kern0pt}\ {\isadigit{3}}{\isachardoublequoteclose}\isanewline
\ \ \ \ \isacommand{then}\isamarkupfalse%
\ \isacommand{obtain}\isamarkupfalse%
\ x\isactrlsub {\isadigit{1}}\ xs\isactrlsub {\isadigit{1}}\ \isakeyword{where}\ {\isacharbrackleft}{\kern0pt}simp{\isacharbrackright}{\kern0pt}{\isacharcolon}{\kern0pt}\ {\isachardoublequoteopen}xs\ {\isacharequal}{\kern0pt}\ x\isactrlsub {\isadigit{1}}{\isacharhash}{\kern0pt}xs\isactrlsub {\isadigit{1}}{\isachardoublequoteclose}\ {\isachardoublequoteopen}length\ xs\isactrlsub {\isadigit{1}}\ {\isacharequal}{\kern0pt}\ {\isadigit{2}}{\isachardoublequoteclose}\isanewline
\ \ \ \ \ \ \isacommand{using}\isamarkupfalse%
\ length{\isacharunderscore}{\kern0pt}Suc{\isacharunderscore}{\kern0pt}conv{\isacharbrackleft}{\kern0pt}of\ xs\ {\isadigit{2}}{\isacharbrackright}{\kern0pt}\ \isacommand{by}\isamarkupfalse%
\ auto\isanewline
\ \ \ \ \isacommand{then}\isamarkupfalse%
\ \isacommand{obtain}\isamarkupfalse%
\ x\isactrlsub {\isadigit{2}}\ xs\isactrlsub {\isadigit{2}}\ \isakeyword{where}\ {\isacharbrackleft}{\kern0pt}simp{\isacharbrackright}{\kern0pt}{\isacharcolon}{\kern0pt}\ {\isachardoublequoteopen}xs\isactrlsub {\isadigit{1}}\ {\isacharequal}{\kern0pt}\ x\isactrlsub {\isadigit{2}}{\isacharhash}{\kern0pt}xs\isactrlsub {\isadigit{2}}{\isachardoublequoteclose}\ {\isachardoublequoteopen}length\ xs\isactrlsub {\isadigit{2}}\ {\isacharequal}{\kern0pt}\ {\isadigit{1}}{\isachardoublequoteclose}\isanewline
\ \ \ \ \ \ \isacommand{using}\isamarkupfalse%
\ length{\isacharunderscore}{\kern0pt}Suc{\isacharunderscore}{\kern0pt}conv{\isacharbrackleft}{\kern0pt}of\ xs\isactrlsub {\isadigit{1}}\ {\isadigit{1}}{\isacharbrackright}{\kern0pt}\ \isacommand{by}\isamarkupfalse%
\ auto\isanewline
\ \ \ \ \isacommand{then}\isamarkupfalse%
\ \isacommand{obtain}\isamarkupfalse%
\ x\isactrlsub {\isadigit{3}}\ \isakeyword{where}\ {\isacharbrackleft}{\kern0pt}simp{\isacharbrackright}{\kern0pt}{\isacharcolon}{\kern0pt}\ {\isachardoublequoteopen}xs\isactrlsub {\isadigit{2}}\ {\isacharequal}{\kern0pt}\ {\isacharbrackleft}{\kern0pt}x\isactrlsub {\isadigit{3}}{\isacharbrackright}{\kern0pt}{\isachardoublequoteclose}\isanewline
\ \ \ \ \ \ \isacommand{using}\isamarkupfalse%
\ length{\isacharunderscore}{\kern0pt}Suc{\isacharunderscore}{\kern0pt}conv{\isacharbrackleft}{\kern0pt}of\ xs\isactrlsub {\isadigit{2}}\ {\isadigit{0}}{\isacharbrackright}{\kern0pt}\ \isacommand{by}\isamarkupfalse%
\ auto\isanewline
\ \ \ \ \isacommand{then}\isamarkupfalse%
\ \isacommand{show}\isamarkupfalse%
\ {\isacharquery}{\kern0pt}thesis\ \isacommand{by}\isamarkupfalse%
\ auto\isanewline
\ \ \isacommand{next}\isamarkupfalse%
\isanewline
\ \ \ \ \isacommand{assume}\isamarkupfalse%
\ {\isachardoublequoteopen}length\ xs\ {\isachargreater}{\kern0pt}\ {\isadigit{3}}{\isachardoublequoteclose}\isanewline
\ \ \ \ \isacommand{then}\isamarkupfalse%
\ \isacommand{show}\isamarkupfalse%
\ {\isacharquery}{\kern0pt}thesis\ \isacommand{using}\isamarkupfalse%
\ Cons\ \isacommand{by}\isamarkupfalse%
\ auto\isanewline
\ \ \isacommand{qed}\isamarkupfalse%
\isanewline
\isacommand{qed}\isamarkupfalse%
\ auto%
\endisatagproof
{\isafoldproof}%
%
\isadelimproof
%
\endisadelimproof
%
\begin{isamarkuptext}%
Any Knight's circuit on a positive board can be rotated to start with (1,1) and 
end with (3,2).%
\end{isamarkuptext}\isamarkuptrue%
\isacommand{corollary}\isamarkupfalse%
\ rotate{\isacharunderscore}{\kern0pt}knights{\isacharunderscore}{\kern0pt}circuit{\isacharcolon}{\kern0pt}\isanewline
\ \ \isakeyword{assumes}\ {\isachardoublequoteopen}knights{\isacharunderscore}{\kern0pt}circuit\ {\isacharparenleft}{\kern0pt}board\ n\ m{\isacharparenright}{\kern0pt}\ ps{\isachardoublequoteclose}\ {\isachardoublequoteopen}min\ n\ m\ {\isasymge}\ {\isadigit{5}}{\isachardoublequoteclose}\isanewline
\ \ \isakeyword{shows}\ {\isachardoublequoteopen}{\isasymexists}ps{\isachardot}{\kern0pt}\ knights{\isacharunderscore}{\kern0pt}circuit\ {\isacharparenleft}{\kern0pt}board\ n\ m{\isacharparenright}{\kern0pt}\ ps\ {\isasymand}\ hd\ ps\ {\isacharequal}{\kern0pt}\ {\isacharparenleft}{\kern0pt}{\isadigit{1}}{\isacharcomma}{\kern0pt}{\isadigit{1}}{\isacharparenright}{\kern0pt}\ {\isasymand}\ last\ ps\ {\isacharequal}{\kern0pt}\ {\isacharparenleft}{\kern0pt}{\isadigit{3}}{\isacharcomma}{\kern0pt}{\isadigit{2}}{\isacharparenright}{\kern0pt}{\isachardoublequoteclose}\isanewline
%
\isadelimproof
\ \ %
\endisadelimproof
%
\isatagproof
\isacommand{using}\isamarkupfalse%
\ assms\isanewline
\isacommand{proof}\isamarkupfalse%
\ {\isacharminus}{\kern0pt}\isanewline
\ \ \isacommand{let}\isamarkupfalse%
\ {\isacharquery}{\kern0pt}b{\isacharequal}{\kern0pt}{\isachardoublequoteopen}board\ n\ m{\isachardoublequoteclose}\isanewline
\ \ \isacommand{have}\isamarkupfalse%
\ {\isachardoublequoteopen}knights{\isacharunderscore}{\kern0pt}path\ {\isacharquery}{\kern0pt}b\ ps{\isachardoublequoteclose}\isanewline
\ \ \ \ \isacommand{using}\isamarkupfalse%
\ assms\ \isacommand{unfolding}\isamarkupfalse%
\ knights{\isacharunderscore}{\kern0pt}circuit{\isacharunderscore}{\kern0pt}def\ \isacommand{by}\isamarkupfalse%
\ auto\isanewline
\ \ \isacommand{then}\isamarkupfalse%
\ \isacommand{have}\isamarkupfalse%
\ {\isachardoublequoteopen}{\isacharparenleft}{\kern0pt}{\isadigit{1}}{\isacharcomma}{\kern0pt}{\isadigit{1}}{\isacharparenright}{\kern0pt}\ {\isasymin}\ set\ ps{\isachardoublequoteclose}\isanewline
\ \ \ \ \isacommand{using}\isamarkupfalse%
\ assms\ knights{\isacharunderscore}{\kern0pt}path{\isacharunderscore}{\kern0pt}set{\isacharunderscore}{\kern0pt}eq\ \isacommand{by}\isamarkupfalse%
\ {\isacharparenleft}{\kern0pt}auto\ simp{\isacharcolon}{\kern0pt}\ board{\isacharunderscore}{\kern0pt}def{\isacharparenright}{\kern0pt}\isanewline
\ \ \isacommand{then}\isamarkupfalse%
\ \isacommand{obtain}\isamarkupfalse%
\ k\ \isakeyword{where}\ {\isachardoublequoteopen}hd\ {\isacharparenleft}{\kern0pt}drop\ k\ ps{\isacharparenright}{\kern0pt}\ {\isacharequal}{\kern0pt}\ {\isacharparenleft}{\kern0pt}{\isadigit{1}}{\isacharcomma}{\kern0pt}{\isadigit{1}}{\isacharparenright}{\kern0pt}{\isachardoublequoteclose}\ {\isachardoublequoteopen}k\ {\isacharless}{\kern0pt}\ length\ ps{\isachardoublequoteclose}\isanewline
\ \ \ \ \isacommand{by}\isamarkupfalse%
\ {\isacharparenleft}{\kern0pt}metis\ hd{\isacharunderscore}{\kern0pt}drop{\isacharunderscore}{\kern0pt}conv{\isacharunderscore}{\kern0pt}nth\ in{\isacharunderscore}{\kern0pt}set{\isacharunderscore}{\kern0pt}conv{\isacharunderscore}{\kern0pt}nth{\isacharparenright}{\kern0pt}\isanewline
\ \ \isacommand{then}\isamarkupfalse%
\ \isacommand{obtain}\isamarkupfalse%
\ ps\isactrlsub r\ \isakeyword{where}\ ps\isactrlsub r{\isacharunderscore}{\kern0pt}prems{\isacharcolon}{\kern0pt}\ {\isachardoublequoteopen}knights{\isacharunderscore}{\kern0pt}circuit\ {\isacharquery}{\kern0pt}b\ ps\isactrlsub r{\isachardoublequoteclose}\ {\isachardoublequoteopen}hd\ ps\isactrlsub r\ {\isacharequal}{\kern0pt}\ {\isacharparenleft}{\kern0pt}{\isadigit{1}}{\isacharcomma}{\kern0pt}{\isadigit{1}}{\isacharparenright}{\kern0pt}{\isachardoublequoteclose}\isanewline
\ \ \ \ \isacommand{using}\isamarkupfalse%
\ assms\ knights{\isacharunderscore}{\kern0pt}circuit{\isacharunderscore}{\kern0pt}rotate{\isacharunderscore}{\kern0pt}to\ \isacommand{by}\isamarkupfalse%
\ blast\isanewline
\ \ \isacommand{then}\isamarkupfalse%
\ \isacommand{have}\isamarkupfalse%
\ kp{\isacharcolon}{\kern0pt}\ {\isachardoublequoteopen}knights{\isacharunderscore}{\kern0pt}path\ {\isacharquery}{\kern0pt}b\ ps\isactrlsub r{\isachardoublequoteclose}\ \isakeyword{and}\ {\isachardoublequoteopen}valid{\isacharunderscore}{\kern0pt}step\ {\isacharparenleft}{\kern0pt}last\ ps\isactrlsub r{\isacharparenright}{\kern0pt}\ {\isacharparenleft}{\kern0pt}{\isadigit{1}}{\isacharcomma}{\kern0pt}{\isadigit{1}}{\isacharparenright}{\kern0pt}{\isachardoublequoteclose}\isanewline
\ \ \ \ \isacommand{unfolding}\isamarkupfalse%
\ knights{\isacharunderscore}{\kern0pt}circuit{\isacharunderscore}{\kern0pt}def\ \isacommand{by}\isamarkupfalse%
\ auto\isanewline
\isanewline
\ \ \isacommand{have}\isamarkupfalse%
\ {\isachardoublequoteopen}{\isacharparenleft}{\kern0pt}{\isadigit{1}}{\isacharcomma}{\kern0pt}{\isadigit{1}}{\isacharparenright}{\kern0pt}\ {\isasymin}\ {\isacharquery}{\kern0pt}b{\isachardoublequoteclose}\ {\isachardoublequoteopen}{\isacharparenleft}{\kern0pt}{\isadigit{1}}{\isacharcomma}{\kern0pt}{\isadigit{2}}{\isacharparenright}{\kern0pt}\ {\isasymin}\ {\isacharquery}{\kern0pt}b{\isachardoublequoteclose}\ {\isachardoublequoteopen}{\isacharparenleft}{\kern0pt}{\isadigit{1}}{\isacharcomma}{\kern0pt}{\isadigit{3}}{\isacharparenright}{\kern0pt}\ {\isasymin}\ {\isacharquery}{\kern0pt}b{\isachardoublequoteclose}\isanewline
\ \ \ \ \isacommand{using}\isamarkupfalse%
\ assms\ \isacommand{unfolding}\isamarkupfalse%
\ board{\isacharunderscore}{\kern0pt}def\ \isacommand{by}\isamarkupfalse%
\ auto\isanewline
\ \ \isacommand{then}\isamarkupfalse%
\ \isacommand{have}\isamarkupfalse%
\ {\isachardoublequoteopen}{\isacharparenleft}{\kern0pt}{\isadigit{1}}{\isacharcomma}{\kern0pt}{\isadigit{1}}{\isacharparenright}{\kern0pt}\ {\isasymin}\ set\ ps\isactrlsub r{\isachardoublequoteclose}\ {\isachardoublequoteopen}{\isacharparenleft}{\kern0pt}{\isadigit{1}}{\isacharcomma}{\kern0pt}{\isadigit{2}}{\isacharparenright}{\kern0pt}\ {\isasymin}\ set\ ps\isactrlsub r{\isachardoublequoteclose}\ {\isachardoublequoteopen}{\isacharparenleft}{\kern0pt}{\isadigit{1}}{\isacharcomma}{\kern0pt}{\isadigit{3}}{\isacharparenright}{\kern0pt}\ {\isasymin}\ set\ ps\isactrlsub r{\isachardoublequoteclose}\isanewline
\ \ \ \ \isacommand{using}\isamarkupfalse%
\ kp\ knights{\isacharunderscore}{\kern0pt}path{\isacharunderscore}{\kern0pt}set{\isacharunderscore}{\kern0pt}eq\ \isacommand{by}\isamarkupfalse%
\ auto\isanewline
\isanewline
\ \ \isacommand{have}\isamarkupfalse%
\ {\isachardoublequoteopen}{\isadigit{3}}\ {\isacharless}{\kern0pt}\ card\ {\isacharquery}{\kern0pt}b{\isachardoublequoteclose}\isanewline
\ \ \ \ \isacommand{using}\isamarkupfalse%
\ assms\ board{\isacharunderscore}{\kern0pt}leq{\isacharunderscore}{\kern0pt}subset\ card{\isacharunderscore}{\kern0pt}board{\isacharbrackleft}{\kern0pt}of\ {\isadigit{5}}\ {\isadigit{5}}{\isacharbrackright}{\kern0pt}\isanewline
\ \ \ \ \ \ \ \ \ \ card{\isacharunderscore}{\kern0pt}mono{\isacharbrackleft}{\kern0pt}OF\ board{\isacharunderscore}{\kern0pt}finite{\isacharbrackleft}{\kern0pt}of\ n\ m{\isacharbrackright}{\kern0pt}{\isacharcomma}{\kern0pt}\ of\ {\isachardoublequoteopen}board\ {\isadigit{5}}\ {\isadigit{5}}{\isachardoublequoteclose}{\isacharbrackright}{\kern0pt}\ \isacommand{by}\isamarkupfalse%
\ auto\isanewline
\ \ \isacommand{then}\isamarkupfalse%
\ \isacommand{have}\isamarkupfalse%
\ {\isachardoublequoteopen}{\isadigit{3}}\ {\isacharless}{\kern0pt}\ length\ ps\isactrlsub r{\isachardoublequoteclose}\isanewline
\ \ \ \ \isacommand{using}\isamarkupfalse%
\ knights{\isacharunderscore}{\kern0pt}path{\isacharunderscore}{\kern0pt}length\ kp\ \isacommand{by}\isamarkupfalse%
\ auto\isanewline
\ \ \isacommand{then}\isamarkupfalse%
\ \isacommand{obtain}\isamarkupfalse%
\ s\isactrlsub j\ ps{\isacharprime}{\kern0pt}\ s\isactrlsub k\ \isakeyword{where}\ {\isacharbrackleft}{\kern0pt}simp{\isacharbrackright}{\kern0pt}{\isacharcolon}{\kern0pt}\ {\isachardoublequoteopen}ps\isactrlsub r\ {\isacharequal}{\kern0pt}\ {\isacharparenleft}{\kern0pt}{\isadigit{1}}{\isacharcomma}{\kern0pt}{\isadigit{1}}{\isacharparenright}{\kern0pt}{\isacharhash}{\kern0pt}s\isactrlsub j{\isacharhash}{\kern0pt}ps{\isacharprime}{\kern0pt}{\isacharat}{\kern0pt}{\isacharbrackleft}{\kern0pt}s\isactrlsub k{\isacharbrackright}{\kern0pt}{\isachardoublequoteclose}\isanewline
\ \ \ \ \isacommand{using}\isamarkupfalse%
\ {\isacartoucheopen}hd\ ps\isactrlsub r\ {\isacharequal}{\kern0pt}\ {\isacharparenleft}{\kern0pt}{\isadigit{1}}{\isacharcomma}{\kern0pt}{\isadigit{1}}{\isacharparenright}{\kern0pt}{\isacartoucheclose}\ list{\isacharunderscore}{\kern0pt}len{\isacharunderscore}{\kern0pt}g{\isacharunderscore}{\kern0pt}{\isadigit{3}}{\isacharunderscore}{\kern0pt}split{\isacharbrackleft}{\kern0pt}of\ ps\isactrlsub r{\isacharbrackright}{\kern0pt}\ \isacommand{by}\isamarkupfalse%
\ auto\isanewline
\ \ \isacommand{have}\isamarkupfalse%
\ {\isachardoublequoteopen}s\isactrlsub j\ {\isasymnoteq}\ s\isactrlsub k{\isachardoublequoteclose}\isanewline
\ \ \ \ \isacommand{using}\isamarkupfalse%
\ kp\ knights{\isacharunderscore}{\kern0pt}path{\isacharunderscore}{\kern0pt}distinct\ \isacommand{by}\isamarkupfalse%
\ force\isanewline
\isanewline
\ \ \isacommand{have}\isamarkupfalse%
\ vs{\isacharunderscore}{\kern0pt}s\isactrlsub k{\isacharcolon}{\kern0pt}\ {\isachardoublequoteopen}valid{\isacharunderscore}{\kern0pt}step\ s\isactrlsub k\ {\isacharparenleft}{\kern0pt}{\isadigit{1}}{\isacharcomma}{\kern0pt}{\isadigit{1}}{\isacharparenright}{\kern0pt}{\isachardoublequoteclose}\isanewline
\ \ \ \ \isacommand{using}\isamarkupfalse%
\ {\isacartoucheopen}valid{\isacharunderscore}{\kern0pt}step\ {\isacharparenleft}{\kern0pt}last\ ps\isactrlsub r{\isacharparenright}{\kern0pt}\ {\isacharparenleft}{\kern0pt}{\isadigit{1}}{\isacharcomma}{\kern0pt}{\isadigit{1}}{\isacharparenright}{\kern0pt}{\isacartoucheclose}\ \isacommand{by}\isamarkupfalse%
\ simp\isanewline
\isanewline
\ \ \isacommand{have}\isamarkupfalse%
\ vs{\isacharunderscore}{\kern0pt}s\isactrlsub j{\isacharcolon}{\kern0pt}\ {\isachardoublequoteopen}valid{\isacharunderscore}{\kern0pt}step\ {\isacharparenleft}{\kern0pt}{\isadigit{1}}{\isacharcomma}{\kern0pt}{\isadigit{1}}{\isacharparenright}{\kern0pt}\ s\isactrlsub j{\isachardoublequoteclose}\ \isakeyword{and}\ kp{\isacharprime}{\kern0pt}{\isacharcolon}{\kern0pt}\ {\isachardoublequoteopen}knights{\isacharunderscore}{\kern0pt}path\ {\isacharparenleft}{\kern0pt}{\isacharquery}{\kern0pt}b\ {\isacharminus}{\kern0pt}\ {\isacharbraceleft}{\kern0pt}{\isacharparenleft}{\kern0pt}{\isadigit{1}}{\isacharcomma}{\kern0pt}{\isadigit{1}}{\isacharparenright}{\kern0pt}{\isacharbraceright}{\kern0pt}{\isacharparenright}{\kern0pt}\ {\isacharparenleft}{\kern0pt}s\isactrlsub j{\isacharhash}{\kern0pt}ps{\isacharprime}{\kern0pt}{\isacharat}{\kern0pt}{\isacharbrackleft}{\kern0pt}s\isactrlsub k{\isacharbrackright}{\kern0pt}{\isacharparenright}{\kern0pt}{\isachardoublequoteclose}\isanewline
\ \ \ \ \isacommand{using}\isamarkupfalse%
\ kp\ \isacommand{by}\isamarkupfalse%
\ {\isacharparenleft}{\kern0pt}auto\ elim{\isacharcolon}{\kern0pt}\ knights{\isacharunderscore}{\kern0pt}path{\isachardot}{\kern0pt}cases{\isacharparenright}{\kern0pt}\isanewline
\isanewline
\ \ \isacommand{have}\isamarkupfalse%
\ {\isachardoublequoteopen}s\isactrlsub j\ {\isasymin}\ set\ ps\isactrlsub r{\isachardoublequoteclose}\ {\isachardoublequoteopen}s\isactrlsub k\ {\isasymin}\ set\ ps\isactrlsub r{\isachardoublequoteclose}\ \isacommand{by}\isamarkupfalse%
\ auto\isanewline
\ \ \isacommand{then}\isamarkupfalse%
\ \isacommand{have}\isamarkupfalse%
\ {\isachardoublequoteopen}s\isactrlsub j\ {\isasymin}\ {\isacharquery}{\kern0pt}b{\isachardoublequoteclose}\ {\isachardoublequoteopen}s\isactrlsub k\ {\isasymin}\ {\isacharquery}{\kern0pt}b{\isachardoublequoteclose}\isanewline
\ \ \ \ \isacommand{using}\isamarkupfalse%
\ kp\ knights{\isacharunderscore}{\kern0pt}path{\isacharunderscore}{\kern0pt}set{\isacharunderscore}{\kern0pt}eq\ \isacommand{by}\isamarkupfalse%
\ blast{\isacharplus}{\kern0pt}\isanewline
\ \ \isacommand{then}\isamarkupfalse%
\ \isacommand{have}\isamarkupfalse%
\ {\isachardoublequoteopen}{\isadigit{0}}\ {\isacharless}{\kern0pt}\ fst\ s\isactrlsub j\ {\isasymand}\ {\isadigit{0}}\ {\isacharless}{\kern0pt}\ snd\ s\isactrlsub j{\isachardoublequoteclose}\ {\isachardoublequoteopen}{\isadigit{0}}\ {\isacharless}{\kern0pt}\ fst\ s\isactrlsub k\ {\isasymand}\ {\isadigit{0}}\ {\isacharless}{\kern0pt}\ snd\ s\isactrlsub k{\isachardoublequoteclose}\isanewline
\ \ \ \ \isacommand{unfolding}\isamarkupfalse%
\ board{\isacharunderscore}{\kern0pt}def\ \isacommand{by}\isamarkupfalse%
\ auto\isanewline
\ \ \isacommand{then}\isamarkupfalse%
\ \isacommand{have}\isamarkupfalse%
\ {\isachardoublequoteopen}s\isactrlsub k\ {\isacharequal}{\kern0pt}\ {\isacharparenleft}{\kern0pt}{\isadigit{2}}{\isacharcomma}{\kern0pt}{\isadigit{3}}{\isacharparenright}{\kern0pt}\ {\isasymor}\ s\isactrlsub k\ {\isacharequal}{\kern0pt}\ {\isacharparenleft}{\kern0pt}{\isadigit{3}}{\isacharcomma}{\kern0pt}{\isadigit{2}}{\isacharparenright}{\kern0pt}{\isachardoublequoteclose}\ {\isachardoublequoteopen}s\isactrlsub j\ {\isacharequal}{\kern0pt}\ {\isacharparenleft}{\kern0pt}{\isadigit{2}}{\isacharcomma}{\kern0pt}{\isadigit{3}}{\isacharparenright}{\kern0pt}\ {\isasymor}\ s\isactrlsub j\ {\isacharequal}{\kern0pt}\ {\isacharparenleft}{\kern0pt}{\isadigit{3}}{\isacharcomma}{\kern0pt}{\isadigit{2}}{\isacharparenright}{\kern0pt}{\isachardoublequoteclose}\isanewline
\ \ \ \ \isacommand{using}\isamarkupfalse%
\ vs{\isacharunderscore}{\kern0pt}s\isactrlsub k\ vs{\isacharunderscore}{\kern0pt}s\isactrlsub j\ valid{\isacharunderscore}{\kern0pt}step{\isacharunderscore}{\kern0pt}{\isadigit{1}}{\isacharunderscore}{\kern0pt}{\isadigit{1}}\ valid{\isacharunderscore}{\kern0pt}step{\isacharunderscore}{\kern0pt}rev\ \isacommand{by}\isamarkupfalse%
\ {\isacharparenleft}{\kern0pt}metis\ prod{\isachardot}{\kern0pt}collapse{\isacharparenright}{\kern0pt}{\isacharplus}{\kern0pt}\isanewline
\ \ \isacommand{then}\isamarkupfalse%
\ \isacommand{have}\isamarkupfalse%
\ {\isachardoublequoteopen}s\isactrlsub k\ {\isacharequal}{\kern0pt}\ {\isacharparenleft}{\kern0pt}{\isadigit{3}}{\isacharcomma}{\kern0pt}{\isadigit{2}}{\isacharparenright}{\kern0pt}\ {\isasymor}\ s\isactrlsub j\ {\isacharequal}{\kern0pt}\ {\isacharparenleft}{\kern0pt}{\isadigit{3}}{\isacharcomma}{\kern0pt}{\isadigit{2}}{\isacharparenright}{\kern0pt}{\isachardoublequoteclose}\isanewline
\ \ \ \ \isacommand{using}\isamarkupfalse%
\ {\isacartoucheopen}s\isactrlsub j\ {\isasymnoteq}\ s\isactrlsub k{\isacartoucheclose}\ \isacommand{by}\isamarkupfalse%
\ auto\isanewline
\ \ \isacommand{then}\isamarkupfalse%
\ \isacommand{show}\isamarkupfalse%
\ {\isacharquery}{\kern0pt}thesis\isanewline
\ \ \isacommand{proof}\isamarkupfalse%
\ {\isacharparenleft}{\kern0pt}elim\ disjE{\isacharparenright}{\kern0pt}\isanewline
\ \ \ \ \isacommand{assume}\isamarkupfalse%
\ {\isachardoublequoteopen}s\isactrlsub k\ {\isacharequal}{\kern0pt}\ {\isacharparenleft}{\kern0pt}{\isadigit{3}}{\isacharcomma}{\kern0pt}{\isadigit{2}}{\isacharparenright}{\kern0pt}{\isachardoublequoteclose}\isanewline
\ \ \ \ \isacommand{then}\isamarkupfalse%
\ \isacommand{have}\isamarkupfalse%
\ {\isachardoublequoteopen}last\ ps\isactrlsub r\ {\isacharequal}{\kern0pt}\ {\isacharparenleft}{\kern0pt}{\isadigit{3}}{\isacharcomma}{\kern0pt}{\isadigit{2}}{\isacharparenright}{\kern0pt}{\isachardoublequoteclose}\ \isacommand{by}\isamarkupfalse%
\ auto\isanewline
\ \ \ \ \isacommand{then}\isamarkupfalse%
\ \isacommand{show}\isamarkupfalse%
\ {\isacharquery}{\kern0pt}thesis\ \isacommand{using}\isamarkupfalse%
\ ps\isactrlsub r{\isacharunderscore}{\kern0pt}prems\ \isacommand{by}\isamarkupfalse%
\ auto\isanewline
\ \ \isacommand{next}\isamarkupfalse%
\isanewline
\ \ \ \ \isacommand{assume}\isamarkupfalse%
\ {\isachardoublequoteopen}s\isactrlsub j\ {\isacharequal}{\kern0pt}\ {\isacharparenleft}{\kern0pt}{\isadigit{3}}{\isacharcomma}{\kern0pt}{\isadigit{2}}{\isacharparenright}{\kern0pt}{\isachardoublequoteclose}\isanewline
\ \ \ \ \isacommand{then}\isamarkupfalse%
\ \isacommand{have}\isamarkupfalse%
\ vs{\isacharcolon}{\kern0pt}\ {\isachardoublequoteopen}valid{\isacharunderscore}{\kern0pt}step\ {\isacharparenleft}{\kern0pt}last\ {\isacharparenleft}{\kern0pt}{\isacharparenleft}{\kern0pt}{\isadigit{1}}{\isacharcomma}{\kern0pt}{\isadigit{1}}{\isacharparenright}{\kern0pt}{\isacharhash}{\kern0pt}rev\ {\isacharparenleft}{\kern0pt}s\isactrlsub j{\isacharhash}{\kern0pt}ps{\isacharprime}{\kern0pt}{\isacharat}{\kern0pt}{\isacharbrackleft}{\kern0pt}s\isactrlsub k{\isacharbrackright}{\kern0pt}{\isacharparenright}{\kern0pt}{\isacharparenright}{\kern0pt}{\isacharparenright}{\kern0pt}\ {\isacharparenleft}{\kern0pt}hd\ {\isacharparenleft}{\kern0pt}{\isacharparenleft}{\kern0pt}{\isadigit{1}}{\isacharcomma}{\kern0pt}{\isadigit{1}}{\isacharparenright}{\kern0pt}{\isacharhash}{\kern0pt}rev\ {\isacharparenleft}{\kern0pt}s\isactrlsub j{\isacharhash}{\kern0pt}ps{\isacharprime}{\kern0pt}{\isacharat}{\kern0pt}{\isacharbrackleft}{\kern0pt}s\isactrlsub k{\isacharbrackright}{\kern0pt}{\isacharparenright}{\kern0pt}{\isacharparenright}{\kern0pt}{\isacharparenright}{\kern0pt}{\isachardoublequoteclose}\isanewline
\ \ \ \ \ \ \isacommand{unfolding}\isamarkupfalse%
\ valid{\isacharunderscore}{\kern0pt}step{\isacharunderscore}{\kern0pt}def\ \isacommand{by}\isamarkupfalse%
\ auto\isanewline
\isanewline
\ \ \ \ \isacommand{have}\isamarkupfalse%
\ rev{\isacharunderscore}{\kern0pt}simp{\isacharcolon}{\kern0pt}\ {\isachardoublequoteopen}rev\ {\isacharparenleft}{\kern0pt}s\isactrlsub j{\isacharhash}{\kern0pt}ps{\isacharprime}{\kern0pt}{\isacharat}{\kern0pt}{\isacharbrackleft}{\kern0pt}s\isactrlsub k{\isacharbrackright}{\kern0pt}{\isacharparenright}{\kern0pt}\ {\isacharequal}{\kern0pt}\ s\isactrlsub k{\isacharhash}{\kern0pt}{\isacharparenleft}{\kern0pt}rev\ ps{\isacharprime}{\kern0pt}{\isacharparenright}{\kern0pt}{\isacharat}{\kern0pt}{\isacharbrackleft}{\kern0pt}s\isactrlsub j{\isacharbrackright}{\kern0pt}{\isachardoublequoteclose}\ \isacommand{by}\isamarkupfalse%
\ auto\isanewline
\isanewline
\ \ \ \ \isacommand{have}\isamarkupfalse%
\ {\isachardoublequoteopen}knights{\isacharunderscore}{\kern0pt}path\ {\isacharparenleft}{\kern0pt}{\isacharquery}{\kern0pt}b\ {\isacharminus}{\kern0pt}\ {\isacharbraceleft}{\kern0pt}{\isacharparenleft}{\kern0pt}{\isadigit{1}}{\isacharcomma}{\kern0pt}{\isadigit{1}}{\isacharparenright}{\kern0pt}{\isacharbraceright}{\kern0pt}{\isacharparenright}{\kern0pt}\ {\isacharparenleft}{\kern0pt}rev\ {\isacharparenleft}{\kern0pt}s\isactrlsub j{\isacharhash}{\kern0pt}ps{\isacharprime}{\kern0pt}{\isacharat}{\kern0pt}{\isacharbrackleft}{\kern0pt}s\isactrlsub k{\isacharbrackright}{\kern0pt}{\isacharparenright}{\kern0pt}{\isacharparenright}{\kern0pt}{\isachardoublequoteclose}\isanewline
\ \ \ \ \ \ \isacommand{using}\isamarkupfalse%
\ knights{\isacharunderscore}{\kern0pt}path{\isacharunderscore}{\kern0pt}rev{\isacharbrackleft}{\kern0pt}OF\ kp{\isacharprime}{\kern0pt}{\isacharbrackright}{\kern0pt}\ \isacommand{by}\isamarkupfalse%
\ auto\isanewline
\ \ \ \ \isacommand{then}\isamarkupfalse%
\ \isacommand{have}\isamarkupfalse%
\ {\isachardoublequoteopen}{\isacharparenleft}{\kern0pt}{\isadigit{1}}{\isacharcomma}{\kern0pt}{\isadigit{1}}{\isacharparenright}{\kern0pt}\ {\isasymnotin}\ {\isacharparenleft}{\kern0pt}{\isacharquery}{\kern0pt}b\ {\isacharminus}{\kern0pt}\ {\isacharbraceleft}{\kern0pt}{\isacharparenleft}{\kern0pt}{\isadigit{1}}{\isacharcomma}{\kern0pt}{\isadigit{1}}{\isacharparenright}{\kern0pt}{\isacharbraceright}{\kern0pt}{\isacharparenright}{\kern0pt}{\isachardoublequoteclose}\ {\isachardoublequoteopen}valid{\isacharunderscore}{\kern0pt}step\ {\isacharparenleft}{\kern0pt}{\isadigit{1}}{\isacharcomma}{\kern0pt}{\isadigit{1}}{\isacharparenright}{\kern0pt}\ s\isactrlsub k{\isachardoublequoteclose}\ \isanewline
\ \ \ \ \ \ \ \ \ {\isachardoublequoteopen}knights{\isacharunderscore}{\kern0pt}path\ {\isacharparenleft}{\kern0pt}{\isacharquery}{\kern0pt}b\ {\isacharminus}{\kern0pt}\ {\isacharbraceleft}{\kern0pt}{\isacharparenleft}{\kern0pt}{\isadigit{1}}{\isacharcomma}{\kern0pt}{\isadigit{1}}{\isacharparenright}{\kern0pt}{\isacharbraceright}{\kern0pt}{\isacharparenright}{\kern0pt}\ {\isacharparenleft}{\kern0pt}s\isactrlsub k{\isacharhash}{\kern0pt}{\isacharparenleft}{\kern0pt}rev\ ps{\isacharprime}{\kern0pt}{\isacharparenright}{\kern0pt}{\isacharat}{\kern0pt}{\isacharbrackleft}{\kern0pt}s\isactrlsub j{\isacharbrackright}{\kern0pt}{\isacharparenright}{\kern0pt}{\isachardoublequoteclose}\isanewline
\ \ \ \ \ \ \isacommand{using}\isamarkupfalse%
\ assms\ vs{\isacharunderscore}{\kern0pt}s\isactrlsub k\ valid{\isacharunderscore}{\kern0pt}step{\isacharunderscore}{\kern0pt}rev\ \isacommand{by}\isamarkupfalse%
\ {\isacharparenleft}{\kern0pt}auto\ simp{\isacharcolon}{\kern0pt}\ rev{\isacharunderscore}{\kern0pt}simp{\isacharparenright}{\kern0pt}\isanewline
\ \ \ \ \isacommand{then}\isamarkupfalse%
\ \isacommand{have}\isamarkupfalse%
\ {\isachardoublequoteopen}knights{\isacharunderscore}{\kern0pt}path\ {\isacharparenleft}{\kern0pt}{\isacharquery}{\kern0pt}b\ {\isacharminus}{\kern0pt}\ {\isacharbraceleft}{\kern0pt}{\isacharparenleft}{\kern0pt}{\isadigit{1}}{\isacharcomma}{\kern0pt}\ {\isadigit{1}}{\isacharparenright}{\kern0pt}{\isacharbraceright}{\kern0pt}\ {\isasymunion}\ {\isacharbraceleft}{\kern0pt}{\isacharparenleft}{\kern0pt}{\isadigit{1}}{\isacharcomma}{\kern0pt}\ {\isadigit{1}}{\isacharparenright}{\kern0pt}{\isacharbraceright}{\kern0pt}{\isacharparenright}{\kern0pt}\ {\isacharparenleft}{\kern0pt}{\isacharparenleft}{\kern0pt}{\isadigit{1}}{\isacharcomma}{\kern0pt}{\isadigit{1}}{\isacharparenright}{\kern0pt}{\isacharhash}{\kern0pt}s\isactrlsub k{\isacharhash}{\kern0pt}{\isacharparenleft}{\kern0pt}rev\ ps{\isacharprime}{\kern0pt}{\isacharparenright}{\kern0pt}{\isacharat}{\kern0pt}{\isacharbrackleft}{\kern0pt}s\isactrlsub j{\isacharbrackright}{\kern0pt}{\isacharparenright}{\kern0pt}{\isachardoublequoteclose}\isanewline
\ \ \ \ \ \ \isacommand{using}\isamarkupfalse%
\ knights{\isacharunderscore}{\kern0pt}path{\isachardot}{\kern0pt}intros{\isacharparenleft}{\kern0pt}{\isadigit{2}}{\isacharparenright}{\kern0pt}{\isacharbrackleft}{\kern0pt}of\ {\isachardoublequoteopen}{\isacharparenleft}{\kern0pt}{\isadigit{1}}{\isacharcomma}{\kern0pt}{\isadigit{1}}{\isacharparenright}{\kern0pt}{\isachardoublequoteclose}\ {\isachardoublequoteopen}{\isacharquery}{\kern0pt}b\ {\isacharminus}{\kern0pt}\ {\isacharbraceleft}{\kern0pt}{\isacharparenleft}{\kern0pt}{\isadigit{1}}{\isacharcomma}{\kern0pt}{\isadigit{1}}{\isacharparenright}{\kern0pt}{\isacharbraceright}{\kern0pt}{\isachardoublequoteclose}\ s\isactrlsub k\ {\isachardoublequoteopen}{\isacharparenleft}{\kern0pt}rev\ ps{\isacharprime}{\kern0pt}{\isacharparenright}{\kern0pt}{\isacharat}{\kern0pt}{\isacharbrackleft}{\kern0pt}s\isactrlsub j{\isacharbrackright}{\kern0pt}{\isachardoublequoteclose}{\isacharbrackright}{\kern0pt}\ \isacommand{by}\isamarkupfalse%
\ auto\isanewline
\ \ \ \ \isacommand{then}\isamarkupfalse%
\ \isacommand{have}\isamarkupfalse%
\ {\isachardoublequoteopen}knights{\isacharunderscore}{\kern0pt}path\ {\isacharquery}{\kern0pt}b\ {\isacharparenleft}{\kern0pt}{\isacharparenleft}{\kern0pt}{\isadigit{1}}{\isacharcomma}{\kern0pt}{\isadigit{1}}{\isacharparenright}{\kern0pt}{\isacharhash}{\kern0pt}rev\ {\isacharparenleft}{\kern0pt}s\isactrlsub j{\isacharhash}{\kern0pt}ps{\isacharprime}{\kern0pt}{\isacharat}{\kern0pt}{\isacharbrackleft}{\kern0pt}s\isactrlsub k{\isacharbrackright}{\kern0pt}{\isacharparenright}{\kern0pt}{\isacharparenright}{\kern0pt}{\isachardoublequoteclose}\isanewline
\ \ \ \ \ \ \isacommand{using}\isamarkupfalse%
\ assms\ \isacommand{by}\isamarkupfalse%
\ {\isacharparenleft}{\kern0pt}simp\ add{\isacharcolon}{\kern0pt}\ board{\isacharunderscore}{\kern0pt}def\ insert{\isacharunderscore}{\kern0pt}absorb\ rev{\isacharunderscore}{\kern0pt}simp{\isacharparenright}{\kern0pt}\isanewline
\ \ \ \ \isacommand{then}\isamarkupfalse%
\ \isacommand{have}\isamarkupfalse%
\ {\isachardoublequoteopen}knights{\isacharunderscore}{\kern0pt}circuit\ {\isacharquery}{\kern0pt}b\ {\isacharparenleft}{\kern0pt}{\isacharparenleft}{\kern0pt}{\isadigit{1}}{\isacharcomma}{\kern0pt}{\isadigit{1}}{\isacharparenright}{\kern0pt}{\isacharhash}{\kern0pt}rev\ {\isacharparenleft}{\kern0pt}s\isactrlsub j{\isacharhash}{\kern0pt}ps{\isacharprime}{\kern0pt}{\isacharat}{\kern0pt}{\isacharbrackleft}{\kern0pt}s\isactrlsub k{\isacharbrackright}{\kern0pt}{\isacharparenright}{\kern0pt}{\isacharparenright}{\kern0pt}{\isachardoublequoteclose}\isanewline
\ \ \ \ \ \ \isacommand{unfolding}\isamarkupfalse%
\ knights{\isacharunderscore}{\kern0pt}circuit{\isacharunderscore}{\kern0pt}def\ \isacommand{using}\isamarkupfalse%
\ vs\ \isacommand{by}\isamarkupfalse%
\ auto\isanewline
\ \ \ \ \isacommand{then}\isamarkupfalse%
\ \isacommand{show}\isamarkupfalse%
\ {\isacharquery}{\kern0pt}thesis\isanewline
\ \ \ \ \ \ \isacommand{using}\isamarkupfalse%
\ {\isacartoucheopen}s\isactrlsub j\ {\isacharequal}{\kern0pt}\ {\isacharparenleft}{\kern0pt}{\isadigit{3}}{\isacharcomma}{\kern0pt}{\isadigit{2}}{\isacharparenright}{\kern0pt}{\isacartoucheclose}\ \isacommand{by}\isamarkupfalse%
\ auto\isanewline
\ \ \isacommand{qed}\isamarkupfalse%
\isanewline
\isacommand{qed}\isamarkupfalse%
%
\endisatagproof
{\isafoldproof}%
%
\isadelimproof
%
\endisadelimproof
%
\isadelimdocument
%
\endisadelimdocument
%
\isatagdocument
%
\isamarkupsection{Transposing Paths and Boards%
}
\isamarkuptrue%
%
\isamarkupsubsection{Implementation of Path and Board Transposition%
}
\isamarkuptrue%
%
\endisatagdocument
{\isafolddocument}%
%
\isadelimdocument
%
\endisadelimdocument
\isacommand{definition}\isamarkupfalse%
\ {\isachardoublequoteopen}transpose{\isacharunderscore}{\kern0pt}square\ s\isactrlsub i\ {\isacharequal}{\kern0pt}\ {\isacharparenleft}{\kern0pt}case\ s\isactrlsub i\ of\ {\isacharparenleft}{\kern0pt}i{\isacharcomma}{\kern0pt}j{\isacharparenright}{\kern0pt}\ {\isasymRightarrow}\ {\isacharparenleft}{\kern0pt}j{\isacharcomma}{\kern0pt}i{\isacharparenright}{\kern0pt}{\isacharparenright}{\kern0pt}{\isachardoublequoteclose}\isanewline
\isanewline
\isacommand{fun}\isamarkupfalse%
\ transpose\ {\isacharcolon}{\kern0pt}{\isacharcolon}{\kern0pt}\ {\isachardoublequoteopen}path\ {\isasymRightarrow}\ path{\isachardoublequoteclose}\ \isakeyword{where}\isanewline
\ \ {\isachardoublequoteopen}transpose\ {\isacharbrackleft}{\kern0pt}{\isacharbrackright}{\kern0pt}\ {\isacharequal}{\kern0pt}\ {\isacharbrackleft}{\kern0pt}{\isacharbrackright}{\kern0pt}{\isachardoublequoteclose}\isanewline
{\isacharbar}{\kern0pt}\ {\isachardoublequoteopen}transpose\ {\isacharparenleft}{\kern0pt}s\isactrlsub i{\isacharhash}{\kern0pt}ps{\isacharparenright}{\kern0pt}\ {\isacharequal}{\kern0pt}\ {\isacharparenleft}{\kern0pt}transpose{\isacharunderscore}{\kern0pt}square\ s\isactrlsub i{\isacharparenright}{\kern0pt}{\isacharhash}{\kern0pt}transpose\ ps{\isachardoublequoteclose}\isanewline
\isanewline
\isacommand{definition}\isamarkupfalse%
\ transpose{\isacharunderscore}{\kern0pt}board\ {\isacharcolon}{\kern0pt}{\isacharcolon}{\kern0pt}\ {\isachardoublequoteopen}board\ {\isasymRightarrow}\ board{\isachardoublequoteclose}\ \isakeyword{where}\isanewline
\ \ {\isachardoublequoteopen}transpose{\isacharunderscore}{\kern0pt}board\ b\ {\isasymequiv}\ {\isacharbraceleft}{\kern0pt}{\isacharparenleft}{\kern0pt}j{\isacharcomma}{\kern0pt}i{\isacharparenright}{\kern0pt}\ {\isacharbar}{\kern0pt}i\ j{\isachardot}{\kern0pt}\ {\isacharparenleft}{\kern0pt}i{\isacharcomma}{\kern0pt}j{\isacharparenright}{\kern0pt}\ {\isasymin}\ b{\isacharbraceright}{\kern0pt}{\isachardoublequoteclose}%
\isadelimdocument
%
\endisadelimdocument
%
\isatagdocument
%
\isamarkupsubsection{Correctness of Path and Board Transposition%
}
\isamarkuptrue%
%
\endisatagdocument
{\isafolddocument}%
%
\isadelimdocument
%
\endisadelimdocument
\isacommand{lemma}\isamarkupfalse%
\ transpose{\isadigit{2}}{\isacharcolon}{\kern0pt}\ {\isachardoublequoteopen}transpose{\isacharunderscore}{\kern0pt}square\ {\isacharparenleft}{\kern0pt}transpose{\isacharunderscore}{\kern0pt}square\ s\isactrlsub i{\isacharparenright}{\kern0pt}\ {\isacharequal}{\kern0pt}\ s\isactrlsub i{\isachardoublequoteclose}\isanewline
%
\isadelimproof
\ \ %
\endisadelimproof
%
\isatagproof
\isacommand{unfolding}\isamarkupfalse%
\ transpose{\isacharunderscore}{\kern0pt}square{\isacharunderscore}{\kern0pt}def\ \isacommand{by}\isamarkupfalse%
\ {\isacharparenleft}{\kern0pt}auto\ split{\isacharcolon}{\kern0pt}\ prod{\isachardot}{\kern0pt}splits{\isacharparenright}{\kern0pt}%
\endisatagproof
{\isafoldproof}%
%
\isadelimproof
\isanewline
%
\endisadelimproof
\isanewline
\isacommand{lemma}\isamarkupfalse%
\ transpose{\isacharunderscore}{\kern0pt}nil{\isacharcolon}{\kern0pt}\ {\isachardoublequoteopen}ps\ {\isacharequal}{\kern0pt}\ {\isacharbrackleft}{\kern0pt}{\isacharbrackright}{\kern0pt}\ {\isasymlongleftrightarrow}\ transpose\ ps\ {\isacharequal}{\kern0pt}\ {\isacharbrackleft}{\kern0pt}{\isacharbrackright}{\kern0pt}{\isachardoublequoteclose}\isanewline
%
\isadelimproof
\ \ %
\endisadelimproof
%
\isatagproof
\isacommand{using}\isamarkupfalse%
\ transpose{\isachardot}{\kern0pt}elims\ \isacommand{by}\isamarkupfalse%
\ blast%
\endisatagproof
{\isafoldproof}%
%
\isadelimproof
\isanewline
%
\endisadelimproof
\isanewline
\isacommand{lemma}\isamarkupfalse%
\ transpose{\isacharunderscore}{\kern0pt}length{\isacharcolon}{\kern0pt}\ {\isachardoublequoteopen}length\ ps\ {\isacharequal}{\kern0pt}\ length\ {\isacharparenleft}{\kern0pt}transpose\ ps{\isacharparenright}{\kern0pt}{\isachardoublequoteclose}\isanewline
%
\isadelimproof
\ \ %
\endisadelimproof
%
\isatagproof
\isacommand{by}\isamarkupfalse%
\ {\isacharparenleft}{\kern0pt}induction\ ps{\isacharparenright}{\kern0pt}\ auto%
\endisatagproof
{\isafoldproof}%
%
\isadelimproof
\isanewline
%
\endisadelimproof
\isanewline
\isacommand{lemma}\isamarkupfalse%
\ hd{\isacharunderscore}{\kern0pt}transpose{\isacharcolon}{\kern0pt}\ {\isachardoublequoteopen}ps\ {\isasymnoteq}{\isacharbrackleft}{\kern0pt}{\isacharbrackright}{\kern0pt}\ {\isasymLongrightarrow}\ hd\ {\isacharparenleft}{\kern0pt}transpose\ ps{\isacharparenright}{\kern0pt}\ {\isacharequal}{\kern0pt}\ transpose{\isacharunderscore}{\kern0pt}square\ {\isacharparenleft}{\kern0pt}hd\ ps{\isacharparenright}{\kern0pt}{\isachardoublequoteclose}\isanewline
%
\isadelimproof
\ \ %
\endisadelimproof
%
\isatagproof
\isacommand{by}\isamarkupfalse%
\ {\isacharparenleft}{\kern0pt}induction\ ps{\isacharparenright}{\kern0pt}\ {\isacharparenleft}{\kern0pt}auto\ simp{\isacharcolon}{\kern0pt}\ transpose{\isacharunderscore}{\kern0pt}square{\isacharunderscore}{\kern0pt}def{\isacharparenright}{\kern0pt}%
\endisatagproof
{\isafoldproof}%
%
\isadelimproof
\isanewline
%
\endisadelimproof
\isanewline
\isacommand{lemma}\isamarkupfalse%
\ last{\isacharunderscore}{\kern0pt}transpose{\isacharcolon}{\kern0pt}\ {\isachardoublequoteopen}ps\ {\isasymnoteq}{\isacharbrackleft}{\kern0pt}{\isacharbrackright}{\kern0pt}\ {\isasymLongrightarrow}\ last\ {\isacharparenleft}{\kern0pt}transpose\ ps{\isacharparenright}{\kern0pt}\ {\isacharequal}{\kern0pt}\ transpose{\isacharunderscore}{\kern0pt}square\ {\isacharparenleft}{\kern0pt}last\ ps{\isacharparenright}{\kern0pt}{\isachardoublequoteclose}\isanewline
%
\isadelimproof
%
\endisadelimproof
%
\isatagproof
\isacommand{proof}\isamarkupfalse%
\ {\isacharparenleft}{\kern0pt}induction\ ps{\isacharparenright}{\kern0pt}\isanewline
\ \ \isacommand{case}\isamarkupfalse%
\ {\isacharparenleft}{\kern0pt}Cons\ s\isactrlsub i\ ps{\isacharparenright}{\kern0pt}\isanewline
\ \ \isacommand{then}\isamarkupfalse%
\ \isacommand{show}\isamarkupfalse%
\ {\isacharquery}{\kern0pt}case\isanewline
\ \ \isacommand{proof}\isamarkupfalse%
\ {\isacharparenleft}{\kern0pt}cases\ {\isachardoublequoteopen}ps\ {\isacharequal}{\kern0pt}\ {\isacharbrackleft}{\kern0pt}{\isacharbrackright}{\kern0pt}{\isachardoublequoteclose}{\isacharparenright}{\kern0pt}\isanewline
\ \ \ \ \isacommand{case}\isamarkupfalse%
\ True\isanewline
\ \ \ \ \isacommand{then}\isamarkupfalse%
\ \isacommand{show}\isamarkupfalse%
\ {\isacharquery}{\kern0pt}thesis\ \isacommand{using}\isamarkupfalse%
\ Cons\ \isacommand{by}\isamarkupfalse%
\ {\isacharparenleft}{\kern0pt}auto\ simp{\isacharcolon}{\kern0pt}\ transpose{\isacharunderscore}{\kern0pt}square{\isacharunderscore}{\kern0pt}def{\isacharparenright}{\kern0pt}\ \ \ \ \ \ \isanewline
\ \ \isacommand{next}\isamarkupfalse%
\isanewline
\ \ \ \ \isacommand{case}\isamarkupfalse%
\ False\isanewline
\ \ \ \ \isacommand{then}\isamarkupfalse%
\ \isacommand{show}\isamarkupfalse%
\ {\isacharquery}{\kern0pt}thesis\ \isacommand{using}\isamarkupfalse%
\ Cons\ transpose{\isacharunderscore}{\kern0pt}nil\ \isacommand{by}\isamarkupfalse%
\ auto\isanewline
\ \ \isacommand{qed}\isamarkupfalse%
\isanewline
\isacommand{qed}\isamarkupfalse%
\ auto%
\endisatagproof
{\isafoldproof}%
%
\isadelimproof
\isanewline
%
\endisadelimproof
\isanewline
\isacommand{lemma}\isamarkupfalse%
\ take{\isacharunderscore}{\kern0pt}transpose{\isacharcolon}{\kern0pt}\ \isanewline
\ \ \isakeyword{shows}\ {\isachardoublequoteopen}take\ k\ {\isacharparenleft}{\kern0pt}transpose\ ps{\isacharparenright}{\kern0pt}\ {\isacharequal}{\kern0pt}\ transpose\ {\isacharparenleft}{\kern0pt}take\ k\ ps{\isacharparenright}{\kern0pt}{\isachardoublequoteclose}\isanewline
%
\isadelimproof
%
\endisadelimproof
%
\isatagproof
\isacommand{proof}\isamarkupfalse%
\ {\isacharparenleft}{\kern0pt}induction\ ps\ arbitrary{\isacharcolon}{\kern0pt}\ k{\isacharparenright}{\kern0pt}\isanewline
\ \ \isacommand{case}\isamarkupfalse%
\ Nil\isanewline
\ \ \isacommand{then}\isamarkupfalse%
\ \isacommand{show}\isamarkupfalse%
\ {\isacharquery}{\kern0pt}case\ \isacommand{by}\isamarkupfalse%
\ auto\isanewline
\isacommand{next}\isamarkupfalse%
\isanewline
\ \ \isacommand{case}\isamarkupfalse%
\ {\isacharparenleft}{\kern0pt}Cons\ s\isactrlsub i\ ps{\isacharparenright}{\kern0pt}\isanewline
\ \ \isacommand{then}\isamarkupfalse%
\ \isacommand{obtain}\isamarkupfalse%
\ i\ j\ \isakeyword{where}\ {\isachardoublequoteopen}s\isactrlsub i\ {\isacharequal}{\kern0pt}\ {\isacharparenleft}{\kern0pt}i{\isacharcomma}{\kern0pt}j{\isacharparenright}{\kern0pt}{\isachardoublequoteclose}\ \isacommand{by}\isamarkupfalse%
\ force\isanewline
\ \ \isacommand{then}\isamarkupfalse%
\ \isacommand{have}\isamarkupfalse%
\ {\isachardoublequoteopen}k\ {\isacharequal}{\kern0pt}\ {\isadigit{0}}\ {\isasymor}\ k\ {\isachargreater}{\kern0pt}\ {\isadigit{0}}{\isachardoublequoteclose}\ \isacommand{by}\isamarkupfalse%
\ auto\isanewline
\ \ \isacommand{then}\isamarkupfalse%
\ \isacommand{show}\isamarkupfalse%
\ {\isacharquery}{\kern0pt}case\ \isanewline
\ \ \isacommand{proof}\isamarkupfalse%
\ {\isacharparenleft}{\kern0pt}elim\ disjE{\isacharparenright}{\kern0pt}\isanewline
\ \ \ \ \isacommand{assume}\isamarkupfalse%
\ {\isachardoublequoteopen}k\ {\isachargreater}{\kern0pt}\ {\isadigit{0}}{\isachardoublequoteclose}\isanewline
\ \ \ \ \isacommand{then}\isamarkupfalse%
\ \isacommand{show}\isamarkupfalse%
\ {\isacharquery}{\kern0pt}thesis\ \isacommand{using}\isamarkupfalse%
\ Cons{\isachardot}{\kern0pt}IH\ \isacommand{by}\isamarkupfalse%
\ {\isacharparenleft}{\kern0pt}auto\ simp{\isacharcolon}{\kern0pt}\ {\isacartoucheopen}s\isactrlsub i\ {\isacharequal}{\kern0pt}\ {\isacharparenleft}{\kern0pt}i{\isacharcomma}{\kern0pt}j{\isacharparenright}{\kern0pt}{\isacartoucheclose}\ take{\isacharunderscore}{\kern0pt}Cons{\isacharprime}{\kern0pt}{\isacharparenright}{\kern0pt}\isanewline
\ \ \isacommand{qed}\isamarkupfalse%
\ auto\isanewline
\isacommand{qed}\isamarkupfalse%
%
\endisatagproof
{\isafoldproof}%
%
\isadelimproof
\isanewline
%
\endisadelimproof
\isanewline
\isacommand{lemma}\isamarkupfalse%
\ drop{\isacharunderscore}{\kern0pt}transpose{\isacharcolon}{\kern0pt}\ \isanewline
\ \ \isakeyword{shows}\ {\isachardoublequoteopen}drop\ k\ {\isacharparenleft}{\kern0pt}transpose\ ps{\isacharparenright}{\kern0pt}\ {\isacharequal}{\kern0pt}\ transpose\ {\isacharparenleft}{\kern0pt}drop\ k\ ps{\isacharparenright}{\kern0pt}{\isachardoublequoteclose}\isanewline
%
\isadelimproof
%
\endisadelimproof
%
\isatagproof
\isacommand{proof}\isamarkupfalse%
\ {\isacharparenleft}{\kern0pt}induction\ ps\ arbitrary{\isacharcolon}{\kern0pt}\ k{\isacharparenright}{\kern0pt}\isanewline
\ \ \isacommand{case}\isamarkupfalse%
\ Nil\isanewline
\ \ \isacommand{then}\isamarkupfalse%
\ \isacommand{show}\isamarkupfalse%
\ {\isacharquery}{\kern0pt}case\ \isacommand{by}\isamarkupfalse%
\ auto\isanewline
\isacommand{next}\isamarkupfalse%
\isanewline
\ \ \isacommand{case}\isamarkupfalse%
\ {\isacharparenleft}{\kern0pt}Cons\ s\isactrlsub i\ ps{\isacharparenright}{\kern0pt}\isanewline
\ \ \isacommand{then}\isamarkupfalse%
\ \isacommand{obtain}\isamarkupfalse%
\ i\ j\ \isakeyword{where}\ {\isachardoublequoteopen}s\isactrlsub i\ {\isacharequal}{\kern0pt}\ {\isacharparenleft}{\kern0pt}i{\isacharcomma}{\kern0pt}j{\isacharparenright}{\kern0pt}{\isachardoublequoteclose}\ \isacommand{by}\isamarkupfalse%
\ force\isanewline
\ \ \isacommand{then}\isamarkupfalse%
\ \isacommand{have}\isamarkupfalse%
\ {\isachardoublequoteopen}k\ {\isacharequal}{\kern0pt}\ {\isadigit{0}}\ {\isasymor}\ k\ {\isachargreater}{\kern0pt}\ {\isadigit{0}}{\isachardoublequoteclose}\ \isacommand{by}\isamarkupfalse%
\ auto\isanewline
\ \ \isacommand{then}\isamarkupfalse%
\ \isacommand{show}\isamarkupfalse%
\ {\isacharquery}{\kern0pt}case\ \isanewline
\ \ \isacommand{proof}\isamarkupfalse%
\ {\isacharparenleft}{\kern0pt}elim\ disjE{\isacharparenright}{\kern0pt}\isanewline
\ \ \ \ \isacommand{assume}\isamarkupfalse%
\ {\isachardoublequoteopen}k\ {\isachargreater}{\kern0pt}\ {\isadigit{0}}{\isachardoublequoteclose}\isanewline
\ \ \ \ \isacommand{then}\isamarkupfalse%
\ \isacommand{show}\isamarkupfalse%
\ {\isacharquery}{\kern0pt}thesis\ \isacommand{using}\isamarkupfalse%
\ Cons{\isachardot}{\kern0pt}IH\ \isacommand{by}\isamarkupfalse%
\ {\isacharparenleft}{\kern0pt}auto\ simp{\isacharcolon}{\kern0pt}\ {\isacartoucheopen}s\isactrlsub i\ {\isacharequal}{\kern0pt}\ {\isacharparenleft}{\kern0pt}i{\isacharcomma}{\kern0pt}j{\isacharparenright}{\kern0pt}{\isacartoucheclose}\ drop{\isacharunderscore}{\kern0pt}Cons{\isacharprime}{\kern0pt}{\isacharparenright}{\kern0pt}\isanewline
\ \ \isacommand{qed}\isamarkupfalse%
\ auto\isanewline
\isacommand{qed}\isamarkupfalse%
%
\endisatagproof
{\isafoldproof}%
%
\isadelimproof
\isanewline
%
\endisadelimproof
\isanewline
\isacommand{lemma}\isamarkupfalse%
\ transpose{\isacharunderscore}{\kern0pt}board{\isacharunderscore}{\kern0pt}correct{\isacharcolon}{\kern0pt}\ {\isachardoublequoteopen}s\isactrlsub i\ {\isasymin}\ b\ {\isasymlongleftrightarrow}\ {\isacharparenleft}{\kern0pt}transpose{\isacharunderscore}{\kern0pt}square\ s\isactrlsub i{\isacharparenright}{\kern0pt}\ {\isasymin}\ transpose{\isacharunderscore}{\kern0pt}board\ b{\isachardoublequoteclose}\isanewline
%
\isadelimproof
\ \ %
\endisadelimproof
%
\isatagproof
\isacommand{unfolding}\isamarkupfalse%
\ transpose{\isacharunderscore}{\kern0pt}board{\isacharunderscore}{\kern0pt}def\ transpose{\isacharunderscore}{\kern0pt}square{\isacharunderscore}{\kern0pt}def\ \isacommand{by}\isamarkupfalse%
\ {\isacharparenleft}{\kern0pt}auto\ split{\isacharcolon}{\kern0pt}\ prod{\isachardot}{\kern0pt}splits{\isacharparenright}{\kern0pt}%
\endisatagproof
{\isafoldproof}%
%
\isadelimproof
\isanewline
%
\endisadelimproof
\isanewline
\isacommand{lemma}\isamarkupfalse%
\ transpose{\isacharunderscore}{\kern0pt}board{\isacharcolon}{\kern0pt}\ {\isachardoublequoteopen}transpose{\isacharunderscore}{\kern0pt}board\ {\isacharparenleft}{\kern0pt}board\ n\ m{\isacharparenright}{\kern0pt}\ {\isacharequal}{\kern0pt}\ board\ m\ n{\isachardoublequoteclose}\isanewline
%
\isadelimproof
\ \ %
\endisadelimproof
%
\isatagproof
\isacommand{unfolding}\isamarkupfalse%
\ board{\isacharunderscore}{\kern0pt}def\ \isacommand{using}\isamarkupfalse%
\ transpose{\isacharunderscore}{\kern0pt}board{\isacharunderscore}{\kern0pt}correct\ \isacommand{by}\isamarkupfalse%
\ {\isacharparenleft}{\kern0pt}auto\ simp{\isacharcolon}{\kern0pt}\ transpose{\isacharunderscore}{\kern0pt}square{\isacharunderscore}{\kern0pt}def{\isacharparenright}{\kern0pt}%
\endisatagproof
{\isafoldproof}%
%
\isadelimproof
\isanewline
%
\endisadelimproof
\isanewline
\isacommand{lemma}\isamarkupfalse%
\ insert{\isacharunderscore}{\kern0pt}transpose{\isacharunderscore}{\kern0pt}board{\isacharcolon}{\kern0pt}\ \isanewline
\ \ {\isachardoublequoteopen}insert\ {\isacharparenleft}{\kern0pt}transpose{\isacharunderscore}{\kern0pt}square\ s\isactrlsub i{\isacharparenright}{\kern0pt}\ {\isacharparenleft}{\kern0pt}transpose{\isacharunderscore}{\kern0pt}board\ b{\isacharparenright}{\kern0pt}\ {\isacharequal}{\kern0pt}\ transpose{\isacharunderscore}{\kern0pt}board\ {\isacharparenleft}{\kern0pt}insert\ s\isactrlsub i\ b{\isacharparenright}{\kern0pt}{\isachardoublequoteclose}\isanewline
%
\isadelimproof
\ \ %
\endisadelimproof
%
\isatagproof
\isacommand{unfolding}\isamarkupfalse%
\ transpose{\isacharunderscore}{\kern0pt}board{\isacharunderscore}{\kern0pt}def\ transpose{\isacharunderscore}{\kern0pt}square{\isacharunderscore}{\kern0pt}def\ \isacommand{by}\isamarkupfalse%
\ {\isacharparenleft}{\kern0pt}auto\ split{\isacharcolon}{\kern0pt}\ prod{\isachardot}{\kern0pt}splits{\isacharparenright}{\kern0pt}%
\endisatagproof
{\isafoldproof}%
%
\isadelimproof
\isanewline
%
\endisadelimproof
\isanewline
\isacommand{lemma}\isamarkupfalse%
\ transpose{\isacharunderscore}{\kern0pt}board{\isadigit{2}}{\isacharcolon}{\kern0pt}\ {\isachardoublequoteopen}transpose{\isacharunderscore}{\kern0pt}board\ {\isacharparenleft}{\kern0pt}transpose{\isacharunderscore}{\kern0pt}board\ b{\isacharparenright}{\kern0pt}\ {\isacharequal}{\kern0pt}\ b{\isachardoublequoteclose}\isanewline
%
\isadelimproof
\ \ %
\endisadelimproof
%
\isatagproof
\isacommand{unfolding}\isamarkupfalse%
\ transpose{\isacharunderscore}{\kern0pt}board{\isacharunderscore}{\kern0pt}def\ \isacommand{by}\isamarkupfalse%
\ auto%
\endisatagproof
{\isafoldproof}%
%
\isadelimproof
\isanewline
%
\endisadelimproof
\isanewline
\isacommand{lemma}\isamarkupfalse%
\ transpose{\isacharunderscore}{\kern0pt}union{\isacharcolon}{\kern0pt}\ {\isachardoublequoteopen}transpose{\isacharunderscore}{\kern0pt}board\ {\isacharparenleft}{\kern0pt}b\isactrlsub {\isadigit{1}}\ {\isasymunion}\ b\isactrlsub {\isadigit{2}}{\isacharparenright}{\kern0pt}\ {\isacharequal}{\kern0pt}\ transpose{\isacharunderscore}{\kern0pt}board\ b\isactrlsub {\isadigit{1}}\ {\isasymunion}\ transpose{\isacharunderscore}{\kern0pt}board\ b\isactrlsub {\isadigit{2}}{\isachardoublequoteclose}\isanewline
%
\isadelimproof
\ \ %
\endisadelimproof
%
\isatagproof
\isacommand{unfolding}\isamarkupfalse%
\ transpose{\isacharunderscore}{\kern0pt}board{\isacharunderscore}{\kern0pt}def\ \isacommand{by}\isamarkupfalse%
\ auto%
\endisatagproof
{\isafoldproof}%
%
\isadelimproof
\isanewline
%
\endisadelimproof
\isanewline
\isacommand{lemma}\isamarkupfalse%
\ transpose{\isacharunderscore}{\kern0pt}valid{\isacharunderscore}{\kern0pt}step{\isacharcolon}{\kern0pt}\ \isanewline
\ \ {\isachardoublequoteopen}valid{\isacharunderscore}{\kern0pt}step\ s\isactrlsub i\ s\isactrlsub j\ {\isasymlongleftrightarrow}\ valid{\isacharunderscore}{\kern0pt}step\ {\isacharparenleft}{\kern0pt}transpose{\isacharunderscore}{\kern0pt}square\ s\isactrlsub i{\isacharparenright}{\kern0pt}\ {\isacharparenleft}{\kern0pt}transpose{\isacharunderscore}{\kern0pt}square\ s\isactrlsub j{\isacharparenright}{\kern0pt}{\isachardoublequoteclose}\isanewline
%
\isadelimproof
\ \ %
\endisadelimproof
%
\isatagproof
\isacommand{unfolding}\isamarkupfalse%
\ valid{\isacharunderscore}{\kern0pt}step{\isacharunderscore}{\kern0pt}def\ transpose{\isacharunderscore}{\kern0pt}square{\isacharunderscore}{\kern0pt}def\ \isacommand{by}\isamarkupfalse%
\ {\isacharparenleft}{\kern0pt}auto\ split{\isacharcolon}{\kern0pt}\ prod{\isachardot}{\kern0pt}splits{\isacharparenright}{\kern0pt}%
\endisatagproof
{\isafoldproof}%
%
\isadelimproof
\isanewline
%
\endisadelimproof
\isanewline
\isacommand{lemma}\isamarkupfalse%
\ transpose{\isacharunderscore}{\kern0pt}knights{\isacharunderscore}{\kern0pt}path{\isacharprime}{\kern0pt}{\isacharcolon}{\kern0pt}\ \ \isanewline
\ \ \isakeyword{assumes}\ {\isachardoublequoteopen}knights{\isacharunderscore}{\kern0pt}path\ b\ ps{\isachardoublequoteclose}\ \isanewline
\ \ \isakeyword{shows}\ {\isachardoublequoteopen}knights{\isacharunderscore}{\kern0pt}path\ {\isacharparenleft}{\kern0pt}transpose{\isacharunderscore}{\kern0pt}board\ b{\isacharparenright}{\kern0pt}\ {\isacharparenleft}{\kern0pt}transpose\ ps{\isacharparenright}{\kern0pt}{\isachardoublequoteclose}\isanewline
%
\isadelimproof
\ \ %
\endisadelimproof
%
\isatagproof
\isacommand{using}\isamarkupfalse%
\ assms\isanewline
\isacommand{proof}\isamarkupfalse%
\ {\isacharparenleft}{\kern0pt}induction\ rule{\isacharcolon}{\kern0pt}\ knights{\isacharunderscore}{\kern0pt}path{\isachardot}{\kern0pt}induct{\isacharparenright}{\kern0pt}\isanewline
\ \ \isacommand{case}\isamarkupfalse%
\ {\isacharparenleft}{\kern0pt}{\isadigit{1}}\ s\isactrlsub i{\isacharparenright}{\kern0pt}\isanewline
\ \ \isacommand{then}\isamarkupfalse%
\ \isacommand{have}\isamarkupfalse%
\ {\isachardoublequoteopen}transpose{\isacharunderscore}{\kern0pt}board\ {\isacharbraceleft}{\kern0pt}s\isactrlsub i{\isacharbraceright}{\kern0pt}\ {\isacharequal}{\kern0pt}\ {\isacharbraceleft}{\kern0pt}transpose{\isacharunderscore}{\kern0pt}square\ s\isactrlsub i{\isacharbraceright}{\kern0pt}{\isachardoublequoteclose}\ {\isachardoublequoteopen}transpose\ {\isacharbrackleft}{\kern0pt}s\isactrlsub i{\isacharbrackright}{\kern0pt}\ {\isacharequal}{\kern0pt}\ {\isacharbrackleft}{\kern0pt}transpose{\isacharunderscore}{\kern0pt}square\ s\isactrlsub i{\isacharbrackright}{\kern0pt}{\isachardoublequoteclose}\isanewline
\ \ \ \ \isacommand{using}\isamarkupfalse%
\ transpose{\isacharunderscore}{\kern0pt}board{\isacharunderscore}{\kern0pt}correct\ \isacommand{by}\isamarkupfalse%
\ {\isacharparenleft}{\kern0pt}auto\ simp{\isacharcolon}{\kern0pt}\ transpose{\isacharunderscore}{\kern0pt}square{\isacharunderscore}{\kern0pt}def\ split{\isacharcolon}{\kern0pt}\ prod{\isachardot}{\kern0pt}splits{\isacharparenright}{\kern0pt}\isanewline
\ \ \isacommand{then}\isamarkupfalse%
\ \isacommand{show}\isamarkupfalse%
\ {\isacharquery}{\kern0pt}case\ \isacommand{by}\isamarkupfalse%
\ {\isacharparenleft}{\kern0pt}auto\ intro{\isacharcolon}{\kern0pt}\ knights{\isacharunderscore}{\kern0pt}path{\isachardot}{\kern0pt}intros{\isacharparenright}{\kern0pt}\isanewline
\isacommand{next}\isamarkupfalse%
\isanewline
\ \ \isacommand{case}\isamarkupfalse%
\ {\isacharparenleft}{\kern0pt}{\isadigit{2}}\ s\isactrlsub i\ b\ s\isactrlsub j\ ps{\isacharparenright}{\kern0pt}\isanewline
\ \ \isacommand{then}\isamarkupfalse%
\ \isacommand{have}\isamarkupfalse%
\ prems{\isacharcolon}{\kern0pt}\ {\isachardoublequoteopen}transpose{\isacharunderscore}{\kern0pt}square\ s\isactrlsub i\ {\isasymnotin}\ transpose{\isacharunderscore}{\kern0pt}board\ b{\isachardoublequoteclose}\ \isanewline
\ \ \ \ \ \ \ \ \ \ \ \ {\isachardoublequoteopen}valid{\isacharunderscore}{\kern0pt}step\ {\isacharparenleft}{\kern0pt}transpose{\isacharunderscore}{\kern0pt}square\ s\isactrlsub i{\isacharparenright}{\kern0pt}\ {\isacharparenleft}{\kern0pt}transpose{\isacharunderscore}{\kern0pt}square\ s\isactrlsub j{\isacharparenright}{\kern0pt}{\isachardoublequoteclose}\ \isanewline
\ \ \ \ \ \ \ \ \ \ \ \ \isakeyword{and}\ {\isachardoublequoteopen}transpose\ {\isacharparenleft}{\kern0pt}s\isactrlsub j{\isacharhash}{\kern0pt}ps{\isacharparenright}{\kern0pt}\ {\isacharequal}{\kern0pt}\ transpose{\isacharunderscore}{\kern0pt}square\ s\isactrlsub j{\isacharhash}{\kern0pt}transpose\ ps{\isachardoublequoteclose}\isanewline
\ \ \ \ \isacommand{using}\isamarkupfalse%
\ {\isadigit{2}}\ transpose{\isacharunderscore}{\kern0pt}board{\isacharunderscore}{\kern0pt}correct\ transpose{\isacharunderscore}{\kern0pt}valid{\isacharunderscore}{\kern0pt}step\ \isacommand{by}\isamarkupfalse%
\ auto\isanewline
\ \ \isacommand{then}\isamarkupfalse%
\ \isacommand{show}\isamarkupfalse%
\ {\isacharquery}{\kern0pt}case\ \isanewline
\ \ \ \ \isacommand{using}\isamarkupfalse%
\ {\isadigit{2}}\ knights{\isacharunderscore}{\kern0pt}path{\isachardot}{\kern0pt}intros{\isacharparenleft}{\kern0pt}{\isadigit{2}}{\isacharparenright}{\kern0pt}{\isacharbrackleft}{\kern0pt}OF\ prems{\isacharbrackright}{\kern0pt}\ insert{\isacharunderscore}{\kern0pt}transpose{\isacharunderscore}{\kern0pt}board\ \isacommand{by}\isamarkupfalse%
\ auto\isanewline
\isacommand{qed}\isamarkupfalse%
%
\endisatagproof
{\isafoldproof}%
%
\isadelimproof
\isanewline
%
\endisadelimproof
\isanewline
\isacommand{corollary}\isamarkupfalse%
\ transpose{\isacharunderscore}{\kern0pt}knights{\isacharunderscore}{\kern0pt}path{\isacharcolon}{\kern0pt}\ \isanewline
\ \ \isakeyword{assumes}\ {\isachardoublequoteopen}knights{\isacharunderscore}{\kern0pt}path\ {\isacharparenleft}{\kern0pt}board\ n\ m{\isacharparenright}{\kern0pt}\ ps{\isachardoublequoteclose}\ \isanewline
\ \ \isakeyword{shows}\ {\isachardoublequoteopen}knights{\isacharunderscore}{\kern0pt}path\ {\isacharparenleft}{\kern0pt}board\ m\ n{\isacharparenright}{\kern0pt}\ {\isacharparenleft}{\kern0pt}transpose\ ps{\isacharparenright}{\kern0pt}{\isachardoublequoteclose}\isanewline
%
\isadelimproof
\ \ %
\endisadelimproof
%
\isatagproof
\isacommand{using}\isamarkupfalse%
\ assms\ transpose{\isacharunderscore}{\kern0pt}knights{\isacharunderscore}{\kern0pt}path{\isacharprime}{\kern0pt}{\isacharbrackleft}{\kern0pt}of\ {\isachardoublequoteopen}board\ n\ m{\isachardoublequoteclose}\ ps{\isacharbrackright}{\kern0pt}\ \isacommand{by}\isamarkupfalse%
\ {\isacharparenleft}{\kern0pt}auto\ simp{\isacharcolon}{\kern0pt}\ transpose{\isacharunderscore}{\kern0pt}board{\isacharparenright}{\kern0pt}%
\endisatagproof
{\isafoldproof}%
%
\isadelimproof
\ \isanewline
%
\endisadelimproof
\isanewline
\isacommand{corollary}\isamarkupfalse%
\ transpose{\isacharunderscore}{\kern0pt}knights{\isacharunderscore}{\kern0pt}circuit{\isacharcolon}{\kern0pt}\ \isanewline
\ \ \isakeyword{assumes}\ {\isachardoublequoteopen}knights{\isacharunderscore}{\kern0pt}circuit\ {\isacharparenleft}{\kern0pt}board\ n\ m{\isacharparenright}{\kern0pt}\ ps{\isachardoublequoteclose}\ \isanewline
\ \ \isakeyword{shows}\ {\isachardoublequoteopen}knights{\isacharunderscore}{\kern0pt}circuit\ {\isacharparenleft}{\kern0pt}board\ m\ n{\isacharparenright}{\kern0pt}\ {\isacharparenleft}{\kern0pt}transpose\ ps{\isacharparenright}{\kern0pt}{\isachardoublequoteclose}\isanewline
%
\isadelimproof
\ \ %
\endisadelimproof
%
\isatagproof
\isacommand{using}\isamarkupfalse%
\ assms\ \isanewline
\isacommand{proof}\isamarkupfalse%
\ {\isacharminus}{\kern0pt}\isanewline
\ \ \isacommand{have}\isamarkupfalse%
\ {\isachardoublequoteopen}knights{\isacharunderscore}{\kern0pt}path\ {\isacharparenleft}{\kern0pt}board\ n\ m{\isacharparenright}{\kern0pt}\ ps{\isachardoublequoteclose}\ \isakeyword{and}\ vs{\isacharcolon}{\kern0pt}\ {\isachardoublequoteopen}valid{\isacharunderscore}{\kern0pt}step\ {\isacharparenleft}{\kern0pt}last\ ps{\isacharparenright}{\kern0pt}\ {\isacharparenleft}{\kern0pt}hd\ ps{\isacharparenright}{\kern0pt}{\isachardoublequoteclose}\isanewline
\ \ \ \ \isacommand{using}\isamarkupfalse%
\ assms\ \isacommand{unfolding}\isamarkupfalse%
\ knights{\isacharunderscore}{\kern0pt}circuit{\isacharunderscore}{\kern0pt}def\ \isacommand{by}\isamarkupfalse%
\ auto\isanewline
\ \ \isacommand{then}\isamarkupfalse%
\ \isacommand{have}\isamarkupfalse%
\ kp{\isacharunderscore}{\kern0pt}t{\isacharcolon}{\kern0pt}\ {\isachardoublequoteopen}knights{\isacharunderscore}{\kern0pt}path\ {\isacharparenleft}{\kern0pt}board\ m\ n{\isacharparenright}{\kern0pt}\ {\isacharparenleft}{\kern0pt}transpose\ ps{\isacharparenright}{\kern0pt}{\isachardoublequoteclose}\ \isakeyword{and}\ {\isachardoublequoteopen}ps\ {\isasymnoteq}\ {\isacharbrackleft}{\kern0pt}{\isacharbrackright}{\kern0pt}{\isachardoublequoteclose}\isanewline
\ \ \ \ \isacommand{using}\isamarkupfalse%
\ transpose{\isacharunderscore}{\kern0pt}knights{\isacharunderscore}{\kern0pt}path\ knights{\isacharunderscore}{\kern0pt}path{\isacharunderscore}{\kern0pt}non{\isacharunderscore}{\kern0pt}nil\ \isacommand{by}\isamarkupfalse%
\ auto\isanewline
\ \ \isacommand{then}\isamarkupfalse%
\ \isacommand{have}\isamarkupfalse%
\ {\isachardoublequoteopen}valid{\isacharunderscore}{\kern0pt}step\ {\isacharparenleft}{\kern0pt}last\ {\isacharparenleft}{\kern0pt}transpose\ ps{\isacharparenright}{\kern0pt}{\isacharparenright}{\kern0pt}\ {\isacharparenleft}{\kern0pt}hd\ {\isacharparenleft}{\kern0pt}transpose\ ps{\isacharparenright}{\kern0pt}{\isacharparenright}{\kern0pt}{\isachardoublequoteclose}\isanewline
\ \ \ \ \isacommand{using}\isamarkupfalse%
\ vs\ hd{\isacharunderscore}{\kern0pt}transpose\ last{\isacharunderscore}{\kern0pt}transpose\ transpose{\isacharunderscore}{\kern0pt}valid{\isacharunderscore}{\kern0pt}step\ \isacommand{by}\isamarkupfalse%
\ auto\isanewline
\ \ \isacommand{then}\isamarkupfalse%
\ \isacommand{show}\isamarkupfalse%
\ {\isacharquery}{\kern0pt}thesis\ \isacommand{using}\isamarkupfalse%
\ kp{\isacharunderscore}{\kern0pt}t\ \isacommand{by}\isamarkupfalse%
\ {\isacharparenleft}{\kern0pt}auto\ simp{\isacharcolon}{\kern0pt}\ knights{\isacharunderscore}{\kern0pt}circuit{\isacharunderscore}{\kern0pt}def{\isacharparenright}{\kern0pt}\isanewline
\isacommand{qed}\isamarkupfalse%
%
\endisatagproof
{\isafoldproof}%
%
\isadelimproof
%
\endisadelimproof
%
\isadelimdocument
%
\endisadelimdocument
%
\isatagdocument
%
\isamarkupsection{Mirroring Paths and Boards%
}
\isamarkuptrue%
%
\isamarkupsubsection{Implementation of Path and Board Mirroring%
}
\isamarkuptrue%
%
\endisatagdocument
{\isafolddocument}%
%
\isadelimdocument
%
\endisadelimdocument
\isacommand{abbreviation}\isamarkupfalse%
\ {\isachardoublequoteopen}min{\isadigit{1}}\ ps\ {\isasymequiv}\ Min\ {\isacharparenleft}{\kern0pt}{\isacharparenleft}{\kern0pt}fst{\isacharparenright}{\kern0pt}\ {\isacharbackquote}{\kern0pt}\ set\ ps{\isacharparenright}{\kern0pt}{\isachardoublequoteclose}\isanewline
\isacommand{abbreviation}\isamarkupfalse%
\ {\isachardoublequoteopen}max{\isadigit{1}}\ ps\ {\isasymequiv}\ Max\ {\isacharparenleft}{\kern0pt}{\isacharparenleft}{\kern0pt}fst{\isacharparenright}{\kern0pt}\ {\isacharbackquote}{\kern0pt}\ set\ ps{\isacharparenright}{\kern0pt}{\isachardoublequoteclose}\isanewline
\isacommand{abbreviation}\isamarkupfalse%
\ {\isachardoublequoteopen}min{\isadigit{2}}\ ps\ {\isasymequiv}\ Min\ {\isacharparenleft}{\kern0pt}{\isacharparenleft}{\kern0pt}snd{\isacharparenright}{\kern0pt}\ {\isacharbackquote}{\kern0pt}\ set\ ps{\isacharparenright}{\kern0pt}{\isachardoublequoteclose}\isanewline
\isacommand{abbreviation}\isamarkupfalse%
\ {\isachardoublequoteopen}max{\isadigit{2}}\ ps\ {\isasymequiv}\ Max\ {\isacharparenleft}{\kern0pt}{\isacharparenleft}{\kern0pt}snd{\isacharparenright}{\kern0pt}\ {\isacharbackquote}{\kern0pt}\ set\ ps{\isacharparenright}{\kern0pt}{\isachardoublequoteclose}\isanewline
\isanewline
\isacommand{definition}\isamarkupfalse%
\ mirror{\isadigit{1}}{\isacharunderscore}{\kern0pt}square\ {\isacharcolon}{\kern0pt}{\isacharcolon}{\kern0pt}\ {\isachardoublequoteopen}int\ {\isasymRightarrow}\ square\ {\isasymRightarrow}\ square{\isachardoublequoteclose}\ \isakeyword{where}\ \isanewline
\ \ {\isachardoublequoteopen}mirror{\isadigit{1}}{\isacharunderscore}{\kern0pt}square\ n\ s\isactrlsub i\ {\isacharequal}{\kern0pt}\ {\isacharparenleft}{\kern0pt}case\ s\isactrlsub i\ of\ {\isacharparenleft}{\kern0pt}i{\isacharcomma}{\kern0pt}j{\isacharparenright}{\kern0pt}\ {\isasymRightarrow}\ {\isacharparenleft}{\kern0pt}n{\isacharminus}{\kern0pt}i{\isacharcomma}{\kern0pt}j{\isacharparenright}{\kern0pt}{\isacharparenright}{\kern0pt}{\isachardoublequoteclose}\isanewline
\isanewline
\isacommand{fun}\isamarkupfalse%
\ mirror{\isadigit{1}}{\isacharunderscore}{\kern0pt}aux\ {\isacharcolon}{\kern0pt}{\isacharcolon}{\kern0pt}\ {\isachardoublequoteopen}int\ {\isasymRightarrow}\ path\ {\isasymRightarrow}\ path{\isachardoublequoteclose}\ \isakeyword{where}\isanewline
\ \ {\isachardoublequoteopen}mirror{\isadigit{1}}{\isacharunderscore}{\kern0pt}aux\ n\ {\isacharbrackleft}{\kern0pt}{\isacharbrackright}{\kern0pt}\ {\isacharequal}{\kern0pt}\ {\isacharbrackleft}{\kern0pt}{\isacharbrackright}{\kern0pt}{\isachardoublequoteclose}\isanewline
{\isacharbar}{\kern0pt}\ {\isachardoublequoteopen}mirror{\isadigit{1}}{\isacharunderscore}{\kern0pt}aux\ n\ {\isacharparenleft}{\kern0pt}s\isactrlsub i{\isacharhash}{\kern0pt}ps{\isacharparenright}{\kern0pt}\ {\isacharequal}{\kern0pt}\ {\isacharparenleft}{\kern0pt}mirror{\isadigit{1}}{\isacharunderscore}{\kern0pt}square\ n\ s\isactrlsub i{\isacharparenright}{\kern0pt}{\isacharhash}{\kern0pt}mirror{\isadigit{1}}{\isacharunderscore}{\kern0pt}aux\ n\ ps{\isachardoublequoteclose}\isanewline
\isanewline
\isacommand{definition}\isamarkupfalse%
\ {\isachardoublequoteopen}mirror{\isadigit{1}}\ ps\ {\isacharequal}{\kern0pt}\ mirror{\isadigit{1}}{\isacharunderscore}{\kern0pt}aux\ {\isacharparenleft}{\kern0pt}max{\isadigit{1}}\ ps\ {\isacharplus}{\kern0pt}\ min{\isadigit{1}}\ ps{\isacharparenright}{\kern0pt}\ ps{\isachardoublequoteclose}\isanewline
\isanewline
\isacommand{definition}\isamarkupfalse%
\ mirror{\isadigit{1}}{\isacharunderscore}{\kern0pt}board\ {\isacharcolon}{\kern0pt}{\isacharcolon}{\kern0pt}\ {\isachardoublequoteopen}int\ {\isasymRightarrow}\ board\ {\isasymRightarrow}\ board{\isachardoublequoteclose}\ \isakeyword{where}\isanewline
\ \ {\isachardoublequoteopen}mirror{\isadigit{1}}{\isacharunderscore}{\kern0pt}board\ n\ b\ {\isasymequiv}\ {\isacharbraceleft}{\kern0pt}mirror{\isadigit{1}}{\isacharunderscore}{\kern0pt}square\ n\ s\isactrlsub i\ {\isacharbar}{\kern0pt}s\isactrlsub i{\isachardot}{\kern0pt}\ s\isactrlsub i\ {\isasymin}\ b{\isacharbraceright}{\kern0pt}{\isachardoublequoteclose}\isanewline
\isanewline
\isacommand{definition}\isamarkupfalse%
\ mirror{\isadigit{2}}{\isacharunderscore}{\kern0pt}square\ {\isacharcolon}{\kern0pt}{\isacharcolon}{\kern0pt}\ {\isachardoublequoteopen}int\ {\isasymRightarrow}\ square\ {\isasymRightarrow}\ square{\isachardoublequoteclose}\ \isakeyword{where}\ \isanewline
\ \ {\isachardoublequoteopen}mirror{\isadigit{2}}{\isacharunderscore}{\kern0pt}square\ m\ s\isactrlsub i\ {\isacharequal}{\kern0pt}\ {\isacharparenleft}{\kern0pt}case\ s\isactrlsub i\ of\ {\isacharparenleft}{\kern0pt}i{\isacharcomma}{\kern0pt}j{\isacharparenright}{\kern0pt}\ {\isasymRightarrow}\ {\isacharparenleft}{\kern0pt}i{\isacharcomma}{\kern0pt}m{\isacharminus}{\kern0pt}j{\isacharparenright}{\kern0pt}{\isacharparenright}{\kern0pt}{\isachardoublequoteclose}\isanewline
\isanewline
\isacommand{fun}\isamarkupfalse%
\ mirror{\isadigit{2}}{\isacharunderscore}{\kern0pt}aux\ {\isacharcolon}{\kern0pt}{\isacharcolon}{\kern0pt}\ {\isachardoublequoteopen}int\ {\isasymRightarrow}\ path\ {\isasymRightarrow}\ path{\isachardoublequoteclose}\ \isakeyword{where}\isanewline
\ \ {\isachardoublequoteopen}mirror{\isadigit{2}}{\isacharunderscore}{\kern0pt}aux\ m\ {\isacharbrackleft}{\kern0pt}{\isacharbrackright}{\kern0pt}\ {\isacharequal}{\kern0pt}\ {\isacharbrackleft}{\kern0pt}{\isacharbrackright}{\kern0pt}{\isachardoublequoteclose}\isanewline
{\isacharbar}{\kern0pt}\ {\isachardoublequoteopen}mirror{\isadigit{2}}{\isacharunderscore}{\kern0pt}aux\ m\ {\isacharparenleft}{\kern0pt}s\isactrlsub i{\isacharhash}{\kern0pt}ps{\isacharparenright}{\kern0pt}\ {\isacharequal}{\kern0pt}\ {\isacharparenleft}{\kern0pt}mirror{\isadigit{2}}{\isacharunderscore}{\kern0pt}square\ m\ s\isactrlsub i{\isacharparenright}{\kern0pt}{\isacharhash}{\kern0pt}mirror{\isadigit{2}}{\isacharunderscore}{\kern0pt}aux\ m\ ps{\isachardoublequoteclose}\isanewline
\isanewline
\isacommand{definition}\isamarkupfalse%
\ {\isachardoublequoteopen}mirror{\isadigit{2}}\ ps\ {\isacharequal}{\kern0pt}\ mirror{\isadigit{2}}{\isacharunderscore}{\kern0pt}aux\ {\isacharparenleft}{\kern0pt}max{\isadigit{2}}\ ps\ {\isacharplus}{\kern0pt}\ min{\isadigit{2}}\ ps{\isacharparenright}{\kern0pt}\ ps{\isachardoublequoteclose}\isanewline
\isanewline
\isacommand{definition}\isamarkupfalse%
\ mirror{\isadigit{2}}{\isacharunderscore}{\kern0pt}board\ {\isacharcolon}{\kern0pt}{\isacharcolon}{\kern0pt}\ {\isachardoublequoteopen}int\ {\isasymRightarrow}\ board\ {\isasymRightarrow}\ board{\isachardoublequoteclose}\ \isakeyword{where}\isanewline
\ \ {\isachardoublequoteopen}mirror{\isadigit{2}}{\isacharunderscore}{\kern0pt}board\ m\ b\ {\isasymequiv}\ {\isacharbraceleft}{\kern0pt}mirror{\isadigit{2}}{\isacharunderscore}{\kern0pt}square\ m\ s\isactrlsub i\ {\isacharbar}{\kern0pt}s\isactrlsub i{\isachardot}{\kern0pt}\ s\isactrlsub i\ {\isasymin}\ b{\isacharbraceright}{\kern0pt}{\isachardoublequoteclose}%
\isadelimdocument
%
\endisadelimdocument
%
\isatagdocument
%
\isamarkupsubsection{Correctness of Path and Board Mirroring%
}
\isamarkuptrue%
%
\endisatagdocument
{\isafolddocument}%
%
\isadelimdocument
%
\endisadelimdocument
\isacommand{lemma}\isamarkupfalse%
\ mirror{\isadigit{1}}{\isacharunderscore}{\kern0pt}board{\isacharunderscore}{\kern0pt}id{\isacharcolon}{\kern0pt}\ {\isachardoublequoteopen}mirror{\isadigit{1}}{\isacharunderscore}{\kern0pt}board\ {\isacharparenleft}{\kern0pt}int\ n{\isacharplus}{\kern0pt}{\isadigit{1}}{\isacharparenright}{\kern0pt}\ {\isacharparenleft}{\kern0pt}board\ n\ m{\isacharparenright}{\kern0pt}\ {\isacharequal}{\kern0pt}\ board\ n\ m{\isachardoublequoteclose}\ {\isacharparenleft}{\kern0pt}\isakeyword{is}\ {\isachardoublequoteopen}{\isacharunderscore}{\kern0pt}\ {\isacharequal}{\kern0pt}\ {\isacharquery}{\kern0pt}b{\isachardoublequoteclose}{\isacharparenright}{\kern0pt}\isanewline
%
\isadelimproof
%
\endisadelimproof
%
\isatagproof
\isacommand{proof}\isamarkupfalse%
\isanewline
\ \ \isacommand{show}\isamarkupfalse%
\ {\isachardoublequoteopen}mirror{\isadigit{1}}{\isacharunderscore}{\kern0pt}board\ {\isacharparenleft}{\kern0pt}int\ n{\isacharplus}{\kern0pt}{\isadigit{1}}{\isacharparenright}{\kern0pt}\ {\isacharquery}{\kern0pt}b\ {\isasymsubseteq}\ {\isacharquery}{\kern0pt}b{\isachardoublequoteclose}\isanewline
\ \ \isacommand{proof}\isamarkupfalse%
\isanewline
\ \ \ \ \isacommand{fix}\isamarkupfalse%
\ s\isactrlsub i{\isacharprime}{\kern0pt}\isanewline
\ \ \ \ \isacommand{assume}\isamarkupfalse%
\ assms{\isacharcolon}{\kern0pt}\ {\isachardoublequoteopen}s\isactrlsub i{\isacharprime}{\kern0pt}\ {\isasymin}\ mirror{\isadigit{1}}{\isacharunderscore}{\kern0pt}board\ {\isacharparenleft}{\kern0pt}int\ n{\isacharplus}{\kern0pt}{\isadigit{1}}{\isacharparenright}{\kern0pt}\ {\isacharquery}{\kern0pt}b{\isachardoublequoteclose}\isanewline
\ \ \ \ \isacommand{then}\isamarkupfalse%
\ \isacommand{obtain}\isamarkupfalse%
\ i{\isacharprime}{\kern0pt}\ j{\isacharprime}{\kern0pt}\ \isakeyword{where}\ {\isacharbrackleft}{\kern0pt}simp{\isacharbrackright}{\kern0pt}{\isacharcolon}{\kern0pt}\ {\isachardoublequoteopen}s\isactrlsub i{\isacharprime}{\kern0pt}\ {\isacharequal}{\kern0pt}\ {\isacharparenleft}{\kern0pt}i{\isacharprime}{\kern0pt}{\isacharcomma}{\kern0pt}j{\isacharprime}{\kern0pt}{\isacharparenright}{\kern0pt}{\isachardoublequoteclose}\ \isacommand{by}\isamarkupfalse%
\ force\isanewline
\ \ \ \ \isacommand{then}\isamarkupfalse%
\ \isacommand{have}\isamarkupfalse%
\ {\isachardoublequoteopen}{\isacharparenleft}{\kern0pt}i{\isacharprime}{\kern0pt}{\isacharcomma}{\kern0pt}j{\isacharprime}{\kern0pt}{\isacharparenright}{\kern0pt}\ {\isasymin}\ mirror{\isadigit{1}}{\isacharunderscore}{\kern0pt}board\ {\isacharparenleft}{\kern0pt}int\ n{\isacharplus}{\kern0pt}{\isadigit{1}}{\isacharparenright}{\kern0pt}\ {\isacharquery}{\kern0pt}b{\isachardoublequoteclose}\isanewline
\ \ \ \ \ \ \isacommand{using}\isamarkupfalse%
\ assms\ \isacommand{by}\isamarkupfalse%
\ auto\isanewline
\ \ \ \ \isacommand{then}\isamarkupfalse%
\ \isacommand{obtain}\isamarkupfalse%
\ i\ j\ \isakeyword{where}\ {\isachardoublequoteopen}{\isacharparenleft}{\kern0pt}i{\isacharcomma}{\kern0pt}j{\isacharparenright}{\kern0pt}\ {\isasymin}\ {\isacharquery}{\kern0pt}b{\isachardoublequoteclose}\ {\isachardoublequoteopen}mirror{\isadigit{1}}{\isacharunderscore}{\kern0pt}square\ {\isacharparenleft}{\kern0pt}int\ n{\isacharplus}{\kern0pt}{\isadigit{1}}{\isacharparenright}{\kern0pt}\ {\isacharparenleft}{\kern0pt}i{\isacharcomma}{\kern0pt}j{\isacharparenright}{\kern0pt}\ {\isacharequal}{\kern0pt}\ {\isacharparenleft}{\kern0pt}i{\isacharprime}{\kern0pt}{\isacharcomma}{\kern0pt}j{\isacharprime}{\kern0pt}{\isacharparenright}{\kern0pt}{\isachardoublequoteclose}\isanewline
\ \ \ \ \ \ \isacommand{unfolding}\isamarkupfalse%
\ mirror{\isadigit{1}}{\isacharunderscore}{\kern0pt}board{\isacharunderscore}{\kern0pt}def\ \isacommand{by}\isamarkupfalse%
\ auto\isanewline
\ \ \ \ \isacommand{then}\isamarkupfalse%
\ \isacommand{have}\isamarkupfalse%
\ {\isachardoublequoteopen}{\isadigit{1}}\ {\isasymle}\ i\ {\isasymand}\ i\ {\isasymle}\ int\ n{\isachardoublequoteclose}\ {\isachardoublequoteopen}{\isadigit{1}}\ {\isasymle}\ j\ {\isasymand}\ j\ {\isasymle}\ int\ m{\isachardoublequoteclose}\ {\isachardoublequoteopen}i{\isacharprime}{\kern0pt}{\isacharequal}{\kern0pt}{\isacharparenleft}{\kern0pt}int\ n{\isacharplus}{\kern0pt}{\isadigit{1}}{\isacharparenright}{\kern0pt}{\isacharminus}{\kern0pt}i{\isachardoublequoteclose}\ {\isachardoublequoteopen}j{\isacharprime}{\kern0pt}{\isacharequal}{\kern0pt}j{\isachardoublequoteclose}\isanewline
\ \ \ \ \ \ \isacommand{unfolding}\isamarkupfalse%
\ board{\isacharunderscore}{\kern0pt}def\ mirror{\isadigit{1}}{\isacharunderscore}{\kern0pt}square{\isacharunderscore}{\kern0pt}def\ \isacommand{by}\isamarkupfalse%
\ auto\isanewline
\ \ \ \ \isacommand{then}\isamarkupfalse%
\ \isacommand{have}\isamarkupfalse%
\ {\isachardoublequoteopen}{\isadigit{1}}\ {\isasymle}\ i{\isacharprime}{\kern0pt}\ {\isasymand}\ i{\isacharprime}{\kern0pt}\ {\isasymle}\ int\ n{\isachardoublequoteclose}\ {\isachardoublequoteopen}{\isadigit{1}}\ {\isasymle}\ j{\isacharprime}{\kern0pt}\ {\isasymand}\ j{\isacharprime}{\kern0pt}\ {\isasymle}\ int\ m{\isachardoublequoteclose}\isanewline
\ \ \ \ \ \ \isacommand{by}\isamarkupfalse%
\ auto\isanewline
\ \ \ \ \isacommand{then}\isamarkupfalse%
\ \isacommand{show}\isamarkupfalse%
\ {\isachardoublequoteopen}s\isactrlsub i{\isacharprime}{\kern0pt}\ {\isasymin}\ {\isacharquery}{\kern0pt}b{\isachardoublequoteclose}\isanewline
\ \ \ \ \ \ \isacommand{unfolding}\isamarkupfalse%
\ board{\isacharunderscore}{\kern0pt}def\ \isacommand{by}\isamarkupfalse%
\ auto\isanewline
\ \ \isacommand{qed}\isamarkupfalse%
\isanewline
\isacommand{next}\isamarkupfalse%
\isanewline
\ \ \isacommand{show}\isamarkupfalse%
\ {\isachardoublequoteopen}{\isacharquery}{\kern0pt}b\ {\isasymsubseteq}\ mirror{\isadigit{1}}{\isacharunderscore}{\kern0pt}board\ {\isacharparenleft}{\kern0pt}int\ n{\isacharplus}{\kern0pt}{\isadigit{1}}{\isacharparenright}{\kern0pt}\ {\isacharquery}{\kern0pt}b{\isachardoublequoteclose}\isanewline
\ \ \isacommand{proof}\isamarkupfalse%
\isanewline
\ \ \ \ \isacommand{fix}\isamarkupfalse%
\ s\isactrlsub i\isanewline
\ \ \ \ \isacommand{assume}\isamarkupfalse%
\ assms{\isacharcolon}{\kern0pt}\ {\isachardoublequoteopen}s\isactrlsub i\ {\isasymin}\ {\isacharquery}{\kern0pt}b{\isachardoublequoteclose}\isanewline
\ \ \ \ \isacommand{then}\isamarkupfalse%
\ \isacommand{obtain}\isamarkupfalse%
\ i\ j\ \isakeyword{where}\ {\isacharbrackleft}{\kern0pt}simp{\isacharbrackright}{\kern0pt}{\isacharcolon}{\kern0pt}\ {\isachardoublequoteopen}s\isactrlsub i\ {\isacharequal}{\kern0pt}\ {\isacharparenleft}{\kern0pt}i{\isacharcomma}{\kern0pt}j{\isacharparenright}{\kern0pt}{\isachardoublequoteclose}\ \isacommand{by}\isamarkupfalse%
\ force\isanewline
\ \ \ \ \isacommand{then}\isamarkupfalse%
\ \isacommand{have}\isamarkupfalse%
\ {\isachardoublequoteopen}{\isacharparenleft}{\kern0pt}i{\isacharcomma}{\kern0pt}j{\isacharparenright}{\kern0pt}\ {\isasymin}\ {\isacharquery}{\kern0pt}b{\isachardoublequoteclose}\isanewline
\ \ \ \ \ \ \isacommand{using}\isamarkupfalse%
\ assms\ \isacommand{by}\isamarkupfalse%
\ auto\isanewline
\ \ \ \ \isacommand{then}\isamarkupfalse%
\ \isacommand{have}\isamarkupfalse%
\ {\isachardoublequoteopen}{\isadigit{1}}\ {\isasymle}\ i\ {\isasymand}\ i\ {\isasymle}\ int\ n{\isachardoublequoteclose}\ {\isachardoublequoteopen}{\isadigit{1}}\ {\isasymle}\ j\ {\isasymand}\ j\ {\isasymle}\ int\ m{\isachardoublequoteclose}\isanewline
\ \ \ \ \ \ \isacommand{unfolding}\isamarkupfalse%
\ board{\isacharunderscore}{\kern0pt}def\ \isacommand{by}\isamarkupfalse%
\ auto\isanewline
\ \ \ \ \isacommand{then}\isamarkupfalse%
\ \isacommand{obtain}\isamarkupfalse%
\ i{\isacharprime}{\kern0pt}\ j{\isacharprime}{\kern0pt}\ \isakeyword{where}\ {\isachardoublequoteopen}i{\isacharprime}{\kern0pt}{\isacharequal}{\kern0pt}{\isacharparenleft}{\kern0pt}int\ n{\isacharplus}{\kern0pt}{\isadigit{1}}{\isacharparenright}{\kern0pt}{\isacharminus}{\kern0pt}i{\isachardoublequoteclose}\ {\isachardoublequoteopen}j{\isacharprime}{\kern0pt}{\isacharequal}{\kern0pt}j{\isachardoublequoteclose}\ \isacommand{by}\isamarkupfalse%
\ auto\isanewline
\ \ \ \ \isacommand{then}\isamarkupfalse%
\ \isacommand{have}\isamarkupfalse%
\ {\isachardoublequoteopen}{\isacharparenleft}{\kern0pt}i{\isacharprime}{\kern0pt}{\isacharcomma}{\kern0pt}j{\isacharprime}{\kern0pt}{\isacharparenright}{\kern0pt}\ {\isasymin}\ {\isacharquery}{\kern0pt}b{\isachardoublequoteclose}\ {\isachardoublequoteopen}mirror{\isadigit{1}}{\isacharunderscore}{\kern0pt}square\ {\isacharparenleft}{\kern0pt}int\ n{\isacharplus}{\kern0pt}{\isadigit{1}}{\isacharparenright}{\kern0pt}\ {\isacharparenleft}{\kern0pt}i{\isacharprime}{\kern0pt}{\isacharcomma}{\kern0pt}j{\isacharprime}{\kern0pt}{\isacharparenright}{\kern0pt}\ {\isacharequal}{\kern0pt}\ {\isacharparenleft}{\kern0pt}i{\isacharcomma}{\kern0pt}j{\isacharparenright}{\kern0pt}{\isachardoublequoteclose}\ \isanewline
\ \ \ \ \ \ \isacommand{using}\isamarkupfalse%
\ {\isacartoucheopen}{\isadigit{1}}\ {\isasymle}\ i\ {\isasymand}\ i\ {\isasymle}\ int\ n{\isacartoucheclose}\ {\isacartoucheopen}{\isadigit{1}}\ {\isasymle}\ j\ {\isasymand}\ j\ {\isasymle}\ int\ m{\isacartoucheclose}\ \isanewline
\ \ \ \ \ \ \isacommand{unfolding}\isamarkupfalse%
\ mirror{\isadigit{1}}{\isacharunderscore}{\kern0pt}square{\isacharunderscore}{\kern0pt}def\ \isacommand{by}\isamarkupfalse%
\ {\isacharparenleft}{\kern0pt}auto\ simp{\isacharcolon}{\kern0pt}\ board{\isacharunderscore}{\kern0pt}def{\isacharparenright}{\kern0pt}\isanewline
\ \ \ \ \isacommand{then}\isamarkupfalse%
\ \isacommand{show}\isamarkupfalse%
\ {\isachardoublequoteopen}s\isactrlsub i\ {\isasymin}\ mirror{\isadigit{1}}{\isacharunderscore}{\kern0pt}board\ {\isacharparenleft}{\kern0pt}int\ n{\isacharplus}{\kern0pt}{\isadigit{1}}{\isacharparenright}{\kern0pt}\ {\isacharquery}{\kern0pt}b{\isachardoublequoteclose}\isanewline
\ \ \ \ \ \ \isacommand{unfolding}\isamarkupfalse%
\ mirror{\isadigit{1}}{\isacharunderscore}{\kern0pt}board{\isacharunderscore}{\kern0pt}def\ \isacommand{by}\isamarkupfalse%
\ force\isanewline
\ \ \isacommand{qed}\isamarkupfalse%
\isanewline
\isacommand{qed}\isamarkupfalse%
%
\endisatagproof
{\isafoldproof}%
%
\isadelimproof
\isanewline
%
\endisadelimproof
\isanewline
\isacommand{lemma}\isamarkupfalse%
\ mirror{\isadigit{2}}{\isacharunderscore}{\kern0pt}board{\isacharunderscore}{\kern0pt}id{\isacharcolon}{\kern0pt}\ {\isachardoublequoteopen}mirror{\isadigit{2}}{\isacharunderscore}{\kern0pt}board\ {\isacharparenleft}{\kern0pt}int\ m{\isacharplus}{\kern0pt}{\isadigit{1}}{\isacharparenright}{\kern0pt}\ {\isacharparenleft}{\kern0pt}board\ n\ m{\isacharparenright}{\kern0pt}\ {\isacharequal}{\kern0pt}\ board\ n\ m{\isachardoublequoteclose}\ {\isacharparenleft}{\kern0pt}\isakeyword{is}\ {\isachardoublequoteopen}{\isacharunderscore}{\kern0pt}\ {\isacharequal}{\kern0pt}\ {\isacharquery}{\kern0pt}b{\isachardoublequoteclose}{\isacharparenright}{\kern0pt}\isanewline
%
\isadelimproof
%
\endisadelimproof
%
\isatagproof
\isacommand{proof}\isamarkupfalse%
\isanewline
\ \ \isacommand{show}\isamarkupfalse%
\ {\isachardoublequoteopen}mirror{\isadigit{2}}{\isacharunderscore}{\kern0pt}board\ {\isacharparenleft}{\kern0pt}int\ m{\isacharplus}{\kern0pt}{\isadigit{1}}{\isacharparenright}{\kern0pt}\ {\isacharquery}{\kern0pt}b\ {\isasymsubseteq}\ {\isacharquery}{\kern0pt}b{\isachardoublequoteclose}\isanewline
\ \ \isacommand{proof}\isamarkupfalse%
\isanewline
\ \ \ \ \isacommand{fix}\isamarkupfalse%
\ s\isactrlsub i{\isacharprime}{\kern0pt}\isanewline
\ \ \ \ \isacommand{assume}\isamarkupfalse%
\ assms{\isacharcolon}{\kern0pt}\ {\isachardoublequoteopen}s\isactrlsub i{\isacharprime}{\kern0pt}\ {\isasymin}\ mirror{\isadigit{2}}{\isacharunderscore}{\kern0pt}board\ {\isacharparenleft}{\kern0pt}int\ m{\isacharplus}{\kern0pt}{\isadigit{1}}{\isacharparenright}{\kern0pt}\ {\isacharquery}{\kern0pt}b{\isachardoublequoteclose}\isanewline
\ \ \ \ \isacommand{then}\isamarkupfalse%
\ \isacommand{obtain}\isamarkupfalse%
\ i{\isacharprime}{\kern0pt}\ j{\isacharprime}{\kern0pt}\ \isakeyword{where}\ {\isacharbrackleft}{\kern0pt}simp{\isacharbrackright}{\kern0pt}{\isacharcolon}{\kern0pt}\ {\isachardoublequoteopen}s\isactrlsub i{\isacharprime}{\kern0pt}\ {\isacharequal}{\kern0pt}\ {\isacharparenleft}{\kern0pt}i{\isacharprime}{\kern0pt}{\isacharcomma}{\kern0pt}j{\isacharprime}{\kern0pt}{\isacharparenright}{\kern0pt}{\isachardoublequoteclose}\ \isacommand{by}\isamarkupfalse%
\ force\isanewline
\ \ \ \ \isacommand{then}\isamarkupfalse%
\ \isacommand{have}\isamarkupfalse%
\ {\isachardoublequoteopen}{\isacharparenleft}{\kern0pt}i{\isacharprime}{\kern0pt}{\isacharcomma}{\kern0pt}j{\isacharprime}{\kern0pt}{\isacharparenright}{\kern0pt}\ {\isasymin}\ mirror{\isadigit{2}}{\isacharunderscore}{\kern0pt}board\ {\isacharparenleft}{\kern0pt}int\ m{\isacharplus}{\kern0pt}{\isadigit{1}}{\isacharparenright}{\kern0pt}\ {\isacharquery}{\kern0pt}b{\isachardoublequoteclose}\isanewline
\ \ \ \ \ \ \isacommand{using}\isamarkupfalse%
\ assms\ \isacommand{by}\isamarkupfalse%
\ auto\isanewline
\ \ \ \ \isacommand{then}\isamarkupfalse%
\ \isacommand{obtain}\isamarkupfalse%
\ i\ j\ \isakeyword{where}\ {\isachardoublequoteopen}{\isacharparenleft}{\kern0pt}i{\isacharcomma}{\kern0pt}j{\isacharparenright}{\kern0pt}\ {\isasymin}\ {\isacharquery}{\kern0pt}b{\isachardoublequoteclose}\ {\isachardoublequoteopen}mirror{\isadigit{2}}{\isacharunderscore}{\kern0pt}square\ {\isacharparenleft}{\kern0pt}int\ m{\isacharplus}{\kern0pt}{\isadigit{1}}{\isacharparenright}{\kern0pt}\ {\isacharparenleft}{\kern0pt}i{\isacharcomma}{\kern0pt}j{\isacharparenright}{\kern0pt}\ {\isacharequal}{\kern0pt}\ {\isacharparenleft}{\kern0pt}i{\isacharprime}{\kern0pt}{\isacharcomma}{\kern0pt}j{\isacharprime}{\kern0pt}{\isacharparenright}{\kern0pt}{\isachardoublequoteclose}\isanewline
\ \ \ \ \ \ \isacommand{unfolding}\isamarkupfalse%
\ mirror{\isadigit{2}}{\isacharunderscore}{\kern0pt}board{\isacharunderscore}{\kern0pt}def\ \isacommand{by}\isamarkupfalse%
\ auto\isanewline
\ \ \ \ \isacommand{then}\isamarkupfalse%
\ \isacommand{have}\isamarkupfalse%
\ {\isachardoublequoteopen}{\isadigit{1}}\ {\isasymle}\ i\ {\isasymand}\ i\ {\isasymle}\ int\ n{\isachardoublequoteclose}\ {\isachardoublequoteopen}{\isadigit{1}}\ {\isasymle}\ j\ {\isasymand}\ j\ {\isasymle}\ int\ m{\isachardoublequoteclose}\ {\isachardoublequoteopen}i{\isacharprime}{\kern0pt}{\isacharequal}{\kern0pt}i{\isachardoublequoteclose}\ {\isachardoublequoteopen}j{\isacharprime}{\kern0pt}{\isacharequal}{\kern0pt}{\isacharparenleft}{\kern0pt}int\ m{\isacharplus}{\kern0pt}{\isadigit{1}}{\isacharparenright}{\kern0pt}{\isacharminus}{\kern0pt}j{\isachardoublequoteclose}\isanewline
\ \ \ \ \ \ \isacommand{unfolding}\isamarkupfalse%
\ board{\isacharunderscore}{\kern0pt}def\ mirror{\isadigit{2}}{\isacharunderscore}{\kern0pt}square{\isacharunderscore}{\kern0pt}def\ \isacommand{by}\isamarkupfalse%
\ auto\isanewline
\ \ \ \ \isacommand{then}\isamarkupfalse%
\ \isacommand{have}\isamarkupfalse%
\ {\isachardoublequoteopen}{\isadigit{1}}\ {\isasymle}\ i{\isacharprime}{\kern0pt}\ {\isasymand}\ i{\isacharprime}{\kern0pt}\ {\isasymle}\ int\ n{\isachardoublequoteclose}\ {\isachardoublequoteopen}{\isadigit{1}}\ {\isasymle}\ j{\isacharprime}{\kern0pt}\ {\isasymand}\ j{\isacharprime}{\kern0pt}\ {\isasymle}\ int\ m{\isachardoublequoteclose}\isanewline
\ \ \ \ \ \ \isacommand{by}\isamarkupfalse%
\ auto\isanewline
\ \ \ \ \isacommand{then}\isamarkupfalse%
\ \isacommand{show}\isamarkupfalse%
\ {\isachardoublequoteopen}s\isactrlsub i{\isacharprime}{\kern0pt}\ {\isasymin}\ {\isacharquery}{\kern0pt}b{\isachardoublequoteclose}\isanewline
\ \ \ \ \ \ \isacommand{unfolding}\isamarkupfalse%
\ board{\isacharunderscore}{\kern0pt}def\ \isacommand{by}\isamarkupfalse%
\ auto\isanewline
\ \ \isacommand{qed}\isamarkupfalse%
\isanewline
\isacommand{next}\isamarkupfalse%
\isanewline
\ \ \isacommand{show}\isamarkupfalse%
\ {\isachardoublequoteopen}{\isacharquery}{\kern0pt}b\ {\isasymsubseteq}\ mirror{\isadigit{2}}{\isacharunderscore}{\kern0pt}board\ {\isacharparenleft}{\kern0pt}int\ m{\isacharplus}{\kern0pt}{\isadigit{1}}{\isacharparenright}{\kern0pt}\ {\isacharquery}{\kern0pt}b{\isachardoublequoteclose}\isanewline
\ \ \isacommand{proof}\isamarkupfalse%
\isanewline
\ \ \ \ \isacommand{fix}\isamarkupfalse%
\ s\isactrlsub i\isanewline
\ \ \ \ \isacommand{assume}\isamarkupfalse%
\ assms{\isacharcolon}{\kern0pt}\ {\isachardoublequoteopen}s\isactrlsub i\ {\isasymin}\ {\isacharquery}{\kern0pt}b{\isachardoublequoteclose}\isanewline
\ \ \ \ \isacommand{then}\isamarkupfalse%
\ \isacommand{obtain}\isamarkupfalse%
\ i\ j\ \isakeyword{where}\ {\isacharbrackleft}{\kern0pt}simp{\isacharbrackright}{\kern0pt}{\isacharcolon}{\kern0pt}\ {\isachardoublequoteopen}s\isactrlsub i\ {\isacharequal}{\kern0pt}\ {\isacharparenleft}{\kern0pt}i{\isacharcomma}{\kern0pt}j{\isacharparenright}{\kern0pt}{\isachardoublequoteclose}\ \isacommand{by}\isamarkupfalse%
\ force\isanewline
\ \ \ \ \isacommand{then}\isamarkupfalse%
\ \isacommand{have}\isamarkupfalse%
\ {\isachardoublequoteopen}{\isacharparenleft}{\kern0pt}i{\isacharcomma}{\kern0pt}j{\isacharparenright}{\kern0pt}\ {\isasymin}\ {\isacharquery}{\kern0pt}b{\isachardoublequoteclose}\isanewline
\ \ \ \ \ \ \isacommand{using}\isamarkupfalse%
\ assms\ \isacommand{by}\isamarkupfalse%
\ auto\isanewline
\ \ \ \ \isacommand{then}\isamarkupfalse%
\ \isacommand{have}\isamarkupfalse%
\ {\isachardoublequoteopen}{\isadigit{1}}\ {\isasymle}\ i\ {\isasymand}\ i\ {\isasymle}\ int\ n{\isachardoublequoteclose}\ {\isachardoublequoteopen}{\isadigit{1}}\ {\isasymle}\ j\ {\isasymand}\ j\ {\isasymle}\ int\ m{\isachardoublequoteclose}\isanewline
\ \ \ \ \ \ \isacommand{unfolding}\isamarkupfalse%
\ board{\isacharunderscore}{\kern0pt}def\ \isacommand{by}\isamarkupfalse%
\ auto\isanewline
\ \ \ \ \isacommand{then}\isamarkupfalse%
\ \isacommand{obtain}\isamarkupfalse%
\ i{\isacharprime}{\kern0pt}\ j{\isacharprime}{\kern0pt}\ \isakeyword{where}\ {\isachardoublequoteopen}i{\isacharprime}{\kern0pt}{\isacharequal}{\kern0pt}i{\isachardoublequoteclose}\ {\isachardoublequoteopen}j{\isacharprime}{\kern0pt}{\isacharequal}{\kern0pt}{\isacharparenleft}{\kern0pt}int\ m{\isacharplus}{\kern0pt}{\isadigit{1}}{\isacharparenright}{\kern0pt}{\isacharminus}{\kern0pt}j{\isachardoublequoteclose}\ \isacommand{by}\isamarkupfalse%
\ auto\isanewline
\ \ \ \ \isacommand{then}\isamarkupfalse%
\ \isacommand{have}\isamarkupfalse%
\ {\isachardoublequoteopen}{\isacharparenleft}{\kern0pt}i{\isacharprime}{\kern0pt}{\isacharcomma}{\kern0pt}j{\isacharprime}{\kern0pt}{\isacharparenright}{\kern0pt}\ {\isasymin}\ {\isacharquery}{\kern0pt}b{\isachardoublequoteclose}\ {\isachardoublequoteopen}mirror{\isadigit{2}}{\isacharunderscore}{\kern0pt}square\ {\isacharparenleft}{\kern0pt}int\ m{\isacharplus}{\kern0pt}{\isadigit{1}}{\isacharparenright}{\kern0pt}\ {\isacharparenleft}{\kern0pt}i{\isacharprime}{\kern0pt}{\isacharcomma}{\kern0pt}j{\isacharprime}{\kern0pt}{\isacharparenright}{\kern0pt}\ {\isacharequal}{\kern0pt}\ {\isacharparenleft}{\kern0pt}i{\isacharcomma}{\kern0pt}j{\isacharparenright}{\kern0pt}{\isachardoublequoteclose}\ \isanewline
\ \ \ \ \ \ \isacommand{using}\isamarkupfalse%
\ {\isacartoucheopen}{\isadigit{1}}\ {\isasymle}\ i\ {\isasymand}\ i\ {\isasymle}\ int\ n{\isacartoucheclose}\ {\isacartoucheopen}{\isadigit{1}}\ {\isasymle}\ j\ {\isasymand}\ j\ {\isasymle}\ int\ m{\isacartoucheclose}\ \isanewline
\ \ \ \ \ \ \isacommand{unfolding}\isamarkupfalse%
\ mirror{\isadigit{2}}{\isacharunderscore}{\kern0pt}square{\isacharunderscore}{\kern0pt}def\ \isacommand{by}\isamarkupfalse%
\ {\isacharparenleft}{\kern0pt}auto\ simp{\isacharcolon}{\kern0pt}\ board{\isacharunderscore}{\kern0pt}def{\isacharparenright}{\kern0pt}\isanewline
\ \ \ \ \isacommand{then}\isamarkupfalse%
\ \isacommand{show}\isamarkupfalse%
\ {\isachardoublequoteopen}s\isactrlsub i\ {\isasymin}\ mirror{\isadigit{2}}{\isacharunderscore}{\kern0pt}board\ {\isacharparenleft}{\kern0pt}int\ m{\isacharplus}{\kern0pt}{\isadigit{1}}{\isacharparenright}{\kern0pt}\ {\isacharquery}{\kern0pt}b{\isachardoublequoteclose}\isanewline
\ \ \ \ \ \ \isacommand{unfolding}\isamarkupfalse%
\ mirror{\isadigit{2}}{\isacharunderscore}{\kern0pt}board{\isacharunderscore}{\kern0pt}def\ \isacommand{by}\isamarkupfalse%
\ force\isanewline
\ \ \isacommand{qed}\isamarkupfalse%
\isanewline
\isacommand{qed}\isamarkupfalse%
%
\endisatagproof
{\isafoldproof}%
%
\isadelimproof
\isanewline
%
\endisadelimproof
\isanewline
\isacommand{lemma}\isamarkupfalse%
\ knights{\isacharunderscore}{\kern0pt}path{\isacharunderscore}{\kern0pt}min{\isadigit{1}}{\isacharcolon}{\kern0pt}\ {\isachardoublequoteopen}knights{\isacharunderscore}{\kern0pt}path\ {\isacharparenleft}{\kern0pt}board\ n\ m{\isacharparenright}{\kern0pt}\ ps\ {\isasymLongrightarrow}\ min{\isadigit{1}}\ ps\ {\isacharequal}{\kern0pt}\ {\isadigit{1}}{\isachardoublequoteclose}\isanewline
%
\isadelimproof
%
\endisadelimproof
%
\isatagproof
\isacommand{proof}\isamarkupfalse%
\ {\isacharminus}{\kern0pt}\isanewline
\ \ \isacommand{assume}\isamarkupfalse%
\ assms{\isacharcolon}{\kern0pt}\ {\isachardoublequoteopen}knights{\isacharunderscore}{\kern0pt}path\ {\isacharparenleft}{\kern0pt}board\ n\ m{\isacharparenright}{\kern0pt}\ ps{\isachardoublequoteclose}\isanewline
\ \ \isacommand{then}\isamarkupfalse%
\ \isacommand{have}\isamarkupfalse%
\ {\isachardoublequoteopen}min\ n\ m\ {\isasymge}\ {\isadigit{1}}{\isachardoublequoteclose}\isanewline
\ \ \ \ \isacommand{using}\isamarkupfalse%
\ knights{\isacharunderscore}{\kern0pt}path{\isacharunderscore}{\kern0pt}board{\isacharunderscore}{\kern0pt}m{\isacharunderscore}{\kern0pt}n{\isacharunderscore}{\kern0pt}geq{\isacharunderscore}{\kern0pt}{\isadigit{1}}\ \isacommand{by}\isamarkupfalse%
\ auto\isanewline
\ \ \isacommand{then}\isamarkupfalse%
\ \isacommand{have}\isamarkupfalse%
\ {\isachardoublequoteopen}{\isacharparenleft}{\kern0pt}{\isadigit{1}}{\isacharcomma}{\kern0pt}{\isadigit{1}}{\isacharparenright}{\kern0pt}\ {\isasymin}\ board\ n\ m{\isachardoublequoteclose}\ \isakeyword{and}\ ge{\isacharunderscore}{\kern0pt}{\isadigit{1}}{\isacharcolon}{\kern0pt}\ {\isachardoublequoteopen}{\isasymforall}{\isacharparenleft}{\kern0pt}i{\isacharcomma}{\kern0pt}j{\isacharparenright}{\kern0pt}\ {\isasymin}\ board\ n\ m{\isachardot}{\kern0pt}\ i\ {\isasymge}\ {\isadigit{1}}{\isachardoublequoteclose}\isanewline
\ \ \ \ \isacommand{unfolding}\isamarkupfalse%
\ board{\isacharunderscore}{\kern0pt}def\ \isacommand{by}\isamarkupfalse%
\ auto\isanewline
\ \ \isacommand{then}\isamarkupfalse%
\ \isacommand{have}\isamarkupfalse%
\ finite{\isacharcolon}{\kern0pt}\ {\isachardoublequoteopen}finite\ {\isacharparenleft}{\kern0pt}{\isacharparenleft}{\kern0pt}fst{\isacharparenright}{\kern0pt}\ {\isacharbackquote}{\kern0pt}\ board\ n\ m{\isacharparenright}{\kern0pt}{\isachardoublequoteclose}\ \isakeyword{and}\ \isanewline
\ \ \ \ \ \ \ \ \ \ non{\isacharunderscore}{\kern0pt}empty{\isacharcolon}{\kern0pt}\ {\isachardoublequoteopen}{\isacharparenleft}{\kern0pt}fst{\isacharparenright}{\kern0pt}\ {\isacharbackquote}{\kern0pt}\ board\ n\ m\ {\isasymnoteq}\ {\isacharbraceleft}{\kern0pt}{\isacharbraceright}{\kern0pt}{\isachardoublequoteclose}\ \isakeyword{and}\isanewline
\ \ \ \ \ \ \ \ \ \ mem{\isacharunderscore}{\kern0pt}{\isadigit{1}}{\isacharcolon}{\kern0pt}\ {\isachardoublequoteopen}{\isadigit{1}}\ {\isasymin}\ {\isacharparenleft}{\kern0pt}fst{\isacharparenright}{\kern0pt}\ {\isacharbackquote}{\kern0pt}\ board\ n\ m{\isachardoublequoteclose}\isanewline
\ \ \ \ \isacommand{using}\isamarkupfalse%
\ board{\isacharunderscore}{\kern0pt}finite\ \isacommand{by}\isamarkupfalse%
\ auto\ {\isacharparenleft}{\kern0pt}metis\ fstI\ image{\isacharunderscore}{\kern0pt}eqI{\isacharparenright}{\kern0pt}\isanewline
\ \ \isacommand{then}\isamarkupfalse%
\ \isacommand{have}\isamarkupfalse%
\ {\isachardoublequoteopen}Min\ {\isacharparenleft}{\kern0pt}{\isacharparenleft}{\kern0pt}fst{\isacharparenright}{\kern0pt}\ {\isacharbackquote}{\kern0pt}\ board\ n\ m{\isacharparenright}{\kern0pt}\ {\isacharequal}{\kern0pt}\ {\isadigit{1}}{\isachardoublequoteclose}\isanewline
\ \ \ \ \isacommand{using}\isamarkupfalse%
\ ge{\isacharunderscore}{\kern0pt}{\isadigit{1}}\ \isacommand{by}\isamarkupfalse%
\ {\isacharparenleft}{\kern0pt}auto\ simp{\isacharcolon}{\kern0pt}\ Min{\isacharunderscore}{\kern0pt}eq{\isacharunderscore}{\kern0pt}iff{\isacharparenright}{\kern0pt}\isanewline
\ \ \isacommand{then}\isamarkupfalse%
\ \isacommand{show}\isamarkupfalse%
\ {\isacharquery}{\kern0pt}thesis\isanewline
\ \ \ \ \isacommand{using}\isamarkupfalse%
\ assms\ knights{\isacharunderscore}{\kern0pt}path{\isacharunderscore}{\kern0pt}set{\isacharunderscore}{\kern0pt}eq\ \isacommand{by}\isamarkupfalse%
\ auto\isanewline
\isacommand{qed}\isamarkupfalse%
%
\endisatagproof
{\isafoldproof}%
%
\isadelimproof
\isanewline
%
\endisadelimproof
\isanewline
\isacommand{lemma}\isamarkupfalse%
\ knights{\isacharunderscore}{\kern0pt}path{\isacharunderscore}{\kern0pt}min{\isadigit{2}}{\isacharcolon}{\kern0pt}\ {\isachardoublequoteopen}knights{\isacharunderscore}{\kern0pt}path\ {\isacharparenleft}{\kern0pt}board\ n\ m{\isacharparenright}{\kern0pt}\ ps\ {\isasymLongrightarrow}\ min{\isadigit{2}}\ ps\ {\isacharequal}{\kern0pt}\ {\isadigit{1}}{\isachardoublequoteclose}\isanewline
%
\isadelimproof
%
\endisadelimproof
%
\isatagproof
\isacommand{proof}\isamarkupfalse%
\ {\isacharminus}{\kern0pt}\isanewline
\ \ \isacommand{assume}\isamarkupfalse%
\ assms{\isacharcolon}{\kern0pt}\ {\isachardoublequoteopen}knights{\isacharunderscore}{\kern0pt}path\ {\isacharparenleft}{\kern0pt}board\ n\ m{\isacharparenright}{\kern0pt}\ ps{\isachardoublequoteclose}\isanewline
\ \ \isacommand{then}\isamarkupfalse%
\ \isacommand{have}\isamarkupfalse%
\ {\isachardoublequoteopen}min\ n\ m\ {\isasymge}\ {\isadigit{1}}{\isachardoublequoteclose}\isanewline
\ \ \ \ \isacommand{using}\isamarkupfalse%
\ knights{\isacharunderscore}{\kern0pt}path{\isacharunderscore}{\kern0pt}board{\isacharunderscore}{\kern0pt}m{\isacharunderscore}{\kern0pt}n{\isacharunderscore}{\kern0pt}geq{\isacharunderscore}{\kern0pt}{\isadigit{1}}\ \isacommand{by}\isamarkupfalse%
\ auto\isanewline
\ \ \isacommand{then}\isamarkupfalse%
\ \isacommand{have}\isamarkupfalse%
\ {\isachardoublequoteopen}{\isacharparenleft}{\kern0pt}{\isadigit{1}}{\isacharcomma}{\kern0pt}{\isadigit{1}}{\isacharparenright}{\kern0pt}\ {\isasymin}\ board\ n\ m{\isachardoublequoteclose}\ \isakeyword{and}\ ge{\isacharunderscore}{\kern0pt}{\isadigit{1}}{\isacharcolon}{\kern0pt}\ {\isachardoublequoteopen}{\isasymforall}{\isacharparenleft}{\kern0pt}i{\isacharcomma}{\kern0pt}j{\isacharparenright}{\kern0pt}\ {\isasymin}\ board\ n\ m{\isachardot}{\kern0pt}\ j\ {\isasymge}\ {\isadigit{1}}{\isachardoublequoteclose}\isanewline
\ \ \ \ \isacommand{unfolding}\isamarkupfalse%
\ board{\isacharunderscore}{\kern0pt}def\ \isacommand{by}\isamarkupfalse%
\ auto\isanewline
\ \ \isacommand{then}\isamarkupfalse%
\ \isacommand{have}\isamarkupfalse%
\ finite{\isacharcolon}{\kern0pt}\ {\isachardoublequoteopen}finite\ {\isacharparenleft}{\kern0pt}{\isacharparenleft}{\kern0pt}snd{\isacharparenright}{\kern0pt}\ {\isacharbackquote}{\kern0pt}\ board\ n\ m{\isacharparenright}{\kern0pt}{\isachardoublequoteclose}\ \isakeyword{and}\ \isanewline
\ \ \ \ \ \ \ \ \ \ non{\isacharunderscore}{\kern0pt}empty{\isacharcolon}{\kern0pt}\ {\isachardoublequoteopen}{\isacharparenleft}{\kern0pt}snd{\isacharparenright}{\kern0pt}\ {\isacharbackquote}{\kern0pt}\ board\ n\ m\ {\isasymnoteq}\ {\isacharbraceleft}{\kern0pt}{\isacharbraceright}{\kern0pt}{\isachardoublequoteclose}\ \isakeyword{and}\isanewline
\ \ \ \ \ \ \ \ \ \ mem{\isacharunderscore}{\kern0pt}{\isadigit{1}}{\isacharcolon}{\kern0pt}\ {\isachardoublequoteopen}{\isadigit{1}}\ {\isasymin}\ {\isacharparenleft}{\kern0pt}snd{\isacharparenright}{\kern0pt}\ {\isacharbackquote}{\kern0pt}\ board\ n\ m{\isachardoublequoteclose}\isanewline
\ \ \ \ \isacommand{using}\isamarkupfalse%
\ board{\isacharunderscore}{\kern0pt}finite\ \isacommand{by}\isamarkupfalse%
\ auto\ {\isacharparenleft}{\kern0pt}metis\ sndI\ image{\isacharunderscore}{\kern0pt}eqI{\isacharparenright}{\kern0pt}\isanewline
\ \ \isacommand{then}\isamarkupfalse%
\ \isacommand{have}\isamarkupfalse%
\ {\isachardoublequoteopen}Min\ {\isacharparenleft}{\kern0pt}{\isacharparenleft}{\kern0pt}snd{\isacharparenright}{\kern0pt}\ {\isacharbackquote}{\kern0pt}\ board\ n\ m{\isacharparenright}{\kern0pt}\ {\isacharequal}{\kern0pt}\ {\isadigit{1}}{\isachardoublequoteclose}\isanewline
\ \ \ \ \isacommand{using}\isamarkupfalse%
\ ge{\isacharunderscore}{\kern0pt}{\isadigit{1}}\ \isacommand{by}\isamarkupfalse%
\ {\isacharparenleft}{\kern0pt}auto\ simp{\isacharcolon}{\kern0pt}\ Min{\isacharunderscore}{\kern0pt}eq{\isacharunderscore}{\kern0pt}iff{\isacharparenright}{\kern0pt}\isanewline
\ \ \isacommand{then}\isamarkupfalse%
\ \isacommand{show}\isamarkupfalse%
\ {\isacharquery}{\kern0pt}thesis\isanewline
\ \ \ \ \isacommand{using}\isamarkupfalse%
\ assms\ knights{\isacharunderscore}{\kern0pt}path{\isacharunderscore}{\kern0pt}set{\isacharunderscore}{\kern0pt}eq\ \isacommand{by}\isamarkupfalse%
\ auto\isanewline
\isacommand{qed}\isamarkupfalse%
%
\endisatagproof
{\isafoldproof}%
%
\isadelimproof
\isanewline
%
\endisadelimproof
\isanewline
\isacommand{lemma}\isamarkupfalse%
\ knights{\isacharunderscore}{\kern0pt}path{\isacharunderscore}{\kern0pt}max{\isadigit{1}}{\isacharcolon}{\kern0pt}\ {\isachardoublequoteopen}knights{\isacharunderscore}{\kern0pt}path\ {\isacharparenleft}{\kern0pt}board\ n\ m{\isacharparenright}{\kern0pt}\ ps\ {\isasymLongrightarrow}\ max{\isadigit{1}}\ ps\ {\isacharequal}{\kern0pt}\ int\ n{\isachardoublequoteclose}\isanewline
%
\isadelimproof
%
\endisadelimproof
%
\isatagproof
\isacommand{proof}\isamarkupfalse%
\ {\isacharminus}{\kern0pt}\isanewline
\ \ \isacommand{assume}\isamarkupfalse%
\ assms{\isacharcolon}{\kern0pt}\ {\isachardoublequoteopen}knights{\isacharunderscore}{\kern0pt}path\ {\isacharparenleft}{\kern0pt}board\ n\ m{\isacharparenright}{\kern0pt}\ ps{\isachardoublequoteclose}\isanewline
\ \ \isacommand{then}\isamarkupfalse%
\ \isacommand{have}\isamarkupfalse%
\ {\isachardoublequoteopen}min\ n\ m\ {\isasymge}\ {\isadigit{1}}{\isachardoublequoteclose}\isanewline
\ \ \ \ \isacommand{using}\isamarkupfalse%
\ knights{\isacharunderscore}{\kern0pt}path{\isacharunderscore}{\kern0pt}board{\isacharunderscore}{\kern0pt}m{\isacharunderscore}{\kern0pt}n{\isacharunderscore}{\kern0pt}geq{\isacharunderscore}{\kern0pt}{\isadigit{1}}\ \isacommand{by}\isamarkupfalse%
\ auto\isanewline
\ \ \isacommand{then}\isamarkupfalse%
\ \isacommand{have}\isamarkupfalse%
\ {\isachardoublequoteopen}{\isacharparenleft}{\kern0pt}int\ n{\isacharcomma}{\kern0pt}{\isadigit{1}}{\isacharparenright}{\kern0pt}\ {\isasymin}\ board\ n\ m{\isachardoublequoteclose}\ \isakeyword{and}\ leq{\isacharunderscore}{\kern0pt}n{\isacharcolon}{\kern0pt}\ {\isachardoublequoteopen}{\isasymforall}{\isacharparenleft}{\kern0pt}i{\isacharcomma}{\kern0pt}j{\isacharparenright}{\kern0pt}\ {\isasymin}\ board\ n\ m{\isachardot}{\kern0pt}\ i\ {\isasymle}\ int\ n{\isachardoublequoteclose}\isanewline
\ \ \ \ \isacommand{unfolding}\isamarkupfalse%
\ board{\isacharunderscore}{\kern0pt}def\ \isacommand{by}\isamarkupfalse%
\ auto\isanewline
\ \ \isacommand{then}\isamarkupfalse%
\ \isacommand{have}\isamarkupfalse%
\ finite{\isacharcolon}{\kern0pt}\ {\isachardoublequoteopen}finite\ {\isacharparenleft}{\kern0pt}{\isacharparenleft}{\kern0pt}fst{\isacharparenright}{\kern0pt}\ {\isacharbackquote}{\kern0pt}\ board\ n\ m{\isacharparenright}{\kern0pt}{\isachardoublequoteclose}\ \isakeyword{and}\ \isanewline
\ \ \ \ \ \ \ \ \ \ non{\isacharunderscore}{\kern0pt}empty{\isacharcolon}{\kern0pt}\ {\isachardoublequoteopen}{\isacharparenleft}{\kern0pt}fst{\isacharparenright}{\kern0pt}\ {\isacharbackquote}{\kern0pt}\ board\ n\ m\ {\isasymnoteq}\ {\isacharbraceleft}{\kern0pt}{\isacharbraceright}{\kern0pt}{\isachardoublequoteclose}\ \isakeyword{and}\isanewline
\ \ \ \ \ \ \ \ \ \ mem{\isacharunderscore}{\kern0pt}{\isadigit{1}}{\isacharcolon}{\kern0pt}\ {\isachardoublequoteopen}int\ n\ {\isasymin}\ {\isacharparenleft}{\kern0pt}fst{\isacharparenright}{\kern0pt}\ {\isacharbackquote}{\kern0pt}\ board\ n\ m{\isachardoublequoteclose}\isanewline
\ \ \ \ \isacommand{using}\isamarkupfalse%
\ board{\isacharunderscore}{\kern0pt}finite\ \isacommand{by}\isamarkupfalse%
\ auto\ {\isacharparenleft}{\kern0pt}metis\ fstI\ image{\isacharunderscore}{\kern0pt}eqI{\isacharparenright}{\kern0pt}\isanewline
\ \ \isacommand{then}\isamarkupfalse%
\ \isacommand{have}\isamarkupfalse%
\ {\isachardoublequoteopen}Max\ {\isacharparenleft}{\kern0pt}{\isacharparenleft}{\kern0pt}fst{\isacharparenright}{\kern0pt}\ {\isacharbackquote}{\kern0pt}\ board\ n\ m{\isacharparenright}{\kern0pt}\ {\isacharequal}{\kern0pt}\ int\ n{\isachardoublequoteclose}\isanewline
\ \ \ \ \isacommand{using}\isamarkupfalse%
\ leq{\isacharunderscore}{\kern0pt}n\ \isacommand{by}\isamarkupfalse%
\ {\isacharparenleft}{\kern0pt}auto\ simp{\isacharcolon}{\kern0pt}\ Max{\isacharunderscore}{\kern0pt}eq{\isacharunderscore}{\kern0pt}iff{\isacharparenright}{\kern0pt}\isanewline
\ \ \isacommand{then}\isamarkupfalse%
\ \isacommand{show}\isamarkupfalse%
\ {\isacharquery}{\kern0pt}thesis\isanewline
\ \ \ \ \isacommand{using}\isamarkupfalse%
\ assms\ knights{\isacharunderscore}{\kern0pt}path{\isacharunderscore}{\kern0pt}set{\isacharunderscore}{\kern0pt}eq\ \isacommand{by}\isamarkupfalse%
\ auto\isanewline
\isacommand{qed}\isamarkupfalse%
%
\endisatagproof
{\isafoldproof}%
%
\isadelimproof
\isanewline
%
\endisadelimproof
\isanewline
\isacommand{lemma}\isamarkupfalse%
\ knights{\isacharunderscore}{\kern0pt}path{\isacharunderscore}{\kern0pt}max{\isadigit{2}}{\isacharcolon}{\kern0pt}\ {\isachardoublequoteopen}knights{\isacharunderscore}{\kern0pt}path\ {\isacharparenleft}{\kern0pt}board\ n\ m{\isacharparenright}{\kern0pt}\ ps\ {\isasymLongrightarrow}\ max{\isadigit{2}}\ ps\ {\isacharequal}{\kern0pt}\ int\ m{\isachardoublequoteclose}\isanewline
%
\isadelimproof
%
\endisadelimproof
%
\isatagproof
\isacommand{proof}\isamarkupfalse%
\ {\isacharminus}{\kern0pt}\isanewline
\ \ \isacommand{assume}\isamarkupfalse%
\ assms{\isacharcolon}{\kern0pt}\ {\isachardoublequoteopen}knights{\isacharunderscore}{\kern0pt}path\ {\isacharparenleft}{\kern0pt}board\ n\ m{\isacharparenright}{\kern0pt}\ ps{\isachardoublequoteclose}\isanewline
\ \ \isacommand{then}\isamarkupfalse%
\ \isacommand{have}\isamarkupfalse%
\ {\isachardoublequoteopen}min\ n\ m\ {\isasymge}\ {\isadigit{1}}{\isachardoublequoteclose}\isanewline
\ \ \ \ \isacommand{using}\isamarkupfalse%
\ knights{\isacharunderscore}{\kern0pt}path{\isacharunderscore}{\kern0pt}board{\isacharunderscore}{\kern0pt}m{\isacharunderscore}{\kern0pt}n{\isacharunderscore}{\kern0pt}geq{\isacharunderscore}{\kern0pt}{\isadigit{1}}\ \isacommand{by}\isamarkupfalse%
\ auto\isanewline
\ \ \isacommand{then}\isamarkupfalse%
\ \isacommand{have}\isamarkupfalse%
\ {\isachardoublequoteopen}{\isacharparenleft}{\kern0pt}{\isadigit{1}}{\isacharcomma}{\kern0pt}int\ m{\isacharparenright}{\kern0pt}\ {\isasymin}\ board\ n\ m{\isachardoublequoteclose}\ \isakeyword{and}\ leq{\isacharunderscore}{\kern0pt}m{\isacharcolon}{\kern0pt}\ {\isachardoublequoteopen}{\isasymforall}{\isacharparenleft}{\kern0pt}i{\isacharcomma}{\kern0pt}j{\isacharparenright}{\kern0pt}\ {\isasymin}\ board\ n\ m{\isachardot}{\kern0pt}\ j\ {\isasymle}\ int\ m{\isachardoublequoteclose}\isanewline
\ \ \ \ \isacommand{unfolding}\isamarkupfalse%
\ board{\isacharunderscore}{\kern0pt}def\ \isacommand{by}\isamarkupfalse%
\ auto\isanewline
\ \ \isacommand{then}\isamarkupfalse%
\ \isacommand{have}\isamarkupfalse%
\ finite{\isacharcolon}{\kern0pt}\ {\isachardoublequoteopen}finite\ {\isacharparenleft}{\kern0pt}{\isacharparenleft}{\kern0pt}snd{\isacharparenright}{\kern0pt}\ {\isacharbackquote}{\kern0pt}\ board\ n\ m{\isacharparenright}{\kern0pt}{\isachardoublequoteclose}\ \isakeyword{and}\ \isanewline
\ \ \ \ \ \ \ \ \ \ non{\isacharunderscore}{\kern0pt}empty{\isacharcolon}{\kern0pt}\ {\isachardoublequoteopen}{\isacharparenleft}{\kern0pt}snd{\isacharparenright}{\kern0pt}\ {\isacharbackquote}{\kern0pt}\ board\ n\ m\ {\isasymnoteq}\ {\isacharbraceleft}{\kern0pt}{\isacharbraceright}{\kern0pt}{\isachardoublequoteclose}\ \isakeyword{and}\isanewline
\ \ \ \ \ \ \ \ \ \ mem{\isacharunderscore}{\kern0pt}{\isadigit{1}}{\isacharcolon}{\kern0pt}\ {\isachardoublequoteopen}int\ m\ {\isasymin}\ {\isacharparenleft}{\kern0pt}snd{\isacharparenright}{\kern0pt}\ {\isacharbackquote}{\kern0pt}\ board\ n\ m{\isachardoublequoteclose}\isanewline
\ \ \ \ \isacommand{using}\isamarkupfalse%
\ board{\isacharunderscore}{\kern0pt}finite\ \isacommand{by}\isamarkupfalse%
\ auto\ {\isacharparenleft}{\kern0pt}metis\ sndI\ image{\isacharunderscore}{\kern0pt}eqI{\isacharparenright}{\kern0pt}\isanewline
\ \ \isacommand{then}\isamarkupfalse%
\ \isacommand{have}\isamarkupfalse%
\ {\isachardoublequoteopen}Max\ {\isacharparenleft}{\kern0pt}{\isacharparenleft}{\kern0pt}snd{\isacharparenright}{\kern0pt}\ {\isacharbackquote}{\kern0pt}\ board\ n\ m{\isacharparenright}{\kern0pt}\ {\isacharequal}{\kern0pt}\ int\ m{\isachardoublequoteclose}\isanewline
\ \ \ \ \isacommand{using}\isamarkupfalse%
\ leq{\isacharunderscore}{\kern0pt}m\ \isacommand{by}\isamarkupfalse%
\ {\isacharparenleft}{\kern0pt}auto\ simp{\isacharcolon}{\kern0pt}\ Max{\isacharunderscore}{\kern0pt}eq{\isacharunderscore}{\kern0pt}iff{\isacharparenright}{\kern0pt}\isanewline
\ \ \isacommand{then}\isamarkupfalse%
\ \isacommand{show}\isamarkupfalse%
\ {\isacharquery}{\kern0pt}thesis\isanewline
\ \ \ \ \isacommand{using}\isamarkupfalse%
\ assms\ knights{\isacharunderscore}{\kern0pt}path{\isacharunderscore}{\kern0pt}set{\isacharunderscore}{\kern0pt}eq\ \isacommand{by}\isamarkupfalse%
\ auto\isanewline
\isacommand{qed}\isamarkupfalse%
%
\endisatagproof
{\isafoldproof}%
%
\isadelimproof
\isanewline
%
\endisadelimproof
\isanewline
\isacommand{lemma}\isamarkupfalse%
\ mirror{\isadigit{1}}{\isacharunderscore}{\kern0pt}aux{\isacharunderscore}{\kern0pt}nil{\isacharcolon}{\kern0pt}\ {\isachardoublequoteopen}ps\ {\isacharequal}{\kern0pt}\ {\isacharbrackleft}{\kern0pt}{\isacharbrackright}{\kern0pt}\ {\isasymlongleftrightarrow}\ mirror{\isadigit{1}}{\isacharunderscore}{\kern0pt}aux\ m\ ps\ {\isacharequal}{\kern0pt}\ {\isacharbrackleft}{\kern0pt}{\isacharbrackright}{\kern0pt}{\isachardoublequoteclose}\isanewline
%
\isadelimproof
\ \ %
\endisadelimproof
%
\isatagproof
\isacommand{using}\isamarkupfalse%
\ mirror{\isadigit{1}}{\isacharunderscore}{\kern0pt}aux{\isachardot}{\kern0pt}elims\ \isacommand{by}\isamarkupfalse%
\ blast%
\endisatagproof
{\isafoldproof}%
%
\isadelimproof
\isanewline
%
\endisadelimproof
\isanewline
\isacommand{lemma}\isamarkupfalse%
\ mirror{\isadigit{1}}{\isacharunderscore}{\kern0pt}nil{\isacharcolon}{\kern0pt}\ {\isachardoublequoteopen}ps\ {\isacharequal}{\kern0pt}\ {\isacharbrackleft}{\kern0pt}{\isacharbrackright}{\kern0pt}\ {\isasymlongleftrightarrow}\ mirror{\isadigit{1}}\ ps\ {\isacharequal}{\kern0pt}\ {\isacharbrackleft}{\kern0pt}{\isacharbrackright}{\kern0pt}{\isachardoublequoteclose}\isanewline
%
\isadelimproof
\ \ %
\endisadelimproof
%
\isatagproof
\isacommand{unfolding}\isamarkupfalse%
\ mirror{\isadigit{1}}{\isacharunderscore}{\kern0pt}def\ \isacommand{using}\isamarkupfalse%
\ mirror{\isadigit{1}}{\isacharunderscore}{\kern0pt}aux{\isacharunderscore}{\kern0pt}nil\ \isacommand{by}\isamarkupfalse%
\ blast%
\endisatagproof
{\isafoldproof}%
%
\isadelimproof
\isanewline
%
\endisadelimproof
\isanewline
\isacommand{lemma}\isamarkupfalse%
\ mirror{\isadigit{2}}{\isacharunderscore}{\kern0pt}aux{\isacharunderscore}{\kern0pt}nil{\isacharcolon}{\kern0pt}\ {\isachardoublequoteopen}ps\ {\isacharequal}{\kern0pt}\ {\isacharbrackleft}{\kern0pt}{\isacharbrackright}{\kern0pt}\ {\isasymlongleftrightarrow}\ mirror{\isadigit{2}}{\isacharunderscore}{\kern0pt}aux\ m\ ps\ {\isacharequal}{\kern0pt}\ {\isacharbrackleft}{\kern0pt}{\isacharbrackright}{\kern0pt}{\isachardoublequoteclose}\isanewline
%
\isadelimproof
\ \ %
\endisadelimproof
%
\isatagproof
\isacommand{using}\isamarkupfalse%
\ mirror{\isadigit{2}}{\isacharunderscore}{\kern0pt}aux{\isachardot}{\kern0pt}elims\ \isacommand{by}\isamarkupfalse%
\ blast%
\endisatagproof
{\isafoldproof}%
%
\isadelimproof
\isanewline
%
\endisadelimproof
\isanewline
\isacommand{lemma}\isamarkupfalse%
\ mirror{\isadigit{2}}{\isacharunderscore}{\kern0pt}nil{\isacharcolon}{\kern0pt}\ {\isachardoublequoteopen}ps\ {\isacharequal}{\kern0pt}\ {\isacharbrackleft}{\kern0pt}{\isacharbrackright}{\kern0pt}\ {\isasymlongleftrightarrow}\ mirror{\isadigit{2}}\ ps\ {\isacharequal}{\kern0pt}\ {\isacharbrackleft}{\kern0pt}{\isacharbrackright}{\kern0pt}{\isachardoublequoteclose}\isanewline
%
\isadelimproof
\ \ %
\endisadelimproof
%
\isatagproof
\isacommand{unfolding}\isamarkupfalse%
\ mirror{\isadigit{2}}{\isacharunderscore}{\kern0pt}def\ \isacommand{using}\isamarkupfalse%
\ mirror{\isadigit{2}}{\isacharunderscore}{\kern0pt}aux{\isacharunderscore}{\kern0pt}nil\ \isacommand{by}\isamarkupfalse%
\ blast%
\endisatagproof
{\isafoldproof}%
%
\isadelimproof
\isanewline
%
\endisadelimproof
\isanewline
\isacommand{lemma}\isamarkupfalse%
\ length{\isacharunderscore}{\kern0pt}mirror{\isadigit{1}}{\isacharunderscore}{\kern0pt}aux{\isacharcolon}{\kern0pt}\ {\isachardoublequoteopen}length\ ps\ {\isacharequal}{\kern0pt}\ length\ {\isacharparenleft}{\kern0pt}mirror{\isadigit{1}}{\isacharunderscore}{\kern0pt}aux\ n\ ps{\isacharparenright}{\kern0pt}{\isachardoublequoteclose}\isanewline
%
\isadelimproof
\ \ %
\endisadelimproof
%
\isatagproof
\isacommand{by}\isamarkupfalse%
\ {\isacharparenleft}{\kern0pt}induction\ ps{\isacharparenright}{\kern0pt}\ auto%
\endisatagproof
{\isafoldproof}%
%
\isadelimproof
\isanewline
%
\endisadelimproof
\isanewline
\isacommand{lemma}\isamarkupfalse%
\ length{\isacharunderscore}{\kern0pt}mirror{\isadigit{1}}{\isacharcolon}{\kern0pt}\ {\isachardoublequoteopen}length\ ps\ {\isacharequal}{\kern0pt}\ length\ {\isacharparenleft}{\kern0pt}mirror{\isadigit{1}}\ ps{\isacharparenright}{\kern0pt}{\isachardoublequoteclose}\isanewline
%
\isadelimproof
\ \ %
\endisadelimproof
%
\isatagproof
\isacommand{unfolding}\isamarkupfalse%
\ mirror{\isadigit{1}}{\isacharunderscore}{\kern0pt}def\ \isacommand{using}\isamarkupfalse%
\ length{\isacharunderscore}{\kern0pt}mirror{\isadigit{1}}{\isacharunderscore}{\kern0pt}aux\ \isacommand{by}\isamarkupfalse%
\ auto%
\endisatagproof
{\isafoldproof}%
%
\isadelimproof
\isanewline
%
\endisadelimproof
\isanewline
\isacommand{lemma}\isamarkupfalse%
\ length{\isacharunderscore}{\kern0pt}mirror{\isadigit{2}}{\isacharunderscore}{\kern0pt}aux{\isacharcolon}{\kern0pt}\ {\isachardoublequoteopen}length\ ps\ {\isacharequal}{\kern0pt}\ length\ {\isacharparenleft}{\kern0pt}mirror{\isadigit{2}}{\isacharunderscore}{\kern0pt}aux\ n\ ps{\isacharparenright}{\kern0pt}{\isachardoublequoteclose}\isanewline
%
\isadelimproof
\ \ %
\endisadelimproof
%
\isatagproof
\isacommand{by}\isamarkupfalse%
\ {\isacharparenleft}{\kern0pt}induction\ ps{\isacharparenright}{\kern0pt}\ auto%
\endisatagproof
{\isafoldproof}%
%
\isadelimproof
\isanewline
%
\endisadelimproof
\isanewline
\isacommand{lemma}\isamarkupfalse%
\ length{\isacharunderscore}{\kern0pt}mirror{\isadigit{2}}{\isacharcolon}{\kern0pt}\ {\isachardoublequoteopen}length\ ps\ {\isacharequal}{\kern0pt}\ length\ {\isacharparenleft}{\kern0pt}mirror{\isadigit{2}}\ ps{\isacharparenright}{\kern0pt}{\isachardoublequoteclose}\isanewline
%
\isadelimproof
\ \ %
\endisadelimproof
%
\isatagproof
\isacommand{unfolding}\isamarkupfalse%
\ mirror{\isadigit{2}}{\isacharunderscore}{\kern0pt}def\ \isacommand{using}\isamarkupfalse%
\ length{\isacharunderscore}{\kern0pt}mirror{\isadigit{2}}{\isacharunderscore}{\kern0pt}aux\ \isacommand{by}\isamarkupfalse%
\ auto%
\endisatagproof
{\isafoldproof}%
%
\isadelimproof
\isanewline
%
\endisadelimproof
\isanewline
\isacommand{lemma}\isamarkupfalse%
\ mirror{\isadigit{1}}{\isacharunderscore}{\kern0pt}board{\isacharunderscore}{\kern0pt}iff{\isacharcolon}{\kern0pt}{\isachardoublequoteopen}s\isactrlsub i\ {\isasymnotin}\ b\ {\isasymlongleftrightarrow}\ mirror{\isadigit{1}}{\isacharunderscore}{\kern0pt}square\ n\ s\isactrlsub i\ {\isasymnotin}\ mirror{\isadigit{1}}{\isacharunderscore}{\kern0pt}board\ n\ b{\isachardoublequoteclose}\isanewline
%
\isadelimproof
\ \ %
\endisadelimproof
%
\isatagproof
\isacommand{unfolding}\isamarkupfalse%
\ mirror{\isadigit{1}}{\isacharunderscore}{\kern0pt}board{\isacharunderscore}{\kern0pt}def\ mirror{\isadigit{1}}{\isacharunderscore}{\kern0pt}square{\isacharunderscore}{\kern0pt}def\ \isacommand{by}\isamarkupfalse%
\ {\isacharparenleft}{\kern0pt}auto\ split{\isacharcolon}{\kern0pt}\ prod{\isachardot}{\kern0pt}splits{\isacharparenright}{\kern0pt}%
\endisatagproof
{\isafoldproof}%
%
\isadelimproof
\isanewline
%
\endisadelimproof
\isanewline
\isacommand{lemma}\isamarkupfalse%
\ mirror{\isadigit{2}}{\isacharunderscore}{\kern0pt}board{\isacharunderscore}{\kern0pt}iff{\isacharcolon}{\kern0pt}{\isachardoublequoteopen}s\isactrlsub i\ {\isasymnotin}\ b\ {\isasymlongleftrightarrow}\ mirror{\isadigit{2}}{\isacharunderscore}{\kern0pt}square\ n\ s\isactrlsub i\ {\isasymnotin}\ mirror{\isadigit{2}}{\isacharunderscore}{\kern0pt}board\ n\ b{\isachardoublequoteclose}\isanewline
%
\isadelimproof
\ \ %
\endisadelimproof
%
\isatagproof
\isacommand{unfolding}\isamarkupfalse%
\ mirror{\isadigit{2}}{\isacharunderscore}{\kern0pt}board{\isacharunderscore}{\kern0pt}def\ mirror{\isadigit{2}}{\isacharunderscore}{\kern0pt}square{\isacharunderscore}{\kern0pt}def\ \isacommand{by}\isamarkupfalse%
\ {\isacharparenleft}{\kern0pt}auto\ split{\isacharcolon}{\kern0pt}\ prod{\isachardot}{\kern0pt}splits{\isacharparenright}{\kern0pt}%
\endisatagproof
{\isafoldproof}%
%
\isadelimproof
\isanewline
%
\endisadelimproof
\isanewline
\isacommand{lemma}\isamarkupfalse%
\ insert{\isacharunderscore}{\kern0pt}mirror{\isadigit{1}}{\isacharunderscore}{\kern0pt}board{\isacharcolon}{\kern0pt}\ \isanewline
\ \ {\isachardoublequoteopen}insert\ {\isacharparenleft}{\kern0pt}mirror{\isadigit{1}}{\isacharunderscore}{\kern0pt}square\ n\ s\isactrlsub i{\isacharparenright}{\kern0pt}\ {\isacharparenleft}{\kern0pt}mirror{\isadigit{1}}{\isacharunderscore}{\kern0pt}board\ n\ b{\isacharparenright}{\kern0pt}\ {\isacharequal}{\kern0pt}\ mirror{\isadigit{1}}{\isacharunderscore}{\kern0pt}board\ n\ {\isacharparenleft}{\kern0pt}insert\ s\isactrlsub i\ b{\isacharparenright}{\kern0pt}{\isachardoublequoteclose}\isanewline
%
\isadelimproof
\ \ %
\endisadelimproof
%
\isatagproof
\isacommand{unfolding}\isamarkupfalse%
\ mirror{\isadigit{1}}{\isacharunderscore}{\kern0pt}board{\isacharunderscore}{\kern0pt}def\ mirror{\isadigit{1}}{\isacharunderscore}{\kern0pt}square{\isacharunderscore}{\kern0pt}def\ \isacommand{by}\isamarkupfalse%
\ {\isacharparenleft}{\kern0pt}auto\ split{\isacharcolon}{\kern0pt}\ prod{\isachardot}{\kern0pt}splits{\isacharparenright}{\kern0pt}%
\endisatagproof
{\isafoldproof}%
%
\isadelimproof
\isanewline
%
\endisadelimproof
\isanewline
\isacommand{lemma}\isamarkupfalse%
\ insert{\isacharunderscore}{\kern0pt}mirror{\isadigit{2}}{\isacharunderscore}{\kern0pt}board{\isacharcolon}{\kern0pt}\ \isanewline
\ \ {\isachardoublequoteopen}insert\ {\isacharparenleft}{\kern0pt}mirror{\isadigit{2}}{\isacharunderscore}{\kern0pt}square\ n\ s\isactrlsub i{\isacharparenright}{\kern0pt}\ {\isacharparenleft}{\kern0pt}mirror{\isadigit{2}}{\isacharunderscore}{\kern0pt}board\ n\ b{\isacharparenright}{\kern0pt}\ {\isacharequal}{\kern0pt}\ mirror{\isadigit{2}}{\isacharunderscore}{\kern0pt}board\ n\ {\isacharparenleft}{\kern0pt}insert\ s\isactrlsub i\ b{\isacharparenright}{\kern0pt}{\isachardoublequoteclose}\isanewline
%
\isadelimproof
\ \ %
\endisadelimproof
%
\isatagproof
\isacommand{unfolding}\isamarkupfalse%
\ mirror{\isadigit{2}}{\isacharunderscore}{\kern0pt}board{\isacharunderscore}{\kern0pt}def\ mirror{\isadigit{2}}{\isacharunderscore}{\kern0pt}square{\isacharunderscore}{\kern0pt}def\ \isacommand{by}\isamarkupfalse%
\ {\isacharparenleft}{\kern0pt}auto\ split{\isacharcolon}{\kern0pt}\ prod{\isachardot}{\kern0pt}splits{\isacharparenright}{\kern0pt}%
\endisatagproof
{\isafoldproof}%
%
\isadelimproof
\isanewline
%
\endisadelimproof
\isanewline
\isacommand{lemma}\isamarkupfalse%
\ {\isachardoublequoteopen}{\isacharparenleft}{\kern0pt}i{\isacharcolon}{\kern0pt}{\isacharcolon}{\kern0pt}int{\isacharparenright}{\kern0pt}\ {\isacharequal}{\kern0pt}\ i{\isacharprime}{\kern0pt}{\isacharplus}{\kern0pt}{\isadigit{1}}\ {\isasymLongrightarrow}\ n{\isacharminus}{\kern0pt}i{\isacharequal}{\kern0pt}n{\isacharminus}{\kern0pt}{\isacharparenleft}{\kern0pt}i{\isacharprime}{\kern0pt}{\isacharplus}{\kern0pt}{\isadigit{1}}{\isacharparenright}{\kern0pt}{\isachardoublequoteclose}\isanewline
%
\isadelimproof
\ \ %
\endisadelimproof
%
\isatagproof
\isacommand{by}\isamarkupfalse%
\ auto%
\endisatagproof
{\isafoldproof}%
%
\isadelimproof
\isanewline
%
\endisadelimproof
\isanewline
\isacommand{lemma}\isamarkupfalse%
\ valid{\isacharunderscore}{\kern0pt}step{\isacharunderscore}{\kern0pt}mirror{\isadigit{1}}{\isacharcolon}{\kern0pt}\ \isanewline
\ \ {\isachardoublequoteopen}valid{\isacharunderscore}{\kern0pt}step\ s\isactrlsub i\ s\isactrlsub j\ {\isasymlongleftrightarrow}\ valid{\isacharunderscore}{\kern0pt}step\ {\isacharparenleft}{\kern0pt}mirror{\isadigit{1}}{\isacharunderscore}{\kern0pt}square\ n\ s\isactrlsub i{\isacharparenright}{\kern0pt}\ {\isacharparenleft}{\kern0pt}mirror{\isadigit{1}}{\isacharunderscore}{\kern0pt}square\ n\ s\isactrlsub j{\isacharparenright}{\kern0pt}{\isachardoublequoteclose}\ \isanewline
%
\isadelimproof
%
\endisadelimproof
%
\isatagproof
\isacommand{proof}\isamarkupfalse%
\isanewline
\ \ \isacommand{assume}\isamarkupfalse%
\ assms{\isacharcolon}{\kern0pt}\ {\isachardoublequoteopen}valid{\isacharunderscore}{\kern0pt}step\ s\isactrlsub i\ s\isactrlsub j{\isachardoublequoteclose}\isanewline
\ \ \isacommand{obtain}\isamarkupfalse%
\ i\ j\ i{\isacharprime}{\kern0pt}\ j{\isacharprime}{\kern0pt}\ \isakeyword{where}\ {\isacharbrackleft}{\kern0pt}simp{\isacharbrackright}{\kern0pt}{\isacharcolon}{\kern0pt}\ {\isachardoublequoteopen}s\isactrlsub i\ {\isacharequal}{\kern0pt}\ {\isacharparenleft}{\kern0pt}i{\isacharcomma}{\kern0pt}j{\isacharparenright}{\kern0pt}{\isachardoublequoteclose}\ {\isachardoublequoteopen}s\isactrlsub j\ {\isacharequal}{\kern0pt}\ {\isacharparenleft}{\kern0pt}i{\isacharprime}{\kern0pt}{\isacharcomma}{\kern0pt}j{\isacharprime}{\kern0pt}{\isacharparenright}{\kern0pt}{\isachardoublequoteclose}\ \isacommand{by}\isamarkupfalse%
\ force\isanewline
\ \ \isacommand{then}\isamarkupfalse%
\ \isacommand{have}\isamarkupfalse%
\ {\isachardoublequoteopen}valid{\isacharunderscore}{\kern0pt}step\ {\isacharparenleft}{\kern0pt}n{\isacharminus}{\kern0pt}i{\isacharcomma}{\kern0pt}j{\isacharparenright}{\kern0pt}\ {\isacharparenleft}{\kern0pt}n{\isacharminus}{\kern0pt}i{\isacharprime}{\kern0pt}{\isacharcomma}{\kern0pt}j{\isacharprime}{\kern0pt}{\isacharparenright}{\kern0pt}{\isachardoublequoteclose}\isanewline
\ \ \ \ \isacommand{using}\isamarkupfalse%
\ assms\ \isacommand{unfolding}\isamarkupfalse%
\ valid{\isacharunderscore}{\kern0pt}step{\isacharunderscore}{\kern0pt}def\isanewline
\ \ \ \ \isacommand{apply}\isamarkupfalse%
\ simp\isanewline
\ \ \ \ \isacommand{apply}\isamarkupfalse%
\ {\isacharparenleft}{\kern0pt}elim\ disjE{\isacharparenright}{\kern0pt}\isanewline
\ \ \ \ \isacommand{apply}\isamarkupfalse%
\ auto\isanewline
\ \ \ \ \isacommand{done}\isamarkupfalse%
\isanewline
\ \ \isacommand{then}\isamarkupfalse%
\ \isacommand{show}\isamarkupfalse%
\ {\isachardoublequoteopen}valid{\isacharunderscore}{\kern0pt}step\ {\isacharparenleft}{\kern0pt}mirror{\isadigit{1}}{\isacharunderscore}{\kern0pt}square\ n\ s\isactrlsub i{\isacharparenright}{\kern0pt}\ {\isacharparenleft}{\kern0pt}mirror{\isadigit{1}}{\isacharunderscore}{\kern0pt}square\ n\ s\isactrlsub j{\isacharparenright}{\kern0pt}{\isachardoublequoteclose}\isanewline
\ \ \ \ \isacommand{unfolding}\isamarkupfalse%
\ mirror{\isadigit{1}}{\isacharunderscore}{\kern0pt}square{\isacharunderscore}{\kern0pt}def\ \isacommand{by}\isamarkupfalse%
\ auto\isanewline
\isacommand{next}\isamarkupfalse%
\isanewline
\ \ \isacommand{assume}\isamarkupfalse%
\ assms{\isacharcolon}{\kern0pt}\ {\isachardoublequoteopen}valid{\isacharunderscore}{\kern0pt}step\ {\isacharparenleft}{\kern0pt}mirror{\isadigit{1}}{\isacharunderscore}{\kern0pt}square\ n\ s\isactrlsub i{\isacharparenright}{\kern0pt}\ {\isacharparenleft}{\kern0pt}mirror{\isadigit{1}}{\isacharunderscore}{\kern0pt}square\ n\ s\isactrlsub j{\isacharparenright}{\kern0pt}{\isachardoublequoteclose}\isanewline
\ \ \isacommand{obtain}\isamarkupfalse%
\ i\ j\ i{\isacharprime}{\kern0pt}\ j{\isacharprime}{\kern0pt}\ \isakeyword{where}\ {\isacharbrackleft}{\kern0pt}simp{\isacharbrackright}{\kern0pt}{\isacharcolon}{\kern0pt}\ {\isachardoublequoteopen}s\isactrlsub i\ {\isacharequal}{\kern0pt}\ {\isacharparenleft}{\kern0pt}i{\isacharcomma}{\kern0pt}j{\isacharparenright}{\kern0pt}{\isachardoublequoteclose}\ {\isachardoublequoteopen}s\isactrlsub j\ {\isacharequal}{\kern0pt}\ {\isacharparenleft}{\kern0pt}i{\isacharprime}{\kern0pt}{\isacharcomma}{\kern0pt}j{\isacharprime}{\kern0pt}{\isacharparenright}{\kern0pt}{\isachardoublequoteclose}\ \isacommand{by}\isamarkupfalse%
\ force\isanewline
\ \ \isacommand{then}\isamarkupfalse%
\ \isacommand{have}\isamarkupfalse%
\ {\isachardoublequoteopen}valid{\isacharunderscore}{\kern0pt}step\ {\isacharparenleft}{\kern0pt}i{\isacharcomma}{\kern0pt}j{\isacharparenright}{\kern0pt}\ {\isacharparenleft}{\kern0pt}i{\isacharprime}{\kern0pt}{\isacharcomma}{\kern0pt}j{\isacharprime}{\kern0pt}{\isacharparenright}{\kern0pt}{\isachardoublequoteclose}\isanewline
\ \ \ \ \isacommand{using}\isamarkupfalse%
\ assms\ \isacommand{unfolding}\isamarkupfalse%
\ valid{\isacharunderscore}{\kern0pt}step{\isacharunderscore}{\kern0pt}def\ mirror{\isadigit{1}}{\isacharunderscore}{\kern0pt}square{\isacharunderscore}{\kern0pt}def\isanewline
\ \ \ \ \isacommand{apply}\isamarkupfalse%
\ simp\isanewline
\ \ \ \ \isacommand{apply}\isamarkupfalse%
\ {\isacharparenleft}{\kern0pt}elim\ disjE{\isacharparenright}{\kern0pt}\isanewline
\ \ \ \ \isacommand{apply}\isamarkupfalse%
\ auto\isanewline
\ \ \ \ \isacommand{done}\isamarkupfalse%
\isanewline
\ \ \isacommand{then}\isamarkupfalse%
\ \isacommand{show}\isamarkupfalse%
\ {\isachardoublequoteopen}valid{\isacharunderscore}{\kern0pt}step\ s\isactrlsub i\ s\isactrlsub j{\isachardoublequoteclose}\isanewline
\ \ \ \ \isacommand{unfolding}\isamarkupfalse%
\ mirror{\isadigit{1}}{\isacharunderscore}{\kern0pt}square{\isacharunderscore}{\kern0pt}def\ \isacommand{by}\isamarkupfalse%
\ auto\isanewline
\isacommand{qed}\isamarkupfalse%
%
\endisatagproof
{\isafoldproof}%
%
\isadelimproof
\isanewline
%
\endisadelimproof
\isanewline
\isacommand{lemma}\isamarkupfalse%
\ valid{\isacharunderscore}{\kern0pt}step{\isacharunderscore}{\kern0pt}mirror{\isadigit{2}}{\isacharcolon}{\kern0pt}\ \isanewline
\ \ {\isachardoublequoteopen}valid{\isacharunderscore}{\kern0pt}step\ s\isactrlsub i\ s\isactrlsub j\ {\isasymlongleftrightarrow}\ valid{\isacharunderscore}{\kern0pt}step\ {\isacharparenleft}{\kern0pt}mirror{\isadigit{2}}{\isacharunderscore}{\kern0pt}square\ m\ s\isactrlsub i{\isacharparenright}{\kern0pt}\ {\isacharparenleft}{\kern0pt}mirror{\isadigit{2}}{\isacharunderscore}{\kern0pt}square\ m\ s\isactrlsub j{\isacharparenright}{\kern0pt}{\isachardoublequoteclose}\isanewline
%
\isadelimproof
%
\endisadelimproof
%
\isatagproof
\isacommand{proof}\isamarkupfalse%
\isanewline
\ \ \isacommand{assume}\isamarkupfalse%
\ assms{\isacharcolon}{\kern0pt}\ {\isachardoublequoteopen}valid{\isacharunderscore}{\kern0pt}step\ s\isactrlsub i\ s\isactrlsub j{\isachardoublequoteclose}\isanewline
\ \ \isacommand{obtain}\isamarkupfalse%
\ i\ j\ i{\isacharprime}{\kern0pt}\ j{\isacharprime}{\kern0pt}\ \isakeyword{where}\ {\isacharbrackleft}{\kern0pt}simp{\isacharbrackright}{\kern0pt}{\isacharcolon}{\kern0pt}\ {\isachardoublequoteopen}s\isactrlsub i\ {\isacharequal}{\kern0pt}\ {\isacharparenleft}{\kern0pt}i{\isacharcomma}{\kern0pt}j{\isacharparenright}{\kern0pt}{\isachardoublequoteclose}\ {\isachardoublequoteopen}s\isactrlsub j\ {\isacharequal}{\kern0pt}\ {\isacharparenleft}{\kern0pt}i{\isacharprime}{\kern0pt}{\isacharcomma}{\kern0pt}j{\isacharprime}{\kern0pt}{\isacharparenright}{\kern0pt}{\isachardoublequoteclose}\ \isacommand{by}\isamarkupfalse%
\ force\isanewline
\ \ \isacommand{then}\isamarkupfalse%
\ \isacommand{have}\isamarkupfalse%
\ {\isachardoublequoteopen}valid{\isacharunderscore}{\kern0pt}step\ {\isacharparenleft}{\kern0pt}i{\isacharcomma}{\kern0pt}m{\isacharminus}{\kern0pt}j{\isacharparenright}{\kern0pt}\ {\isacharparenleft}{\kern0pt}i{\isacharprime}{\kern0pt}{\isacharcomma}{\kern0pt}m{\isacharminus}{\kern0pt}j{\isacharprime}{\kern0pt}{\isacharparenright}{\kern0pt}{\isachardoublequoteclose}\isanewline
\ \ \ \ \isacommand{using}\isamarkupfalse%
\ assms\ \isacommand{unfolding}\isamarkupfalse%
\ valid{\isacharunderscore}{\kern0pt}step{\isacharunderscore}{\kern0pt}def\isanewline
\ \ \ \ \isacommand{apply}\isamarkupfalse%
\ simp\isanewline
\ \ \ \ \isacommand{apply}\isamarkupfalse%
\ {\isacharparenleft}{\kern0pt}elim\ disjE{\isacharparenright}{\kern0pt}\isanewline
\ \ \ \ \isacommand{apply}\isamarkupfalse%
\ auto\isanewline
\ \ \ \ \isacommand{done}\isamarkupfalse%
\isanewline
\ \ \isacommand{then}\isamarkupfalse%
\ \isacommand{show}\isamarkupfalse%
\ {\isachardoublequoteopen}valid{\isacharunderscore}{\kern0pt}step\ {\isacharparenleft}{\kern0pt}mirror{\isadigit{2}}{\isacharunderscore}{\kern0pt}square\ m\ s\isactrlsub i{\isacharparenright}{\kern0pt}\ {\isacharparenleft}{\kern0pt}mirror{\isadigit{2}}{\isacharunderscore}{\kern0pt}square\ m\ s\isactrlsub j{\isacharparenright}{\kern0pt}{\isachardoublequoteclose}\isanewline
\ \ \ \ \isacommand{unfolding}\isamarkupfalse%
\ mirror{\isadigit{2}}{\isacharunderscore}{\kern0pt}square{\isacharunderscore}{\kern0pt}def\ \isacommand{by}\isamarkupfalse%
\ auto\isanewline
\isacommand{next}\isamarkupfalse%
\isanewline
\ \ \isacommand{assume}\isamarkupfalse%
\ assms{\isacharcolon}{\kern0pt}\ {\isachardoublequoteopen}valid{\isacharunderscore}{\kern0pt}step\ {\isacharparenleft}{\kern0pt}mirror{\isadigit{2}}{\isacharunderscore}{\kern0pt}square\ m\ s\isactrlsub i{\isacharparenright}{\kern0pt}\ {\isacharparenleft}{\kern0pt}mirror{\isadigit{2}}{\isacharunderscore}{\kern0pt}square\ m\ s\isactrlsub j{\isacharparenright}{\kern0pt}{\isachardoublequoteclose}\isanewline
\ \ \isacommand{obtain}\isamarkupfalse%
\ i\ j\ i{\isacharprime}{\kern0pt}\ j{\isacharprime}{\kern0pt}\ \isakeyword{where}\ {\isacharbrackleft}{\kern0pt}simp{\isacharbrackright}{\kern0pt}{\isacharcolon}{\kern0pt}\ {\isachardoublequoteopen}s\isactrlsub i\ {\isacharequal}{\kern0pt}\ {\isacharparenleft}{\kern0pt}i{\isacharcomma}{\kern0pt}j{\isacharparenright}{\kern0pt}{\isachardoublequoteclose}\ {\isachardoublequoteopen}s\isactrlsub j\ {\isacharequal}{\kern0pt}\ {\isacharparenleft}{\kern0pt}i{\isacharprime}{\kern0pt}{\isacharcomma}{\kern0pt}j{\isacharprime}{\kern0pt}{\isacharparenright}{\kern0pt}{\isachardoublequoteclose}\ \isacommand{by}\isamarkupfalse%
\ force\isanewline
\ \ \isacommand{then}\isamarkupfalse%
\ \isacommand{have}\isamarkupfalse%
\ {\isachardoublequoteopen}valid{\isacharunderscore}{\kern0pt}step\ {\isacharparenleft}{\kern0pt}i{\isacharcomma}{\kern0pt}j{\isacharparenright}{\kern0pt}\ {\isacharparenleft}{\kern0pt}i{\isacharprime}{\kern0pt}{\isacharcomma}{\kern0pt}j{\isacharprime}{\kern0pt}{\isacharparenright}{\kern0pt}{\isachardoublequoteclose}\isanewline
\ \ \ \ \isacommand{using}\isamarkupfalse%
\ assms\ \isacommand{unfolding}\isamarkupfalse%
\ valid{\isacharunderscore}{\kern0pt}step{\isacharunderscore}{\kern0pt}def\ mirror{\isadigit{2}}{\isacharunderscore}{\kern0pt}square{\isacharunderscore}{\kern0pt}def\isanewline
\ \ \ \ \isacommand{apply}\isamarkupfalse%
\ simp\isanewline
\ \ \ \ \isacommand{apply}\isamarkupfalse%
\ {\isacharparenleft}{\kern0pt}elim\ disjE{\isacharparenright}{\kern0pt}\isanewline
\ \ \ \ \isacommand{apply}\isamarkupfalse%
\ auto\isanewline
\ \ \ \ \isacommand{done}\isamarkupfalse%
\isanewline
\ \ \isacommand{then}\isamarkupfalse%
\ \isacommand{show}\isamarkupfalse%
\ {\isachardoublequoteopen}valid{\isacharunderscore}{\kern0pt}step\ s\isactrlsub i\ s\isactrlsub j{\isachardoublequoteclose}\isanewline
\ \ \ \ \isacommand{unfolding}\isamarkupfalse%
\ mirror{\isadigit{1}}{\isacharunderscore}{\kern0pt}square{\isacharunderscore}{\kern0pt}def\ \isacommand{by}\isamarkupfalse%
\ auto\isanewline
\isacommand{qed}\isamarkupfalse%
%
\endisatagproof
{\isafoldproof}%
%
\isadelimproof
\isanewline
%
\endisadelimproof
\isanewline
\isacommand{lemma}\isamarkupfalse%
\ hd{\isacharunderscore}{\kern0pt}mirror{\isadigit{1}}{\isacharcolon}{\kern0pt}\isanewline
\ \ \isakeyword{assumes}\ {\isachardoublequoteopen}knights{\isacharunderscore}{\kern0pt}path\ {\isacharparenleft}{\kern0pt}board\ n\ m{\isacharparenright}{\kern0pt}\ ps{\isachardoublequoteclose}\ {\isachardoublequoteopen}hd\ ps\ {\isacharequal}{\kern0pt}\ {\isacharparenleft}{\kern0pt}i{\isacharcomma}{\kern0pt}j{\isacharparenright}{\kern0pt}{\isachardoublequoteclose}\isanewline
\ \ \isakeyword{shows}\ {\isachardoublequoteopen}hd\ {\isacharparenleft}{\kern0pt}mirror{\isadigit{1}}\ ps{\isacharparenright}{\kern0pt}\ {\isacharequal}{\kern0pt}\ {\isacharparenleft}{\kern0pt}int\ n{\isacharplus}{\kern0pt}{\isadigit{1}}{\isacharminus}{\kern0pt}i{\isacharcomma}{\kern0pt}j{\isacharparenright}{\kern0pt}{\isachardoublequoteclose}\isanewline
%
\isadelimproof
\ \ %
\endisadelimproof
%
\isatagproof
\isacommand{using}\isamarkupfalse%
\ assms\isanewline
\isacommand{proof}\isamarkupfalse%
\ {\isacharminus}{\kern0pt}\isanewline
\ \ \isacommand{have}\isamarkupfalse%
\ {\isachardoublequoteopen}hd\ {\isacharparenleft}{\kern0pt}mirror{\isadigit{1}}\ ps{\isacharparenright}{\kern0pt}\ {\isacharequal}{\kern0pt}\ hd\ {\isacharparenleft}{\kern0pt}mirror{\isadigit{1}}{\isacharunderscore}{\kern0pt}aux\ {\isacharparenleft}{\kern0pt}int\ n{\isacharplus}{\kern0pt}{\isadigit{1}}{\isacharparenright}{\kern0pt}\ ps{\isacharparenright}{\kern0pt}{\isachardoublequoteclose}\isanewline
\ \ \ \ \isacommand{unfolding}\isamarkupfalse%
\ mirror{\isadigit{1}}{\isacharunderscore}{\kern0pt}def\ \isacommand{using}\isamarkupfalse%
\ assms\ knights{\isacharunderscore}{\kern0pt}path{\isacharunderscore}{\kern0pt}min{\isadigit{1}}\ knights{\isacharunderscore}{\kern0pt}path{\isacharunderscore}{\kern0pt}max{\isadigit{1}}\ \isacommand{by}\isamarkupfalse%
\ auto\isanewline
\ \ \isacommand{also}\isamarkupfalse%
\ \isacommand{have}\isamarkupfalse%
\ {\isachardoublequoteopen}{\isachardot}{\kern0pt}{\isachardot}{\kern0pt}{\isachardot}{\kern0pt}\ {\isacharequal}{\kern0pt}\ hd\ {\isacharparenleft}{\kern0pt}mirror{\isadigit{1}}{\isacharunderscore}{\kern0pt}aux\ {\isacharparenleft}{\kern0pt}int\ n{\isacharplus}{\kern0pt}{\isadigit{1}}{\isacharparenright}{\kern0pt}\ {\isacharparenleft}{\kern0pt}{\isacharparenleft}{\kern0pt}hd\ ps{\isacharparenright}{\kern0pt}{\isacharhash}{\kern0pt}{\isacharparenleft}{\kern0pt}tl\ ps{\isacharparenright}{\kern0pt}{\isacharparenright}{\kern0pt}{\isacharparenright}{\kern0pt}{\isachardoublequoteclose}\isanewline
\ \ \ \ \isacommand{using}\isamarkupfalse%
\ assms\ knights{\isacharunderscore}{\kern0pt}path{\isacharunderscore}{\kern0pt}non{\isacharunderscore}{\kern0pt}nil\ \isacommand{by}\isamarkupfalse%
\ {\isacharparenleft}{\kern0pt}metis\ list{\isachardot}{\kern0pt}collapse{\isacharparenright}{\kern0pt}\isanewline
\ \ \isacommand{also}\isamarkupfalse%
\ \isacommand{have}\isamarkupfalse%
\ {\isachardoublequoteopen}{\isachardot}{\kern0pt}{\isachardot}{\kern0pt}{\isachardot}{\kern0pt}\ {\isacharequal}{\kern0pt}\ {\isacharparenleft}{\kern0pt}int\ n{\isacharplus}{\kern0pt}{\isadigit{1}}{\isacharminus}{\kern0pt}i{\isacharcomma}{\kern0pt}j{\isacharparenright}{\kern0pt}{\isachardoublequoteclose}\isanewline
\ \ \ \ \isacommand{using}\isamarkupfalse%
\ assms\ \isacommand{by}\isamarkupfalse%
\ {\isacharparenleft}{\kern0pt}auto\ simp{\isacharcolon}{\kern0pt}\ mirror{\isadigit{1}}{\isacharunderscore}{\kern0pt}square{\isacharunderscore}{\kern0pt}def{\isacharparenright}{\kern0pt}\isanewline
\ \ \isacommand{finally}\isamarkupfalse%
\ \isacommand{show}\isamarkupfalse%
\ {\isacharquery}{\kern0pt}thesis\ \isacommand{{\isachardot}{\kern0pt}}\isamarkupfalse%
\isanewline
\isacommand{qed}\isamarkupfalse%
%
\endisatagproof
{\isafoldproof}%
%
\isadelimproof
\isanewline
%
\endisadelimproof
\isanewline
\isacommand{lemma}\isamarkupfalse%
\ last{\isacharunderscore}{\kern0pt}mirror{\isadigit{1}}{\isacharunderscore}{\kern0pt}aux{\isacharcolon}{\kern0pt}\isanewline
\ \ \isakeyword{assumes}\ {\isachardoublequoteopen}ps\ {\isasymnoteq}\ {\isacharbrackleft}{\kern0pt}{\isacharbrackright}{\kern0pt}{\isachardoublequoteclose}\ {\isachardoublequoteopen}last\ ps\ {\isacharequal}{\kern0pt}\ {\isacharparenleft}{\kern0pt}i{\isacharcomma}{\kern0pt}j{\isacharparenright}{\kern0pt}{\isachardoublequoteclose}\isanewline
\ \ \isakeyword{shows}\ {\isachardoublequoteopen}last\ {\isacharparenleft}{\kern0pt}mirror{\isadigit{1}}{\isacharunderscore}{\kern0pt}aux\ n\ ps{\isacharparenright}{\kern0pt}\ {\isacharequal}{\kern0pt}\ {\isacharparenleft}{\kern0pt}n{\isacharminus}{\kern0pt}i{\isacharcomma}{\kern0pt}j{\isacharparenright}{\kern0pt}{\isachardoublequoteclose}\isanewline
%
\isadelimproof
\ \ %
\endisadelimproof
%
\isatagproof
\isacommand{using}\isamarkupfalse%
\ assms\isanewline
\isacommand{proof}\isamarkupfalse%
\ {\isacharparenleft}{\kern0pt}induction\ ps{\isacharparenright}{\kern0pt}\isanewline
\ \ \isacommand{case}\isamarkupfalse%
\ {\isacharparenleft}{\kern0pt}Cons\ s\isactrlsub i\ ps{\isacharparenright}{\kern0pt}\isanewline
\ \ \isacommand{then}\isamarkupfalse%
\ \isacommand{show}\isamarkupfalse%
\ {\isacharquery}{\kern0pt}case\ \isanewline
\ \ \ \ \isacommand{using}\isamarkupfalse%
\ mirror{\isadigit{1}}{\isacharunderscore}{\kern0pt}aux{\isacharunderscore}{\kern0pt}nil\ Cons\ \isacommand{by}\isamarkupfalse%
\ {\isacharparenleft}{\kern0pt}cases\ {\isachardoublequoteopen}ps\ {\isacharequal}{\kern0pt}\ {\isacharbrackleft}{\kern0pt}{\isacharbrackright}{\kern0pt}{\isachardoublequoteclose}{\isacharparenright}{\kern0pt}\ {\isacharparenleft}{\kern0pt}auto\ simp{\isacharcolon}{\kern0pt}\ mirror{\isadigit{1}}{\isacharunderscore}{\kern0pt}square{\isacharunderscore}{\kern0pt}def{\isacharparenright}{\kern0pt}\isanewline
\isacommand{qed}\isamarkupfalse%
\ auto%
\endisatagproof
{\isafoldproof}%
%
\isadelimproof
\isanewline
%
\endisadelimproof
\isanewline
\isacommand{lemma}\isamarkupfalse%
\ last{\isacharunderscore}{\kern0pt}mirror{\isadigit{1}}{\isacharcolon}{\kern0pt}\isanewline
\ \ \isakeyword{assumes}\ {\isachardoublequoteopen}knights{\isacharunderscore}{\kern0pt}path\ {\isacharparenleft}{\kern0pt}board\ n\ m{\isacharparenright}{\kern0pt}\ ps{\isachardoublequoteclose}\ {\isachardoublequoteopen}last\ ps\ {\isacharequal}{\kern0pt}\ {\isacharparenleft}{\kern0pt}i{\isacharcomma}{\kern0pt}j{\isacharparenright}{\kern0pt}{\isachardoublequoteclose}\isanewline
\ \ \isakeyword{shows}\ {\isachardoublequoteopen}last\ {\isacharparenleft}{\kern0pt}mirror{\isadigit{1}}\ ps{\isacharparenright}{\kern0pt}\ {\isacharequal}{\kern0pt}\ {\isacharparenleft}{\kern0pt}int\ n{\isacharplus}{\kern0pt}{\isadigit{1}}{\isacharminus}{\kern0pt}i{\isacharcomma}{\kern0pt}j{\isacharparenright}{\kern0pt}{\isachardoublequoteclose}\isanewline
%
\isadelimproof
\ \ %
\endisadelimproof
%
\isatagproof
\isacommand{unfolding}\isamarkupfalse%
\ mirror{\isadigit{1}}{\isacharunderscore}{\kern0pt}def\ \isacommand{using}\isamarkupfalse%
\ assms\ last{\isacharunderscore}{\kern0pt}mirror{\isadigit{1}}{\isacharunderscore}{\kern0pt}aux\ knights{\isacharunderscore}{\kern0pt}path{\isacharunderscore}{\kern0pt}non{\isacharunderscore}{\kern0pt}nil\isanewline
\ \ \isacommand{by}\isamarkupfalse%
\ {\isacharparenleft}{\kern0pt}simp\ add{\isacharcolon}{\kern0pt}\ knights{\isacharunderscore}{\kern0pt}path{\isacharunderscore}{\kern0pt}max{\isadigit{1}}\ knights{\isacharunderscore}{\kern0pt}path{\isacharunderscore}{\kern0pt}min{\isadigit{1}}{\isacharparenright}{\kern0pt}%
\endisatagproof
{\isafoldproof}%
%
\isadelimproof
\isanewline
%
\endisadelimproof
\isanewline
\isacommand{lemma}\isamarkupfalse%
\ hd{\isacharunderscore}{\kern0pt}mirror{\isadigit{2}}{\isacharcolon}{\kern0pt}\isanewline
\ \ \isakeyword{assumes}\ {\isachardoublequoteopen}knights{\isacharunderscore}{\kern0pt}path\ {\isacharparenleft}{\kern0pt}board\ n\ m{\isacharparenright}{\kern0pt}\ ps{\isachardoublequoteclose}\ {\isachardoublequoteopen}hd\ ps\ {\isacharequal}{\kern0pt}\ {\isacharparenleft}{\kern0pt}i{\isacharcomma}{\kern0pt}j{\isacharparenright}{\kern0pt}{\isachardoublequoteclose}\isanewline
\ \ \isakeyword{shows}\ {\isachardoublequoteopen}hd\ {\isacharparenleft}{\kern0pt}mirror{\isadigit{2}}\ ps{\isacharparenright}{\kern0pt}\ {\isacharequal}{\kern0pt}\ {\isacharparenleft}{\kern0pt}i{\isacharcomma}{\kern0pt}int\ m{\isacharplus}{\kern0pt}{\isadigit{1}}{\isacharminus}{\kern0pt}j{\isacharparenright}{\kern0pt}{\isachardoublequoteclose}\isanewline
%
\isadelimproof
\ \ %
\endisadelimproof
%
\isatagproof
\isacommand{using}\isamarkupfalse%
\ assms\isanewline
\isacommand{proof}\isamarkupfalse%
\ {\isacharminus}{\kern0pt}\isanewline
\ \ \isacommand{have}\isamarkupfalse%
\ {\isachardoublequoteopen}hd\ {\isacharparenleft}{\kern0pt}mirror{\isadigit{2}}\ ps{\isacharparenright}{\kern0pt}\ {\isacharequal}{\kern0pt}\ hd\ {\isacharparenleft}{\kern0pt}mirror{\isadigit{2}}{\isacharunderscore}{\kern0pt}aux\ {\isacharparenleft}{\kern0pt}int\ m{\isacharplus}{\kern0pt}{\isadigit{1}}{\isacharparenright}{\kern0pt}\ ps{\isacharparenright}{\kern0pt}{\isachardoublequoteclose}\isanewline
\ \ \ \ \isacommand{unfolding}\isamarkupfalse%
\ mirror{\isadigit{2}}{\isacharunderscore}{\kern0pt}def\ \isacommand{using}\isamarkupfalse%
\ assms\ knights{\isacharunderscore}{\kern0pt}path{\isacharunderscore}{\kern0pt}min{\isadigit{2}}\ knights{\isacharunderscore}{\kern0pt}path{\isacharunderscore}{\kern0pt}max{\isadigit{2}}\ \isacommand{by}\isamarkupfalse%
\ auto\isanewline
\ \ \isacommand{also}\isamarkupfalse%
\ \isacommand{have}\isamarkupfalse%
\ {\isachardoublequoteopen}{\isachardot}{\kern0pt}{\isachardot}{\kern0pt}{\isachardot}{\kern0pt}\ {\isacharequal}{\kern0pt}\ hd\ {\isacharparenleft}{\kern0pt}mirror{\isadigit{2}}{\isacharunderscore}{\kern0pt}aux\ {\isacharparenleft}{\kern0pt}int\ m{\isacharplus}{\kern0pt}{\isadigit{1}}{\isacharparenright}{\kern0pt}\ {\isacharparenleft}{\kern0pt}{\isacharparenleft}{\kern0pt}hd\ ps{\isacharparenright}{\kern0pt}{\isacharhash}{\kern0pt}{\isacharparenleft}{\kern0pt}tl\ ps{\isacharparenright}{\kern0pt}{\isacharparenright}{\kern0pt}{\isacharparenright}{\kern0pt}{\isachardoublequoteclose}\isanewline
\ \ \ \ \isacommand{using}\isamarkupfalse%
\ assms\ knights{\isacharunderscore}{\kern0pt}path{\isacharunderscore}{\kern0pt}non{\isacharunderscore}{\kern0pt}nil\ \isacommand{by}\isamarkupfalse%
\ {\isacharparenleft}{\kern0pt}metis\ list{\isachardot}{\kern0pt}collapse{\isacharparenright}{\kern0pt}\isanewline
\ \ \isacommand{also}\isamarkupfalse%
\ \isacommand{have}\isamarkupfalse%
\ {\isachardoublequoteopen}{\isachardot}{\kern0pt}{\isachardot}{\kern0pt}{\isachardot}{\kern0pt}\ {\isacharequal}{\kern0pt}\ {\isacharparenleft}{\kern0pt}i{\isacharcomma}{\kern0pt}int\ m{\isacharplus}{\kern0pt}{\isadigit{1}}{\isacharminus}{\kern0pt}j{\isacharparenright}{\kern0pt}{\isachardoublequoteclose}\isanewline
\ \ \ \ \isacommand{using}\isamarkupfalse%
\ assms\ \isacommand{by}\isamarkupfalse%
\ {\isacharparenleft}{\kern0pt}auto\ simp{\isacharcolon}{\kern0pt}\ mirror{\isadigit{2}}{\isacharunderscore}{\kern0pt}square{\isacharunderscore}{\kern0pt}def{\isacharparenright}{\kern0pt}\isanewline
\ \ \isacommand{finally}\isamarkupfalse%
\ \isacommand{show}\isamarkupfalse%
\ {\isacharquery}{\kern0pt}thesis\ \isacommand{{\isachardot}{\kern0pt}}\isamarkupfalse%
\isanewline
\isacommand{qed}\isamarkupfalse%
%
\endisatagproof
{\isafoldproof}%
%
\isadelimproof
\isanewline
%
\endisadelimproof
\isanewline
\isacommand{lemma}\isamarkupfalse%
\ last{\isacharunderscore}{\kern0pt}mirror{\isadigit{2}}{\isacharunderscore}{\kern0pt}aux{\isacharcolon}{\kern0pt}\isanewline
\ \ \isakeyword{assumes}\ {\isachardoublequoteopen}ps\ {\isasymnoteq}\ {\isacharbrackleft}{\kern0pt}{\isacharbrackright}{\kern0pt}{\isachardoublequoteclose}\ {\isachardoublequoteopen}last\ ps\ {\isacharequal}{\kern0pt}\ {\isacharparenleft}{\kern0pt}i{\isacharcomma}{\kern0pt}j{\isacharparenright}{\kern0pt}{\isachardoublequoteclose}\isanewline
\ \ \isakeyword{shows}\ {\isachardoublequoteopen}last\ {\isacharparenleft}{\kern0pt}mirror{\isadigit{2}}{\isacharunderscore}{\kern0pt}aux\ m\ ps{\isacharparenright}{\kern0pt}\ {\isacharequal}{\kern0pt}\ {\isacharparenleft}{\kern0pt}i{\isacharcomma}{\kern0pt}m{\isacharminus}{\kern0pt}j{\isacharparenright}{\kern0pt}{\isachardoublequoteclose}\isanewline
%
\isadelimproof
\ \ %
\endisadelimproof
%
\isatagproof
\isacommand{using}\isamarkupfalse%
\ assms\isanewline
\isacommand{proof}\isamarkupfalse%
\ {\isacharparenleft}{\kern0pt}induction\ ps{\isacharparenright}{\kern0pt}\isanewline
\ \ \isacommand{case}\isamarkupfalse%
\ {\isacharparenleft}{\kern0pt}Cons\ s\isactrlsub i\ ps{\isacharparenright}{\kern0pt}\isanewline
\ \ \isacommand{then}\isamarkupfalse%
\ \isacommand{show}\isamarkupfalse%
\ {\isacharquery}{\kern0pt}case\ \isanewline
\ \ \ \ \isacommand{using}\isamarkupfalse%
\ mirror{\isadigit{2}}{\isacharunderscore}{\kern0pt}aux{\isacharunderscore}{\kern0pt}nil\ Cons\ \isacommand{by}\isamarkupfalse%
\ {\isacharparenleft}{\kern0pt}cases\ {\isachardoublequoteopen}ps\ {\isacharequal}{\kern0pt}\ {\isacharbrackleft}{\kern0pt}{\isacharbrackright}{\kern0pt}{\isachardoublequoteclose}{\isacharparenright}{\kern0pt}\ {\isacharparenleft}{\kern0pt}auto\ simp{\isacharcolon}{\kern0pt}\ mirror{\isadigit{2}}{\isacharunderscore}{\kern0pt}square{\isacharunderscore}{\kern0pt}def{\isacharparenright}{\kern0pt}\isanewline
\isacommand{qed}\isamarkupfalse%
\ auto%
\endisatagproof
{\isafoldproof}%
%
\isadelimproof
\isanewline
%
\endisadelimproof
\isanewline
\isacommand{lemma}\isamarkupfalse%
\ last{\isacharunderscore}{\kern0pt}mirror{\isadigit{2}}{\isacharcolon}{\kern0pt}\isanewline
\ \ \isakeyword{assumes}\ {\isachardoublequoteopen}knights{\isacharunderscore}{\kern0pt}path\ {\isacharparenleft}{\kern0pt}board\ n\ m{\isacharparenright}{\kern0pt}\ ps{\isachardoublequoteclose}\ {\isachardoublequoteopen}last\ ps\ {\isacharequal}{\kern0pt}\ {\isacharparenleft}{\kern0pt}i{\isacharcomma}{\kern0pt}j{\isacharparenright}{\kern0pt}{\isachardoublequoteclose}\isanewline
\ \ \isakeyword{shows}\ {\isachardoublequoteopen}last\ {\isacharparenleft}{\kern0pt}mirror{\isadigit{2}}\ ps{\isacharparenright}{\kern0pt}\ {\isacharequal}{\kern0pt}\ {\isacharparenleft}{\kern0pt}i{\isacharcomma}{\kern0pt}int\ m{\isacharplus}{\kern0pt}{\isadigit{1}}{\isacharminus}{\kern0pt}j{\isacharparenright}{\kern0pt}{\isachardoublequoteclose}\isanewline
%
\isadelimproof
\ \ %
\endisadelimproof
%
\isatagproof
\isacommand{unfolding}\isamarkupfalse%
\ mirror{\isadigit{2}}{\isacharunderscore}{\kern0pt}def\ \isacommand{using}\isamarkupfalse%
\ assms\ last{\isacharunderscore}{\kern0pt}mirror{\isadigit{2}}{\isacharunderscore}{\kern0pt}aux\ knights{\isacharunderscore}{\kern0pt}path{\isacharunderscore}{\kern0pt}non{\isacharunderscore}{\kern0pt}nil\isanewline
\ \ \isacommand{by}\isamarkupfalse%
\ {\isacharparenleft}{\kern0pt}simp\ add{\isacharcolon}{\kern0pt}\ knights{\isacharunderscore}{\kern0pt}path{\isacharunderscore}{\kern0pt}max{\isadigit{2}}\ knights{\isacharunderscore}{\kern0pt}path{\isacharunderscore}{\kern0pt}min{\isadigit{2}}{\isacharparenright}{\kern0pt}%
\endisatagproof
{\isafoldproof}%
%
\isadelimproof
\isanewline
%
\endisadelimproof
\isanewline
\isacommand{lemma}\isamarkupfalse%
\ mirror{\isadigit{1}}{\isacharunderscore}{\kern0pt}aux{\isacharunderscore}{\kern0pt}knights{\isacharunderscore}{\kern0pt}path{\isacharcolon}{\kern0pt}\isanewline
\ \ \isakeyword{assumes}\ {\isachardoublequoteopen}knights{\isacharunderscore}{\kern0pt}path\ b\ ps{\isachardoublequoteclose}\ \isanewline
\ \ \isakeyword{shows}\ {\isachardoublequoteopen}knights{\isacharunderscore}{\kern0pt}path\ {\isacharparenleft}{\kern0pt}mirror{\isadigit{1}}{\isacharunderscore}{\kern0pt}board\ n\ b{\isacharparenright}{\kern0pt}\ {\isacharparenleft}{\kern0pt}mirror{\isadigit{1}}{\isacharunderscore}{\kern0pt}aux\ n\ ps{\isacharparenright}{\kern0pt}{\isachardoublequoteclose}\isanewline
%
\isadelimproof
\ \ %
\endisadelimproof
%
\isatagproof
\isacommand{using}\isamarkupfalse%
\ assms\isanewline
\isacommand{proof}\isamarkupfalse%
\ {\isacharparenleft}{\kern0pt}induction\ rule{\isacharcolon}{\kern0pt}\ knights{\isacharunderscore}{\kern0pt}path{\isachardot}{\kern0pt}induct{\isacharparenright}{\kern0pt}\isanewline
\ \ \isacommand{case}\isamarkupfalse%
\ {\isacharparenleft}{\kern0pt}{\isadigit{1}}\ s\isactrlsub i{\isacharparenright}{\kern0pt}\isanewline
\ \ \isacommand{then}\isamarkupfalse%
\ \isacommand{have}\isamarkupfalse%
\ {\isachardoublequoteopen}mirror{\isadigit{1}}{\isacharunderscore}{\kern0pt}board\ n\ {\isacharbraceleft}{\kern0pt}s\isactrlsub i{\isacharbraceright}{\kern0pt}\ {\isacharequal}{\kern0pt}\ {\isacharbraceleft}{\kern0pt}mirror{\isadigit{1}}{\isacharunderscore}{\kern0pt}square\ n\ s\isactrlsub i{\isacharbraceright}{\kern0pt}{\isachardoublequoteclose}\ \isanewline
\ \ \ \ \isacommand{unfolding}\isamarkupfalse%
\ mirror{\isadigit{1}}{\isacharunderscore}{\kern0pt}board{\isacharunderscore}{\kern0pt}def\ \isacommand{by}\isamarkupfalse%
\ blast\isanewline
\ \ \isacommand{then}\isamarkupfalse%
\ \isacommand{show}\isamarkupfalse%
\ {\isacharquery}{\kern0pt}case\ \isacommand{by}\isamarkupfalse%
\ {\isacharparenleft}{\kern0pt}auto\ intro{\isacharcolon}{\kern0pt}\ knights{\isacharunderscore}{\kern0pt}path{\isachardot}{\kern0pt}intros{\isacharparenright}{\kern0pt}\isanewline
\isacommand{next}\isamarkupfalse%
\isanewline
\ \ \isacommand{case}\isamarkupfalse%
\ {\isacharparenleft}{\kern0pt}{\isadigit{2}}\ s\isactrlsub i\ b\ s\isactrlsub j\ ps{\isacharparenright}{\kern0pt}\isanewline
\ \ \isacommand{then}\isamarkupfalse%
\ \isacommand{have}\isamarkupfalse%
\ prems{\isacharcolon}{\kern0pt}\ {\isachardoublequoteopen}mirror{\isadigit{1}}{\isacharunderscore}{\kern0pt}square\ n\ s\isactrlsub i\ {\isasymnotin}\ mirror{\isadigit{1}}{\isacharunderscore}{\kern0pt}board\ n\ b{\isachardoublequoteclose}\ \isanewline
\ \ \ \ \ \ \ \ \ \ \ \ {\isachardoublequoteopen}valid{\isacharunderscore}{\kern0pt}step\ {\isacharparenleft}{\kern0pt}mirror{\isadigit{1}}{\isacharunderscore}{\kern0pt}square\ n\ s\isactrlsub i{\isacharparenright}{\kern0pt}\ {\isacharparenleft}{\kern0pt}mirror{\isadigit{1}}{\isacharunderscore}{\kern0pt}square\ n\ s\isactrlsub j{\isacharparenright}{\kern0pt}{\isachardoublequoteclose}\ \isanewline
\ \ \ \ \ \ \ \ \ \ \ \ \isakeyword{and}\ {\isachardoublequoteopen}mirror{\isadigit{1}}{\isacharunderscore}{\kern0pt}aux\ n\ {\isacharparenleft}{\kern0pt}s\isactrlsub j{\isacharhash}{\kern0pt}ps{\isacharparenright}{\kern0pt}\ {\isacharequal}{\kern0pt}\ mirror{\isadigit{1}}{\isacharunderscore}{\kern0pt}square\ n\ s\isactrlsub j{\isacharhash}{\kern0pt}mirror{\isadigit{1}}{\isacharunderscore}{\kern0pt}aux\ n\ ps{\isachardoublequoteclose}\isanewline
\ \ \ \ \isacommand{using}\isamarkupfalse%
\ {\isadigit{2}}\ mirror{\isadigit{1}}{\isacharunderscore}{\kern0pt}board{\isacharunderscore}{\kern0pt}iff\ valid{\isacharunderscore}{\kern0pt}step{\isacharunderscore}{\kern0pt}mirror{\isadigit{1}}\ \isacommand{by}\isamarkupfalse%
\ auto\isanewline
\ \ \isacommand{then}\isamarkupfalse%
\ \isacommand{show}\isamarkupfalse%
\ {\isacharquery}{\kern0pt}case\ \isanewline
\ \ \ \ \isacommand{using}\isamarkupfalse%
\ {\isadigit{2}}\ knights{\isacharunderscore}{\kern0pt}path{\isachardot}{\kern0pt}intros{\isacharparenleft}{\kern0pt}{\isadigit{2}}{\isacharparenright}{\kern0pt}{\isacharbrackleft}{\kern0pt}OF\ prems{\isacharbrackright}{\kern0pt}\ insert{\isacharunderscore}{\kern0pt}mirror{\isadigit{1}}{\isacharunderscore}{\kern0pt}board\ \isacommand{by}\isamarkupfalse%
\ auto\isanewline
\isacommand{qed}\isamarkupfalse%
%
\endisatagproof
{\isafoldproof}%
%
\isadelimproof
\isanewline
%
\endisadelimproof
\isanewline
\isacommand{corollary}\isamarkupfalse%
\ mirror{\isadigit{1}}{\isacharunderscore}{\kern0pt}knights{\isacharunderscore}{\kern0pt}path{\isacharcolon}{\kern0pt}\isanewline
\ \ \isakeyword{assumes}\ {\isachardoublequoteopen}knights{\isacharunderscore}{\kern0pt}path\ {\isacharparenleft}{\kern0pt}board\ n\ m{\isacharparenright}{\kern0pt}\ ps{\isachardoublequoteclose}\ \isanewline
\ \ \isakeyword{shows}\ {\isachardoublequoteopen}knights{\isacharunderscore}{\kern0pt}path\ {\isacharparenleft}{\kern0pt}board\ n\ m{\isacharparenright}{\kern0pt}\ {\isacharparenleft}{\kern0pt}mirror{\isadigit{1}}\ ps{\isacharparenright}{\kern0pt}{\isachardoublequoteclose}\isanewline
%
\isadelimproof
\ \ %
\endisadelimproof
%
\isatagproof
\isacommand{using}\isamarkupfalse%
\ assms\isanewline
\isacommand{proof}\isamarkupfalse%
\ {\isacharminus}{\kern0pt}\isanewline
\ \ \isacommand{have}\isamarkupfalse%
\ {\isacharbrackleft}{\kern0pt}simp{\isacharbrackright}{\kern0pt}{\isacharcolon}{\kern0pt}\ {\isachardoublequoteopen}min{\isadigit{1}}\ ps\ {\isacharequal}{\kern0pt}\ {\isadigit{1}}{\isachardoublequoteclose}\ {\isachardoublequoteopen}max{\isadigit{1}}\ ps\ {\isacharequal}{\kern0pt}\ int\ n{\isachardoublequoteclose}\isanewline
\ \ \ \ \isacommand{using}\isamarkupfalse%
\ assms\ knights{\isacharunderscore}{\kern0pt}path{\isacharunderscore}{\kern0pt}min{\isadigit{1}}\ knights{\isacharunderscore}{\kern0pt}path{\isacharunderscore}{\kern0pt}max{\isadigit{1}}\ \isacommand{by}\isamarkupfalse%
\ auto\isanewline
\ \ \isacommand{then}\isamarkupfalse%
\ \isacommand{have}\isamarkupfalse%
\ {\isachardoublequoteopen}mirror{\isadigit{1}}{\isacharunderscore}{\kern0pt}board\ {\isacharparenleft}{\kern0pt}int\ n{\isacharplus}{\kern0pt}{\isadigit{1}}{\isacharparenright}{\kern0pt}\ {\isacharparenleft}{\kern0pt}board\ n\ m{\isacharparenright}{\kern0pt}\ {\isacharequal}{\kern0pt}\ {\isacharparenleft}{\kern0pt}board\ n\ m{\isacharparenright}{\kern0pt}{\isachardoublequoteclose}\isanewline
\ \ \ \ \isacommand{using}\isamarkupfalse%
\ mirror{\isadigit{1}}{\isacharunderscore}{\kern0pt}board{\isacharunderscore}{\kern0pt}id\ \isacommand{by}\isamarkupfalse%
\ auto\isanewline
\ \ \isacommand{then}\isamarkupfalse%
\ \isacommand{have}\isamarkupfalse%
\ {\isachardoublequoteopen}knights{\isacharunderscore}{\kern0pt}path\ {\isacharparenleft}{\kern0pt}board\ n\ m{\isacharparenright}{\kern0pt}\ {\isacharparenleft}{\kern0pt}mirror{\isadigit{1}}{\isacharunderscore}{\kern0pt}aux\ {\isacharparenleft}{\kern0pt}int\ n{\isacharplus}{\kern0pt}{\isadigit{1}}{\isacharparenright}{\kern0pt}\ ps{\isacharparenright}{\kern0pt}{\isachardoublequoteclose}\isanewline
\ \ \ \ \isacommand{using}\isamarkupfalse%
\ assms\ mirror{\isadigit{1}}{\isacharunderscore}{\kern0pt}aux{\isacharunderscore}{\kern0pt}knights{\isacharunderscore}{\kern0pt}path{\isacharbrackleft}{\kern0pt}of\ {\isachardoublequoteopen}board\ n\ m{\isachardoublequoteclose}\ ps\ {\isachardoublequoteopen}int\ n{\isacharplus}{\kern0pt}{\isadigit{1}}{\isachardoublequoteclose}{\isacharbrackright}{\kern0pt}\ \isacommand{by}\isamarkupfalse%
\ auto\isanewline
\ \ \isacommand{then}\isamarkupfalse%
\ \isacommand{show}\isamarkupfalse%
\ {\isacharquery}{\kern0pt}thesis\ \isacommand{unfolding}\isamarkupfalse%
\ mirror{\isadigit{1}}{\isacharunderscore}{\kern0pt}def\ \isacommand{by}\isamarkupfalse%
\ auto\isanewline
\isacommand{qed}\isamarkupfalse%
%
\endisatagproof
{\isafoldproof}%
%
\isadelimproof
\isanewline
%
\endisadelimproof
\isanewline
\isacommand{lemma}\isamarkupfalse%
\ mirror{\isadigit{2}}{\isacharunderscore}{\kern0pt}aux{\isacharunderscore}{\kern0pt}knights{\isacharunderscore}{\kern0pt}path{\isacharcolon}{\kern0pt}\isanewline
\ \ \isakeyword{assumes}\ {\isachardoublequoteopen}knights{\isacharunderscore}{\kern0pt}path\ b\ ps{\isachardoublequoteclose}\ \isanewline
\ \ \isakeyword{shows}\ {\isachardoublequoteopen}knights{\isacharunderscore}{\kern0pt}path\ {\isacharparenleft}{\kern0pt}mirror{\isadigit{2}}{\isacharunderscore}{\kern0pt}board\ n\ b{\isacharparenright}{\kern0pt}\ {\isacharparenleft}{\kern0pt}mirror{\isadigit{2}}{\isacharunderscore}{\kern0pt}aux\ n\ ps{\isacharparenright}{\kern0pt}{\isachardoublequoteclose}\isanewline
%
\isadelimproof
\ \ %
\endisadelimproof
%
\isatagproof
\isacommand{using}\isamarkupfalse%
\ assms\isanewline
\isacommand{proof}\isamarkupfalse%
\ {\isacharparenleft}{\kern0pt}induction\ rule{\isacharcolon}{\kern0pt}\ knights{\isacharunderscore}{\kern0pt}path{\isachardot}{\kern0pt}induct{\isacharparenright}{\kern0pt}\isanewline
\ \ \isacommand{case}\isamarkupfalse%
\ {\isacharparenleft}{\kern0pt}{\isadigit{1}}\ s\isactrlsub i{\isacharparenright}{\kern0pt}\isanewline
\ \ \isacommand{then}\isamarkupfalse%
\ \isacommand{have}\isamarkupfalse%
\ {\isachardoublequoteopen}mirror{\isadigit{2}}{\isacharunderscore}{\kern0pt}board\ n\ {\isacharbraceleft}{\kern0pt}s\isactrlsub i{\isacharbraceright}{\kern0pt}\ {\isacharequal}{\kern0pt}\ {\isacharbraceleft}{\kern0pt}mirror{\isadigit{2}}{\isacharunderscore}{\kern0pt}square\ n\ s\isactrlsub i{\isacharbraceright}{\kern0pt}{\isachardoublequoteclose}\ \isanewline
\ \ \ \ \isacommand{unfolding}\isamarkupfalse%
\ mirror{\isadigit{2}}{\isacharunderscore}{\kern0pt}board{\isacharunderscore}{\kern0pt}def\ \isacommand{by}\isamarkupfalse%
\ blast\isanewline
\ \ \isacommand{then}\isamarkupfalse%
\ \isacommand{show}\isamarkupfalse%
\ {\isacharquery}{\kern0pt}case\ \isacommand{by}\isamarkupfalse%
\ {\isacharparenleft}{\kern0pt}auto\ intro{\isacharcolon}{\kern0pt}\ knights{\isacharunderscore}{\kern0pt}path{\isachardot}{\kern0pt}intros{\isacharparenright}{\kern0pt}\isanewline
\isacommand{next}\isamarkupfalse%
\isanewline
\ \ \isacommand{case}\isamarkupfalse%
\ {\isacharparenleft}{\kern0pt}{\isadigit{2}}\ s\isactrlsub i\ b\ s\isactrlsub j\ ps{\isacharparenright}{\kern0pt}\isanewline
\ \ \isacommand{then}\isamarkupfalse%
\ \isacommand{have}\isamarkupfalse%
\ prems{\isacharcolon}{\kern0pt}\ {\isachardoublequoteopen}mirror{\isadigit{2}}{\isacharunderscore}{\kern0pt}square\ n\ s\isactrlsub i\ {\isasymnotin}\ mirror{\isadigit{2}}{\isacharunderscore}{\kern0pt}board\ n\ b{\isachardoublequoteclose}\ \isanewline
\ \ \ \ \ \ \ \ \ \ \ \ {\isachardoublequoteopen}valid{\isacharunderscore}{\kern0pt}step\ {\isacharparenleft}{\kern0pt}mirror{\isadigit{2}}{\isacharunderscore}{\kern0pt}square\ n\ s\isactrlsub i{\isacharparenright}{\kern0pt}\ {\isacharparenleft}{\kern0pt}mirror{\isadigit{2}}{\isacharunderscore}{\kern0pt}square\ n\ s\isactrlsub j{\isacharparenright}{\kern0pt}{\isachardoublequoteclose}\ \isanewline
\ \ \ \ \ \ \ \ \ \ \ \ \isakeyword{and}\ {\isachardoublequoteopen}mirror{\isadigit{2}}{\isacharunderscore}{\kern0pt}aux\ n\ {\isacharparenleft}{\kern0pt}s\isactrlsub j{\isacharhash}{\kern0pt}ps{\isacharparenright}{\kern0pt}\ {\isacharequal}{\kern0pt}\ mirror{\isadigit{2}}{\isacharunderscore}{\kern0pt}square\ n\ s\isactrlsub j{\isacharhash}{\kern0pt}mirror{\isadigit{2}}{\isacharunderscore}{\kern0pt}aux\ n\ ps{\isachardoublequoteclose}\isanewline
\ \ \ \ \isacommand{using}\isamarkupfalse%
\ {\isadigit{2}}\ mirror{\isadigit{2}}{\isacharunderscore}{\kern0pt}board{\isacharunderscore}{\kern0pt}iff\ valid{\isacharunderscore}{\kern0pt}step{\isacharunderscore}{\kern0pt}mirror{\isadigit{2}}\ \isacommand{by}\isamarkupfalse%
\ auto\isanewline
\ \ \isacommand{then}\isamarkupfalse%
\ \isacommand{show}\isamarkupfalse%
\ {\isacharquery}{\kern0pt}case\ \isanewline
\ \ \ \ \isacommand{using}\isamarkupfalse%
\ {\isadigit{2}}\ knights{\isacharunderscore}{\kern0pt}path{\isachardot}{\kern0pt}intros{\isacharparenleft}{\kern0pt}{\isadigit{2}}{\isacharparenright}{\kern0pt}{\isacharbrackleft}{\kern0pt}OF\ prems{\isacharbrackright}{\kern0pt}\ insert{\isacharunderscore}{\kern0pt}mirror{\isadigit{2}}{\isacharunderscore}{\kern0pt}board\ \isacommand{by}\isamarkupfalse%
\ auto\isanewline
\isacommand{qed}\isamarkupfalse%
%
\endisatagproof
{\isafoldproof}%
%
\isadelimproof
\isanewline
%
\endisadelimproof
\isanewline
\isacommand{corollary}\isamarkupfalse%
\ mirror{\isadigit{2}}{\isacharunderscore}{\kern0pt}knights{\isacharunderscore}{\kern0pt}path{\isacharcolon}{\kern0pt}\isanewline
\ \ \isakeyword{assumes}\ {\isachardoublequoteopen}knights{\isacharunderscore}{\kern0pt}path\ {\isacharparenleft}{\kern0pt}board\ n\ m{\isacharparenright}{\kern0pt}\ ps{\isachardoublequoteclose}\ \isanewline
\ \ \isakeyword{shows}\ {\isachardoublequoteopen}knights{\isacharunderscore}{\kern0pt}path\ {\isacharparenleft}{\kern0pt}board\ n\ m{\isacharparenright}{\kern0pt}\ {\isacharparenleft}{\kern0pt}mirror{\isadigit{2}}\ ps{\isacharparenright}{\kern0pt}{\isachardoublequoteclose}\isanewline
%
\isadelimproof
%
\endisadelimproof
%
\isatagproof
\isacommand{proof}\isamarkupfalse%
\ {\isacharminus}{\kern0pt}\isanewline
\ \ \isacommand{have}\isamarkupfalse%
\ {\isacharbrackleft}{\kern0pt}simp{\isacharbrackright}{\kern0pt}{\isacharcolon}{\kern0pt}\ {\isachardoublequoteopen}min{\isadigit{2}}\ ps\ {\isacharequal}{\kern0pt}\ {\isadigit{1}}{\isachardoublequoteclose}\ {\isachardoublequoteopen}max{\isadigit{2}}\ ps\ {\isacharequal}{\kern0pt}\ int\ m{\isachardoublequoteclose}\isanewline
\ \ \ \ \isacommand{using}\isamarkupfalse%
\ assms\ knights{\isacharunderscore}{\kern0pt}path{\isacharunderscore}{\kern0pt}min{\isadigit{2}}\ knights{\isacharunderscore}{\kern0pt}path{\isacharunderscore}{\kern0pt}max{\isadigit{2}}\ \isacommand{by}\isamarkupfalse%
\ auto\isanewline
\ \ \isacommand{then}\isamarkupfalse%
\ \isacommand{have}\isamarkupfalse%
\ {\isachardoublequoteopen}mirror{\isadigit{2}}{\isacharunderscore}{\kern0pt}board\ {\isacharparenleft}{\kern0pt}int\ m{\isacharplus}{\kern0pt}{\isadigit{1}}{\isacharparenright}{\kern0pt}\ {\isacharparenleft}{\kern0pt}board\ n\ m{\isacharparenright}{\kern0pt}\ {\isacharequal}{\kern0pt}\ {\isacharparenleft}{\kern0pt}board\ n\ m{\isacharparenright}{\kern0pt}{\isachardoublequoteclose}\isanewline
\ \ \ \ \isacommand{using}\isamarkupfalse%
\ mirror{\isadigit{2}}{\isacharunderscore}{\kern0pt}board{\isacharunderscore}{\kern0pt}id\ \isacommand{by}\isamarkupfalse%
\ auto\isanewline
\ \ \isacommand{then}\isamarkupfalse%
\ \isacommand{have}\isamarkupfalse%
\ {\isachardoublequoteopen}knights{\isacharunderscore}{\kern0pt}path\ {\isacharparenleft}{\kern0pt}board\ n\ m{\isacharparenright}{\kern0pt}\ {\isacharparenleft}{\kern0pt}mirror{\isadigit{2}}{\isacharunderscore}{\kern0pt}aux\ {\isacharparenleft}{\kern0pt}int\ m{\isacharplus}{\kern0pt}{\isadigit{1}}{\isacharparenright}{\kern0pt}\ ps{\isacharparenright}{\kern0pt}{\isachardoublequoteclose}\isanewline
\ \ \ \ \isacommand{using}\isamarkupfalse%
\ assms\ mirror{\isadigit{2}}{\isacharunderscore}{\kern0pt}aux{\isacharunderscore}{\kern0pt}knights{\isacharunderscore}{\kern0pt}path{\isacharbrackleft}{\kern0pt}of\ {\isachardoublequoteopen}board\ n\ m{\isachardoublequoteclose}\ ps\ {\isachardoublequoteopen}int\ m{\isacharplus}{\kern0pt}{\isadigit{1}}{\isachardoublequoteclose}{\isacharbrackright}{\kern0pt}\ \isacommand{by}\isamarkupfalse%
\ auto\isanewline
\ \ \isacommand{then}\isamarkupfalse%
\ \isacommand{show}\isamarkupfalse%
\ {\isacharquery}{\kern0pt}thesis\ \isacommand{unfolding}\isamarkupfalse%
\ mirror{\isadigit{2}}{\isacharunderscore}{\kern0pt}def\ \isacommand{by}\isamarkupfalse%
\ auto\isanewline
\isacommand{qed}\isamarkupfalse%
%
\endisatagproof
{\isafoldproof}%
%
\isadelimproof
%
\endisadelimproof
%
\isadelimdocument
%
\endisadelimdocument
%
\isatagdocument
%
\isamarkupsubsection{Rotate Knight's Paths%
}
\isamarkuptrue%
%
\endisatagdocument
{\isafolddocument}%
%
\isadelimdocument
%
\endisadelimdocument
%
\begin{isamarkuptext}%
Transposing (\isa{transpose}) and mirroring (along first axis \isa{mirror{\isadigit{1}}}) a Knight's path 
preserves the Knight's path's property. Tranpose+Mirror1 equals a 90deg-clockwise turn.%
\end{isamarkuptext}\isamarkuptrue%
\isacommand{corollary}\isamarkupfalse%
\ rot{\isadigit{9}}{\isadigit{0}}{\isacharunderscore}{\kern0pt}knights{\isacharunderscore}{\kern0pt}path{\isacharcolon}{\kern0pt}\isanewline
\ \ \isakeyword{assumes}\ {\isachardoublequoteopen}knights{\isacharunderscore}{\kern0pt}path\ {\isacharparenleft}{\kern0pt}board\ n\ m{\isacharparenright}{\kern0pt}\ ps{\isachardoublequoteclose}\ \isanewline
\ \ \isakeyword{shows}\ {\isachardoublequoteopen}knights{\isacharunderscore}{\kern0pt}path\ {\isacharparenleft}{\kern0pt}board\ m\ n{\isacharparenright}{\kern0pt}\ {\isacharparenleft}{\kern0pt}mirror{\isadigit{1}}\ {\isacharparenleft}{\kern0pt}transpose\ ps{\isacharparenright}{\kern0pt}{\isacharparenright}{\kern0pt}{\isachardoublequoteclose}\isanewline
%
\isadelimproof
\ \ %
\endisadelimproof
%
\isatagproof
\isacommand{using}\isamarkupfalse%
\ assms\ transpose{\isacharunderscore}{\kern0pt}knights{\isacharunderscore}{\kern0pt}path\ mirror{\isadigit{1}}{\isacharunderscore}{\kern0pt}knights{\isacharunderscore}{\kern0pt}path\ \isacommand{by}\isamarkupfalse%
\ auto%
\endisatagproof
{\isafoldproof}%
%
\isadelimproof
\isanewline
%
\endisadelimproof
\isanewline
\isacommand{lemma}\isamarkupfalse%
\ hd{\isacharunderscore}{\kern0pt}rot{\isadigit{9}}{\isadigit{0}}{\isacharunderscore}{\kern0pt}knights{\isacharunderscore}{\kern0pt}path{\isacharcolon}{\kern0pt}\ \isanewline
\ \ \isakeyword{assumes}\ {\isachardoublequoteopen}knights{\isacharunderscore}{\kern0pt}path\ {\isacharparenleft}{\kern0pt}board\ n\ m{\isacharparenright}{\kern0pt}\ ps{\isachardoublequoteclose}\ {\isachardoublequoteopen}hd\ ps\ {\isacharequal}{\kern0pt}\ {\isacharparenleft}{\kern0pt}i{\isacharcomma}{\kern0pt}j{\isacharparenright}{\kern0pt}{\isachardoublequoteclose}\isanewline
\ \ \isakeyword{shows}\ {\isachardoublequoteopen}hd\ {\isacharparenleft}{\kern0pt}mirror{\isadigit{1}}\ {\isacharparenleft}{\kern0pt}transpose\ ps{\isacharparenright}{\kern0pt}{\isacharparenright}{\kern0pt}\ {\isacharequal}{\kern0pt}\ {\isacharparenleft}{\kern0pt}int\ m{\isacharplus}{\kern0pt}{\isadigit{1}}{\isacharminus}{\kern0pt}j{\isacharcomma}{\kern0pt}i{\isacharparenright}{\kern0pt}{\isachardoublequoteclose}\isanewline
%
\isadelimproof
\ \ %
\endisadelimproof
%
\isatagproof
\isacommand{using}\isamarkupfalse%
\ assms\isanewline
\isacommand{proof}\isamarkupfalse%
\ {\isacharminus}{\kern0pt}\isanewline
\ \ \isacommand{have}\isamarkupfalse%
\ {\isachardoublequoteopen}hd\ {\isacharparenleft}{\kern0pt}transpose\ ps{\isacharparenright}{\kern0pt}\ {\isacharequal}{\kern0pt}\ {\isacharparenleft}{\kern0pt}j{\isacharcomma}{\kern0pt}i{\isacharparenright}{\kern0pt}{\isachardoublequoteclose}\ {\isachardoublequoteopen}knights{\isacharunderscore}{\kern0pt}path\ {\isacharparenleft}{\kern0pt}board\ m\ n{\isacharparenright}{\kern0pt}\ {\isacharparenleft}{\kern0pt}transpose\ ps{\isacharparenright}{\kern0pt}{\isachardoublequoteclose}\isanewline
\ \ \ \ \isacommand{using}\isamarkupfalse%
\ assms\ knights{\isacharunderscore}{\kern0pt}path{\isacharunderscore}{\kern0pt}non{\isacharunderscore}{\kern0pt}nil\ hd{\isacharunderscore}{\kern0pt}transpose\ transpose{\isacharunderscore}{\kern0pt}knights{\isacharunderscore}{\kern0pt}path\ \isanewline
\ \ \ \ \isacommand{by}\isamarkupfalse%
\ {\isacharparenleft}{\kern0pt}auto\ simp{\isacharcolon}{\kern0pt}\ transpose{\isacharunderscore}{\kern0pt}square{\isacharunderscore}{\kern0pt}def{\isacharparenright}{\kern0pt}\isanewline
\ \ \isacommand{then}\isamarkupfalse%
\ \isacommand{show}\isamarkupfalse%
\ {\isacharquery}{\kern0pt}thesis\ \isacommand{using}\isamarkupfalse%
\ hd{\isacharunderscore}{\kern0pt}mirror{\isadigit{1}}\ \isacommand{by}\isamarkupfalse%
\ auto\isanewline
\isacommand{qed}\isamarkupfalse%
%
\endisatagproof
{\isafoldproof}%
%
\isadelimproof
\isanewline
%
\endisadelimproof
\isanewline
\isacommand{lemma}\isamarkupfalse%
\ last{\isacharunderscore}{\kern0pt}rot{\isadigit{9}}{\isadigit{0}}{\isacharunderscore}{\kern0pt}knights{\isacharunderscore}{\kern0pt}path{\isacharcolon}{\kern0pt}\ \isanewline
\ \ \isakeyword{assumes}\ {\isachardoublequoteopen}knights{\isacharunderscore}{\kern0pt}path\ {\isacharparenleft}{\kern0pt}board\ n\ m{\isacharparenright}{\kern0pt}\ ps{\isachardoublequoteclose}\ {\isachardoublequoteopen}last\ ps\ {\isacharequal}{\kern0pt}\ {\isacharparenleft}{\kern0pt}i{\isacharcomma}{\kern0pt}j{\isacharparenright}{\kern0pt}{\isachardoublequoteclose}\isanewline
\ \ \isakeyword{shows}\ {\isachardoublequoteopen}last\ {\isacharparenleft}{\kern0pt}mirror{\isadigit{1}}\ {\isacharparenleft}{\kern0pt}transpose\ ps{\isacharparenright}{\kern0pt}{\isacharparenright}{\kern0pt}\ {\isacharequal}{\kern0pt}\ {\isacharparenleft}{\kern0pt}int\ m{\isacharplus}{\kern0pt}{\isadigit{1}}{\isacharminus}{\kern0pt}j{\isacharcomma}{\kern0pt}i{\isacharparenright}{\kern0pt}{\isachardoublequoteclose}\isanewline
%
\isadelimproof
\ \ %
\endisadelimproof
%
\isatagproof
\isacommand{using}\isamarkupfalse%
\ assms\isanewline
\isacommand{proof}\isamarkupfalse%
\ {\isacharminus}{\kern0pt}\isanewline
\ \ \isacommand{have}\isamarkupfalse%
\ {\isachardoublequoteopen}last\ {\isacharparenleft}{\kern0pt}transpose\ ps{\isacharparenright}{\kern0pt}\ {\isacharequal}{\kern0pt}\ {\isacharparenleft}{\kern0pt}j{\isacharcomma}{\kern0pt}i{\isacharparenright}{\kern0pt}{\isachardoublequoteclose}\ {\isachardoublequoteopen}knights{\isacharunderscore}{\kern0pt}path\ {\isacharparenleft}{\kern0pt}board\ m\ n{\isacharparenright}{\kern0pt}\ {\isacharparenleft}{\kern0pt}transpose\ ps{\isacharparenright}{\kern0pt}{\isachardoublequoteclose}\isanewline
\ \ \ \ \isacommand{using}\isamarkupfalse%
\ assms\ knights{\isacharunderscore}{\kern0pt}path{\isacharunderscore}{\kern0pt}non{\isacharunderscore}{\kern0pt}nil\ last{\isacharunderscore}{\kern0pt}transpose\ transpose{\isacharunderscore}{\kern0pt}knights{\isacharunderscore}{\kern0pt}path\ \isanewline
\ \ \ \ \isacommand{by}\isamarkupfalse%
\ {\isacharparenleft}{\kern0pt}auto\ simp{\isacharcolon}{\kern0pt}\ transpose{\isacharunderscore}{\kern0pt}square{\isacharunderscore}{\kern0pt}def{\isacharparenright}{\kern0pt}\isanewline
\ \ \isacommand{then}\isamarkupfalse%
\ \isacommand{show}\isamarkupfalse%
\ {\isacharquery}{\kern0pt}thesis\ \isacommand{using}\isamarkupfalse%
\ last{\isacharunderscore}{\kern0pt}mirror{\isadigit{1}}\ \isacommand{by}\isamarkupfalse%
\ auto\isanewline
\isacommand{qed}\isamarkupfalse%
%
\endisatagproof
{\isafoldproof}%
%
\isadelimproof
%
\endisadelimproof
%
\isadelimdocument
%
\endisadelimdocument
%
\isatagdocument
%
\isamarkupsection{Translating Paths and Boards%
}
\isamarkuptrue%
%
\endisatagdocument
{\isafolddocument}%
%
\isadelimdocument
%
\endisadelimdocument
%
\begin{isamarkuptext}%
When constructing knight's paths for larger boards multiple knight's paths for smaller boards
are concatenated. To concatenate paths the the coordinates in the path need to be translated. 
Therefore, simple auxiliary functions are provided.%
\end{isamarkuptext}\isamarkuptrue%
%
\isadelimdocument
%
\endisadelimdocument
%
\isatagdocument
%
\isamarkupsubsection{Implementation of Path and Board Translation%
}
\isamarkuptrue%
%
\endisatagdocument
{\isafolddocument}%
%
\isadelimdocument
%
\endisadelimdocument
%
\begin{isamarkuptext}%
Translate the coordinates for a path by \isa{{\isacharparenleft}{\kern0pt}k\isactrlsub {\isadigit{1}}{\isacharcomma}{\kern0pt}k\isactrlsub {\isadigit{2}}{\isacharparenright}{\kern0pt}}.%
\end{isamarkuptext}\isamarkuptrue%
\isacommand{fun}\isamarkupfalse%
\ trans{\isacharunderscore}{\kern0pt}path\ {\isacharcolon}{\kern0pt}{\isacharcolon}{\kern0pt}\ {\isachardoublequoteopen}int\ {\isasymtimes}\ int\ {\isasymRightarrow}\ path\ {\isasymRightarrow}\ path{\isachardoublequoteclose}\ \isakeyword{where}\isanewline
\ \ {\isachardoublequoteopen}trans{\isacharunderscore}{\kern0pt}path\ {\isacharparenleft}{\kern0pt}k\isactrlsub {\isadigit{1}}{\isacharcomma}{\kern0pt}k\isactrlsub {\isadigit{2}}{\isacharparenright}{\kern0pt}\ {\isacharbrackleft}{\kern0pt}{\isacharbrackright}{\kern0pt}\ {\isacharequal}{\kern0pt}\ {\isacharbrackleft}{\kern0pt}{\isacharbrackright}{\kern0pt}{\isachardoublequoteclose}\isanewline
{\isacharbar}{\kern0pt}\ {\isachardoublequoteopen}trans{\isacharunderscore}{\kern0pt}path\ {\isacharparenleft}{\kern0pt}k\isactrlsub {\isadigit{1}}{\isacharcomma}{\kern0pt}k\isactrlsub {\isadigit{2}}{\isacharparenright}{\kern0pt}\ {\isacharparenleft}{\kern0pt}{\isacharparenleft}{\kern0pt}i{\isacharcomma}{\kern0pt}j{\isacharparenright}{\kern0pt}{\isacharhash}{\kern0pt}xs{\isacharparenright}{\kern0pt}\ {\isacharequal}{\kern0pt}\ {\isacharparenleft}{\kern0pt}i{\isacharplus}{\kern0pt}k\isactrlsub {\isadigit{1}}{\isacharcomma}{\kern0pt}j{\isacharplus}{\kern0pt}k\isactrlsub {\isadigit{2}}{\isacharparenright}{\kern0pt}{\isacharhash}{\kern0pt}{\isacharparenleft}{\kern0pt}trans{\isacharunderscore}{\kern0pt}path\ {\isacharparenleft}{\kern0pt}k\isactrlsub {\isadigit{1}}{\isacharcomma}{\kern0pt}k\isactrlsub {\isadigit{2}}{\isacharparenright}{\kern0pt}\ xs{\isacharparenright}{\kern0pt}{\isachardoublequoteclose}%
\begin{isamarkuptext}%
Translate the coordinates of a board by \isa{{\isacharparenleft}{\kern0pt}k\isactrlsub {\isadigit{1}}{\isacharcomma}{\kern0pt}k\isactrlsub {\isadigit{2}}{\isacharparenright}{\kern0pt}}.%
\end{isamarkuptext}\isamarkuptrue%
\isacommand{definition}\isamarkupfalse%
\ trans{\isacharunderscore}{\kern0pt}board\ {\isacharcolon}{\kern0pt}{\isacharcolon}{\kern0pt}\ {\isachardoublequoteopen}int\ {\isasymtimes}\ int\ {\isasymRightarrow}\ board\ {\isasymRightarrow}\ board{\isachardoublequoteclose}\ \isakeyword{where}\ \isanewline
\ \ {\isachardoublequoteopen}trans{\isacharunderscore}{\kern0pt}board\ t\ b\ {\isasymequiv}\ {\isacharparenleft}{\kern0pt}case\ t\ of\ {\isacharparenleft}{\kern0pt}k\isactrlsub {\isadigit{1}}{\isacharcomma}{\kern0pt}k\isactrlsub {\isadigit{2}}{\isacharparenright}{\kern0pt}\ {\isasymRightarrow}\ {\isacharbraceleft}{\kern0pt}{\isacharparenleft}{\kern0pt}i{\isacharplus}{\kern0pt}k\isactrlsub {\isadigit{1}}{\isacharcomma}{\kern0pt}j{\isacharplus}{\kern0pt}k\isactrlsub {\isadigit{2}}{\isacharparenright}{\kern0pt}{\isacharbar}{\kern0pt}i\ j{\isachardot}{\kern0pt}\ {\isacharparenleft}{\kern0pt}i{\isacharcomma}{\kern0pt}j{\isacharparenright}{\kern0pt}\ {\isasymin}\ b{\isacharbraceright}{\kern0pt}{\isacharparenright}{\kern0pt}{\isachardoublequoteclose}%
\isadelimdocument
%
\endisadelimdocument
%
\isatagdocument
%
\isamarkupsubsection{Correctness of Path and Board Translation%
}
\isamarkuptrue%
%
\endisatagdocument
{\isafolddocument}%
%
\isadelimdocument
%
\endisadelimdocument
\isacommand{lemma}\isamarkupfalse%
\ trans{\isacharunderscore}{\kern0pt}path{\isacharunderscore}{\kern0pt}length{\isacharcolon}{\kern0pt}\ {\isachardoublequoteopen}length\ ps\ {\isacharequal}{\kern0pt}\ length\ {\isacharparenleft}{\kern0pt}trans{\isacharunderscore}{\kern0pt}path\ {\isacharparenleft}{\kern0pt}k\isactrlsub {\isadigit{1}}{\isacharcomma}{\kern0pt}k\isactrlsub {\isadigit{2}}{\isacharparenright}{\kern0pt}\ ps{\isacharparenright}{\kern0pt}{\isachardoublequoteclose}\isanewline
%
\isadelimproof
\ \ %
\endisadelimproof
%
\isatagproof
\isacommand{by}\isamarkupfalse%
\ {\isacharparenleft}{\kern0pt}induction\ ps{\isacharparenright}{\kern0pt}\ auto%
\endisatagproof
{\isafoldproof}%
%
\isadelimproof
\isanewline
%
\endisadelimproof
\isanewline
\isacommand{lemma}\isamarkupfalse%
\ trans{\isacharunderscore}{\kern0pt}path{\isacharunderscore}{\kern0pt}non{\isacharunderscore}{\kern0pt}nil{\isacharcolon}{\kern0pt}\ {\isachardoublequoteopen}ps\ {\isasymnoteq}\ {\isacharbrackleft}{\kern0pt}{\isacharbrackright}{\kern0pt}\ {\isasymLongrightarrow}\ trans{\isacharunderscore}{\kern0pt}path\ {\isacharparenleft}{\kern0pt}k\isactrlsub {\isadigit{1}}{\isacharcomma}{\kern0pt}k\isactrlsub {\isadigit{2}}{\isacharparenright}{\kern0pt}\ ps\ {\isasymnoteq}\ {\isacharbrackleft}{\kern0pt}{\isacharbrackright}{\kern0pt}{\isachardoublequoteclose}\isanewline
%
\isadelimproof
\ \ %
\endisadelimproof
%
\isatagproof
\isacommand{by}\isamarkupfalse%
\ {\isacharparenleft}{\kern0pt}induction\ ps{\isacharparenright}{\kern0pt}\ auto%
\endisatagproof
{\isafoldproof}%
%
\isadelimproof
\isanewline
%
\endisadelimproof
\isanewline
\isacommand{lemma}\isamarkupfalse%
\ trans{\isacharunderscore}{\kern0pt}path{\isacharunderscore}{\kern0pt}correct{\isacharcolon}{\kern0pt}\ {\isachardoublequoteopen}{\isacharparenleft}{\kern0pt}i{\isacharcomma}{\kern0pt}j{\isacharparenright}{\kern0pt}\ {\isasymin}\ set\ ps\ {\isasymlongleftrightarrow}\ {\isacharparenleft}{\kern0pt}i{\isacharplus}{\kern0pt}k\isactrlsub {\isadigit{1}}{\isacharcomma}{\kern0pt}j{\isacharplus}{\kern0pt}k\isactrlsub {\isadigit{2}}{\isacharparenright}{\kern0pt}\ {\isasymin}\ set\ {\isacharparenleft}{\kern0pt}trans{\isacharunderscore}{\kern0pt}path\ {\isacharparenleft}{\kern0pt}k\isactrlsub {\isadigit{1}}{\isacharcomma}{\kern0pt}k\isactrlsub {\isadigit{2}}{\isacharparenright}{\kern0pt}\ ps{\isacharparenright}{\kern0pt}{\isachardoublequoteclose}\isanewline
%
\isadelimproof
%
\endisadelimproof
%
\isatagproof
\isacommand{proof}\isamarkupfalse%
\ {\isacharparenleft}{\kern0pt}induction\ ps{\isacharparenright}{\kern0pt}\isanewline
\ \ \isacommand{case}\isamarkupfalse%
\ {\isacharparenleft}{\kern0pt}Cons\ s\isactrlsub i\ ps{\isacharparenright}{\kern0pt}\isanewline
\ \ \isacommand{then}\isamarkupfalse%
\ \isacommand{show}\isamarkupfalse%
\ {\isacharquery}{\kern0pt}case\ \isacommand{by}\isamarkupfalse%
\ {\isacharparenleft}{\kern0pt}cases\ s\isactrlsub i{\isacharparenright}{\kern0pt}\ auto\isanewline
\isacommand{qed}\isamarkupfalse%
\ auto%
\endisatagproof
{\isafoldproof}%
%
\isadelimproof
\isanewline
%
\endisadelimproof
\isanewline
\isacommand{lemma}\isamarkupfalse%
\ trans{\isacharunderscore}{\kern0pt}path{\isacharunderscore}{\kern0pt}non{\isacharunderscore}{\kern0pt}nil{\isacharunderscore}{\kern0pt}last{\isacharcolon}{\kern0pt}\ \isanewline
\ \ {\isachardoublequoteopen}ps\ {\isasymnoteq}\ {\isacharbrackleft}{\kern0pt}{\isacharbrackright}{\kern0pt}\ {\isasymLongrightarrow}\ last\ {\isacharparenleft}{\kern0pt}trans{\isacharunderscore}{\kern0pt}path\ {\isacharparenleft}{\kern0pt}k\isactrlsub {\isadigit{1}}{\isacharcomma}{\kern0pt}k\isactrlsub {\isadigit{2}}{\isacharparenright}{\kern0pt}\ ps{\isacharparenright}{\kern0pt}\ {\isacharequal}{\kern0pt}\ last\ {\isacharparenleft}{\kern0pt}trans{\isacharunderscore}{\kern0pt}path\ {\isacharparenleft}{\kern0pt}k\isactrlsub {\isadigit{1}}{\isacharcomma}{\kern0pt}k\isactrlsub {\isadigit{2}}{\isacharparenright}{\kern0pt}\ {\isacharparenleft}{\kern0pt}{\isacharparenleft}{\kern0pt}i{\isacharcomma}{\kern0pt}j{\isacharparenright}{\kern0pt}{\isacharhash}{\kern0pt}ps{\isacharparenright}{\kern0pt}{\isacharparenright}{\kern0pt}{\isachardoublequoteclose}\isanewline
%
\isadelimproof
\ \ %
\endisadelimproof
%
\isatagproof
\isacommand{using}\isamarkupfalse%
\ trans{\isacharunderscore}{\kern0pt}path{\isacharunderscore}{\kern0pt}non{\isacharunderscore}{\kern0pt}nil\ \isacommand{by}\isamarkupfalse%
\ {\isacharparenleft}{\kern0pt}induction\ ps{\isacharparenright}{\kern0pt}\ auto%
\endisatagproof
{\isafoldproof}%
%
\isadelimproof
\isanewline
%
\endisadelimproof
\isanewline
\isacommand{lemma}\isamarkupfalse%
\ hd{\isacharunderscore}{\kern0pt}trans{\isacharunderscore}{\kern0pt}path{\isacharcolon}{\kern0pt}\isanewline
\ \ \isakeyword{assumes}\ {\isachardoublequoteopen}ps\ {\isasymnoteq}\ {\isacharbrackleft}{\kern0pt}{\isacharbrackright}{\kern0pt}{\isachardoublequoteclose}\ {\isachardoublequoteopen}hd\ ps\ {\isacharequal}{\kern0pt}\ {\isacharparenleft}{\kern0pt}i{\isacharcomma}{\kern0pt}j{\isacharparenright}{\kern0pt}{\isachardoublequoteclose}\isanewline
\ \ \isakeyword{shows}\ {\isachardoublequoteopen}hd\ {\isacharparenleft}{\kern0pt}trans{\isacharunderscore}{\kern0pt}path\ {\isacharparenleft}{\kern0pt}k\isactrlsub {\isadigit{1}}{\isacharcomma}{\kern0pt}k\isactrlsub {\isadigit{2}}{\isacharparenright}{\kern0pt}\ ps{\isacharparenright}{\kern0pt}\ {\isacharequal}{\kern0pt}\ {\isacharparenleft}{\kern0pt}i{\isacharplus}{\kern0pt}k\isactrlsub {\isadigit{1}}{\isacharcomma}{\kern0pt}j{\isacharplus}{\kern0pt}k\isactrlsub {\isadigit{2}}{\isacharparenright}{\kern0pt}{\isachardoublequoteclose}\isanewline
%
\isadelimproof
\ \ %
\endisadelimproof
%
\isatagproof
\isacommand{using}\isamarkupfalse%
\ assms\ \isacommand{by}\isamarkupfalse%
\ {\isacharparenleft}{\kern0pt}induction\ ps{\isacharparenright}{\kern0pt}\ auto%
\endisatagproof
{\isafoldproof}%
%
\isadelimproof
\isanewline
%
\endisadelimproof
\isanewline
\isacommand{lemma}\isamarkupfalse%
\ last{\isacharunderscore}{\kern0pt}trans{\isacharunderscore}{\kern0pt}path{\isacharcolon}{\kern0pt}\isanewline
\ \ \isakeyword{assumes}\ {\isachardoublequoteopen}ps\ {\isasymnoteq}\ {\isacharbrackleft}{\kern0pt}{\isacharbrackright}{\kern0pt}{\isachardoublequoteclose}\ {\isachardoublequoteopen}last\ ps\ {\isacharequal}{\kern0pt}\ {\isacharparenleft}{\kern0pt}i{\isacharcomma}{\kern0pt}j{\isacharparenright}{\kern0pt}{\isachardoublequoteclose}\isanewline
\ \ \isakeyword{shows}\ {\isachardoublequoteopen}last\ {\isacharparenleft}{\kern0pt}trans{\isacharunderscore}{\kern0pt}path\ {\isacharparenleft}{\kern0pt}k\isactrlsub {\isadigit{1}}{\isacharcomma}{\kern0pt}k\isactrlsub {\isadigit{2}}{\isacharparenright}{\kern0pt}\ ps{\isacharparenright}{\kern0pt}\ {\isacharequal}{\kern0pt}\ {\isacharparenleft}{\kern0pt}i{\isacharplus}{\kern0pt}k\isactrlsub {\isadigit{1}}{\isacharcomma}{\kern0pt}j{\isacharplus}{\kern0pt}k\isactrlsub {\isadigit{2}}{\isacharparenright}{\kern0pt}{\isachardoublequoteclose}\isanewline
%
\isadelimproof
\ \ %
\endisadelimproof
%
\isatagproof
\isacommand{using}\isamarkupfalse%
\ assms\isanewline
\isacommand{proof}\isamarkupfalse%
\ {\isacharparenleft}{\kern0pt}induction\ ps{\isacharparenright}{\kern0pt}\isanewline
\ \ \isacommand{case}\isamarkupfalse%
\ {\isacharparenleft}{\kern0pt}Cons\ s\isactrlsub i\ ps{\isacharparenright}{\kern0pt}\isanewline
\ \ \isacommand{then}\isamarkupfalse%
\ \isacommand{show}\isamarkupfalse%
\ {\isacharquery}{\kern0pt}case\ \isanewline
\ \ \ \ \isacommand{using}\isamarkupfalse%
\ trans{\isacharunderscore}{\kern0pt}path{\isacharunderscore}{\kern0pt}non{\isacharunderscore}{\kern0pt}nil{\isacharunderscore}{\kern0pt}last{\isacharbrackleft}{\kern0pt}symmetric{\isacharbrackright}{\kern0pt}\ \isanewline
\ \ \ \ \isacommand{apply}\isamarkupfalse%
\ {\isacharparenleft}{\kern0pt}cases\ s\isactrlsub i{\isacharparenright}{\kern0pt}\ \isanewline
\ \ \ \ \isacommand{apply}\isamarkupfalse%
\ {\isacharparenleft}{\kern0pt}cases\ {\isachardoublequoteopen}ps\ {\isacharequal}{\kern0pt}\ {\isacharbrackleft}{\kern0pt}{\isacharbrackright}{\kern0pt}{\isachardoublequoteclose}{\isacharparenright}{\kern0pt}\isanewline
\ \ \ \ \isacommand{apply}\isamarkupfalse%
\ auto\isanewline
\ \ \ \ \isacommand{done}\isamarkupfalse%
\isanewline
\isacommand{qed}\isamarkupfalse%
\ {\isacharparenleft}{\kern0pt}auto{\isacharparenright}{\kern0pt}%
\endisatagproof
{\isafoldproof}%
%
\isadelimproof
\isanewline
%
\endisadelimproof
\isanewline
\isacommand{lemma}\isamarkupfalse%
\ take{\isacharunderscore}{\kern0pt}trans{\isacharcolon}{\kern0pt}\ \isanewline
\ \ \isakeyword{shows}\ {\isachardoublequoteopen}take\ k\ {\isacharparenleft}{\kern0pt}trans{\isacharunderscore}{\kern0pt}path\ {\isacharparenleft}{\kern0pt}k\isactrlsub {\isadigit{1}}{\isacharcomma}{\kern0pt}k\isactrlsub {\isadigit{2}}{\isacharparenright}{\kern0pt}\ ps{\isacharparenright}{\kern0pt}\ {\isacharequal}{\kern0pt}\ trans{\isacharunderscore}{\kern0pt}path\ {\isacharparenleft}{\kern0pt}k\isactrlsub {\isadigit{1}}{\isacharcomma}{\kern0pt}k\isactrlsub {\isadigit{2}}{\isacharparenright}{\kern0pt}\ {\isacharparenleft}{\kern0pt}take\ k\ ps{\isacharparenright}{\kern0pt}{\isachardoublequoteclose}\isanewline
%
\isadelimproof
%
\endisadelimproof
%
\isatagproof
\isacommand{proof}\isamarkupfalse%
\ {\isacharparenleft}{\kern0pt}induction\ ps\ arbitrary{\isacharcolon}{\kern0pt}\ k{\isacharparenright}{\kern0pt}\isanewline
\ \ \isacommand{case}\isamarkupfalse%
\ Nil\isanewline
\ \ \isacommand{then}\isamarkupfalse%
\ \isacommand{show}\isamarkupfalse%
\ {\isacharquery}{\kern0pt}case\ \isacommand{by}\isamarkupfalse%
\ auto\isanewline
\isacommand{next}\isamarkupfalse%
\isanewline
\ \ \isacommand{case}\isamarkupfalse%
\ {\isacharparenleft}{\kern0pt}Cons\ s\isactrlsub i\ ps{\isacharparenright}{\kern0pt}\isanewline
\ \ \isacommand{then}\isamarkupfalse%
\ \isacommand{obtain}\isamarkupfalse%
\ i\ j\ \isakeyword{where}\ {\isachardoublequoteopen}s\isactrlsub i\ {\isacharequal}{\kern0pt}\ {\isacharparenleft}{\kern0pt}i{\isacharcomma}{\kern0pt}j{\isacharparenright}{\kern0pt}{\isachardoublequoteclose}\ \isacommand{by}\isamarkupfalse%
\ force\isanewline
\ \ \isacommand{then}\isamarkupfalse%
\ \isacommand{have}\isamarkupfalse%
\ {\isachardoublequoteopen}k\ {\isacharequal}{\kern0pt}\ {\isadigit{0}}\ {\isasymor}\ k\ {\isachargreater}{\kern0pt}\ {\isadigit{0}}{\isachardoublequoteclose}\ \isacommand{by}\isamarkupfalse%
\ auto\isanewline
\ \ \isacommand{then}\isamarkupfalse%
\ \isacommand{show}\isamarkupfalse%
\ {\isacharquery}{\kern0pt}case\ \isanewline
\ \ \isacommand{proof}\isamarkupfalse%
\ {\isacharparenleft}{\kern0pt}elim\ disjE{\isacharparenright}{\kern0pt}\isanewline
\ \ \ \ \isacommand{assume}\isamarkupfalse%
\ {\isachardoublequoteopen}k\ {\isachargreater}{\kern0pt}\ {\isadigit{0}}{\isachardoublequoteclose}\isanewline
\ \ \ \ \isacommand{then}\isamarkupfalse%
\ \isacommand{show}\isamarkupfalse%
\ {\isacharquery}{\kern0pt}thesis\ \isacommand{using}\isamarkupfalse%
\ Cons{\isachardot}{\kern0pt}IH\ \isacommand{by}\isamarkupfalse%
\ {\isacharparenleft}{\kern0pt}auto\ simp{\isacharcolon}{\kern0pt}\ {\isacartoucheopen}s\isactrlsub i\ {\isacharequal}{\kern0pt}\ {\isacharparenleft}{\kern0pt}i{\isacharcomma}{\kern0pt}j{\isacharparenright}{\kern0pt}{\isacartoucheclose}\ take{\isacharunderscore}{\kern0pt}Cons{\isacharprime}{\kern0pt}{\isacharparenright}{\kern0pt}\isanewline
\ \ \isacommand{qed}\isamarkupfalse%
\ auto\isanewline
\isacommand{qed}\isamarkupfalse%
%
\endisatagproof
{\isafoldproof}%
%
\isadelimproof
\isanewline
%
\endisadelimproof
\isanewline
\isacommand{lemma}\isamarkupfalse%
\ drop{\isacharunderscore}{\kern0pt}trans{\isacharcolon}{\kern0pt}\ \isanewline
\ \ \isakeyword{shows}\ {\isachardoublequoteopen}drop\ k\ {\isacharparenleft}{\kern0pt}trans{\isacharunderscore}{\kern0pt}path\ {\isacharparenleft}{\kern0pt}k\isactrlsub {\isadigit{1}}{\isacharcomma}{\kern0pt}k\isactrlsub {\isadigit{2}}{\isacharparenright}{\kern0pt}\ ps{\isacharparenright}{\kern0pt}\ {\isacharequal}{\kern0pt}\ trans{\isacharunderscore}{\kern0pt}path\ {\isacharparenleft}{\kern0pt}k\isactrlsub {\isadigit{1}}{\isacharcomma}{\kern0pt}k\isactrlsub {\isadigit{2}}{\isacharparenright}{\kern0pt}\ {\isacharparenleft}{\kern0pt}drop\ k\ ps{\isacharparenright}{\kern0pt}{\isachardoublequoteclose}\isanewline
%
\isadelimproof
%
\endisadelimproof
%
\isatagproof
\isacommand{proof}\isamarkupfalse%
\ {\isacharparenleft}{\kern0pt}induction\ ps\ arbitrary{\isacharcolon}{\kern0pt}\ k{\isacharparenright}{\kern0pt}\isanewline
\ \ \isacommand{case}\isamarkupfalse%
\ Nil\isanewline
\ \ \isacommand{then}\isamarkupfalse%
\ \isacommand{show}\isamarkupfalse%
\ {\isacharquery}{\kern0pt}case\ \isacommand{by}\isamarkupfalse%
\ auto\isanewline
\isacommand{next}\isamarkupfalse%
\isanewline
\ \ \isacommand{case}\isamarkupfalse%
\ {\isacharparenleft}{\kern0pt}Cons\ s\isactrlsub i\ ps{\isacharparenright}{\kern0pt}\isanewline
\ \ \isacommand{then}\isamarkupfalse%
\ \isacommand{obtain}\isamarkupfalse%
\ i\ j\ \isakeyword{where}\ {\isachardoublequoteopen}s\isactrlsub i\ {\isacharequal}{\kern0pt}\ {\isacharparenleft}{\kern0pt}i{\isacharcomma}{\kern0pt}j{\isacharparenright}{\kern0pt}{\isachardoublequoteclose}\ \isacommand{by}\isamarkupfalse%
\ force\isanewline
\ \ \isacommand{then}\isamarkupfalse%
\ \isacommand{have}\isamarkupfalse%
\ {\isachardoublequoteopen}k\ {\isacharequal}{\kern0pt}\ {\isadigit{0}}\ {\isasymor}\ k\ {\isachargreater}{\kern0pt}\ {\isadigit{0}}{\isachardoublequoteclose}\ \isacommand{by}\isamarkupfalse%
\ auto\isanewline
\ \ \isacommand{then}\isamarkupfalse%
\ \isacommand{show}\isamarkupfalse%
\ {\isacharquery}{\kern0pt}case\ \isanewline
\ \ \isacommand{proof}\isamarkupfalse%
\ {\isacharparenleft}{\kern0pt}elim\ disjE{\isacharparenright}{\kern0pt}\isanewline
\ \ \ \ \isacommand{assume}\isamarkupfalse%
\ {\isachardoublequoteopen}k\ {\isachargreater}{\kern0pt}\ {\isadigit{0}}{\isachardoublequoteclose}\isanewline
\ \ \ \ \isacommand{then}\isamarkupfalse%
\ \isacommand{show}\isamarkupfalse%
\ {\isacharquery}{\kern0pt}thesis\ \isacommand{using}\isamarkupfalse%
\ Cons{\isachardot}{\kern0pt}IH\ \isacommand{by}\isamarkupfalse%
\ {\isacharparenleft}{\kern0pt}auto\ simp{\isacharcolon}{\kern0pt}\ {\isacartoucheopen}s\isactrlsub i\ {\isacharequal}{\kern0pt}\ {\isacharparenleft}{\kern0pt}i{\isacharcomma}{\kern0pt}j{\isacharparenright}{\kern0pt}{\isacartoucheclose}\ drop{\isacharunderscore}{\kern0pt}Cons{\isacharprime}{\kern0pt}{\isacharparenright}{\kern0pt}\isanewline
\ \ \isacommand{qed}\isamarkupfalse%
\ auto\isanewline
\isacommand{qed}\isamarkupfalse%
%
\endisatagproof
{\isafoldproof}%
%
\isadelimproof
\isanewline
%
\endisadelimproof
\isanewline
\isacommand{lemma}\isamarkupfalse%
\ trans{\isacharunderscore}{\kern0pt}board{\isacharunderscore}{\kern0pt}correct{\isacharcolon}{\kern0pt}\ {\isachardoublequoteopen}{\isacharparenleft}{\kern0pt}i{\isacharcomma}{\kern0pt}j{\isacharparenright}{\kern0pt}\ {\isasymin}\ b\ {\isasymlongleftrightarrow}\ {\isacharparenleft}{\kern0pt}i{\isacharplus}{\kern0pt}k\isactrlsub {\isadigit{1}}{\isacharcomma}{\kern0pt}j{\isacharplus}{\kern0pt}k\isactrlsub {\isadigit{2}}{\isacharparenright}{\kern0pt}\ {\isasymin}\ trans{\isacharunderscore}{\kern0pt}board\ {\isacharparenleft}{\kern0pt}k\isactrlsub {\isadigit{1}}{\isacharcomma}{\kern0pt}k\isactrlsub {\isadigit{2}}{\isacharparenright}{\kern0pt}\ b{\isachardoublequoteclose}\isanewline
%
\isadelimproof
\ \ %
\endisadelimproof
%
\isatagproof
\isacommand{unfolding}\isamarkupfalse%
\ trans{\isacharunderscore}{\kern0pt}board{\isacharunderscore}{\kern0pt}def\ \isacommand{by}\isamarkupfalse%
\ auto%
\endisatagproof
{\isafoldproof}%
%
\isadelimproof
\isanewline
%
\endisadelimproof
\isanewline
\isacommand{lemma}\isamarkupfalse%
\ board{\isacharunderscore}{\kern0pt}subset{\isacharcolon}{\kern0pt}\ {\isachardoublequoteopen}n\isactrlsub {\isadigit{1}}\ {\isasymle}\ n\isactrlsub {\isadigit{2}}\ {\isasymLongrightarrow}\ m\isactrlsub {\isadigit{1}}\ {\isasymle}\ m\isactrlsub {\isadigit{2}}\ {\isasymLongrightarrow}\ board\ n\isactrlsub {\isadigit{1}}\ m\isactrlsub {\isadigit{1}}\ {\isasymsubseteq}\ board\ n\isactrlsub {\isadigit{2}}\ m\isactrlsub {\isadigit{2}}{\isachardoublequoteclose}\isanewline
%
\isadelimproof
\ \ %
\endisadelimproof
%
\isatagproof
\isacommand{unfolding}\isamarkupfalse%
\ board{\isacharunderscore}{\kern0pt}def\ \isacommand{by}\isamarkupfalse%
\ auto%
\endisatagproof
{\isafoldproof}%
%
\isadelimproof
%
\endisadelimproof
%
\begin{isamarkuptext}%
Board concatenation%
\end{isamarkuptext}\isamarkuptrue%
\isacommand{corollary}\isamarkupfalse%
\ board{\isacharunderscore}{\kern0pt}concat{\isacharcolon}{\kern0pt}\ \isanewline
\ \ \isakeyword{shows}\ {\isachardoublequoteopen}board\ n\ m\isactrlsub {\isadigit{1}}\ {\isasymunion}\ trans{\isacharunderscore}{\kern0pt}board\ {\isacharparenleft}{\kern0pt}{\isadigit{0}}{\isacharcomma}{\kern0pt}int\ m\isactrlsub {\isadigit{1}}{\isacharparenright}{\kern0pt}\ {\isacharparenleft}{\kern0pt}board\ n\ m\isactrlsub {\isadigit{2}}{\isacharparenright}{\kern0pt}\ {\isacharequal}{\kern0pt}\ board\ n\ {\isacharparenleft}{\kern0pt}m\isactrlsub {\isadigit{1}}{\isacharplus}{\kern0pt}m\isactrlsub {\isadigit{2}}{\isacharparenright}{\kern0pt}{\isachardoublequoteclose}\ {\isacharparenleft}{\kern0pt}\isakeyword{is}\ {\isachardoublequoteopen}{\isacharquery}{\kern0pt}b{\isadigit{1}}\ {\isasymunion}\ {\isacharquery}{\kern0pt}b{\isadigit{2}}\ {\isacharequal}{\kern0pt}\ {\isacharquery}{\kern0pt}b{\isachardoublequoteclose}{\isacharparenright}{\kern0pt}\isanewline
%
\isadelimproof
%
\endisadelimproof
%
\isatagproof
\isacommand{proof}\isamarkupfalse%
\isanewline
\ \ \isacommand{show}\isamarkupfalse%
\ {\isachardoublequoteopen}{\isacharquery}{\kern0pt}b{\isadigit{1}}\ {\isasymunion}\ {\isacharquery}{\kern0pt}b{\isadigit{2}}\ {\isasymsubseteq}\ {\isacharquery}{\kern0pt}b{\isachardoublequoteclose}\ \isacommand{unfolding}\isamarkupfalse%
\ board{\isacharunderscore}{\kern0pt}def\ trans{\isacharunderscore}{\kern0pt}board{\isacharunderscore}{\kern0pt}def\ \isacommand{by}\isamarkupfalse%
\ auto\isanewline
\isacommand{next}\isamarkupfalse%
\isanewline
\ \ \isacommand{show}\isamarkupfalse%
\ {\isachardoublequoteopen}{\isacharquery}{\kern0pt}b\ {\isasymsubseteq}\ {\isacharquery}{\kern0pt}b{\isadigit{1}}\ {\isasymunion}\ {\isacharquery}{\kern0pt}b{\isadigit{2}}{\isachardoublequoteclose}\isanewline
\ \ \isacommand{proof}\isamarkupfalse%
\isanewline
\ \ \ \ \isacommand{fix}\isamarkupfalse%
\ x\isanewline
\ \ \ \ \isacommand{assume}\isamarkupfalse%
\ {\isachardoublequoteopen}x\ {\isasymin}\ {\isacharquery}{\kern0pt}b{\isachardoublequoteclose}\isanewline
\ \ \ \ \isacommand{then}\isamarkupfalse%
\ \isacommand{obtain}\isamarkupfalse%
\ i\ j\ \isakeyword{where}\ x{\isacharunderscore}{\kern0pt}split{\isacharcolon}{\kern0pt}\ {\isachardoublequoteopen}x\ {\isacharequal}{\kern0pt}\ {\isacharparenleft}{\kern0pt}i{\isacharcomma}{\kern0pt}j{\isacharparenright}{\kern0pt}{\isachardoublequoteclose}\ {\isachardoublequoteopen}{\isadigit{1}}\ {\isasymle}\ i\ {\isasymand}\ i\ {\isasymle}\ int\ n{\isachardoublequoteclose}\ {\isachardoublequoteopen}{\isadigit{1}}\ {\isasymle}\ j\ {\isasymand}\ j\ {\isasymle}\ int\ {\isacharparenleft}{\kern0pt}m\isactrlsub {\isadigit{1}}{\isacharplus}{\kern0pt}m\isactrlsub {\isadigit{2}}{\isacharparenright}{\kern0pt}{\isachardoublequoteclose}\ \isanewline
\ \ \ \ \ \ \isacommand{unfolding}\isamarkupfalse%
\ board{\isacharunderscore}{\kern0pt}def\ \isacommand{by}\isamarkupfalse%
\ auto\isanewline
\ \ \ \ \isacommand{then}\isamarkupfalse%
\ \isacommand{have}\isamarkupfalse%
\ {\isachardoublequoteopen}j\ {\isasymle}\ int\ m\isactrlsub {\isadigit{1}}\ {\isasymor}\ {\isacharparenleft}{\kern0pt}int\ m\isactrlsub {\isadigit{1}}\ {\isacharless}{\kern0pt}\ j\ {\isasymand}\ j\ {\isasymle}\ int\ {\isacharparenleft}{\kern0pt}m\isactrlsub {\isadigit{1}}{\isacharplus}{\kern0pt}m\isactrlsub {\isadigit{2}}{\isacharparenright}{\kern0pt}{\isacharparenright}{\kern0pt}{\isachardoublequoteclose}\ \isacommand{by}\isamarkupfalse%
\ auto\isanewline
\ \ \ \ \isacommand{then}\isamarkupfalse%
\ \isacommand{show}\isamarkupfalse%
\ {\isachardoublequoteopen}x\ {\isasymin}\ {\isacharquery}{\kern0pt}b{\isadigit{1}}\ {\isasymunion}\ {\isacharquery}{\kern0pt}b{\isadigit{2}}{\isachardoublequoteclose}\isanewline
\ \ \ \ \isacommand{proof}\isamarkupfalse%
\isanewline
\ \ \ \ \ \ \isacommand{assume}\isamarkupfalse%
\ {\isachardoublequoteopen}j\ {\isasymle}\ int\ m\isactrlsub {\isadigit{1}}{\isachardoublequoteclose}\isanewline
\ \ \ \ \ \ \isacommand{then}\isamarkupfalse%
\ \isacommand{show}\isamarkupfalse%
\ {\isachardoublequoteopen}x\ {\isasymin}\ {\isacharquery}{\kern0pt}b{\isadigit{1}}\ {\isasymunion}\ {\isacharquery}{\kern0pt}b{\isadigit{2}}{\isachardoublequoteclose}\ \isacommand{using}\isamarkupfalse%
\ x{\isacharunderscore}{\kern0pt}split\ \isacommand{unfolding}\isamarkupfalse%
\ board{\isacharunderscore}{\kern0pt}def\ \isacommand{by}\isamarkupfalse%
\ auto\isanewline
\ \ \ \ \isacommand{next}\isamarkupfalse%
\isanewline
\ \ \ \ \ \ \isacommand{assume}\isamarkupfalse%
\ asm{\isacharcolon}{\kern0pt}\ {\isachardoublequoteopen}int\ m\isactrlsub {\isadigit{1}}\ {\isacharless}{\kern0pt}\ j\ {\isasymand}\ j\ {\isasymle}\ int\ {\isacharparenleft}{\kern0pt}m\isactrlsub {\isadigit{1}}{\isacharplus}{\kern0pt}m\isactrlsub {\isadigit{2}}{\isacharparenright}{\kern0pt}{\isachardoublequoteclose}\isanewline
\ \ \ \ \ \ \isacommand{then}\isamarkupfalse%
\ \isacommand{have}\isamarkupfalse%
\ {\isachardoublequoteopen}{\isacharparenleft}{\kern0pt}i{\isacharcomma}{\kern0pt}j{\isacharminus}{\kern0pt}int\ m\isactrlsub {\isadigit{1}}{\isacharparenright}{\kern0pt}\ {\isasymin}\ board\ n\ m\isactrlsub {\isadigit{2}}{\isachardoublequoteclose}\ \isacommand{using}\isamarkupfalse%
\ x{\isacharunderscore}{\kern0pt}split\ \isacommand{unfolding}\isamarkupfalse%
\ board{\isacharunderscore}{\kern0pt}def\ \isacommand{by}\isamarkupfalse%
\ auto\isanewline
\ \ \ \ \ \ \isacommand{then}\isamarkupfalse%
\ \isacommand{show}\isamarkupfalse%
\ {\isachardoublequoteopen}x\ {\isasymin}\ {\isacharquery}{\kern0pt}b{\isadigit{1}}\ {\isasymunion}\ {\isacharquery}{\kern0pt}b{\isadigit{2}}{\isachardoublequoteclose}\ \isanewline
\ \ \ \ \ \ \ \ \isacommand{using}\isamarkupfalse%
\ x{\isacharunderscore}{\kern0pt}split\ asm\ trans{\isacharunderscore}{\kern0pt}board{\isacharunderscore}{\kern0pt}correct{\isacharbrackleft}{\kern0pt}of\ i\ {\isachardoublequoteopen}j{\isacharminus}{\kern0pt}int\ m\isactrlsub {\isadigit{1}}{\isachardoublequoteclose}\ {\isachardoublequoteopen}board\ n\ m\isactrlsub {\isadigit{2}}{\isachardoublequoteclose}\ {\isadigit{0}}\ {\isachardoublequoteopen}int\ m\isactrlsub {\isadigit{1}}{\isachardoublequoteclose}{\isacharbrackright}{\kern0pt}\ \isacommand{by}\isamarkupfalse%
\ auto\isanewline
\ \ \ \ \isacommand{qed}\isamarkupfalse%
\isanewline
\ \ \isacommand{qed}\isamarkupfalse%
\isanewline
\isacommand{qed}\isamarkupfalse%
%
\endisatagproof
{\isafoldproof}%
%
\isadelimproof
\isanewline
%
\endisadelimproof
\isanewline
\isacommand{lemma}\isamarkupfalse%
\ transpose{\isacharunderscore}{\kern0pt}trans{\isacharunderscore}{\kern0pt}board{\isacharcolon}{\kern0pt}\ \isanewline
\ \ {\isachardoublequoteopen}transpose{\isacharunderscore}{\kern0pt}board\ {\isacharparenleft}{\kern0pt}trans{\isacharunderscore}{\kern0pt}board\ {\isacharparenleft}{\kern0pt}k\isactrlsub {\isadigit{1}}{\isacharcomma}{\kern0pt}k\isactrlsub {\isadigit{2}}{\isacharparenright}{\kern0pt}\ b{\isacharparenright}{\kern0pt}\ {\isacharequal}{\kern0pt}\ trans{\isacharunderscore}{\kern0pt}board\ {\isacharparenleft}{\kern0pt}k\isactrlsub {\isadigit{2}}{\isacharcomma}{\kern0pt}k\isactrlsub {\isadigit{1}}{\isacharparenright}{\kern0pt}\ {\isacharparenleft}{\kern0pt}transpose{\isacharunderscore}{\kern0pt}board\ b{\isacharparenright}{\kern0pt}{\isachardoublequoteclose}\isanewline
%
\isadelimproof
\ \ %
\endisadelimproof
%
\isatagproof
\isacommand{unfolding}\isamarkupfalse%
\ transpose{\isacharunderscore}{\kern0pt}board{\isacharunderscore}{\kern0pt}def\ trans{\isacharunderscore}{\kern0pt}board{\isacharunderscore}{\kern0pt}def\ \isacommand{by}\isamarkupfalse%
\ blast%
\endisatagproof
{\isafoldproof}%
%
\isadelimproof
\isanewline
%
\endisadelimproof
\isanewline
\isacommand{corollary}\isamarkupfalse%
\ board{\isacharunderscore}{\kern0pt}concatT{\isacharcolon}{\kern0pt}\ \isanewline
\ \ \isakeyword{shows}\ {\isachardoublequoteopen}board\ n\isactrlsub {\isadigit{1}}\ m\ {\isasymunion}\ trans{\isacharunderscore}{\kern0pt}board\ {\isacharparenleft}{\kern0pt}int\ n\isactrlsub {\isadigit{1}}{\isacharcomma}{\kern0pt}{\isadigit{0}}{\isacharparenright}{\kern0pt}\ {\isacharparenleft}{\kern0pt}board\ n\isactrlsub {\isadigit{2}}\ m{\isacharparenright}{\kern0pt}\ {\isacharequal}{\kern0pt}\ board\ {\isacharparenleft}{\kern0pt}n\isactrlsub {\isadigit{1}}{\isacharplus}{\kern0pt}n\isactrlsub {\isadigit{2}}{\isacharparenright}{\kern0pt}\ m{\isachardoublequoteclose}\ {\isacharparenleft}{\kern0pt}\isakeyword{is}\ {\isachardoublequoteopen}{\isacharquery}{\kern0pt}b\isactrlsub {\isadigit{1}}\ {\isasymunion}\ {\isacharquery}{\kern0pt}b\isactrlsub {\isadigit{2}}\ {\isacharequal}{\kern0pt}\ {\isacharquery}{\kern0pt}b{\isachardoublequoteclose}{\isacharparenright}{\kern0pt}\isanewline
%
\isadelimproof
%
\endisadelimproof
%
\isatagproof
\isacommand{proof}\isamarkupfalse%
\ {\isacharminus}{\kern0pt}\isanewline
\ \ \isacommand{let}\isamarkupfalse%
\ {\isacharquery}{\kern0pt}b\isactrlsub {\isadigit{1}}T{\isacharequal}{\kern0pt}{\isachardoublequoteopen}board\ m\ n\isactrlsub {\isadigit{1}}{\isachardoublequoteclose}\isanewline
\ \ \isacommand{let}\isamarkupfalse%
\ {\isacharquery}{\kern0pt}b\isactrlsub {\isadigit{2}}T{\isacharequal}{\kern0pt}{\isachardoublequoteopen}trans{\isacharunderscore}{\kern0pt}board\ {\isacharparenleft}{\kern0pt}{\isadigit{0}}{\isacharcomma}{\kern0pt}int\ n\isactrlsub {\isadigit{1}}{\isacharparenright}{\kern0pt}\ {\isacharparenleft}{\kern0pt}board\ m\ n\isactrlsub {\isadigit{2}}{\isacharparenright}{\kern0pt}{\isachardoublequoteclose}\isanewline
\ \ \isacommand{have}\isamarkupfalse%
\ {\isachardoublequoteopen}{\isacharquery}{\kern0pt}b\isactrlsub {\isadigit{1}}\ {\isasymunion}\ {\isacharquery}{\kern0pt}b\isactrlsub {\isadigit{2}}\ {\isacharequal}{\kern0pt}\ transpose{\isacharunderscore}{\kern0pt}board\ {\isacharparenleft}{\kern0pt}{\isacharquery}{\kern0pt}b\isactrlsub {\isadigit{1}}T\ {\isasymunion}\ {\isacharquery}{\kern0pt}b\isactrlsub {\isadigit{2}}T{\isacharparenright}{\kern0pt}\ {\isachardoublequoteclose}\isanewline
\ \ \ \ \isacommand{using}\isamarkupfalse%
\ transpose{\isacharunderscore}{\kern0pt}board{\isadigit{2}}\ transpose{\isacharunderscore}{\kern0pt}union\ transpose{\isacharunderscore}{\kern0pt}board\ transpose{\isacharunderscore}{\kern0pt}trans{\isacharunderscore}{\kern0pt}board\ \isacommand{by}\isamarkupfalse%
\ auto\isanewline
\ \ \isacommand{also}\isamarkupfalse%
\ \isacommand{have}\isamarkupfalse%
\ {\isachardoublequoteopen}{\isachardot}{\kern0pt}{\isachardot}{\kern0pt}{\isachardot}{\kern0pt}\ {\isacharequal}{\kern0pt}\ transpose{\isacharunderscore}{\kern0pt}board\ {\isacharparenleft}{\kern0pt}board\ m\ {\isacharparenleft}{\kern0pt}n\isactrlsub {\isadigit{1}}{\isacharplus}{\kern0pt}n\isactrlsub {\isadigit{2}}{\isacharparenright}{\kern0pt}{\isacharparenright}{\kern0pt}{\isachardoublequoteclose}\isanewline
\ \ \ \ \isacommand{using}\isamarkupfalse%
\ board{\isacharunderscore}{\kern0pt}concat\ \isacommand{by}\isamarkupfalse%
\ auto\isanewline
\ \ \isacommand{also}\isamarkupfalse%
\ \isacommand{have}\isamarkupfalse%
\ {\isachardoublequoteopen}{\isachardot}{\kern0pt}{\isachardot}{\kern0pt}{\isachardot}{\kern0pt}\ {\isacharequal}{\kern0pt}\ board\ {\isacharparenleft}{\kern0pt}n\isactrlsub {\isadigit{1}}{\isacharplus}{\kern0pt}n\isactrlsub {\isadigit{2}}{\isacharparenright}{\kern0pt}\ m{\isachardoublequoteclose}\isanewline
\ \ \ \ \isacommand{using}\isamarkupfalse%
\ transpose{\isacharunderscore}{\kern0pt}board\ \isacommand{by}\isamarkupfalse%
\ auto\isanewline
\ \ \isacommand{finally}\isamarkupfalse%
\ \isacommand{show}\isamarkupfalse%
\ {\isacharquery}{\kern0pt}thesis\ \isacommand{{\isachardot}{\kern0pt}}\isamarkupfalse%
\isanewline
\isacommand{qed}\isamarkupfalse%
%
\endisatagproof
{\isafoldproof}%
%
\isadelimproof
\isanewline
%
\endisadelimproof
\isanewline
\isacommand{lemma}\isamarkupfalse%
\ trans{\isacharunderscore}{\kern0pt}valid{\isacharunderscore}{\kern0pt}step{\isacharcolon}{\kern0pt}\ \isanewline
\ \ {\isachardoublequoteopen}valid{\isacharunderscore}{\kern0pt}step\ {\isacharparenleft}{\kern0pt}i{\isacharcomma}{\kern0pt}j{\isacharparenright}{\kern0pt}\ {\isacharparenleft}{\kern0pt}i{\isacharprime}{\kern0pt}{\isacharcomma}{\kern0pt}j{\isacharprime}{\kern0pt}{\isacharparenright}{\kern0pt}\ {\isasymLongrightarrow}\ valid{\isacharunderscore}{\kern0pt}step\ {\isacharparenleft}{\kern0pt}i{\isacharplus}{\kern0pt}k\isactrlsub {\isadigit{1}}{\isacharcomma}{\kern0pt}j{\isacharplus}{\kern0pt}k\isactrlsub {\isadigit{2}}{\isacharparenright}{\kern0pt}\ {\isacharparenleft}{\kern0pt}i{\isacharprime}{\kern0pt}{\isacharplus}{\kern0pt}k\isactrlsub {\isadigit{1}}{\isacharcomma}{\kern0pt}j{\isacharprime}{\kern0pt}{\isacharplus}{\kern0pt}k\isactrlsub {\isadigit{2}}{\isacharparenright}{\kern0pt}{\isachardoublequoteclose}\isanewline
%
\isadelimproof
\ \ %
\endisadelimproof
%
\isatagproof
\isacommand{unfolding}\isamarkupfalse%
\ valid{\isacharunderscore}{\kern0pt}step{\isacharunderscore}{\kern0pt}def\ \isacommand{by}\isamarkupfalse%
\ auto%
\endisatagproof
{\isafoldproof}%
%
\isadelimproof
%
\endisadelimproof
%
\begin{isamarkuptext}%
Translating a path and a boards preserves the validity.%
\end{isamarkuptext}\isamarkuptrue%
\isacommand{lemma}\isamarkupfalse%
\ trans{\isacharunderscore}{\kern0pt}knights{\isacharunderscore}{\kern0pt}path{\isacharcolon}{\kern0pt}\isanewline
\ \ \isakeyword{assumes}\ {\isachardoublequoteopen}knights{\isacharunderscore}{\kern0pt}path\ b\ ps{\isachardoublequoteclose}\isanewline
\ \ \isakeyword{shows}\ {\isachardoublequoteopen}knights{\isacharunderscore}{\kern0pt}path\ {\isacharparenleft}{\kern0pt}trans{\isacharunderscore}{\kern0pt}board\ {\isacharparenleft}{\kern0pt}k\isactrlsub {\isadigit{1}}{\isacharcomma}{\kern0pt}k\isactrlsub {\isadigit{2}}{\isacharparenright}{\kern0pt}\ b{\isacharparenright}{\kern0pt}\ {\isacharparenleft}{\kern0pt}trans{\isacharunderscore}{\kern0pt}path\ {\isacharparenleft}{\kern0pt}k\isactrlsub {\isadigit{1}}{\isacharcomma}{\kern0pt}k\isactrlsub {\isadigit{2}}{\isacharparenright}{\kern0pt}\ ps{\isacharparenright}{\kern0pt}{\isachardoublequoteclose}\isanewline
%
\isadelimproof
\ \ %
\endisadelimproof
%
\isatagproof
\isacommand{using}\isamarkupfalse%
\ assms\isanewline
\isacommand{proof}\isamarkupfalse%
\ {\isacharparenleft}{\kern0pt}induction\ rule{\isacharcolon}{\kern0pt}\ knights{\isacharunderscore}{\kern0pt}path{\isachardot}{\kern0pt}induct{\isacharparenright}{\kern0pt}\isanewline
\ \ \isacommand{case}\isamarkupfalse%
\ {\isacharparenleft}{\kern0pt}{\isadigit{2}}\ s\isactrlsub i\ b\ s\isactrlsub j\ xs{\isacharparenright}{\kern0pt}\isanewline
\ \ \isacommand{then}\isamarkupfalse%
\ \isacommand{obtain}\isamarkupfalse%
\ i\ j\ i{\isacharprime}{\kern0pt}\ j{\isacharprime}{\kern0pt}\ \isakeyword{where}\ split{\isacharcolon}{\kern0pt}\ {\isachardoublequoteopen}s\isactrlsub i\ {\isacharequal}{\kern0pt}\ {\isacharparenleft}{\kern0pt}i{\isacharcomma}{\kern0pt}j{\isacharparenright}{\kern0pt}{\isachardoublequoteclose}\ {\isachardoublequoteopen}s\isactrlsub j\ {\isacharequal}{\kern0pt}\ {\isacharparenleft}{\kern0pt}i{\isacharprime}{\kern0pt}{\isacharcomma}{\kern0pt}j{\isacharprime}{\kern0pt}{\isacharparenright}{\kern0pt}{\isachardoublequoteclose}\ \isacommand{by}\isamarkupfalse%
\ force\isanewline
\ \ \isacommand{let}\isamarkupfalse%
\ {\isacharquery}{\kern0pt}s\isactrlsub i{\isacharequal}{\kern0pt}{\isachardoublequoteopen}{\isacharparenleft}{\kern0pt}i{\isacharplus}{\kern0pt}k\isactrlsub {\isadigit{1}}{\isacharcomma}{\kern0pt}j{\isacharplus}{\kern0pt}k\isactrlsub {\isadigit{2}}{\isacharparenright}{\kern0pt}{\isachardoublequoteclose}\isanewline
\ \ \isacommand{let}\isamarkupfalse%
\ {\isacharquery}{\kern0pt}s\isactrlsub j{\isacharequal}{\kern0pt}{\isachardoublequoteopen}{\isacharparenleft}{\kern0pt}i{\isacharprime}{\kern0pt}{\isacharplus}{\kern0pt}k\isactrlsub {\isadigit{1}}{\isacharcomma}{\kern0pt}j{\isacharprime}{\kern0pt}{\isacharplus}{\kern0pt}k\isactrlsub {\isadigit{2}}{\isacharparenright}{\kern0pt}{\isachardoublequoteclose}\isanewline
\ \ \isacommand{let}\isamarkupfalse%
\ {\isacharquery}{\kern0pt}xs{\isacharequal}{\kern0pt}{\isachardoublequoteopen}trans{\isacharunderscore}{\kern0pt}path\ {\isacharparenleft}{\kern0pt}k\isactrlsub {\isadigit{1}}{\isacharcomma}{\kern0pt}k\isactrlsub {\isadigit{2}}{\isacharparenright}{\kern0pt}\ xs{\isachardoublequoteclose}\isanewline
\ \ \isacommand{let}\isamarkupfalse%
\ {\isacharquery}{\kern0pt}b{\isacharequal}{\kern0pt}{\isachardoublequoteopen}trans{\isacharunderscore}{\kern0pt}board\ {\isacharparenleft}{\kern0pt}k\isactrlsub {\isadigit{1}}{\isacharcomma}{\kern0pt}k\isactrlsub {\isadigit{2}}{\isacharparenright}{\kern0pt}\ b{\isachardoublequoteclose}\isanewline
\ \ \isacommand{have}\isamarkupfalse%
\ simps{\isacharcolon}{\kern0pt}\ {\isachardoublequoteopen}trans{\isacharunderscore}{\kern0pt}path\ {\isacharparenleft}{\kern0pt}k\isactrlsub {\isadigit{1}}{\isacharcomma}{\kern0pt}k\isactrlsub {\isadigit{2}}{\isacharparenright}{\kern0pt}\ {\isacharparenleft}{\kern0pt}s\isactrlsub i{\isacharhash}{\kern0pt}s\isactrlsub j{\isacharhash}{\kern0pt}xs{\isacharparenright}{\kern0pt}\ {\isacharequal}{\kern0pt}\ {\isacharquery}{\kern0pt}s\isactrlsub i{\isacharhash}{\kern0pt}{\isacharquery}{\kern0pt}s\isactrlsub j{\isacharhash}{\kern0pt}{\isacharquery}{\kern0pt}xs{\isachardoublequoteclose}\ \isanewline
\ \ \ \ \ \ \ \ \ \ \ \ \ \ {\isachardoublequoteopen}{\isacharquery}{\kern0pt}b\ {\isasymunion}\ {\isacharbraceleft}{\kern0pt}{\isacharquery}{\kern0pt}s\isactrlsub i{\isacharbraceright}{\kern0pt}\ {\isacharequal}{\kern0pt}\ trans{\isacharunderscore}{\kern0pt}board\ {\isacharparenleft}{\kern0pt}k\isactrlsub {\isadigit{1}}{\isacharcomma}{\kern0pt}k\isactrlsub {\isadigit{2}}{\isacharparenright}{\kern0pt}\ {\isacharparenleft}{\kern0pt}b\ {\isasymunion}\ {\isacharbraceleft}{\kern0pt}s\isactrlsub i{\isacharbraceright}{\kern0pt}{\isacharparenright}{\kern0pt}{\isachardoublequoteclose}\isanewline
\ \ \ \ \isacommand{unfolding}\isamarkupfalse%
\ trans{\isacharunderscore}{\kern0pt}board{\isacharunderscore}{\kern0pt}def\ \isacommand{using}\isamarkupfalse%
\ split\ \isacommand{by}\isamarkupfalse%
\ auto\isanewline
\ \ \isacommand{have}\isamarkupfalse%
\ {\isachardoublequoteopen}{\isacharquery}{\kern0pt}s\isactrlsub i\ {\isasymnotin}\ {\isacharquery}{\kern0pt}b{\isachardoublequoteclose}\ {\isachardoublequoteopen}valid{\isacharunderscore}{\kern0pt}step\ {\isacharquery}{\kern0pt}s\isactrlsub i\ {\isacharquery}{\kern0pt}s\isactrlsub j{\isachardoublequoteclose}\ {\isachardoublequoteopen}knights{\isacharunderscore}{\kern0pt}path\ {\isacharquery}{\kern0pt}b\ {\isacharparenleft}{\kern0pt}{\isacharquery}{\kern0pt}s\isactrlsub j{\isacharhash}{\kern0pt}{\isacharquery}{\kern0pt}xs{\isacharparenright}{\kern0pt}{\isachardoublequoteclose}\isanewline
\ \ \ \ \isacommand{using}\isamarkupfalse%
\ {\isadigit{2}}\ split\ trans{\isacharunderscore}{\kern0pt}valid{\isacharunderscore}{\kern0pt}step\ \isacommand{by}\isamarkupfalse%
\ {\isacharparenleft}{\kern0pt}auto\ simp{\isacharcolon}{\kern0pt}\ trans{\isacharunderscore}{\kern0pt}board{\isacharunderscore}{\kern0pt}def{\isacharparenright}{\kern0pt}\isanewline
\ \ \isacommand{then}\isamarkupfalse%
\ \isacommand{have}\isamarkupfalse%
\ {\isachardoublequoteopen}knights{\isacharunderscore}{\kern0pt}path\ {\isacharparenleft}{\kern0pt}{\isacharquery}{\kern0pt}b\ {\isasymunion}\ {\isacharbraceleft}{\kern0pt}{\isacharquery}{\kern0pt}s\isactrlsub i{\isacharbraceright}{\kern0pt}{\isacharparenright}{\kern0pt}\ {\isacharparenleft}{\kern0pt}{\isacharquery}{\kern0pt}s\isactrlsub i{\isacharhash}{\kern0pt}{\isacharquery}{\kern0pt}s\isactrlsub j{\isacharhash}{\kern0pt}{\isacharquery}{\kern0pt}xs{\isacharparenright}{\kern0pt}{\isachardoublequoteclose}\isanewline
\ \ \ \ \isacommand{using}\isamarkupfalse%
\ knights{\isacharunderscore}{\kern0pt}path{\isachardot}{\kern0pt}intros\ \isacommand{by}\isamarkupfalse%
\ auto\isanewline
\ \ \isacommand{then}\isamarkupfalse%
\ \isacommand{show}\isamarkupfalse%
\ {\isacharquery}{\kern0pt}case\ \isacommand{using}\isamarkupfalse%
\ simps\ \isacommand{by}\isamarkupfalse%
\ auto\isanewline
\isacommand{qed}\isamarkupfalse%
\ {\isacharparenleft}{\kern0pt}auto\ simp{\isacharcolon}{\kern0pt}\ trans{\isacharunderscore}{\kern0pt}board{\isacharunderscore}{\kern0pt}def\ intro{\isacharcolon}{\kern0pt}\ knights{\isacharunderscore}{\kern0pt}path{\isachardot}{\kern0pt}intros{\isacharparenright}{\kern0pt}%
\endisatagproof
{\isafoldproof}%
%
\isadelimproof
%
\endisadelimproof
%
\begin{isamarkuptext}%
Predicate that indicates if two squares \isa{s\isactrlsub i} and \isa{s\isactrlsub j} are adjacent in \isa{ps}.%
\end{isamarkuptext}\isamarkuptrue%
\isacommand{definition}\isamarkupfalse%
\ step{\isacharunderscore}{\kern0pt}in\ {\isacharcolon}{\kern0pt}{\isacharcolon}{\kern0pt}\ {\isachardoublequoteopen}path\ {\isasymRightarrow}\ square\ {\isasymRightarrow}\ square\ {\isasymRightarrow}\ bool{\isachardoublequoteclose}\ \isakeyword{where}\isanewline
\ \ {\isachardoublequoteopen}step{\isacharunderscore}{\kern0pt}in\ ps\ s\isactrlsub i\ s\isactrlsub j\ {\isasymequiv}\ {\isacharparenleft}{\kern0pt}{\isasymexists}k{\isachardot}{\kern0pt}\ {\isadigit{0}}\ {\isacharless}{\kern0pt}\ k\ {\isasymand}\ k\ {\isacharless}{\kern0pt}\ length\ ps\ {\isasymand}\ last\ {\isacharparenleft}{\kern0pt}take\ k\ ps{\isacharparenright}{\kern0pt}\ {\isacharequal}{\kern0pt}\ s\isactrlsub i\ {\isasymand}\ hd\ {\isacharparenleft}{\kern0pt}drop\ k\ ps{\isacharparenright}{\kern0pt}\ {\isacharequal}{\kern0pt}\ s\isactrlsub j{\isacharparenright}{\kern0pt}{\isachardoublequoteclose}\isanewline
\isanewline
\isacommand{lemma}\isamarkupfalse%
\ step{\isacharunderscore}{\kern0pt}in{\isacharunderscore}{\kern0pt}Cons{\isacharcolon}{\kern0pt}\ {\isachardoublequoteopen}step{\isacharunderscore}{\kern0pt}in\ ps\ s\isactrlsub i\ s\isactrlsub j\ {\isasymLongrightarrow}\ step{\isacharunderscore}{\kern0pt}in\ {\isacharparenleft}{\kern0pt}s\isactrlsub k{\isacharhash}{\kern0pt}ps{\isacharparenright}{\kern0pt}\ s\isactrlsub i\ s\isactrlsub j{\isachardoublequoteclose}\isanewline
%
\isadelimproof
%
\endisadelimproof
%
\isatagproof
\isacommand{proof}\isamarkupfalse%
\ {\isacharminus}{\kern0pt}\isanewline
\ \ \isacommand{assume}\isamarkupfalse%
\ {\isachardoublequoteopen}step{\isacharunderscore}{\kern0pt}in\ ps\ s\isactrlsub i\ s\isactrlsub j{\isachardoublequoteclose}\isanewline
\ \ \isacommand{then}\isamarkupfalse%
\ \isacommand{obtain}\isamarkupfalse%
\ k\ \isakeyword{where}\ {\isachardoublequoteopen}{\isadigit{0}}\ {\isacharless}{\kern0pt}\ k\ {\isasymand}\ k\ {\isacharless}{\kern0pt}\ length\ ps{\isachardoublequoteclose}\ {\isachardoublequoteopen}last\ {\isacharparenleft}{\kern0pt}take\ k\ ps{\isacharparenright}{\kern0pt}\ {\isacharequal}{\kern0pt}\ s\isactrlsub i{\isachardoublequoteclose}\ {\isachardoublequoteopen}hd\ {\isacharparenleft}{\kern0pt}drop\ k\ ps{\isacharparenright}{\kern0pt}\ {\isacharequal}{\kern0pt}\ s\isactrlsub j{\isachardoublequoteclose}\isanewline
\ \ \ \ \isacommand{unfolding}\isamarkupfalse%
\ step{\isacharunderscore}{\kern0pt}in{\isacharunderscore}{\kern0pt}def\ \isacommand{by}\isamarkupfalse%
\ auto\ \isanewline
\ \ \isacommand{then}\isamarkupfalse%
\ \isacommand{have}\isamarkupfalse%
\ {\isachardoublequoteopen}{\isadigit{0}}\ {\isacharless}{\kern0pt}\ k{\isacharplus}{\kern0pt}{\isadigit{1}}\ {\isasymand}\ k{\isacharplus}{\kern0pt}{\isadigit{1}}\ {\isacharless}{\kern0pt}\ length\ {\isacharparenleft}{\kern0pt}s\isactrlsub k{\isacharhash}{\kern0pt}ps{\isacharparenright}{\kern0pt}{\isachardoublequoteclose}\ \isanewline
\ \ \ \ \ \ {\isachardoublequoteopen}last\ {\isacharparenleft}{\kern0pt}take\ {\isacharparenleft}{\kern0pt}k{\isacharplus}{\kern0pt}{\isadigit{1}}{\isacharparenright}{\kern0pt}\ {\isacharparenleft}{\kern0pt}s\isactrlsub k{\isacharhash}{\kern0pt}ps{\isacharparenright}{\kern0pt}{\isacharparenright}{\kern0pt}\ {\isacharequal}{\kern0pt}\ s\isactrlsub i{\isachardoublequoteclose}\ {\isachardoublequoteopen}hd\ {\isacharparenleft}{\kern0pt}drop\ {\isacharparenleft}{\kern0pt}k{\isacharplus}{\kern0pt}{\isadigit{1}}{\isacharparenright}{\kern0pt}\ {\isacharparenleft}{\kern0pt}s\isactrlsub k{\isacharhash}{\kern0pt}ps{\isacharparenright}{\kern0pt}{\isacharparenright}{\kern0pt}\ {\isacharequal}{\kern0pt}\ s\isactrlsub j{\isachardoublequoteclose}\isanewline
\ \ \ \ \isacommand{by}\isamarkupfalse%
\ auto\isanewline
\ \ \isacommand{then}\isamarkupfalse%
\ \isacommand{show}\isamarkupfalse%
\ {\isacharquery}{\kern0pt}thesis\isanewline
\ \ \ \ \isacommand{by}\isamarkupfalse%
\ {\isacharparenleft}{\kern0pt}auto\ simp{\isacharcolon}{\kern0pt}\ step{\isacharunderscore}{\kern0pt}in{\isacharunderscore}{\kern0pt}def{\isacharparenright}{\kern0pt}\isanewline
\isacommand{qed}\isamarkupfalse%
%
\endisatagproof
{\isafoldproof}%
%
\isadelimproof
\isanewline
%
\endisadelimproof
\isanewline
\isacommand{lemma}\isamarkupfalse%
\ step{\isacharunderscore}{\kern0pt}in{\isacharunderscore}{\kern0pt}append{\isacharcolon}{\kern0pt}\ {\isachardoublequoteopen}step{\isacharunderscore}{\kern0pt}in\ ps\ s\isactrlsub i\ s\isactrlsub j\ {\isasymLongrightarrow}\ step{\isacharunderscore}{\kern0pt}in\ {\isacharparenleft}{\kern0pt}ps{\isacharat}{\kern0pt}ps{\isacharprime}{\kern0pt}{\isacharparenright}{\kern0pt}\ s\isactrlsub i\ s\isactrlsub j{\isachardoublequoteclose}\isanewline
%
\isadelimproof
%
\endisadelimproof
%
\isatagproof
\isacommand{proof}\isamarkupfalse%
\ {\isacharminus}{\kern0pt}\isanewline
\ \ \isacommand{assume}\isamarkupfalse%
\ {\isachardoublequoteopen}step{\isacharunderscore}{\kern0pt}in\ ps\ s\isactrlsub i\ s\isactrlsub j{\isachardoublequoteclose}\isanewline
\ \ \isacommand{then}\isamarkupfalse%
\ \isacommand{obtain}\isamarkupfalse%
\ k\ \isakeyword{where}\ {\isachardoublequoteopen}{\isadigit{0}}\ {\isacharless}{\kern0pt}\ k\ {\isasymand}\ k\ {\isacharless}{\kern0pt}\ length\ ps{\isachardoublequoteclose}\ {\isachardoublequoteopen}last\ {\isacharparenleft}{\kern0pt}take\ k\ ps{\isacharparenright}{\kern0pt}\ {\isacharequal}{\kern0pt}\ s\isactrlsub i{\isachardoublequoteclose}\ {\isachardoublequoteopen}hd\ {\isacharparenleft}{\kern0pt}drop\ k\ ps{\isacharparenright}{\kern0pt}\ {\isacharequal}{\kern0pt}\ s\isactrlsub j{\isachardoublequoteclose}\isanewline
\ \ \ \ \isacommand{unfolding}\isamarkupfalse%
\ step{\isacharunderscore}{\kern0pt}in{\isacharunderscore}{\kern0pt}def\ \isacommand{by}\isamarkupfalse%
\ auto\ \isanewline
\ \ \isacommand{then}\isamarkupfalse%
\ \isacommand{have}\isamarkupfalse%
\ {\isachardoublequoteopen}{\isadigit{0}}\ {\isacharless}{\kern0pt}\ k\ {\isasymand}\ k\ {\isacharless}{\kern0pt}\ length\ {\isacharparenleft}{\kern0pt}ps{\isacharat}{\kern0pt}ps{\isacharprime}{\kern0pt}{\isacharparenright}{\kern0pt}{\isachardoublequoteclose}\ \isanewline
\ \ \ \ \ \ {\isachardoublequoteopen}last\ {\isacharparenleft}{\kern0pt}take\ k\ {\isacharparenleft}{\kern0pt}ps{\isacharat}{\kern0pt}ps{\isacharprime}{\kern0pt}{\isacharparenright}{\kern0pt}{\isacharparenright}{\kern0pt}\ {\isacharequal}{\kern0pt}\ s\isactrlsub i{\isachardoublequoteclose}\ {\isachardoublequoteopen}hd\ {\isacharparenleft}{\kern0pt}drop\ k\ {\isacharparenleft}{\kern0pt}ps{\isacharat}{\kern0pt}ps{\isacharprime}{\kern0pt}{\isacharparenright}{\kern0pt}{\isacharparenright}{\kern0pt}\ {\isacharequal}{\kern0pt}\ s\isactrlsub j{\isachardoublequoteclose}\isanewline
\ \ \ \ \isacommand{by}\isamarkupfalse%
\ auto\isanewline
\ \ \isacommand{then}\isamarkupfalse%
\ \isacommand{show}\isamarkupfalse%
\ {\isacharquery}{\kern0pt}thesis\isanewline
\ \ \ \ \isacommand{by}\isamarkupfalse%
\ {\isacharparenleft}{\kern0pt}auto\ simp{\isacharcolon}{\kern0pt}\ step{\isacharunderscore}{\kern0pt}in{\isacharunderscore}{\kern0pt}def{\isacharparenright}{\kern0pt}\isanewline
\isacommand{qed}\isamarkupfalse%
%
\endisatagproof
{\isafoldproof}%
%
\isadelimproof
\isanewline
%
\endisadelimproof
\isanewline
\isacommand{lemma}\isamarkupfalse%
\ step{\isacharunderscore}{\kern0pt}in{\isacharunderscore}{\kern0pt}prepend{\isacharcolon}{\kern0pt}\ {\isachardoublequoteopen}step{\isacharunderscore}{\kern0pt}in\ ps\ s\isactrlsub i\ s\isactrlsub j\ {\isasymLongrightarrow}\ step{\isacharunderscore}{\kern0pt}in\ {\isacharparenleft}{\kern0pt}ps{\isacharprime}{\kern0pt}{\isacharat}{\kern0pt}ps{\isacharparenright}{\kern0pt}\ s\isactrlsub i\ s\isactrlsub j{\isachardoublequoteclose}\isanewline
%
\isadelimproof
\ \ %
\endisadelimproof
%
\isatagproof
\isacommand{using}\isamarkupfalse%
\ step{\isacharunderscore}{\kern0pt}in{\isacharunderscore}{\kern0pt}Cons\ \isacommand{by}\isamarkupfalse%
\ {\isacharparenleft}{\kern0pt}induction\ ps{\isacharprime}{\kern0pt}\ arbitrary{\isacharcolon}{\kern0pt}\ ps{\isacharparenright}{\kern0pt}\ auto%
\endisatagproof
{\isafoldproof}%
%
\isadelimproof
\isanewline
%
\endisadelimproof
\isanewline
\isacommand{lemma}\isamarkupfalse%
\ step{\isacharunderscore}{\kern0pt}in{\isacharunderscore}{\kern0pt}valid{\isacharunderscore}{\kern0pt}step{\isacharcolon}{\kern0pt}\ {\isachardoublequoteopen}knights{\isacharunderscore}{\kern0pt}path\ b\ ps\ {\isasymLongrightarrow}\ step{\isacharunderscore}{\kern0pt}in\ ps\ s\isactrlsub i\ s\isactrlsub j\ {\isasymLongrightarrow}\ valid{\isacharunderscore}{\kern0pt}step\ s\isactrlsub i\ s\isactrlsub j{\isachardoublequoteclose}\isanewline
%
\isadelimproof
%
\endisadelimproof
%
\isatagproof
\isacommand{proof}\isamarkupfalse%
\ {\isacharminus}{\kern0pt}\isanewline
\ \ \isacommand{assume}\isamarkupfalse%
\ assms{\isacharcolon}{\kern0pt}\ {\isachardoublequoteopen}knights{\isacharunderscore}{\kern0pt}path\ b\ ps{\isachardoublequoteclose}\ {\isachardoublequoteopen}step{\isacharunderscore}{\kern0pt}in\ ps\ s\isactrlsub i\ s\isactrlsub j{\isachardoublequoteclose}\isanewline
\ \ \isacommand{then}\isamarkupfalse%
\ \isacommand{obtain}\isamarkupfalse%
\ k\ \isakeyword{where}\ k{\isacharunderscore}{\kern0pt}prems{\isacharcolon}{\kern0pt}\ {\isachardoublequoteopen}{\isadigit{0}}\ {\isacharless}{\kern0pt}\ k\ {\isasymand}\ k\ {\isacharless}{\kern0pt}\ length\ ps{\isachardoublequoteclose}\ {\isachardoublequoteopen}last\ {\isacharparenleft}{\kern0pt}take\ k\ ps{\isacharparenright}{\kern0pt}\ {\isacharequal}{\kern0pt}\ s\isactrlsub i{\isachardoublequoteclose}\ {\isachardoublequoteopen}hd\ {\isacharparenleft}{\kern0pt}drop\ k\ ps{\isacharparenright}{\kern0pt}\ {\isacharequal}{\kern0pt}\ s\isactrlsub j{\isachardoublequoteclose}\isanewline
\ \ \ \ \isacommand{unfolding}\isamarkupfalse%
\ step{\isacharunderscore}{\kern0pt}in{\isacharunderscore}{\kern0pt}def\ \isacommand{by}\isamarkupfalse%
\ auto\isanewline
\ \ \isacommand{then}\isamarkupfalse%
\ \isacommand{have}\isamarkupfalse%
\ {\isachardoublequoteopen}k\ {\isacharequal}{\kern0pt}\ {\isadigit{1}}\ {\isasymor}\ k\ {\isachargreater}{\kern0pt}\ {\isadigit{1}}{\isachardoublequoteclose}\ \isacommand{by}\isamarkupfalse%
\ auto\isanewline
\ \ \isacommand{then}\isamarkupfalse%
\ \isacommand{show}\isamarkupfalse%
\ {\isacharquery}{\kern0pt}thesis\isanewline
\ \ \isacommand{proof}\isamarkupfalse%
\ {\isacharparenleft}{\kern0pt}elim\ disjE{\isacharparenright}{\kern0pt}\isanewline
\ \ \ \ \isacommand{assume}\isamarkupfalse%
\ {\isachardoublequoteopen}k\ {\isacharequal}{\kern0pt}\ {\isadigit{1}}{\isachardoublequoteclose}\isanewline
\ \ \ \ \isacommand{then}\isamarkupfalse%
\ \isacommand{obtain}\isamarkupfalse%
\ ps{\isacharprime}{\kern0pt}\ \isakeyword{where}\ {\isachardoublequoteopen}ps\ {\isacharequal}{\kern0pt}\ s\isactrlsub i{\isacharhash}{\kern0pt}s\isactrlsub j{\isacharhash}{\kern0pt}ps{\isacharprime}{\kern0pt}{\isachardoublequoteclose}\isanewline
\ \ \ \ \ \ \isacommand{using}\isamarkupfalse%
\ k{\isacharunderscore}{\kern0pt}prems\ list{\isacharunderscore}{\kern0pt}len{\isacharunderscore}{\kern0pt}g{\isacharunderscore}{\kern0pt}{\isadigit{1}}{\isacharunderscore}{\kern0pt}split\ \isacommand{by}\isamarkupfalse%
\ fastforce\isanewline
\ \ \ \ \isacommand{then}\isamarkupfalse%
\ \isacommand{show}\isamarkupfalse%
\ {\isacharquery}{\kern0pt}thesis\isanewline
\ \ \ \ \ \ \isacommand{using}\isamarkupfalse%
\ assms\ \isacommand{by}\isamarkupfalse%
\ {\isacharparenleft}{\kern0pt}auto\ elim{\isacharcolon}{\kern0pt}\ knights{\isacharunderscore}{\kern0pt}path{\isachardot}{\kern0pt}cases{\isacharparenright}{\kern0pt}\isanewline
\ \ \isacommand{next}\isamarkupfalse%
\isanewline
\ \ \ \ \isacommand{assume}\isamarkupfalse%
\ {\isachardoublequoteopen}k\ {\isachargreater}{\kern0pt}\ {\isadigit{1}}{\isachardoublequoteclose}\isanewline
\ \ \ \ \isacommand{then}\isamarkupfalse%
\ \isacommand{have}\isamarkupfalse%
\ {\isachardoublequoteopen}{\isadigit{0}}\ {\isacharless}{\kern0pt}\ k{\isacharminus}{\kern0pt}{\isadigit{1}}\ {\isasymand}\ k{\isacharminus}{\kern0pt}{\isadigit{1}}\ {\isacharless}{\kern0pt}\ length\ ps{\isachardoublequoteclose}\isanewline
\ \ \ \ \ \ \isacommand{using}\isamarkupfalse%
\ k{\isacharunderscore}{\kern0pt}prems\ \isacommand{by}\isamarkupfalse%
\ auto\isanewline
\ \ \ \ \isacommand{then}\isamarkupfalse%
\ \isacommand{obtain}\isamarkupfalse%
\ b\ \isakeyword{where}\ {\isachardoublequoteopen}knights{\isacharunderscore}{\kern0pt}path\ b\ {\isacharparenleft}{\kern0pt}drop\ {\isacharparenleft}{\kern0pt}k{\isacharminus}{\kern0pt}{\isadigit{1}}{\isacharparenright}{\kern0pt}\ ps{\isacharparenright}{\kern0pt}{\isachardoublequoteclose}\isanewline
\ \ \ \ \ \ \isacommand{using}\isamarkupfalse%
\ assms\ knights{\isacharunderscore}{\kern0pt}path{\isacharunderscore}{\kern0pt}split\ \isacommand{by}\isamarkupfalse%
\ blast\isanewline
\isanewline
\ \ \ \ \isacommand{obtain}\isamarkupfalse%
\ ps{\isacharprime}{\kern0pt}\ \isakeyword{where}\ {\isachardoublequoteopen}drop\ {\isacharparenleft}{\kern0pt}k{\isacharminus}{\kern0pt}{\isadigit{1}}{\isacharparenright}{\kern0pt}\ ps\ {\isacharequal}{\kern0pt}\ s\isactrlsub i{\isacharhash}{\kern0pt}s\isactrlsub j{\isacharhash}{\kern0pt}ps{\isacharprime}{\kern0pt}{\isachardoublequoteclose}\isanewline
\ \ \ \ \ \ \isacommand{using}\isamarkupfalse%
\ k{\isacharunderscore}{\kern0pt}prems\ {\isacartoucheopen}{\isadigit{0}}\ {\isacharless}{\kern0pt}\ k\ {\isacharminus}{\kern0pt}\ {\isadigit{1}}\ {\isasymand}\ k\ {\isacharminus}{\kern0pt}\ {\isadigit{1}}\ {\isacharless}{\kern0pt}\ length\ ps{\isacartoucheclose}\isanewline
\ \ \ \ \ \ \isacommand{by}\isamarkupfalse%
\ {\isacharparenleft}{\kern0pt}metis\ Cons{\isacharunderscore}{\kern0pt}nth{\isacharunderscore}{\kern0pt}drop{\isacharunderscore}{\kern0pt}Suc\ Suc{\isacharunderscore}{\kern0pt}diff{\isacharunderscore}{\kern0pt}{\isadigit{1}}\ hd{\isacharunderscore}{\kern0pt}drop{\isacharunderscore}{\kern0pt}conv{\isacharunderscore}{\kern0pt}nth\ last{\isacharunderscore}{\kern0pt}snoc\ take{\isacharunderscore}{\kern0pt}hd{\isacharunderscore}{\kern0pt}drop{\isacharparenright}{\kern0pt}\isanewline
\ \ \ \ \isacommand{then}\isamarkupfalse%
\ \isacommand{show}\isamarkupfalse%
\ {\isacharquery}{\kern0pt}thesis\isanewline
\ \ \ \ \ \ \isacommand{using}\isamarkupfalse%
\ {\isacartoucheopen}knights{\isacharunderscore}{\kern0pt}path\ b\ {\isacharparenleft}{\kern0pt}drop\ {\isacharparenleft}{\kern0pt}k{\isacharminus}{\kern0pt}{\isadigit{1}}{\isacharparenright}{\kern0pt}\ ps{\isacharparenright}{\kern0pt}{\isacartoucheclose}\ \isacommand{by}\isamarkupfalse%
\ {\isacharparenleft}{\kern0pt}auto\ elim{\isacharcolon}{\kern0pt}\ knights{\isacharunderscore}{\kern0pt}path{\isachardot}{\kern0pt}cases{\isacharparenright}{\kern0pt}\isanewline
\ \ \isacommand{qed}\isamarkupfalse%
\isanewline
\isacommand{qed}\isamarkupfalse%
%
\endisatagproof
{\isafoldproof}%
%
\isadelimproof
\isanewline
%
\endisadelimproof
\isanewline
\isacommand{lemma}\isamarkupfalse%
\ trans{\isacharunderscore}{\kern0pt}step{\isacharunderscore}{\kern0pt}in{\isacharcolon}{\kern0pt}\ \isanewline
\ \ {\isachardoublequoteopen}step{\isacharunderscore}{\kern0pt}in\ ps\ {\isacharparenleft}{\kern0pt}i{\isacharcomma}{\kern0pt}j{\isacharparenright}{\kern0pt}\ {\isacharparenleft}{\kern0pt}i{\isacharprime}{\kern0pt}{\isacharcomma}{\kern0pt}j{\isacharprime}{\kern0pt}{\isacharparenright}{\kern0pt}\ {\isasymLongrightarrow}\ step{\isacharunderscore}{\kern0pt}in\ {\isacharparenleft}{\kern0pt}trans{\isacharunderscore}{\kern0pt}path\ {\isacharparenleft}{\kern0pt}k\isactrlsub {\isadigit{1}}{\isacharcomma}{\kern0pt}k\isactrlsub {\isadigit{2}}{\isacharparenright}{\kern0pt}\ ps{\isacharparenright}{\kern0pt}\ {\isacharparenleft}{\kern0pt}i{\isacharplus}{\kern0pt}k\isactrlsub {\isadigit{1}}{\isacharcomma}{\kern0pt}j{\isacharplus}{\kern0pt}k\isactrlsub {\isadigit{2}}{\isacharparenright}{\kern0pt}\ {\isacharparenleft}{\kern0pt}i{\isacharprime}{\kern0pt}{\isacharplus}{\kern0pt}k\isactrlsub {\isadigit{1}}{\isacharcomma}{\kern0pt}j{\isacharprime}{\kern0pt}{\isacharplus}{\kern0pt}k\isactrlsub {\isadigit{2}}{\isacharparenright}{\kern0pt}{\isachardoublequoteclose}\isanewline
%
\isadelimproof
%
\endisadelimproof
%
\isatagproof
\isacommand{proof}\isamarkupfalse%
\ {\isacharminus}{\kern0pt}\isanewline
\ \ \isacommand{let}\isamarkupfalse%
\ {\isacharquery}{\kern0pt}ps{\isacharprime}{\kern0pt}{\isacharequal}{\kern0pt}{\isachardoublequoteopen}trans{\isacharunderscore}{\kern0pt}path\ {\isacharparenleft}{\kern0pt}k\isactrlsub {\isadigit{1}}{\isacharcomma}{\kern0pt}k\isactrlsub {\isadigit{2}}{\isacharparenright}{\kern0pt}\ ps{\isachardoublequoteclose}\isanewline
\ \ \isacommand{assume}\isamarkupfalse%
\ {\isachardoublequoteopen}step{\isacharunderscore}{\kern0pt}in\ ps\ {\isacharparenleft}{\kern0pt}i{\isacharcomma}{\kern0pt}j{\isacharparenright}{\kern0pt}\ {\isacharparenleft}{\kern0pt}i{\isacharprime}{\kern0pt}{\isacharcomma}{\kern0pt}j{\isacharprime}{\kern0pt}{\isacharparenright}{\kern0pt}{\isachardoublequoteclose}\isanewline
\ \ \isacommand{then}\isamarkupfalse%
\ \isacommand{obtain}\isamarkupfalse%
\ k\ \isakeyword{where}\ {\isachardoublequoteopen}{\isadigit{0}}\ {\isacharless}{\kern0pt}\ k\ {\isasymand}\ k\ {\isacharless}{\kern0pt}\ length\ ps{\isachardoublequoteclose}\ {\isachardoublequoteopen}last\ {\isacharparenleft}{\kern0pt}take\ k\ ps{\isacharparenright}{\kern0pt}\ {\isacharequal}{\kern0pt}\ {\isacharparenleft}{\kern0pt}i{\isacharcomma}{\kern0pt}j{\isacharparenright}{\kern0pt}{\isachardoublequoteclose}\ {\isachardoublequoteopen}hd\ {\isacharparenleft}{\kern0pt}drop\ k\ ps{\isacharparenright}{\kern0pt}\ {\isacharequal}{\kern0pt}\ {\isacharparenleft}{\kern0pt}i{\isacharprime}{\kern0pt}{\isacharcomma}{\kern0pt}j{\isacharprime}{\kern0pt}{\isacharparenright}{\kern0pt}{\isachardoublequoteclose}\isanewline
\ \ \ \ \isacommand{unfolding}\isamarkupfalse%
\ step{\isacharunderscore}{\kern0pt}in{\isacharunderscore}{\kern0pt}def\ \isacommand{by}\isamarkupfalse%
\ auto\isanewline
\ \ \isacommand{then}\isamarkupfalse%
\ \isacommand{have}\isamarkupfalse%
\ {\isachardoublequoteopen}take\ k\ ps\ {\isasymnoteq}\ {\isacharbrackleft}{\kern0pt}{\isacharbrackright}{\kern0pt}{\isachardoublequoteclose}\ {\isachardoublequoteopen}drop\ k\ ps\ {\isasymnoteq}\ {\isacharbrackleft}{\kern0pt}{\isacharbrackright}{\kern0pt}{\isachardoublequoteclose}\ \isacommand{by}\isamarkupfalse%
\ fastforce{\isacharplus}{\kern0pt}\isanewline
\ \ \isacommand{then}\isamarkupfalse%
\ \isacommand{have}\isamarkupfalse%
\ {\isachardoublequoteopen}{\isadigit{0}}\ {\isacharless}{\kern0pt}\ k\ {\isasymand}\ k\ {\isacharless}{\kern0pt}\ length\ {\isacharquery}{\kern0pt}ps{\isacharprime}{\kern0pt}{\isachardoublequoteclose}\ \isanewline
\ \ \ \ \ \ {\isachardoublequoteopen}last\ {\isacharparenleft}{\kern0pt}take\ k\ {\isacharquery}{\kern0pt}ps{\isacharprime}{\kern0pt}{\isacharparenright}{\kern0pt}\ {\isacharequal}{\kern0pt}\ {\isacharparenleft}{\kern0pt}i{\isacharplus}{\kern0pt}k\isactrlsub {\isadigit{1}}{\isacharcomma}{\kern0pt}j{\isacharplus}{\kern0pt}k\isactrlsub {\isadigit{2}}{\isacharparenright}{\kern0pt}{\isachardoublequoteclose}\ {\isachardoublequoteopen}hd\ {\isacharparenleft}{\kern0pt}drop\ k\ {\isacharquery}{\kern0pt}ps{\isacharprime}{\kern0pt}{\isacharparenright}{\kern0pt}\ {\isacharequal}{\kern0pt}\ {\isacharparenleft}{\kern0pt}i{\isacharprime}{\kern0pt}{\isacharplus}{\kern0pt}k\isactrlsub {\isadigit{1}}{\isacharcomma}{\kern0pt}j{\isacharprime}{\kern0pt}{\isacharplus}{\kern0pt}k\isactrlsub {\isadigit{2}}{\isacharparenright}{\kern0pt}{\isachardoublequoteclose}\isanewline
\ \ \ \ \isacommand{using}\isamarkupfalse%
\ trans{\isacharunderscore}{\kern0pt}path{\isacharunderscore}{\kern0pt}length\isanewline
\ \ \ \ \ \ \ \ \ \ last{\isacharunderscore}{\kern0pt}trans{\isacharunderscore}{\kern0pt}path{\isacharbrackleft}{\kern0pt}OF\ {\isacartoucheopen}take\ k\ ps\ {\isasymnoteq}\ {\isacharbrackleft}{\kern0pt}{\isacharbrackright}{\kern0pt}{\isacartoucheclose}\ {\isacartoucheopen}last\ {\isacharparenleft}{\kern0pt}take\ k\ ps{\isacharparenright}{\kern0pt}\ {\isacharequal}{\kern0pt}\ {\isacharparenleft}{\kern0pt}i{\isacharcomma}{\kern0pt}j{\isacharparenright}{\kern0pt}{\isacartoucheclose}{\isacharbrackright}{\kern0pt}\ take{\isacharunderscore}{\kern0pt}trans\ \isanewline
\ \ \ \ \ \ \ \ \ \ hd{\isacharunderscore}{\kern0pt}trans{\isacharunderscore}{\kern0pt}path{\isacharbrackleft}{\kern0pt}OF\ {\isacartoucheopen}drop\ k\ ps\ {\isasymnoteq}\ {\isacharbrackleft}{\kern0pt}{\isacharbrackright}{\kern0pt}{\isacartoucheclose}\ {\isacartoucheopen}hd\ {\isacharparenleft}{\kern0pt}drop\ k\ ps{\isacharparenright}{\kern0pt}\ {\isacharequal}{\kern0pt}\ {\isacharparenleft}{\kern0pt}i{\isacharprime}{\kern0pt}{\isacharcomma}{\kern0pt}j{\isacharprime}{\kern0pt}{\isacharparenright}{\kern0pt}{\isacartoucheclose}{\isacharbrackright}{\kern0pt}\ drop{\isacharunderscore}{\kern0pt}trans\isanewline
\ \ \ \ \isacommand{by}\isamarkupfalse%
\ auto\isanewline
\ \ \isacommand{then}\isamarkupfalse%
\ \isacommand{show}\isamarkupfalse%
\ {\isacharquery}{\kern0pt}thesis\isanewline
\ \ \ \ \isacommand{by}\isamarkupfalse%
\ {\isacharparenleft}{\kern0pt}auto\ simp{\isacharcolon}{\kern0pt}\ step{\isacharunderscore}{\kern0pt}in{\isacharunderscore}{\kern0pt}def{\isacharparenright}{\kern0pt}\ \isanewline
\isacommand{qed}\isamarkupfalse%
%
\endisatagproof
{\isafoldproof}%
%
\isadelimproof
\isanewline
%
\endisadelimproof
\isanewline
\isacommand{lemma}\isamarkupfalse%
\ transpose{\isacharunderscore}{\kern0pt}step{\isacharunderscore}{\kern0pt}in{\isacharcolon}{\kern0pt}\ \isanewline
\ \ {\isachardoublequoteopen}step{\isacharunderscore}{\kern0pt}in\ ps\ s\isactrlsub i\ s\isactrlsub j\ {\isasymLongrightarrow}\ step{\isacharunderscore}{\kern0pt}in\ {\isacharparenleft}{\kern0pt}transpose\ ps{\isacharparenright}{\kern0pt}\ {\isacharparenleft}{\kern0pt}transpose{\isacharunderscore}{\kern0pt}square\ s\isactrlsub i{\isacharparenright}{\kern0pt}\ {\isacharparenleft}{\kern0pt}transpose{\isacharunderscore}{\kern0pt}square\ s\isactrlsub j{\isacharparenright}{\kern0pt}{\isachardoublequoteclose}\isanewline
\ \ {\isacharparenleft}{\kern0pt}\isakeyword{is}\ {\isachardoublequoteopen}{\isacharunderscore}{\kern0pt}\ {\isasymLongrightarrow}\ step{\isacharunderscore}{\kern0pt}in\ {\isacharquery}{\kern0pt}psT\ {\isacharquery}{\kern0pt}s\isactrlsub iT\ {\isacharquery}{\kern0pt}s\isactrlsub jT{\isachardoublequoteclose}{\isacharparenright}{\kern0pt}\isanewline
%
\isadelimproof
%
\endisadelimproof
%
\isatagproof
\isacommand{proof}\isamarkupfalse%
\ {\isacharminus}{\kern0pt}\isanewline
\ \ \isacommand{assume}\isamarkupfalse%
\ {\isachardoublequoteopen}step{\isacharunderscore}{\kern0pt}in\ ps\ s\isactrlsub i\ s\isactrlsub j{\isachardoublequoteclose}\isanewline
\ \ \isacommand{then}\isamarkupfalse%
\ \isacommand{obtain}\isamarkupfalse%
\ k\ \isakeyword{where}\ \isanewline
\ \ \ \ \ \ k{\isacharunderscore}{\kern0pt}prems{\isacharcolon}{\kern0pt}\ {\isachardoublequoteopen}{\isadigit{0}}\ {\isacharless}{\kern0pt}\ k{\isachardoublequoteclose}\ {\isachardoublequoteopen}k\ {\isacharless}{\kern0pt}\ length\ ps{\isachardoublequoteclose}\ {\isachardoublequoteopen}last\ {\isacharparenleft}{\kern0pt}take\ k\ ps{\isacharparenright}{\kern0pt}\ {\isacharequal}{\kern0pt}\ s\isactrlsub i{\isachardoublequoteclose}\ {\isachardoublequoteopen}hd\ {\isacharparenleft}{\kern0pt}drop\ k\ ps{\isacharparenright}{\kern0pt}\ {\isacharequal}{\kern0pt}\ s\isactrlsub j{\isachardoublequoteclose}\isanewline
\ \ \ \ \isacommand{unfolding}\isamarkupfalse%
\ step{\isacharunderscore}{\kern0pt}in{\isacharunderscore}{\kern0pt}def\ \isacommand{by}\isamarkupfalse%
\ auto\isanewline
\ \ \isacommand{then}\isamarkupfalse%
\ \isacommand{have}\isamarkupfalse%
\ non{\isacharunderscore}{\kern0pt}nil{\isacharcolon}{\kern0pt}\ {\isachardoublequoteopen}take\ k\ ps\ {\isasymnoteq}\ {\isacharbrackleft}{\kern0pt}{\isacharbrackright}{\kern0pt}{\isachardoublequoteclose}\ {\isachardoublequoteopen}drop\ k\ ps\ {\isasymnoteq}\ {\isacharbrackleft}{\kern0pt}{\isacharbrackright}{\kern0pt}{\isachardoublequoteclose}\ \isacommand{by}\isamarkupfalse%
\ fastforce{\isacharplus}{\kern0pt}\isanewline
\ \ \isacommand{have}\isamarkupfalse%
\ {\isachardoublequoteopen}take\ k\ {\isacharquery}{\kern0pt}psT\ {\isacharequal}{\kern0pt}\ transpose\ {\isacharparenleft}{\kern0pt}take\ k\ ps{\isacharparenright}{\kern0pt}{\isachardoublequoteclose}\ {\isachardoublequoteopen}drop\ k\ {\isacharquery}{\kern0pt}psT\ {\isacharequal}{\kern0pt}\ transpose\ {\isacharparenleft}{\kern0pt}drop\ k\ ps{\isacharparenright}{\kern0pt}{\isachardoublequoteclose}\isanewline
\ \ \ \ \isacommand{using}\isamarkupfalse%
\ take{\isacharunderscore}{\kern0pt}transpose\ drop{\isacharunderscore}{\kern0pt}transpose\ \isacommand{by}\isamarkupfalse%
\ auto\isanewline
\ \ \isacommand{then}\isamarkupfalse%
\ \isacommand{have}\isamarkupfalse%
\ {\isachardoublequoteopen}last\ {\isacharparenleft}{\kern0pt}take\ k\ {\isacharquery}{\kern0pt}psT{\isacharparenright}{\kern0pt}\ {\isacharequal}{\kern0pt}\ {\isacharquery}{\kern0pt}s\isactrlsub iT{\isachardoublequoteclose}\ {\isachardoublequoteopen}hd\ {\isacharparenleft}{\kern0pt}drop\ k\ {\isacharquery}{\kern0pt}psT{\isacharparenright}{\kern0pt}\ {\isacharequal}{\kern0pt}\ {\isacharquery}{\kern0pt}s\isactrlsub jT{\isachardoublequoteclose}\isanewline
\ \ \ \ \isacommand{using}\isamarkupfalse%
\ non{\isacharunderscore}{\kern0pt}nil\ k{\isacharunderscore}{\kern0pt}prems\ hd{\isacharunderscore}{\kern0pt}transpose\ last{\isacharunderscore}{\kern0pt}transpose\ \isacommand{by}\isamarkupfalse%
\ auto\isanewline
\ \ \isacommand{then}\isamarkupfalse%
\ \isacommand{show}\isamarkupfalse%
\ {\isachardoublequoteopen}step{\isacharunderscore}{\kern0pt}in\ {\isacharquery}{\kern0pt}psT\ {\isacharquery}{\kern0pt}s\isactrlsub iT\ {\isacharquery}{\kern0pt}s\isactrlsub jT{\isachardoublequoteclose}\isanewline
\ \ \ \ \isacommand{unfolding}\isamarkupfalse%
\ step{\isacharunderscore}{\kern0pt}in{\isacharunderscore}{\kern0pt}def\ \isacommand{using}\isamarkupfalse%
\ k{\isacharunderscore}{\kern0pt}prems\ transpose{\isacharunderscore}{\kern0pt}length\ \isacommand{by}\isamarkupfalse%
\ auto\isanewline
\isacommand{qed}\isamarkupfalse%
%
\endisatagproof
{\isafoldproof}%
%
\isadelimproof
\isanewline
%
\endisadelimproof
\ \ \isanewline
\isacommand{lemma}\isamarkupfalse%
\ hd{\isacharunderscore}{\kern0pt}take{\isacharcolon}{\kern0pt}\ {\isachardoublequoteopen}{\isadigit{0}}\ {\isacharless}{\kern0pt}\ k\ {\isasymLongrightarrow}\ hd\ xs\ {\isacharequal}{\kern0pt}\ hd\ {\isacharparenleft}{\kern0pt}take\ k\ xs{\isacharparenright}{\kern0pt}{\isachardoublequoteclose}\isanewline
%
\isadelimproof
\ \ %
\endisadelimproof
%
\isatagproof
\isacommand{by}\isamarkupfalse%
\ {\isacharparenleft}{\kern0pt}induction\ xs{\isacharparenright}{\kern0pt}\ auto%
\endisatagproof
{\isafoldproof}%
%
\isadelimproof
\isanewline
%
\endisadelimproof
\isanewline
\isacommand{lemma}\isamarkupfalse%
\ last{\isacharunderscore}{\kern0pt}drop{\isacharcolon}{\kern0pt}\ {\isachardoublequoteopen}k\ {\isacharless}{\kern0pt}\ length\ xs\ {\isasymLongrightarrow}\ last\ xs\ {\isacharequal}{\kern0pt}\ last\ {\isacharparenleft}{\kern0pt}drop\ k\ xs{\isacharparenright}{\kern0pt}{\isachardoublequoteclose}\isanewline
%
\isadelimproof
\ \ %
\endisadelimproof
%
\isatagproof
\isacommand{by}\isamarkupfalse%
\ {\isacharparenleft}{\kern0pt}induction\ xs{\isacharparenright}{\kern0pt}\ auto%
\endisatagproof
{\isafoldproof}%
%
\isadelimproof
%
\endisadelimproof
%
\isadelimdocument
%
\endisadelimdocument
%
\isatagdocument
%
\isamarkupsubsection{Concatenate Knight's Paths and Circuits%
}
\isamarkuptrue%
%
\endisatagdocument
{\isafolddocument}%
%
\isadelimdocument
%
\endisadelimdocument
%
\begin{isamarkuptext}%
Concatenate two knight's path on a \isa{n{\isasymtimes}m}-board along the 2nd axis if the first path contains
the step \isa{s\isactrlsub i\ {\isasymrightarrow}\ s\isactrlsub j} and there are valid steps \isa{s\isactrlsub i\ {\isasymrightarrow}\ hd\ ps\isactrlsub {\isadigit{2}}{\isacharprime}{\kern0pt}} and \isa{s\isactrlsub j\ {\isasymrightarrow}\ last\ ps\isactrlsub {\isadigit{2}}{\isacharprime}{\kern0pt}}, where 
\isa{ps\isactrlsub {\isadigit{2}}{\isacharprime}{\kern0pt}} is \isa{ps\isactrlsub {\isadigit{2}}} is translated by \isa{m\isactrlsub {\isadigit{1}}}. An arbitrary step in \isa{ps\isactrlsub {\isadigit{2}}} is preserved.%
\end{isamarkuptext}\isamarkuptrue%
\isacommand{corollary}\isamarkupfalse%
\ knights{\isacharunderscore}{\kern0pt}path{\isacharunderscore}{\kern0pt}split{\isacharunderscore}{\kern0pt}concat{\isacharunderscore}{\kern0pt}si{\isacharunderscore}{\kern0pt}prev{\isacharcolon}{\kern0pt}\isanewline
\ \ \isakeyword{assumes}\ {\isachardoublequoteopen}knights{\isacharunderscore}{\kern0pt}path\ {\isacharparenleft}{\kern0pt}board\ n\ m\isactrlsub {\isadigit{1}}{\isacharparenright}{\kern0pt}\ ps\isactrlsub {\isadigit{1}}{\isachardoublequoteclose}\ {\isachardoublequoteopen}knights{\isacharunderscore}{\kern0pt}path\ {\isacharparenleft}{\kern0pt}board\ n\ m\isactrlsub {\isadigit{2}}{\isacharparenright}{\kern0pt}\ ps\isactrlsub {\isadigit{2}}{\isachardoublequoteclose}\ \isanewline
\ \ \ \ \ \ \ \ \ \ {\isachardoublequoteopen}step{\isacharunderscore}{\kern0pt}in\ ps\isactrlsub {\isadigit{1}}\ s\isactrlsub i\ s\isactrlsub j{\isachardoublequoteclose}\ {\isachardoublequoteopen}hd\ ps\isactrlsub {\isadigit{2}}\ {\isacharequal}{\kern0pt}\ {\isacharparenleft}{\kern0pt}i\isactrlsub h{\isacharcomma}{\kern0pt}j\isactrlsub h{\isacharparenright}{\kern0pt}{\isachardoublequoteclose}\ {\isachardoublequoteopen}last\ ps\isactrlsub {\isadigit{2}}\ {\isacharequal}{\kern0pt}\ {\isacharparenleft}{\kern0pt}i\isactrlsub l{\isacharcomma}{\kern0pt}j\isactrlsub l{\isacharparenright}{\kern0pt}{\isachardoublequoteclose}\ {\isachardoublequoteopen}step{\isacharunderscore}{\kern0pt}in\ ps\isactrlsub {\isadigit{2}}\ {\isacharparenleft}{\kern0pt}i{\isacharcomma}{\kern0pt}j{\isacharparenright}{\kern0pt}\ {\isacharparenleft}{\kern0pt}i{\isacharprime}{\kern0pt}{\isacharcomma}{\kern0pt}j{\isacharprime}{\kern0pt}{\isacharparenright}{\kern0pt}{\isachardoublequoteclose}\isanewline
\ \ \ \ \ \ \ \ \ \ {\isachardoublequoteopen}valid{\isacharunderscore}{\kern0pt}step\ s\isactrlsub i\ {\isacharparenleft}{\kern0pt}i\isactrlsub h{\isacharcomma}{\kern0pt}int\ m\isactrlsub {\isadigit{1}}{\isacharplus}{\kern0pt}j\isactrlsub h{\isacharparenright}{\kern0pt}{\isachardoublequoteclose}\ {\isachardoublequoteopen}valid{\isacharunderscore}{\kern0pt}step\ {\isacharparenleft}{\kern0pt}i\isactrlsub l{\isacharcomma}{\kern0pt}int\ m\isactrlsub {\isadigit{1}}{\isacharplus}{\kern0pt}j\isactrlsub l{\isacharparenright}{\kern0pt}\ s\isactrlsub j{\isachardoublequoteclose}\isanewline
\ \ \isakeyword{shows}\ {\isachardoublequoteopen}{\isasymexists}ps{\isachardot}{\kern0pt}\ knights{\isacharunderscore}{\kern0pt}path\ {\isacharparenleft}{\kern0pt}board\ n\ {\isacharparenleft}{\kern0pt}m\isactrlsub {\isadigit{1}}{\isacharplus}{\kern0pt}m\isactrlsub {\isadigit{2}}{\isacharparenright}{\kern0pt}{\isacharparenright}{\kern0pt}\ ps\ {\isasymand}\ hd\ ps\ {\isacharequal}{\kern0pt}\ hd\ ps\isactrlsub {\isadigit{1}}\ \ \isanewline
\ \ \ \ {\isasymand}\ last\ ps\ {\isacharequal}{\kern0pt}\ last\ ps\isactrlsub {\isadigit{1}}\ {\isasymand}\ step{\isacharunderscore}{\kern0pt}in\ ps\ {\isacharparenleft}{\kern0pt}i{\isacharcomma}{\kern0pt}int\ m\isactrlsub {\isadigit{1}}{\isacharplus}{\kern0pt}j{\isacharparenright}{\kern0pt}\ {\isacharparenleft}{\kern0pt}i{\isacharprime}{\kern0pt}{\isacharcomma}{\kern0pt}int\ m\isactrlsub {\isadigit{1}}{\isacharplus}{\kern0pt}j{\isacharprime}{\kern0pt}{\isacharparenright}{\kern0pt}{\isachardoublequoteclose}\isanewline
%
\isadelimproof
\ \ %
\endisadelimproof
%
\isatagproof
\isacommand{using}\isamarkupfalse%
\ assms\isanewline
\isacommand{proof}\isamarkupfalse%
\ {\isacharminus}{\kern0pt}\isanewline
\ \ \isacommand{let}\isamarkupfalse%
\ {\isacharquery}{\kern0pt}b\isactrlsub {\isadigit{1}}{\isacharequal}{\kern0pt}{\isachardoublequoteopen}board\ n\ m\isactrlsub {\isadigit{1}}{\isachardoublequoteclose}\isanewline
\ \ \isacommand{let}\isamarkupfalse%
\ {\isacharquery}{\kern0pt}b\isactrlsub {\isadigit{2}}{\isacharequal}{\kern0pt}{\isachardoublequoteopen}board\ n\ m\isactrlsub {\isadigit{2}}{\isachardoublequoteclose}\isanewline
\ \ \isacommand{let}\isamarkupfalse%
\ {\isacharquery}{\kern0pt}ps\isactrlsub {\isadigit{2}}{\isacharprime}{\kern0pt}{\isacharequal}{\kern0pt}{\isachardoublequoteopen}trans{\isacharunderscore}{\kern0pt}path\ {\isacharparenleft}{\kern0pt}{\isadigit{0}}{\isacharcomma}{\kern0pt}int\ m\isactrlsub {\isadigit{1}}{\isacharparenright}{\kern0pt}\ ps\isactrlsub {\isadigit{2}}{\isachardoublequoteclose}\isanewline
\ \ \isacommand{let}\isamarkupfalse%
\ {\isacharquery}{\kern0pt}b{\isacharprime}{\kern0pt}{\isacharequal}{\kern0pt}{\isachardoublequoteopen}trans{\isacharunderscore}{\kern0pt}board\ {\isacharparenleft}{\kern0pt}{\isadigit{0}}{\isacharcomma}{\kern0pt}int\ m\isactrlsub {\isadigit{1}}{\isacharparenright}{\kern0pt}\ {\isacharquery}{\kern0pt}b\isactrlsub {\isadigit{2}}{\isachardoublequoteclose}\isanewline
\ \ \isacommand{have}\isamarkupfalse%
\ kp{\isadigit{2}}{\isacharprime}{\kern0pt}{\isacharcolon}{\kern0pt}\ {\isachardoublequoteopen}knights{\isacharunderscore}{\kern0pt}path\ {\isacharquery}{\kern0pt}b{\isacharprime}{\kern0pt}\ {\isacharquery}{\kern0pt}ps\isactrlsub {\isadigit{2}}{\isacharprime}{\kern0pt}{\isachardoublequoteclose}\ \isacommand{using}\isamarkupfalse%
\ assms\ trans{\isacharunderscore}{\kern0pt}knights{\isacharunderscore}{\kern0pt}path\ \isacommand{by}\isamarkupfalse%
\ auto\isanewline
\ \ \isacommand{then}\isamarkupfalse%
\ \isacommand{have}\isamarkupfalse%
\ {\isachardoublequoteopen}{\isacharquery}{\kern0pt}ps\isactrlsub {\isadigit{2}}{\isacharprime}{\kern0pt}\ {\isasymnoteq}\ {\isacharbrackleft}{\kern0pt}{\isacharbrackright}{\kern0pt}{\isachardoublequoteclose}\ \isacommand{using}\isamarkupfalse%
\ knights{\isacharunderscore}{\kern0pt}path{\isacharunderscore}{\kern0pt}non{\isacharunderscore}{\kern0pt}nil\ \isacommand{by}\isamarkupfalse%
\ auto\isanewline
\isanewline
\ \ \isacommand{obtain}\isamarkupfalse%
\ k\ \isakeyword{where}\ k{\isacharunderscore}{\kern0pt}prems{\isacharcolon}{\kern0pt}\ \isanewline
\ \ \ \ {\isachardoublequoteopen}{\isadigit{0}}\ {\isacharless}{\kern0pt}\ k{\isachardoublequoteclose}\ {\isachardoublequoteopen}k\ {\isacharless}{\kern0pt}\ length\ ps\isactrlsub {\isadigit{1}}{\isachardoublequoteclose}\ {\isachardoublequoteopen}last\ {\isacharparenleft}{\kern0pt}take\ k\ ps\isactrlsub {\isadigit{1}}{\isacharparenright}{\kern0pt}\ {\isacharequal}{\kern0pt}\ s\isactrlsub i{\isachardoublequoteclose}\ {\isachardoublequoteopen}hd\ {\isacharparenleft}{\kern0pt}drop\ k\ ps\isactrlsub {\isadigit{1}}{\isacharparenright}{\kern0pt}\ {\isacharequal}{\kern0pt}\ s\isactrlsub j{\isachardoublequoteclose}\isanewline
\ \ \ \ \isacommand{using}\isamarkupfalse%
\ assms\ \isacommand{unfolding}\isamarkupfalse%
\ step{\isacharunderscore}{\kern0pt}in{\isacharunderscore}{\kern0pt}def\ \isacommand{by}\isamarkupfalse%
\ auto\isanewline
\ \ \isacommand{let}\isamarkupfalse%
\ {\isacharquery}{\kern0pt}ps{\isacharequal}{\kern0pt}{\isachardoublequoteopen}{\isacharparenleft}{\kern0pt}take\ k\ ps\isactrlsub {\isadigit{1}}{\isacharparenright}{\kern0pt}\ {\isacharat}{\kern0pt}\ {\isacharquery}{\kern0pt}ps\isactrlsub {\isadigit{2}}{\isacharprime}{\kern0pt}\ {\isacharat}{\kern0pt}\ {\isacharparenleft}{\kern0pt}drop\ k\ ps\isactrlsub {\isadigit{1}}{\isacharparenright}{\kern0pt}{\isachardoublequoteclose}\isanewline
\ \ \isacommand{obtain}\isamarkupfalse%
\ b\isactrlsub {\isadigit{1}}\ b\isactrlsub {\isadigit{2}}\ \isakeyword{where}\ b{\isacharunderscore}{\kern0pt}prems{\isacharcolon}{\kern0pt}\ {\isachardoublequoteopen}knights{\isacharunderscore}{\kern0pt}path\ b\isactrlsub {\isadigit{1}}\ {\isacharparenleft}{\kern0pt}take\ k\ ps\isactrlsub {\isadigit{1}}{\isacharparenright}{\kern0pt}{\isachardoublequoteclose}\ {\isachardoublequoteopen}knights{\isacharunderscore}{\kern0pt}path\ b\isactrlsub {\isadigit{2}}\ {\isacharparenleft}{\kern0pt}drop\ k\ ps\isactrlsub {\isadigit{1}}{\isacharparenright}{\kern0pt}{\isachardoublequoteclose}\ \isanewline
\ \ \ \ \ \ {\isachardoublequoteopen}b\isactrlsub {\isadigit{1}}\ {\isasymunion}\ b\isactrlsub {\isadigit{2}}\ {\isacharequal}{\kern0pt}\ {\isacharquery}{\kern0pt}b\isactrlsub {\isadigit{1}}{\isachardoublequoteclose}\ {\isachardoublequoteopen}b\isactrlsub {\isadigit{1}}\ {\isasyminter}\ b\isactrlsub {\isadigit{2}}\ {\isacharequal}{\kern0pt}\ {\isacharbraceleft}{\kern0pt}{\isacharbraceright}{\kern0pt}{\isachardoublequoteclose}\isanewline
\ \ \ \ \isacommand{using}\isamarkupfalse%
\ assms\ {\isacartoucheopen}{\isadigit{0}}\ {\isacharless}{\kern0pt}\ k{\isacartoucheclose}\ {\isacartoucheopen}k\ {\isacharless}{\kern0pt}\ length\ ps\isactrlsub {\isadigit{1}}{\isacartoucheclose}\ knights{\isacharunderscore}{\kern0pt}path{\isacharunderscore}{\kern0pt}split\ \isacommand{by}\isamarkupfalse%
\ blast\isanewline
\isanewline
\ \ \isacommand{have}\isamarkupfalse%
\ {\isachardoublequoteopen}hd\ {\isacharquery}{\kern0pt}ps\isactrlsub {\isadigit{2}}{\isacharprime}{\kern0pt}\ {\isacharequal}{\kern0pt}\ {\isacharparenleft}{\kern0pt}i\isactrlsub h{\isacharcomma}{\kern0pt}int\ m\isactrlsub {\isadigit{1}}{\isacharplus}{\kern0pt}j\isactrlsub h{\isacharparenright}{\kern0pt}{\isachardoublequoteclose}\ {\isachardoublequoteopen}last\ {\isacharquery}{\kern0pt}ps\isactrlsub {\isadigit{2}}{\isacharprime}{\kern0pt}\ {\isacharequal}{\kern0pt}\ {\isacharparenleft}{\kern0pt}i\isactrlsub l{\isacharcomma}{\kern0pt}int\ m\isactrlsub {\isadigit{1}}{\isacharplus}{\kern0pt}j\isactrlsub l{\isacharparenright}{\kern0pt}{\isachardoublequoteclose}\isanewline
\ \ \ \ \isacommand{using}\isamarkupfalse%
\ assms\ knights{\isacharunderscore}{\kern0pt}path{\isacharunderscore}{\kern0pt}non{\isacharunderscore}{\kern0pt}nil\ hd{\isacharunderscore}{\kern0pt}trans{\isacharunderscore}{\kern0pt}path\ last{\isacharunderscore}{\kern0pt}trans{\isacharunderscore}{\kern0pt}path\ \isacommand{by}\isamarkupfalse%
\ auto\isanewline
\ \ \isacommand{then}\isamarkupfalse%
\ \isacommand{have}\isamarkupfalse%
\ {\isachardoublequoteopen}hd\ {\isacharquery}{\kern0pt}ps\isactrlsub {\isadigit{2}}{\isacharprime}{\kern0pt}\ {\isacharequal}{\kern0pt}\ {\isacharparenleft}{\kern0pt}i\isactrlsub h{\isacharcomma}{\kern0pt}int\ m\isactrlsub {\isadigit{1}}{\isacharplus}{\kern0pt}j\isactrlsub h{\isacharparenright}{\kern0pt}{\isachardoublequoteclose}\ {\isachardoublequoteopen}last\ {\isacharparenleft}{\kern0pt}{\isacharparenleft}{\kern0pt}take\ k\ ps\isactrlsub {\isadigit{1}}{\isacharparenright}{\kern0pt}\ {\isacharat}{\kern0pt}\ {\isacharquery}{\kern0pt}ps\isactrlsub {\isadigit{2}}{\isacharprime}{\kern0pt}{\isacharparenright}{\kern0pt}\ {\isacharequal}{\kern0pt}\ {\isacharparenleft}{\kern0pt}i\isactrlsub l{\isacharcomma}{\kern0pt}int\ m\isactrlsub {\isadigit{1}}{\isacharplus}{\kern0pt}j\isactrlsub l{\isacharparenright}{\kern0pt}{\isachardoublequoteclose}\isanewline
\ \ \ \ \isacommand{using}\isamarkupfalse%
\ {\isacartoucheopen}{\isacharquery}{\kern0pt}ps\isactrlsub {\isadigit{2}}{\isacharprime}{\kern0pt}\ {\isasymnoteq}\ {\isacharbrackleft}{\kern0pt}{\isacharbrackright}{\kern0pt}{\isacartoucheclose}\ \isacommand{by}\isamarkupfalse%
\ auto\isanewline
\ \ \isacommand{then}\isamarkupfalse%
\ \isacommand{have}\isamarkupfalse%
\ vs{\isacharcolon}{\kern0pt}\ {\isachardoublequoteopen}valid{\isacharunderscore}{\kern0pt}step\ {\isacharparenleft}{\kern0pt}last\ {\isacharparenleft}{\kern0pt}take\ k\ ps\isactrlsub {\isadigit{1}}{\isacharparenright}{\kern0pt}{\isacharparenright}{\kern0pt}\ {\isacharparenleft}{\kern0pt}hd\ {\isacharquery}{\kern0pt}ps\isactrlsub {\isadigit{2}}{\isacharprime}{\kern0pt}{\isacharparenright}{\kern0pt}{\isachardoublequoteclose}\ \isanewline
\ \ \ \ \ \ {\isachardoublequoteopen}valid{\isacharunderscore}{\kern0pt}step\ {\isacharparenleft}{\kern0pt}last\ {\isacharparenleft}{\kern0pt}{\isacharparenleft}{\kern0pt}take\ k\ ps\isactrlsub {\isadigit{1}}{\isacharparenright}{\kern0pt}\ {\isacharat}{\kern0pt}\ {\isacharquery}{\kern0pt}ps\isactrlsub {\isadigit{2}}{\isacharprime}{\kern0pt}{\isacharparenright}{\kern0pt}{\isacharparenright}{\kern0pt}\ {\isacharparenleft}{\kern0pt}hd\ {\isacharparenleft}{\kern0pt}drop\ k\ ps\isactrlsub {\isadigit{1}}{\isacharparenright}{\kern0pt}{\isacharparenright}{\kern0pt}{\isachardoublequoteclose}\isanewline
\ \ \ \ \isacommand{using}\isamarkupfalse%
\ assms\ k{\isacharunderscore}{\kern0pt}prems\ \isacommand{by}\isamarkupfalse%
\ auto\isanewline
\isanewline
\ \ \isacommand{have}\isamarkupfalse%
\ {\isachardoublequoteopen}{\isacharquery}{\kern0pt}b\isactrlsub {\isadigit{1}}\ {\isasyminter}\ {\isacharquery}{\kern0pt}b{\isacharprime}{\kern0pt}\ {\isacharequal}{\kern0pt}\ {\isacharbraceleft}{\kern0pt}{\isacharbraceright}{\kern0pt}{\isachardoublequoteclose}\ \isacommand{unfolding}\isamarkupfalse%
\ board{\isacharunderscore}{\kern0pt}def\ trans{\isacharunderscore}{\kern0pt}board{\isacharunderscore}{\kern0pt}def\ \isacommand{by}\isamarkupfalse%
\ auto\isanewline
\ \ \isacommand{then}\isamarkupfalse%
\ \isacommand{have}\isamarkupfalse%
\ {\isachardoublequoteopen}b\isactrlsub {\isadigit{1}}\ {\isasyminter}\ {\isacharquery}{\kern0pt}b{\isacharprime}{\kern0pt}\ {\isacharequal}{\kern0pt}\ {\isacharbraceleft}{\kern0pt}{\isacharbraceright}{\kern0pt}\ {\isasymand}\ {\isacharparenleft}{\kern0pt}b\isactrlsub {\isadigit{1}}\ {\isasymunion}\ {\isacharquery}{\kern0pt}b{\isacharprime}{\kern0pt}{\isacharparenright}{\kern0pt}\ {\isasyminter}\ b\isactrlsub {\isadigit{2}}\ {\isacharequal}{\kern0pt}\ {\isacharbraceleft}{\kern0pt}{\isacharbraceright}{\kern0pt}{\isachardoublequoteclose}\ \isacommand{using}\isamarkupfalse%
\ b{\isacharunderscore}{\kern0pt}prems\ \isacommand{by}\isamarkupfalse%
\ blast\isanewline
\ \ \isacommand{then}\isamarkupfalse%
\ \isacommand{have}\isamarkupfalse%
\ inter{\isacharunderscore}{\kern0pt}empty{\isacharcolon}{\kern0pt}\ {\isachardoublequoteopen}b\isactrlsub {\isadigit{1}}\ {\isasyminter}\ {\isacharquery}{\kern0pt}b{\isacharprime}{\kern0pt}\ {\isacharequal}{\kern0pt}\ {\isacharbraceleft}{\kern0pt}{\isacharbraceright}{\kern0pt}{\isachardoublequoteclose}\ {\isachardoublequoteopen}{\isacharparenleft}{\kern0pt}b\isactrlsub {\isadigit{1}}\ {\isasymunion}\ {\isacharquery}{\kern0pt}b{\isacharprime}{\kern0pt}{\isacharparenright}{\kern0pt}\ {\isasyminter}\ b\isactrlsub {\isadigit{2}}\ {\isacharequal}{\kern0pt}\ {\isacharbraceleft}{\kern0pt}{\isacharbraceright}{\kern0pt}{\isachardoublequoteclose}\ \isacommand{by}\isamarkupfalse%
\ auto\isanewline
\isanewline
\ \ \isacommand{have}\isamarkupfalse%
\ {\isachardoublequoteopen}knights{\isacharunderscore}{\kern0pt}path\ {\isacharparenleft}{\kern0pt}b\isactrlsub {\isadigit{1}}\ {\isasymunion}\ {\isacharquery}{\kern0pt}b{\isacharprime}{\kern0pt}{\isacharparenright}{\kern0pt}\ {\isacharparenleft}{\kern0pt}{\isacharparenleft}{\kern0pt}take\ k\ ps\isactrlsub {\isadigit{1}}{\isacharparenright}{\kern0pt}\ {\isacharat}{\kern0pt}\ {\isacharquery}{\kern0pt}ps\isactrlsub {\isadigit{2}}{\isacharprime}{\kern0pt}{\isacharparenright}{\kern0pt}{\isachardoublequoteclose}\isanewline
\ \ \ \ \isacommand{using}\isamarkupfalse%
\ kp{\isadigit{2}}{\isacharprime}{\kern0pt}\ b{\isacharunderscore}{\kern0pt}prems\ inter{\isacharunderscore}{\kern0pt}empty\ vs\ knights{\isacharunderscore}{\kern0pt}path{\isacharunderscore}{\kern0pt}append\ \isacommand{by}\isamarkupfalse%
\ auto\isanewline
\ \ \isacommand{then}\isamarkupfalse%
\ \isacommand{have}\isamarkupfalse%
\ {\isachardoublequoteopen}knights{\isacharunderscore}{\kern0pt}path\ {\isacharparenleft}{\kern0pt}b\isactrlsub {\isadigit{1}}\ {\isasymunion}\ {\isacharquery}{\kern0pt}b{\isacharprime}{\kern0pt}\ {\isasymunion}\ b\isactrlsub {\isadigit{2}}{\isacharparenright}{\kern0pt}\ {\isacharquery}{\kern0pt}ps{\isachardoublequoteclose}\isanewline
\ \ \ \ \isacommand{using}\isamarkupfalse%
\ b{\isacharunderscore}{\kern0pt}prems\ inter{\isacharunderscore}{\kern0pt}empty\ vs\ knights{\isacharunderscore}{\kern0pt}path{\isacharunderscore}{\kern0pt}append{\isacharbrackleft}{\kern0pt}\isakeyword{where}\ ps\isactrlsub {\isadigit{1}}{\isacharequal}{\kern0pt}{\isachardoublequoteopen}{\isacharparenleft}{\kern0pt}take\ k\ ps\isactrlsub {\isadigit{1}}{\isacharparenright}{\kern0pt}\ {\isacharat}{\kern0pt}\ {\isacharquery}{\kern0pt}ps\isactrlsub {\isadigit{2}}{\isacharprime}{\kern0pt}{\isachardoublequoteclose}{\isacharbrackright}{\kern0pt}\ \isacommand{by}\isamarkupfalse%
\ auto\isanewline
\ \ \isacommand{then}\isamarkupfalse%
\ \isacommand{have}\isamarkupfalse%
\ {\isachardoublequoteopen}knights{\isacharunderscore}{\kern0pt}path\ {\isacharparenleft}{\kern0pt}{\isacharquery}{\kern0pt}b\isactrlsub {\isadigit{1}}\ {\isasymunion}\ {\isacharquery}{\kern0pt}b{\isacharprime}{\kern0pt}{\isacharparenright}{\kern0pt}\ {\isacharquery}{\kern0pt}ps{\isachardoublequoteclose}\ \isanewline
\ \ \ \ \isacommand{using}\isamarkupfalse%
\ b{\isacharunderscore}{\kern0pt}prems\ Un{\isacharunderscore}{\kern0pt}commute\ Un{\isacharunderscore}{\kern0pt}assoc\ \isacommand{by}\isamarkupfalse%
\ metis\isanewline
\ \ \isacommand{then}\isamarkupfalse%
\ \isacommand{have}\isamarkupfalse%
\ kp{\isacharcolon}{\kern0pt}\ {\isachardoublequoteopen}knights{\isacharunderscore}{\kern0pt}path\ {\isacharparenleft}{\kern0pt}board\ n\ {\isacharparenleft}{\kern0pt}m\isactrlsub {\isadigit{1}}{\isacharplus}{\kern0pt}m\isactrlsub {\isadigit{2}}{\isacharparenright}{\kern0pt}{\isacharparenright}{\kern0pt}\ {\isacharquery}{\kern0pt}ps{\isachardoublequoteclose}\isanewline
\ \ \ \ \isacommand{using}\isamarkupfalse%
\ board{\isacharunderscore}{\kern0pt}concat{\isacharbrackleft}{\kern0pt}of\ n\ m\isactrlsub {\isadigit{1}}\ m\isactrlsub {\isadigit{2}}{\isacharbrackright}{\kern0pt}\ \isacommand{by}\isamarkupfalse%
\ auto\isanewline
\isanewline
\ \ \isacommand{have}\isamarkupfalse%
\ hd{\isacharcolon}{\kern0pt}\ {\isachardoublequoteopen}hd\ {\isacharquery}{\kern0pt}ps\ {\isacharequal}{\kern0pt}\ hd\ ps\isactrlsub {\isadigit{1}}{\isachardoublequoteclose}\isanewline
\ \ \ \ \isacommand{using}\isamarkupfalse%
\ assms\ {\isacartoucheopen}{\isadigit{0}}\ {\isacharless}{\kern0pt}\ k{\isacartoucheclose}\ knights{\isacharunderscore}{\kern0pt}path{\isacharunderscore}{\kern0pt}non{\isacharunderscore}{\kern0pt}nil\ hd{\isacharunderscore}{\kern0pt}take\ \isacommand{by}\isamarkupfalse%
\ auto\isanewline
\isanewline
\ \ \isacommand{have}\isamarkupfalse%
\ last{\isacharcolon}{\kern0pt}\ {\isachardoublequoteopen}last\ {\isacharquery}{\kern0pt}ps\ {\isacharequal}{\kern0pt}\ last\ ps\isactrlsub {\isadigit{1}}{\isachardoublequoteclose}\isanewline
\ \ \ \ \isacommand{using}\isamarkupfalse%
\ assms\ {\isacartoucheopen}k\ {\isacharless}{\kern0pt}\ length\ ps\isactrlsub {\isadigit{1}}{\isacartoucheclose}\ knights{\isacharunderscore}{\kern0pt}path{\isacharunderscore}{\kern0pt}non{\isacharunderscore}{\kern0pt}nil\ last{\isacharunderscore}{\kern0pt}drop\ \isacommand{by}\isamarkupfalse%
\ auto\isanewline
\isanewline
\ \ \isacommand{have}\isamarkupfalse%
\ m{\isacharunderscore}{\kern0pt}simps{\isacharcolon}{\kern0pt}\ {\isachardoublequoteopen}j{\isacharplus}{\kern0pt}int\ m\isactrlsub {\isadigit{1}}\ {\isacharequal}{\kern0pt}\ int\ m\isactrlsub {\isadigit{1}}{\isacharplus}{\kern0pt}j{\isachardoublequoteclose}\ {\isachardoublequoteopen}j{\isacharprime}{\kern0pt}{\isacharplus}{\kern0pt}int\ m\isactrlsub {\isadigit{1}}\ {\isacharequal}{\kern0pt}\ int\ m\isactrlsub {\isadigit{1}}{\isacharplus}{\kern0pt}j{\isacharprime}{\kern0pt}{\isachardoublequoteclose}\ \isacommand{by}\isamarkupfalse%
\ auto\isanewline
\ \ \isacommand{have}\isamarkupfalse%
\ si{\isacharcolon}{\kern0pt}\ {\isachardoublequoteopen}step{\isacharunderscore}{\kern0pt}in\ {\isacharquery}{\kern0pt}ps\ {\isacharparenleft}{\kern0pt}i{\isacharcomma}{\kern0pt}int\ m\isactrlsub {\isadigit{1}}{\isacharplus}{\kern0pt}j{\isacharparenright}{\kern0pt}\ {\isacharparenleft}{\kern0pt}i{\isacharprime}{\kern0pt}{\isacharcomma}{\kern0pt}int\ m\isactrlsub {\isadigit{1}}{\isacharplus}{\kern0pt}j{\isacharprime}{\kern0pt}{\isacharparenright}{\kern0pt}{\isachardoublequoteclose}\isanewline
\ \ \ \ \isacommand{using}\isamarkupfalse%
\ assms\ step{\isacharunderscore}{\kern0pt}in{\isacharunderscore}{\kern0pt}append{\isacharbrackleft}{\kern0pt}OF\ step{\isacharunderscore}{\kern0pt}in{\isacharunderscore}{\kern0pt}prepend{\isacharbrackleft}{\kern0pt}OF\ trans{\isacharunderscore}{\kern0pt}step{\isacharunderscore}{\kern0pt}in{\isacharbrackright}{\kern0pt}{\isacharcomma}{\kern0pt}\ \isanewline
\ \ \ \ \ \ \ \ \ \ \ \ \ \ \ \ \ \ of\ ps\isactrlsub {\isadigit{2}}\ i\ j\ i{\isacharprime}{\kern0pt}\ j{\isacharprime}{\kern0pt}\ {\isachardoublequoteopen}take\ k\ ps\isactrlsub {\isadigit{1}}{\isachardoublequoteclose}\ {\isadigit{0}}\ {\isachardoublequoteopen}int\ m\isactrlsub {\isadigit{1}}{\isachardoublequoteclose}\ {\isachardoublequoteopen}drop\ k\ ps\isactrlsub {\isadigit{1}}{\isachardoublequoteclose}{\isacharbrackright}{\kern0pt}\ \isanewline
\ \ \ \ \isacommand{by}\isamarkupfalse%
\ {\isacharparenleft}{\kern0pt}auto\ simp{\isacharcolon}{\kern0pt}\ m{\isacharunderscore}{\kern0pt}simps{\isacharparenright}{\kern0pt}\isanewline
\ \ \isanewline
\ \ \isacommand{show}\isamarkupfalse%
\ {\isacharquery}{\kern0pt}thesis\ \isacommand{using}\isamarkupfalse%
\ kp\ hd\ last\ si\ \isacommand{by}\isamarkupfalse%
\ auto\isanewline
\isacommand{qed}\isamarkupfalse%
%
\endisatagproof
{\isafoldproof}%
%
\isadelimproof
\isanewline
%
\endisadelimproof
\isanewline
\isacommand{lemma}\isamarkupfalse%
\ len{\isadigit{1}}{\isacharunderscore}{\kern0pt}hd{\isacharunderscore}{\kern0pt}last{\isacharcolon}{\kern0pt}\ {\isachardoublequoteopen}length\ xs\ {\isacharequal}{\kern0pt}\ {\isadigit{1}}\ {\isasymLongrightarrow}\ hd\ xs\ {\isacharequal}{\kern0pt}\ last\ xs{\isachardoublequoteclose}\isanewline
%
\isadelimproof
\ \ %
\endisadelimproof
%
\isatagproof
\isacommand{by}\isamarkupfalse%
\ {\isacharparenleft}{\kern0pt}induction\ xs{\isacharparenright}{\kern0pt}\ auto%
\endisatagproof
{\isafoldproof}%
%
\isadelimproof
%
\endisadelimproof
%
\begin{isamarkuptext}%
Weaker version of \isa{{\isasymlbrakk}knights{\isacharunderscore}{\kern0pt}path\ {\isacharparenleft}{\kern0pt}board\ {\isacharquery}{\kern0pt}n\ {\isacharquery}{\kern0pt}m\isactrlsub {\isadigit{1}}{\isacharparenright}{\kern0pt}\ {\isacharquery}{\kern0pt}ps\isactrlsub {\isadigit{1}}{\isacharsemicolon}{\kern0pt}\ knights{\isacharunderscore}{\kern0pt}path\ {\isacharparenleft}{\kern0pt}board\ {\isacharquery}{\kern0pt}n\ {\isacharquery}{\kern0pt}m\isactrlsub {\isadigit{2}}{\isacharparenright}{\kern0pt}\ {\isacharquery}{\kern0pt}ps\isactrlsub {\isadigit{2}}{\isacharsemicolon}{\kern0pt}\ step{\isacharunderscore}{\kern0pt}in\ {\isacharquery}{\kern0pt}ps\isactrlsub {\isadigit{1}}\ {\isacharquery}{\kern0pt}s\isactrlsub i\ {\isacharquery}{\kern0pt}s\isactrlsub j{\isacharsemicolon}{\kern0pt}\ hd\ {\isacharquery}{\kern0pt}ps\isactrlsub {\isadigit{2}}\ {\isacharequal}{\kern0pt}\ {\isacharparenleft}{\kern0pt}{\isacharquery}{\kern0pt}i\isactrlsub h{\isacharcomma}{\kern0pt}\ {\isacharquery}{\kern0pt}j\isactrlsub h{\isacharparenright}{\kern0pt}{\isacharsemicolon}{\kern0pt}\ last\ {\isacharquery}{\kern0pt}ps\isactrlsub {\isadigit{2}}\ {\isacharequal}{\kern0pt}\ {\isacharparenleft}{\kern0pt}{\isacharquery}{\kern0pt}i\isactrlsub l{\isacharcomma}{\kern0pt}\ {\isacharquery}{\kern0pt}j\isactrlsub l{\isacharparenright}{\kern0pt}{\isacharsemicolon}{\kern0pt}\ step{\isacharunderscore}{\kern0pt}in\ {\isacharquery}{\kern0pt}ps\isactrlsub {\isadigit{2}}\ {\isacharparenleft}{\kern0pt}{\isacharquery}{\kern0pt}i{\isacharcomma}{\kern0pt}\ {\isacharquery}{\kern0pt}j{\isacharparenright}{\kern0pt}\ {\isacharparenleft}{\kern0pt}{\isacharquery}{\kern0pt}i{\isacharprime}{\kern0pt}{\isacharcomma}{\kern0pt}\ {\isacharquery}{\kern0pt}j{\isacharprime}{\kern0pt}{\isacharparenright}{\kern0pt}{\isacharsemicolon}{\kern0pt}\ valid{\isacharunderscore}{\kern0pt}step\ {\isacharquery}{\kern0pt}s\isactrlsub i\ {\isacharparenleft}{\kern0pt}{\isacharquery}{\kern0pt}i\isactrlsub h{\isacharcomma}{\kern0pt}\ int\ {\isacharquery}{\kern0pt}m\isactrlsub {\isadigit{1}}\ {\isacharplus}{\kern0pt}\ {\isacharquery}{\kern0pt}j\isactrlsub h{\isacharparenright}{\kern0pt}{\isacharsemicolon}{\kern0pt}\ valid{\isacharunderscore}{\kern0pt}step\ {\isacharparenleft}{\kern0pt}{\isacharquery}{\kern0pt}i\isactrlsub l{\isacharcomma}{\kern0pt}\ int\ {\isacharquery}{\kern0pt}m\isactrlsub {\isadigit{1}}\ {\isacharplus}{\kern0pt}\ {\isacharquery}{\kern0pt}j\isactrlsub l{\isacharparenright}{\kern0pt}\ {\isacharquery}{\kern0pt}s\isactrlsub j{\isasymrbrakk}\ {\isasymLongrightarrow}\ {\isasymexists}ps{\isachardot}{\kern0pt}\ knights{\isacharunderscore}{\kern0pt}path\ {\isacharparenleft}{\kern0pt}board\ {\isacharquery}{\kern0pt}n\ {\isacharparenleft}{\kern0pt}{\isacharquery}{\kern0pt}m\isactrlsub {\isadigit{1}}\ {\isacharplus}{\kern0pt}\ {\isacharquery}{\kern0pt}m\isactrlsub {\isadigit{2}}{\isacharparenright}{\kern0pt}{\isacharparenright}{\kern0pt}\ ps\ {\isasymand}\ hd\ ps\ {\isacharequal}{\kern0pt}\ hd\ {\isacharquery}{\kern0pt}ps\isactrlsub {\isadigit{1}}\ {\isasymand}\ last\ ps\ {\isacharequal}{\kern0pt}\ last\ {\isacharquery}{\kern0pt}ps\isactrlsub {\isadigit{1}}\ {\isasymand}\ step{\isacharunderscore}{\kern0pt}in\ ps\ {\isacharparenleft}{\kern0pt}{\isacharquery}{\kern0pt}i{\isacharcomma}{\kern0pt}\ int\ {\isacharquery}{\kern0pt}m\isactrlsub {\isadigit{1}}\ {\isacharplus}{\kern0pt}\ {\isacharquery}{\kern0pt}j{\isacharparenright}{\kern0pt}\ {\isacharparenleft}{\kern0pt}{\isacharquery}{\kern0pt}i{\isacharprime}{\kern0pt}{\isacharcomma}{\kern0pt}\ int\ {\isacharquery}{\kern0pt}m\isactrlsub {\isadigit{1}}\ {\isacharplus}{\kern0pt}\ {\isacharquery}{\kern0pt}j{\isacharprime}{\kern0pt}{\isacharparenright}{\kern0pt}}.%
\end{isamarkuptext}\isamarkuptrue%
\isacommand{corollary}\isamarkupfalse%
\ knights{\isacharunderscore}{\kern0pt}path{\isacharunderscore}{\kern0pt}split{\isacharunderscore}{\kern0pt}concat{\isacharcolon}{\kern0pt}\isanewline
\ \ \isakeyword{assumes}\ {\isachardoublequoteopen}knights{\isacharunderscore}{\kern0pt}path\ {\isacharparenleft}{\kern0pt}board\ n\ m\isactrlsub {\isadigit{1}}{\isacharparenright}{\kern0pt}\ ps\isactrlsub {\isadigit{1}}{\isachardoublequoteclose}\ {\isachardoublequoteopen}knights{\isacharunderscore}{\kern0pt}path\ {\isacharparenleft}{\kern0pt}board\ n\ m\isactrlsub {\isadigit{2}}{\isacharparenright}{\kern0pt}\ ps\isactrlsub {\isadigit{2}}{\isachardoublequoteclose}\ \isanewline
\ \ \ \ \ \ \ \ \ \ {\isachardoublequoteopen}step{\isacharunderscore}{\kern0pt}in\ ps\isactrlsub {\isadigit{1}}\ s\isactrlsub i\ s\isactrlsub j{\isachardoublequoteclose}\ {\isachardoublequoteopen}hd\ ps\isactrlsub {\isadigit{2}}\ {\isacharequal}{\kern0pt}\ {\isacharparenleft}{\kern0pt}i\isactrlsub h{\isacharcomma}{\kern0pt}j\isactrlsub h{\isacharparenright}{\kern0pt}{\isachardoublequoteclose}\ {\isachardoublequoteopen}last\ ps\isactrlsub {\isadigit{2}}\ {\isacharequal}{\kern0pt}\ {\isacharparenleft}{\kern0pt}i\isactrlsub l{\isacharcomma}{\kern0pt}j\isactrlsub l{\isacharparenright}{\kern0pt}{\isachardoublequoteclose}\isanewline
\ \ \ \ \ \ \ \ \ \ {\isachardoublequoteopen}valid{\isacharunderscore}{\kern0pt}step\ s\isactrlsub i\ {\isacharparenleft}{\kern0pt}i\isactrlsub h{\isacharcomma}{\kern0pt}int\ m\isactrlsub {\isadigit{1}}{\isacharplus}{\kern0pt}j\isactrlsub h{\isacharparenright}{\kern0pt}{\isachardoublequoteclose}\ {\isachardoublequoteopen}valid{\isacharunderscore}{\kern0pt}step\ {\isacharparenleft}{\kern0pt}i\isactrlsub l{\isacharcomma}{\kern0pt}int\ m\isactrlsub {\isadigit{1}}{\isacharplus}{\kern0pt}j\isactrlsub l{\isacharparenright}{\kern0pt}\ s\isactrlsub j{\isachardoublequoteclose}\isanewline
\ \ \isakeyword{shows}\ {\isachardoublequoteopen}{\isasymexists}ps{\isachardot}{\kern0pt}\ knights{\isacharunderscore}{\kern0pt}path\ {\isacharparenleft}{\kern0pt}board\ n\ {\isacharparenleft}{\kern0pt}m\isactrlsub {\isadigit{1}}{\isacharplus}{\kern0pt}m\isactrlsub {\isadigit{2}}{\isacharparenright}{\kern0pt}{\isacharparenright}{\kern0pt}\ ps\ {\isasymand}\ hd\ ps\ {\isacharequal}{\kern0pt}\ hd\ ps\isactrlsub {\isadigit{1}}\ {\isasymand}\ last\ ps\ {\isacharequal}{\kern0pt}\ last\ ps\isactrlsub {\isadigit{1}}{\isachardoublequoteclose}\isanewline
%
\isadelimproof
%
\endisadelimproof
%
\isatagproof
\isacommand{proof}\isamarkupfalse%
\ {\isacharminus}{\kern0pt}\isanewline
\ \ \isacommand{have}\isamarkupfalse%
\ {\isachardoublequoteopen}length\ ps\isactrlsub {\isadigit{2}}\ {\isacharequal}{\kern0pt}\ {\isadigit{1}}\ {\isasymor}\ length\ ps\isactrlsub {\isadigit{2}}\ {\isachargreater}{\kern0pt}\ {\isadigit{1}}{\isachardoublequoteclose}\isanewline
\ \ \ \ \isacommand{using}\isamarkupfalse%
\ assms\ knights{\isacharunderscore}{\kern0pt}path{\isacharunderscore}{\kern0pt}non{\isacharunderscore}{\kern0pt}nil\ \isacommand{by}\isamarkupfalse%
\ {\isacharparenleft}{\kern0pt}meson\ length{\isacharunderscore}{\kern0pt}{\isadigit{0}}{\isacharunderscore}{\kern0pt}conv\ less{\isacharunderscore}{\kern0pt}one\ linorder{\isacharunderscore}{\kern0pt}neqE{\isacharunderscore}{\kern0pt}nat{\isacharparenright}{\kern0pt}\isanewline
\ \ \isacommand{then}\isamarkupfalse%
\ \isacommand{show}\isamarkupfalse%
\ {\isacharquery}{\kern0pt}thesis\isanewline
\ \ \isacommand{proof}\isamarkupfalse%
\ {\isacharparenleft}{\kern0pt}elim\ disjE{\isacharparenright}{\kern0pt}\isanewline
\ \ \ \ \isacommand{let}\isamarkupfalse%
\ {\isacharquery}{\kern0pt}s\isactrlsub k{\isacharequal}{\kern0pt}{\isachardoublequoteopen}{\isacharparenleft}{\kern0pt}i\isactrlsub h{\isacharcomma}{\kern0pt}int\ m\isactrlsub {\isadigit{1}}{\isacharplus}{\kern0pt}j\isactrlsub h{\isacharparenright}{\kern0pt}{\isachardoublequoteclose}\isanewline
\ \ \ \ \isacommand{assume}\isamarkupfalse%
\ {\isachardoublequoteopen}length\ ps\isactrlsub {\isadigit{2}}\ {\isacharequal}{\kern0pt}\ {\isadigit{1}}{\isachardoublequoteclose}\isanewline
\ \ \ \ \isanewline
\ \ \ \ \isacommand{then}\isamarkupfalse%
\ \isacommand{have}\isamarkupfalse%
\ {\isachardoublequoteopen}{\isacharparenleft}{\kern0pt}i\isactrlsub h{\isacharcomma}{\kern0pt}j\isactrlsub h{\isacharparenright}{\kern0pt}\ {\isacharequal}{\kern0pt}\ {\isacharparenleft}{\kern0pt}i\isactrlsub l{\isacharcomma}{\kern0pt}j\isactrlsub l{\isacharparenright}{\kern0pt}{\isachardoublequoteclose}\isanewline
\ \ \ \ \ \ \isacommand{using}\isamarkupfalse%
\ assms\ len{\isadigit{1}}{\isacharunderscore}{\kern0pt}hd{\isacharunderscore}{\kern0pt}last\ \isacommand{by}\isamarkupfalse%
\ metis\isanewline
\ \ \ \ \isacommand{then}\isamarkupfalse%
\ \isacommand{have}\isamarkupfalse%
\ {\isachardoublequoteopen}valid{\isacharunderscore}{\kern0pt}step\ s\isactrlsub i\ {\isacharquery}{\kern0pt}s\isactrlsub k{\isachardoublequoteclose}\ {\isachardoublequoteopen}valid{\isacharunderscore}{\kern0pt}step\ {\isacharquery}{\kern0pt}s\isactrlsub k\ s\isactrlsub j{\isachardoublequoteclose}\ {\isachardoublequoteopen}valid{\isacharunderscore}{\kern0pt}step\ s\isactrlsub i\ s\isactrlsub j{\isachardoublequoteclose}\isanewline
\ \ \ \ \ \ \isacommand{using}\isamarkupfalse%
\ assms\ step{\isacharunderscore}{\kern0pt}in{\isacharunderscore}{\kern0pt}valid{\isacharunderscore}{\kern0pt}step\ \isacommand{by}\isamarkupfalse%
\ auto\isanewline
\ \ \ \ \isacommand{then}\isamarkupfalse%
\ \isacommand{show}\isamarkupfalse%
\ {\isacharquery}{\kern0pt}thesis\isanewline
\ \ \ \ \ \ \isacommand{using}\isamarkupfalse%
\ valid{\isacharunderscore}{\kern0pt}step{\isacharunderscore}{\kern0pt}non{\isacharunderscore}{\kern0pt}transitive\ \isacommand{by}\isamarkupfalse%
\ blast\isanewline
\ \ \isacommand{next}\isamarkupfalse%
\isanewline
\ \ \ \ \isacommand{assume}\isamarkupfalse%
\ {\isachardoublequoteopen}length\ ps\isactrlsub {\isadigit{2}}\ {\isachargreater}{\kern0pt}\ {\isadigit{1}}{\isachardoublequoteclose}\isanewline
\ \ \ \ \isacommand{then}\isamarkupfalse%
\ \isacommand{obtain}\isamarkupfalse%
\ i\isactrlsub {\isadigit{1}}\ j\isactrlsub {\isadigit{1}}\ i\isactrlsub {\isadigit{2}}\ j\isactrlsub {\isadigit{2}}\ ps\isactrlsub {\isadigit{2}}{\isacharprime}{\kern0pt}\ \isakeyword{where}\ {\isachardoublequoteopen}ps\isactrlsub {\isadigit{2}}\ {\isacharequal}{\kern0pt}\ {\isacharparenleft}{\kern0pt}i\isactrlsub {\isadigit{1}}{\isacharcomma}{\kern0pt}j\isactrlsub {\isadigit{1}}{\isacharparenright}{\kern0pt}{\isacharhash}{\kern0pt}{\isacharparenleft}{\kern0pt}i\isactrlsub {\isadigit{2}}{\isacharcomma}{\kern0pt}j\isactrlsub {\isadigit{2}}{\isacharparenright}{\kern0pt}{\isacharhash}{\kern0pt}ps\isactrlsub {\isadigit{2}}{\isacharprime}{\kern0pt}{\isachardoublequoteclose}\isanewline
\ \ \ \ \ \ \isacommand{using}\isamarkupfalse%
\ list{\isacharunderscore}{\kern0pt}len{\isacharunderscore}{\kern0pt}g{\isacharunderscore}{\kern0pt}{\isadigit{1}}{\isacharunderscore}{\kern0pt}split\ \isacommand{by}\isamarkupfalse%
\ fastforce\isanewline
\ \ \ \ \isacommand{then}\isamarkupfalse%
\ \isacommand{have}\isamarkupfalse%
\ {\isachardoublequoteopen}last\ {\isacharparenleft}{\kern0pt}take\ {\isadigit{1}}\ ps\isactrlsub {\isadigit{2}}{\isacharparenright}{\kern0pt}\ {\isacharequal}{\kern0pt}\ {\isacharparenleft}{\kern0pt}i\isactrlsub {\isadigit{1}}{\isacharcomma}{\kern0pt}j\isactrlsub {\isadigit{1}}{\isacharparenright}{\kern0pt}{\isachardoublequoteclose}\ {\isachardoublequoteopen}hd\ {\isacharparenleft}{\kern0pt}drop\ {\isadigit{1}}\ ps\isactrlsub {\isadigit{2}}{\isacharparenright}{\kern0pt}\ {\isacharequal}{\kern0pt}\ {\isacharparenleft}{\kern0pt}i\isactrlsub {\isadigit{2}}{\isacharcomma}{\kern0pt}j\isactrlsub {\isadigit{2}}{\isacharparenright}{\kern0pt}{\isachardoublequoteclose}\ \isacommand{by}\isamarkupfalse%
\ auto\isanewline
\ \ \ \ \isacommand{then}\isamarkupfalse%
\ \isacommand{have}\isamarkupfalse%
\ {\isachardoublequoteopen}step{\isacharunderscore}{\kern0pt}in\ ps\isactrlsub {\isadigit{2}}\ {\isacharparenleft}{\kern0pt}i\isactrlsub {\isadigit{1}}{\isacharcomma}{\kern0pt}j\isactrlsub {\isadigit{1}}{\isacharparenright}{\kern0pt}\ {\isacharparenleft}{\kern0pt}i\isactrlsub {\isadigit{2}}{\isacharcomma}{\kern0pt}j\isactrlsub {\isadigit{2}}{\isacharparenright}{\kern0pt}{\isachardoublequoteclose}\ \isacommand{using}\isamarkupfalse%
\ {\isacartoucheopen}length\ ps\isactrlsub {\isadigit{2}}\ {\isachargreater}{\kern0pt}\ {\isadigit{1}}{\isacartoucheclose}\ \isacommand{by}\isamarkupfalse%
\ {\isacharparenleft}{\kern0pt}auto\ simp{\isacharcolon}{\kern0pt}\ step{\isacharunderscore}{\kern0pt}in{\isacharunderscore}{\kern0pt}def{\isacharparenright}{\kern0pt}\isanewline
\ \ \ \ \isacommand{then}\isamarkupfalse%
\ \isacommand{show}\isamarkupfalse%
\ {\isacharquery}{\kern0pt}thesis\isanewline
\ \ \ \ \ \ \isacommand{using}\isamarkupfalse%
\ assms\ knights{\isacharunderscore}{\kern0pt}path{\isacharunderscore}{\kern0pt}split{\isacharunderscore}{\kern0pt}concat{\isacharunderscore}{\kern0pt}si{\isacharunderscore}{\kern0pt}prev\ \isacommand{by}\isamarkupfalse%
\ blast\isanewline
\ \ \isacommand{qed}\isamarkupfalse%
\isanewline
\isacommand{qed}\isamarkupfalse%
%
\endisatagproof
{\isafoldproof}%
%
\isadelimproof
%
\endisadelimproof
%
\begin{isamarkuptext}%
Concatenate two knight's path on a \isa{n{\isasymtimes}m}-board along the 1st axis.%
\end{isamarkuptext}\isamarkuptrue%
\isacommand{corollary}\isamarkupfalse%
\ knights{\isacharunderscore}{\kern0pt}path{\isacharunderscore}{\kern0pt}split{\isacharunderscore}{\kern0pt}concatT{\isacharcolon}{\kern0pt}\isanewline
\ \ \isakeyword{assumes}\ {\isachardoublequoteopen}knights{\isacharunderscore}{\kern0pt}path\ {\isacharparenleft}{\kern0pt}board\ n\isactrlsub {\isadigit{1}}\ m{\isacharparenright}{\kern0pt}\ ps\isactrlsub {\isadigit{1}}{\isachardoublequoteclose}\ {\isachardoublequoteopen}knights{\isacharunderscore}{\kern0pt}path\ {\isacharparenleft}{\kern0pt}board\ n\isactrlsub {\isadigit{2}}\ m{\isacharparenright}{\kern0pt}\ ps\isactrlsub {\isadigit{2}}{\isachardoublequoteclose}\ \isanewline
\ \ \ \ \ \ \ \ \ \ {\isachardoublequoteopen}step{\isacharunderscore}{\kern0pt}in\ ps\isactrlsub {\isadigit{1}}\ s\isactrlsub i\ s\isactrlsub j{\isachardoublequoteclose}\ {\isachardoublequoteopen}hd\ ps\isactrlsub {\isadigit{2}}\ {\isacharequal}{\kern0pt}\ {\isacharparenleft}{\kern0pt}i\isactrlsub h{\isacharcomma}{\kern0pt}j\isactrlsub h{\isacharparenright}{\kern0pt}{\isachardoublequoteclose}\ {\isachardoublequoteopen}last\ ps\isactrlsub {\isadigit{2}}\ {\isacharequal}{\kern0pt}\ {\isacharparenleft}{\kern0pt}i\isactrlsub l{\isacharcomma}{\kern0pt}j\isactrlsub l{\isacharparenright}{\kern0pt}{\isachardoublequoteclose}\isanewline
\ \ \ \ \ \ \ \ \ \ {\isachardoublequoteopen}valid{\isacharunderscore}{\kern0pt}step\ s\isactrlsub i\ {\isacharparenleft}{\kern0pt}int\ n\isactrlsub {\isadigit{1}}{\isacharplus}{\kern0pt}i\isactrlsub h{\isacharcomma}{\kern0pt}j\isactrlsub h{\isacharparenright}{\kern0pt}{\isachardoublequoteclose}\ {\isachardoublequoteopen}valid{\isacharunderscore}{\kern0pt}step\ {\isacharparenleft}{\kern0pt}int\ n\isactrlsub {\isadigit{1}}{\isacharplus}{\kern0pt}i\isactrlsub l{\isacharcomma}{\kern0pt}j\isactrlsub l{\isacharparenright}{\kern0pt}\ s\isactrlsub j{\isachardoublequoteclose}\isanewline
\ \ \isakeyword{shows}\ {\isachardoublequoteopen}{\isasymexists}ps{\isachardot}{\kern0pt}\ knights{\isacharunderscore}{\kern0pt}path\ {\isacharparenleft}{\kern0pt}board\ {\isacharparenleft}{\kern0pt}n\isactrlsub {\isadigit{1}}{\isacharplus}{\kern0pt}n\isactrlsub {\isadigit{2}}{\isacharparenright}{\kern0pt}\ m{\isacharparenright}{\kern0pt}\ ps\ {\isasymand}\ hd\ ps\ {\isacharequal}{\kern0pt}\ hd\ ps\isactrlsub {\isadigit{1}}\ {\isasymand}\ last\ ps\ {\isacharequal}{\kern0pt}\ last\ ps\isactrlsub {\isadigit{1}}{\isachardoublequoteclose}\isanewline
%
\isadelimproof
\ \ %
\endisadelimproof
%
\isatagproof
\isacommand{using}\isamarkupfalse%
\ assms\isanewline
\isacommand{proof}\isamarkupfalse%
\ {\isacharminus}{\kern0pt}\isanewline
\ \ \isacommand{let}\isamarkupfalse%
\ {\isacharquery}{\kern0pt}ps\isactrlsub {\isadigit{1}}T{\isacharequal}{\kern0pt}{\isachardoublequoteopen}transpose\ ps\isactrlsub {\isadigit{1}}{\isachardoublequoteclose}\isanewline
\ \ \isacommand{let}\isamarkupfalse%
\ {\isacharquery}{\kern0pt}ps\isactrlsub {\isadigit{2}}T{\isacharequal}{\kern0pt}{\isachardoublequoteopen}transpose\ ps\isactrlsub {\isadigit{2}}{\isachardoublequoteclose}\isanewline
\ \ \isacommand{have}\isamarkupfalse%
\ kps{\isacharcolon}{\kern0pt}\ {\isachardoublequoteopen}knights{\isacharunderscore}{\kern0pt}path\ {\isacharparenleft}{\kern0pt}board\ m\ n\isactrlsub {\isadigit{1}}{\isacharparenright}{\kern0pt}\ {\isacharquery}{\kern0pt}ps\isactrlsub {\isadigit{1}}T{\isachardoublequoteclose}\ {\isachardoublequoteopen}knights{\isacharunderscore}{\kern0pt}path\ {\isacharparenleft}{\kern0pt}board\ m\ n\isactrlsub {\isadigit{2}}{\isacharparenright}{\kern0pt}\ {\isacharquery}{\kern0pt}ps\isactrlsub {\isadigit{2}}T{\isachardoublequoteclose}\isanewline
\ \ \ \ \isacommand{using}\isamarkupfalse%
\ assms\ transpose{\isacharunderscore}{\kern0pt}knights{\isacharunderscore}{\kern0pt}path\ \isacommand{by}\isamarkupfalse%
\ auto\isanewline
\isanewline
\ \ \isacommand{let}\isamarkupfalse%
\ {\isacharquery}{\kern0pt}s\isactrlsub iT{\isacharequal}{\kern0pt}{\isachardoublequoteopen}transpose{\isacharunderscore}{\kern0pt}square\ s\isactrlsub i{\isachardoublequoteclose}\isanewline
\ \ \isacommand{let}\isamarkupfalse%
\ {\isacharquery}{\kern0pt}s\isactrlsub jT{\isacharequal}{\kern0pt}{\isachardoublequoteopen}transpose{\isacharunderscore}{\kern0pt}square\ s\isactrlsub j{\isachardoublequoteclose}\isanewline
\ \ \isacommand{have}\isamarkupfalse%
\ si{\isacharcolon}{\kern0pt}\ {\isachardoublequoteopen}step{\isacharunderscore}{\kern0pt}in\ {\isacharquery}{\kern0pt}ps\isactrlsub {\isadigit{1}}T\ {\isacharquery}{\kern0pt}s\isactrlsub iT\ {\isacharquery}{\kern0pt}s\isactrlsub jT{\isachardoublequoteclose}\isanewline
\ \ \ \ \isacommand{using}\isamarkupfalse%
\ assms\ transpose{\isacharunderscore}{\kern0pt}step{\isacharunderscore}{\kern0pt}in\ \isacommand{by}\isamarkupfalse%
\ auto\isanewline
\isanewline
\ \ \isacommand{have}\isamarkupfalse%
\ {\isachardoublequoteopen}ps\isactrlsub {\isadigit{1}}\ {\isasymnoteq}\ {\isacharbrackleft}{\kern0pt}{\isacharbrackright}{\kern0pt}{\isachardoublequoteclose}\ {\isachardoublequoteopen}ps\isactrlsub {\isadigit{2}}\ {\isasymnoteq}\ {\isacharbrackleft}{\kern0pt}{\isacharbrackright}{\kern0pt}{\isachardoublequoteclose}\isanewline
\ \ \ \ \isacommand{using}\isamarkupfalse%
\ assms\ knights{\isacharunderscore}{\kern0pt}path{\isacharunderscore}{\kern0pt}non{\isacharunderscore}{\kern0pt}nil\ \isacommand{by}\isamarkupfalse%
\ auto\isanewline
\ \ \isacommand{then}\isamarkupfalse%
\ \isacommand{have}\isamarkupfalse%
\ hd{\isacharunderscore}{\kern0pt}last{\isadigit{2}}{\isacharcolon}{\kern0pt}\ {\isachardoublequoteopen}hd\ {\isacharquery}{\kern0pt}ps\isactrlsub {\isadigit{2}}T\ {\isacharequal}{\kern0pt}\ {\isacharparenleft}{\kern0pt}j\isactrlsub h{\isacharcomma}{\kern0pt}i\isactrlsub h{\isacharparenright}{\kern0pt}{\isachardoublequoteclose}\ {\isachardoublequoteopen}last\ {\isacharquery}{\kern0pt}ps\isactrlsub {\isadigit{2}}T\ {\isacharequal}{\kern0pt}\ {\isacharparenleft}{\kern0pt}j\isactrlsub l{\isacharcomma}{\kern0pt}i\isactrlsub l{\isacharparenright}{\kern0pt}{\isachardoublequoteclose}\isanewline
\ \ \ \ \isacommand{using}\isamarkupfalse%
\ assms\ hd{\isacharunderscore}{\kern0pt}transpose\ last{\isacharunderscore}{\kern0pt}transpose\ \isacommand{by}\isamarkupfalse%
\ {\isacharparenleft}{\kern0pt}auto\ simp{\isacharcolon}{\kern0pt}\ transpose{\isacharunderscore}{\kern0pt}square{\isacharunderscore}{\kern0pt}def{\isacharparenright}{\kern0pt}\isanewline
\isanewline
\ \ \isacommand{have}\isamarkupfalse%
\ vs{\isacharcolon}{\kern0pt}\ {\isachardoublequoteopen}valid{\isacharunderscore}{\kern0pt}step\ {\isacharquery}{\kern0pt}s\isactrlsub iT\ {\isacharparenleft}{\kern0pt}j\isactrlsub h{\isacharcomma}{\kern0pt}int\ n\isactrlsub {\isadigit{1}}{\isacharplus}{\kern0pt}i\isactrlsub h{\isacharparenright}{\kern0pt}{\isachardoublequoteclose}\ {\isachardoublequoteopen}valid{\isacharunderscore}{\kern0pt}step\ {\isacharparenleft}{\kern0pt}j\isactrlsub l{\isacharcomma}{\kern0pt}int\ n\isactrlsub {\isadigit{1}}{\isacharplus}{\kern0pt}i\isactrlsub l{\isacharparenright}{\kern0pt}\ {\isacharquery}{\kern0pt}s\isactrlsub jT{\isachardoublequoteclose}\isanewline
\ \ \ \ \isacommand{using}\isamarkupfalse%
\ assms\ transpose{\isacharunderscore}{\kern0pt}valid{\isacharunderscore}{\kern0pt}step\ \isacommand{by}\isamarkupfalse%
\ {\isacharparenleft}{\kern0pt}auto\ simp{\isacharcolon}{\kern0pt}\ transpose{\isacharunderscore}{\kern0pt}square{\isacharunderscore}{\kern0pt}def\ split{\isacharcolon}{\kern0pt}\ prod{\isachardot}{\kern0pt}splits{\isacharparenright}{\kern0pt}\isanewline
\isanewline
\ \ \isacommand{then}\isamarkupfalse%
\ \isacommand{obtain}\isamarkupfalse%
\ ps\ \isakeyword{where}\ \isanewline
\ \ \ \ ps{\isacharunderscore}{\kern0pt}prems{\isacharcolon}{\kern0pt}\ {\isachardoublequoteopen}knights{\isacharunderscore}{\kern0pt}path\ {\isacharparenleft}{\kern0pt}board\ m\ {\isacharparenleft}{\kern0pt}n\isactrlsub {\isadigit{1}}{\isacharplus}{\kern0pt}n\isactrlsub {\isadigit{2}}{\isacharparenright}{\kern0pt}{\isacharparenright}{\kern0pt}\ ps{\isachardoublequoteclose}\ {\isachardoublequoteopen}hd\ ps\ {\isacharequal}{\kern0pt}\ hd\ {\isacharquery}{\kern0pt}ps\isactrlsub {\isadigit{1}}T{\isachardoublequoteclose}\ {\isachardoublequoteopen}last\ ps\ {\isacharequal}{\kern0pt}\ last\ {\isacharquery}{\kern0pt}ps\isactrlsub {\isadigit{1}}T{\isachardoublequoteclose}\isanewline
\ \ \ \ \isacommand{using}\isamarkupfalse%
\ knights{\isacharunderscore}{\kern0pt}path{\isacharunderscore}{\kern0pt}split{\isacharunderscore}{\kern0pt}concat{\isacharbrackleft}{\kern0pt}OF\ kps\ si\ hd{\isacharunderscore}{\kern0pt}last{\isadigit{2}}\ vs{\isacharbrackright}{\kern0pt}\ \isacommand{by}\isamarkupfalse%
\ auto\isanewline
\ \ \isacommand{then}\isamarkupfalse%
\ \isacommand{have}\isamarkupfalse%
\ {\isachardoublequoteopen}ps\ {\isasymnoteq}\ {\isacharbrackleft}{\kern0pt}{\isacharbrackright}{\kern0pt}{\isachardoublequoteclose}\ \isacommand{using}\isamarkupfalse%
\ knights{\isacharunderscore}{\kern0pt}path{\isacharunderscore}{\kern0pt}non{\isacharunderscore}{\kern0pt}nil\ \isacommand{by}\isamarkupfalse%
\ auto\isanewline
\ \ \isacommand{let}\isamarkupfalse%
\ {\isacharquery}{\kern0pt}psT{\isacharequal}{\kern0pt}{\isachardoublequoteopen}transpose\ ps{\isachardoublequoteclose}\isanewline
\ \ \isacommand{have}\isamarkupfalse%
\ {\isachardoublequoteopen}knights{\isacharunderscore}{\kern0pt}path\ {\isacharparenleft}{\kern0pt}board\ {\isacharparenleft}{\kern0pt}n\isactrlsub {\isadigit{1}}{\isacharplus}{\kern0pt}n\isactrlsub {\isadigit{2}}{\isacharparenright}{\kern0pt}\ m{\isacharparenright}{\kern0pt}\ {\isacharquery}{\kern0pt}psT{\isachardoublequoteclose}\ {\isachardoublequoteopen}hd\ {\isacharquery}{\kern0pt}psT\ {\isacharequal}{\kern0pt}\ hd\ ps\isactrlsub {\isadigit{1}}{\isachardoublequoteclose}\ {\isachardoublequoteopen}last\ {\isacharquery}{\kern0pt}psT\ {\isacharequal}{\kern0pt}\ last\ ps\isactrlsub {\isadigit{1}}{\isachardoublequoteclose}\isanewline
\ \ \ \ \isacommand{using}\isamarkupfalse%
\ {\isacartoucheopen}ps\isactrlsub {\isadigit{1}}\ {\isasymnoteq}\ {\isacharbrackleft}{\kern0pt}{\isacharbrackright}{\kern0pt}{\isacartoucheclose}\ {\isacartoucheopen}ps\ {\isasymnoteq}\ {\isacharbrackleft}{\kern0pt}{\isacharbrackright}{\kern0pt}{\isacartoucheclose}\ ps{\isacharunderscore}{\kern0pt}prems\ transpose{\isacharunderscore}{\kern0pt}knights{\isacharunderscore}{\kern0pt}path\ hd{\isacharunderscore}{\kern0pt}transpose\ last{\isacharunderscore}{\kern0pt}transpose\ \isanewline
\ \ \ \ \isacommand{by}\isamarkupfalse%
\ {\isacharparenleft}{\kern0pt}auto\ simp{\isacharcolon}{\kern0pt}\ transpose{\isadigit{2}}{\isacharparenright}{\kern0pt}\isanewline
\ \ \isacommand{then}\isamarkupfalse%
\ \isacommand{show}\isamarkupfalse%
\ {\isacharquery}{\kern0pt}thesis\ \isacommand{by}\isamarkupfalse%
\ auto\isanewline
\isacommand{qed}\isamarkupfalse%
%
\endisatagproof
{\isafoldproof}%
%
\isadelimproof
%
\endisadelimproof
%
\begin{isamarkuptext}%
Concatenate two Knight's path along the 2nd axis. There is a valid step from the last square 
in the first Knight's path \isa{ps\isactrlsub {\isadigit{1}}} to the first square in the second Knight's path \isa{ps\isactrlsub {\isadigit{2}}}.%
\end{isamarkuptext}\isamarkuptrue%
\isacommand{corollary}\isamarkupfalse%
\ knights{\isacharunderscore}{\kern0pt}path{\isacharunderscore}{\kern0pt}concat{\isacharcolon}{\kern0pt}\isanewline
\ \ \isakeyword{assumes}\ {\isachardoublequoteopen}knights{\isacharunderscore}{\kern0pt}path\ {\isacharparenleft}{\kern0pt}board\ n\ m\isactrlsub {\isadigit{1}}{\isacharparenright}{\kern0pt}\ ps\isactrlsub {\isadigit{1}}{\isachardoublequoteclose}\ {\isachardoublequoteopen}knights{\isacharunderscore}{\kern0pt}path\ {\isacharparenleft}{\kern0pt}board\ n\ m\isactrlsub {\isadigit{2}}{\isacharparenright}{\kern0pt}\ ps\isactrlsub {\isadigit{2}}{\isachardoublequoteclose}\ \isanewline
\ \ \ \ \ \ \ \ \ \ {\isachardoublequoteopen}hd\ ps\isactrlsub {\isadigit{2}}\ {\isacharequal}{\kern0pt}\ {\isacharparenleft}{\kern0pt}i\isactrlsub h{\isacharcomma}{\kern0pt}j\isactrlsub h{\isacharparenright}{\kern0pt}{\isachardoublequoteclose}\ {\isachardoublequoteopen}valid{\isacharunderscore}{\kern0pt}step\ {\isacharparenleft}{\kern0pt}last\ ps\isactrlsub {\isadigit{1}}{\isacharparenright}{\kern0pt}\ {\isacharparenleft}{\kern0pt}i\isactrlsub h{\isacharcomma}{\kern0pt}int\ m\isactrlsub {\isadigit{1}}{\isacharplus}{\kern0pt}j\isactrlsub h{\isacharparenright}{\kern0pt}{\isachardoublequoteclose}\isanewline
\ \ \isakeyword{shows}\ {\isachardoublequoteopen}knights{\isacharunderscore}{\kern0pt}path\ {\isacharparenleft}{\kern0pt}board\ n\ {\isacharparenleft}{\kern0pt}m\isactrlsub {\isadigit{1}}{\isacharplus}{\kern0pt}m\isactrlsub {\isadigit{2}}{\isacharparenright}{\kern0pt}{\isacharparenright}{\kern0pt}\ {\isacharparenleft}{\kern0pt}ps\isactrlsub {\isadigit{1}}\ {\isacharat}{\kern0pt}\ {\isacharparenleft}{\kern0pt}trans{\isacharunderscore}{\kern0pt}path\ {\isacharparenleft}{\kern0pt}{\isadigit{0}}{\isacharcomma}{\kern0pt}int\ m\isactrlsub {\isadigit{1}}{\isacharparenright}{\kern0pt}\ ps\isactrlsub {\isadigit{2}}{\isacharparenright}{\kern0pt}{\isacharparenright}{\kern0pt}{\isachardoublequoteclose}\isanewline
%
\isadelimproof
%
\endisadelimproof
%
\isatagproof
\isacommand{proof}\isamarkupfalse%
\ {\isacharminus}{\kern0pt}\isanewline
\ \ \isacommand{let}\isamarkupfalse%
\ {\isacharquery}{\kern0pt}ps\isactrlsub {\isadigit{2}}{\isacharprime}{\kern0pt}{\isacharequal}{\kern0pt}{\isachardoublequoteopen}trans{\isacharunderscore}{\kern0pt}path\ {\isacharparenleft}{\kern0pt}{\isadigit{0}}{\isacharcomma}{\kern0pt}int\ m\isactrlsub {\isadigit{1}}{\isacharparenright}{\kern0pt}\ ps\isactrlsub {\isadigit{2}}{\isachardoublequoteclose}\isanewline
\ \ \isacommand{let}\isamarkupfalse%
\ {\isacharquery}{\kern0pt}b{\isacharequal}{\kern0pt}{\isachardoublequoteopen}trans{\isacharunderscore}{\kern0pt}board\ {\isacharparenleft}{\kern0pt}{\isadigit{0}}{\isacharcomma}{\kern0pt}int\ m\isactrlsub {\isadigit{1}}{\isacharparenright}{\kern0pt}\ {\isacharparenleft}{\kern0pt}board\ n\ m\isactrlsub {\isadigit{2}}{\isacharparenright}{\kern0pt}{\isachardoublequoteclose}\isanewline
\ \ \isacommand{have}\isamarkupfalse%
\ inter{\isacharunderscore}{\kern0pt}empty{\isacharcolon}{\kern0pt}\ {\isachardoublequoteopen}board\ n\ m\isactrlsub {\isadigit{1}}\ {\isasyminter}\ {\isacharquery}{\kern0pt}b\ {\isacharequal}{\kern0pt}\ {\isacharbraceleft}{\kern0pt}{\isacharbraceright}{\kern0pt}{\isachardoublequoteclose}\isanewline
\ \ \ \ \isacommand{unfolding}\isamarkupfalse%
\ board{\isacharunderscore}{\kern0pt}def\ trans{\isacharunderscore}{\kern0pt}board{\isacharunderscore}{\kern0pt}def\ \isacommand{by}\isamarkupfalse%
\ auto\isanewline
\ \ \isacommand{have}\isamarkupfalse%
\ {\isachardoublequoteopen}hd\ {\isacharquery}{\kern0pt}ps\isactrlsub {\isadigit{2}}{\isacharprime}{\kern0pt}\ {\isacharequal}{\kern0pt}\ {\isacharparenleft}{\kern0pt}i\isactrlsub h{\isacharcomma}{\kern0pt}int\ m\isactrlsub {\isadigit{1}}{\isacharplus}{\kern0pt}j\isactrlsub h{\isacharparenright}{\kern0pt}{\isachardoublequoteclose}\isanewline
\ \ \ \ \isacommand{using}\isamarkupfalse%
\ assms\ knights{\isacharunderscore}{\kern0pt}path{\isacharunderscore}{\kern0pt}non{\isacharunderscore}{\kern0pt}nil\ hd{\isacharunderscore}{\kern0pt}trans{\isacharunderscore}{\kern0pt}path\ \isacommand{by}\isamarkupfalse%
\ auto\isanewline
\ \ \isacommand{then}\isamarkupfalse%
\ \isacommand{have}\isamarkupfalse%
\ kp{\isacharcolon}{\kern0pt}\ {\isachardoublequoteopen}knights{\isacharunderscore}{\kern0pt}path\ {\isacharparenleft}{\kern0pt}board\ n\ m\isactrlsub {\isadigit{1}}{\isacharparenright}{\kern0pt}\ ps\isactrlsub {\isadigit{1}}{\isachardoublequoteclose}\ {\isachardoublequoteopen}knights{\isacharunderscore}{\kern0pt}path\ {\isacharquery}{\kern0pt}b\ {\isacharquery}{\kern0pt}ps\isactrlsub {\isadigit{2}}{\isacharprime}{\kern0pt}{\isachardoublequoteclose}\ \isakeyword{and}\ \isanewline
\ \ \ \ \ \ \ \ vs{\isacharcolon}{\kern0pt}\ {\isachardoublequoteopen}valid{\isacharunderscore}{\kern0pt}step\ {\isacharparenleft}{\kern0pt}last\ ps\isactrlsub {\isadigit{1}}{\isacharparenright}{\kern0pt}\ {\isacharparenleft}{\kern0pt}hd\ {\isacharquery}{\kern0pt}ps\isactrlsub {\isadigit{2}}{\isacharprime}{\kern0pt}{\isacharparenright}{\kern0pt}{\isachardoublequoteclose}\isanewline
\ \ \ \ \isacommand{using}\isamarkupfalse%
\ assms\ trans{\isacharunderscore}{\kern0pt}knights{\isacharunderscore}{\kern0pt}path\ \isacommand{by}\isamarkupfalse%
\ auto\isanewline
\ \ \isacommand{then}\isamarkupfalse%
\ \isacommand{show}\isamarkupfalse%
\ {\isachardoublequoteopen}knights{\isacharunderscore}{\kern0pt}path\ {\isacharparenleft}{\kern0pt}board\ n\ {\isacharparenleft}{\kern0pt}m\isactrlsub {\isadigit{1}}{\isacharplus}{\kern0pt}m\isactrlsub {\isadigit{2}}{\isacharparenright}{\kern0pt}{\isacharparenright}{\kern0pt}\ {\isacharparenleft}{\kern0pt}ps\isactrlsub {\isadigit{1}}\ {\isacharat}{\kern0pt}\ {\isacharquery}{\kern0pt}ps\isactrlsub {\isadigit{2}}{\isacharprime}{\kern0pt}{\isacharparenright}{\kern0pt}{\isachardoublequoteclose}\isanewline
\ \ \ \ \isacommand{using}\isamarkupfalse%
\ knights{\isacharunderscore}{\kern0pt}path{\isacharunderscore}{\kern0pt}append{\isacharbrackleft}{\kern0pt}OF\ kp\ inter{\isacharunderscore}{\kern0pt}empty\ vs{\isacharbrackright}{\kern0pt}\ board{\isacharunderscore}{\kern0pt}concat\ \isacommand{by}\isamarkupfalse%
\ auto\isanewline
\isacommand{qed}\isamarkupfalse%
%
\endisatagproof
{\isafoldproof}%
%
\isadelimproof
%
\endisadelimproof
%
\begin{isamarkuptext}%
Concatenate two Knight's path along the 2nd axis. The first Knight's path end in 
\isa{{\isacharparenleft}{\kern0pt}{\isadigit{2}}{\isacharcomma}{\kern0pt}m\isactrlsub {\isadigit{1}}{\isacharminus}{\kern0pt}{\isadigit{1}}{\isacharparenright}{\kern0pt}} (lower-right) and the second Knight's paths start in \isa{{\isacharparenleft}{\kern0pt}{\isadigit{1}}{\isacharcomma}{\kern0pt}{\isadigit{1}}{\isacharparenright}{\kern0pt}} (lower-left).%
\end{isamarkuptext}\isamarkuptrue%
\isacommand{corollary}\isamarkupfalse%
\ knights{\isacharunderscore}{\kern0pt}path{\isacharunderscore}{\kern0pt}lr{\isacharunderscore}{\kern0pt}concat{\isacharcolon}{\kern0pt}\isanewline
\ \ \isakeyword{assumes}\ {\isachardoublequoteopen}knights{\isacharunderscore}{\kern0pt}path\ {\isacharparenleft}{\kern0pt}board\ n\ m\isactrlsub {\isadigit{1}}{\isacharparenright}{\kern0pt}\ ps\isactrlsub {\isadigit{1}}{\isachardoublequoteclose}\ {\isachardoublequoteopen}knights{\isacharunderscore}{\kern0pt}path\ {\isacharparenleft}{\kern0pt}board\ n\ m\isactrlsub {\isadigit{2}}{\isacharparenright}{\kern0pt}\ ps\isactrlsub {\isadigit{2}}{\isachardoublequoteclose}\ \isanewline
\ \ \ \ \ \ \ \ \ \ {\isachardoublequoteopen}last\ ps\isactrlsub {\isadigit{1}}\ {\isacharequal}{\kern0pt}\ {\isacharparenleft}{\kern0pt}{\isadigit{2}}{\isacharcomma}{\kern0pt}int\ m\isactrlsub {\isadigit{1}}{\isacharminus}{\kern0pt}{\isadigit{1}}{\isacharparenright}{\kern0pt}{\isachardoublequoteclose}\ {\isachardoublequoteopen}hd\ ps\isactrlsub {\isadigit{2}}\ {\isacharequal}{\kern0pt}\ {\isacharparenleft}{\kern0pt}{\isadigit{1}}{\isacharcomma}{\kern0pt}{\isadigit{1}}{\isacharparenright}{\kern0pt}{\isachardoublequoteclose}\isanewline
\ \ \isakeyword{shows}\ {\isachardoublequoteopen}knights{\isacharunderscore}{\kern0pt}path\ {\isacharparenleft}{\kern0pt}board\ n\ {\isacharparenleft}{\kern0pt}m\isactrlsub {\isadigit{1}}{\isacharplus}{\kern0pt}m\isactrlsub {\isadigit{2}}{\isacharparenright}{\kern0pt}{\isacharparenright}{\kern0pt}\ {\isacharparenleft}{\kern0pt}ps\isactrlsub {\isadigit{1}}\ {\isacharat}{\kern0pt}\ {\isacharparenleft}{\kern0pt}trans{\isacharunderscore}{\kern0pt}path\ {\isacharparenleft}{\kern0pt}{\isadigit{0}}{\isacharcomma}{\kern0pt}int\ m\isactrlsub {\isadigit{1}}{\isacharparenright}{\kern0pt}\ ps\isactrlsub {\isadigit{2}}{\isacharparenright}{\kern0pt}{\isacharparenright}{\kern0pt}{\isachardoublequoteclose}\isanewline
%
\isadelimproof
%
\endisadelimproof
%
\isatagproof
\isacommand{proof}\isamarkupfalse%
\ {\isacharminus}{\kern0pt}\isanewline
\ \ \isacommand{have}\isamarkupfalse%
\ {\isachardoublequoteopen}valid{\isacharunderscore}{\kern0pt}step\ {\isacharparenleft}{\kern0pt}last\ ps\isactrlsub {\isadigit{1}}{\isacharparenright}{\kern0pt}\ {\isacharparenleft}{\kern0pt}{\isadigit{1}}{\isacharcomma}{\kern0pt}int\ m\isactrlsub {\isadigit{1}}{\isacharplus}{\kern0pt}{\isadigit{1}}{\isacharparenright}{\kern0pt}{\isachardoublequoteclose}\isanewline
\ \ \ \ \isacommand{using}\isamarkupfalse%
\ assms\ \isacommand{unfolding}\isamarkupfalse%
\ valid{\isacharunderscore}{\kern0pt}step{\isacharunderscore}{\kern0pt}def\ \isacommand{by}\isamarkupfalse%
\ auto\isanewline
\ \ \isacommand{then}\isamarkupfalse%
\ \isacommand{show}\isamarkupfalse%
\ {\isacharquery}{\kern0pt}thesis\isanewline
\ \ \ \ \isacommand{using}\isamarkupfalse%
\ assms\ knights{\isacharunderscore}{\kern0pt}path{\isacharunderscore}{\kern0pt}concat\ \isacommand{by}\isamarkupfalse%
\ auto\isanewline
\isacommand{qed}\isamarkupfalse%
%
\endisatagproof
{\isafoldproof}%
%
\isadelimproof
%
\endisadelimproof
%
\begin{isamarkuptext}%
Concatenate two Knight's circuits along the 2nd axis. In the first Knight's path the 
squares \isa{{\isacharparenleft}{\kern0pt}{\isadigit{2}}{\isacharcomma}{\kern0pt}m\isactrlsub {\isadigit{1}}{\isacharminus}{\kern0pt}{\isadigit{1}}{\isacharparenright}{\kern0pt}} and \isa{{\isacharparenleft}{\kern0pt}{\isadigit{4}}{\isacharcomma}{\kern0pt}m\isactrlsub {\isadigit{1}}{\isacharparenright}{\kern0pt}} are adjacent and the second Knight's cirucit starts in \isa{{\isacharparenleft}{\kern0pt}{\isadigit{1}}{\isacharcomma}{\kern0pt}{\isadigit{1}}{\isacharparenright}{\kern0pt}} 
(lower-left) and end in \isa{{\isacharparenleft}{\kern0pt}{\isadigit{3}}{\isacharcomma}{\kern0pt}{\isadigit{2}}{\isacharparenright}{\kern0pt}}.%
\end{isamarkuptext}\isamarkuptrue%
\isacommand{corollary}\isamarkupfalse%
\ knights{\isacharunderscore}{\kern0pt}circuit{\isacharunderscore}{\kern0pt}lr{\isacharunderscore}{\kern0pt}concat{\isacharcolon}{\kern0pt}\isanewline
\ \ \isakeyword{assumes}\ {\isachardoublequoteopen}knights{\isacharunderscore}{\kern0pt}circuit\ {\isacharparenleft}{\kern0pt}board\ n\ m\isactrlsub {\isadigit{1}}{\isacharparenright}{\kern0pt}\ ps\isactrlsub {\isadigit{1}}{\isachardoublequoteclose}\ {\isachardoublequoteopen}knights{\isacharunderscore}{\kern0pt}circuit\ {\isacharparenleft}{\kern0pt}board\ n\ m\isactrlsub {\isadigit{2}}{\isacharparenright}{\kern0pt}\ ps\isactrlsub {\isadigit{2}}{\isachardoublequoteclose}\isanewline
\ \ \ \ \ \ \ \ \ \ {\isachardoublequoteopen}step{\isacharunderscore}{\kern0pt}in\ ps\isactrlsub {\isadigit{1}}\ {\isacharparenleft}{\kern0pt}{\isadigit{2}}{\isacharcomma}{\kern0pt}int\ m\isactrlsub {\isadigit{1}}{\isacharminus}{\kern0pt}{\isadigit{1}}{\isacharparenright}{\kern0pt}\ {\isacharparenleft}{\kern0pt}{\isadigit{4}}{\isacharcomma}{\kern0pt}int\ m\isactrlsub {\isadigit{1}}{\isacharparenright}{\kern0pt}{\isachardoublequoteclose}\ \isanewline
\ \ \ \ \ \ \ \ \ \ {\isachardoublequoteopen}hd\ ps\isactrlsub {\isadigit{2}}\ {\isacharequal}{\kern0pt}\ {\isacharparenleft}{\kern0pt}{\isadigit{1}}{\isacharcomma}{\kern0pt}{\isadigit{1}}{\isacharparenright}{\kern0pt}{\isachardoublequoteclose}\ {\isachardoublequoteopen}last\ ps\isactrlsub {\isadigit{2}}\ {\isacharequal}{\kern0pt}\ {\isacharparenleft}{\kern0pt}{\isadigit{3}}{\isacharcomma}{\kern0pt}{\isadigit{2}}{\isacharparenright}{\kern0pt}{\isachardoublequoteclose}\ {\isachardoublequoteopen}step{\isacharunderscore}{\kern0pt}in\ ps\isactrlsub {\isadigit{2}}\ {\isacharparenleft}{\kern0pt}{\isadigit{2}}{\isacharcomma}{\kern0pt}int\ m\isactrlsub {\isadigit{2}}{\isacharminus}{\kern0pt}{\isadigit{1}}{\isacharparenright}{\kern0pt}\ {\isacharparenleft}{\kern0pt}{\isadigit{4}}{\isacharcomma}{\kern0pt}int\ m\isactrlsub {\isadigit{2}}{\isacharparenright}{\kern0pt}{\isachardoublequoteclose}\isanewline
\ \ \isakeyword{shows}\ {\isachardoublequoteopen}{\isasymexists}ps{\isachardot}{\kern0pt}\ knights{\isacharunderscore}{\kern0pt}circuit\ {\isacharparenleft}{\kern0pt}board\ n\ {\isacharparenleft}{\kern0pt}m\isactrlsub {\isadigit{1}}{\isacharplus}{\kern0pt}m\isactrlsub {\isadigit{2}}{\isacharparenright}{\kern0pt}{\isacharparenright}{\kern0pt}\ ps\ {\isasymand}\ step{\isacharunderscore}{\kern0pt}in\ ps\ {\isacharparenleft}{\kern0pt}{\isadigit{2}}{\isacharcomma}{\kern0pt}int\ {\isacharparenleft}{\kern0pt}m\isactrlsub {\isadigit{1}}{\isacharplus}{\kern0pt}m\isactrlsub {\isadigit{2}}{\isacharparenright}{\kern0pt}{\isacharminus}{\kern0pt}{\isadigit{1}}{\isacharparenright}{\kern0pt}\ {\isacharparenleft}{\kern0pt}{\isadigit{4}}{\isacharcomma}{\kern0pt}int\ {\isacharparenleft}{\kern0pt}m\isactrlsub {\isadigit{1}}{\isacharplus}{\kern0pt}m\isactrlsub {\isadigit{2}}{\isacharparenright}{\kern0pt}{\isacharparenright}{\kern0pt}{\isachardoublequoteclose}\isanewline
%
\isadelimproof
%
\endisadelimproof
%
\isatagproof
\isacommand{proof}\isamarkupfalse%
\ {\isacharminus}{\kern0pt}\isanewline
\ \ \isacommand{have}\isamarkupfalse%
\ kp{\isadigit{1}}{\isacharcolon}{\kern0pt}\ {\isachardoublequoteopen}knights{\isacharunderscore}{\kern0pt}path\ {\isacharparenleft}{\kern0pt}board\ n\ m\isactrlsub {\isadigit{1}}{\isacharparenright}{\kern0pt}\ ps\isactrlsub {\isadigit{1}}{\isachardoublequoteclose}\ \isakeyword{and}\ kp{\isadigit{2}}{\isacharcolon}{\kern0pt}\ {\isachardoublequoteopen}knights{\isacharunderscore}{\kern0pt}path\ {\isacharparenleft}{\kern0pt}board\ n\ m\isactrlsub {\isadigit{2}}{\isacharparenright}{\kern0pt}\ ps\isactrlsub {\isadigit{2}}{\isachardoublequoteclose}\ \isanewline
\ \ \ \ \isakeyword{and}\ vs{\isacharcolon}{\kern0pt}\ {\isachardoublequoteopen}valid{\isacharunderscore}{\kern0pt}step\ {\isacharparenleft}{\kern0pt}last\ ps\isactrlsub {\isadigit{1}}{\isacharparenright}{\kern0pt}\ {\isacharparenleft}{\kern0pt}hd\ ps\isactrlsub {\isadigit{1}}{\isacharparenright}{\kern0pt}{\isachardoublequoteclose}\isanewline
\ \ \ \ \isacommand{using}\isamarkupfalse%
\ assms\ \isacommand{unfolding}\isamarkupfalse%
\ knights{\isacharunderscore}{\kern0pt}circuit{\isacharunderscore}{\kern0pt}def\ \isacommand{by}\isamarkupfalse%
\ auto\isanewline
\isanewline
\ \ \isacommand{have}\isamarkupfalse%
\ m{\isacharunderscore}{\kern0pt}simps{\isacharcolon}{\kern0pt}\ {\isachardoublequoteopen}int\ m\isactrlsub {\isadigit{1}}\ {\isacharplus}{\kern0pt}\ {\isacharparenleft}{\kern0pt}int\ m\isactrlsub {\isadigit{2}}{\isacharminus}{\kern0pt}{\isadigit{1}}{\isacharparenright}{\kern0pt}\ {\isacharequal}{\kern0pt}\ int\ {\isacharparenleft}{\kern0pt}m\isactrlsub {\isadigit{1}}{\isacharplus}{\kern0pt}m\isactrlsub {\isadigit{2}}{\isacharparenright}{\kern0pt}{\isacharminus}{\kern0pt}{\isadigit{1}}{\isachardoublequoteclose}\ {\isachardoublequoteopen}int\ m\isactrlsub {\isadigit{1}}\ {\isacharplus}{\kern0pt}\ int\ m\isactrlsub {\isadigit{2}}\ {\isacharequal}{\kern0pt}\ int\ {\isacharparenleft}{\kern0pt}m\isactrlsub {\isadigit{1}}{\isacharplus}{\kern0pt}m\isactrlsub {\isadigit{2}}{\isacharparenright}{\kern0pt}{\isachardoublequoteclose}\ \isacommand{by}\isamarkupfalse%
\ auto\isanewline
\isanewline
\ \ \isacommand{have}\isamarkupfalse%
\ {\isachardoublequoteopen}valid{\isacharunderscore}{\kern0pt}step\ {\isacharparenleft}{\kern0pt}{\isadigit{2}}{\isacharcomma}{\kern0pt}int\ m\isactrlsub {\isadigit{1}}{\isacharminus}{\kern0pt}{\isadigit{1}}{\isacharparenright}{\kern0pt}\ {\isacharparenleft}{\kern0pt}{\isadigit{1}}{\isacharcomma}{\kern0pt}int\ m\isactrlsub {\isadigit{1}}{\isacharplus}{\kern0pt}{\isadigit{1}}{\isacharparenright}{\kern0pt}{\isachardoublequoteclose}\ {\isachardoublequoteopen}valid{\isacharunderscore}{\kern0pt}step\ {\isacharparenleft}{\kern0pt}{\isadigit{3}}{\isacharcomma}{\kern0pt}int\ m\isactrlsub {\isadigit{1}}{\isacharplus}{\kern0pt}{\isadigit{2}}{\isacharparenright}{\kern0pt}\ {\isacharparenleft}{\kern0pt}{\isadigit{4}}{\isacharcomma}{\kern0pt}int\ m\isactrlsub {\isadigit{1}}{\isacharparenright}{\kern0pt}{\isachardoublequoteclose}\isanewline
\ \ \ \ \isacommand{unfolding}\isamarkupfalse%
\ valid{\isacharunderscore}{\kern0pt}step{\isacharunderscore}{\kern0pt}def\ \isacommand{by}\isamarkupfalse%
\ auto\isanewline
\ \ \isacommand{then}\isamarkupfalse%
\ \isacommand{obtain}\isamarkupfalse%
\ ps\ \isakeyword{where}\ {\isachardoublequoteopen}knights{\isacharunderscore}{\kern0pt}path\ {\isacharparenleft}{\kern0pt}board\ n\ {\isacharparenleft}{\kern0pt}m\isactrlsub {\isadigit{1}}{\isacharplus}{\kern0pt}m\isactrlsub {\isadigit{2}}{\isacharparenright}{\kern0pt}{\isacharparenright}{\kern0pt}\ ps{\isachardoublequoteclose}\ {\isachardoublequoteopen}hd\ ps\ {\isacharequal}{\kern0pt}\ hd\ ps\isactrlsub {\isadigit{1}}{\isachardoublequoteclose}\ {\isachardoublequoteopen}last\ ps\ {\isacharequal}{\kern0pt}\ last\ ps\isactrlsub {\isadigit{1}}{\isachardoublequoteclose}\ \isakeyword{and}\ \isanewline
\ \ \ \ \ \ si{\isacharcolon}{\kern0pt}\ {\isachardoublequoteopen}step{\isacharunderscore}{\kern0pt}in\ ps\ {\isacharparenleft}{\kern0pt}{\isadigit{2}}{\isacharcomma}{\kern0pt}int\ {\isacharparenleft}{\kern0pt}m\isactrlsub {\isadigit{1}}{\isacharplus}{\kern0pt}m\isactrlsub {\isadigit{2}}{\isacharparenright}{\kern0pt}{\isacharminus}{\kern0pt}{\isadigit{1}}{\isacharparenright}{\kern0pt}\ {\isacharparenleft}{\kern0pt}{\isadigit{4}}{\isacharcomma}{\kern0pt}int\ {\isacharparenleft}{\kern0pt}m\isactrlsub {\isadigit{1}}{\isacharplus}{\kern0pt}m\isactrlsub {\isadigit{2}}{\isacharparenright}{\kern0pt}{\isacharparenright}{\kern0pt}{\isachardoublequoteclose}\isanewline
\ \ \ \ \isacommand{using}\isamarkupfalse%
\ assms\ kp{\isadigit{1}}\ kp{\isadigit{2}}\ \isanewline
\ \ \ \ \ \ \ \ \ \ knights{\isacharunderscore}{\kern0pt}path{\isacharunderscore}{\kern0pt}split{\isacharunderscore}{\kern0pt}concat{\isacharunderscore}{\kern0pt}si{\isacharunderscore}{\kern0pt}prev{\isacharbrackleft}{\kern0pt}of\ n\ m\isactrlsub {\isadigit{1}}\ ps\isactrlsub {\isadigit{1}}\ m\isactrlsub {\isadigit{2}}\ ps\isactrlsub {\isadigit{2}}\ {\isachardoublequoteopen}{\isacharparenleft}{\kern0pt}{\isadigit{2}}{\isacharcomma}{\kern0pt}int\ m\isactrlsub {\isadigit{1}}{\isacharminus}{\kern0pt}{\isadigit{1}}{\isacharparenright}{\kern0pt}{\isachardoublequoteclose}\ \isanewline
\ \ \ \ \ \ \ \ \ \ \ \ \ \ \ \ \ \ \ \ \ \ \ \ \ \ \ \ \ \ \ \ \ \ \ \ \ \ \ \ \ \ \ \ \ \ {\isachardoublequoteopen}{\isacharparenleft}{\kern0pt}{\isadigit{4}}{\isacharcomma}{\kern0pt}int\ m\isactrlsub {\isadigit{1}}{\isacharparenright}{\kern0pt}{\isachardoublequoteclose}\ {\isadigit{1}}\ {\isadigit{1}}\ {\isadigit{3}}\ {\isadigit{2}}\ {\isadigit{2}}\ {\isachardoublequoteopen}int\ m\isactrlsub {\isadigit{2}}{\isacharminus}{\kern0pt}{\isadigit{1}}{\isachardoublequoteclose}\ {\isadigit{4}}\ {\isachardoublequoteopen}int\ m\isactrlsub {\isadigit{2}}{\isachardoublequoteclose}{\isacharbrackright}{\kern0pt}\ \isanewline
\ \ \ \ \isacommand{by}\isamarkupfalse%
\ {\isacharparenleft}{\kern0pt}auto\ simp\ only{\isacharcolon}{\kern0pt}\ m{\isacharunderscore}{\kern0pt}simps{\isacharparenright}{\kern0pt}\isanewline
\ \ \isacommand{then}\isamarkupfalse%
\ \isacommand{have}\isamarkupfalse%
\ {\isachardoublequoteopen}knights{\isacharunderscore}{\kern0pt}circuit\ {\isacharparenleft}{\kern0pt}board\ n\ {\isacharparenleft}{\kern0pt}m\isactrlsub {\isadigit{1}}{\isacharplus}{\kern0pt}m\isactrlsub {\isadigit{2}}{\isacharparenright}{\kern0pt}{\isacharparenright}{\kern0pt}\ ps{\isachardoublequoteclose}\isanewline
\ \ \ \ \isacommand{using}\isamarkupfalse%
\ vs\ \isacommand{by}\isamarkupfalse%
\ {\isacharparenleft}{\kern0pt}auto\ simp{\isacharcolon}{\kern0pt}\ knights{\isacharunderscore}{\kern0pt}circuit{\isacharunderscore}{\kern0pt}def{\isacharparenright}{\kern0pt}\isanewline
\ \ \isacommand{then}\isamarkupfalse%
\ \isacommand{show}\isamarkupfalse%
\ {\isacharquery}{\kern0pt}thesis\isanewline
\ \ \ \ \isacommand{using}\isamarkupfalse%
\ si\ \isacommand{by}\isamarkupfalse%
\ auto\isanewline
\isacommand{qed}\isamarkupfalse%
%
\endisatagproof
{\isafoldproof}%
%
\isadelimproof
%
\endisadelimproof
%
\isadelimdocument
%
\endisadelimdocument
%
\isatagdocument
%
\isamarkupsection{Parsing Paths%
}
\isamarkuptrue%
%
\endisatagdocument
{\isafolddocument}%
%
\isadelimdocument
%
\endisadelimdocument
%
\begin{isamarkuptext}%
In this section functions are implemented to parse and construct paths. The parser converts 
the matrix representation (\isa{{\isacharparenleft}{\kern0pt}nat\ list{\isacharparenright}{\kern0pt}\ list}) used in \cite{cull_decurtins_1987} to a path 
(\isa{path}).%
\end{isamarkuptext}\isamarkuptrue%
%
\begin{isamarkuptext}%
for debugging%
\end{isamarkuptext}\isamarkuptrue%
\isacommand{fun}\isamarkupfalse%
\ test{\isacharunderscore}{\kern0pt}path\ {\isacharcolon}{\kern0pt}{\isacharcolon}{\kern0pt}\ {\isachardoublequoteopen}path\ {\isasymRightarrow}\ bool{\isachardoublequoteclose}\ \isakeyword{where}\isanewline
\ \ {\isachardoublequoteopen}test{\isacharunderscore}{\kern0pt}path\ {\isacharparenleft}{\kern0pt}s\isactrlsub i{\isacharhash}{\kern0pt}s\isactrlsub j{\isacharhash}{\kern0pt}xs{\isacharparenright}{\kern0pt}\ {\isacharequal}{\kern0pt}\ {\isacharparenleft}{\kern0pt}step{\isacharunderscore}{\kern0pt}checker\ s\isactrlsub i\ s\isactrlsub j\ {\isasymand}\ test{\isacharunderscore}{\kern0pt}path\ {\isacharparenleft}{\kern0pt}s\isactrlsub j{\isacharhash}{\kern0pt}xs{\isacharparenright}{\kern0pt}{\isacharparenright}{\kern0pt}{\isachardoublequoteclose}\isanewline
{\isacharbar}{\kern0pt}\ {\isachardoublequoteopen}test{\isacharunderscore}{\kern0pt}path\ {\isacharunderscore}{\kern0pt}\ {\isacharequal}{\kern0pt}\ True{\isachardoublequoteclose}\isanewline
\isanewline
\isacommand{fun}\isamarkupfalse%
\ f{\isacharunderscore}{\kern0pt}opt\ {\isacharcolon}{\kern0pt}{\isacharcolon}{\kern0pt}\ {\isachardoublequoteopen}{\isacharparenleft}{\kern0pt}{\isacharprime}{\kern0pt}a\ {\isasymRightarrow}\ {\isacharprime}{\kern0pt}a{\isacharparenright}{\kern0pt}\ {\isasymRightarrow}\ {\isacharprime}{\kern0pt}a\ option\ {\isasymRightarrow}\ {\isacharprime}{\kern0pt}a\ option{\isachardoublequoteclose}\ \isakeyword{where}\isanewline
\ \ {\isachardoublequoteopen}f{\isacharunderscore}{\kern0pt}opt\ {\isacharunderscore}{\kern0pt}\ None\ {\isacharequal}{\kern0pt}\ None{\isachardoublequoteclose}\isanewline
{\isacharbar}{\kern0pt}\ {\isachardoublequoteopen}f{\isacharunderscore}{\kern0pt}opt\ f\ {\isacharparenleft}{\kern0pt}Some\ a{\isacharparenright}{\kern0pt}\ {\isacharequal}{\kern0pt}\ Some\ {\isacharparenleft}{\kern0pt}f\ a{\isacharparenright}{\kern0pt}{\isachardoublequoteclose}\isanewline
\isanewline
\isacommand{fun}\isamarkupfalse%
\ add{\isacharunderscore}{\kern0pt}opt{\isacharunderscore}{\kern0pt}fst{\isacharunderscore}{\kern0pt}sq\ {\isacharcolon}{\kern0pt}{\isacharcolon}{\kern0pt}\ {\isachardoublequoteopen}int\ {\isasymRightarrow}\ square\ option\ {\isasymRightarrow}\ square\ option{\isachardoublequoteclose}\ \isakeyword{where}\isanewline
\ \ {\isachardoublequoteopen}add{\isacharunderscore}{\kern0pt}opt{\isacharunderscore}{\kern0pt}fst{\isacharunderscore}{\kern0pt}sq\ {\isacharunderscore}{\kern0pt}\ None\ {\isacharequal}{\kern0pt}\ None{\isachardoublequoteclose}\isanewline
{\isacharbar}{\kern0pt}\ {\isachardoublequoteopen}add{\isacharunderscore}{\kern0pt}opt{\isacharunderscore}{\kern0pt}fst{\isacharunderscore}{\kern0pt}sq\ k\ {\isacharparenleft}{\kern0pt}Some\ {\isacharparenleft}{\kern0pt}i{\isacharcomma}{\kern0pt}j{\isacharparenright}{\kern0pt}{\isacharparenright}{\kern0pt}\ {\isacharequal}{\kern0pt}\ Some\ {\isacharparenleft}{\kern0pt}k{\isacharplus}{\kern0pt}i{\isacharcomma}{\kern0pt}j{\isacharparenright}{\kern0pt}{\isachardoublequoteclose}\isanewline
\isanewline
\isacommand{fun}\isamarkupfalse%
\ find{\isacharunderscore}{\kern0pt}k{\isacharunderscore}{\kern0pt}in{\isacharunderscore}{\kern0pt}col\ {\isacharcolon}{\kern0pt}{\isacharcolon}{\kern0pt}\ {\isachardoublequoteopen}nat\ {\isasymRightarrow}\ nat\ list\ {\isasymRightarrow}\ int\ option{\isachardoublequoteclose}\ \isakeyword{where}\isanewline
\ \ {\isachardoublequoteopen}find{\isacharunderscore}{\kern0pt}k{\isacharunderscore}{\kern0pt}in{\isacharunderscore}{\kern0pt}col\ k\ {\isacharbrackleft}{\kern0pt}{\isacharbrackright}{\kern0pt}\ {\isacharequal}{\kern0pt}\ None{\isachardoublequoteclose}\isanewline
{\isacharbar}{\kern0pt}\ {\isachardoublequoteopen}find{\isacharunderscore}{\kern0pt}k{\isacharunderscore}{\kern0pt}in{\isacharunderscore}{\kern0pt}col\ k\ {\isacharparenleft}{\kern0pt}c{\isacharhash}{\kern0pt}cs{\isacharparenright}{\kern0pt}\ {\isacharequal}{\kern0pt}\ {\isacharparenleft}{\kern0pt}if\ c\ {\isacharequal}{\kern0pt}\ k\ then\ Some\ {\isadigit{1}}\ else\ f{\isacharunderscore}{\kern0pt}opt\ {\isacharparenleft}{\kern0pt}{\isacharparenleft}{\kern0pt}{\isacharplus}{\kern0pt}{\isacharparenright}{\kern0pt}\ {\isadigit{1}}{\isacharparenright}{\kern0pt}\ {\isacharparenleft}{\kern0pt}find{\isacharunderscore}{\kern0pt}k{\isacharunderscore}{\kern0pt}in{\isacharunderscore}{\kern0pt}col\ k\ cs{\isacharparenright}{\kern0pt}{\isacharparenright}{\kern0pt}{\isachardoublequoteclose}\isanewline
\isanewline
\isacommand{fun}\isamarkupfalse%
\ find{\isacharunderscore}{\kern0pt}k{\isacharunderscore}{\kern0pt}sqr\ {\isacharcolon}{\kern0pt}{\isacharcolon}{\kern0pt}\ {\isachardoublequoteopen}nat\ {\isasymRightarrow}\ {\isacharparenleft}{\kern0pt}nat\ list{\isacharparenright}{\kern0pt}\ list\ {\isasymRightarrow}\ square\ option{\isachardoublequoteclose}\ \isakeyword{where}\isanewline
\ \ {\isachardoublequoteopen}find{\isacharunderscore}{\kern0pt}k{\isacharunderscore}{\kern0pt}sqr\ k\ {\isacharbrackleft}{\kern0pt}{\isacharbrackright}{\kern0pt}\ {\isacharequal}{\kern0pt}\ None{\isachardoublequoteclose}\isanewline
{\isacharbar}{\kern0pt}\ {\isachardoublequoteopen}find{\isacharunderscore}{\kern0pt}k{\isacharunderscore}{\kern0pt}sqr\ k\ {\isacharparenleft}{\kern0pt}r{\isacharhash}{\kern0pt}rs{\isacharparenright}{\kern0pt}\ {\isacharequal}{\kern0pt}\ {\isacharparenleft}{\kern0pt}case\ find{\isacharunderscore}{\kern0pt}k{\isacharunderscore}{\kern0pt}in{\isacharunderscore}{\kern0pt}col\ k\ r\ of\ \isanewline
\ \ \ \ \ \ None\ {\isasymRightarrow}\ f{\isacharunderscore}{\kern0pt}opt\ {\isacharparenleft}{\kern0pt}{\isasymlambda}{\isacharparenleft}{\kern0pt}i{\isacharcomma}{\kern0pt}j{\isacharparenright}{\kern0pt}{\isachardot}{\kern0pt}\ {\isacharparenleft}{\kern0pt}i{\isacharplus}{\kern0pt}{\isadigit{1}}{\isacharcomma}{\kern0pt}j{\isacharparenright}{\kern0pt}{\isacharparenright}{\kern0pt}\ {\isacharparenleft}{\kern0pt}find{\isacharunderscore}{\kern0pt}k{\isacharunderscore}{\kern0pt}sqr\ k\ rs{\isacharparenright}{\kern0pt}\isanewline
\ \ \ \ {\isacharbar}{\kern0pt}\ Some\ j\ {\isasymRightarrow}\ Some\ {\isacharparenleft}{\kern0pt}{\isadigit{1}}{\isacharcomma}{\kern0pt}j{\isacharparenright}{\kern0pt}{\isacharparenright}{\kern0pt}{\isachardoublequoteclose}%
\begin{isamarkuptext}%
Auxiliary function to easily parse pre-computed boards from paper.%
\end{isamarkuptext}\isamarkuptrue%
\isacommand{fun}\isamarkupfalse%
\ to{\isacharunderscore}{\kern0pt}sqrs\ {\isacharcolon}{\kern0pt}{\isacharcolon}{\kern0pt}\ {\isachardoublequoteopen}nat\ {\isasymRightarrow}\ {\isacharparenleft}{\kern0pt}nat\ list{\isacharparenright}{\kern0pt}\ list\ {\isasymRightarrow}\ path\ option{\isachardoublequoteclose}\ \isakeyword{where}\isanewline
\ \ {\isachardoublequoteopen}to{\isacharunderscore}{\kern0pt}sqrs\ {\isadigit{0}}\ rs\ {\isacharequal}{\kern0pt}\ Some\ {\isacharbrackleft}{\kern0pt}{\isacharbrackright}{\kern0pt}{\isachardoublequoteclose}\isanewline
{\isacharbar}{\kern0pt}\ {\isachardoublequoteopen}to{\isacharunderscore}{\kern0pt}sqrs\ k\ rs\ {\isacharequal}{\kern0pt}\ {\isacharparenleft}{\kern0pt}case\ find{\isacharunderscore}{\kern0pt}k{\isacharunderscore}{\kern0pt}sqr\ k\ rs\ of\isanewline
\ \ \ \ \ \ None\ {\isasymRightarrow}\ None\isanewline
\ \ \ \ {\isacharbar}{\kern0pt}\ Some\ s\isactrlsub i\ {\isasymRightarrow}\ f{\isacharunderscore}{\kern0pt}opt\ {\isacharparenleft}{\kern0pt}{\isasymlambda}ps{\isachardot}{\kern0pt}\ ps{\isacharat}{\kern0pt}{\isacharbrackleft}{\kern0pt}s\isactrlsub i{\isacharbrackright}{\kern0pt}{\isacharparenright}{\kern0pt}\ {\isacharparenleft}{\kern0pt}to{\isacharunderscore}{\kern0pt}sqrs\ {\isacharparenleft}{\kern0pt}k{\isacharminus}{\kern0pt}{\isadigit{1}}{\isacharparenright}{\kern0pt}\ rs{\isacharparenright}{\kern0pt}{\isacharparenright}{\kern0pt}{\isachardoublequoteclose}\isanewline
\isanewline
\isacommand{fun}\isamarkupfalse%
\ num{\isacharunderscore}{\kern0pt}elems\ {\isacharcolon}{\kern0pt}{\isacharcolon}{\kern0pt}\ {\isachardoublequoteopen}{\isacharparenleft}{\kern0pt}nat\ list{\isacharparenright}{\kern0pt}\ list\ {\isasymRightarrow}\ nat{\isachardoublequoteclose}\ \isakeyword{where}\isanewline
\ \ {\isachardoublequoteopen}num{\isacharunderscore}{\kern0pt}elems\ {\isacharparenleft}{\kern0pt}r{\isacharhash}{\kern0pt}rs{\isacharparenright}{\kern0pt}\ {\isacharequal}{\kern0pt}\ length\ r\ {\isacharasterisk}{\kern0pt}\ length\ {\isacharparenleft}{\kern0pt}r{\isacharhash}{\kern0pt}rs{\isacharparenright}{\kern0pt}{\isachardoublequoteclose}%
\begin{isamarkuptext}%
Convert a matrix (\isa{nat\ list\ list}) to a path (\isa{path}). With this function we implicitly 
define the lower-left corner to be \isa{{\isacharparenleft}{\kern0pt}{\isadigit{1}}{\isacharcomma}{\kern0pt}{\isadigit{1}}{\isacharparenright}{\kern0pt}} and the upper-right corner to be \isa{{\isacharparenleft}{\kern0pt}n{\isacharcomma}{\kern0pt}m{\isacharparenright}{\kern0pt}}.%
\end{isamarkuptext}\isamarkuptrue%
\isacommand{definition}\isamarkupfalse%
\ {\isachardoublequoteopen}to{\isacharunderscore}{\kern0pt}path\ rs\ {\isasymequiv}\ to{\isacharunderscore}{\kern0pt}sqrs\ {\isacharparenleft}{\kern0pt}num{\isacharunderscore}{\kern0pt}elems\ rs{\isacharparenright}{\kern0pt}\ {\isacharparenleft}{\kern0pt}rev\ rs{\isacharparenright}{\kern0pt}{\isachardoublequoteclose}%
\begin{isamarkuptext}%
Example%
\end{isamarkuptext}\isamarkuptrue%
\isacommand{value}\isamarkupfalse%
\ {\isachardoublequoteopen}to{\isacharunderscore}{\kern0pt}path\ \isanewline
\ \ {\isacharbrackleft}{\kern0pt}{\isacharbrackleft}{\kern0pt}{\isadigit{3}}{\isacharcomma}{\kern0pt}{\isadigit{2}}{\isadigit{2}}{\isacharcomma}{\kern0pt}{\isadigit{1}}{\isadigit{3}}{\isacharcomma}{\kern0pt}{\isadigit{1}}{\isadigit{6}}{\isacharcomma}{\kern0pt}{\isadigit{5}}{\isacharbrackright}{\kern0pt}{\isacharcomma}{\kern0pt}\isanewline
\ \ {\isacharbrackleft}{\kern0pt}{\isadigit{1}}{\isadigit{2}}{\isacharcomma}{\kern0pt}{\isadigit{1}}{\isadigit{7}}{\isacharcomma}{\kern0pt}{\isadigit{4}}{\isacharcomma}{\kern0pt}{\isadigit{2}}{\isadigit{1}}{\isacharcomma}{\kern0pt}{\isadigit{1}}{\isadigit{4}}{\isacharbrackright}{\kern0pt}{\isacharcomma}{\kern0pt}\isanewline
\ \ {\isacharbrackleft}{\kern0pt}{\isadigit{2}}{\isadigit{3}}{\isacharcomma}{\kern0pt}{\isadigit{2}}{\isacharcomma}{\kern0pt}{\isadigit{1}}{\isadigit{5}}{\isacharcomma}{\kern0pt}{\isadigit{6}}{\isacharcomma}{\kern0pt}{\isadigit{9}}{\isacharbrackright}{\kern0pt}{\isacharcomma}{\kern0pt}\isanewline
\ \ {\isacharbrackleft}{\kern0pt}{\isadigit{1}}{\isadigit{8}}{\isacharcomma}{\kern0pt}{\isadigit{1}}{\isadigit{1}}{\isacharcomma}{\kern0pt}{\isadigit{8}}{\isacharcomma}{\kern0pt}{\isadigit{2}}{\isadigit{5}}{\isacharcomma}{\kern0pt}{\isadigit{2}}{\isadigit{0}}{\isacharbrackright}{\kern0pt}{\isacharcomma}{\kern0pt}\isanewline
\ \ {\isacharbrackleft}{\kern0pt}{\isadigit{1}}{\isacharcomma}{\kern0pt}{\isadigit{2}}{\isadigit{4}}{\isacharcomma}{\kern0pt}{\isadigit{1}}{\isadigit{9}}{\isacharcomma}{\kern0pt}{\isadigit{1}}{\isadigit{0}}{\isacharcomma}{\kern0pt}{\isadigit{7}}{\isacharcolon}{\kern0pt}{\isacharcolon}{\kern0pt}nat{\isacharbrackright}{\kern0pt}{\isacharbrackright}{\kern0pt}{\isachardoublequoteclose}%
\isadelimdocument
%
\endisadelimdocument
%
\isatagdocument
%
\isamarkupsection{Knight's Paths for \isa{{\isadigit{5}}{\isasymtimes}m}-Boards%
}
\isamarkuptrue%
%
\endisatagdocument
{\isafolddocument}%
%
\isadelimdocument
%
\endisadelimdocument
%
\begin{isamarkuptext}%
Given here are knight's paths, \isa{kp{\isadigit{5}}xmlr} and \isa{kp{\isadigit{5}}xmur}, for the \isa{{\isacharparenleft}{\kern0pt}{\isadigit{5}}{\isasymtimes}m{\isacharparenright}{\kern0pt}}-board that start 
in the lower-left corner for \isa{m{\isasymin}{\isacharbraceleft}{\kern0pt}{\isadigit{5}}{\isacharcomma}{\kern0pt}{\isadigit{6}}{\isacharcomma}{\kern0pt}{\isadigit{7}}{\isacharcomma}{\kern0pt}{\isadigit{8}}{\isacharcomma}{\kern0pt}{\isadigit{9}}{\isacharbraceright}{\kern0pt}}. The path \isa{kp{\isadigit{5}}xmlr} ends in the lower-right corner, 
whereas the path \isa{kp{\isadigit{5}}xmur} ends in the upper-right corner. 
The tables show the visited squares numbered in ascending order.%
\end{isamarkuptext}\isamarkuptrue%
\isacommand{abbreviation}\isamarkupfalse%
\ {\isachardoublequoteopen}b{\isadigit{5}}x{\isadigit{5}}\ {\isasymequiv}\ board\ {\isadigit{5}}\ {\isadigit{5}}{\isachardoublequoteclose}%
\begin{isamarkuptext}%
A Knight's path for the \isa{{\isacharparenleft}{\kern0pt}{\isadigit{5}}{\isasymtimes}{\isadigit{5}}{\isacharparenright}{\kern0pt}}-board that starts in the lower-left and ends in the 
lower-right.
  \begin{table}[H]
    \begin{tabular}{lllll}
       3 & 22 & 13 & 16 &  5 \\
      12 & 17 &  4 & 21 & 14 \\
      23 &  2 & 15 &  6 &  9 \\
      18 & 11 &  8 & 25 & 20 \\
       1 & 24 & 19 & 10 &  7
    \end{tabular}
  \end{table}%
\end{isamarkuptext}\isamarkuptrue%
\isacommand{abbreviation}\isamarkupfalse%
\ {\isachardoublequoteopen}kp{\isadigit{5}}x{\isadigit{5}}lr\ {\isasymequiv}\ the\ {\isacharparenleft}{\kern0pt}to{\isacharunderscore}{\kern0pt}path\ \isanewline
\ \ {\isacharbrackleft}{\kern0pt}{\isacharbrackleft}{\kern0pt}{\isadigit{3}}{\isacharcomma}{\kern0pt}{\isadigit{2}}{\isadigit{2}}{\isacharcomma}{\kern0pt}{\isadigit{1}}{\isadigit{3}}{\isacharcomma}{\kern0pt}{\isadigit{1}}{\isadigit{6}}{\isacharcomma}{\kern0pt}{\isadigit{5}}{\isacharbrackright}{\kern0pt}{\isacharcomma}{\kern0pt}\isanewline
\ \ {\isacharbrackleft}{\kern0pt}{\isadigit{1}}{\isadigit{2}}{\isacharcomma}{\kern0pt}{\isadigit{1}}{\isadigit{7}}{\isacharcomma}{\kern0pt}{\isadigit{4}}{\isacharcomma}{\kern0pt}{\isadigit{2}}{\isadigit{1}}{\isacharcomma}{\kern0pt}{\isadigit{1}}{\isadigit{4}}{\isacharbrackright}{\kern0pt}{\isacharcomma}{\kern0pt}\isanewline
\ \ {\isacharbrackleft}{\kern0pt}{\isadigit{2}}{\isadigit{3}}{\isacharcomma}{\kern0pt}{\isadigit{2}}{\isacharcomma}{\kern0pt}{\isadigit{1}}{\isadigit{5}}{\isacharcomma}{\kern0pt}{\isadigit{6}}{\isacharcomma}{\kern0pt}{\isadigit{9}}{\isacharbrackright}{\kern0pt}{\isacharcomma}{\kern0pt}\isanewline
\ \ {\isacharbrackleft}{\kern0pt}{\isadigit{1}}{\isadigit{8}}{\isacharcomma}{\kern0pt}{\isadigit{1}}{\isadigit{1}}{\isacharcomma}{\kern0pt}{\isadigit{8}}{\isacharcomma}{\kern0pt}{\isadigit{2}}{\isadigit{5}}{\isacharcomma}{\kern0pt}{\isadigit{2}}{\isadigit{0}}{\isacharbrackright}{\kern0pt}{\isacharcomma}{\kern0pt}\isanewline
\ \ {\isacharbrackleft}{\kern0pt}{\isadigit{1}}{\isacharcomma}{\kern0pt}{\isadigit{2}}{\isadigit{4}}{\isacharcomma}{\kern0pt}{\isadigit{1}}{\isadigit{9}}{\isacharcomma}{\kern0pt}{\isadigit{1}}{\isadigit{0}}{\isacharcomma}{\kern0pt}{\isadigit{7}}{\isacharbrackright}{\kern0pt}{\isacharbrackright}{\kern0pt}{\isacharparenright}{\kern0pt}{\isachardoublequoteclose}\isanewline
\isacommand{lemma}\isamarkupfalse%
\ kp{\isacharunderscore}{\kern0pt}{\isadigit{5}}x{\isadigit{5}}{\isacharunderscore}{\kern0pt}lr{\isacharcolon}{\kern0pt}\ {\isachardoublequoteopen}knights{\isacharunderscore}{\kern0pt}path\ b{\isadigit{5}}x{\isadigit{5}}\ kp{\isadigit{5}}x{\isadigit{5}}lr{\isachardoublequoteclose}\isanewline
%
\isadelimproof
\ \ %
\endisadelimproof
%
\isatagproof
\isacommand{by}\isamarkupfalse%
\ {\isacharparenleft}{\kern0pt}simp\ only{\isacharcolon}{\kern0pt}\ knights{\isacharunderscore}{\kern0pt}path{\isacharunderscore}{\kern0pt}exec{\isacharunderscore}{\kern0pt}simp{\isacharparenright}{\kern0pt}\ eval%
\endisatagproof
{\isafoldproof}%
%
\isadelimproof
\isanewline
%
\endisadelimproof
\isanewline
\isacommand{lemma}\isamarkupfalse%
\ kp{\isacharunderscore}{\kern0pt}{\isadigit{5}}x{\isadigit{5}}{\isacharunderscore}{\kern0pt}lr{\isacharunderscore}{\kern0pt}hd{\isacharcolon}{\kern0pt}\ {\isachardoublequoteopen}hd\ kp{\isadigit{5}}x{\isadigit{5}}lr\ {\isacharequal}{\kern0pt}\ {\isacharparenleft}{\kern0pt}{\isadigit{1}}{\isacharcomma}{\kern0pt}{\isadigit{1}}{\isacharparenright}{\kern0pt}{\isachardoublequoteclose}%
\isadelimproof
\ %
\endisadelimproof
%
\isatagproof
\isacommand{by}\isamarkupfalse%
\ eval%
\endisatagproof
{\isafoldproof}%
%
\isadelimproof
%
\endisadelimproof
\isanewline
\isanewline
\isacommand{lemma}\isamarkupfalse%
\ kp{\isacharunderscore}{\kern0pt}{\isadigit{5}}x{\isadigit{5}}{\isacharunderscore}{\kern0pt}lr{\isacharunderscore}{\kern0pt}last{\isacharcolon}{\kern0pt}\ {\isachardoublequoteopen}last\ kp{\isadigit{5}}x{\isadigit{5}}lr\ {\isacharequal}{\kern0pt}\ {\isacharparenleft}{\kern0pt}{\isadigit{2}}{\isacharcomma}{\kern0pt}{\isadigit{4}}{\isacharparenright}{\kern0pt}{\isachardoublequoteclose}%
\isadelimproof
\ %
\endisadelimproof
%
\isatagproof
\isacommand{by}\isamarkupfalse%
\ eval%
\endisatagproof
{\isafoldproof}%
%
\isadelimproof
%
\endisadelimproof
\isanewline
\isanewline
\isacommand{lemma}\isamarkupfalse%
\ kp{\isacharunderscore}{\kern0pt}{\isadigit{5}}x{\isadigit{5}}{\isacharunderscore}{\kern0pt}lr{\isacharunderscore}{\kern0pt}non{\isacharunderscore}{\kern0pt}nil{\isacharcolon}{\kern0pt}\ {\isachardoublequoteopen}kp{\isadigit{5}}x{\isadigit{5}}lr\ {\isasymnoteq}\ {\isacharbrackleft}{\kern0pt}{\isacharbrackright}{\kern0pt}{\isachardoublequoteclose}%
\isadelimproof
\ %
\endisadelimproof
%
\isatagproof
\isacommand{by}\isamarkupfalse%
\ eval%
\endisatagproof
{\isafoldproof}%
%
\isadelimproof
%
\endisadelimproof
%
\begin{isamarkuptext}%
A Knight's path for the \isa{{\isacharparenleft}{\kern0pt}{\isadigit{5}}{\isasymtimes}{\isadigit{5}}{\isacharparenright}{\kern0pt}}-board that starts in the lower-left and ends in the 
upper-right.
  \begin{table}[H]
    \begin{tabular}{lllll}
       7 & 12 & 15 & 20 &  5 \\
      16 & 21 &  6 & 25 & 14 \\
      11 &  8 & 13 &  4 & 19 \\
      22 & 17 &  2 &  9 & 24 \\
       1 & 10 & 23 & 18 &  3
    \end{tabular}
  \end{table}%
\end{isamarkuptext}\isamarkuptrue%
\isacommand{abbreviation}\isamarkupfalse%
\ {\isachardoublequoteopen}kp{\isadigit{5}}x{\isadigit{5}}ur\ {\isasymequiv}\ the\ {\isacharparenleft}{\kern0pt}to{\isacharunderscore}{\kern0pt}path\ \isanewline
\ \ {\isacharbrackleft}{\kern0pt}{\isacharbrackleft}{\kern0pt}{\isadigit{7}}{\isacharcomma}{\kern0pt}{\isadigit{1}}{\isadigit{2}}{\isacharcomma}{\kern0pt}{\isadigit{1}}{\isadigit{5}}{\isacharcomma}{\kern0pt}{\isadigit{2}}{\isadigit{0}}{\isacharcomma}{\kern0pt}{\isadigit{5}}{\isacharbrackright}{\kern0pt}{\isacharcomma}{\kern0pt}\isanewline
\ \ {\isacharbrackleft}{\kern0pt}{\isadigit{1}}{\isadigit{6}}{\isacharcomma}{\kern0pt}{\isadigit{2}}{\isadigit{1}}{\isacharcomma}{\kern0pt}{\isadigit{6}}{\isacharcomma}{\kern0pt}{\isadigit{2}}{\isadigit{5}}{\isacharcomma}{\kern0pt}{\isadigit{1}}{\isadigit{4}}{\isacharbrackright}{\kern0pt}{\isacharcomma}{\kern0pt}\isanewline
\ \ {\isacharbrackleft}{\kern0pt}{\isadigit{1}}{\isadigit{1}}{\isacharcomma}{\kern0pt}{\isadigit{8}}{\isacharcomma}{\kern0pt}{\isadigit{1}}{\isadigit{3}}{\isacharcomma}{\kern0pt}{\isadigit{4}}{\isacharcomma}{\kern0pt}{\isadigit{1}}{\isadigit{9}}{\isacharbrackright}{\kern0pt}{\isacharcomma}{\kern0pt}\isanewline
\ \ {\isacharbrackleft}{\kern0pt}{\isadigit{2}}{\isadigit{2}}{\isacharcomma}{\kern0pt}{\isadigit{1}}{\isadigit{7}}{\isacharcomma}{\kern0pt}{\isadigit{2}}{\isacharcomma}{\kern0pt}{\isadigit{9}}{\isacharcomma}{\kern0pt}{\isadigit{2}}{\isadigit{4}}{\isacharbrackright}{\kern0pt}{\isacharcomma}{\kern0pt}\isanewline
\ \ {\isacharbrackleft}{\kern0pt}{\isadigit{1}}{\isacharcomma}{\kern0pt}{\isadigit{1}}{\isadigit{0}}{\isacharcomma}{\kern0pt}{\isadigit{2}}{\isadigit{3}}{\isacharcomma}{\kern0pt}{\isadigit{1}}{\isadigit{8}}{\isacharcomma}{\kern0pt}{\isadigit{3}}{\isacharbrackright}{\kern0pt}{\isacharbrackright}{\kern0pt}{\isacharparenright}{\kern0pt}{\isachardoublequoteclose}\isanewline
\isacommand{lemma}\isamarkupfalse%
\ kp{\isacharunderscore}{\kern0pt}{\isadigit{5}}x{\isadigit{5}}{\isacharunderscore}{\kern0pt}ur{\isacharcolon}{\kern0pt}\ {\isachardoublequoteopen}knights{\isacharunderscore}{\kern0pt}path\ b{\isadigit{5}}x{\isadigit{5}}\ kp{\isadigit{5}}x{\isadigit{5}}ur{\isachardoublequoteclose}\isanewline
%
\isadelimproof
\ \ %
\endisadelimproof
%
\isatagproof
\isacommand{by}\isamarkupfalse%
\ {\isacharparenleft}{\kern0pt}simp\ only{\isacharcolon}{\kern0pt}\ knights{\isacharunderscore}{\kern0pt}path{\isacharunderscore}{\kern0pt}exec{\isacharunderscore}{\kern0pt}simp{\isacharparenright}{\kern0pt}\ eval%
\endisatagproof
{\isafoldproof}%
%
\isadelimproof
\isanewline
%
\endisadelimproof
\isanewline
\isacommand{lemma}\isamarkupfalse%
\ kp{\isacharunderscore}{\kern0pt}{\isadigit{5}}x{\isadigit{5}}{\isacharunderscore}{\kern0pt}ur{\isacharunderscore}{\kern0pt}hd{\isacharcolon}{\kern0pt}\ {\isachardoublequoteopen}hd\ kp{\isadigit{5}}x{\isadigit{5}}ur\ {\isacharequal}{\kern0pt}\ {\isacharparenleft}{\kern0pt}{\isadigit{1}}{\isacharcomma}{\kern0pt}{\isadigit{1}}{\isacharparenright}{\kern0pt}{\isachardoublequoteclose}%
\isadelimproof
\ %
\endisadelimproof
%
\isatagproof
\isacommand{by}\isamarkupfalse%
\ eval%
\endisatagproof
{\isafoldproof}%
%
\isadelimproof
%
\endisadelimproof
\isanewline
\isanewline
\isacommand{lemma}\isamarkupfalse%
\ kp{\isacharunderscore}{\kern0pt}{\isadigit{5}}x{\isadigit{5}}{\isacharunderscore}{\kern0pt}ur{\isacharunderscore}{\kern0pt}last{\isacharcolon}{\kern0pt}\ {\isachardoublequoteopen}last\ kp{\isadigit{5}}x{\isadigit{5}}ur\ {\isacharequal}{\kern0pt}\ {\isacharparenleft}{\kern0pt}{\isadigit{4}}{\isacharcomma}{\kern0pt}{\isadigit{4}}{\isacharparenright}{\kern0pt}{\isachardoublequoteclose}%
\isadelimproof
\ %
\endisadelimproof
%
\isatagproof
\isacommand{by}\isamarkupfalse%
\ eval%
\endisatagproof
{\isafoldproof}%
%
\isadelimproof
%
\endisadelimproof
\isanewline
\isanewline
\isacommand{lemma}\isamarkupfalse%
\ kp{\isacharunderscore}{\kern0pt}{\isadigit{5}}x{\isadigit{5}}{\isacharunderscore}{\kern0pt}ur{\isacharunderscore}{\kern0pt}non{\isacharunderscore}{\kern0pt}nil{\isacharcolon}{\kern0pt}\ {\isachardoublequoteopen}kp{\isadigit{5}}x{\isadigit{5}}ur\ {\isasymnoteq}\ {\isacharbrackleft}{\kern0pt}{\isacharbrackright}{\kern0pt}{\isachardoublequoteclose}%
\isadelimproof
\ %
\endisadelimproof
%
\isatagproof
\isacommand{by}\isamarkupfalse%
\ eval%
\endisatagproof
{\isafoldproof}%
%
\isadelimproof
%
\endisadelimproof
\isanewline
\isanewline
\isacommand{abbreviation}\isamarkupfalse%
\ {\isachardoublequoteopen}b{\isadigit{5}}x{\isadigit{6}}\ {\isasymequiv}\ board\ {\isadigit{5}}\ {\isadigit{6}}{\isachardoublequoteclose}%
\begin{isamarkuptext}%
A Knight's path for the \isa{{\isacharparenleft}{\kern0pt}{\isadigit{5}}{\isasymtimes}{\isadigit{6}}{\isacharparenright}{\kern0pt}}-board that starts in the lower-left and ends in the 
lower-right.
  \begin{table}[H]
    \begin{tabular}{llllll}
       7 & 14 & 21 & 28 &  5 & 12 \\
      22 & 27 &  6 & 13 & 20 & 29 \\
      15 &  8 & 17 & 24 & 11 &  4 \\
      26 & 23 &  2 &  9 & 30 & 19 \\
       1 & 16 & 25 & 18 &  3 & 10
    \end{tabular}
  \end{table}%
\end{isamarkuptext}\isamarkuptrue%
\isacommand{abbreviation}\isamarkupfalse%
\ {\isachardoublequoteopen}kp{\isadigit{5}}x{\isadigit{6}}lr\ {\isasymequiv}\ the\ {\isacharparenleft}{\kern0pt}to{\isacharunderscore}{\kern0pt}path\ \isanewline
\ \ {\isacharbrackleft}{\kern0pt}{\isacharbrackleft}{\kern0pt}{\isadigit{7}}{\isacharcomma}{\kern0pt}{\isadigit{1}}{\isadigit{4}}{\isacharcomma}{\kern0pt}{\isadigit{2}}{\isadigit{1}}{\isacharcomma}{\kern0pt}{\isadigit{2}}{\isadigit{8}}{\isacharcomma}{\kern0pt}{\isadigit{5}}{\isacharcomma}{\kern0pt}{\isadigit{1}}{\isadigit{2}}{\isacharbrackright}{\kern0pt}{\isacharcomma}{\kern0pt}\isanewline
\ \ {\isacharbrackleft}{\kern0pt}{\isadigit{2}}{\isadigit{2}}{\isacharcomma}{\kern0pt}{\isadigit{2}}{\isadigit{7}}{\isacharcomma}{\kern0pt}{\isadigit{6}}{\isacharcomma}{\kern0pt}{\isadigit{1}}{\isadigit{3}}{\isacharcomma}{\kern0pt}{\isadigit{2}}{\isadigit{0}}{\isacharcomma}{\kern0pt}{\isadigit{2}}{\isadigit{9}}{\isacharbrackright}{\kern0pt}{\isacharcomma}{\kern0pt}\isanewline
\ \ {\isacharbrackleft}{\kern0pt}{\isadigit{1}}{\isadigit{5}}{\isacharcomma}{\kern0pt}{\isadigit{8}}{\isacharcomma}{\kern0pt}{\isadigit{1}}{\isadigit{7}}{\isacharcomma}{\kern0pt}{\isadigit{2}}{\isadigit{4}}{\isacharcomma}{\kern0pt}{\isadigit{1}}{\isadigit{1}}{\isacharcomma}{\kern0pt}{\isadigit{4}}{\isacharbrackright}{\kern0pt}{\isacharcomma}{\kern0pt}\isanewline
\ \ {\isacharbrackleft}{\kern0pt}{\isadigit{2}}{\isadigit{6}}{\isacharcomma}{\kern0pt}{\isadigit{2}}{\isadigit{3}}{\isacharcomma}{\kern0pt}{\isadigit{2}}{\isacharcomma}{\kern0pt}{\isadigit{9}}{\isacharcomma}{\kern0pt}{\isadigit{3}}{\isadigit{0}}{\isacharcomma}{\kern0pt}{\isadigit{1}}{\isadigit{9}}{\isacharbrackright}{\kern0pt}{\isacharcomma}{\kern0pt}\isanewline
\ \ {\isacharbrackleft}{\kern0pt}{\isadigit{1}}{\isacharcomma}{\kern0pt}{\isadigit{1}}{\isadigit{6}}{\isacharcomma}{\kern0pt}{\isadigit{2}}{\isadigit{5}}{\isacharcomma}{\kern0pt}{\isadigit{1}}{\isadigit{8}}{\isacharcomma}{\kern0pt}{\isadigit{3}}{\isacharcomma}{\kern0pt}{\isadigit{1}}{\isadigit{0}}{\isacharbrackright}{\kern0pt}{\isacharbrackright}{\kern0pt}{\isacharparenright}{\kern0pt}{\isachardoublequoteclose}\isanewline
\isacommand{lemma}\isamarkupfalse%
\ kp{\isacharunderscore}{\kern0pt}{\isadigit{5}}x{\isadigit{6}}{\isacharunderscore}{\kern0pt}lr{\isacharcolon}{\kern0pt}\ {\isachardoublequoteopen}knights{\isacharunderscore}{\kern0pt}path\ b{\isadigit{5}}x{\isadigit{6}}\ kp{\isadigit{5}}x{\isadigit{6}}lr{\isachardoublequoteclose}\isanewline
%
\isadelimproof
\ \ %
\endisadelimproof
%
\isatagproof
\isacommand{by}\isamarkupfalse%
\ {\isacharparenleft}{\kern0pt}simp\ only{\isacharcolon}{\kern0pt}\ knights{\isacharunderscore}{\kern0pt}path{\isacharunderscore}{\kern0pt}exec{\isacharunderscore}{\kern0pt}simp{\isacharparenright}{\kern0pt}\ eval%
\endisatagproof
{\isafoldproof}%
%
\isadelimproof
\isanewline
%
\endisadelimproof
\isanewline
\isacommand{lemma}\isamarkupfalse%
\ kp{\isacharunderscore}{\kern0pt}{\isadigit{5}}x{\isadigit{6}}{\isacharunderscore}{\kern0pt}lr{\isacharunderscore}{\kern0pt}hd{\isacharcolon}{\kern0pt}\ {\isachardoublequoteopen}hd\ kp{\isadigit{5}}x{\isadigit{6}}lr\ {\isacharequal}{\kern0pt}\ {\isacharparenleft}{\kern0pt}{\isadigit{1}}{\isacharcomma}{\kern0pt}{\isadigit{1}}{\isacharparenright}{\kern0pt}{\isachardoublequoteclose}%
\isadelimproof
\ %
\endisadelimproof
%
\isatagproof
\isacommand{by}\isamarkupfalse%
\ eval%
\endisatagproof
{\isafoldproof}%
%
\isadelimproof
%
\endisadelimproof
\isanewline
\isanewline
\isacommand{lemma}\isamarkupfalse%
\ kp{\isacharunderscore}{\kern0pt}{\isadigit{5}}x{\isadigit{6}}{\isacharunderscore}{\kern0pt}lr{\isacharunderscore}{\kern0pt}last{\isacharcolon}{\kern0pt}\ {\isachardoublequoteopen}last\ kp{\isadigit{5}}x{\isadigit{6}}lr\ {\isacharequal}{\kern0pt}\ {\isacharparenleft}{\kern0pt}{\isadigit{2}}{\isacharcomma}{\kern0pt}{\isadigit{5}}{\isacharparenright}{\kern0pt}{\isachardoublequoteclose}%
\isadelimproof
\ %
\endisadelimproof
%
\isatagproof
\isacommand{by}\isamarkupfalse%
\ eval%
\endisatagproof
{\isafoldproof}%
%
\isadelimproof
%
\endisadelimproof
\isanewline
\isanewline
\isacommand{lemma}\isamarkupfalse%
\ kp{\isacharunderscore}{\kern0pt}{\isadigit{5}}x{\isadigit{6}}{\isacharunderscore}{\kern0pt}lr{\isacharunderscore}{\kern0pt}non{\isacharunderscore}{\kern0pt}nil{\isacharcolon}{\kern0pt}\ {\isachardoublequoteopen}kp{\isadigit{5}}x{\isadigit{6}}lr\ {\isasymnoteq}\ {\isacharbrackleft}{\kern0pt}{\isacharbrackright}{\kern0pt}{\isachardoublequoteclose}%
\isadelimproof
\ %
\endisadelimproof
%
\isatagproof
\isacommand{by}\isamarkupfalse%
\ eval%
\endisatagproof
{\isafoldproof}%
%
\isadelimproof
%
\endisadelimproof
%
\begin{isamarkuptext}%
A Knight's path for the \isa{{\isacharparenleft}{\kern0pt}{\isadigit{5}}{\isasymtimes}{\isadigit{6}}{\isacharparenright}{\kern0pt}}-board that starts in the lower-left and ends in the 
upper-right.
  \begin{table}[H]
    \begin{tabular}{llllll}
       3 & 10 & 29 & 20 &  5 & 12 \\
      28 & 19 &  4 & 11 & 30 & 21 \\
       9 &  2 & 17 & 24 & 13 &  6 \\
      18 & 27 &  8 & 15 & 22 & 25 \\
       1 & 16 & 23 & 26 &  7 & 14
    \end{tabular}
  \end{table}%
\end{isamarkuptext}\isamarkuptrue%
\isacommand{abbreviation}\isamarkupfalse%
\ {\isachardoublequoteopen}kp{\isadigit{5}}x{\isadigit{6}}ur\ {\isasymequiv}\ the\ {\isacharparenleft}{\kern0pt}to{\isacharunderscore}{\kern0pt}path\ \isanewline
\ \ {\isacharbrackleft}{\kern0pt}{\isacharbrackleft}{\kern0pt}{\isadigit{3}}{\isacharcomma}{\kern0pt}{\isadigit{1}}{\isadigit{0}}{\isacharcomma}{\kern0pt}{\isadigit{2}}{\isadigit{9}}{\isacharcomma}{\kern0pt}{\isadigit{2}}{\isadigit{0}}{\isacharcomma}{\kern0pt}{\isadigit{5}}{\isacharcomma}{\kern0pt}{\isadigit{1}}{\isadigit{2}}{\isacharbrackright}{\kern0pt}{\isacharcomma}{\kern0pt}\isanewline
\ \ {\isacharbrackleft}{\kern0pt}{\isadigit{2}}{\isadigit{8}}{\isacharcomma}{\kern0pt}{\isadigit{1}}{\isadigit{9}}{\isacharcomma}{\kern0pt}{\isadigit{4}}{\isacharcomma}{\kern0pt}{\isadigit{1}}{\isadigit{1}}{\isacharcomma}{\kern0pt}{\isadigit{3}}{\isadigit{0}}{\isacharcomma}{\kern0pt}{\isadigit{2}}{\isadigit{1}}{\isacharbrackright}{\kern0pt}{\isacharcomma}{\kern0pt}\isanewline
\ \ {\isacharbrackleft}{\kern0pt}{\isadigit{9}}{\isacharcomma}{\kern0pt}{\isadigit{2}}{\isacharcomma}{\kern0pt}{\isadigit{1}}{\isadigit{7}}{\isacharcomma}{\kern0pt}{\isadigit{2}}{\isadigit{4}}{\isacharcomma}{\kern0pt}{\isadigit{1}}{\isadigit{3}}{\isacharcomma}{\kern0pt}{\isadigit{6}}{\isacharbrackright}{\kern0pt}{\isacharcomma}{\kern0pt}\isanewline
\ \ {\isacharbrackleft}{\kern0pt}{\isadigit{1}}{\isadigit{8}}{\isacharcomma}{\kern0pt}{\isadigit{2}}{\isadigit{7}}{\isacharcomma}{\kern0pt}{\isadigit{8}}{\isacharcomma}{\kern0pt}{\isadigit{1}}{\isadigit{5}}{\isacharcomma}{\kern0pt}{\isadigit{2}}{\isadigit{2}}{\isacharcomma}{\kern0pt}{\isadigit{2}}{\isadigit{5}}{\isacharbrackright}{\kern0pt}{\isacharcomma}{\kern0pt}\isanewline
\ \ {\isacharbrackleft}{\kern0pt}{\isadigit{1}}{\isacharcomma}{\kern0pt}{\isadigit{1}}{\isadigit{6}}{\isacharcomma}{\kern0pt}{\isadigit{2}}{\isadigit{3}}{\isacharcomma}{\kern0pt}{\isadigit{2}}{\isadigit{6}}{\isacharcomma}{\kern0pt}{\isadigit{7}}{\isacharcomma}{\kern0pt}{\isadigit{1}}{\isadigit{4}}{\isacharbrackright}{\kern0pt}{\isacharbrackright}{\kern0pt}{\isacharparenright}{\kern0pt}{\isachardoublequoteclose}\isanewline
\isacommand{lemma}\isamarkupfalse%
\ kp{\isacharunderscore}{\kern0pt}{\isadigit{5}}x{\isadigit{6}}{\isacharunderscore}{\kern0pt}ur{\isacharcolon}{\kern0pt}\ {\isachardoublequoteopen}knights{\isacharunderscore}{\kern0pt}path\ b{\isadigit{5}}x{\isadigit{6}}\ kp{\isadigit{5}}x{\isadigit{6}}ur{\isachardoublequoteclose}\isanewline
%
\isadelimproof
\ \ %
\endisadelimproof
%
\isatagproof
\isacommand{by}\isamarkupfalse%
\ {\isacharparenleft}{\kern0pt}simp\ only{\isacharcolon}{\kern0pt}\ knights{\isacharunderscore}{\kern0pt}path{\isacharunderscore}{\kern0pt}exec{\isacharunderscore}{\kern0pt}simp{\isacharparenright}{\kern0pt}\ eval%
\endisatagproof
{\isafoldproof}%
%
\isadelimproof
\isanewline
%
\endisadelimproof
\isanewline
\isacommand{lemma}\isamarkupfalse%
\ kp{\isacharunderscore}{\kern0pt}{\isadigit{5}}x{\isadigit{6}}{\isacharunderscore}{\kern0pt}ur{\isacharunderscore}{\kern0pt}hd{\isacharcolon}{\kern0pt}\ {\isachardoublequoteopen}hd\ kp{\isadigit{5}}x{\isadigit{6}}ur\ {\isacharequal}{\kern0pt}\ {\isacharparenleft}{\kern0pt}{\isadigit{1}}{\isacharcomma}{\kern0pt}{\isadigit{1}}{\isacharparenright}{\kern0pt}{\isachardoublequoteclose}%
\isadelimproof
\ %
\endisadelimproof
%
\isatagproof
\isacommand{by}\isamarkupfalse%
\ eval%
\endisatagproof
{\isafoldproof}%
%
\isadelimproof
%
\endisadelimproof
\isanewline
\isanewline
\isacommand{lemma}\isamarkupfalse%
\ kp{\isacharunderscore}{\kern0pt}{\isadigit{5}}x{\isadigit{6}}{\isacharunderscore}{\kern0pt}ur{\isacharunderscore}{\kern0pt}last{\isacharcolon}{\kern0pt}\ {\isachardoublequoteopen}last\ kp{\isadigit{5}}x{\isadigit{6}}ur\ {\isacharequal}{\kern0pt}\ {\isacharparenleft}{\kern0pt}{\isadigit{4}}{\isacharcomma}{\kern0pt}{\isadigit{5}}{\isacharparenright}{\kern0pt}{\isachardoublequoteclose}%
\isadelimproof
\ %
\endisadelimproof
%
\isatagproof
\isacommand{by}\isamarkupfalse%
\ eval%
\endisatagproof
{\isafoldproof}%
%
\isadelimproof
%
\endisadelimproof
\isanewline
\isanewline
\isacommand{lemma}\isamarkupfalse%
\ kp{\isacharunderscore}{\kern0pt}{\isadigit{5}}x{\isadigit{6}}{\isacharunderscore}{\kern0pt}ur{\isacharunderscore}{\kern0pt}non{\isacharunderscore}{\kern0pt}nil{\isacharcolon}{\kern0pt}\ {\isachardoublequoteopen}kp{\isadigit{5}}x{\isadigit{6}}ur\ {\isasymnoteq}\ {\isacharbrackleft}{\kern0pt}{\isacharbrackright}{\kern0pt}{\isachardoublequoteclose}%
\isadelimproof
\ %
\endisadelimproof
%
\isatagproof
\isacommand{by}\isamarkupfalse%
\ eval%
\endisatagproof
{\isafoldproof}%
%
\isadelimproof
%
\endisadelimproof
\isanewline
\isanewline
\isacommand{abbreviation}\isamarkupfalse%
\ {\isachardoublequoteopen}b{\isadigit{5}}x{\isadigit{7}}\ {\isasymequiv}\ board\ {\isadigit{5}}\ {\isadigit{7}}{\isachardoublequoteclose}%
\begin{isamarkuptext}%
A Knight's path for the \isa{{\isacharparenleft}{\kern0pt}{\isadigit{5}}{\isasymtimes}{\isadigit{7}}{\isacharparenright}{\kern0pt}}-board that starts in the lower-left and ends in the 
lower-right.
  \begin{table}[H]
    \begin{tabular}{lllllll}
       3 & 12 & 21 & 30 &  5 & 14 & 23 \\
      20 & 29 &  4 & 13 & 22 & 31 &  6 \\
      11 &  2 & 19 & 32 &  7 & 24 & 15 \\
      28 & 33 & 10 & 17 & 26 & 35 &  8 \\
       1 & 18 & 27 & 34 &  9 & 16 & 25
    \end{tabular}
  \end{table}%
\end{isamarkuptext}\isamarkuptrue%
\isacommand{abbreviation}\isamarkupfalse%
\ {\isachardoublequoteopen}kp{\isadigit{5}}x{\isadigit{7}}lr\ {\isasymequiv}\ the\ {\isacharparenleft}{\kern0pt}to{\isacharunderscore}{\kern0pt}path\ \isanewline
\ \ {\isacharbrackleft}{\kern0pt}{\isacharbrackleft}{\kern0pt}{\isadigit{3}}{\isacharcomma}{\kern0pt}{\isadigit{1}}{\isadigit{2}}{\isacharcomma}{\kern0pt}{\isadigit{2}}{\isadigit{1}}{\isacharcomma}{\kern0pt}{\isadigit{3}}{\isadigit{0}}{\isacharcomma}{\kern0pt}{\isadigit{5}}{\isacharcomma}{\kern0pt}{\isadigit{1}}{\isadigit{4}}{\isacharcomma}{\kern0pt}{\isadigit{2}}{\isadigit{3}}{\isacharbrackright}{\kern0pt}{\isacharcomma}{\kern0pt}\isanewline
\ \ {\isacharbrackleft}{\kern0pt}{\isadigit{2}}{\isadigit{0}}{\isacharcomma}{\kern0pt}{\isadigit{2}}{\isadigit{9}}{\isacharcomma}{\kern0pt}{\isadigit{4}}{\isacharcomma}{\kern0pt}{\isadigit{1}}{\isadigit{3}}{\isacharcomma}{\kern0pt}{\isadigit{2}}{\isadigit{2}}{\isacharcomma}{\kern0pt}{\isadigit{3}}{\isadigit{1}}{\isacharcomma}{\kern0pt}{\isadigit{6}}{\isacharbrackright}{\kern0pt}{\isacharcomma}{\kern0pt}\isanewline
\ \ {\isacharbrackleft}{\kern0pt}{\isadigit{1}}{\isadigit{1}}{\isacharcomma}{\kern0pt}{\isadigit{2}}{\isacharcomma}{\kern0pt}{\isadigit{1}}{\isadigit{9}}{\isacharcomma}{\kern0pt}{\isadigit{3}}{\isadigit{2}}{\isacharcomma}{\kern0pt}{\isadigit{7}}{\isacharcomma}{\kern0pt}{\isadigit{2}}{\isadigit{4}}{\isacharcomma}{\kern0pt}{\isadigit{1}}{\isadigit{5}}{\isacharbrackright}{\kern0pt}{\isacharcomma}{\kern0pt}\isanewline
\ \ {\isacharbrackleft}{\kern0pt}{\isadigit{2}}{\isadigit{8}}{\isacharcomma}{\kern0pt}{\isadigit{3}}{\isadigit{3}}{\isacharcomma}{\kern0pt}{\isadigit{1}}{\isadigit{0}}{\isacharcomma}{\kern0pt}{\isadigit{1}}{\isadigit{7}}{\isacharcomma}{\kern0pt}{\isadigit{2}}{\isadigit{6}}{\isacharcomma}{\kern0pt}{\isadigit{3}}{\isadigit{5}}{\isacharcomma}{\kern0pt}{\isadigit{8}}{\isacharbrackright}{\kern0pt}{\isacharcomma}{\kern0pt}\isanewline
\ \ {\isacharbrackleft}{\kern0pt}{\isadigit{1}}{\isacharcomma}{\kern0pt}{\isadigit{1}}{\isadigit{8}}{\isacharcomma}{\kern0pt}{\isadigit{2}}{\isadigit{7}}{\isacharcomma}{\kern0pt}{\isadigit{3}}{\isadigit{4}}{\isacharcomma}{\kern0pt}{\isadigit{9}}{\isacharcomma}{\kern0pt}{\isadigit{1}}{\isadigit{6}}{\isacharcomma}{\kern0pt}{\isadigit{2}}{\isadigit{5}}{\isacharbrackright}{\kern0pt}{\isacharbrackright}{\kern0pt}{\isacharparenright}{\kern0pt}{\isachardoublequoteclose}\isanewline
\isacommand{lemma}\isamarkupfalse%
\ kp{\isacharunderscore}{\kern0pt}{\isadigit{5}}x{\isadigit{7}}{\isacharunderscore}{\kern0pt}lr{\isacharcolon}{\kern0pt}\ {\isachardoublequoteopen}knights{\isacharunderscore}{\kern0pt}path\ b{\isadigit{5}}x{\isadigit{7}}\ kp{\isadigit{5}}x{\isadigit{7}}lr{\isachardoublequoteclose}\isanewline
%
\isadelimproof
\ \ %
\endisadelimproof
%
\isatagproof
\isacommand{by}\isamarkupfalse%
\ {\isacharparenleft}{\kern0pt}simp\ only{\isacharcolon}{\kern0pt}\ knights{\isacharunderscore}{\kern0pt}path{\isacharunderscore}{\kern0pt}exec{\isacharunderscore}{\kern0pt}simp{\isacharparenright}{\kern0pt}\ eval%
\endisatagproof
{\isafoldproof}%
%
\isadelimproof
\isanewline
%
\endisadelimproof
\isanewline
\isacommand{lemma}\isamarkupfalse%
\ kp{\isacharunderscore}{\kern0pt}{\isadigit{5}}x{\isadigit{7}}{\isacharunderscore}{\kern0pt}lr{\isacharunderscore}{\kern0pt}hd{\isacharcolon}{\kern0pt}\ {\isachardoublequoteopen}hd\ kp{\isadigit{5}}x{\isadigit{7}}lr\ {\isacharequal}{\kern0pt}\ {\isacharparenleft}{\kern0pt}{\isadigit{1}}{\isacharcomma}{\kern0pt}{\isadigit{1}}{\isacharparenright}{\kern0pt}{\isachardoublequoteclose}%
\isadelimproof
\ %
\endisadelimproof
%
\isatagproof
\isacommand{by}\isamarkupfalse%
\ eval%
\endisatagproof
{\isafoldproof}%
%
\isadelimproof
%
\endisadelimproof
\isanewline
\isanewline
\isacommand{lemma}\isamarkupfalse%
\ kp{\isacharunderscore}{\kern0pt}{\isadigit{5}}x{\isadigit{7}}{\isacharunderscore}{\kern0pt}lr{\isacharunderscore}{\kern0pt}last{\isacharcolon}{\kern0pt}\ {\isachardoublequoteopen}last\ kp{\isadigit{5}}x{\isadigit{7}}lr\ {\isacharequal}{\kern0pt}\ {\isacharparenleft}{\kern0pt}{\isadigit{2}}{\isacharcomma}{\kern0pt}{\isadigit{6}}{\isacharparenright}{\kern0pt}{\isachardoublequoteclose}%
\isadelimproof
\ %
\endisadelimproof
%
\isatagproof
\isacommand{by}\isamarkupfalse%
\ eval%
\endisatagproof
{\isafoldproof}%
%
\isadelimproof
%
\endisadelimproof
\isanewline
\isanewline
\isacommand{lemma}\isamarkupfalse%
\ kp{\isacharunderscore}{\kern0pt}{\isadigit{5}}x{\isadigit{7}}{\isacharunderscore}{\kern0pt}lr{\isacharunderscore}{\kern0pt}non{\isacharunderscore}{\kern0pt}nil{\isacharcolon}{\kern0pt}\ {\isachardoublequoteopen}kp{\isadigit{5}}x{\isadigit{7}}lr\ {\isasymnoteq}\ {\isacharbrackleft}{\kern0pt}{\isacharbrackright}{\kern0pt}{\isachardoublequoteclose}%
\isadelimproof
\ %
\endisadelimproof
%
\isatagproof
\isacommand{by}\isamarkupfalse%
\ eval%
\endisatagproof
{\isafoldproof}%
%
\isadelimproof
%
\endisadelimproof
%
\begin{isamarkuptext}%
A Knight's path for the \isa{{\isacharparenleft}{\kern0pt}{\isadigit{5}}{\isasymtimes}{\isadigit{7}}{\isacharparenright}{\kern0pt}}-board that starts in the lower-left and ends in the 
upper-right.
  \begin{table}[H]
    \begin{tabular}{lllllll}
       3 & 32 & 11 & 34 &  5 & 26 & 13 \\
      10 & 19 &  4 & 25 & 12 & 35 &  6 \\
      31 &  2 & 33 & 20 & 23 & 14 & 27 \\
      18 &  9 & 24 & 29 & 16 &  7 & 22 \\
       1 & 30 & 17 &  8 & 21 & 28 & 15
    \end{tabular}
  \end{table}%
\end{isamarkuptext}\isamarkuptrue%
\isacommand{abbreviation}\isamarkupfalse%
\ {\isachardoublequoteopen}kp{\isadigit{5}}x{\isadigit{7}}ur\ {\isasymequiv}\ the\ {\isacharparenleft}{\kern0pt}to{\isacharunderscore}{\kern0pt}path\ \isanewline
\ \ {\isacharbrackleft}{\kern0pt}{\isacharbrackleft}{\kern0pt}{\isadigit{3}}{\isacharcomma}{\kern0pt}{\isadigit{3}}{\isadigit{2}}{\isacharcomma}{\kern0pt}{\isadigit{1}}{\isadigit{1}}{\isacharcomma}{\kern0pt}{\isadigit{3}}{\isadigit{4}}{\isacharcomma}{\kern0pt}{\isadigit{5}}{\isacharcomma}{\kern0pt}{\isadigit{2}}{\isadigit{6}}{\isacharcomma}{\kern0pt}{\isadigit{1}}{\isadigit{3}}{\isacharbrackright}{\kern0pt}{\isacharcomma}{\kern0pt}\isanewline
\ \ {\isacharbrackleft}{\kern0pt}{\isadigit{1}}{\isadigit{0}}{\isacharcomma}{\kern0pt}{\isadigit{1}}{\isadigit{9}}{\isacharcomma}{\kern0pt}{\isadigit{4}}{\isacharcomma}{\kern0pt}{\isadigit{2}}{\isadigit{5}}{\isacharcomma}{\kern0pt}{\isadigit{1}}{\isadigit{2}}{\isacharcomma}{\kern0pt}{\isadigit{3}}{\isadigit{5}}{\isacharcomma}{\kern0pt}{\isadigit{6}}{\isacharbrackright}{\kern0pt}{\isacharcomma}{\kern0pt}\isanewline
\ \ {\isacharbrackleft}{\kern0pt}{\isadigit{3}}{\isadigit{1}}{\isacharcomma}{\kern0pt}{\isadigit{2}}{\isacharcomma}{\kern0pt}{\isadigit{3}}{\isadigit{3}}{\isacharcomma}{\kern0pt}{\isadigit{2}}{\isadigit{0}}{\isacharcomma}{\kern0pt}{\isadigit{2}}{\isadigit{3}}{\isacharcomma}{\kern0pt}{\isadigit{1}}{\isadigit{4}}{\isacharcomma}{\kern0pt}{\isadigit{2}}{\isadigit{7}}{\isacharbrackright}{\kern0pt}{\isacharcomma}{\kern0pt}\isanewline
\ \ {\isacharbrackleft}{\kern0pt}{\isadigit{1}}{\isadigit{8}}{\isacharcomma}{\kern0pt}{\isadigit{9}}{\isacharcomma}{\kern0pt}{\isadigit{2}}{\isadigit{4}}{\isacharcomma}{\kern0pt}{\isadigit{2}}{\isadigit{9}}{\isacharcomma}{\kern0pt}{\isadigit{1}}{\isadigit{6}}{\isacharcomma}{\kern0pt}{\isadigit{7}}{\isacharcomma}{\kern0pt}{\isadigit{2}}{\isadigit{2}}{\isacharbrackright}{\kern0pt}{\isacharcomma}{\kern0pt}\isanewline
\ \ {\isacharbrackleft}{\kern0pt}{\isadigit{1}}{\isacharcomma}{\kern0pt}{\isadigit{3}}{\isadigit{0}}{\isacharcomma}{\kern0pt}{\isadigit{1}}{\isadigit{7}}{\isacharcomma}{\kern0pt}{\isadigit{8}}{\isacharcomma}{\kern0pt}{\isadigit{2}}{\isadigit{1}}{\isacharcomma}{\kern0pt}{\isadigit{2}}{\isadigit{8}}{\isacharcomma}{\kern0pt}{\isadigit{1}}{\isadigit{5}}{\isacharbrackright}{\kern0pt}{\isacharbrackright}{\kern0pt}{\isacharparenright}{\kern0pt}{\isachardoublequoteclose}\isanewline
\isacommand{lemma}\isamarkupfalse%
\ kp{\isacharunderscore}{\kern0pt}{\isadigit{5}}x{\isadigit{7}}{\isacharunderscore}{\kern0pt}ur{\isacharcolon}{\kern0pt}\ {\isachardoublequoteopen}knights{\isacharunderscore}{\kern0pt}path\ b{\isadigit{5}}x{\isadigit{7}}\ kp{\isadigit{5}}x{\isadigit{7}}ur{\isachardoublequoteclose}\isanewline
%
\isadelimproof
\ \ %
\endisadelimproof
%
\isatagproof
\isacommand{by}\isamarkupfalse%
\ {\isacharparenleft}{\kern0pt}simp\ only{\isacharcolon}{\kern0pt}\ knights{\isacharunderscore}{\kern0pt}path{\isacharunderscore}{\kern0pt}exec{\isacharunderscore}{\kern0pt}simp{\isacharparenright}{\kern0pt}\ eval%
\endisatagproof
{\isafoldproof}%
%
\isadelimproof
\isanewline
%
\endisadelimproof
\isanewline
\isacommand{lemma}\isamarkupfalse%
\ kp{\isacharunderscore}{\kern0pt}{\isadigit{5}}x{\isadigit{7}}{\isacharunderscore}{\kern0pt}ur{\isacharunderscore}{\kern0pt}hd{\isacharcolon}{\kern0pt}\ {\isachardoublequoteopen}hd\ kp{\isadigit{5}}x{\isadigit{7}}ur\ {\isacharequal}{\kern0pt}\ {\isacharparenleft}{\kern0pt}{\isadigit{1}}{\isacharcomma}{\kern0pt}{\isadigit{1}}{\isacharparenright}{\kern0pt}{\isachardoublequoteclose}%
\isadelimproof
\ %
\endisadelimproof
%
\isatagproof
\isacommand{by}\isamarkupfalse%
\ eval%
\endisatagproof
{\isafoldproof}%
%
\isadelimproof
%
\endisadelimproof
\isanewline
\isanewline
\isacommand{lemma}\isamarkupfalse%
\ kp{\isacharunderscore}{\kern0pt}{\isadigit{5}}x{\isadigit{7}}{\isacharunderscore}{\kern0pt}ur{\isacharunderscore}{\kern0pt}last{\isacharcolon}{\kern0pt}\ {\isachardoublequoteopen}last\ kp{\isadigit{5}}x{\isadigit{7}}ur\ {\isacharequal}{\kern0pt}\ {\isacharparenleft}{\kern0pt}{\isadigit{4}}{\isacharcomma}{\kern0pt}{\isadigit{6}}{\isacharparenright}{\kern0pt}{\isachardoublequoteclose}%
\isadelimproof
\ %
\endisadelimproof
%
\isatagproof
\isacommand{by}\isamarkupfalse%
\ eval%
\endisatagproof
{\isafoldproof}%
%
\isadelimproof
%
\endisadelimproof
\isanewline
\isanewline
\isacommand{lemma}\isamarkupfalse%
\ kp{\isacharunderscore}{\kern0pt}{\isadigit{5}}x{\isadigit{7}}{\isacharunderscore}{\kern0pt}ur{\isacharunderscore}{\kern0pt}non{\isacharunderscore}{\kern0pt}nil{\isacharcolon}{\kern0pt}\ {\isachardoublequoteopen}kp{\isadigit{5}}x{\isadigit{7}}ur\ {\isasymnoteq}\ {\isacharbrackleft}{\kern0pt}{\isacharbrackright}{\kern0pt}{\isachardoublequoteclose}%
\isadelimproof
\ %
\endisadelimproof
%
\isatagproof
\isacommand{by}\isamarkupfalse%
\ eval%
\endisatagproof
{\isafoldproof}%
%
\isadelimproof
%
\endisadelimproof
\isanewline
\isanewline
\isacommand{abbreviation}\isamarkupfalse%
\ {\isachardoublequoteopen}b{\isadigit{5}}x{\isadigit{8}}\ {\isasymequiv}\ board\ {\isadigit{5}}\ {\isadigit{8}}{\isachardoublequoteclose}%
\begin{isamarkuptext}%
A Knight's path for the \isa{{\isacharparenleft}{\kern0pt}{\isadigit{5}}{\isasymtimes}{\isadigit{8}}{\isacharparenright}{\kern0pt}}-board that starts in the lower-left and ends in the 
lower-right.
  \begin{table}[H]
    \begin{tabular}{llllllll}
       3 & 12 & 37 & 26 &  5 & 14 & 17 & 28 \\
      34 & 23 &  4 & 13 & 36 & 27 &  6 & 15 \\
      11 &  2 & 35 & 38 & 25 & 16 & 29 & 18 \\
      22 & 33 & 24 &  9 & 20 & 31 & 40 &  7 \\
       1 & 10 & 21 & 32 & 39 &  8 & 19 & 30
    \end{tabular}
  \end{table}%
\end{isamarkuptext}\isamarkuptrue%
\isacommand{abbreviation}\isamarkupfalse%
\ {\isachardoublequoteopen}kp{\isadigit{5}}x{\isadigit{8}}lr\ {\isasymequiv}\ the\ {\isacharparenleft}{\kern0pt}to{\isacharunderscore}{\kern0pt}path\ \isanewline
\ \ {\isacharbrackleft}{\kern0pt}{\isacharbrackleft}{\kern0pt}{\isadigit{3}}{\isacharcomma}{\kern0pt}{\isadigit{1}}{\isadigit{2}}{\isacharcomma}{\kern0pt}{\isadigit{3}}{\isadigit{7}}{\isacharcomma}{\kern0pt}{\isadigit{2}}{\isadigit{6}}{\isacharcomma}{\kern0pt}{\isadigit{5}}{\isacharcomma}{\kern0pt}{\isadigit{1}}{\isadigit{4}}{\isacharcomma}{\kern0pt}{\isadigit{1}}{\isadigit{7}}{\isacharcomma}{\kern0pt}{\isadigit{2}}{\isadigit{8}}{\isacharbrackright}{\kern0pt}{\isacharcomma}{\kern0pt}\isanewline
\ \ {\isacharbrackleft}{\kern0pt}{\isadigit{3}}{\isadigit{4}}{\isacharcomma}{\kern0pt}{\isadigit{2}}{\isadigit{3}}{\isacharcomma}{\kern0pt}{\isadigit{4}}{\isacharcomma}{\kern0pt}{\isadigit{1}}{\isadigit{3}}{\isacharcomma}{\kern0pt}{\isadigit{3}}{\isadigit{6}}{\isacharcomma}{\kern0pt}{\isadigit{2}}{\isadigit{7}}{\isacharcomma}{\kern0pt}{\isadigit{6}}{\isacharcomma}{\kern0pt}{\isadigit{1}}{\isadigit{5}}{\isacharbrackright}{\kern0pt}{\isacharcomma}{\kern0pt}\isanewline
\ \ {\isacharbrackleft}{\kern0pt}{\isadigit{1}}{\isadigit{1}}{\isacharcomma}{\kern0pt}{\isadigit{2}}{\isacharcomma}{\kern0pt}{\isadigit{3}}{\isadigit{5}}{\isacharcomma}{\kern0pt}{\isadigit{3}}{\isadigit{8}}{\isacharcomma}{\kern0pt}{\isadigit{2}}{\isadigit{5}}{\isacharcomma}{\kern0pt}{\isadigit{1}}{\isadigit{6}}{\isacharcomma}{\kern0pt}{\isadigit{2}}{\isadigit{9}}{\isacharcomma}{\kern0pt}{\isadigit{1}}{\isadigit{8}}{\isacharbrackright}{\kern0pt}{\isacharcomma}{\kern0pt}\isanewline
\ \ {\isacharbrackleft}{\kern0pt}{\isadigit{2}}{\isadigit{2}}{\isacharcomma}{\kern0pt}{\isadigit{3}}{\isadigit{3}}{\isacharcomma}{\kern0pt}{\isadigit{2}}{\isadigit{4}}{\isacharcomma}{\kern0pt}{\isadigit{9}}{\isacharcomma}{\kern0pt}{\isadigit{2}}{\isadigit{0}}{\isacharcomma}{\kern0pt}{\isadigit{3}}{\isadigit{1}}{\isacharcomma}{\kern0pt}{\isadigit{4}}{\isadigit{0}}{\isacharcomma}{\kern0pt}{\isadigit{7}}{\isacharbrackright}{\kern0pt}{\isacharcomma}{\kern0pt}\isanewline
\ \ {\isacharbrackleft}{\kern0pt}{\isadigit{1}}{\isacharcomma}{\kern0pt}{\isadigit{1}}{\isadigit{0}}{\isacharcomma}{\kern0pt}{\isadigit{2}}{\isadigit{1}}{\isacharcomma}{\kern0pt}{\isadigit{3}}{\isadigit{2}}{\isacharcomma}{\kern0pt}{\isadigit{3}}{\isadigit{9}}{\isacharcomma}{\kern0pt}{\isadigit{8}}{\isacharcomma}{\kern0pt}{\isadigit{1}}{\isadigit{9}}{\isacharcomma}{\kern0pt}{\isadigit{3}}{\isadigit{0}}{\isacharbrackright}{\kern0pt}{\isacharbrackright}{\kern0pt}{\isacharparenright}{\kern0pt}{\isachardoublequoteclose}\isanewline
\isacommand{lemma}\isamarkupfalse%
\ kp{\isacharunderscore}{\kern0pt}{\isadigit{5}}x{\isadigit{8}}{\isacharunderscore}{\kern0pt}lr{\isacharcolon}{\kern0pt}\ {\isachardoublequoteopen}knights{\isacharunderscore}{\kern0pt}path\ b{\isadigit{5}}x{\isadigit{8}}\ kp{\isadigit{5}}x{\isadigit{8}}lr{\isachardoublequoteclose}\isanewline
%
\isadelimproof
\ \ %
\endisadelimproof
%
\isatagproof
\isacommand{by}\isamarkupfalse%
\ {\isacharparenleft}{\kern0pt}simp\ only{\isacharcolon}{\kern0pt}\ knights{\isacharunderscore}{\kern0pt}path{\isacharunderscore}{\kern0pt}exec{\isacharunderscore}{\kern0pt}simp{\isacharparenright}{\kern0pt}\ eval%
\endisatagproof
{\isafoldproof}%
%
\isadelimproof
\isanewline
%
\endisadelimproof
\isanewline
\isacommand{lemma}\isamarkupfalse%
\ kp{\isacharunderscore}{\kern0pt}{\isadigit{5}}x{\isadigit{8}}{\isacharunderscore}{\kern0pt}lr{\isacharunderscore}{\kern0pt}hd{\isacharcolon}{\kern0pt}\ {\isachardoublequoteopen}hd\ kp{\isadigit{5}}x{\isadigit{8}}lr\ {\isacharequal}{\kern0pt}\ {\isacharparenleft}{\kern0pt}{\isadigit{1}}{\isacharcomma}{\kern0pt}{\isadigit{1}}{\isacharparenright}{\kern0pt}{\isachardoublequoteclose}%
\isadelimproof
\ %
\endisadelimproof
%
\isatagproof
\isacommand{by}\isamarkupfalse%
\ eval%
\endisatagproof
{\isafoldproof}%
%
\isadelimproof
%
\endisadelimproof
\isanewline
\isanewline
\isacommand{lemma}\isamarkupfalse%
\ kp{\isacharunderscore}{\kern0pt}{\isadigit{5}}x{\isadigit{8}}{\isacharunderscore}{\kern0pt}lr{\isacharunderscore}{\kern0pt}last{\isacharcolon}{\kern0pt}\ {\isachardoublequoteopen}last\ kp{\isadigit{5}}x{\isadigit{8}}lr\ {\isacharequal}{\kern0pt}\ {\isacharparenleft}{\kern0pt}{\isadigit{2}}{\isacharcomma}{\kern0pt}{\isadigit{7}}{\isacharparenright}{\kern0pt}{\isachardoublequoteclose}%
\isadelimproof
\ %
\endisadelimproof
%
\isatagproof
\isacommand{by}\isamarkupfalse%
\ eval%
\endisatagproof
{\isafoldproof}%
%
\isadelimproof
%
\endisadelimproof
\isanewline
\isanewline
\isacommand{lemma}\isamarkupfalse%
\ kp{\isacharunderscore}{\kern0pt}{\isadigit{5}}x{\isadigit{8}}{\isacharunderscore}{\kern0pt}lr{\isacharunderscore}{\kern0pt}non{\isacharunderscore}{\kern0pt}nil{\isacharcolon}{\kern0pt}\ {\isachardoublequoteopen}kp{\isadigit{5}}x{\isadigit{8}}lr\ {\isasymnoteq}\ {\isacharbrackleft}{\kern0pt}{\isacharbrackright}{\kern0pt}{\isachardoublequoteclose}%
\isadelimproof
\ %
\endisadelimproof
%
\isatagproof
\isacommand{by}\isamarkupfalse%
\ eval%
\endisatagproof
{\isafoldproof}%
%
\isadelimproof
%
\endisadelimproof
%
\begin{isamarkuptext}%
A Knight's path for the \isa{{\isacharparenleft}{\kern0pt}{\isadigit{5}}{\isasymtimes}{\isadigit{8}}{\isacharparenright}{\kern0pt}}-board that starts in the lower-left and ends in the 
upper-right.
  \begin{table}[H]
    \begin{tabular}{llllllll}
      33 &  8 & 17 & 38 & 35 &  6 & 15 & 24 \\
      18 & 37 & 34 &  7 & 16 & 25 & 40 &  5 \\
       9 & 32 & 29 & 36 & 39 & 14 & 23 & 26 \\
      30 & 19 &  2 & 11 & 28 & 21 &  4 & 13 \\
       1 & 10 & 31 & 20 &  3 & 12 & 27 & 22
    \end{tabular}
  \end{table}%
\end{isamarkuptext}\isamarkuptrue%
\isacommand{abbreviation}\isamarkupfalse%
\ {\isachardoublequoteopen}kp{\isadigit{5}}x{\isadigit{8}}ur\ {\isasymequiv}\ the\ {\isacharparenleft}{\kern0pt}to{\isacharunderscore}{\kern0pt}path\ \isanewline
\ \ {\isacharbrackleft}{\kern0pt}{\isacharbrackleft}{\kern0pt}{\isadigit{3}}{\isadigit{3}}{\isacharcomma}{\kern0pt}{\isadigit{8}}{\isacharcomma}{\kern0pt}{\isadigit{1}}{\isadigit{7}}{\isacharcomma}{\kern0pt}{\isadigit{3}}{\isadigit{8}}{\isacharcomma}{\kern0pt}{\isadigit{3}}{\isadigit{5}}{\isacharcomma}{\kern0pt}{\isadigit{6}}{\isacharcomma}{\kern0pt}{\isadigit{1}}{\isadigit{5}}{\isacharcomma}{\kern0pt}{\isadigit{2}}{\isadigit{4}}{\isacharbrackright}{\kern0pt}{\isacharcomma}{\kern0pt}\isanewline
\ \ {\isacharbrackleft}{\kern0pt}{\isadigit{1}}{\isadigit{8}}{\isacharcomma}{\kern0pt}{\isadigit{3}}{\isadigit{7}}{\isacharcomma}{\kern0pt}{\isadigit{3}}{\isadigit{4}}{\isacharcomma}{\kern0pt}{\isadigit{7}}{\isacharcomma}{\kern0pt}{\isadigit{1}}{\isadigit{6}}{\isacharcomma}{\kern0pt}{\isadigit{2}}{\isadigit{5}}{\isacharcomma}{\kern0pt}{\isadigit{4}}{\isadigit{0}}{\isacharcomma}{\kern0pt}{\isadigit{5}}{\isacharbrackright}{\kern0pt}{\isacharcomma}{\kern0pt}\isanewline
\ \ {\isacharbrackleft}{\kern0pt}{\isadigit{9}}{\isacharcomma}{\kern0pt}{\isadigit{3}}{\isadigit{2}}{\isacharcomma}{\kern0pt}{\isadigit{2}}{\isadigit{9}}{\isacharcomma}{\kern0pt}{\isadigit{3}}{\isadigit{6}}{\isacharcomma}{\kern0pt}{\isadigit{3}}{\isadigit{9}}{\isacharcomma}{\kern0pt}{\isadigit{1}}{\isadigit{4}}{\isacharcomma}{\kern0pt}{\isadigit{2}}{\isadigit{3}}{\isacharcomma}{\kern0pt}{\isadigit{2}}{\isadigit{6}}{\isacharbrackright}{\kern0pt}{\isacharcomma}{\kern0pt}\isanewline
\ \ {\isacharbrackleft}{\kern0pt}{\isadigit{3}}{\isadigit{0}}{\isacharcomma}{\kern0pt}{\isadigit{1}}{\isadigit{9}}{\isacharcomma}{\kern0pt}{\isadigit{2}}{\isacharcomma}{\kern0pt}{\isadigit{1}}{\isadigit{1}}{\isacharcomma}{\kern0pt}{\isadigit{2}}{\isadigit{8}}{\isacharcomma}{\kern0pt}{\isadigit{2}}{\isadigit{1}}{\isacharcomma}{\kern0pt}{\isadigit{4}}{\isacharcomma}{\kern0pt}{\isadigit{1}}{\isadigit{3}}{\isacharbrackright}{\kern0pt}{\isacharcomma}{\kern0pt}\isanewline
\ \ {\isacharbrackleft}{\kern0pt}{\isadigit{1}}{\isacharcomma}{\kern0pt}{\isadigit{1}}{\isadigit{0}}{\isacharcomma}{\kern0pt}{\isadigit{3}}{\isadigit{1}}{\isacharcomma}{\kern0pt}{\isadigit{2}}{\isadigit{0}}{\isacharcomma}{\kern0pt}{\isadigit{3}}{\isacharcomma}{\kern0pt}{\isadigit{1}}{\isadigit{2}}{\isacharcomma}{\kern0pt}{\isadigit{2}}{\isadigit{7}}{\isacharcomma}{\kern0pt}{\isadigit{2}}{\isadigit{2}}{\isacharbrackright}{\kern0pt}{\isacharbrackright}{\kern0pt}{\isacharparenright}{\kern0pt}{\isachardoublequoteclose}\isanewline
\isacommand{lemma}\isamarkupfalse%
\ kp{\isacharunderscore}{\kern0pt}{\isadigit{5}}x{\isadigit{8}}{\isacharunderscore}{\kern0pt}ur{\isacharcolon}{\kern0pt}\ {\isachardoublequoteopen}knights{\isacharunderscore}{\kern0pt}path\ b{\isadigit{5}}x{\isadigit{8}}\ kp{\isadigit{5}}x{\isadigit{8}}ur{\isachardoublequoteclose}\isanewline
%
\isadelimproof
\ \ %
\endisadelimproof
%
\isatagproof
\isacommand{by}\isamarkupfalse%
\ {\isacharparenleft}{\kern0pt}simp\ only{\isacharcolon}{\kern0pt}\ knights{\isacharunderscore}{\kern0pt}path{\isacharunderscore}{\kern0pt}exec{\isacharunderscore}{\kern0pt}simp{\isacharparenright}{\kern0pt}\ eval%
\endisatagproof
{\isafoldproof}%
%
\isadelimproof
\isanewline
%
\endisadelimproof
\isanewline
\isacommand{lemma}\isamarkupfalse%
\ kp{\isacharunderscore}{\kern0pt}{\isadigit{5}}x{\isadigit{8}}{\isacharunderscore}{\kern0pt}ur{\isacharunderscore}{\kern0pt}hd{\isacharcolon}{\kern0pt}\ {\isachardoublequoteopen}hd\ kp{\isadigit{5}}x{\isadigit{8}}ur\ {\isacharequal}{\kern0pt}\ {\isacharparenleft}{\kern0pt}{\isadigit{1}}{\isacharcomma}{\kern0pt}{\isadigit{1}}{\isacharparenright}{\kern0pt}{\isachardoublequoteclose}%
\isadelimproof
\ %
\endisadelimproof
%
\isatagproof
\isacommand{by}\isamarkupfalse%
\ eval%
\endisatagproof
{\isafoldproof}%
%
\isadelimproof
%
\endisadelimproof
\isanewline
\isanewline
\isacommand{lemma}\isamarkupfalse%
\ kp{\isacharunderscore}{\kern0pt}{\isadigit{5}}x{\isadigit{8}}{\isacharunderscore}{\kern0pt}ur{\isacharunderscore}{\kern0pt}last{\isacharcolon}{\kern0pt}\ {\isachardoublequoteopen}last\ kp{\isadigit{5}}x{\isadigit{8}}ur\ {\isacharequal}{\kern0pt}\ {\isacharparenleft}{\kern0pt}{\isadigit{4}}{\isacharcomma}{\kern0pt}{\isadigit{7}}{\isacharparenright}{\kern0pt}{\isachardoublequoteclose}%
\isadelimproof
\ %
\endisadelimproof
%
\isatagproof
\isacommand{by}\isamarkupfalse%
\ eval%
\endisatagproof
{\isafoldproof}%
%
\isadelimproof
%
\endisadelimproof
\isanewline
\isanewline
\isacommand{lemma}\isamarkupfalse%
\ kp{\isacharunderscore}{\kern0pt}{\isadigit{5}}x{\isadigit{8}}{\isacharunderscore}{\kern0pt}ur{\isacharunderscore}{\kern0pt}non{\isacharunderscore}{\kern0pt}nil{\isacharcolon}{\kern0pt}\ {\isachardoublequoteopen}kp{\isadigit{5}}x{\isadigit{8}}ur\ {\isasymnoteq}\ {\isacharbrackleft}{\kern0pt}{\isacharbrackright}{\kern0pt}{\isachardoublequoteclose}%
\isadelimproof
\ %
\endisadelimproof
%
\isatagproof
\isacommand{by}\isamarkupfalse%
\ eval%
\endisatagproof
{\isafoldproof}%
%
\isadelimproof
%
\endisadelimproof
\isanewline
\isanewline
\isacommand{abbreviation}\isamarkupfalse%
\ {\isachardoublequoteopen}b{\isadigit{5}}x{\isadigit{9}}\ {\isasymequiv}\ board\ {\isadigit{5}}\ {\isadigit{9}}{\isachardoublequoteclose}%
\begin{isamarkuptext}%
A Knight's path for the \isa{{\isacharparenleft}{\kern0pt}{\isadigit{5}}{\isasymtimes}{\isadigit{9}}{\isacharparenright}{\kern0pt}}-board that starts in the lower-left and ends in the lower-right.
  \begin{table}[H]
    \begin{tabular}{lllllllll}
       9 &  4 & 11 & 16 & 23 & 42 & 33 & 36 & 25 \\
      12 & 17 &  8 &  3 & 32 & 37 & 24 & 41 & 34 \\
       5 & 10 & 15 & 20 & 43 & 22 & 35 & 26 & 29 \\
      18 & 13 &  2 &  7 & 38 & 31 & 28 & 45 & 40 \\
       1 &  6 & 19 & 14 & 21 & 44 & 39 & 30 & 27
    \end{tabular}
  \end{table}%
\end{isamarkuptext}\isamarkuptrue%
\isacommand{abbreviation}\isamarkupfalse%
\ {\isachardoublequoteopen}kp{\isadigit{5}}x{\isadigit{9}}lr\ {\isasymequiv}\ the\ {\isacharparenleft}{\kern0pt}to{\isacharunderscore}{\kern0pt}path\ \isanewline
\ \ {\isacharbrackleft}{\kern0pt}{\isacharbrackleft}{\kern0pt}{\isadigit{9}}{\isacharcomma}{\kern0pt}{\isadigit{4}}{\isacharcomma}{\kern0pt}{\isadigit{1}}{\isadigit{1}}{\isacharcomma}{\kern0pt}{\isadigit{1}}{\isadigit{6}}{\isacharcomma}{\kern0pt}{\isadigit{2}}{\isadigit{3}}{\isacharcomma}{\kern0pt}{\isadigit{4}}{\isadigit{2}}{\isacharcomma}{\kern0pt}{\isadigit{3}}{\isadigit{3}}{\isacharcomma}{\kern0pt}{\isadigit{3}}{\isadigit{6}}{\isacharcomma}{\kern0pt}{\isadigit{2}}{\isadigit{5}}{\isacharbrackright}{\kern0pt}{\isacharcomma}{\kern0pt}\isanewline
\ \ {\isacharbrackleft}{\kern0pt}{\isadigit{1}}{\isadigit{2}}{\isacharcomma}{\kern0pt}{\isadigit{1}}{\isadigit{7}}{\isacharcomma}{\kern0pt}{\isadigit{8}}{\isacharcomma}{\kern0pt}{\isadigit{3}}{\isacharcomma}{\kern0pt}{\isadigit{3}}{\isadigit{2}}{\isacharcomma}{\kern0pt}{\isadigit{3}}{\isadigit{7}}{\isacharcomma}{\kern0pt}{\isadigit{2}}{\isadigit{4}}{\isacharcomma}{\kern0pt}{\isadigit{4}}{\isadigit{1}}{\isacharcomma}{\kern0pt}{\isadigit{3}}{\isadigit{4}}{\isacharbrackright}{\kern0pt}{\isacharcomma}{\kern0pt}\isanewline
\ \ {\isacharbrackleft}{\kern0pt}{\isadigit{5}}{\isacharcomma}{\kern0pt}{\isadigit{1}}{\isadigit{0}}{\isacharcomma}{\kern0pt}{\isadigit{1}}{\isadigit{5}}{\isacharcomma}{\kern0pt}{\isadigit{2}}{\isadigit{0}}{\isacharcomma}{\kern0pt}{\isadigit{4}}{\isadigit{3}}{\isacharcomma}{\kern0pt}{\isadigit{2}}{\isadigit{2}}{\isacharcomma}{\kern0pt}{\isadigit{3}}{\isadigit{5}}{\isacharcomma}{\kern0pt}{\isadigit{2}}{\isadigit{6}}{\isacharcomma}{\kern0pt}{\isadigit{2}}{\isadigit{9}}{\isacharbrackright}{\kern0pt}{\isacharcomma}{\kern0pt}\isanewline
\ \ {\isacharbrackleft}{\kern0pt}{\isadigit{1}}{\isadigit{8}}{\isacharcomma}{\kern0pt}{\isadigit{1}}{\isadigit{3}}{\isacharcomma}{\kern0pt}{\isadigit{2}}{\isacharcomma}{\kern0pt}{\isadigit{7}}{\isacharcomma}{\kern0pt}{\isadigit{3}}{\isadigit{8}}{\isacharcomma}{\kern0pt}{\isadigit{3}}{\isadigit{1}}{\isacharcomma}{\kern0pt}{\isadigit{2}}{\isadigit{8}}{\isacharcomma}{\kern0pt}{\isadigit{4}}{\isadigit{5}}{\isacharcomma}{\kern0pt}{\isadigit{4}}{\isadigit{0}}{\isacharbrackright}{\kern0pt}{\isacharcomma}{\kern0pt}\isanewline
\ \ {\isacharbrackleft}{\kern0pt}{\isadigit{1}}{\isacharcomma}{\kern0pt}{\isadigit{6}}{\isacharcomma}{\kern0pt}{\isadigit{1}}{\isadigit{9}}{\isacharcomma}{\kern0pt}{\isadigit{1}}{\isadigit{4}}{\isacharcomma}{\kern0pt}{\isadigit{2}}{\isadigit{1}}{\isacharcomma}{\kern0pt}{\isadigit{4}}{\isadigit{4}}{\isacharcomma}{\kern0pt}{\isadigit{3}}{\isadigit{9}}{\isacharcomma}{\kern0pt}{\isadigit{3}}{\isadigit{0}}{\isacharcomma}{\kern0pt}{\isadigit{2}}{\isadigit{7}}{\isacharbrackright}{\kern0pt}{\isacharbrackright}{\kern0pt}{\isacharparenright}{\kern0pt}{\isachardoublequoteclose}\isanewline
\isacommand{lemma}\isamarkupfalse%
\ kp{\isacharunderscore}{\kern0pt}{\isadigit{5}}x{\isadigit{9}}{\isacharunderscore}{\kern0pt}lr{\isacharcolon}{\kern0pt}\ {\isachardoublequoteopen}knights{\isacharunderscore}{\kern0pt}path\ b{\isadigit{5}}x{\isadigit{9}}\ kp{\isadigit{5}}x{\isadigit{9}}lr{\isachardoublequoteclose}\isanewline
%
\isadelimproof
\ \ %
\endisadelimproof
%
\isatagproof
\isacommand{by}\isamarkupfalse%
\ {\isacharparenleft}{\kern0pt}simp\ only{\isacharcolon}{\kern0pt}\ knights{\isacharunderscore}{\kern0pt}path{\isacharunderscore}{\kern0pt}exec{\isacharunderscore}{\kern0pt}simp{\isacharparenright}{\kern0pt}\ eval%
\endisatagproof
{\isafoldproof}%
%
\isadelimproof
\isanewline
%
\endisadelimproof
\isanewline
\isacommand{lemma}\isamarkupfalse%
\ kp{\isacharunderscore}{\kern0pt}{\isadigit{5}}x{\isadigit{9}}{\isacharunderscore}{\kern0pt}lr{\isacharunderscore}{\kern0pt}hd{\isacharcolon}{\kern0pt}\ {\isachardoublequoteopen}hd\ kp{\isadigit{5}}x{\isadigit{9}}lr\ {\isacharequal}{\kern0pt}\ {\isacharparenleft}{\kern0pt}{\isadigit{1}}{\isacharcomma}{\kern0pt}{\isadigit{1}}{\isacharparenright}{\kern0pt}{\isachardoublequoteclose}%
\isadelimproof
\ %
\endisadelimproof
%
\isatagproof
\isacommand{by}\isamarkupfalse%
\ eval%
\endisatagproof
{\isafoldproof}%
%
\isadelimproof
%
\endisadelimproof
\isanewline
\isanewline
\isacommand{lemma}\isamarkupfalse%
\ kp{\isacharunderscore}{\kern0pt}{\isadigit{5}}x{\isadigit{9}}{\isacharunderscore}{\kern0pt}lr{\isacharunderscore}{\kern0pt}last{\isacharcolon}{\kern0pt}\ {\isachardoublequoteopen}last\ kp{\isadigit{5}}x{\isadigit{9}}lr\ {\isacharequal}{\kern0pt}\ {\isacharparenleft}{\kern0pt}{\isadigit{2}}{\isacharcomma}{\kern0pt}{\isadigit{8}}{\isacharparenright}{\kern0pt}{\isachardoublequoteclose}%
\isadelimproof
\ %
\endisadelimproof
%
\isatagproof
\isacommand{by}\isamarkupfalse%
\ eval%
\endisatagproof
{\isafoldproof}%
%
\isadelimproof
%
\endisadelimproof
\isanewline
\isanewline
\isacommand{lemma}\isamarkupfalse%
\ kp{\isacharunderscore}{\kern0pt}{\isadigit{5}}x{\isadigit{9}}{\isacharunderscore}{\kern0pt}lr{\isacharunderscore}{\kern0pt}non{\isacharunderscore}{\kern0pt}nil{\isacharcolon}{\kern0pt}\ {\isachardoublequoteopen}kp{\isadigit{5}}x{\isadigit{9}}lr\ {\isasymnoteq}\ {\isacharbrackleft}{\kern0pt}{\isacharbrackright}{\kern0pt}{\isachardoublequoteclose}%
\isadelimproof
\ %
\endisadelimproof
%
\isatagproof
\isacommand{by}\isamarkupfalse%
\ eval%
\endisatagproof
{\isafoldproof}%
%
\isadelimproof
%
\endisadelimproof
%
\begin{isamarkuptext}%
A Knight's path for the \isa{{\isacharparenleft}{\kern0pt}{\isadigit{5}}{\isasymtimes}{\isadigit{9}}{\isacharparenright}{\kern0pt}}-board that starts in the lower-left and ends in the upper-right.
  \begin{table}[H]
    \begin{tabular}{lllllllll}
       9 &  4 & 11 & 16 & 27 & 32 & 35 & 40 & 25 \\
      12 & 17 &  8 &  3 & 36 & 41 & 26 & 45 & 34 \\
       5 & 10 & 15 & 20 & 31 & 28 & 33 & 24 & 39 \\
      18 & 13 &  2 &  7 & 42 & 37 & 22 & 29 & 44 \\
       1 &  6 & 19 & 14 & 21 & 30 & 43 & 38 & 23
    \end{tabular}
  \end{table}%
\end{isamarkuptext}\isamarkuptrue%
\isacommand{abbreviation}\isamarkupfalse%
\ {\isachardoublequoteopen}kp{\isadigit{5}}x{\isadigit{9}}ur\ {\isasymequiv}\ the\ {\isacharparenleft}{\kern0pt}to{\isacharunderscore}{\kern0pt}path\ \isanewline
\ \ {\isacharbrackleft}{\kern0pt}{\isacharbrackleft}{\kern0pt}{\isadigit{9}}{\isacharcomma}{\kern0pt}{\isadigit{4}}{\isacharcomma}{\kern0pt}{\isadigit{1}}{\isadigit{1}}{\isacharcomma}{\kern0pt}{\isadigit{1}}{\isadigit{6}}{\isacharcomma}{\kern0pt}{\isadigit{2}}{\isadigit{7}}{\isacharcomma}{\kern0pt}{\isadigit{3}}{\isadigit{2}}{\isacharcomma}{\kern0pt}{\isadigit{3}}{\isadigit{5}}{\isacharcomma}{\kern0pt}{\isadigit{4}}{\isadigit{0}}{\isacharcomma}{\kern0pt}{\isadigit{2}}{\isadigit{5}}{\isacharbrackright}{\kern0pt}{\isacharcomma}{\kern0pt}\isanewline
\ \ {\isacharbrackleft}{\kern0pt}{\isadigit{1}}{\isadigit{2}}{\isacharcomma}{\kern0pt}{\isadigit{1}}{\isadigit{7}}{\isacharcomma}{\kern0pt}{\isadigit{8}}{\isacharcomma}{\kern0pt}{\isadigit{3}}{\isacharcomma}{\kern0pt}{\isadigit{3}}{\isadigit{6}}{\isacharcomma}{\kern0pt}{\isadigit{4}}{\isadigit{1}}{\isacharcomma}{\kern0pt}{\isadigit{2}}{\isadigit{6}}{\isacharcomma}{\kern0pt}{\isadigit{4}}{\isadigit{5}}{\isacharcomma}{\kern0pt}{\isadigit{3}}{\isadigit{4}}{\isacharbrackright}{\kern0pt}{\isacharcomma}{\kern0pt}\isanewline
\ \ {\isacharbrackleft}{\kern0pt}{\isadigit{5}}{\isacharcomma}{\kern0pt}{\isadigit{1}}{\isadigit{0}}{\isacharcomma}{\kern0pt}{\isadigit{1}}{\isadigit{5}}{\isacharcomma}{\kern0pt}{\isadigit{2}}{\isadigit{0}}{\isacharcomma}{\kern0pt}{\isadigit{3}}{\isadigit{1}}{\isacharcomma}{\kern0pt}{\isadigit{2}}{\isadigit{8}}{\isacharcomma}{\kern0pt}{\isadigit{3}}{\isadigit{3}}{\isacharcomma}{\kern0pt}{\isadigit{2}}{\isadigit{4}}{\isacharcomma}{\kern0pt}{\isadigit{3}}{\isadigit{9}}{\isacharbrackright}{\kern0pt}{\isacharcomma}{\kern0pt}\isanewline
\ \ {\isacharbrackleft}{\kern0pt}{\isadigit{1}}{\isadigit{8}}{\isacharcomma}{\kern0pt}{\isadigit{1}}{\isadigit{3}}{\isacharcomma}{\kern0pt}{\isadigit{2}}{\isacharcomma}{\kern0pt}{\isadigit{7}}{\isacharcomma}{\kern0pt}{\isadigit{4}}{\isadigit{2}}{\isacharcomma}{\kern0pt}{\isadigit{3}}{\isadigit{7}}{\isacharcomma}{\kern0pt}{\isadigit{2}}{\isadigit{2}}{\isacharcomma}{\kern0pt}{\isadigit{2}}{\isadigit{9}}{\isacharcomma}{\kern0pt}{\isadigit{4}}{\isadigit{4}}{\isacharbrackright}{\kern0pt}{\isacharcomma}{\kern0pt}\isanewline
\ \ {\isacharbrackleft}{\kern0pt}{\isadigit{1}}{\isacharcomma}{\kern0pt}{\isadigit{6}}{\isacharcomma}{\kern0pt}{\isadigit{1}}{\isadigit{9}}{\isacharcomma}{\kern0pt}{\isadigit{1}}{\isadigit{4}}{\isacharcomma}{\kern0pt}{\isadigit{2}}{\isadigit{1}}{\isacharcomma}{\kern0pt}{\isadigit{3}}{\isadigit{0}}{\isacharcomma}{\kern0pt}{\isadigit{4}}{\isadigit{3}}{\isacharcomma}{\kern0pt}{\isadigit{3}}{\isadigit{8}}{\isacharcomma}{\kern0pt}{\isadigit{2}}{\isadigit{3}}{\isacharbrackright}{\kern0pt}{\isacharbrackright}{\kern0pt}{\isacharparenright}{\kern0pt}{\isachardoublequoteclose}\isanewline
\isacommand{lemma}\isamarkupfalse%
\ kp{\isacharunderscore}{\kern0pt}{\isadigit{5}}x{\isadigit{9}}{\isacharunderscore}{\kern0pt}ur{\isacharcolon}{\kern0pt}\ {\isachardoublequoteopen}knights{\isacharunderscore}{\kern0pt}path\ b{\isadigit{5}}x{\isadigit{9}}\ kp{\isadigit{5}}x{\isadigit{9}}ur{\isachardoublequoteclose}\isanewline
%
\isadelimproof
\ \ %
\endisadelimproof
%
\isatagproof
\isacommand{by}\isamarkupfalse%
\ {\isacharparenleft}{\kern0pt}simp\ only{\isacharcolon}{\kern0pt}\ knights{\isacharunderscore}{\kern0pt}path{\isacharunderscore}{\kern0pt}exec{\isacharunderscore}{\kern0pt}simp{\isacharparenright}{\kern0pt}\ eval%
\endisatagproof
{\isafoldproof}%
%
\isadelimproof
\isanewline
%
\endisadelimproof
\isanewline
\isacommand{lemma}\isamarkupfalse%
\ kp{\isacharunderscore}{\kern0pt}{\isadigit{5}}x{\isadigit{9}}{\isacharunderscore}{\kern0pt}ur{\isacharunderscore}{\kern0pt}hd{\isacharcolon}{\kern0pt}\ {\isachardoublequoteopen}hd\ kp{\isadigit{5}}x{\isadigit{9}}ur\ {\isacharequal}{\kern0pt}\ {\isacharparenleft}{\kern0pt}{\isadigit{1}}{\isacharcomma}{\kern0pt}{\isadigit{1}}{\isacharparenright}{\kern0pt}{\isachardoublequoteclose}%
\isadelimproof
\ %
\endisadelimproof
%
\isatagproof
\isacommand{by}\isamarkupfalse%
\ eval%
\endisatagproof
{\isafoldproof}%
%
\isadelimproof
%
\endisadelimproof
\isanewline
\isanewline
\isacommand{lemma}\isamarkupfalse%
\ kp{\isacharunderscore}{\kern0pt}{\isadigit{5}}x{\isadigit{9}}{\isacharunderscore}{\kern0pt}ur{\isacharunderscore}{\kern0pt}last{\isacharcolon}{\kern0pt}\ {\isachardoublequoteopen}last\ kp{\isadigit{5}}x{\isadigit{9}}ur\ {\isacharequal}{\kern0pt}\ {\isacharparenleft}{\kern0pt}{\isadigit{4}}{\isacharcomma}{\kern0pt}{\isadigit{8}}{\isacharparenright}{\kern0pt}{\isachardoublequoteclose}%
\isadelimproof
\ %
\endisadelimproof
%
\isatagproof
\isacommand{by}\isamarkupfalse%
\ eval%
\endisatagproof
{\isafoldproof}%
%
\isadelimproof
%
\endisadelimproof
\isanewline
\isanewline
\isacommand{lemma}\isamarkupfalse%
\ kp{\isacharunderscore}{\kern0pt}{\isadigit{5}}x{\isadigit{9}}{\isacharunderscore}{\kern0pt}ur{\isacharunderscore}{\kern0pt}non{\isacharunderscore}{\kern0pt}nil{\isacharcolon}{\kern0pt}\ {\isachardoublequoteopen}kp{\isadigit{5}}x{\isadigit{9}}ur\ {\isasymnoteq}\ {\isacharbrackleft}{\kern0pt}{\isacharbrackright}{\kern0pt}{\isachardoublequoteclose}%
\isadelimproof
\ %
\endisadelimproof
%
\isatagproof
\isacommand{by}\isamarkupfalse%
\ eval%
\endisatagproof
{\isafoldproof}%
%
\isadelimproof
%
\endisadelimproof
\isanewline
\isanewline
\isacommand{lemmas}\isamarkupfalse%
\ kp{\isacharunderscore}{\kern0pt}{\isadigit{5}}xm{\isacharunderscore}{\kern0pt}lr\ {\isacharequal}{\kern0pt}\ \isanewline
\ \ kp{\isacharunderscore}{\kern0pt}{\isadigit{5}}x{\isadigit{5}}{\isacharunderscore}{\kern0pt}lr\ kp{\isacharunderscore}{\kern0pt}{\isadigit{5}}x{\isadigit{5}}{\isacharunderscore}{\kern0pt}lr{\isacharunderscore}{\kern0pt}hd\ kp{\isacharunderscore}{\kern0pt}{\isadigit{5}}x{\isadigit{5}}{\isacharunderscore}{\kern0pt}lr{\isacharunderscore}{\kern0pt}last\ kp{\isacharunderscore}{\kern0pt}{\isadigit{5}}x{\isadigit{5}}{\isacharunderscore}{\kern0pt}lr{\isacharunderscore}{\kern0pt}non{\isacharunderscore}{\kern0pt}nil\isanewline
\ \ kp{\isacharunderscore}{\kern0pt}{\isadigit{5}}x{\isadigit{6}}{\isacharunderscore}{\kern0pt}lr\ kp{\isacharunderscore}{\kern0pt}{\isadigit{5}}x{\isadigit{6}}{\isacharunderscore}{\kern0pt}lr{\isacharunderscore}{\kern0pt}hd\ kp{\isacharunderscore}{\kern0pt}{\isadigit{5}}x{\isadigit{6}}{\isacharunderscore}{\kern0pt}lr{\isacharunderscore}{\kern0pt}last\ kp{\isacharunderscore}{\kern0pt}{\isadigit{5}}x{\isadigit{6}}{\isacharunderscore}{\kern0pt}lr{\isacharunderscore}{\kern0pt}non{\isacharunderscore}{\kern0pt}nil\isanewline
\ \ kp{\isacharunderscore}{\kern0pt}{\isadigit{5}}x{\isadigit{7}}{\isacharunderscore}{\kern0pt}lr\ kp{\isacharunderscore}{\kern0pt}{\isadigit{5}}x{\isadigit{7}}{\isacharunderscore}{\kern0pt}lr{\isacharunderscore}{\kern0pt}hd\ kp{\isacharunderscore}{\kern0pt}{\isadigit{5}}x{\isadigit{7}}{\isacharunderscore}{\kern0pt}lr{\isacharunderscore}{\kern0pt}last\ kp{\isacharunderscore}{\kern0pt}{\isadigit{5}}x{\isadigit{7}}{\isacharunderscore}{\kern0pt}lr{\isacharunderscore}{\kern0pt}non{\isacharunderscore}{\kern0pt}nil\isanewline
\ \ kp{\isacharunderscore}{\kern0pt}{\isadigit{5}}x{\isadigit{8}}{\isacharunderscore}{\kern0pt}lr\ kp{\isacharunderscore}{\kern0pt}{\isadigit{5}}x{\isadigit{8}}{\isacharunderscore}{\kern0pt}lr{\isacharunderscore}{\kern0pt}hd\ kp{\isacharunderscore}{\kern0pt}{\isadigit{5}}x{\isadigit{8}}{\isacharunderscore}{\kern0pt}lr{\isacharunderscore}{\kern0pt}last\ kp{\isacharunderscore}{\kern0pt}{\isadigit{5}}x{\isadigit{8}}{\isacharunderscore}{\kern0pt}lr{\isacharunderscore}{\kern0pt}non{\isacharunderscore}{\kern0pt}nil\isanewline
\ \ kp{\isacharunderscore}{\kern0pt}{\isadigit{5}}x{\isadigit{9}}{\isacharunderscore}{\kern0pt}lr\ kp{\isacharunderscore}{\kern0pt}{\isadigit{5}}x{\isadigit{9}}{\isacharunderscore}{\kern0pt}lr{\isacharunderscore}{\kern0pt}hd\ kp{\isacharunderscore}{\kern0pt}{\isadigit{5}}x{\isadigit{9}}{\isacharunderscore}{\kern0pt}lr{\isacharunderscore}{\kern0pt}last\ kp{\isacharunderscore}{\kern0pt}{\isadigit{5}}x{\isadigit{9}}{\isacharunderscore}{\kern0pt}lr{\isacharunderscore}{\kern0pt}non{\isacharunderscore}{\kern0pt}nil\isanewline
\isanewline
\isacommand{lemmas}\isamarkupfalse%
\ kp{\isacharunderscore}{\kern0pt}{\isadigit{5}}xm{\isacharunderscore}{\kern0pt}ur\ {\isacharequal}{\kern0pt}\ \isanewline
\ \ kp{\isacharunderscore}{\kern0pt}{\isadigit{5}}x{\isadigit{5}}{\isacharunderscore}{\kern0pt}ur\ kp{\isacharunderscore}{\kern0pt}{\isadigit{5}}x{\isadigit{5}}{\isacharunderscore}{\kern0pt}ur{\isacharunderscore}{\kern0pt}hd\ kp{\isacharunderscore}{\kern0pt}{\isadigit{5}}x{\isadigit{5}}{\isacharunderscore}{\kern0pt}ur{\isacharunderscore}{\kern0pt}last\ kp{\isacharunderscore}{\kern0pt}{\isadigit{5}}x{\isadigit{5}}{\isacharunderscore}{\kern0pt}ur{\isacharunderscore}{\kern0pt}non{\isacharunderscore}{\kern0pt}nil\isanewline
\ \ kp{\isacharunderscore}{\kern0pt}{\isadigit{5}}x{\isadigit{6}}{\isacharunderscore}{\kern0pt}ur\ kp{\isacharunderscore}{\kern0pt}{\isadigit{5}}x{\isadigit{6}}{\isacharunderscore}{\kern0pt}ur{\isacharunderscore}{\kern0pt}hd\ kp{\isacharunderscore}{\kern0pt}{\isadigit{5}}x{\isadigit{6}}{\isacharunderscore}{\kern0pt}ur{\isacharunderscore}{\kern0pt}last\ kp{\isacharunderscore}{\kern0pt}{\isadigit{5}}x{\isadigit{6}}{\isacharunderscore}{\kern0pt}ur{\isacharunderscore}{\kern0pt}non{\isacharunderscore}{\kern0pt}nil\isanewline
\ \ kp{\isacharunderscore}{\kern0pt}{\isadigit{5}}x{\isadigit{7}}{\isacharunderscore}{\kern0pt}ur\ kp{\isacharunderscore}{\kern0pt}{\isadigit{5}}x{\isadigit{7}}{\isacharunderscore}{\kern0pt}ur{\isacharunderscore}{\kern0pt}hd\ kp{\isacharunderscore}{\kern0pt}{\isadigit{5}}x{\isadigit{7}}{\isacharunderscore}{\kern0pt}ur{\isacharunderscore}{\kern0pt}last\ kp{\isacharunderscore}{\kern0pt}{\isadigit{5}}x{\isadigit{7}}{\isacharunderscore}{\kern0pt}ur{\isacharunderscore}{\kern0pt}non{\isacharunderscore}{\kern0pt}nil\isanewline
\ \ kp{\isacharunderscore}{\kern0pt}{\isadigit{5}}x{\isadigit{8}}{\isacharunderscore}{\kern0pt}ur\ kp{\isacharunderscore}{\kern0pt}{\isadigit{5}}x{\isadigit{8}}{\isacharunderscore}{\kern0pt}ur{\isacharunderscore}{\kern0pt}hd\ kp{\isacharunderscore}{\kern0pt}{\isadigit{5}}x{\isadigit{8}}{\isacharunderscore}{\kern0pt}ur{\isacharunderscore}{\kern0pt}last\ kp{\isacharunderscore}{\kern0pt}{\isadigit{5}}x{\isadigit{8}}{\isacharunderscore}{\kern0pt}ur{\isacharunderscore}{\kern0pt}non{\isacharunderscore}{\kern0pt}nil\isanewline
\ \ kp{\isacharunderscore}{\kern0pt}{\isadigit{5}}x{\isadigit{9}}{\isacharunderscore}{\kern0pt}ur\ kp{\isacharunderscore}{\kern0pt}{\isadigit{5}}x{\isadigit{9}}{\isacharunderscore}{\kern0pt}ur{\isacharunderscore}{\kern0pt}hd\ kp{\isacharunderscore}{\kern0pt}{\isadigit{5}}x{\isadigit{9}}{\isacharunderscore}{\kern0pt}ur{\isacharunderscore}{\kern0pt}last\ kp{\isacharunderscore}{\kern0pt}{\isadigit{5}}x{\isadigit{9}}{\isacharunderscore}{\kern0pt}ur{\isacharunderscore}{\kern0pt}non{\isacharunderscore}{\kern0pt}nil%
\begin{isamarkuptext}%
For every \isa{{\isadigit{5}}{\isasymtimes}m}-board with \isa{m\ {\isasymge}\ {\isadigit{5}}} there exists a knight's path that starts in 
\isa{{\isacharparenleft}{\kern0pt}{\isadigit{1}}{\isacharcomma}{\kern0pt}{\isadigit{1}}{\isacharparenright}{\kern0pt}} (bottom-left) and ends in \isa{{\isacharparenleft}{\kern0pt}{\isadigit{2}}{\isacharcomma}{\kern0pt}m{\isacharminus}{\kern0pt}{\isadigit{1}}{\isacharparenright}{\kern0pt}} (bottom-right).%
\end{isamarkuptext}\isamarkuptrue%
\isacommand{lemma}\isamarkupfalse%
\ knights{\isacharunderscore}{\kern0pt}path{\isacharunderscore}{\kern0pt}{\isadigit{5}}xm{\isacharunderscore}{\kern0pt}lr{\isacharunderscore}{\kern0pt}exists{\isacharcolon}{\kern0pt}\ \isanewline
\ \ \isakeyword{assumes}\ {\isachardoublequoteopen}m\ {\isasymge}\ {\isadigit{5}}{\isachardoublequoteclose}\ \isanewline
\ \ \isakeyword{shows}\ {\isachardoublequoteopen}{\isasymexists}ps{\isachardot}{\kern0pt}\ knights{\isacharunderscore}{\kern0pt}path\ {\isacharparenleft}{\kern0pt}board\ {\isadigit{5}}\ m{\isacharparenright}{\kern0pt}\ ps\ {\isasymand}\ hd\ ps\ {\isacharequal}{\kern0pt}\ {\isacharparenleft}{\kern0pt}{\isadigit{1}}{\isacharcomma}{\kern0pt}{\isadigit{1}}{\isacharparenright}{\kern0pt}\ {\isasymand}\ last\ ps\ {\isacharequal}{\kern0pt}\ {\isacharparenleft}{\kern0pt}{\isadigit{2}}{\isacharcomma}{\kern0pt}int\ m{\isacharminus}{\kern0pt}{\isadigit{1}}{\isacharparenright}{\kern0pt}{\isachardoublequoteclose}\isanewline
%
\isadelimproof
\ \ %
\endisadelimproof
%
\isatagproof
\isacommand{using}\isamarkupfalse%
\ assms\isanewline
\isacommand{proof}\isamarkupfalse%
\ {\isacharparenleft}{\kern0pt}induction\ m\ rule{\isacharcolon}{\kern0pt}\ less{\isacharunderscore}{\kern0pt}induct{\isacharparenright}{\kern0pt}\isanewline
\ \ \isacommand{case}\isamarkupfalse%
\ {\isacharparenleft}{\kern0pt}less\ m{\isacharparenright}{\kern0pt}\isanewline
\ \ \isacommand{then}\isamarkupfalse%
\ \isacommand{have}\isamarkupfalse%
\ {\isachardoublequoteopen}m\ {\isasymin}\ {\isacharbraceleft}{\kern0pt}{\isadigit{5}}{\isacharcomma}{\kern0pt}{\isadigit{6}}{\isacharcomma}{\kern0pt}{\isadigit{7}}{\isacharcomma}{\kern0pt}{\isadigit{8}}{\isacharcomma}{\kern0pt}{\isadigit{9}}{\isacharbraceright}{\kern0pt}\ {\isasymor}\ {\isadigit{5}}\ {\isasymle}\ m{\isacharminus}{\kern0pt}{\isadigit{5}}{\isachardoublequoteclose}\ \isacommand{by}\isamarkupfalse%
\ auto\isanewline
\ \ \isacommand{then}\isamarkupfalse%
\ \isacommand{show}\isamarkupfalse%
\ {\isacharquery}{\kern0pt}case\isanewline
\ \ \isacommand{proof}\isamarkupfalse%
\ {\isacharparenleft}{\kern0pt}elim\ disjE{\isacharparenright}{\kern0pt}\isanewline
\ \ \ \ \isacommand{assume}\isamarkupfalse%
\ {\isachardoublequoteopen}m\ {\isasymin}\ {\isacharbraceleft}{\kern0pt}{\isadigit{5}}{\isacharcomma}{\kern0pt}{\isadigit{6}}{\isacharcomma}{\kern0pt}{\isadigit{7}}{\isacharcomma}{\kern0pt}{\isadigit{8}}{\isacharcomma}{\kern0pt}{\isadigit{9}}{\isacharbraceright}{\kern0pt}{\isachardoublequoteclose}\isanewline
\ \ \ \ \isacommand{then}\isamarkupfalse%
\ \isacommand{show}\isamarkupfalse%
\ {\isacharquery}{\kern0pt}thesis\ \isacommand{using}\isamarkupfalse%
\ kp{\isacharunderscore}{\kern0pt}{\isadigit{5}}xm{\isacharunderscore}{\kern0pt}lr\ \isacommand{by}\isamarkupfalse%
\ fastforce\isanewline
\ \ \isacommand{next}\isamarkupfalse%
\isanewline
\ \ \ \ \isacommand{assume}\isamarkupfalse%
\ m{\isacharunderscore}{\kern0pt}ge{\isacharcolon}{\kern0pt}\ {\isachardoublequoteopen}{\isadigit{5}}\ {\isasymle}\ m{\isacharminus}{\kern0pt}{\isadigit{5}}{\isachardoublequoteclose}\ \isanewline
\ \ \ \ \isacommand{then}\isamarkupfalse%
\ \isacommand{obtain}\isamarkupfalse%
\ ps\isactrlsub {\isadigit{1}}\ \isakeyword{where}\ ps\isactrlsub {\isadigit{1}}{\isacharunderscore}{\kern0pt}IH{\isacharcolon}{\kern0pt}\ {\isachardoublequoteopen}knights{\isacharunderscore}{\kern0pt}path\ {\isacharparenleft}{\kern0pt}board\ {\isadigit{5}}\ {\isacharparenleft}{\kern0pt}m{\isacharminus}{\kern0pt}{\isadigit{5}}{\isacharparenright}{\kern0pt}{\isacharparenright}{\kern0pt}\ ps\isactrlsub {\isadigit{1}}{\isachardoublequoteclose}\ {\isachardoublequoteopen}hd\ ps\isactrlsub {\isadigit{1}}\ {\isacharequal}{\kern0pt}\ {\isacharparenleft}{\kern0pt}{\isadigit{1}}{\isacharcomma}{\kern0pt}{\isadigit{1}}{\isacharparenright}{\kern0pt}{\isachardoublequoteclose}\ \isanewline
\ \ \ \ \ \ \ \ \ \ \ \ \ \ \ \ \ \ \ \ \ \ \ \ \ \ \ \ \ \ \ \ {\isachardoublequoteopen}last\ ps\isactrlsub {\isadigit{1}}\ {\isacharequal}{\kern0pt}\ {\isacharparenleft}{\kern0pt}{\isadigit{2}}{\isacharcomma}{\kern0pt}int\ {\isacharparenleft}{\kern0pt}m{\isacharminus}{\kern0pt}{\isadigit{5}}{\isacharparenright}{\kern0pt}{\isacharminus}{\kern0pt}{\isadigit{1}}{\isacharparenright}{\kern0pt}{\isachardoublequoteclose}\ {\isachardoublequoteopen}ps\isactrlsub {\isadigit{1}}\ {\isasymnoteq}\ {\isacharbrackleft}{\kern0pt}{\isacharbrackright}{\kern0pt}{\isachardoublequoteclose}\isanewline
\ \ \ \ \ \ \isacommand{using}\isamarkupfalse%
\ less{\isachardot}{\kern0pt}IH{\isacharbrackleft}{\kern0pt}of\ {\isachardoublequoteopen}m{\isacharminus}{\kern0pt}{\isadigit{5}}{\isachardoublequoteclose}{\isacharbrackright}{\kern0pt}\ knights{\isacharunderscore}{\kern0pt}path{\isacharunderscore}{\kern0pt}non{\isacharunderscore}{\kern0pt}nil\ \isacommand{by}\isamarkupfalse%
\ auto\isanewline
\isanewline
\ \ \ \ \isacommand{let}\isamarkupfalse%
\ {\isacharquery}{\kern0pt}ps\isactrlsub {\isadigit{2}}{\isacharequal}{\kern0pt}{\isachardoublequoteopen}kp{\isadigit{5}}x{\isadigit{5}}lr{\isachardoublequoteclose}\isanewline
\ \ \ \ \isacommand{let}\isamarkupfalse%
\ {\isacharquery}{\kern0pt}ps\isactrlsub {\isadigit{2}}{\isacharprime}{\kern0pt}{\isacharequal}{\kern0pt}{\isachardoublequoteopen}ps\isactrlsub {\isadigit{1}}\ {\isacharat}{\kern0pt}\ trans{\isacharunderscore}{\kern0pt}path\ {\isacharparenleft}{\kern0pt}{\isadigit{0}}{\isacharcomma}{\kern0pt}int\ {\isacharparenleft}{\kern0pt}m{\isacharminus}{\kern0pt}{\isadigit{5}}{\isacharparenright}{\kern0pt}{\isacharparenright}{\kern0pt}\ {\isacharquery}{\kern0pt}ps\isactrlsub {\isadigit{2}}{\isachardoublequoteclose}\isanewline
\ \ \ \ \isacommand{have}\isamarkupfalse%
\ {\isachardoublequoteopen}knights{\isacharunderscore}{\kern0pt}path\ b{\isadigit{5}}x{\isadigit{5}}\ {\isacharquery}{\kern0pt}ps\isactrlsub {\isadigit{2}}{\isachardoublequoteclose}\ {\isachardoublequoteopen}hd\ {\isacharquery}{\kern0pt}ps\isactrlsub {\isadigit{2}}\ {\isacharequal}{\kern0pt}\ {\isacharparenleft}{\kern0pt}{\isadigit{1}}{\isacharcomma}{\kern0pt}\ {\isadigit{1}}{\isacharparenright}{\kern0pt}{\isachardoublequoteclose}\ {\isachardoublequoteopen}{\isacharquery}{\kern0pt}ps\isactrlsub {\isadigit{2}}\ {\isasymnoteq}\ {\isacharbrackleft}{\kern0pt}{\isacharbrackright}{\kern0pt}{\isachardoublequoteclose}\ {\isachardoublequoteopen}last\ {\isacharquery}{\kern0pt}ps\isactrlsub {\isadigit{2}}\ {\isacharequal}{\kern0pt}\ {\isacharparenleft}{\kern0pt}{\isadigit{2}}{\isacharcomma}{\kern0pt}{\isadigit{4}}{\isacharparenright}{\kern0pt}{\isachardoublequoteclose}\isanewline
\ \ \ \ \ \ \isacommand{using}\isamarkupfalse%
\ kp{\isacharunderscore}{\kern0pt}{\isadigit{5}}xm{\isacharunderscore}{\kern0pt}lr\ \isacommand{by}\isamarkupfalse%
\ auto\isanewline
\ \ \ \ \isacommand{then}\isamarkupfalse%
\ \isacommand{have}\isamarkupfalse%
\ {\isadigit{1}}{\isacharcolon}{\kern0pt}\ {\isachardoublequoteopen}knights{\isacharunderscore}{\kern0pt}path\ {\isacharparenleft}{\kern0pt}board\ {\isadigit{5}}\ m{\isacharparenright}{\kern0pt}\ {\isacharquery}{\kern0pt}ps\isactrlsub {\isadigit{2}}{\isacharprime}{\kern0pt}{\isachardoublequoteclose}\ \ \ \ \ \ \ \ \ \isanewline
\ \ \ \ \ \ \isacommand{using}\isamarkupfalse%
\ m{\isacharunderscore}{\kern0pt}ge\ ps\isactrlsub {\isadigit{1}}{\isacharunderscore}{\kern0pt}IH\ knights{\isacharunderscore}{\kern0pt}path{\isacharunderscore}{\kern0pt}lr{\isacharunderscore}{\kern0pt}concat{\isacharbrackleft}{\kern0pt}of\ {\isadigit{5}}\ {\isachardoublequoteopen}m{\isacharminus}{\kern0pt}{\isadigit{5}}{\isachardoublequoteclose}\ ps\isactrlsub {\isadigit{1}}\ {\isadigit{5}}\ {\isacharquery}{\kern0pt}ps\isactrlsub {\isadigit{2}}{\isacharbrackright}{\kern0pt}\ \isacommand{by}\isamarkupfalse%
\ auto\isanewline
\isanewline
\ \ \ \ \isacommand{have}\isamarkupfalse%
\ {\isadigit{2}}{\isacharcolon}{\kern0pt}\ {\isachardoublequoteopen}hd\ {\isacharquery}{\kern0pt}ps\isactrlsub {\isadigit{2}}{\isacharprime}{\kern0pt}\ {\isacharequal}{\kern0pt}\ {\isacharparenleft}{\kern0pt}{\isadigit{1}}{\isacharcomma}{\kern0pt}{\isadigit{1}}{\isacharparenright}{\kern0pt}{\isachardoublequoteclose}\ \isacommand{using}\isamarkupfalse%
\ ps\isactrlsub {\isadigit{1}}{\isacharunderscore}{\kern0pt}IH\ \isacommand{by}\isamarkupfalse%
\ auto\isanewline
\isanewline
\ \ \ \ \isacommand{have}\isamarkupfalse%
\ {\isachardoublequoteopen}last\ {\isacharparenleft}{\kern0pt}trans{\isacharunderscore}{\kern0pt}path\ {\isacharparenleft}{\kern0pt}{\isadigit{0}}{\isacharcomma}{\kern0pt}int\ {\isacharparenleft}{\kern0pt}m{\isacharminus}{\kern0pt}{\isadigit{5}}{\isacharparenright}{\kern0pt}{\isacharparenright}{\kern0pt}\ {\isacharquery}{\kern0pt}ps\isactrlsub {\isadigit{2}}{\isacharparenright}{\kern0pt}\ {\isacharequal}{\kern0pt}\ {\isacharparenleft}{\kern0pt}{\isadigit{2}}{\isacharcomma}{\kern0pt}int\ m{\isacharminus}{\kern0pt}{\isadigit{1}}{\isacharparenright}{\kern0pt}{\isachardoublequoteclose}\isanewline
\ \ \ \ \ \ \isacommand{using}\isamarkupfalse%
\ m{\isacharunderscore}{\kern0pt}ge\ last{\isacharunderscore}{\kern0pt}trans{\isacharunderscore}{\kern0pt}path{\isacharbrackleft}{\kern0pt}OF\ {\isacartoucheopen}{\isacharquery}{\kern0pt}ps\isactrlsub {\isadigit{2}}\ {\isasymnoteq}\ {\isacharbrackleft}{\kern0pt}{\isacharbrackright}{\kern0pt}{\isacartoucheclose}\ {\isacartoucheopen}last\ {\isacharquery}{\kern0pt}ps\isactrlsub {\isadigit{2}}\ {\isacharequal}{\kern0pt}\ {\isacharparenleft}{\kern0pt}{\isadigit{2}}{\isacharcomma}{\kern0pt}{\isadigit{4}}{\isacharparenright}{\kern0pt}{\isacartoucheclose}{\isacharbrackright}{\kern0pt}\ \isacommand{by}\isamarkupfalse%
\ auto\isanewline
\ \ \ \ \isacommand{then}\isamarkupfalse%
\ \isacommand{have}\isamarkupfalse%
\ {\isadigit{3}}{\isacharcolon}{\kern0pt}\ {\isachardoublequoteopen}last\ {\isacharquery}{\kern0pt}ps\isactrlsub {\isadigit{2}}{\isacharprime}{\kern0pt}\ {\isacharequal}{\kern0pt}\ {\isacharparenleft}{\kern0pt}{\isadigit{2}}{\isacharcomma}{\kern0pt}int\ m{\isacharminus}{\kern0pt}{\isadigit{1}}{\isacharparenright}{\kern0pt}{\isachardoublequoteclose}\isanewline
\ \ \ \ \ \ \isacommand{using}\isamarkupfalse%
\ last{\isacharunderscore}{\kern0pt}appendR{\isacharbrackleft}{\kern0pt}OF\ trans{\isacharunderscore}{\kern0pt}path{\isacharunderscore}{\kern0pt}non{\isacharunderscore}{\kern0pt}nil{\isacharbrackleft}{\kern0pt}OF\ {\isacartoucheopen}{\isacharquery}{\kern0pt}ps\isactrlsub {\isadigit{2}}\ {\isasymnoteq}\ {\isacharbrackleft}{\kern0pt}{\isacharbrackright}{\kern0pt}{\isacartoucheclose}{\isacharbrackright}{\kern0pt}{\isacharcomma}{\kern0pt}symmetric{\isacharbrackright}{\kern0pt}\ \isacommand{by}\isamarkupfalse%
\ metis\isanewline
\ \ \ \ \isanewline
\ \ \ \ \isacommand{show}\isamarkupfalse%
\ {\isacharquery}{\kern0pt}thesis\ \isacommand{using}\isamarkupfalse%
\ {\isadigit{1}}\ {\isadigit{2}}\ {\isadigit{3}}\ \isacommand{by}\isamarkupfalse%
\ auto\isanewline
\ \ \isacommand{qed}\isamarkupfalse%
\isanewline
\isacommand{qed}\isamarkupfalse%
%
\endisatagproof
{\isafoldproof}%
%
\isadelimproof
%
\endisadelimproof
%
\begin{isamarkuptext}%
For every \isa{{\isadigit{5}}{\isasymtimes}m}-board with \isa{m\ {\isasymge}\ {\isadigit{5}}} there exists a knight's path that starts in 
\isa{{\isacharparenleft}{\kern0pt}{\isadigit{1}}{\isacharcomma}{\kern0pt}{\isadigit{1}}{\isacharparenright}{\kern0pt}} (bottom-left) and ends in \isa{{\isacharparenleft}{\kern0pt}{\isadigit{4}}{\isacharcomma}{\kern0pt}m{\isacharminus}{\kern0pt}{\isadigit{1}}{\isacharparenright}{\kern0pt}} (top-right).%
\end{isamarkuptext}\isamarkuptrue%
\isacommand{lemma}\isamarkupfalse%
\ knights{\isacharunderscore}{\kern0pt}path{\isacharunderscore}{\kern0pt}{\isadigit{5}}xm{\isacharunderscore}{\kern0pt}ur{\isacharunderscore}{\kern0pt}exists{\isacharcolon}{\kern0pt}\ \isanewline
\ \ \isakeyword{assumes}\ {\isachardoublequoteopen}m\ {\isasymge}\ {\isadigit{5}}{\isachardoublequoteclose}\ \isanewline
\ \ \isakeyword{shows}\ {\isachardoublequoteopen}{\isasymexists}ps{\isachardot}{\kern0pt}\ knights{\isacharunderscore}{\kern0pt}path\ {\isacharparenleft}{\kern0pt}board\ {\isadigit{5}}\ m{\isacharparenright}{\kern0pt}\ ps\ {\isasymand}\ hd\ ps\ {\isacharequal}{\kern0pt}\ {\isacharparenleft}{\kern0pt}{\isadigit{1}}{\isacharcomma}{\kern0pt}{\isadigit{1}}{\isacharparenright}{\kern0pt}\ {\isasymand}\ last\ ps\ {\isacharequal}{\kern0pt}\ {\isacharparenleft}{\kern0pt}{\isadigit{4}}{\isacharcomma}{\kern0pt}int\ m{\isacharminus}{\kern0pt}{\isadigit{1}}{\isacharparenright}{\kern0pt}{\isachardoublequoteclose}\isanewline
%
\isadelimproof
\ \ %
\endisadelimproof
%
\isatagproof
\isacommand{using}\isamarkupfalse%
\ assms\isanewline
\isacommand{proof}\isamarkupfalse%
\ {\isacharminus}{\kern0pt}\isanewline
\ \ \isacommand{have}\isamarkupfalse%
\ {\isachardoublequoteopen}m\ {\isasymin}\ {\isacharbraceleft}{\kern0pt}{\isadigit{5}}{\isacharcomma}{\kern0pt}{\isadigit{6}}{\isacharcomma}{\kern0pt}{\isadigit{7}}{\isacharcomma}{\kern0pt}{\isadigit{8}}{\isacharcomma}{\kern0pt}{\isadigit{9}}{\isacharbraceright}{\kern0pt}\ {\isasymor}\ {\isadigit{5}}\ {\isasymle}\ m{\isacharminus}{\kern0pt}{\isadigit{5}}{\isachardoublequoteclose}\ \isacommand{using}\isamarkupfalse%
\ assms\ \isacommand{by}\isamarkupfalse%
\ auto\isanewline
\ \ \isacommand{then}\isamarkupfalse%
\ \isacommand{show}\isamarkupfalse%
\ {\isacharquery}{\kern0pt}thesis\isanewline
\ \ \isacommand{proof}\isamarkupfalse%
\ {\isacharparenleft}{\kern0pt}elim\ disjE{\isacharparenright}{\kern0pt}\isanewline
\ \ \ \ \isacommand{assume}\isamarkupfalse%
\ {\isachardoublequoteopen}m\ {\isasymin}\ {\isacharbraceleft}{\kern0pt}{\isadigit{5}}{\isacharcomma}{\kern0pt}{\isadigit{6}}{\isacharcomma}{\kern0pt}{\isadigit{7}}{\isacharcomma}{\kern0pt}{\isadigit{8}}{\isacharcomma}{\kern0pt}{\isadigit{9}}{\isacharbraceright}{\kern0pt}{\isachardoublequoteclose}\isanewline
\ \ \ \ \isacommand{then}\isamarkupfalse%
\ \isacommand{show}\isamarkupfalse%
\ {\isacharquery}{\kern0pt}thesis\ \isacommand{using}\isamarkupfalse%
\ kp{\isacharunderscore}{\kern0pt}{\isadigit{5}}xm{\isacharunderscore}{\kern0pt}ur\ \isacommand{by}\isamarkupfalse%
\ fastforce\isanewline
\ \ \isacommand{next}\isamarkupfalse%
\isanewline
\ \ \ \ \isacommand{assume}\isamarkupfalse%
\ m{\isacharunderscore}{\kern0pt}ge{\isacharcolon}{\kern0pt}\ {\isachardoublequoteopen}{\isadigit{5}}\ {\isasymle}\ m{\isacharminus}{\kern0pt}{\isadigit{5}}{\isachardoublequoteclose}\ \isanewline
\ \ \ \ \isacommand{then}\isamarkupfalse%
\ \isacommand{obtain}\isamarkupfalse%
\ ps\isactrlsub {\isadigit{1}}\ \isakeyword{where}\ ps{\isacharunderscore}{\kern0pt}prems{\isacharcolon}{\kern0pt}\ {\isachardoublequoteopen}knights{\isacharunderscore}{\kern0pt}path\ {\isacharparenleft}{\kern0pt}board\ {\isadigit{5}}\ {\isacharparenleft}{\kern0pt}m{\isacharminus}{\kern0pt}{\isadigit{5}}{\isacharparenright}{\kern0pt}{\isacharparenright}{\kern0pt}\ ps\isactrlsub {\isadigit{1}}{\isachardoublequoteclose}\ {\isachardoublequoteopen}hd\ ps\isactrlsub {\isadigit{1}}\ {\isacharequal}{\kern0pt}\ {\isacharparenleft}{\kern0pt}{\isadigit{1}}{\isacharcomma}{\kern0pt}{\isadigit{1}}{\isacharparenright}{\kern0pt}{\isachardoublequoteclose}\ \isanewline
\ \ \ \ \ \ \ \ \ \ \ \ \ \ \ \ \ \ \ \ \ \ \ \ \ \ \ \ \ \ \ \ \ \ \ {\isachardoublequoteopen}last\ ps\isactrlsub {\isadigit{1}}\ {\isacharequal}{\kern0pt}\ {\isacharparenleft}{\kern0pt}{\isadigit{2}}{\isacharcomma}{\kern0pt}int\ {\isacharparenleft}{\kern0pt}m{\isacharminus}{\kern0pt}{\isadigit{5}}{\isacharparenright}{\kern0pt}{\isacharminus}{\kern0pt}{\isadigit{1}}{\isacharparenright}{\kern0pt}{\isachardoublequoteclose}\ {\isachardoublequoteopen}ps\isactrlsub {\isadigit{1}}\ {\isasymnoteq}\ {\isacharbrackleft}{\kern0pt}{\isacharbrackright}{\kern0pt}{\isachardoublequoteclose}\isanewline
\ \ \ \ \ \ \isacommand{using}\isamarkupfalse%
\ knights{\isacharunderscore}{\kern0pt}path{\isacharunderscore}{\kern0pt}{\isadigit{5}}xm{\isacharunderscore}{\kern0pt}lr{\isacharunderscore}{\kern0pt}exists{\isacharbrackleft}{\kern0pt}of\ {\isachardoublequoteopen}{\isacharparenleft}{\kern0pt}m{\isacharminus}{\kern0pt}{\isadigit{5}}{\isacharparenright}{\kern0pt}{\isachardoublequoteclose}{\isacharbrackright}{\kern0pt}\ knights{\isacharunderscore}{\kern0pt}path{\isacharunderscore}{\kern0pt}non{\isacharunderscore}{\kern0pt}nil\ \isacommand{by}\isamarkupfalse%
\ auto\isanewline
\ \ \ \ \isacommand{let}\isamarkupfalse%
\ {\isacharquery}{\kern0pt}ps\isactrlsub {\isadigit{2}}{\isacharequal}{\kern0pt}{\isachardoublequoteopen}kp{\isadigit{5}}x{\isadigit{5}}ur{\isachardoublequoteclose}\isanewline
\ \ \ \ \isacommand{let}\isamarkupfalse%
\ {\isacharquery}{\kern0pt}ps{\isacharprime}{\kern0pt}{\isacharequal}{\kern0pt}{\isachardoublequoteopen}ps\isactrlsub {\isadigit{1}}\ {\isacharat}{\kern0pt}\ trans{\isacharunderscore}{\kern0pt}path\ {\isacharparenleft}{\kern0pt}{\isadigit{0}}{\isacharcomma}{\kern0pt}int\ {\isacharparenleft}{\kern0pt}m{\isacharminus}{\kern0pt}{\isadigit{5}}{\isacharparenright}{\kern0pt}{\isacharparenright}{\kern0pt}\ {\isacharquery}{\kern0pt}ps\isactrlsub {\isadigit{2}}{\isachardoublequoteclose}\isanewline
\ \ \ \ \isacommand{have}\isamarkupfalse%
\ {\isachardoublequoteopen}knights{\isacharunderscore}{\kern0pt}path\ b{\isadigit{5}}x{\isadigit{5}}\ {\isacharquery}{\kern0pt}ps\isactrlsub {\isadigit{2}}{\isachardoublequoteclose}\ {\isachardoublequoteopen}hd\ {\isacharquery}{\kern0pt}ps\isactrlsub {\isadigit{2}}\ {\isacharequal}{\kern0pt}\ {\isacharparenleft}{\kern0pt}{\isadigit{1}}{\isacharcomma}{\kern0pt}\ {\isadigit{1}}{\isacharparenright}{\kern0pt}{\isachardoublequoteclose}\ {\isachardoublequoteopen}{\isacharquery}{\kern0pt}ps\isactrlsub {\isadigit{2}}\ {\isasymnoteq}\ {\isacharbrackleft}{\kern0pt}{\isacharbrackright}{\kern0pt}{\isachardoublequoteclose}\ \isanewline
\ \ \ \ \ \ \ \ \ {\isachardoublequoteopen}last\ {\isacharquery}{\kern0pt}ps\isactrlsub {\isadigit{2}}\ {\isacharequal}{\kern0pt}\ {\isacharparenleft}{\kern0pt}{\isadigit{4}}{\isacharcomma}{\kern0pt}{\isadigit{4}}{\isacharparenright}{\kern0pt}{\isachardoublequoteclose}\isanewline
\ \ \ \ \ \ \isacommand{using}\isamarkupfalse%
\ kp{\isacharunderscore}{\kern0pt}{\isadigit{5}}xm{\isacharunderscore}{\kern0pt}ur\ \isacommand{by}\isamarkupfalse%
\ auto\isanewline
\ \ \ \ \isacommand{then}\isamarkupfalse%
\ \isacommand{have}\isamarkupfalse%
\ {\isadigit{1}}{\isacharcolon}{\kern0pt}\ {\isachardoublequoteopen}knights{\isacharunderscore}{\kern0pt}path\ {\isacharparenleft}{\kern0pt}board\ {\isadigit{5}}\ m{\isacharparenright}{\kern0pt}\ {\isacharquery}{\kern0pt}ps{\isacharprime}{\kern0pt}{\isachardoublequoteclose}\ \ \ \ \ \ \ \ \ \ \ \ \ \ \isanewline
\ \ \ \ \ \ \isacommand{using}\isamarkupfalse%
\ m{\isacharunderscore}{\kern0pt}ge\ ps{\isacharunderscore}{\kern0pt}prems\ knights{\isacharunderscore}{\kern0pt}path{\isacharunderscore}{\kern0pt}lr{\isacharunderscore}{\kern0pt}concat{\isacharbrackleft}{\kern0pt}of\ {\isadigit{5}}\ {\isachardoublequoteopen}m{\isacharminus}{\kern0pt}{\isadigit{5}}{\isachardoublequoteclose}\ ps\isactrlsub {\isadigit{1}}\ {\isadigit{5}}\ {\isacharquery}{\kern0pt}ps\isactrlsub {\isadigit{2}}{\isacharbrackright}{\kern0pt}\ \isacommand{by}\isamarkupfalse%
\ auto\isanewline
\isanewline
\ \ \ \ \isacommand{have}\isamarkupfalse%
\ {\isadigit{2}}{\isacharcolon}{\kern0pt}\ {\isachardoublequoteopen}hd\ {\isacharquery}{\kern0pt}ps{\isacharprime}{\kern0pt}\ {\isacharequal}{\kern0pt}\ {\isacharparenleft}{\kern0pt}{\isadigit{1}}{\isacharcomma}{\kern0pt}{\isadigit{1}}{\isacharparenright}{\kern0pt}{\isachardoublequoteclose}\ \isacommand{using}\isamarkupfalse%
\ ps{\isacharunderscore}{\kern0pt}prems\ \isacommand{by}\isamarkupfalse%
\ auto\isanewline
\isanewline
\ \ \ \ \isacommand{have}\isamarkupfalse%
\ {\isachardoublequoteopen}last\ {\isacharparenleft}{\kern0pt}trans{\isacharunderscore}{\kern0pt}path\ {\isacharparenleft}{\kern0pt}{\isadigit{0}}{\isacharcomma}{\kern0pt}int\ {\isacharparenleft}{\kern0pt}m{\isacharminus}{\kern0pt}{\isadigit{5}}{\isacharparenright}{\kern0pt}{\isacharparenright}{\kern0pt}\ {\isacharquery}{\kern0pt}ps\isactrlsub {\isadigit{2}}{\isacharparenright}{\kern0pt}\ {\isacharequal}{\kern0pt}\ {\isacharparenleft}{\kern0pt}{\isadigit{4}}{\isacharcomma}{\kern0pt}int\ m{\isacharminus}{\kern0pt}{\isadigit{1}}{\isacharparenright}{\kern0pt}{\isachardoublequoteclose}\isanewline
\ \ \ \ \ \ \isacommand{using}\isamarkupfalse%
\ m{\isacharunderscore}{\kern0pt}ge\ last{\isacharunderscore}{\kern0pt}trans{\isacharunderscore}{\kern0pt}path{\isacharbrackleft}{\kern0pt}OF\ {\isacartoucheopen}{\isacharquery}{\kern0pt}ps\isactrlsub {\isadigit{2}}\ {\isasymnoteq}\ {\isacharbrackleft}{\kern0pt}{\isacharbrackright}{\kern0pt}{\isacartoucheclose}\ {\isacartoucheopen}last\ {\isacharquery}{\kern0pt}ps\isactrlsub {\isadigit{2}}\ {\isacharequal}{\kern0pt}\ {\isacharparenleft}{\kern0pt}{\isadigit{4}}{\isacharcomma}{\kern0pt}{\isadigit{4}}{\isacharparenright}{\kern0pt}{\isacartoucheclose}{\isacharbrackright}{\kern0pt}\ \isacommand{by}\isamarkupfalse%
\ auto\isanewline
\ \ \ \ \isacommand{then}\isamarkupfalse%
\ \isacommand{have}\isamarkupfalse%
\ {\isadigit{3}}{\isacharcolon}{\kern0pt}\ {\isachardoublequoteopen}last\ {\isacharquery}{\kern0pt}ps{\isacharprime}{\kern0pt}\ {\isacharequal}{\kern0pt}\ {\isacharparenleft}{\kern0pt}{\isadigit{4}}{\isacharcomma}{\kern0pt}int\ m{\isacharminus}{\kern0pt}{\isadigit{1}}{\isacharparenright}{\kern0pt}{\isachardoublequoteclose}\isanewline
\ \ \ \ \ \ \isacommand{using}\isamarkupfalse%
\ last{\isacharunderscore}{\kern0pt}appendR{\isacharbrackleft}{\kern0pt}OF\ trans{\isacharunderscore}{\kern0pt}path{\isacharunderscore}{\kern0pt}non{\isacharunderscore}{\kern0pt}nil{\isacharbrackleft}{\kern0pt}OF\ {\isacartoucheopen}{\isacharquery}{\kern0pt}ps\isactrlsub {\isadigit{2}}\ {\isasymnoteq}\ {\isacharbrackleft}{\kern0pt}{\isacharbrackright}{\kern0pt}{\isacartoucheclose}{\isacharbrackright}{\kern0pt}{\isacharcomma}{\kern0pt}symmetric{\isacharbrackright}{\kern0pt}\ \isacommand{by}\isamarkupfalse%
\ metis\isanewline
\ \ \ \ \isanewline
\ \ \ \ \isacommand{show}\isamarkupfalse%
\ {\isacharquery}{\kern0pt}thesis\ \isacommand{using}\isamarkupfalse%
\ {\isadigit{1}}\ {\isadigit{2}}\ {\isadigit{3}}\ \isacommand{by}\isamarkupfalse%
\ auto\isanewline
\ \ \isacommand{qed}\isamarkupfalse%
\isanewline
\isacommand{qed}\isamarkupfalse%
%
\endisatagproof
{\isafoldproof}%
%
\isadelimproof
%
\endisadelimproof
%
\begin{isamarkuptext}%
\isa{{\isadigit{5}}\ {\isasymle}\ {\isacharquery}{\kern0pt}m\ {\isasymLongrightarrow}\ {\isasymexists}ps{\isachardot}{\kern0pt}\ knights{\isacharunderscore}{\kern0pt}path\ {\isacharparenleft}{\kern0pt}board\ {\isadigit{5}}\ {\isacharquery}{\kern0pt}m{\isacharparenright}{\kern0pt}\ ps\ {\isasymand}\ hd\ ps\ {\isacharequal}{\kern0pt}\ {\isacharparenleft}{\kern0pt}{\isadigit{1}}{\isacharcomma}{\kern0pt}\ {\isadigit{1}}{\isacharparenright}{\kern0pt}\ {\isasymand}\ last\ ps\ {\isacharequal}{\kern0pt}\ {\isacharparenleft}{\kern0pt}{\isadigit{2}}{\isacharcomma}{\kern0pt}\ int\ {\isacharquery}{\kern0pt}m\ {\isacharminus}{\kern0pt}\ {\isadigit{1}}{\isacharparenright}{\kern0pt}} and \isa{{\isadigit{5}}\ {\isasymle}\ {\isacharquery}{\kern0pt}m\ {\isasymLongrightarrow}\ {\isasymexists}ps{\isachardot}{\kern0pt}\ knights{\isacharunderscore}{\kern0pt}path\ {\isacharparenleft}{\kern0pt}board\ {\isadigit{5}}\ {\isacharquery}{\kern0pt}m{\isacharparenright}{\kern0pt}\ ps\ {\isasymand}\ hd\ ps\ {\isacharequal}{\kern0pt}\ {\isacharparenleft}{\kern0pt}{\isadigit{1}}{\isacharcomma}{\kern0pt}\ {\isadigit{1}}{\isacharparenright}{\kern0pt}\ {\isasymand}\ last\ ps\ {\isacharequal}{\kern0pt}\ {\isacharparenleft}{\kern0pt}{\isadigit{2}}{\isacharcomma}{\kern0pt}\ int\ {\isacharquery}{\kern0pt}m\ {\isacharminus}{\kern0pt}\ {\isadigit{1}}{\isacharparenright}{\kern0pt}} formalize Lemma 1 
from \cite{cull_decurtins_1987}.%
\end{isamarkuptext}\isamarkuptrue%
\isacommand{lemmas}\isamarkupfalse%
\ knights{\isacharunderscore}{\kern0pt}path{\isacharunderscore}{\kern0pt}{\isadigit{5}}xm{\isacharunderscore}{\kern0pt}exists\ {\isacharequal}{\kern0pt}\ knights{\isacharunderscore}{\kern0pt}path{\isacharunderscore}{\kern0pt}{\isadigit{5}}xm{\isacharunderscore}{\kern0pt}lr{\isacharunderscore}{\kern0pt}exists\ knights{\isacharunderscore}{\kern0pt}path{\isacharunderscore}{\kern0pt}{\isadigit{5}}xm{\isacharunderscore}{\kern0pt}ur{\isacharunderscore}{\kern0pt}exists%
\isadelimdocument
%
\endisadelimdocument
%
\isatagdocument
%
\isamarkupsection{Knight's Paths and Circuits for \isa{{\isadigit{6}}{\isasymtimes}m}-Boards%
}
\isamarkuptrue%
%
\endisatagdocument
{\isafolddocument}%
%
\isadelimdocument
%
\endisadelimdocument
\isacommand{abbreviation}\isamarkupfalse%
\ {\isachardoublequoteopen}b{\isadigit{6}}x{\isadigit{5}}\ {\isasymequiv}\ board\ {\isadigit{6}}\ {\isadigit{5}}{\isachardoublequoteclose}%
\begin{isamarkuptext}%
A Knight's path for the \isa{{\isacharparenleft}{\kern0pt}{\isadigit{6}}{\isasymtimes}{\isadigit{5}}{\isacharparenright}{\kern0pt}}-board that starts in the lower-left and ends in the upper-left.
  \begin{table}[H]
    \begin{tabular}{lllll}
      10 & 19 &  4 & 29 & 12 \\
       3 & 30 & 11 & 20 &  5 \\
      18 &  9 & 24 & 13 & 28 \\
      25 &  2 & 17 &  6 & 21 \\
      16 & 23 &  8 & 27 & 14 \\
       1 & 26 & 15 & 22 &  7
    \end{tabular}
  \end{table}%
\end{isamarkuptext}\isamarkuptrue%
\isacommand{abbreviation}\isamarkupfalse%
\ {\isachardoublequoteopen}kp{\isadigit{6}}x{\isadigit{5}}ul\ {\isasymequiv}\ the\ {\isacharparenleft}{\kern0pt}to{\isacharunderscore}{\kern0pt}path\ \isanewline
\ \ {\isacharbrackleft}{\kern0pt}{\isacharbrackleft}{\kern0pt}{\isadigit{1}}{\isadigit{0}}{\isacharcomma}{\kern0pt}{\isadigit{1}}{\isadigit{9}}{\isacharcomma}{\kern0pt}{\isadigit{4}}{\isacharcomma}{\kern0pt}{\isadigit{2}}{\isadigit{9}}{\isacharcomma}{\kern0pt}{\isadigit{1}}{\isadigit{2}}{\isacharbrackright}{\kern0pt}{\isacharcomma}{\kern0pt}\isanewline
\ \ {\isacharbrackleft}{\kern0pt}{\isadigit{3}}{\isacharcomma}{\kern0pt}{\isadigit{3}}{\isadigit{0}}{\isacharcomma}{\kern0pt}{\isadigit{1}}{\isadigit{1}}{\isacharcomma}{\kern0pt}{\isadigit{2}}{\isadigit{0}}{\isacharcomma}{\kern0pt}{\isadigit{5}}{\isacharbrackright}{\kern0pt}{\isacharcomma}{\kern0pt}\isanewline
\ \ {\isacharbrackleft}{\kern0pt}{\isadigit{1}}{\isadigit{8}}{\isacharcomma}{\kern0pt}{\isadigit{9}}{\isacharcomma}{\kern0pt}{\isadigit{2}}{\isadigit{4}}{\isacharcomma}{\kern0pt}{\isadigit{1}}{\isadigit{3}}{\isacharcomma}{\kern0pt}{\isadigit{2}}{\isadigit{8}}{\isacharbrackright}{\kern0pt}{\isacharcomma}{\kern0pt}\isanewline
\ \ {\isacharbrackleft}{\kern0pt}{\isadigit{2}}{\isadigit{5}}{\isacharcomma}{\kern0pt}{\isadigit{2}}{\isacharcomma}{\kern0pt}{\isadigit{1}}{\isadigit{7}}{\isacharcomma}{\kern0pt}{\isadigit{6}}{\isacharcomma}{\kern0pt}{\isadigit{2}}{\isadigit{1}}{\isacharbrackright}{\kern0pt}{\isacharcomma}{\kern0pt}\isanewline
\ \ {\isacharbrackleft}{\kern0pt}{\isadigit{1}}{\isadigit{6}}{\isacharcomma}{\kern0pt}{\isadigit{2}}{\isadigit{3}}{\isacharcomma}{\kern0pt}{\isadigit{8}}{\isacharcomma}{\kern0pt}{\isadigit{2}}{\isadigit{7}}{\isacharcomma}{\kern0pt}{\isadigit{1}}{\isadigit{4}}{\isacharbrackright}{\kern0pt}{\isacharcomma}{\kern0pt}\isanewline
\ \ {\isacharbrackleft}{\kern0pt}{\isadigit{1}}{\isacharcomma}{\kern0pt}{\isadigit{2}}{\isadigit{6}}{\isacharcomma}{\kern0pt}{\isadigit{1}}{\isadigit{5}}{\isacharcomma}{\kern0pt}{\isadigit{2}}{\isadigit{2}}{\isacharcomma}{\kern0pt}{\isadigit{7}}{\isacharbrackright}{\kern0pt}{\isacharbrackright}{\kern0pt}{\isacharparenright}{\kern0pt}{\isachardoublequoteclose}\isanewline
\isacommand{lemma}\isamarkupfalse%
\ kp{\isacharunderscore}{\kern0pt}{\isadigit{6}}x{\isadigit{5}}{\isacharunderscore}{\kern0pt}ul{\isacharcolon}{\kern0pt}\ {\isachardoublequoteopen}knights{\isacharunderscore}{\kern0pt}path\ b{\isadigit{6}}x{\isadigit{5}}\ kp{\isadigit{6}}x{\isadigit{5}}ul{\isachardoublequoteclose}\isanewline
%
\isadelimproof
\ \ %
\endisadelimproof
%
\isatagproof
\isacommand{by}\isamarkupfalse%
\ {\isacharparenleft}{\kern0pt}simp\ only{\isacharcolon}{\kern0pt}\ knights{\isacharunderscore}{\kern0pt}path{\isacharunderscore}{\kern0pt}exec{\isacharunderscore}{\kern0pt}simp{\isacharparenright}{\kern0pt}\ eval%
\endisatagproof
{\isafoldproof}%
%
\isadelimproof
\isanewline
%
\endisadelimproof
\isanewline
\isacommand{lemma}\isamarkupfalse%
\ kp{\isacharunderscore}{\kern0pt}{\isadigit{6}}x{\isadigit{5}}{\isacharunderscore}{\kern0pt}ul{\isacharunderscore}{\kern0pt}hd{\isacharcolon}{\kern0pt}\ {\isachardoublequoteopen}hd\ kp{\isadigit{6}}x{\isadigit{5}}ul\ {\isacharequal}{\kern0pt}\ {\isacharparenleft}{\kern0pt}{\isadigit{1}}{\isacharcomma}{\kern0pt}{\isadigit{1}}{\isacharparenright}{\kern0pt}{\isachardoublequoteclose}%
\isadelimproof
\ %
\endisadelimproof
%
\isatagproof
\isacommand{by}\isamarkupfalse%
\ eval%
\endisatagproof
{\isafoldproof}%
%
\isadelimproof
%
\endisadelimproof
\isanewline
\isanewline
\isacommand{lemma}\isamarkupfalse%
\ kp{\isacharunderscore}{\kern0pt}{\isadigit{6}}x{\isadigit{5}}{\isacharunderscore}{\kern0pt}ul{\isacharunderscore}{\kern0pt}last{\isacharcolon}{\kern0pt}\ {\isachardoublequoteopen}last\ kp{\isadigit{6}}x{\isadigit{5}}ul\ {\isacharequal}{\kern0pt}\ {\isacharparenleft}{\kern0pt}{\isadigit{5}}{\isacharcomma}{\kern0pt}{\isadigit{2}}{\isacharparenright}{\kern0pt}{\isachardoublequoteclose}%
\isadelimproof
\ %
\endisadelimproof
%
\isatagproof
\isacommand{by}\isamarkupfalse%
\ eval%
\endisatagproof
{\isafoldproof}%
%
\isadelimproof
%
\endisadelimproof
\isanewline
\isanewline
\isacommand{lemma}\isamarkupfalse%
\ kp{\isacharunderscore}{\kern0pt}{\isadigit{6}}x{\isadigit{5}}{\isacharunderscore}{\kern0pt}ul{\isacharunderscore}{\kern0pt}non{\isacharunderscore}{\kern0pt}nil{\isacharcolon}{\kern0pt}\ {\isachardoublequoteopen}kp{\isadigit{6}}x{\isadigit{5}}ul\ {\isasymnoteq}\ {\isacharbrackleft}{\kern0pt}{\isacharbrackright}{\kern0pt}{\isachardoublequoteclose}%
\isadelimproof
\ %
\endisadelimproof
%
\isatagproof
\isacommand{by}\isamarkupfalse%
\ eval%
\endisatagproof
{\isafoldproof}%
%
\isadelimproof
%
\endisadelimproof
%
\begin{isamarkuptext}%
A Knight's circuit for the \isa{{\isacharparenleft}{\kern0pt}{\isadigit{6}}{\isasymtimes}{\isadigit{5}}{\isacharparenright}{\kern0pt}}-board.
  \begin{table}[H]
    \begin{tabular}{lllll}
      16 &  9 &  6 & 27 & 18 \\
       7 & 26 & 17 & 14 &  5 \\
      10 & 15 &  8 & 19 & 28 \\
      25 & 30 & 23 &  4 & 13 \\
      22 & 11 &  2 & 29 & 20 \\
       1 & 24 & 21 & 12 &  3
    \end{tabular}
  \end{table}%
\end{isamarkuptext}\isamarkuptrue%
\isacommand{abbreviation}\isamarkupfalse%
\ {\isachardoublequoteopen}kc{\isadigit{6}}x{\isadigit{5}}\ {\isasymequiv}\ the\ {\isacharparenleft}{\kern0pt}to{\isacharunderscore}{\kern0pt}path\ \isanewline
\ \ {\isacharbrackleft}{\kern0pt}{\isacharbrackleft}{\kern0pt}{\isadigit{1}}{\isadigit{6}}{\isacharcomma}{\kern0pt}{\isadigit{9}}{\isacharcomma}{\kern0pt}{\isadigit{6}}{\isacharcomma}{\kern0pt}{\isadigit{2}}{\isadigit{7}}{\isacharcomma}{\kern0pt}{\isadigit{1}}{\isadigit{8}}{\isacharbrackright}{\kern0pt}{\isacharcomma}{\kern0pt}\isanewline
\ \ {\isacharbrackleft}{\kern0pt}{\isadigit{7}}{\isacharcomma}{\kern0pt}{\isadigit{2}}{\isadigit{6}}{\isacharcomma}{\kern0pt}{\isadigit{1}}{\isadigit{7}}{\isacharcomma}{\kern0pt}{\isadigit{1}}{\isadigit{4}}{\isacharcomma}{\kern0pt}{\isadigit{5}}{\isacharbrackright}{\kern0pt}{\isacharcomma}{\kern0pt}\isanewline
\ \ {\isacharbrackleft}{\kern0pt}{\isadigit{1}}{\isadigit{0}}{\isacharcomma}{\kern0pt}{\isadigit{1}}{\isadigit{5}}{\isacharcomma}{\kern0pt}{\isadigit{8}}{\isacharcomma}{\kern0pt}{\isadigit{1}}{\isadigit{9}}{\isacharcomma}{\kern0pt}{\isadigit{2}}{\isadigit{8}}{\isacharbrackright}{\kern0pt}{\isacharcomma}{\kern0pt}\isanewline
\ \ {\isacharbrackleft}{\kern0pt}{\isadigit{2}}{\isadigit{5}}{\isacharcomma}{\kern0pt}{\isadigit{3}}{\isadigit{0}}{\isacharcomma}{\kern0pt}{\isadigit{2}}{\isadigit{3}}{\isacharcomma}{\kern0pt}{\isadigit{4}}{\isacharcomma}{\kern0pt}{\isadigit{1}}{\isadigit{3}}{\isacharbrackright}{\kern0pt}{\isacharcomma}{\kern0pt}\isanewline
\ \ {\isacharbrackleft}{\kern0pt}{\isadigit{2}}{\isadigit{2}}{\isacharcomma}{\kern0pt}{\isadigit{1}}{\isadigit{1}}{\isacharcomma}{\kern0pt}{\isadigit{2}}{\isacharcomma}{\kern0pt}{\isadigit{2}}{\isadigit{9}}{\isacharcomma}{\kern0pt}{\isadigit{2}}{\isadigit{0}}{\isacharbrackright}{\kern0pt}{\isacharcomma}{\kern0pt}\isanewline
\ \ {\isacharbrackleft}{\kern0pt}{\isadigit{1}}{\isacharcomma}{\kern0pt}{\isadigit{2}}{\isadigit{4}}{\isacharcomma}{\kern0pt}{\isadigit{2}}{\isadigit{1}}{\isacharcomma}{\kern0pt}{\isadigit{1}}{\isadigit{2}}{\isacharcomma}{\kern0pt}{\isadigit{3}}{\isacharbrackright}{\kern0pt}{\isacharbrackright}{\kern0pt}{\isacharparenright}{\kern0pt}{\isachardoublequoteclose}\isanewline
\isacommand{lemma}\isamarkupfalse%
\ kc{\isacharunderscore}{\kern0pt}{\isadigit{6}}x{\isadigit{5}}{\isacharcolon}{\kern0pt}\ {\isachardoublequoteopen}knights{\isacharunderscore}{\kern0pt}circuit\ b{\isadigit{6}}x{\isadigit{5}}\ kc{\isadigit{6}}x{\isadigit{5}}{\isachardoublequoteclose}\isanewline
%
\isadelimproof
\ \ %
\endisadelimproof
%
\isatagproof
\isacommand{by}\isamarkupfalse%
\ {\isacharparenleft}{\kern0pt}simp\ only{\isacharcolon}{\kern0pt}\ knights{\isacharunderscore}{\kern0pt}circuit{\isacharunderscore}{\kern0pt}exec{\isacharunderscore}{\kern0pt}simp{\isacharparenright}{\kern0pt}\ eval%
\endisatagproof
{\isafoldproof}%
%
\isadelimproof
\isanewline
%
\endisadelimproof
\isanewline
\isacommand{lemma}\isamarkupfalse%
\ kc{\isacharunderscore}{\kern0pt}{\isadigit{6}}x{\isadigit{5}}{\isacharunderscore}{\kern0pt}hd{\isacharcolon}{\kern0pt}\ {\isachardoublequoteopen}hd\ kc{\isadigit{6}}x{\isadigit{5}}\ {\isacharequal}{\kern0pt}\ {\isacharparenleft}{\kern0pt}{\isadigit{1}}{\isacharcomma}{\kern0pt}{\isadigit{1}}{\isacharparenright}{\kern0pt}{\isachardoublequoteclose}%
\isadelimproof
\ %
\endisadelimproof
%
\isatagproof
\isacommand{by}\isamarkupfalse%
\ eval%
\endisatagproof
{\isafoldproof}%
%
\isadelimproof
%
\endisadelimproof
\isanewline
\isanewline
\isacommand{lemma}\isamarkupfalse%
\ kc{\isacharunderscore}{\kern0pt}{\isadigit{6}}x{\isadigit{5}}{\isacharunderscore}{\kern0pt}non{\isacharunderscore}{\kern0pt}nil{\isacharcolon}{\kern0pt}\ {\isachardoublequoteopen}kc{\isadigit{6}}x{\isadigit{5}}\ {\isasymnoteq}\ {\isacharbrackleft}{\kern0pt}{\isacharbrackright}{\kern0pt}{\isachardoublequoteclose}%
\isadelimproof
\ %
\endisadelimproof
%
\isatagproof
\isacommand{by}\isamarkupfalse%
\ eval%
\endisatagproof
{\isafoldproof}%
%
\isadelimproof
%
\endisadelimproof
\isanewline
\isanewline
\isacommand{abbreviation}\isamarkupfalse%
\ {\isachardoublequoteopen}b{\isadigit{6}}x{\isadigit{6}}\ {\isasymequiv}\ board\ {\isadigit{6}}\ {\isadigit{6}}{\isachardoublequoteclose}%
\begin{isamarkuptext}%
The path given for the \isa{{\isadigit{6}}{\isasymtimes}{\isadigit{6}}}-board that ends in the upper-left is wrong. The Knight cannot 
move from square 26 to square 27.
  \begin{table}[H]
    \begin{tabular}{llllll}
      14 & 23 &  6 & 28 & 12 & 21 \\
       7 & 36 & 13 & 22 &  5 & \color{red}{27} \\
      24 & 15 & 29 & 35 & 20 & 11 \\
      30 &  8 & 17 & \color{red}{26} & 34 &  4 \\
      16 & 25 &  2 & 32 & 10 & 19 \\
       1 & 31 &  9 & 18 &  3 & 33
    \end{tabular}
  \end{table}%
\end{isamarkuptext}\isamarkuptrue%
\isacommand{abbreviation}\isamarkupfalse%
\ {\isachardoublequoteopen}kp{\isadigit{6}}x{\isadigit{6}}ul{\isacharunderscore}{\kern0pt}false\ {\isasymequiv}\ the\ {\isacharparenleft}{\kern0pt}to{\isacharunderscore}{\kern0pt}path\ \isanewline
\ \ {\isacharbrackleft}{\kern0pt}{\isacharbrackleft}{\kern0pt}{\isadigit{1}}{\isadigit{4}}{\isacharcomma}{\kern0pt}{\isadigit{2}}{\isadigit{3}}{\isacharcomma}{\kern0pt}{\isadigit{6}}{\isacharcomma}{\kern0pt}{\isadigit{2}}{\isadigit{8}}{\isacharcomma}{\kern0pt}{\isadigit{1}}{\isadigit{2}}{\isacharcomma}{\kern0pt}{\isadigit{2}}{\isadigit{1}}{\isacharbrackright}{\kern0pt}{\isacharcomma}{\kern0pt}\isanewline
\ \ {\isacharbrackleft}{\kern0pt}{\isadigit{7}}{\isacharcomma}{\kern0pt}{\isadigit{3}}{\isadigit{6}}{\isacharcomma}{\kern0pt}{\isadigit{1}}{\isadigit{3}}{\isacharcomma}{\kern0pt}{\isadigit{2}}{\isadigit{2}}{\isacharcomma}{\kern0pt}{\isadigit{5}}{\isacharcomma}{\kern0pt}{\isadigit{2}}{\isadigit{7}}{\isacharbrackright}{\kern0pt}{\isacharcomma}{\kern0pt}\isanewline
\ \ {\isacharbrackleft}{\kern0pt}{\isadigit{2}}{\isadigit{4}}{\isacharcomma}{\kern0pt}{\isadigit{1}}{\isadigit{5}}{\isacharcomma}{\kern0pt}{\isadigit{2}}{\isadigit{9}}{\isacharcomma}{\kern0pt}{\isadigit{3}}{\isadigit{5}}{\isacharcomma}{\kern0pt}{\isadigit{2}}{\isadigit{0}}{\isacharcomma}{\kern0pt}{\isadigit{1}}{\isadigit{1}}{\isacharbrackright}{\kern0pt}{\isacharcomma}{\kern0pt}\isanewline
\ \ {\isacharbrackleft}{\kern0pt}{\isadigit{3}}{\isadigit{0}}{\isacharcomma}{\kern0pt}{\isadigit{8}}{\isacharcomma}{\kern0pt}{\isadigit{1}}{\isadigit{7}}{\isacharcomma}{\kern0pt}{\isadigit{2}}{\isadigit{6}}{\isacharcomma}{\kern0pt}{\isadigit{3}}{\isadigit{4}}{\isacharcomma}{\kern0pt}{\isadigit{4}}{\isacharbrackright}{\kern0pt}{\isacharcomma}{\kern0pt}\isanewline
\ \ {\isacharbrackleft}{\kern0pt}{\isadigit{1}}{\isadigit{6}}{\isacharcomma}{\kern0pt}{\isadigit{2}}{\isadigit{5}}{\isacharcomma}{\kern0pt}{\isadigit{2}}{\isacharcomma}{\kern0pt}{\isadigit{3}}{\isadigit{2}}{\isacharcomma}{\kern0pt}{\isadigit{1}}{\isadigit{0}}{\isacharcomma}{\kern0pt}{\isadigit{1}}{\isadigit{9}}{\isacharbrackright}{\kern0pt}{\isacharcomma}{\kern0pt}\isanewline
\ \ {\isacharbrackleft}{\kern0pt}{\isadigit{1}}{\isacharcomma}{\kern0pt}{\isadigit{3}}{\isadigit{1}}{\isacharcomma}{\kern0pt}{\isadigit{9}}{\isacharcomma}{\kern0pt}{\isadigit{1}}{\isadigit{8}}{\isacharcomma}{\kern0pt}{\isadigit{3}}{\isacharcomma}{\kern0pt}{\isadigit{3}}{\isadigit{3}}{\isacharbrackright}{\kern0pt}{\isacharbrackright}{\kern0pt}{\isacharparenright}{\kern0pt}{\isachardoublequoteclose}\isanewline
\isanewline
\isacommand{lemma}\isamarkupfalse%
\ {\isachardoublequoteopen}{\isasymnot}knights{\isacharunderscore}{\kern0pt}path\ b{\isadigit{6}}x{\isadigit{6}}\ kp{\isadigit{6}}x{\isadigit{6}}ul{\isacharunderscore}{\kern0pt}false{\isachardoublequoteclose}\isanewline
%
\isadelimproof
\ \ %
\endisadelimproof
%
\isatagproof
\isacommand{by}\isamarkupfalse%
\ {\isacharparenleft}{\kern0pt}simp\ only{\isacharcolon}{\kern0pt}\ knights{\isacharunderscore}{\kern0pt}path{\isacharunderscore}{\kern0pt}exec{\isacharunderscore}{\kern0pt}simp{\isacharparenright}{\kern0pt}\ eval%
\endisatagproof
{\isafoldproof}%
%
\isadelimproof
%
\endisadelimproof
%
\begin{isamarkuptext}%
I have computed a correct Knight's path for the \isa{{\isadigit{6}}{\isasymtimes}{\isadigit{6}}}-board that ends in the upper-left.
A Knight's path for the \isa{{\isacharparenleft}{\kern0pt}{\isadigit{6}}{\isasymtimes}{\isadigit{6}}{\isacharparenright}{\kern0pt}}-board that starts in the lower-left and ends in the upper-left.
  \begin{table}[H]
    \begin{tabular}{llllll}
       8 & 25 & 10 & 21 &  6 & 23 \\
      11 & 36 &  7 & 24 & 33 & 20 \\
      26 &  9 & 34 &  3 & 22 &  5 \\
      35 & 12 & 15 & 30 & 19 & 32 \\
      14 & 27 &  2 & 17 &  4 & 29 \\
       1 & 16 & 13 & 28 & 31 & 18
    \end{tabular}
  \end{table}%
\end{isamarkuptext}\isamarkuptrue%
\isacommand{abbreviation}\isamarkupfalse%
\ {\isachardoublequoteopen}kp{\isadigit{6}}x{\isadigit{6}}ul\ {\isasymequiv}\ the\ {\isacharparenleft}{\kern0pt}to{\isacharunderscore}{\kern0pt}path\ \isanewline
\ \ {\isacharbrackleft}{\kern0pt}{\isacharbrackleft}{\kern0pt}{\isadigit{8}}{\isacharcomma}{\kern0pt}{\isadigit{2}}{\isadigit{5}}{\isacharcomma}{\kern0pt}{\isadigit{1}}{\isadigit{0}}{\isacharcomma}{\kern0pt}{\isadigit{2}}{\isadigit{1}}{\isacharcomma}{\kern0pt}{\isadigit{6}}{\isacharcomma}{\kern0pt}{\isadigit{2}}{\isadigit{3}}{\isacharbrackright}{\kern0pt}{\isacharcomma}{\kern0pt}\isanewline
\ \ {\isacharbrackleft}{\kern0pt}{\isadigit{1}}{\isadigit{1}}{\isacharcomma}{\kern0pt}{\isadigit{3}}{\isadigit{6}}{\isacharcomma}{\kern0pt}{\isadigit{7}}{\isacharcomma}{\kern0pt}{\isadigit{2}}{\isadigit{4}}{\isacharcomma}{\kern0pt}{\isadigit{3}}{\isadigit{3}}{\isacharcomma}{\kern0pt}{\isadigit{2}}{\isadigit{0}}{\isacharbrackright}{\kern0pt}{\isacharcomma}{\kern0pt}\isanewline
\ \ {\isacharbrackleft}{\kern0pt}{\isadigit{2}}{\isadigit{6}}{\isacharcomma}{\kern0pt}{\isadigit{9}}{\isacharcomma}{\kern0pt}{\isadigit{3}}{\isadigit{4}}{\isacharcomma}{\kern0pt}{\isadigit{3}}{\isacharcomma}{\kern0pt}{\isadigit{2}}{\isadigit{2}}{\isacharcomma}{\kern0pt}{\isadigit{5}}{\isacharbrackright}{\kern0pt}{\isacharcomma}{\kern0pt}\isanewline
\ \ {\isacharbrackleft}{\kern0pt}{\isadigit{3}}{\isadigit{5}}{\isacharcomma}{\kern0pt}{\isadigit{1}}{\isadigit{2}}{\isacharcomma}{\kern0pt}{\isadigit{1}}{\isadigit{5}}{\isacharcomma}{\kern0pt}{\isadigit{3}}{\isadigit{0}}{\isacharcomma}{\kern0pt}{\isadigit{1}}{\isadigit{9}}{\isacharcomma}{\kern0pt}{\isadigit{3}}{\isadigit{2}}{\isacharbrackright}{\kern0pt}{\isacharcomma}{\kern0pt}\isanewline
\ \ {\isacharbrackleft}{\kern0pt}{\isadigit{1}}{\isadigit{4}}{\isacharcomma}{\kern0pt}{\isadigit{2}}{\isadigit{7}}{\isacharcomma}{\kern0pt}{\isadigit{2}}{\isacharcomma}{\kern0pt}{\isadigit{1}}{\isadigit{7}}{\isacharcomma}{\kern0pt}{\isadigit{4}}{\isacharcomma}{\kern0pt}{\isadigit{2}}{\isadigit{9}}{\isacharbrackright}{\kern0pt}{\isacharcomma}{\kern0pt}\isanewline
\ \ {\isacharbrackleft}{\kern0pt}{\isadigit{1}}{\isacharcomma}{\kern0pt}{\isadigit{1}}{\isadigit{6}}{\isacharcomma}{\kern0pt}{\isadigit{1}}{\isadigit{3}}{\isacharcomma}{\kern0pt}{\isadigit{2}}{\isadigit{8}}{\isacharcomma}{\kern0pt}{\isadigit{3}}{\isadigit{1}}{\isacharcomma}{\kern0pt}{\isadigit{1}}{\isadigit{8}}{\isacharbrackright}{\kern0pt}{\isacharbrackright}{\kern0pt}{\isacharparenright}{\kern0pt}{\isachardoublequoteclose}\isanewline
\isacommand{lemma}\isamarkupfalse%
\ kp{\isacharunderscore}{\kern0pt}{\isadigit{6}}x{\isadigit{6}}{\isacharunderscore}{\kern0pt}ul{\isacharcolon}{\kern0pt}\ {\isachardoublequoteopen}knights{\isacharunderscore}{\kern0pt}path\ b{\isadigit{6}}x{\isadigit{6}}\ kp{\isadigit{6}}x{\isadigit{6}}ul{\isachardoublequoteclose}\isanewline
%
\isadelimproof
\ \ %
\endisadelimproof
%
\isatagproof
\isacommand{by}\isamarkupfalse%
\ {\isacharparenleft}{\kern0pt}simp\ only{\isacharcolon}{\kern0pt}\ knights{\isacharunderscore}{\kern0pt}path{\isacharunderscore}{\kern0pt}exec{\isacharunderscore}{\kern0pt}simp{\isacharparenright}{\kern0pt}\ eval%
\endisatagproof
{\isafoldproof}%
%
\isadelimproof
\isanewline
%
\endisadelimproof
\isanewline
\isacommand{lemma}\isamarkupfalse%
\ kp{\isacharunderscore}{\kern0pt}{\isadigit{6}}x{\isadigit{6}}{\isacharunderscore}{\kern0pt}ul{\isacharunderscore}{\kern0pt}hd{\isacharcolon}{\kern0pt}\ {\isachardoublequoteopen}hd\ kp{\isadigit{6}}x{\isadigit{6}}ul\ {\isacharequal}{\kern0pt}\ {\isacharparenleft}{\kern0pt}{\isadigit{1}}{\isacharcomma}{\kern0pt}{\isadigit{1}}{\isacharparenright}{\kern0pt}{\isachardoublequoteclose}%
\isadelimproof
\ %
\endisadelimproof
%
\isatagproof
\isacommand{by}\isamarkupfalse%
\ eval%
\endisatagproof
{\isafoldproof}%
%
\isadelimproof
%
\endisadelimproof
\isanewline
\isanewline
\isacommand{lemma}\isamarkupfalse%
\ kp{\isacharunderscore}{\kern0pt}{\isadigit{6}}x{\isadigit{6}}{\isacharunderscore}{\kern0pt}ul{\isacharunderscore}{\kern0pt}last{\isacharcolon}{\kern0pt}\ {\isachardoublequoteopen}last\ kp{\isadigit{6}}x{\isadigit{6}}ul\ {\isacharequal}{\kern0pt}\ {\isacharparenleft}{\kern0pt}{\isadigit{5}}{\isacharcomma}{\kern0pt}{\isadigit{2}}{\isacharparenright}{\kern0pt}{\isachardoublequoteclose}%
\isadelimproof
\ %
\endisadelimproof
%
\isatagproof
\isacommand{by}\isamarkupfalse%
\ eval%
\endisatagproof
{\isafoldproof}%
%
\isadelimproof
%
\endisadelimproof
\isanewline
\isanewline
\isacommand{lemma}\isamarkupfalse%
\ kp{\isacharunderscore}{\kern0pt}{\isadigit{6}}x{\isadigit{6}}{\isacharunderscore}{\kern0pt}ul{\isacharunderscore}{\kern0pt}non{\isacharunderscore}{\kern0pt}nil{\isacharcolon}{\kern0pt}\ {\isachardoublequoteopen}kp{\isadigit{6}}x{\isadigit{6}}ul\ {\isasymnoteq}\ {\isacharbrackleft}{\kern0pt}{\isacharbrackright}{\kern0pt}{\isachardoublequoteclose}%
\isadelimproof
\ %
\endisadelimproof
%
\isatagproof
\isacommand{by}\isamarkupfalse%
\ eval%
\endisatagproof
{\isafoldproof}%
%
\isadelimproof
%
\endisadelimproof
%
\begin{isamarkuptext}%
A Knight's circuit for the \isa{{\isacharparenleft}{\kern0pt}{\isadigit{6}}{\isasymtimes}{\isadigit{6}}{\isacharparenright}{\kern0pt}}-board.
  \begin{table}[H]
    \begin{tabular}{llllll}
       4 & 25 & 34 & 15 & 18 &  7 \\
      35 & 14 &  5 &  8 & 33 & 16 \\
      24 &  3 & 26 & 17 &  6 & 19 \\
      13 & 36 & 23 & 30 &  9 & 32 \\
      22 & 27 &  2 & 11 & 20 & 29 \\
       1 & 12 & 21 & 28 & 31 & 10
    \end{tabular}
  \end{table}%
\end{isamarkuptext}\isamarkuptrue%
\isacommand{abbreviation}\isamarkupfalse%
\ {\isachardoublequoteopen}kc{\isadigit{6}}x{\isadigit{6}}\ {\isasymequiv}\ the\ {\isacharparenleft}{\kern0pt}to{\isacharunderscore}{\kern0pt}path\ \isanewline
\ \ {\isacharbrackleft}{\kern0pt}{\isacharbrackleft}{\kern0pt}{\isadigit{4}}{\isacharcomma}{\kern0pt}{\isadigit{2}}{\isadigit{5}}{\isacharcomma}{\kern0pt}{\isadigit{3}}{\isadigit{4}}{\isacharcomma}{\kern0pt}{\isadigit{1}}{\isadigit{5}}{\isacharcomma}{\kern0pt}{\isadigit{1}}{\isadigit{8}}{\isacharcomma}{\kern0pt}{\isadigit{7}}{\isacharbrackright}{\kern0pt}{\isacharcomma}{\kern0pt}\isanewline
\ \ {\isacharbrackleft}{\kern0pt}{\isadigit{3}}{\isadigit{5}}{\isacharcomma}{\kern0pt}{\isadigit{1}}{\isadigit{4}}{\isacharcomma}{\kern0pt}{\isadigit{5}}{\isacharcomma}{\kern0pt}{\isadigit{8}}{\isacharcomma}{\kern0pt}{\isadigit{3}}{\isadigit{3}}{\isacharcomma}{\kern0pt}{\isadigit{1}}{\isadigit{6}}{\isacharbrackright}{\kern0pt}{\isacharcomma}{\kern0pt}\isanewline
\ \ {\isacharbrackleft}{\kern0pt}{\isadigit{2}}{\isadigit{4}}{\isacharcomma}{\kern0pt}{\isadigit{3}}{\isacharcomma}{\kern0pt}{\isadigit{2}}{\isadigit{6}}{\isacharcomma}{\kern0pt}{\isadigit{1}}{\isadigit{7}}{\isacharcomma}{\kern0pt}{\isadigit{6}}{\isacharcomma}{\kern0pt}{\isadigit{1}}{\isadigit{9}}{\isacharbrackright}{\kern0pt}{\isacharcomma}{\kern0pt}\isanewline
\ \ {\isacharbrackleft}{\kern0pt}{\isadigit{1}}{\isadigit{3}}{\isacharcomma}{\kern0pt}{\isadigit{3}}{\isadigit{6}}{\isacharcomma}{\kern0pt}{\isadigit{2}}{\isadigit{3}}{\isacharcomma}{\kern0pt}{\isadigit{3}}{\isadigit{0}}{\isacharcomma}{\kern0pt}{\isadigit{9}}{\isacharcomma}{\kern0pt}{\isadigit{3}}{\isadigit{2}}{\isacharbrackright}{\kern0pt}{\isacharcomma}{\kern0pt}\isanewline
\ \ {\isacharbrackleft}{\kern0pt}{\isadigit{2}}{\isadigit{2}}{\isacharcomma}{\kern0pt}{\isadigit{2}}{\isadigit{7}}{\isacharcomma}{\kern0pt}{\isadigit{2}}{\isacharcomma}{\kern0pt}{\isadigit{1}}{\isadigit{1}}{\isacharcomma}{\kern0pt}{\isadigit{2}}{\isadigit{0}}{\isacharcomma}{\kern0pt}{\isadigit{2}}{\isadigit{9}}{\isacharbrackright}{\kern0pt}{\isacharcomma}{\kern0pt}\isanewline
\ \ {\isacharbrackleft}{\kern0pt}{\isadigit{1}}{\isacharcomma}{\kern0pt}{\isadigit{1}}{\isadigit{2}}{\isacharcomma}{\kern0pt}{\isadigit{2}}{\isadigit{1}}{\isacharcomma}{\kern0pt}{\isadigit{2}}{\isadigit{8}}{\isacharcomma}{\kern0pt}{\isadigit{3}}{\isadigit{1}}{\isacharcomma}{\kern0pt}{\isadigit{1}}{\isadigit{0}}{\isacharbrackright}{\kern0pt}{\isacharbrackright}{\kern0pt}{\isacharparenright}{\kern0pt}{\isachardoublequoteclose}\isanewline
\isacommand{lemma}\isamarkupfalse%
\ kc{\isacharunderscore}{\kern0pt}{\isadigit{6}}x{\isadigit{6}}{\isacharcolon}{\kern0pt}\ {\isachardoublequoteopen}knights{\isacharunderscore}{\kern0pt}circuit\ b{\isadigit{6}}x{\isadigit{6}}\ kc{\isadigit{6}}x{\isadigit{6}}{\isachardoublequoteclose}\isanewline
%
\isadelimproof
\ \ %
\endisadelimproof
%
\isatagproof
\isacommand{by}\isamarkupfalse%
\ {\isacharparenleft}{\kern0pt}simp\ only{\isacharcolon}{\kern0pt}\ knights{\isacharunderscore}{\kern0pt}circuit{\isacharunderscore}{\kern0pt}exec{\isacharunderscore}{\kern0pt}simp{\isacharparenright}{\kern0pt}\ eval%
\endisatagproof
{\isafoldproof}%
%
\isadelimproof
\isanewline
%
\endisadelimproof
\isanewline
\isacommand{lemma}\isamarkupfalse%
\ kc{\isacharunderscore}{\kern0pt}{\isadigit{6}}x{\isadigit{6}}{\isacharunderscore}{\kern0pt}hd{\isacharcolon}{\kern0pt}\ {\isachardoublequoteopen}hd\ kc{\isadigit{6}}x{\isadigit{6}}\ {\isacharequal}{\kern0pt}\ {\isacharparenleft}{\kern0pt}{\isadigit{1}}{\isacharcomma}{\kern0pt}{\isadigit{1}}{\isacharparenright}{\kern0pt}{\isachardoublequoteclose}%
\isadelimproof
\ %
\endisadelimproof
%
\isatagproof
\isacommand{by}\isamarkupfalse%
\ eval%
\endisatagproof
{\isafoldproof}%
%
\isadelimproof
%
\endisadelimproof
\isanewline
\isanewline
\isacommand{lemma}\isamarkupfalse%
\ kc{\isacharunderscore}{\kern0pt}{\isadigit{6}}x{\isadigit{6}}{\isacharunderscore}{\kern0pt}non{\isacharunderscore}{\kern0pt}nil{\isacharcolon}{\kern0pt}\ {\isachardoublequoteopen}kc{\isadigit{6}}x{\isadigit{6}}\ {\isasymnoteq}\ {\isacharbrackleft}{\kern0pt}{\isacharbrackright}{\kern0pt}{\isachardoublequoteclose}%
\isadelimproof
\ %
\endisadelimproof
%
\isatagproof
\isacommand{by}\isamarkupfalse%
\ eval%
\endisatagproof
{\isafoldproof}%
%
\isadelimproof
%
\endisadelimproof
\isanewline
\isanewline
\isacommand{abbreviation}\isamarkupfalse%
\ {\isachardoublequoteopen}b{\isadigit{6}}x{\isadigit{7}}\ {\isasymequiv}\ board\ {\isadigit{6}}\ {\isadigit{7}}{\isachardoublequoteclose}%
\begin{isamarkuptext}%
A Knight's path for the \isa{{\isacharparenleft}{\kern0pt}{\isadigit{6}}{\isasymtimes}{\isadigit{7}}{\isacharparenright}{\kern0pt}}-board that starts in the lower-left and ends in the 
upper-left.
  \begin{table}[H]
    \begin{tabular}{lllllll}
      18 & 23 &  8 & 39 & 16 & 25 &  6 \\
       9 & 42 & 17 & 24 &  7 & 40 & 15 \\
      22 & 19 & 32 & 41 & 38 &  5 & 26 \\
      33 & 10 & 21 & 28 & 31 & 14 & 37 \\
      20 & 29 &  2 & 35 & 12 & 27 &  4 \\
       1 & 34 & 11 & 30 &  3 & 36 & 13
    \end{tabular}
  \end{table}%
\end{isamarkuptext}\isamarkuptrue%
\isacommand{abbreviation}\isamarkupfalse%
\ {\isachardoublequoteopen}kp{\isadigit{6}}x{\isadigit{7}}ul\ {\isasymequiv}\ the\ {\isacharparenleft}{\kern0pt}to{\isacharunderscore}{\kern0pt}path\ \isanewline
\ \ {\isacharbrackleft}{\kern0pt}{\isacharbrackleft}{\kern0pt}{\isadigit{1}}{\isadigit{8}}{\isacharcomma}{\kern0pt}{\isadigit{2}}{\isadigit{3}}{\isacharcomma}{\kern0pt}{\isadigit{8}}{\isacharcomma}{\kern0pt}{\isadigit{3}}{\isadigit{9}}{\isacharcomma}{\kern0pt}{\isadigit{1}}{\isadigit{6}}{\isacharcomma}{\kern0pt}{\isadigit{2}}{\isadigit{5}}{\isacharcomma}{\kern0pt}{\isadigit{6}}{\isacharbrackright}{\kern0pt}{\isacharcomma}{\kern0pt}\isanewline
\ \ {\isacharbrackleft}{\kern0pt}{\isadigit{9}}{\isacharcomma}{\kern0pt}{\isadigit{4}}{\isadigit{2}}{\isacharcomma}{\kern0pt}{\isadigit{1}}{\isadigit{7}}{\isacharcomma}{\kern0pt}{\isadigit{2}}{\isadigit{4}}{\isacharcomma}{\kern0pt}{\isadigit{7}}{\isacharcomma}{\kern0pt}{\isadigit{4}}{\isadigit{0}}{\isacharcomma}{\kern0pt}{\isadigit{1}}{\isadigit{5}}{\isacharbrackright}{\kern0pt}{\isacharcomma}{\kern0pt}\isanewline
\ \ {\isacharbrackleft}{\kern0pt}{\isadigit{2}}{\isadigit{2}}{\isacharcomma}{\kern0pt}{\isadigit{1}}{\isadigit{9}}{\isacharcomma}{\kern0pt}{\isadigit{3}}{\isadigit{2}}{\isacharcomma}{\kern0pt}{\isadigit{4}}{\isadigit{1}}{\isacharcomma}{\kern0pt}{\isadigit{3}}{\isadigit{8}}{\isacharcomma}{\kern0pt}{\isadigit{5}}{\isacharcomma}{\kern0pt}{\isadigit{2}}{\isadigit{6}}{\isacharbrackright}{\kern0pt}{\isacharcomma}{\kern0pt}\isanewline
\ \ {\isacharbrackleft}{\kern0pt}{\isadigit{3}}{\isadigit{3}}{\isacharcomma}{\kern0pt}{\isadigit{1}}{\isadigit{0}}{\isacharcomma}{\kern0pt}{\isadigit{2}}{\isadigit{1}}{\isacharcomma}{\kern0pt}{\isadigit{2}}{\isadigit{8}}{\isacharcomma}{\kern0pt}{\isadigit{3}}{\isadigit{1}}{\isacharcomma}{\kern0pt}{\isadigit{1}}{\isadigit{4}}{\isacharcomma}{\kern0pt}{\isadigit{3}}{\isadigit{7}}{\isacharbrackright}{\kern0pt}{\isacharcomma}{\kern0pt}\isanewline
\ \ {\isacharbrackleft}{\kern0pt}{\isadigit{2}}{\isadigit{0}}{\isacharcomma}{\kern0pt}{\isadigit{2}}{\isadigit{9}}{\isacharcomma}{\kern0pt}{\isadigit{2}}{\isacharcomma}{\kern0pt}{\isadigit{3}}{\isadigit{5}}{\isacharcomma}{\kern0pt}{\isadigit{1}}{\isadigit{2}}{\isacharcomma}{\kern0pt}{\isadigit{2}}{\isadigit{7}}{\isacharcomma}{\kern0pt}{\isadigit{4}}{\isacharbrackright}{\kern0pt}{\isacharcomma}{\kern0pt}\isanewline
\ \ {\isacharbrackleft}{\kern0pt}{\isadigit{1}}{\isacharcomma}{\kern0pt}{\isadigit{3}}{\isadigit{4}}{\isacharcomma}{\kern0pt}{\isadigit{1}}{\isadigit{1}}{\isacharcomma}{\kern0pt}{\isadigit{3}}{\isadigit{0}}{\isacharcomma}{\kern0pt}{\isadigit{3}}{\isacharcomma}{\kern0pt}{\isadigit{3}}{\isadigit{6}}{\isacharcomma}{\kern0pt}{\isadigit{1}}{\isadigit{3}}{\isacharbrackright}{\kern0pt}{\isacharbrackright}{\kern0pt}{\isacharparenright}{\kern0pt}{\isachardoublequoteclose}\isanewline
\isacommand{lemma}\isamarkupfalse%
\ kp{\isacharunderscore}{\kern0pt}{\isadigit{6}}x{\isadigit{7}}{\isacharunderscore}{\kern0pt}ul{\isacharcolon}{\kern0pt}\ {\isachardoublequoteopen}knights{\isacharunderscore}{\kern0pt}path\ b{\isadigit{6}}x{\isadigit{7}}\ kp{\isadigit{6}}x{\isadigit{7}}ul{\isachardoublequoteclose}\isanewline
%
\isadelimproof
\ \ %
\endisadelimproof
%
\isatagproof
\isacommand{by}\isamarkupfalse%
\ {\isacharparenleft}{\kern0pt}simp\ only{\isacharcolon}{\kern0pt}\ knights{\isacharunderscore}{\kern0pt}path{\isacharunderscore}{\kern0pt}exec{\isacharunderscore}{\kern0pt}simp{\isacharparenright}{\kern0pt}\ eval%
\endisatagproof
{\isafoldproof}%
%
\isadelimproof
\isanewline
%
\endisadelimproof
\isanewline
\isacommand{lemma}\isamarkupfalse%
\ kp{\isacharunderscore}{\kern0pt}{\isadigit{6}}x{\isadigit{7}}{\isacharunderscore}{\kern0pt}ul{\isacharunderscore}{\kern0pt}hd{\isacharcolon}{\kern0pt}\ {\isachardoublequoteopen}hd\ kp{\isadigit{6}}x{\isadigit{7}}ul\ {\isacharequal}{\kern0pt}\ {\isacharparenleft}{\kern0pt}{\isadigit{1}}{\isacharcomma}{\kern0pt}{\isadigit{1}}{\isacharparenright}{\kern0pt}{\isachardoublequoteclose}%
\isadelimproof
\ %
\endisadelimproof
%
\isatagproof
\isacommand{by}\isamarkupfalse%
\ eval%
\endisatagproof
{\isafoldproof}%
%
\isadelimproof
%
\endisadelimproof
\isanewline
\isanewline
\isacommand{lemma}\isamarkupfalse%
\ kp{\isacharunderscore}{\kern0pt}{\isadigit{6}}x{\isadigit{7}}{\isacharunderscore}{\kern0pt}ul{\isacharunderscore}{\kern0pt}last{\isacharcolon}{\kern0pt}\ {\isachardoublequoteopen}last\ kp{\isadigit{6}}x{\isadigit{7}}ul\ {\isacharequal}{\kern0pt}\ {\isacharparenleft}{\kern0pt}{\isadigit{5}}{\isacharcomma}{\kern0pt}{\isadigit{2}}{\isacharparenright}{\kern0pt}{\isachardoublequoteclose}%
\isadelimproof
\ %
\endisadelimproof
%
\isatagproof
\isacommand{by}\isamarkupfalse%
\ eval%
\endisatagproof
{\isafoldproof}%
%
\isadelimproof
%
\endisadelimproof
\isanewline
\isanewline
\isacommand{lemma}\isamarkupfalse%
\ kp{\isacharunderscore}{\kern0pt}{\isadigit{6}}x{\isadigit{7}}{\isacharunderscore}{\kern0pt}ul{\isacharunderscore}{\kern0pt}non{\isacharunderscore}{\kern0pt}nil{\isacharcolon}{\kern0pt}\ {\isachardoublequoteopen}kp{\isadigit{6}}x{\isadigit{7}}ul\ {\isasymnoteq}\ {\isacharbrackleft}{\kern0pt}{\isacharbrackright}{\kern0pt}{\isachardoublequoteclose}%
\isadelimproof
\ %
\endisadelimproof
%
\isatagproof
\isacommand{by}\isamarkupfalse%
\ eval%
\endisatagproof
{\isafoldproof}%
%
\isadelimproof
%
\endisadelimproof
%
\begin{isamarkuptext}%
A Knight's circuit for the \isa{{\isacharparenleft}{\kern0pt}{\isadigit{6}}{\isasymtimes}{\isadigit{7}}{\isacharparenright}{\kern0pt}}-board.
  \begin{table}[H]
    \begin{tabular}{lllllll}
      26 & 37 &  8 & 17 & 28 & 31 &  6 \\
       9 & 18 & 27 & 36 &  7 & 16 & 29 \\
      38 & 25 & 10 & 19 & 30 &  5 & 32 \\
      11 & 42 & 23 & 40 & 35 & 20 & 15 \\
      24 & 39 &  2 & 13 & 22 & 33 &  4 \\
       1 & 12 & 41 & 34 &  3 & 14 & 21
    \end{tabular}
  \end{table}%
\end{isamarkuptext}\isamarkuptrue%
\isacommand{abbreviation}\isamarkupfalse%
\ {\isachardoublequoteopen}kc{\isadigit{6}}x{\isadigit{7}}\ {\isasymequiv}\ the\ {\isacharparenleft}{\kern0pt}to{\isacharunderscore}{\kern0pt}path\ \isanewline
\ \ {\isacharbrackleft}{\kern0pt}{\isacharbrackleft}{\kern0pt}{\isadigit{2}}{\isadigit{6}}{\isacharcomma}{\kern0pt}{\isadigit{3}}{\isadigit{7}}{\isacharcomma}{\kern0pt}{\isadigit{8}}{\isacharcomma}{\kern0pt}{\isadigit{1}}{\isadigit{7}}{\isacharcomma}{\kern0pt}{\isadigit{2}}{\isadigit{8}}{\isacharcomma}{\kern0pt}{\isadigit{3}}{\isadigit{1}}{\isacharcomma}{\kern0pt}{\isadigit{6}}{\isacharbrackright}{\kern0pt}{\isacharcomma}{\kern0pt}\isanewline
\ \ {\isacharbrackleft}{\kern0pt}{\isadigit{9}}{\isacharcomma}{\kern0pt}{\isadigit{1}}{\isadigit{8}}{\isacharcomma}{\kern0pt}{\isadigit{2}}{\isadigit{7}}{\isacharcomma}{\kern0pt}{\isadigit{3}}{\isadigit{6}}{\isacharcomma}{\kern0pt}{\isadigit{7}}{\isacharcomma}{\kern0pt}{\isadigit{1}}{\isadigit{6}}{\isacharcomma}{\kern0pt}{\isadigit{2}}{\isadigit{9}}{\isacharbrackright}{\kern0pt}{\isacharcomma}{\kern0pt}\isanewline
\ \ {\isacharbrackleft}{\kern0pt}{\isadigit{3}}{\isadigit{8}}{\isacharcomma}{\kern0pt}{\isadigit{2}}{\isadigit{5}}{\isacharcomma}{\kern0pt}{\isadigit{1}}{\isadigit{0}}{\isacharcomma}{\kern0pt}{\isadigit{1}}{\isadigit{9}}{\isacharcomma}{\kern0pt}{\isadigit{3}}{\isadigit{0}}{\isacharcomma}{\kern0pt}{\isadigit{5}}{\isacharcomma}{\kern0pt}{\isadigit{3}}{\isadigit{2}}{\isacharbrackright}{\kern0pt}{\isacharcomma}{\kern0pt}\isanewline
\ \ {\isacharbrackleft}{\kern0pt}{\isadigit{1}}{\isadigit{1}}{\isacharcomma}{\kern0pt}{\isadigit{4}}{\isadigit{2}}{\isacharcomma}{\kern0pt}{\isadigit{2}}{\isadigit{3}}{\isacharcomma}{\kern0pt}{\isadigit{4}}{\isadigit{0}}{\isacharcomma}{\kern0pt}{\isadigit{3}}{\isadigit{5}}{\isacharcomma}{\kern0pt}{\isadigit{2}}{\isadigit{0}}{\isacharcomma}{\kern0pt}{\isadigit{1}}{\isadigit{5}}{\isacharbrackright}{\kern0pt}{\isacharcomma}{\kern0pt}\isanewline
\ \ {\isacharbrackleft}{\kern0pt}{\isadigit{2}}{\isadigit{4}}{\isacharcomma}{\kern0pt}{\isadigit{3}}{\isadigit{9}}{\isacharcomma}{\kern0pt}{\isadigit{2}}{\isacharcomma}{\kern0pt}{\isadigit{1}}{\isadigit{3}}{\isacharcomma}{\kern0pt}{\isadigit{2}}{\isadigit{2}}{\isacharcomma}{\kern0pt}{\isadigit{3}}{\isadigit{3}}{\isacharcomma}{\kern0pt}{\isadigit{4}}{\isacharbrackright}{\kern0pt}{\isacharcomma}{\kern0pt}\isanewline
\ \ {\isacharbrackleft}{\kern0pt}{\isadigit{1}}{\isacharcomma}{\kern0pt}{\isadigit{1}}{\isadigit{2}}{\isacharcomma}{\kern0pt}{\isadigit{4}}{\isadigit{1}}{\isacharcomma}{\kern0pt}{\isadigit{3}}{\isadigit{4}}{\isacharcomma}{\kern0pt}{\isadigit{3}}{\isacharcomma}{\kern0pt}{\isadigit{1}}{\isadigit{4}}{\isacharcomma}{\kern0pt}{\isadigit{2}}{\isadigit{1}}{\isacharbrackright}{\kern0pt}{\isacharbrackright}{\kern0pt}{\isacharparenright}{\kern0pt}{\isachardoublequoteclose}\isanewline
\isacommand{lemma}\isamarkupfalse%
\ kc{\isacharunderscore}{\kern0pt}{\isadigit{6}}x{\isadigit{7}}{\isacharcolon}{\kern0pt}\ {\isachardoublequoteopen}knights{\isacharunderscore}{\kern0pt}circuit\ b{\isadigit{6}}x{\isadigit{7}}\ kc{\isadigit{6}}x{\isadigit{7}}{\isachardoublequoteclose}\isanewline
%
\isadelimproof
\ \ %
\endisadelimproof
%
\isatagproof
\isacommand{by}\isamarkupfalse%
\ {\isacharparenleft}{\kern0pt}simp\ only{\isacharcolon}{\kern0pt}\ knights{\isacharunderscore}{\kern0pt}circuit{\isacharunderscore}{\kern0pt}exec{\isacharunderscore}{\kern0pt}simp{\isacharparenright}{\kern0pt}\ eval%
\endisatagproof
{\isafoldproof}%
%
\isadelimproof
\isanewline
%
\endisadelimproof
\isanewline
\isacommand{lemma}\isamarkupfalse%
\ kc{\isacharunderscore}{\kern0pt}{\isadigit{6}}x{\isadigit{7}}{\isacharunderscore}{\kern0pt}hd{\isacharcolon}{\kern0pt}\ {\isachardoublequoteopen}hd\ kc{\isadigit{6}}x{\isadigit{7}}\ {\isacharequal}{\kern0pt}\ {\isacharparenleft}{\kern0pt}{\isadigit{1}}{\isacharcomma}{\kern0pt}{\isadigit{1}}{\isacharparenright}{\kern0pt}{\isachardoublequoteclose}%
\isadelimproof
\ %
\endisadelimproof
%
\isatagproof
\isacommand{by}\isamarkupfalse%
\ eval%
\endisatagproof
{\isafoldproof}%
%
\isadelimproof
%
\endisadelimproof
\isanewline
\isanewline
\isacommand{lemma}\isamarkupfalse%
\ kc{\isacharunderscore}{\kern0pt}{\isadigit{6}}x{\isadigit{7}}{\isacharunderscore}{\kern0pt}non{\isacharunderscore}{\kern0pt}nil{\isacharcolon}{\kern0pt}\ {\isachardoublequoteopen}kc{\isadigit{6}}x{\isadigit{7}}\ {\isasymnoteq}\ {\isacharbrackleft}{\kern0pt}{\isacharbrackright}{\kern0pt}{\isachardoublequoteclose}%
\isadelimproof
\ %
\endisadelimproof
%
\isatagproof
\isacommand{by}\isamarkupfalse%
\ eval%
\endisatagproof
{\isafoldproof}%
%
\isadelimproof
%
\endisadelimproof
\isanewline
\isanewline
\isacommand{abbreviation}\isamarkupfalse%
\ {\isachardoublequoteopen}b{\isadigit{6}}x{\isadigit{8}}\ {\isasymequiv}\ board\ {\isadigit{6}}\ {\isadigit{8}}{\isachardoublequoteclose}%
\begin{isamarkuptext}%
A Knight's path for the \isa{{\isacharparenleft}{\kern0pt}{\isadigit{6}}{\isasymtimes}{\isadigit{8}}{\isacharparenright}{\kern0pt}}-board that starts in the lower-left and ends in the 
upper-left.
  \begin{table}[H]
    \begin{tabular}{llllllll}
      18 & 31 &  8 & 35 & 16 & 33 &  6 & 45 \\
       9 & 48 & 17 & 32 &  7 & 46 & 15 & 26 \\
      30 & 19 & 36 & 47 & 34 & 27 & 44 &  5 \\
      37 & 10 & 21 & 28 & 43 & 40 & 25 & 14 \\
      20 & 29 &  2 & 39 & 12 & 23 &  4 & 41 \\
       1 & 38 & 11 & 22 &  3 & 42 & 13 & 24
    \end{tabular}
  \end{table}%
\end{isamarkuptext}\isamarkuptrue%
\isacommand{abbreviation}\isamarkupfalse%
\ {\isachardoublequoteopen}kp{\isadigit{6}}x{\isadigit{8}}ul\ {\isasymequiv}\ the\ {\isacharparenleft}{\kern0pt}to{\isacharunderscore}{\kern0pt}path\ \isanewline
\ \ {\isacharbrackleft}{\kern0pt}{\isacharbrackleft}{\kern0pt}{\isadigit{1}}{\isadigit{8}}{\isacharcomma}{\kern0pt}{\isadigit{3}}{\isadigit{1}}{\isacharcomma}{\kern0pt}{\isadigit{8}}{\isacharcomma}{\kern0pt}{\isadigit{3}}{\isadigit{5}}{\isacharcomma}{\kern0pt}{\isadigit{1}}{\isadigit{6}}{\isacharcomma}{\kern0pt}{\isadigit{3}}{\isadigit{3}}{\isacharcomma}{\kern0pt}{\isadigit{6}}{\isacharcomma}{\kern0pt}{\isadigit{4}}{\isadigit{5}}{\isacharbrackright}{\kern0pt}{\isacharcomma}{\kern0pt}\isanewline
\ \ {\isacharbrackleft}{\kern0pt}{\isadigit{9}}{\isacharcomma}{\kern0pt}{\isadigit{4}}{\isadigit{8}}{\isacharcomma}{\kern0pt}{\isadigit{1}}{\isadigit{7}}{\isacharcomma}{\kern0pt}{\isadigit{3}}{\isadigit{2}}{\isacharcomma}{\kern0pt}{\isadigit{7}}{\isacharcomma}{\kern0pt}{\isadigit{4}}{\isadigit{6}}{\isacharcomma}{\kern0pt}{\isadigit{1}}{\isadigit{5}}{\isacharcomma}{\kern0pt}{\isadigit{2}}{\isadigit{6}}{\isacharbrackright}{\kern0pt}{\isacharcomma}{\kern0pt}\isanewline
\ \ {\isacharbrackleft}{\kern0pt}{\isadigit{3}}{\isadigit{0}}{\isacharcomma}{\kern0pt}{\isadigit{1}}{\isadigit{9}}{\isacharcomma}{\kern0pt}{\isadigit{3}}{\isadigit{6}}{\isacharcomma}{\kern0pt}{\isadigit{4}}{\isadigit{7}}{\isacharcomma}{\kern0pt}{\isadigit{3}}{\isadigit{4}}{\isacharcomma}{\kern0pt}{\isadigit{2}}{\isadigit{7}}{\isacharcomma}{\kern0pt}{\isadigit{4}}{\isadigit{4}}{\isacharcomma}{\kern0pt}{\isadigit{5}}{\isacharbrackright}{\kern0pt}{\isacharcomma}{\kern0pt}\isanewline
\ \ {\isacharbrackleft}{\kern0pt}{\isadigit{3}}{\isadigit{7}}{\isacharcomma}{\kern0pt}{\isadigit{1}}{\isadigit{0}}{\isacharcomma}{\kern0pt}{\isadigit{2}}{\isadigit{1}}{\isacharcomma}{\kern0pt}{\isadigit{2}}{\isadigit{8}}{\isacharcomma}{\kern0pt}{\isadigit{4}}{\isadigit{3}}{\isacharcomma}{\kern0pt}{\isadigit{4}}{\isadigit{0}}{\isacharcomma}{\kern0pt}{\isadigit{2}}{\isadigit{5}}{\isacharcomma}{\kern0pt}{\isadigit{1}}{\isadigit{4}}{\isacharbrackright}{\kern0pt}{\isacharcomma}{\kern0pt}\isanewline
\ \ {\isacharbrackleft}{\kern0pt}{\isadigit{2}}{\isadigit{0}}{\isacharcomma}{\kern0pt}{\isadigit{2}}{\isadigit{9}}{\isacharcomma}{\kern0pt}{\isadigit{2}}{\isacharcomma}{\kern0pt}{\isadigit{3}}{\isadigit{9}}{\isacharcomma}{\kern0pt}{\isadigit{1}}{\isadigit{2}}{\isacharcomma}{\kern0pt}{\isadigit{2}}{\isadigit{3}}{\isacharcomma}{\kern0pt}{\isadigit{4}}{\isacharcomma}{\kern0pt}{\isadigit{4}}{\isadigit{1}}{\isacharbrackright}{\kern0pt}{\isacharcomma}{\kern0pt}\isanewline
\ \ {\isacharbrackleft}{\kern0pt}{\isadigit{1}}{\isacharcomma}{\kern0pt}{\isadigit{3}}{\isadigit{8}}{\isacharcomma}{\kern0pt}{\isadigit{1}}{\isadigit{1}}{\isacharcomma}{\kern0pt}{\isadigit{2}}{\isadigit{2}}{\isacharcomma}{\kern0pt}{\isadigit{3}}{\isacharcomma}{\kern0pt}{\isadigit{4}}{\isadigit{2}}{\isacharcomma}{\kern0pt}{\isadigit{1}}{\isadigit{3}}{\isacharcomma}{\kern0pt}{\isadigit{2}}{\isadigit{4}}{\isacharbrackright}{\kern0pt}{\isacharbrackright}{\kern0pt}{\isacharparenright}{\kern0pt}{\isachardoublequoteclose}\isanewline
\isacommand{lemma}\isamarkupfalse%
\ kp{\isacharunderscore}{\kern0pt}{\isadigit{6}}x{\isadigit{8}}{\isacharunderscore}{\kern0pt}ul{\isacharcolon}{\kern0pt}\ {\isachardoublequoteopen}knights{\isacharunderscore}{\kern0pt}path\ b{\isadigit{6}}x{\isadigit{8}}\ kp{\isadigit{6}}x{\isadigit{8}}ul{\isachardoublequoteclose}\isanewline
%
\isadelimproof
\ \ %
\endisadelimproof
%
\isatagproof
\isacommand{by}\isamarkupfalse%
\ {\isacharparenleft}{\kern0pt}simp\ only{\isacharcolon}{\kern0pt}\ knights{\isacharunderscore}{\kern0pt}path{\isacharunderscore}{\kern0pt}exec{\isacharunderscore}{\kern0pt}simp{\isacharparenright}{\kern0pt}\ eval%
\endisatagproof
{\isafoldproof}%
%
\isadelimproof
\isanewline
%
\endisadelimproof
\isanewline
\isacommand{lemma}\isamarkupfalse%
\ kp{\isacharunderscore}{\kern0pt}{\isadigit{6}}x{\isadigit{8}}{\isacharunderscore}{\kern0pt}ul{\isacharunderscore}{\kern0pt}hd{\isacharcolon}{\kern0pt}\ {\isachardoublequoteopen}hd\ kp{\isadigit{6}}x{\isadigit{8}}ul\ {\isacharequal}{\kern0pt}\ {\isacharparenleft}{\kern0pt}{\isadigit{1}}{\isacharcomma}{\kern0pt}{\isadigit{1}}{\isacharparenright}{\kern0pt}{\isachardoublequoteclose}%
\isadelimproof
\ %
\endisadelimproof
%
\isatagproof
\isacommand{by}\isamarkupfalse%
\ eval%
\endisatagproof
{\isafoldproof}%
%
\isadelimproof
%
\endisadelimproof
\isanewline
\isanewline
\isacommand{lemma}\isamarkupfalse%
\ kp{\isacharunderscore}{\kern0pt}{\isadigit{6}}x{\isadigit{8}}{\isacharunderscore}{\kern0pt}ul{\isacharunderscore}{\kern0pt}last{\isacharcolon}{\kern0pt}\ {\isachardoublequoteopen}last\ kp{\isadigit{6}}x{\isadigit{8}}ul\ {\isacharequal}{\kern0pt}\ {\isacharparenleft}{\kern0pt}{\isadigit{5}}{\isacharcomma}{\kern0pt}{\isadigit{2}}{\isacharparenright}{\kern0pt}{\isachardoublequoteclose}%
\isadelimproof
\ %
\endisadelimproof
%
\isatagproof
\isacommand{by}\isamarkupfalse%
\ eval%
\endisatagproof
{\isafoldproof}%
%
\isadelimproof
%
\endisadelimproof
\isanewline
\isanewline
\isacommand{lemma}\isamarkupfalse%
\ kp{\isacharunderscore}{\kern0pt}{\isadigit{6}}x{\isadigit{8}}{\isacharunderscore}{\kern0pt}ul{\isacharunderscore}{\kern0pt}non{\isacharunderscore}{\kern0pt}nil{\isacharcolon}{\kern0pt}\ {\isachardoublequoteopen}kp{\isadigit{6}}x{\isadigit{8}}ul\ {\isasymnoteq}\ {\isacharbrackleft}{\kern0pt}{\isacharbrackright}{\kern0pt}{\isachardoublequoteclose}%
\isadelimproof
\ %
\endisadelimproof
%
\isatagproof
\isacommand{by}\isamarkupfalse%
\ eval%
\endisatagproof
{\isafoldproof}%
%
\isadelimproof
%
\endisadelimproof
%
\begin{isamarkuptext}%
A Knight's circuit for the \isa{{\isacharparenleft}{\kern0pt}{\isadigit{6}}{\isasymtimes}{\isadigit{8}}{\isacharparenright}{\kern0pt}}-board.
  \begin{table}[H]
    \begin{tabular}{llllllll}
      30 & 35 &  8 & 15 & 28 & 39 &  6 & 13 \\
       9 & 16 & 29 & 36 &  7 & 14 & 27 & 38 \\
      34 & 31 & 10 & 23 & 40 & 37 & 12 &  5 \\
      17 & 48 & 33 & 46 & 11 & 22 & 41 & 26 \\
      32 & 45 &  2 & 19 & 24 & 43 &  4 & 21 \\
       1 & 18 & 47 & 44 &  3 & 20 & 25 & 42
    \end{tabular}
  \end{table}%
\end{isamarkuptext}\isamarkuptrue%
\isacommand{abbreviation}\isamarkupfalse%
\ {\isachardoublequoteopen}kc{\isadigit{6}}x{\isadigit{8}}\ {\isasymequiv}\ the\ {\isacharparenleft}{\kern0pt}to{\isacharunderscore}{\kern0pt}path\ \isanewline
\ \ {\isacharbrackleft}{\kern0pt}{\isacharbrackleft}{\kern0pt}{\isadigit{3}}{\isadigit{0}}{\isacharcomma}{\kern0pt}{\isadigit{3}}{\isadigit{5}}{\isacharcomma}{\kern0pt}{\isadigit{8}}{\isacharcomma}{\kern0pt}{\isadigit{1}}{\isadigit{5}}{\isacharcomma}{\kern0pt}{\isadigit{2}}{\isadigit{8}}{\isacharcomma}{\kern0pt}{\isadigit{3}}{\isadigit{9}}{\isacharcomma}{\kern0pt}{\isadigit{6}}{\isacharcomma}{\kern0pt}{\isadigit{1}}{\isadigit{3}}{\isacharbrackright}{\kern0pt}{\isacharcomma}{\kern0pt}\isanewline
\ \ {\isacharbrackleft}{\kern0pt}{\isadigit{9}}{\isacharcomma}{\kern0pt}{\isadigit{1}}{\isadigit{6}}{\isacharcomma}{\kern0pt}{\isadigit{2}}{\isadigit{9}}{\isacharcomma}{\kern0pt}{\isadigit{3}}{\isadigit{6}}{\isacharcomma}{\kern0pt}{\isadigit{7}}{\isacharcomma}{\kern0pt}{\isadigit{1}}{\isadigit{4}}{\isacharcomma}{\kern0pt}{\isadigit{2}}{\isadigit{7}}{\isacharcomma}{\kern0pt}{\isadigit{3}}{\isadigit{8}}{\isacharbrackright}{\kern0pt}{\isacharcomma}{\kern0pt}\isanewline
\ \ {\isacharbrackleft}{\kern0pt}{\isadigit{3}}{\isadigit{4}}{\isacharcomma}{\kern0pt}{\isadigit{3}}{\isadigit{1}}{\isacharcomma}{\kern0pt}{\isadigit{1}}{\isadigit{0}}{\isacharcomma}{\kern0pt}{\isadigit{2}}{\isadigit{3}}{\isacharcomma}{\kern0pt}{\isadigit{4}}{\isadigit{0}}{\isacharcomma}{\kern0pt}{\isadigit{3}}{\isadigit{7}}{\isacharcomma}{\kern0pt}{\isadigit{1}}{\isadigit{2}}{\isacharcomma}{\kern0pt}{\isadigit{5}}{\isacharbrackright}{\kern0pt}{\isacharcomma}{\kern0pt}\isanewline
\ \ {\isacharbrackleft}{\kern0pt}{\isadigit{1}}{\isadigit{7}}{\isacharcomma}{\kern0pt}{\isadigit{4}}{\isadigit{8}}{\isacharcomma}{\kern0pt}{\isadigit{3}}{\isadigit{3}}{\isacharcomma}{\kern0pt}{\isadigit{4}}{\isadigit{6}}{\isacharcomma}{\kern0pt}{\isadigit{1}}{\isadigit{1}}{\isacharcomma}{\kern0pt}{\isadigit{2}}{\isadigit{2}}{\isacharcomma}{\kern0pt}{\isadigit{4}}{\isadigit{1}}{\isacharcomma}{\kern0pt}{\isadigit{2}}{\isadigit{6}}{\isacharbrackright}{\kern0pt}{\isacharcomma}{\kern0pt}\isanewline
\ \ {\isacharbrackleft}{\kern0pt}{\isadigit{3}}{\isadigit{2}}{\isacharcomma}{\kern0pt}{\isadigit{4}}{\isadigit{5}}{\isacharcomma}{\kern0pt}{\isadigit{2}}{\isacharcomma}{\kern0pt}{\isadigit{1}}{\isadigit{9}}{\isacharcomma}{\kern0pt}{\isadigit{2}}{\isadigit{4}}{\isacharcomma}{\kern0pt}{\isadigit{4}}{\isadigit{3}}{\isacharcomma}{\kern0pt}{\isadigit{4}}{\isacharcomma}{\kern0pt}{\isadigit{2}}{\isadigit{1}}{\isacharbrackright}{\kern0pt}{\isacharcomma}{\kern0pt}\isanewline
\ \ {\isacharbrackleft}{\kern0pt}{\isadigit{1}}{\isacharcomma}{\kern0pt}{\isadigit{1}}{\isadigit{8}}{\isacharcomma}{\kern0pt}{\isadigit{4}}{\isadigit{7}}{\isacharcomma}{\kern0pt}{\isadigit{4}}{\isadigit{4}}{\isacharcomma}{\kern0pt}{\isadigit{3}}{\isacharcomma}{\kern0pt}{\isadigit{2}}{\isadigit{0}}{\isacharcomma}{\kern0pt}{\isadigit{2}}{\isadigit{5}}{\isacharcomma}{\kern0pt}{\isadigit{4}}{\isadigit{2}}{\isacharbrackright}{\kern0pt}{\isacharbrackright}{\kern0pt}{\isacharparenright}{\kern0pt}{\isachardoublequoteclose}\isanewline
\isacommand{lemma}\isamarkupfalse%
\ kc{\isacharunderscore}{\kern0pt}{\isadigit{6}}x{\isadigit{8}}{\isacharcolon}{\kern0pt}\ {\isachardoublequoteopen}knights{\isacharunderscore}{\kern0pt}circuit\ b{\isadigit{6}}x{\isadigit{8}}\ kc{\isadigit{6}}x{\isadigit{8}}{\isachardoublequoteclose}\isanewline
%
\isadelimproof
\ \ %
\endisadelimproof
%
\isatagproof
\isacommand{by}\isamarkupfalse%
\ {\isacharparenleft}{\kern0pt}simp\ only{\isacharcolon}{\kern0pt}\ knights{\isacharunderscore}{\kern0pt}circuit{\isacharunderscore}{\kern0pt}exec{\isacharunderscore}{\kern0pt}simp{\isacharparenright}{\kern0pt}\ eval%
\endisatagproof
{\isafoldproof}%
%
\isadelimproof
\isanewline
%
\endisadelimproof
\isanewline
\isacommand{lemma}\isamarkupfalse%
\ kc{\isacharunderscore}{\kern0pt}{\isadigit{6}}x{\isadigit{8}}{\isacharunderscore}{\kern0pt}hd{\isacharcolon}{\kern0pt}\ {\isachardoublequoteopen}hd\ kc{\isadigit{6}}x{\isadigit{8}}\ {\isacharequal}{\kern0pt}\ {\isacharparenleft}{\kern0pt}{\isadigit{1}}{\isacharcomma}{\kern0pt}{\isadigit{1}}{\isacharparenright}{\kern0pt}{\isachardoublequoteclose}%
\isadelimproof
\ %
\endisadelimproof
%
\isatagproof
\isacommand{by}\isamarkupfalse%
\ eval%
\endisatagproof
{\isafoldproof}%
%
\isadelimproof
%
\endisadelimproof
\isanewline
\isanewline
\isacommand{lemma}\isamarkupfalse%
\ kc{\isacharunderscore}{\kern0pt}{\isadigit{6}}x{\isadigit{8}}{\isacharunderscore}{\kern0pt}non{\isacharunderscore}{\kern0pt}nil{\isacharcolon}{\kern0pt}\ {\isachardoublequoteopen}kc{\isadigit{6}}x{\isadigit{8}}\ {\isasymnoteq}\ {\isacharbrackleft}{\kern0pt}{\isacharbrackright}{\kern0pt}{\isachardoublequoteclose}%
\isadelimproof
\ %
\endisadelimproof
%
\isatagproof
\isacommand{by}\isamarkupfalse%
\ eval%
\endisatagproof
{\isafoldproof}%
%
\isadelimproof
%
\endisadelimproof
\isanewline
\isanewline
\isacommand{abbreviation}\isamarkupfalse%
\ {\isachardoublequoteopen}b{\isadigit{6}}x{\isadigit{9}}\ {\isasymequiv}\ board\ {\isadigit{6}}\ {\isadigit{9}}{\isachardoublequoteclose}%
\begin{isamarkuptext}%
A Knight's path for the \isa{{\isacharparenleft}{\kern0pt}{\isadigit{6}}{\isasymtimes}{\isadigit{9}}{\isacharparenright}{\kern0pt}}-board that starts in the lower-left and ends in the 
upper-left.
  \begin{table}[H]
    \begin{tabular}{lllllllll}
      22 & 45 & 10 & 53 & 20 & 47 &  8 & 35 & 18 \\
      11 & 54 & 21 & 46 &  9 & 36 & 19 & 48 &  7 \\
      44 & 23 & 42 & 37 & 52 & 49 & 32 & 17 & 34 \\
      41 & 12 & 25 & 50 & 27 & 38 & 29 &  6 & 31 \\
      24 & 43 &  2 & 39 & 14 & 51 &  4 & 33 & 16 \\
       1 & 40 & 13 & 26 &  3 & 28 & 15 & 30 &  5
    \end{tabular}
  \end{table}%
\end{isamarkuptext}\isamarkuptrue%
\isacommand{abbreviation}\isamarkupfalse%
\ {\isachardoublequoteopen}kp{\isadigit{6}}x{\isadigit{9}}ul\ {\isasymequiv}\ the\ {\isacharparenleft}{\kern0pt}to{\isacharunderscore}{\kern0pt}path\ \isanewline
\ \ {\isacharbrackleft}{\kern0pt}{\isacharbrackleft}{\kern0pt}{\isadigit{2}}{\isadigit{2}}{\isacharcomma}{\kern0pt}{\isadigit{4}}{\isadigit{5}}{\isacharcomma}{\kern0pt}{\isadigit{1}}{\isadigit{0}}{\isacharcomma}{\kern0pt}{\isadigit{5}}{\isadigit{3}}{\isacharcomma}{\kern0pt}{\isadigit{2}}{\isadigit{0}}{\isacharcomma}{\kern0pt}{\isadigit{4}}{\isadigit{7}}{\isacharcomma}{\kern0pt}{\isadigit{8}}{\isacharcomma}{\kern0pt}{\isadigit{3}}{\isadigit{5}}{\isacharcomma}{\kern0pt}{\isadigit{1}}{\isadigit{8}}{\isacharbrackright}{\kern0pt}{\isacharcomma}{\kern0pt}\isanewline
\ \ {\isacharbrackleft}{\kern0pt}{\isadigit{1}}{\isadigit{1}}{\isacharcomma}{\kern0pt}{\isadigit{5}}{\isadigit{4}}{\isacharcomma}{\kern0pt}{\isadigit{2}}{\isadigit{1}}{\isacharcomma}{\kern0pt}{\isadigit{4}}{\isadigit{6}}{\isacharcomma}{\kern0pt}{\isadigit{9}}{\isacharcomma}{\kern0pt}{\isadigit{3}}{\isadigit{6}}{\isacharcomma}{\kern0pt}{\isadigit{1}}{\isadigit{9}}{\isacharcomma}{\kern0pt}{\isadigit{4}}{\isadigit{8}}{\isacharcomma}{\kern0pt}{\isadigit{7}}{\isacharbrackright}{\kern0pt}{\isacharcomma}{\kern0pt}\isanewline
\ \ {\isacharbrackleft}{\kern0pt}{\isadigit{4}}{\isadigit{4}}{\isacharcomma}{\kern0pt}{\isadigit{2}}{\isadigit{3}}{\isacharcomma}{\kern0pt}{\isadigit{4}}{\isadigit{2}}{\isacharcomma}{\kern0pt}{\isadigit{3}}{\isadigit{7}}{\isacharcomma}{\kern0pt}{\isadigit{5}}{\isadigit{2}}{\isacharcomma}{\kern0pt}{\isadigit{4}}{\isadigit{9}}{\isacharcomma}{\kern0pt}{\isadigit{3}}{\isadigit{2}}{\isacharcomma}{\kern0pt}{\isadigit{1}}{\isadigit{7}}{\isacharcomma}{\kern0pt}{\isadigit{3}}{\isadigit{4}}{\isacharbrackright}{\kern0pt}{\isacharcomma}{\kern0pt}\isanewline
\ \ {\isacharbrackleft}{\kern0pt}{\isadigit{4}}{\isadigit{1}}{\isacharcomma}{\kern0pt}{\isadigit{1}}{\isadigit{2}}{\isacharcomma}{\kern0pt}{\isadigit{2}}{\isadigit{5}}{\isacharcomma}{\kern0pt}{\isadigit{5}}{\isadigit{0}}{\isacharcomma}{\kern0pt}{\isadigit{2}}{\isadigit{7}}{\isacharcomma}{\kern0pt}{\isadigit{3}}{\isadigit{8}}{\isacharcomma}{\kern0pt}{\isadigit{2}}{\isadigit{9}}{\isacharcomma}{\kern0pt}{\isadigit{6}}{\isacharcomma}{\kern0pt}{\isadigit{3}}{\isadigit{1}}{\isacharbrackright}{\kern0pt}{\isacharcomma}{\kern0pt}\isanewline
\ \ {\isacharbrackleft}{\kern0pt}{\isadigit{2}}{\isadigit{4}}{\isacharcomma}{\kern0pt}{\isadigit{4}}{\isadigit{3}}{\isacharcomma}{\kern0pt}{\isadigit{2}}{\isacharcomma}{\kern0pt}{\isadigit{3}}{\isadigit{9}}{\isacharcomma}{\kern0pt}{\isadigit{1}}{\isadigit{4}}{\isacharcomma}{\kern0pt}{\isadigit{5}}{\isadigit{1}}{\isacharcomma}{\kern0pt}{\isadigit{4}}{\isacharcomma}{\kern0pt}{\isadigit{3}}{\isadigit{3}}{\isacharcomma}{\kern0pt}{\isadigit{1}}{\isadigit{6}}{\isacharbrackright}{\kern0pt}{\isacharcomma}{\kern0pt}\isanewline
\ \ {\isacharbrackleft}{\kern0pt}{\isadigit{1}}{\isacharcomma}{\kern0pt}{\isadigit{4}}{\isadigit{0}}{\isacharcomma}{\kern0pt}{\isadigit{1}}{\isadigit{3}}{\isacharcomma}{\kern0pt}{\isadigit{2}}{\isadigit{6}}{\isacharcomma}{\kern0pt}{\isadigit{3}}{\isacharcomma}{\kern0pt}{\isadigit{2}}{\isadigit{8}}{\isacharcomma}{\kern0pt}{\isadigit{1}}{\isadigit{5}}{\isacharcomma}{\kern0pt}{\isadigit{3}}{\isadigit{0}}{\isacharcomma}{\kern0pt}{\isadigit{5}}{\isacharbrackright}{\kern0pt}{\isacharbrackright}{\kern0pt}{\isacharparenright}{\kern0pt}{\isachardoublequoteclose}\isanewline
\isacommand{lemma}\isamarkupfalse%
\ kp{\isacharunderscore}{\kern0pt}{\isadigit{6}}x{\isadigit{9}}{\isacharunderscore}{\kern0pt}ul{\isacharcolon}{\kern0pt}\ {\isachardoublequoteopen}knights{\isacharunderscore}{\kern0pt}path\ b{\isadigit{6}}x{\isadigit{9}}\ kp{\isadigit{6}}x{\isadigit{9}}ul{\isachardoublequoteclose}\isanewline
%
\isadelimproof
\ \ %
\endisadelimproof
%
\isatagproof
\isacommand{by}\isamarkupfalse%
\ {\isacharparenleft}{\kern0pt}simp\ only{\isacharcolon}{\kern0pt}\ knights{\isacharunderscore}{\kern0pt}path{\isacharunderscore}{\kern0pt}exec{\isacharunderscore}{\kern0pt}simp{\isacharparenright}{\kern0pt}\ eval%
\endisatagproof
{\isafoldproof}%
%
\isadelimproof
\isanewline
%
\endisadelimproof
\isanewline
\isacommand{lemma}\isamarkupfalse%
\ kp{\isacharunderscore}{\kern0pt}{\isadigit{6}}x{\isadigit{9}}{\isacharunderscore}{\kern0pt}ul{\isacharunderscore}{\kern0pt}hd{\isacharcolon}{\kern0pt}\ {\isachardoublequoteopen}hd\ kp{\isadigit{6}}x{\isadigit{9}}ul\ {\isacharequal}{\kern0pt}\ {\isacharparenleft}{\kern0pt}{\isadigit{1}}{\isacharcomma}{\kern0pt}{\isadigit{1}}{\isacharparenright}{\kern0pt}{\isachardoublequoteclose}%
\isadelimproof
\ %
\endisadelimproof
%
\isatagproof
\isacommand{by}\isamarkupfalse%
\ eval%
\endisatagproof
{\isafoldproof}%
%
\isadelimproof
%
\endisadelimproof
\isanewline
\isanewline
\isacommand{lemma}\isamarkupfalse%
\ kp{\isacharunderscore}{\kern0pt}{\isadigit{6}}x{\isadigit{9}}{\isacharunderscore}{\kern0pt}ul{\isacharunderscore}{\kern0pt}last{\isacharcolon}{\kern0pt}\ {\isachardoublequoteopen}last\ kp{\isadigit{6}}x{\isadigit{9}}ul\ {\isacharequal}{\kern0pt}\ {\isacharparenleft}{\kern0pt}{\isadigit{5}}{\isacharcomma}{\kern0pt}{\isadigit{2}}{\isacharparenright}{\kern0pt}{\isachardoublequoteclose}%
\isadelimproof
\ %
\endisadelimproof
%
\isatagproof
\isacommand{by}\isamarkupfalse%
\ eval%
\endisatagproof
{\isafoldproof}%
%
\isadelimproof
%
\endisadelimproof
\isanewline
\isanewline
\isacommand{lemma}\isamarkupfalse%
\ kp{\isacharunderscore}{\kern0pt}{\isadigit{6}}x{\isadigit{9}}{\isacharunderscore}{\kern0pt}ul{\isacharunderscore}{\kern0pt}non{\isacharunderscore}{\kern0pt}nil{\isacharcolon}{\kern0pt}\ {\isachardoublequoteopen}kp{\isadigit{6}}x{\isadigit{9}}ul\ {\isasymnoteq}\ {\isacharbrackleft}{\kern0pt}{\isacharbrackright}{\kern0pt}{\isachardoublequoteclose}%
\isadelimproof
\ %
\endisadelimproof
%
\isatagproof
\isacommand{by}\isamarkupfalse%
\ eval%
\endisatagproof
{\isafoldproof}%
%
\isadelimproof
%
\endisadelimproof
%
\begin{isamarkuptext}%
A Knight's circuit for the \isa{{\isacharparenleft}{\kern0pt}{\isadigit{6}}{\isasymtimes}{\isadigit{9}}{\isacharparenright}{\kern0pt}}-board.
  \begin{table}[H]
    \begin{tabular}{lllllllll}
      14 & 49 &  4 & 51 & 24 & 39 &  6 & 29 & 22 \\
       3 & 52 & 13 & 40 &  5 & 32 & 23 & 42 &  7 \\
      48 & 15 & 50 & 25 & 38 & 41 & 28 & 21 & 30 \\
      53 &  2 & 37 & 12 & 33 & 26 & 31 &  8 & 43 \\
      16 & 47 & 54 & 35 & 18 & 45 & 10 & 27 & 20 \\
       1 & 36 & 17 & 46 & 11 & 34 & 19 & 44 &  9
    \end{tabular}
  \end{table}%
\end{isamarkuptext}\isamarkuptrue%
\isacommand{abbreviation}\isamarkupfalse%
\ {\isachardoublequoteopen}kc{\isadigit{6}}x{\isadigit{9}}\ {\isasymequiv}\ the\ {\isacharparenleft}{\kern0pt}to{\isacharunderscore}{\kern0pt}path\ \isanewline
\ \ {\isacharbrackleft}{\kern0pt}{\isacharbrackleft}{\kern0pt}{\isadigit{1}}{\isadigit{4}}{\isacharcomma}{\kern0pt}{\isadigit{4}}{\isadigit{9}}{\isacharcomma}{\kern0pt}{\isadigit{4}}{\isacharcomma}{\kern0pt}{\isadigit{5}}{\isadigit{1}}{\isacharcomma}{\kern0pt}{\isadigit{2}}{\isadigit{4}}{\isacharcomma}{\kern0pt}{\isadigit{3}}{\isadigit{9}}{\isacharcomma}{\kern0pt}{\isadigit{6}}{\isacharcomma}{\kern0pt}{\isadigit{2}}{\isadigit{9}}{\isacharcomma}{\kern0pt}{\isadigit{2}}{\isadigit{2}}{\isacharbrackright}{\kern0pt}{\isacharcomma}{\kern0pt}\isanewline
\ \ {\isacharbrackleft}{\kern0pt}{\isadigit{3}}{\isacharcomma}{\kern0pt}{\isadigit{5}}{\isadigit{2}}{\isacharcomma}{\kern0pt}{\isadigit{1}}{\isadigit{3}}{\isacharcomma}{\kern0pt}{\isadigit{4}}{\isadigit{0}}{\isacharcomma}{\kern0pt}{\isadigit{5}}{\isacharcomma}{\kern0pt}{\isadigit{3}}{\isadigit{2}}{\isacharcomma}{\kern0pt}{\isadigit{2}}{\isadigit{3}}{\isacharcomma}{\kern0pt}{\isadigit{4}}{\isadigit{2}}{\isacharcomma}{\kern0pt}{\isadigit{7}}{\isacharbrackright}{\kern0pt}{\isacharcomma}{\kern0pt}\isanewline
\ \ {\isacharbrackleft}{\kern0pt}{\isadigit{4}}{\isadigit{8}}{\isacharcomma}{\kern0pt}{\isadigit{1}}{\isadigit{5}}{\isacharcomma}{\kern0pt}{\isadigit{5}}{\isadigit{0}}{\isacharcomma}{\kern0pt}{\isadigit{2}}{\isadigit{5}}{\isacharcomma}{\kern0pt}{\isadigit{3}}{\isadigit{8}}{\isacharcomma}{\kern0pt}{\isadigit{4}}{\isadigit{1}}{\isacharcomma}{\kern0pt}{\isadigit{2}}{\isadigit{8}}{\isacharcomma}{\kern0pt}{\isadigit{2}}{\isadigit{1}}{\isacharcomma}{\kern0pt}{\isadigit{3}}{\isadigit{0}}{\isacharbrackright}{\kern0pt}{\isacharcomma}{\kern0pt}\isanewline
\ \ {\isacharbrackleft}{\kern0pt}{\isadigit{5}}{\isadigit{3}}{\isacharcomma}{\kern0pt}{\isadigit{2}}{\isacharcomma}{\kern0pt}{\isadigit{3}}{\isadigit{7}}{\isacharcomma}{\kern0pt}{\isadigit{1}}{\isadigit{2}}{\isacharcomma}{\kern0pt}{\isadigit{3}}{\isadigit{3}}{\isacharcomma}{\kern0pt}{\isadigit{2}}{\isadigit{6}}{\isacharcomma}{\kern0pt}{\isadigit{3}}{\isadigit{1}}{\isacharcomma}{\kern0pt}{\isadigit{8}}{\isacharcomma}{\kern0pt}{\isadigit{4}}{\isadigit{3}}{\isacharbrackright}{\kern0pt}{\isacharcomma}{\kern0pt}\isanewline
\ \ {\isacharbrackleft}{\kern0pt}{\isadigit{1}}{\isadigit{6}}{\isacharcomma}{\kern0pt}{\isadigit{4}}{\isadigit{7}}{\isacharcomma}{\kern0pt}{\isadigit{5}}{\isadigit{4}}{\isacharcomma}{\kern0pt}{\isadigit{3}}{\isadigit{5}}{\isacharcomma}{\kern0pt}{\isadigit{1}}{\isadigit{8}}{\isacharcomma}{\kern0pt}{\isadigit{4}}{\isadigit{5}}{\isacharcomma}{\kern0pt}{\isadigit{1}}{\isadigit{0}}{\isacharcomma}{\kern0pt}{\isadigit{2}}{\isadigit{7}}{\isacharcomma}{\kern0pt}{\isadigit{2}}{\isadigit{0}}{\isacharbrackright}{\kern0pt}{\isacharcomma}{\kern0pt}\isanewline
\ \ {\isacharbrackleft}{\kern0pt}{\isadigit{1}}{\isacharcomma}{\kern0pt}{\isadigit{3}}{\isadigit{6}}{\isacharcomma}{\kern0pt}{\isadigit{1}}{\isadigit{7}}{\isacharcomma}{\kern0pt}{\isadigit{4}}{\isadigit{6}}{\isacharcomma}{\kern0pt}{\isadigit{1}}{\isadigit{1}}{\isacharcomma}{\kern0pt}{\isadigit{3}}{\isadigit{4}}{\isacharcomma}{\kern0pt}{\isadigit{1}}{\isadigit{9}}{\isacharcomma}{\kern0pt}{\isadigit{4}}{\isadigit{4}}{\isacharcomma}{\kern0pt}{\isadigit{9}}{\isacharbrackright}{\kern0pt}{\isacharbrackright}{\kern0pt}{\isacharparenright}{\kern0pt}{\isachardoublequoteclose}\isanewline
\isacommand{lemma}\isamarkupfalse%
\ kc{\isacharunderscore}{\kern0pt}{\isadigit{6}}x{\isadigit{9}}{\isacharcolon}{\kern0pt}\ {\isachardoublequoteopen}knights{\isacharunderscore}{\kern0pt}circuit\ b{\isadigit{6}}x{\isadigit{9}}\ kc{\isadigit{6}}x{\isadigit{9}}{\isachardoublequoteclose}\isanewline
%
\isadelimproof
\ \ %
\endisadelimproof
%
\isatagproof
\isacommand{by}\isamarkupfalse%
\ {\isacharparenleft}{\kern0pt}simp\ only{\isacharcolon}{\kern0pt}\ knights{\isacharunderscore}{\kern0pt}circuit{\isacharunderscore}{\kern0pt}exec{\isacharunderscore}{\kern0pt}simp{\isacharparenright}{\kern0pt}\ eval%
\endisatagproof
{\isafoldproof}%
%
\isadelimproof
\isanewline
%
\endisadelimproof
\isanewline
\isacommand{lemma}\isamarkupfalse%
\ kc{\isacharunderscore}{\kern0pt}{\isadigit{6}}x{\isadigit{9}}{\isacharunderscore}{\kern0pt}hd{\isacharcolon}{\kern0pt}\ {\isachardoublequoteopen}hd\ kc{\isadigit{6}}x{\isadigit{9}}\ {\isacharequal}{\kern0pt}\ {\isacharparenleft}{\kern0pt}{\isadigit{1}}{\isacharcomma}{\kern0pt}{\isadigit{1}}{\isacharparenright}{\kern0pt}{\isachardoublequoteclose}%
\isadelimproof
\ %
\endisadelimproof
%
\isatagproof
\isacommand{by}\isamarkupfalse%
\ eval%
\endisatagproof
{\isafoldproof}%
%
\isadelimproof
%
\endisadelimproof
\isanewline
\isanewline
\isacommand{lemma}\isamarkupfalse%
\ kc{\isacharunderscore}{\kern0pt}{\isadigit{6}}x{\isadigit{9}}{\isacharunderscore}{\kern0pt}non{\isacharunderscore}{\kern0pt}nil{\isacharcolon}{\kern0pt}\ {\isachardoublequoteopen}kc{\isadigit{6}}x{\isadigit{9}}\ {\isasymnoteq}\ {\isacharbrackleft}{\kern0pt}{\isacharbrackright}{\kern0pt}{\isachardoublequoteclose}%
\isadelimproof
\ %
\endisadelimproof
%
\isatagproof
\isacommand{by}\isamarkupfalse%
\ eval%
\endisatagproof
{\isafoldproof}%
%
\isadelimproof
%
\endisadelimproof
\isanewline
\isanewline
\isacommand{lemmas}\isamarkupfalse%
\ kp{\isacharunderscore}{\kern0pt}{\isadigit{6}}xm{\isacharunderscore}{\kern0pt}ul\ {\isacharequal}{\kern0pt}\ \isanewline
\ \ kp{\isacharunderscore}{\kern0pt}{\isadigit{6}}x{\isadigit{5}}{\isacharunderscore}{\kern0pt}ul\ kp{\isacharunderscore}{\kern0pt}{\isadigit{6}}x{\isadigit{5}}{\isacharunderscore}{\kern0pt}ul{\isacharunderscore}{\kern0pt}hd\ kp{\isacharunderscore}{\kern0pt}{\isadigit{6}}x{\isadigit{5}}{\isacharunderscore}{\kern0pt}ul{\isacharunderscore}{\kern0pt}last\ kp{\isacharunderscore}{\kern0pt}{\isadigit{6}}x{\isadigit{5}}{\isacharunderscore}{\kern0pt}ul{\isacharunderscore}{\kern0pt}non{\isacharunderscore}{\kern0pt}nil\isanewline
\ \ kp{\isacharunderscore}{\kern0pt}{\isadigit{6}}x{\isadigit{6}}{\isacharunderscore}{\kern0pt}ul\ kp{\isacharunderscore}{\kern0pt}{\isadigit{6}}x{\isadigit{6}}{\isacharunderscore}{\kern0pt}ul{\isacharunderscore}{\kern0pt}hd\ kp{\isacharunderscore}{\kern0pt}{\isadigit{6}}x{\isadigit{6}}{\isacharunderscore}{\kern0pt}ul{\isacharunderscore}{\kern0pt}last\ kp{\isacharunderscore}{\kern0pt}{\isadigit{6}}x{\isadigit{6}}{\isacharunderscore}{\kern0pt}ul{\isacharunderscore}{\kern0pt}non{\isacharunderscore}{\kern0pt}nil\isanewline
\ \ kp{\isacharunderscore}{\kern0pt}{\isadigit{6}}x{\isadigit{7}}{\isacharunderscore}{\kern0pt}ul\ kp{\isacharunderscore}{\kern0pt}{\isadigit{6}}x{\isadigit{7}}{\isacharunderscore}{\kern0pt}ul{\isacharunderscore}{\kern0pt}hd\ kp{\isacharunderscore}{\kern0pt}{\isadigit{6}}x{\isadigit{7}}{\isacharunderscore}{\kern0pt}ul{\isacharunderscore}{\kern0pt}last\ kp{\isacharunderscore}{\kern0pt}{\isadigit{6}}x{\isadigit{7}}{\isacharunderscore}{\kern0pt}ul{\isacharunderscore}{\kern0pt}non{\isacharunderscore}{\kern0pt}nil\isanewline
\ \ kp{\isacharunderscore}{\kern0pt}{\isadigit{6}}x{\isadigit{8}}{\isacharunderscore}{\kern0pt}ul\ kp{\isacharunderscore}{\kern0pt}{\isadigit{6}}x{\isadigit{8}}{\isacharunderscore}{\kern0pt}ul{\isacharunderscore}{\kern0pt}hd\ kp{\isacharunderscore}{\kern0pt}{\isadigit{6}}x{\isadigit{8}}{\isacharunderscore}{\kern0pt}ul{\isacharunderscore}{\kern0pt}last\ kp{\isacharunderscore}{\kern0pt}{\isadigit{6}}x{\isadigit{8}}{\isacharunderscore}{\kern0pt}ul{\isacharunderscore}{\kern0pt}non{\isacharunderscore}{\kern0pt}nil\isanewline
\ \ kp{\isacharunderscore}{\kern0pt}{\isadigit{6}}x{\isadigit{9}}{\isacharunderscore}{\kern0pt}ul\ kp{\isacharunderscore}{\kern0pt}{\isadigit{6}}x{\isadigit{9}}{\isacharunderscore}{\kern0pt}ul{\isacharunderscore}{\kern0pt}hd\ kp{\isacharunderscore}{\kern0pt}{\isadigit{6}}x{\isadigit{9}}{\isacharunderscore}{\kern0pt}ul{\isacharunderscore}{\kern0pt}last\ kp{\isacharunderscore}{\kern0pt}{\isadigit{6}}x{\isadigit{9}}{\isacharunderscore}{\kern0pt}ul{\isacharunderscore}{\kern0pt}non{\isacharunderscore}{\kern0pt}nil\isanewline
\isanewline
\isacommand{lemmas}\isamarkupfalse%
\ kc{\isacharunderscore}{\kern0pt}{\isadigit{6}}xm\ {\isacharequal}{\kern0pt}\ \isanewline
\ \ kc{\isacharunderscore}{\kern0pt}{\isadigit{6}}x{\isadigit{5}}\ kc{\isacharunderscore}{\kern0pt}{\isadigit{6}}x{\isadigit{5}}{\isacharunderscore}{\kern0pt}hd\ kc{\isacharunderscore}{\kern0pt}{\isadigit{6}}x{\isadigit{5}}{\isacharunderscore}{\kern0pt}non{\isacharunderscore}{\kern0pt}nil\isanewline
\ \ kc{\isacharunderscore}{\kern0pt}{\isadigit{6}}x{\isadigit{6}}\ kc{\isacharunderscore}{\kern0pt}{\isadigit{6}}x{\isadigit{6}}{\isacharunderscore}{\kern0pt}hd\ kc{\isacharunderscore}{\kern0pt}{\isadigit{6}}x{\isadigit{6}}{\isacharunderscore}{\kern0pt}non{\isacharunderscore}{\kern0pt}nil\isanewline
\ \ kc{\isacharunderscore}{\kern0pt}{\isadigit{6}}x{\isadigit{7}}\ kc{\isacharunderscore}{\kern0pt}{\isadigit{6}}x{\isadigit{7}}{\isacharunderscore}{\kern0pt}hd\ kc{\isacharunderscore}{\kern0pt}{\isadigit{6}}x{\isadigit{7}}{\isacharunderscore}{\kern0pt}non{\isacharunderscore}{\kern0pt}nil\isanewline
\ \ kc{\isacharunderscore}{\kern0pt}{\isadigit{6}}x{\isadigit{8}}\ kc{\isacharunderscore}{\kern0pt}{\isadigit{6}}x{\isadigit{8}}{\isacharunderscore}{\kern0pt}hd\ kc{\isacharunderscore}{\kern0pt}{\isadigit{6}}x{\isadigit{8}}{\isacharunderscore}{\kern0pt}non{\isacharunderscore}{\kern0pt}nil\isanewline
\ \ kc{\isacharunderscore}{\kern0pt}{\isadigit{6}}x{\isadigit{9}}\ kc{\isacharunderscore}{\kern0pt}{\isadigit{6}}x{\isadigit{9}}{\isacharunderscore}{\kern0pt}hd\ kc{\isacharunderscore}{\kern0pt}{\isadigit{6}}x{\isadigit{9}}{\isacharunderscore}{\kern0pt}non{\isacharunderscore}{\kern0pt}nil%
\begin{isamarkuptext}%
For every \isa{{\isadigit{6}}{\isasymtimes}m}-board with \isa{m\ {\isasymge}\ {\isadigit{5}}} there exists a knight's path that starts in 
\isa{{\isacharparenleft}{\kern0pt}{\isadigit{1}}{\isacharcomma}{\kern0pt}{\isadigit{1}}{\isacharparenright}{\kern0pt}} (bottom-left) and ends in \isa{{\isacharparenleft}{\kern0pt}{\isadigit{5}}{\isacharcomma}{\kern0pt}{\isadigit{2}}{\isacharparenright}{\kern0pt}} (top-left).%
\end{isamarkuptext}\isamarkuptrue%
\isacommand{lemma}\isamarkupfalse%
\ knights{\isacharunderscore}{\kern0pt}path{\isacharunderscore}{\kern0pt}{\isadigit{6}}xm{\isacharunderscore}{\kern0pt}ul{\isacharunderscore}{\kern0pt}exists{\isacharcolon}{\kern0pt}\ \isanewline
\ \ \isakeyword{assumes}\ {\isachardoublequoteopen}m\ {\isasymge}\ {\isadigit{5}}{\isachardoublequoteclose}\ \isanewline
\ \ \isakeyword{shows}\ {\isachardoublequoteopen}{\isasymexists}ps{\isachardot}{\kern0pt}\ knights{\isacharunderscore}{\kern0pt}path\ {\isacharparenleft}{\kern0pt}board\ {\isadigit{6}}\ m{\isacharparenright}{\kern0pt}\ ps\ {\isasymand}\ hd\ ps\ {\isacharequal}{\kern0pt}\ {\isacharparenleft}{\kern0pt}{\isadigit{1}}{\isacharcomma}{\kern0pt}{\isadigit{1}}{\isacharparenright}{\kern0pt}\ {\isasymand}\ last\ ps\ {\isacharequal}{\kern0pt}\ {\isacharparenleft}{\kern0pt}{\isadigit{5}}{\isacharcomma}{\kern0pt}{\isadigit{2}}{\isacharparenright}{\kern0pt}{\isachardoublequoteclose}\isanewline
%
\isadelimproof
\ \ %
\endisadelimproof
%
\isatagproof
\isacommand{using}\isamarkupfalse%
\ assms\isanewline
\isacommand{proof}\isamarkupfalse%
\ {\isacharparenleft}{\kern0pt}induction\ m\ rule{\isacharcolon}{\kern0pt}\ less{\isacharunderscore}{\kern0pt}induct{\isacharparenright}{\kern0pt}\isanewline
\ \ \isacommand{case}\isamarkupfalse%
\ {\isacharparenleft}{\kern0pt}less\ m{\isacharparenright}{\kern0pt}\isanewline
\ \ \isacommand{then}\isamarkupfalse%
\ \isacommand{have}\isamarkupfalse%
\ {\isachardoublequoteopen}m\ {\isasymin}\ {\isacharbraceleft}{\kern0pt}{\isadigit{5}}{\isacharcomma}{\kern0pt}{\isadigit{6}}{\isacharcomma}{\kern0pt}{\isadigit{7}}{\isacharcomma}{\kern0pt}{\isadigit{8}}{\isacharcomma}{\kern0pt}{\isadigit{9}}{\isacharbraceright}{\kern0pt}\ {\isasymor}\ {\isadigit{5}}\ {\isasymle}\ m{\isacharminus}{\kern0pt}{\isadigit{5}}{\isachardoublequoteclose}\ \isacommand{by}\isamarkupfalse%
\ auto\isanewline
\ \ \isacommand{then}\isamarkupfalse%
\ \isacommand{show}\isamarkupfalse%
\ {\isacharquery}{\kern0pt}case\isanewline
\ \ \isacommand{proof}\isamarkupfalse%
\ {\isacharparenleft}{\kern0pt}elim\ disjE{\isacharparenright}{\kern0pt}\isanewline
\ \ \ \ \isacommand{assume}\isamarkupfalse%
\ {\isachardoublequoteopen}m\ {\isasymin}\ {\isacharbraceleft}{\kern0pt}{\isadigit{5}}{\isacharcomma}{\kern0pt}{\isadigit{6}}{\isacharcomma}{\kern0pt}{\isadigit{7}}{\isacharcomma}{\kern0pt}{\isadigit{8}}{\isacharcomma}{\kern0pt}{\isadigit{9}}{\isacharbraceright}{\kern0pt}{\isachardoublequoteclose}\isanewline
\ \ \ \ \isacommand{then}\isamarkupfalse%
\ \isacommand{show}\isamarkupfalse%
\ {\isacharquery}{\kern0pt}thesis\ \isacommand{using}\isamarkupfalse%
\ kp{\isacharunderscore}{\kern0pt}{\isadigit{6}}xm{\isacharunderscore}{\kern0pt}ul\ \isacommand{by}\isamarkupfalse%
\ fastforce\isanewline
\ \ \isacommand{next}\isamarkupfalse%
\isanewline
\ \ \ \ \isacommand{let}\isamarkupfalse%
\ {\isacharquery}{\kern0pt}ps\isactrlsub {\isadigit{1}}{\isacharequal}{\kern0pt}{\isachardoublequoteopen}kp{\isadigit{6}}x{\isadigit{5}}ul{\isachardoublequoteclose}\isanewline
\ \ \ \ \isacommand{let}\isamarkupfalse%
\ {\isacharquery}{\kern0pt}b\isactrlsub {\isadigit{1}}{\isacharequal}{\kern0pt}{\isachardoublequoteopen}board\ {\isadigit{6}}\ {\isadigit{5}}{\isachardoublequoteclose}\isanewline
\ \ \ \ \isacommand{have}\isamarkupfalse%
\ ps\isactrlsub {\isadigit{1}}{\isacharunderscore}{\kern0pt}prems{\isacharcolon}{\kern0pt}\ {\isachardoublequoteopen}knights{\isacharunderscore}{\kern0pt}path\ {\isacharquery}{\kern0pt}b\isactrlsub {\isadigit{1}}\ {\isacharquery}{\kern0pt}ps\isactrlsub {\isadigit{1}}{\isachardoublequoteclose}\ {\isachardoublequoteopen}hd\ {\isacharquery}{\kern0pt}ps\isactrlsub {\isadigit{1}}\ {\isacharequal}{\kern0pt}\ {\isacharparenleft}{\kern0pt}{\isadigit{1}}{\isacharcomma}{\kern0pt}{\isadigit{1}}{\isacharparenright}{\kern0pt}{\isachardoublequoteclose}\ {\isachardoublequoteopen}last\ {\isacharquery}{\kern0pt}ps\isactrlsub {\isadigit{1}}\ {\isacharequal}{\kern0pt}\ {\isacharparenleft}{\kern0pt}{\isadigit{5}}{\isacharcomma}{\kern0pt}{\isadigit{2}}{\isacharparenright}{\kern0pt}{\isachardoublequoteclose}\ \isanewline
\ \ \ \ \ \ \isacommand{using}\isamarkupfalse%
\ kp{\isacharunderscore}{\kern0pt}{\isadigit{6}}xm{\isacharunderscore}{\kern0pt}ul\ \isacommand{by}\isamarkupfalse%
\ auto\isanewline
\ \ \ \ \isacommand{assume}\isamarkupfalse%
\ m{\isacharunderscore}{\kern0pt}ge{\isacharcolon}{\kern0pt}\ {\isachardoublequoteopen}{\isadigit{5}}\ {\isasymle}\ m{\isacharminus}{\kern0pt}{\isadigit{5}}{\isachardoublequoteclose}\ \isanewline
\ \ \ \ \isacommand{then}\isamarkupfalse%
\ \isacommand{obtain}\isamarkupfalse%
\ ps\isactrlsub {\isadigit{2}}\ \isakeyword{where}\ ps\isactrlsub {\isadigit{2}}{\isacharunderscore}{\kern0pt}IH{\isacharcolon}{\kern0pt}\ {\isachardoublequoteopen}knights{\isacharunderscore}{\kern0pt}path\ {\isacharparenleft}{\kern0pt}board\ {\isadigit{6}}\ {\isacharparenleft}{\kern0pt}m{\isacharminus}{\kern0pt}{\isadigit{5}}{\isacharparenright}{\kern0pt}{\isacharparenright}{\kern0pt}\ ps\isactrlsub {\isadigit{2}}{\isachardoublequoteclose}\ {\isachardoublequoteopen}hd\ ps\isactrlsub {\isadigit{2}}\ {\isacharequal}{\kern0pt}\ {\isacharparenleft}{\kern0pt}{\isadigit{1}}{\isacharcomma}{\kern0pt}{\isadigit{1}}{\isacharparenright}{\kern0pt}{\isachardoublequoteclose}\ \isanewline
\ \ \ \ \ \ \ \ \ \ \ \ \ \ \ \ \ \ \ \ \ \ \ \ \ \ \ \ \ \ \ \ {\isachardoublequoteopen}last\ ps\isactrlsub {\isadigit{2}}\ {\isacharequal}{\kern0pt}\ {\isacharparenleft}{\kern0pt}{\isadigit{5}}{\isacharcomma}{\kern0pt}{\isadigit{2}}{\isacharparenright}{\kern0pt}{\isachardoublequoteclose}\isanewline
\ \ \ \ \ \ \isacommand{using}\isamarkupfalse%
\ less{\isachardot}{\kern0pt}IH{\isacharbrackleft}{\kern0pt}of\ {\isachardoublequoteopen}m{\isacharminus}{\kern0pt}{\isadigit{5}}{\isachardoublequoteclose}{\isacharbrackright}{\kern0pt}\ knights{\isacharunderscore}{\kern0pt}path{\isacharunderscore}{\kern0pt}non{\isacharunderscore}{\kern0pt}nil\ \isacommand{by}\isamarkupfalse%
\ auto\isanewline
\isanewline
\ \ \ \ \isacommand{have}\isamarkupfalse%
\ {\isachardoublequoteopen}{\isadigit{2}}{\isadigit{7}}\ {\isacharless}{\kern0pt}\ length\ {\isacharquery}{\kern0pt}ps\isactrlsub {\isadigit{1}}{\isachardoublequoteclose}\ {\isachardoublequoteopen}last\ {\isacharparenleft}{\kern0pt}take\ {\isadigit{2}}{\isadigit{7}}\ {\isacharquery}{\kern0pt}ps\isactrlsub {\isadigit{1}}{\isacharparenright}{\kern0pt}\ {\isacharequal}{\kern0pt}\ {\isacharparenleft}{\kern0pt}{\isadigit{2}}{\isacharcomma}{\kern0pt}{\isadigit{4}}{\isacharparenright}{\kern0pt}{\isachardoublequoteclose}\ {\isachardoublequoteopen}hd\ {\isacharparenleft}{\kern0pt}drop\ {\isadigit{2}}{\isadigit{7}}\ {\isacharquery}{\kern0pt}ps\isactrlsub {\isadigit{1}}{\isacharparenright}{\kern0pt}\ {\isacharequal}{\kern0pt}\ {\isacharparenleft}{\kern0pt}{\isadigit{4}}{\isacharcomma}{\kern0pt}{\isadigit{5}}{\isacharparenright}{\kern0pt}{\isachardoublequoteclose}\ \isacommand{by}\isamarkupfalse%
\ eval{\isacharplus}{\kern0pt}\isanewline
\ \ \ \ \isacommand{then}\isamarkupfalse%
\ \isacommand{have}\isamarkupfalse%
\ {\isachardoublequoteopen}step{\isacharunderscore}{\kern0pt}in\ {\isacharquery}{\kern0pt}ps\isactrlsub {\isadigit{1}}\ {\isacharparenleft}{\kern0pt}{\isadigit{2}}{\isacharcomma}{\kern0pt}{\isadigit{4}}{\isacharparenright}{\kern0pt}\ {\isacharparenleft}{\kern0pt}{\isadigit{4}}{\isacharcomma}{\kern0pt}{\isadigit{5}}{\isacharparenright}{\kern0pt}{\isachardoublequoteclose}\isanewline
\ \ \ \ \ \ \isacommand{unfolding}\isamarkupfalse%
\ step{\isacharunderscore}{\kern0pt}in{\isacharunderscore}{\kern0pt}def\ \isacommand{using}\isamarkupfalse%
\ zero{\isacharunderscore}{\kern0pt}less{\isacharunderscore}{\kern0pt}numeral\ \isacommand{by}\isamarkupfalse%
\ blast\isanewline
\ \ \ \ \isacommand{then}\isamarkupfalse%
\ \isacommand{have}\isamarkupfalse%
\ {\isachardoublequoteopen}step{\isacharunderscore}{\kern0pt}in\ {\isacharquery}{\kern0pt}ps\isactrlsub {\isadigit{1}}\ {\isacharparenleft}{\kern0pt}{\isadigit{2}}{\isacharcomma}{\kern0pt}{\isadigit{4}}{\isacharparenright}{\kern0pt}\ {\isacharparenleft}{\kern0pt}{\isadigit{4}}{\isacharcomma}{\kern0pt}{\isadigit{5}}{\isacharparenright}{\kern0pt}{\isachardoublequoteclose}\ \isanewline
\ \ \ \ \ \ \ \ \ \ \ \ \ \ {\isachardoublequoteopen}valid{\isacharunderscore}{\kern0pt}step\ {\isacharparenleft}{\kern0pt}{\isadigit{2}}{\isacharcomma}{\kern0pt}{\isadigit{4}}{\isacharparenright}{\kern0pt}\ {\isacharparenleft}{\kern0pt}{\isadigit{1}}{\isacharcomma}{\kern0pt}int\ {\isadigit{5}}{\isacharplus}{\kern0pt}{\isadigit{1}}{\isacharparenright}{\kern0pt}{\isachardoublequoteclose}\ \isanewline
\ \ \ \ \ \ \ \ \ \ \ \ \ \ {\isachardoublequoteopen}valid{\isacharunderscore}{\kern0pt}step\ {\isacharparenleft}{\kern0pt}{\isadigit{5}}{\isacharcomma}{\kern0pt}int\ {\isadigit{5}}{\isacharplus}{\kern0pt}{\isadigit{2}}{\isacharparenright}{\kern0pt}\ {\isacharparenleft}{\kern0pt}{\isadigit{4}}{\isacharcomma}{\kern0pt}{\isadigit{5}}{\isacharparenright}{\kern0pt}{\isachardoublequoteclose}\isanewline
\ \ \ \ \ \ \isacommand{unfolding}\isamarkupfalse%
\ valid{\isacharunderscore}{\kern0pt}step{\isacharunderscore}{\kern0pt}def\ \isacommand{by}\isamarkupfalse%
\ auto\isanewline
\ \ \ \ \isacommand{then}\isamarkupfalse%
\ \isacommand{show}\isamarkupfalse%
\ {\isacharquery}{\kern0pt}thesis\isanewline
\ \ \ \ \ \ \isacommand{using}\isamarkupfalse%
\ {\isacartoucheopen}{\isadigit{5}}\ {\isasymle}\ m{\isacharminus}{\kern0pt}{\isadigit{5}}{\isacartoucheclose}\ ps\isactrlsub {\isadigit{1}}{\isacharunderscore}{\kern0pt}prems\ ps\isactrlsub {\isadigit{2}}{\isacharunderscore}{\kern0pt}IH\ knights{\isacharunderscore}{\kern0pt}path{\isacharunderscore}{\kern0pt}split{\isacharunderscore}{\kern0pt}concat{\isacharbrackleft}{\kern0pt}of\ {\isadigit{6}}\ {\isadigit{5}}\ {\isacharquery}{\kern0pt}ps\isactrlsub {\isadigit{1}}\ {\isachardoublequoteopen}m{\isacharminus}{\kern0pt}{\isadigit{5}}{\isachardoublequoteclose}\ ps\isactrlsub {\isadigit{2}}{\isacharbrackright}{\kern0pt}\ \isacommand{by}\isamarkupfalse%
\ auto\isanewline
\ \ \isacommand{qed}\isamarkupfalse%
\isanewline
\isacommand{qed}\isamarkupfalse%
%
\endisatagproof
{\isafoldproof}%
%
\isadelimproof
%
\endisadelimproof
%
\begin{isamarkuptext}%
For every \isa{{\isadigit{6}}{\isasymtimes}m}-board with \isa{m\ {\isasymge}\ {\isadigit{5}}} there exists a knight's circuit.%
\end{isamarkuptext}\isamarkuptrue%
\isacommand{lemma}\isamarkupfalse%
\ knights{\isacharunderscore}{\kern0pt}circuit{\isacharunderscore}{\kern0pt}{\isadigit{6}}xm{\isacharunderscore}{\kern0pt}exists{\isacharcolon}{\kern0pt}\ \isanewline
\ \ \isakeyword{assumes}\ {\isachardoublequoteopen}m\ {\isasymge}\ {\isadigit{5}}{\isachardoublequoteclose}\ \isanewline
\ \ \isakeyword{shows}\ {\isachardoublequoteopen}{\isasymexists}ps{\isachardot}{\kern0pt}\ knights{\isacharunderscore}{\kern0pt}circuit\ {\isacharparenleft}{\kern0pt}board\ {\isadigit{6}}\ m{\isacharparenright}{\kern0pt}\ ps{\isachardoublequoteclose}\isanewline
%
\isadelimproof
\ \ %
\endisadelimproof
%
\isatagproof
\isacommand{using}\isamarkupfalse%
\ assms\isanewline
\isacommand{proof}\isamarkupfalse%
\ {\isacharminus}{\kern0pt}\isanewline
\ \ \isacommand{have}\isamarkupfalse%
\ {\isachardoublequoteopen}m\ {\isasymin}\ {\isacharbraceleft}{\kern0pt}{\isadigit{5}}{\isacharcomma}{\kern0pt}{\isadigit{6}}{\isacharcomma}{\kern0pt}{\isadigit{7}}{\isacharcomma}{\kern0pt}{\isadigit{8}}{\isacharcomma}{\kern0pt}{\isadigit{9}}{\isacharbraceright}{\kern0pt}\ {\isasymor}\ {\isadigit{5}}\ {\isasymle}\ m{\isacharminus}{\kern0pt}{\isadigit{5}}{\isachardoublequoteclose}\ \isacommand{using}\isamarkupfalse%
\ assms\ \isacommand{by}\isamarkupfalse%
\ auto\isanewline
\ \ \isacommand{then}\isamarkupfalse%
\ \isacommand{show}\isamarkupfalse%
\ {\isacharquery}{\kern0pt}thesis\isanewline
\ \ \isacommand{proof}\isamarkupfalse%
\ {\isacharparenleft}{\kern0pt}elim\ disjE{\isacharparenright}{\kern0pt}\isanewline
\ \ \ \ \isacommand{assume}\isamarkupfalse%
\ {\isachardoublequoteopen}m\ {\isasymin}\ {\isacharbraceleft}{\kern0pt}{\isadigit{5}}{\isacharcomma}{\kern0pt}{\isadigit{6}}{\isacharcomma}{\kern0pt}{\isadigit{7}}{\isacharcomma}{\kern0pt}{\isadigit{8}}{\isacharcomma}{\kern0pt}{\isadigit{9}}{\isacharbraceright}{\kern0pt}{\isachardoublequoteclose}\isanewline
\ \ \ \ \isacommand{then}\isamarkupfalse%
\ \isacommand{show}\isamarkupfalse%
\ {\isacharquery}{\kern0pt}thesis\ \isacommand{using}\isamarkupfalse%
\ kc{\isacharunderscore}{\kern0pt}{\isadigit{6}}xm\ \isacommand{by}\isamarkupfalse%
\ fastforce\isanewline
\ \ \isacommand{next}\isamarkupfalse%
\isanewline
\ \ \ \ \isacommand{let}\isamarkupfalse%
\ {\isacharquery}{\kern0pt}ps\isactrlsub {\isadigit{1}}{\isacharequal}{\kern0pt}{\isachardoublequoteopen}rev\ kc{\isadigit{6}}x{\isadigit{5}}{\isachardoublequoteclose}\isanewline
\ \ \ \ \isacommand{have}\isamarkupfalse%
\ {\isachardoublequoteopen}knights{\isacharunderscore}{\kern0pt}circuit\ b{\isadigit{6}}x{\isadigit{5}}\ {\isacharquery}{\kern0pt}ps\isactrlsub {\isadigit{1}}{\isachardoublequoteclose}\ {\isachardoublequoteopen}last\ {\isacharquery}{\kern0pt}ps\isactrlsub {\isadigit{1}}\ {\isacharequal}{\kern0pt}\ {\isacharparenleft}{\kern0pt}{\isadigit{1}}{\isacharcomma}{\kern0pt}{\isadigit{1}}{\isacharparenright}{\kern0pt}{\isachardoublequoteclose}\isanewline
\ \ \ \ \ \ \isacommand{using}\isamarkupfalse%
\ kc{\isacharunderscore}{\kern0pt}{\isadigit{6}}xm\ knights{\isacharunderscore}{\kern0pt}circuit{\isacharunderscore}{\kern0pt}rev\ \isacommand{by}\isamarkupfalse%
\ {\isacharparenleft}{\kern0pt}auto\ simp{\isacharcolon}{\kern0pt}\ last{\isacharunderscore}{\kern0pt}rev{\isacharparenright}{\kern0pt}\isanewline
\ \ \ \ \isacommand{then}\isamarkupfalse%
\ \isacommand{have}\isamarkupfalse%
\ ps\isactrlsub {\isadigit{1}}{\isacharunderscore}{\kern0pt}prems{\isacharcolon}{\kern0pt}\ {\isachardoublequoteopen}knights{\isacharunderscore}{\kern0pt}path\ b{\isadigit{6}}x{\isadigit{5}}\ {\isacharquery}{\kern0pt}ps\isactrlsub {\isadigit{1}}{\isachardoublequoteclose}\ {\isachardoublequoteopen}valid{\isacharunderscore}{\kern0pt}step\ {\isacharparenleft}{\kern0pt}last\ {\isacharquery}{\kern0pt}ps\isactrlsub {\isadigit{1}}{\isacharparenright}{\kern0pt}\ {\isacharparenleft}{\kern0pt}hd\ {\isacharquery}{\kern0pt}ps\isactrlsub {\isadigit{1}}{\isacharparenright}{\kern0pt}{\isachardoublequoteclose}\isanewline
\ \ \ \ \ \ \isacommand{unfolding}\isamarkupfalse%
\ knights{\isacharunderscore}{\kern0pt}circuit{\isacharunderscore}{\kern0pt}def\ \isacommand{using}\isamarkupfalse%
\ valid{\isacharunderscore}{\kern0pt}step{\isacharunderscore}{\kern0pt}rev\ \isacommand{by}\isamarkupfalse%
\ auto\isanewline
\ \ \ \ \isacommand{assume}\isamarkupfalse%
\ m{\isacharunderscore}{\kern0pt}ge{\isacharcolon}{\kern0pt}\ {\isachardoublequoteopen}{\isadigit{5}}\ {\isasymle}\ m{\isacharminus}{\kern0pt}{\isadigit{5}}{\isachardoublequoteclose}\ \isanewline
\ \ \ \ \isacommand{then}\isamarkupfalse%
\ \isacommand{obtain}\isamarkupfalse%
\ ps\isactrlsub {\isadigit{2}}\ \isakeyword{where}\ ps{\isadigit{2}}{\isacharunderscore}{\kern0pt}prems{\isacharcolon}{\kern0pt}\ {\isachardoublequoteopen}knights{\isacharunderscore}{\kern0pt}path\ {\isacharparenleft}{\kern0pt}board\ {\isadigit{6}}\ {\isacharparenleft}{\kern0pt}m{\isacharminus}{\kern0pt}{\isadigit{5}}{\isacharparenright}{\kern0pt}{\isacharparenright}{\kern0pt}\ ps\isactrlsub {\isadigit{2}}{\isachardoublequoteclose}\ {\isachardoublequoteopen}hd\ ps\isactrlsub {\isadigit{2}}\ {\isacharequal}{\kern0pt}\ {\isacharparenleft}{\kern0pt}{\isadigit{1}}{\isacharcomma}{\kern0pt}{\isadigit{1}}{\isacharparenright}{\kern0pt}{\isachardoublequoteclose}\ \isanewline
\ \ \ \ \ \ \ \ \ \ \ \ \ \ \ \ \ \ \ \ \ \ \ \ \ \ \ \ \ \ \ \ \ \ \ {\isachardoublequoteopen}last\ ps\isactrlsub {\isadigit{2}}\ {\isacharequal}{\kern0pt}\ {\isacharparenleft}{\kern0pt}{\isadigit{5}}{\isacharcomma}{\kern0pt}{\isadigit{2}}{\isacharparenright}{\kern0pt}{\isachardoublequoteclose}\isanewline
\ \ \ \ \ \ \isacommand{using}\isamarkupfalse%
\ knights{\isacharunderscore}{\kern0pt}path{\isacharunderscore}{\kern0pt}{\isadigit{6}}xm{\isacharunderscore}{\kern0pt}ul{\isacharunderscore}{\kern0pt}exists{\isacharbrackleft}{\kern0pt}of\ {\isachardoublequoteopen}{\isacharparenleft}{\kern0pt}m{\isacharminus}{\kern0pt}{\isadigit{5}}{\isacharparenright}{\kern0pt}{\isachardoublequoteclose}{\isacharbrackright}{\kern0pt}\ knights{\isacharunderscore}{\kern0pt}path{\isacharunderscore}{\kern0pt}non{\isacharunderscore}{\kern0pt}nil\ \isacommand{by}\isamarkupfalse%
\ auto\isanewline
\isanewline
\ \ \ \ \isacommand{have}\isamarkupfalse%
\ {\isachardoublequoteopen}{\isadigit{2}}\ {\isacharless}{\kern0pt}\ length\ {\isacharquery}{\kern0pt}ps\isactrlsub {\isadigit{1}}{\isachardoublequoteclose}\ {\isachardoublequoteopen}last\ {\isacharparenleft}{\kern0pt}take\ {\isadigit{2}}\ {\isacharquery}{\kern0pt}ps\isactrlsub {\isadigit{1}}{\isacharparenright}{\kern0pt}\ {\isacharequal}{\kern0pt}\ {\isacharparenleft}{\kern0pt}{\isadigit{2}}{\isacharcomma}{\kern0pt}{\isadigit{4}}{\isacharparenright}{\kern0pt}{\isachardoublequoteclose}\ {\isachardoublequoteopen}hd\ {\isacharparenleft}{\kern0pt}drop\ {\isadigit{2}}\ {\isacharquery}{\kern0pt}ps\isactrlsub {\isadigit{1}}{\isacharparenright}{\kern0pt}\ {\isacharequal}{\kern0pt}\ {\isacharparenleft}{\kern0pt}{\isadigit{4}}{\isacharcomma}{\kern0pt}{\isadigit{5}}{\isacharparenright}{\kern0pt}{\isachardoublequoteclose}\ \isacommand{by}\isamarkupfalse%
\ eval{\isacharplus}{\kern0pt}\isanewline
\ \ \ \ \isacommand{then}\isamarkupfalse%
\ \isacommand{have}\isamarkupfalse%
\ {\isachardoublequoteopen}step{\isacharunderscore}{\kern0pt}in\ {\isacharquery}{\kern0pt}ps\isactrlsub {\isadigit{1}}\ {\isacharparenleft}{\kern0pt}{\isadigit{2}}{\isacharcomma}{\kern0pt}{\isadigit{4}}{\isacharparenright}{\kern0pt}\ {\isacharparenleft}{\kern0pt}{\isadigit{4}}{\isacharcomma}{\kern0pt}{\isadigit{5}}{\isacharparenright}{\kern0pt}{\isachardoublequoteclose}\isanewline
\ \ \ \ \ \ \isacommand{unfolding}\isamarkupfalse%
\ step{\isacharunderscore}{\kern0pt}in{\isacharunderscore}{\kern0pt}def\ \isacommand{using}\isamarkupfalse%
\ zero{\isacharunderscore}{\kern0pt}less{\isacharunderscore}{\kern0pt}numeral\ \isacommand{by}\isamarkupfalse%
\ blast\isanewline
\ \ \ \ \isacommand{then}\isamarkupfalse%
\ \isacommand{have}\isamarkupfalse%
\ {\isachardoublequoteopen}step{\isacharunderscore}{\kern0pt}in\ {\isacharquery}{\kern0pt}ps\isactrlsub {\isadigit{1}}\ {\isacharparenleft}{\kern0pt}{\isadigit{2}}{\isacharcomma}{\kern0pt}{\isadigit{4}}{\isacharparenright}{\kern0pt}\ {\isacharparenleft}{\kern0pt}{\isadigit{4}}{\isacharcomma}{\kern0pt}{\isadigit{5}}{\isacharparenright}{\kern0pt}{\isachardoublequoteclose}\ \isanewline
\ \ \ \ \ \ \ \ \ \ \ \ \ \ {\isachardoublequoteopen}valid{\isacharunderscore}{\kern0pt}step\ {\isacharparenleft}{\kern0pt}{\isadigit{2}}{\isacharcomma}{\kern0pt}{\isadigit{4}}{\isacharparenright}{\kern0pt}\ {\isacharparenleft}{\kern0pt}{\isadigit{1}}{\isacharcomma}{\kern0pt}int\ {\isadigit{5}}{\isacharplus}{\kern0pt}{\isadigit{1}}{\isacharparenright}{\kern0pt}{\isachardoublequoteclose}\ \isanewline
\ \ \ \ \ \ \ \ \ \ \ \ \ \ {\isachardoublequoteopen}valid{\isacharunderscore}{\kern0pt}step\ {\isacharparenleft}{\kern0pt}{\isadigit{5}}{\isacharcomma}{\kern0pt}int\ {\isadigit{5}}{\isacharplus}{\kern0pt}{\isadigit{2}}{\isacharparenright}{\kern0pt}\ {\isacharparenleft}{\kern0pt}{\isadigit{4}}{\isacharcomma}{\kern0pt}{\isadigit{5}}{\isacharparenright}{\kern0pt}{\isachardoublequoteclose}\isanewline
\ \ \ \ \ \ \isacommand{unfolding}\isamarkupfalse%
\ valid{\isacharunderscore}{\kern0pt}step{\isacharunderscore}{\kern0pt}def\ \isacommand{by}\isamarkupfalse%
\ auto\isanewline
\ \ \ \ \isacommand{then}\isamarkupfalse%
\ \isacommand{have}\isamarkupfalse%
\ {\isachardoublequoteopen}{\isasymexists}ps{\isachardot}{\kern0pt}\ knights{\isacharunderscore}{\kern0pt}path\ {\isacharparenleft}{\kern0pt}board\ {\isadigit{6}}\ m{\isacharparenright}{\kern0pt}\ ps\ {\isasymand}\ hd\ ps\ {\isacharequal}{\kern0pt}\ hd\ {\isacharquery}{\kern0pt}ps\isactrlsub {\isadigit{1}}\ {\isasymand}\ last\ ps\ {\isacharequal}{\kern0pt}\ last\ {\isacharquery}{\kern0pt}ps\isactrlsub {\isadigit{1}}{\isachardoublequoteclose}\ \ \ \ \ \ \ \ \ \ \ \ \ \ \isanewline
\ \ \ \ \ \ \isacommand{using}\isamarkupfalse%
\ m{\isacharunderscore}{\kern0pt}ge\ ps\isactrlsub {\isadigit{1}}{\isacharunderscore}{\kern0pt}prems\ ps{\isadigit{2}}{\isacharunderscore}{\kern0pt}prems\ knights{\isacharunderscore}{\kern0pt}path{\isacharunderscore}{\kern0pt}split{\isacharunderscore}{\kern0pt}concat{\isacharbrackleft}{\kern0pt}of\ {\isadigit{6}}\ {\isadigit{5}}\ {\isacharquery}{\kern0pt}ps\isactrlsub {\isadigit{1}}\ {\isachardoublequoteopen}m{\isacharminus}{\kern0pt}{\isadigit{5}}{\isachardoublequoteclose}\ ps\isactrlsub {\isadigit{2}}{\isacharbrackright}{\kern0pt}\ \isacommand{by}\isamarkupfalse%
\ auto\isanewline
\ \ \ \ \isacommand{then}\isamarkupfalse%
\ \isacommand{show}\isamarkupfalse%
\ {\isacharquery}{\kern0pt}thesis\ \isacommand{using}\isamarkupfalse%
\ ps\isactrlsub {\isadigit{1}}{\isacharunderscore}{\kern0pt}prems\ \isacommand{by}\isamarkupfalse%
\ {\isacharparenleft}{\kern0pt}auto\ simp{\isacharcolon}{\kern0pt}\ knights{\isacharunderscore}{\kern0pt}circuit{\isacharunderscore}{\kern0pt}def{\isacharparenright}{\kern0pt}\isanewline
\ \ \isacommand{qed}\isamarkupfalse%
\isanewline
\isacommand{qed}\isamarkupfalse%
%
\endisatagproof
{\isafoldproof}%
%
\isadelimproof
%
\endisadelimproof
%
\begin{isamarkuptext}%
\isa{{\isadigit{5}}\ {\isasymle}\ {\isacharquery}{\kern0pt}m\ {\isasymLongrightarrow}\ {\isasymexists}ps{\isachardot}{\kern0pt}\ knights{\isacharunderscore}{\kern0pt}path\ {\isacharparenleft}{\kern0pt}board\ {\isadigit{6}}\ {\isacharquery}{\kern0pt}m{\isacharparenright}{\kern0pt}\ ps\ {\isasymand}\ hd\ ps\ {\isacharequal}{\kern0pt}\ {\isacharparenleft}{\kern0pt}{\isadigit{1}}{\isacharcomma}{\kern0pt}\ {\isadigit{1}}{\isacharparenright}{\kern0pt}\ {\isasymand}\ last\ ps\ {\isacharequal}{\kern0pt}\ {\isacharparenleft}{\kern0pt}{\isadigit{5}}{\isacharcomma}{\kern0pt}\ {\isadigit{2}}{\isacharparenright}{\kern0pt}} and \isa{{\isadigit{5}}\ {\isasymle}\ {\isacharquery}{\kern0pt}m\ {\isasymLongrightarrow}\ {\isasymexists}ps{\isachardot}{\kern0pt}\ knights{\isacharunderscore}{\kern0pt}circuit\ {\isacharparenleft}{\kern0pt}board\ {\isadigit{6}}\ {\isacharquery}{\kern0pt}m{\isacharparenright}{\kern0pt}\ ps} formalize Lemma 2 
from \cite{cull_decurtins_1987}.%
\end{isamarkuptext}\isamarkuptrue%
\isacommand{lemmas}\isamarkupfalse%
\ knights{\isacharunderscore}{\kern0pt}path{\isacharunderscore}{\kern0pt}{\isadigit{6}}xm{\isacharunderscore}{\kern0pt}exists\ {\isacharequal}{\kern0pt}\ knights{\isacharunderscore}{\kern0pt}path{\isacharunderscore}{\kern0pt}{\isadigit{6}}xm{\isacharunderscore}{\kern0pt}ul{\isacharunderscore}{\kern0pt}exists\ knights{\isacharunderscore}{\kern0pt}circuit{\isacharunderscore}{\kern0pt}{\isadigit{6}}xm{\isacharunderscore}{\kern0pt}exists%
\isadelimdocument
%
\endisadelimdocument
%
\isatagdocument
%
\isamarkupsection{Knight's Paths and Circuits for \isa{{\isadigit{8}}{\isasymtimes}m}-Boards%
}
\isamarkuptrue%
%
\endisatagdocument
{\isafolddocument}%
%
\isadelimdocument
%
\endisadelimdocument
\isacommand{abbreviation}\isamarkupfalse%
\ {\isachardoublequoteopen}b{\isadigit{8}}x{\isadigit{5}}\ {\isasymequiv}\ board\ {\isadigit{8}}\ {\isadigit{5}}{\isachardoublequoteclose}%
\begin{isamarkuptext}%
A Knight's path for the \isa{{\isacharparenleft}{\kern0pt}{\isadigit{8}}{\isasymtimes}{\isadigit{5}}{\isacharparenright}{\kern0pt}}-board that starts in the lower-left and ends in the 
upper-left.
  \begin{table}[H]
    \begin{tabular}{lllll}
      28 &  7 & 22 & 39 & 26 \\
      23 & 40 & 27 &  6 & 21 \\
       8 & 29 & 38 & 25 & 14 \\
      37 & 24 & 15 & 20 &  5 \\
      16 &  9 & 30 & 13 & 34 \\
      31 & 36 & 33 &  4 & 19 \\
      10 & 17 &  2 & 35 & 12 \\
       1 & 32 & 11 & 18 &  3
    \end{tabular}
  \end{table}%
\end{isamarkuptext}\isamarkuptrue%
\isacommand{abbreviation}\isamarkupfalse%
\ {\isachardoublequoteopen}kp{\isadigit{8}}x{\isadigit{5}}ul\ {\isasymequiv}\ the\ {\isacharparenleft}{\kern0pt}to{\isacharunderscore}{\kern0pt}path\ \isanewline
\ \ {\isacharbrackleft}{\kern0pt}{\isacharbrackleft}{\kern0pt}{\isadigit{2}}{\isadigit{8}}{\isacharcomma}{\kern0pt}{\isadigit{7}}{\isacharcomma}{\kern0pt}{\isadigit{2}}{\isadigit{2}}{\isacharcomma}{\kern0pt}{\isadigit{3}}{\isadigit{9}}{\isacharcomma}{\kern0pt}{\isadigit{2}}{\isadigit{6}}{\isacharbrackright}{\kern0pt}{\isacharcomma}{\kern0pt}\isanewline
\ \ {\isacharbrackleft}{\kern0pt}{\isadigit{2}}{\isadigit{3}}{\isacharcomma}{\kern0pt}{\isadigit{4}}{\isadigit{0}}{\isacharcomma}{\kern0pt}{\isadigit{2}}{\isadigit{7}}{\isacharcomma}{\kern0pt}{\isadigit{6}}{\isacharcomma}{\kern0pt}{\isadigit{2}}{\isadigit{1}}{\isacharbrackright}{\kern0pt}{\isacharcomma}{\kern0pt}\isanewline
\ \ {\isacharbrackleft}{\kern0pt}{\isadigit{8}}{\isacharcomma}{\kern0pt}{\isadigit{2}}{\isadigit{9}}{\isacharcomma}{\kern0pt}{\isadigit{3}}{\isadigit{8}}{\isacharcomma}{\kern0pt}{\isadigit{2}}{\isadigit{5}}{\isacharcomma}{\kern0pt}{\isadigit{1}}{\isadigit{4}}{\isacharbrackright}{\kern0pt}{\isacharcomma}{\kern0pt}\isanewline
\ \ {\isacharbrackleft}{\kern0pt}{\isadigit{3}}{\isadigit{7}}{\isacharcomma}{\kern0pt}{\isadigit{2}}{\isadigit{4}}{\isacharcomma}{\kern0pt}{\isadigit{1}}{\isadigit{5}}{\isacharcomma}{\kern0pt}{\isadigit{2}}{\isadigit{0}}{\isacharcomma}{\kern0pt}{\isadigit{5}}{\isacharbrackright}{\kern0pt}{\isacharcomma}{\kern0pt}\isanewline
\ \ {\isacharbrackleft}{\kern0pt}{\isadigit{1}}{\isadigit{6}}{\isacharcomma}{\kern0pt}{\isadigit{9}}{\isacharcomma}{\kern0pt}{\isadigit{3}}{\isadigit{0}}{\isacharcomma}{\kern0pt}{\isadigit{1}}{\isadigit{3}}{\isacharcomma}{\kern0pt}{\isadigit{3}}{\isadigit{4}}{\isacharbrackright}{\kern0pt}{\isacharcomma}{\kern0pt}\isanewline
\ \ {\isacharbrackleft}{\kern0pt}{\isadigit{3}}{\isadigit{1}}{\isacharcomma}{\kern0pt}{\isadigit{3}}{\isadigit{6}}{\isacharcomma}{\kern0pt}{\isadigit{3}}{\isadigit{3}}{\isacharcomma}{\kern0pt}{\isadigit{4}}{\isacharcomma}{\kern0pt}{\isadigit{1}}{\isadigit{9}}{\isacharbrackright}{\kern0pt}{\isacharcomma}{\kern0pt}\isanewline
\ \ {\isacharbrackleft}{\kern0pt}{\isadigit{1}}{\isadigit{0}}{\isacharcomma}{\kern0pt}{\isadigit{1}}{\isadigit{7}}{\isacharcomma}{\kern0pt}{\isadigit{2}}{\isacharcomma}{\kern0pt}{\isadigit{3}}{\isadigit{5}}{\isacharcomma}{\kern0pt}{\isadigit{1}}{\isadigit{2}}{\isacharbrackright}{\kern0pt}{\isacharcomma}{\kern0pt}\isanewline
\ \ {\isacharbrackleft}{\kern0pt}{\isadigit{1}}{\isacharcomma}{\kern0pt}{\isadigit{3}}{\isadigit{2}}{\isacharcomma}{\kern0pt}{\isadigit{1}}{\isadigit{1}}{\isacharcomma}{\kern0pt}{\isadigit{1}}{\isadigit{8}}{\isacharcomma}{\kern0pt}{\isadigit{3}}{\isacharbrackright}{\kern0pt}{\isacharbrackright}{\kern0pt}{\isacharparenright}{\kern0pt}{\isachardoublequoteclose}\isanewline
\isacommand{lemma}\isamarkupfalse%
\ kp{\isacharunderscore}{\kern0pt}{\isadigit{8}}x{\isadigit{5}}{\isacharunderscore}{\kern0pt}ul{\isacharcolon}{\kern0pt}\ {\isachardoublequoteopen}knights{\isacharunderscore}{\kern0pt}path\ b{\isadigit{8}}x{\isadigit{5}}\ kp{\isadigit{8}}x{\isadigit{5}}ul{\isachardoublequoteclose}\isanewline
%
\isadelimproof
\ \ %
\endisadelimproof
%
\isatagproof
\isacommand{by}\isamarkupfalse%
\ {\isacharparenleft}{\kern0pt}simp\ only{\isacharcolon}{\kern0pt}\ knights{\isacharunderscore}{\kern0pt}path{\isacharunderscore}{\kern0pt}exec{\isacharunderscore}{\kern0pt}simp{\isacharparenright}{\kern0pt}\ eval%
\endisatagproof
{\isafoldproof}%
%
\isadelimproof
\isanewline
%
\endisadelimproof
\isanewline
\isacommand{lemma}\isamarkupfalse%
\ kp{\isacharunderscore}{\kern0pt}{\isadigit{8}}x{\isadigit{5}}{\isacharunderscore}{\kern0pt}ul{\isacharunderscore}{\kern0pt}hd{\isacharcolon}{\kern0pt}\ {\isachardoublequoteopen}hd\ kp{\isadigit{8}}x{\isadigit{5}}ul\ {\isacharequal}{\kern0pt}\ {\isacharparenleft}{\kern0pt}{\isadigit{1}}{\isacharcomma}{\kern0pt}{\isadigit{1}}{\isacharparenright}{\kern0pt}{\isachardoublequoteclose}%
\isadelimproof
\ %
\endisadelimproof
%
\isatagproof
\isacommand{by}\isamarkupfalse%
\ eval%
\endisatagproof
{\isafoldproof}%
%
\isadelimproof
%
\endisadelimproof
\isanewline
\isanewline
\isacommand{lemma}\isamarkupfalse%
\ kp{\isacharunderscore}{\kern0pt}{\isadigit{8}}x{\isadigit{5}}{\isacharunderscore}{\kern0pt}ul{\isacharunderscore}{\kern0pt}last{\isacharcolon}{\kern0pt}\ {\isachardoublequoteopen}last\ kp{\isadigit{8}}x{\isadigit{5}}ul\ {\isacharequal}{\kern0pt}\ {\isacharparenleft}{\kern0pt}{\isadigit{7}}{\isacharcomma}{\kern0pt}{\isadigit{2}}{\isacharparenright}{\kern0pt}{\isachardoublequoteclose}%
\isadelimproof
\ %
\endisadelimproof
%
\isatagproof
\isacommand{by}\isamarkupfalse%
\ eval%
\endisatagproof
{\isafoldproof}%
%
\isadelimproof
%
\endisadelimproof
\isanewline
\isanewline
\isacommand{lemma}\isamarkupfalse%
\ kp{\isacharunderscore}{\kern0pt}{\isadigit{8}}x{\isadigit{5}}{\isacharunderscore}{\kern0pt}ul{\isacharunderscore}{\kern0pt}non{\isacharunderscore}{\kern0pt}nil{\isacharcolon}{\kern0pt}\ {\isachardoublequoteopen}kp{\isadigit{8}}x{\isadigit{5}}ul\ {\isasymnoteq}\ {\isacharbrackleft}{\kern0pt}{\isacharbrackright}{\kern0pt}{\isachardoublequoteclose}%
\isadelimproof
\ %
\endisadelimproof
%
\isatagproof
\isacommand{by}\isamarkupfalse%
\ eval%
\endisatagproof
{\isafoldproof}%
%
\isadelimproof
%
\endisadelimproof
%
\begin{isamarkuptext}%
A Knight's circuit for the \isa{{\isacharparenleft}{\kern0pt}{\isadigit{8}}{\isasymtimes}{\isadigit{5}}{\isacharparenright}{\kern0pt}}-board.
  \begin{table}[H]
    \begin{tabular}{lllll}
      26 &  7 & 28 & 15 & 24 \\
      31 & 16 & 25 &  6 & 29 \\
       8 & 27 & 30 & 23 & 14 \\
      17 & 32 & 39 & 34 &  5 \\
      38 &  9 & 18 & 13 & 22 \\
      19 & 40 & 33 &  4 & 35 \\
      10 & 37 &  2 & 21 & 12 \\
       1 & 20 & 11 & 36 &  3
    \end{tabular}
  \end{table}%
\end{isamarkuptext}\isamarkuptrue%
\isacommand{abbreviation}\isamarkupfalse%
\ {\isachardoublequoteopen}kc{\isadigit{8}}x{\isadigit{5}}\ {\isasymequiv}\ the\ {\isacharparenleft}{\kern0pt}to{\isacharunderscore}{\kern0pt}path\ \isanewline
\ \ {\isacharbrackleft}{\kern0pt}{\isacharbrackleft}{\kern0pt}{\isadigit{2}}{\isadigit{6}}{\isacharcomma}{\kern0pt}{\isadigit{7}}{\isacharcomma}{\kern0pt}{\isadigit{2}}{\isadigit{8}}{\isacharcomma}{\kern0pt}{\isadigit{1}}{\isadigit{5}}{\isacharcomma}{\kern0pt}{\isadigit{2}}{\isadigit{4}}{\isacharbrackright}{\kern0pt}{\isacharcomma}{\kern0pt}\isanewline
\ \ {\isacharbrackleft}{\kern0pt}{\isadigit{3}}{\isadigit{1}}{\isacharcomma}{\kern0pt}{\isadigit{1}}{\isadigit{6}}{\isacharcomma}{\kern0pt}{\isadigit{2}}{\isadigit{5}}{\isacharcomma}{\kern0pt}{\isadigit{6}}{\isacharcomma}{\kern0pt}{\isadigit{2}}{\isadigit{9}}{\isacharbrackright}{\kern0pt}{\isacharcomma}{\kern0pt}\isanewline
\ \ {\isacharbrackleft}{\kern0pt}{\isadigit{8}}{\isacharcomma}{\kern0pt}{\isadigit{2}}{\isadigit{7}}{\isacharcomma}{\kern0pt}{\isadigit{3}}{\isadigit{0}}{\isacharcomma}{\kern0pt}{\isadigit{2}}{\isadigit{3}}{\isacharcomma}{\kern0pt}{\isadigit{1}}{\isadigit{4}}{\isacharbrackright}{\kern0pt}{\isacharcomma}{\kern0pt}\isanewline
\ \ {\isacharbrackleft}{\kern0pt}{\isadigit{1}}{\isadigit{7}}{\isacharcomma}{\kern0pt}{\isadigit{3}}{\isadigit{2}}{\isacharcomma}{\kern0pt}{\isadigit{3}}{\isadigit{9}}{\isacharcomma}{\kern0pt}{\isadigit{3}}{\isadigit{4}}{\isacharcomma}{\kern0pt}{\isadigit{5}}{\isacharbrackright}{\kern0pt}{\isacharcomma}{\kern0pt}\isanewline
\ \ {\isacharbrackleft}{\kern0pt}{\isadigit{3}}{\isadigit{8}}{\isacharcomma}{\kern0pt}{\isadigit{9}}{\isacharcomma}{\kern0pt}{\isadigit{1}}{\isadigit{8}}{\isacharcomma}{\kern0pt}{\isadigit{1}}{\isadigit{3}}{\isacharcomma}{\kern0pt}{\isadigit{2}}{\isadigit{2}}{\isacharbrackright}{\kern0pt}{\isacharcomma}{\kern0pt}\isanewline
\ \ {\isacharbrackleft}{\kern0pt}{\isadigit{1}}{\isadigit{9}}{\isacharcomma}{\kern0pt}{\isadigit{4}}{\isadigit{0}}{\isacharcomma}{\kern0pt}{\isadigit{3}}{\isadigit{3}}{\isacharcomma}{\kern0pt}{\isadigit{4}}{\isacharcomma}{\kern0pt}{\isadigit{3}}{\isadigit{5}}{\isacharbrackright}{\kern0pt}{\isacharcomma}{\kern0pt}\isanewline
\ \ {\isacharbrackleft}{\kern0pt}{\isadigit{1}}{\isadigit{0}}{\isacharcomma}{\kern0pt}{\isadigit{3}}{\isadigit{7}}{\isacharcomma}{\kern0pt}{\isadigit{2}}{\isacharcomma}{\kern0pt}{\isadigit{2}}{\isadigit{1}}{\isacharcomma}{\kern0pt}{\isadigit{1}}{\isadigit{2}}{\isacharbrackright}{\kern0pt}{\isacharcomma}{\kern0pt}\isanewline
\ \ {\isacharbrackleft}{\kern0pt}{\isadigit{1}}{\isacharcomma}{\kern0pt}{\isadigit{2}}{\isadigit{0}}{\isacharcomma}{\kern0pt}{\isadigit{1}}{\isadigit{1}}{\isacharcomma}{\kern0pt}{\isadigit{3}}{\isadigit{6}}{\isacharcomma}{\kern0pt}{\isadigit{3}}{\isacharbrackright}{\kern0pt}{\isacharbrackright}{\kern0pt}{\isacharparenright}{\kern0pt}{\isachardoublequoteclose}\isanewline
\isacommand{lemma}\isamarkupfalse%
\ kc{\isacharunderscore}{\kern0pt}{\isadigit{8}}x{\isadigit{5}}{\isacharcolon}{\kern0pt}\ {\isachardoublequoteopen}knights{\isacharunderscore}{\kern0pt}circuit\ b{\isadigit{8}}x{\isadigit{5}}\ kc{\isadigit{8}}x{\isadigit{5}}{\isachardoublequoteclose}\isanewline
%
\isadelimproof
\ \ %
\endisadelimproof
%
\isatagproof
\isacommand{by}\isamarkupfalse%
\ {\isacharparenleft}{\kern0pt}simp\ only{\isacharcolon}{\kern0pt}\ knights{\isacharunderscore}{\kern0pt}circuit{\isacharunderscore}{\kern0pt}exec{\isacharunderscore}{\kern0pt}simp{\isacharparenright}{\kern0pt}\ eval%
\endisatagproof
{\isafoldproof}%
%
\isadelimproof
\isanewline
%
\endisadelimproof
\isanewline
\isacommand{lemma}\isamarkupfalse%
\ kc{\isacharunderscore}{\kern0pt}{\isadigit{8}}x{\isadigit{5}}{\isacharunderscore}{\kern0pt}hd{\isacharcolon}{\kern0pt}\ {\isachardoublequoteopen}hd\ kc{\isadigit{8}}x{\isadigit{5}}\ {\isacharequal}{\kern0pt}\ {\isacharparenleft}{\kern0pt}{\isadigit{1}}{\isacharcomma}{\kern0pt}{\isadigit{1}}{\isacharparenright}{\kern0pt}{\isachardoublequoteclose}%
\isadelimproof
\ %
\endisadelimproof
%
\isatagproof
\isacommand{by}\isamarkupfalse%
\ eval%
\endisatagproof
{\isafoldproof}%
%
\isadelimproof
%
\endisadelimproof
\isanewline
\isanewline
\isacommand{lemma}\isamarkupfalse%
\ kc{\isacharunderscore}{\kern0pt}{\isadigit{8}}x{\isadigit{5}}{\isacharunderscore}{\kern0pt}last{\isacharcolon}{\kern0pt}\ {\isachardoublequoteopen}last\ kc{\isadigit{8}}x{\isadigit{5}}\ {\isacharequal}{\kern0pt}\ {\isacharparenleft}{\kern0pt}{\isadigit{3}}{\isacharcomma}{\kern0pt}{\isadigit{2}}{\isacharparenright}{\kern0pt}{\isachardoublequoteclose}%
\isadelimproof
\ %
\endisadelimproof
%
\isatagproof
\isacommand{by}\isamarkupfalse%
\ eval%
\endisatagproof
{\isafoldproof}%
%
\isadelimproof
%
\endisadelimproof
\isanewline
\isanewline
\isacommand{lemma}\isamarkupfalse%
\ kc{\isacharunderscore}{\kern0pt}{\isadigit{8}}x{\isadigit{5}}{\isacharunderscore}{\kern0pt}non{\isacharunderscore}{\kern0pt}nil{\isacharcolon}{\kern0pt}\ {\isachardoublequoteopen}kc{\isadigit{8}}x{\isadigit{5}}\ {\isasymnoteq}\ {\isacharbrackleft}{\kern0pt}{\isacharbrackright}{\kern0pt}{\isachardoublequoteclose}%
\isadelimproof
\ %
\endisadelimproof
%
\isatagproof
\isacommand{by}\isamarkupfalse%
\ eval%
\endisatagproof
{\isafoldproof}%
%
\isadelimproof
%
\endisadelimproof
\isanewline
\isanewline
\isacommand{lemma}\isamarkupfalse%
\ kc{\isacharunderscore}{\kern0pt}{\isadigit{8}}x{\isadigit{5}}{\isacharunderscore}{\kern0pt}si{\isacharcolon}{\kern0pt}\ {\isachardoublequoteopen}step{\isacharunderscore}{\kern0pt}in\ kc{\isadigit{8}}x{\isadigit{5}}\ {\isacharparenleft}{\kern0pt}{\isadigit{2}}{\isacharcomma}{\kern0pt}{\isadigit{4}}{\isacharparenright}{\kern0pt}\ {\isacharparenleft}{\kern0pt}{\isadigit{4}}{\isacharcomma}{\kern0pt}{\isadigit{5}}{\isacharparenright}{\kern0pt}{\isachardoublequoteclose}\ \ {\isacharparenleft}{\kern0pt}\isakeyword{is}\ {\isachardoublequoteopen}step{\isacharunderscore}{\kern0pt}in\ {\isacharquery}{\kern0pt}ps\ {\isacharunderscore}{\kern0pt}\ {\isacharunderscore}{\kern0pt}{\isachardoublequoteclose}{\isacharparenright}{\kern0pt}\isanewline
%
\isadelimproof
%
\endisadelimproof
%
\isatagproof
\isacommand{proof}\isamarkupfalse%
\ {\isacharminus}{\kern0pt}\isanewline
\ \ \isacommand{have}\isamarkupfalse%
\ {\isachardoublequoteopen}{\isadigit{0}}\ {\isacharless}{\kern0pt}\ {\isacharparenleft}{\kern0pt}{\isadigit{2}}{\isadigit{1}}{\isacharcolon}{\kern0pt}{\isacharcolon}{\kern0pt}nat{\isacharparenright}{\kern0pt}{\isachardoublequoteclose}\ {\isachardoublequoteopen}{\isadigit{2}}{\isadigit{1}}\ {\isacharless}{\kern0pt}\ length\ {\isacharquery}{\kern0pt}ps{\isachardoublequoteclose}\ {\isachardoublequoteopen}last\ {\isacharparenleft}{\kern0pt}take\ {\isadigit{2}}{\isadigit{1}}\ {\isacharquery}{\kern0pt}ps{\isacharparenright}{\kern0pt}\ {\isacharequal}{\kern0pt}\ {\isacharparenleft}{\kern0pt}{\isadigit{2}}{\isacharcomma}{\kern0pt}{\isadigit{4}}{\isacharparenright}{\kern0pt}{\isachardoublequoteclose}\ {\isachardoublequoteopen}hd\ {\isacharparenleft}{\kern0pt}drop\ {\isadigit{2}}{\isadigit{1}}\ {\isacharquery}{\kern0pt}ps{\isacharparenright}{\kern0pt}\ {\isacharequal}{\kern0pt}\ {\isacharparenleft}{\kern0pt}{\isadigit{4}}{\isacharcomma}{\kern0pt}{\isadigit{5}}{\isacharparenright}{\kern0pt}{\isachardoublequoteclose}\ \isanewline
\ \ \ \ \isacommand{by}\isamarkupfalse%
\ eval{\isacharplus}{\kern0pt}\isanewline
\ \ \isacommand{then}\isamarkupfalse%
\ \isacommand{show}\isamarkupfalse%
\ {\isacharquery}{\kern0pt}thesis\ \isacommand{unfolding}\isamarkupfalse%
\ step{\isacharunderscore}{\kern0pt}in{\isacharunderscore}{\kern0pt}def\ \isacommand{by}\isamarkupfalse%
\ blast\isanewline
\isacommand{qed}\isamarkupfalse%
%
\endisatagproof
{\isafoldproof}%
%
\isadelimproof
\isanewline
%
\endisadelimproof
\isanewline
\isacommand{abbreviation}\isamarkupfalse%
\ {\isachardoublequoteopen}b{\isadigit{8}}x{\isadigit{6}}\ {\isasymequiv}\ board\ {\isadigit{8}}\ {\isadigit{6}}{\isachardoublequoteclose}%
\begin{isamarkuptext}%
A Knight's path for the \isa{{\isacharparenleft}{\kern0pt}{\isadigit{8}}{\isasymtimes}{\isadigit{6}}{\isacharparenright}{\kern0pt}}-board that starts in the lower-left and ends in the 
upper-left.
  \begin{table}[H]
    \begin{tabular}{llllll}
      42 & 11 & 26 &  9 & 34 & 13 \\
      25 & 48 & 43 & 12 & 27 &  8 \\
      44 & 41 & 10 & 33 & 14 & 35 \\
      47 & 24 & 45 & 20 &  7 & 28 \\
      40 & 19 & 32 &  3 & 36 & 15 \\
      23 & 46 & 21 &  6 & 29 &  4 \\
      18 & 39 &  2 & 31 & 16 & 37 \\
       1 & 22 & 17 & 38 &  5 & 30
    \end{tabular}
  \end{table}%
\end{isamarkuptext}\isamarkuptrue%
\isacommand{abbreviation}\isamarkupfalse%
\ {\isachardoublequoteopen}kp{\isadigit{8}}x{\isadigit{6}}ul\ {\isasymequiv}\ the\ {\isacharparenleft}{\kern0pt}to{\isacharunderscore}{\kern0pt}path\ \isanewline
\ \ {\isacharbrackleft}{\kern0pt}{\isacharbrackleft}{\kern0pt}{\isadigit{4}}{\isadigit{2}}{\isacharcomma}{\kern0pt}{\isadigit{1}}{\isadigit{1}}{\isacharcomma}{\kern0pt}{\isadigit{2}}{\isadigit{6}}{\isacharcomma}{\kern0pt}{\isadigit{9}}{\isacharcomma}{\kern0pt}{\isadigit{3}}{\isadigit{4}}{\isacharcomma}{\kern0pt}{\isadigit{1}}{\isadigit{3}}{\isacharbrackright}{\kern0pt}{\isacharcomma}{\kern0pt}\isanewline
\ \ {\isacharbrackleft}{\kern0pt}{\isadigit{2}}{\isadigit{5}}{\isacharcomma}{\kern0pt}{\isadigit{4}}{\isadigit{8}}{\isacharcomma}{\kern0pt}{\isadigit{4}}{\isadigit{3}}{\isacharcomma}{\kern0pt}{\isadigit{1}}{\isadigit{2}}{\isacharcomma}{\kern0pt}{\isadigit{2}}{\isadigit{7}}{\isacharcomma}{\kern0pt}{\isadigit{8}}{\isacharbrackright}{\kern0pt}{\isacharcomma}{\kern0pt}\isanewline
\ \ {\isacharbrackleft}{\kern0pt}{\isadigit{4}}{\isadigit{4}}{\isacharcomma}{\kern0pt}{\isadigit{4}}{\isadigit{1}}{\isacharcomma}{\kern0pt}{\isadigit{1}}{\isadigit{0}}{\isacharcomma}{\kern0pt}{\isadigit{3}}{\isadigit{3}}{\isacharcomma}{\kern0pt}{\isadigit{1}}{\isadigit{4}}{\isacharcomma}{\kern0pt}{\isadigit{3}}{\isadigit{5}}{\isacharbrackright}{\kern0pt}{\isacharcomma}{\kern0pt}\isanewline
\ \ {\isacharbrackleft}{\kern0pt}{\isadigit{4}}{\isadigit{7}}{\isacharcomma}{\kern0pt}{\isadigit{2}}{\isadigit{4}}{\isacharcomma}{\kern0pt}{\isadigit{4}}{\isadigit{5}}{\isacharcomma}{\kern0pt}{\isadigit{2}}{\isadigit{0}}{\isacharcomma}{\kern0pt}{\isadigit{7}}{\isacharcomma}{\kern0pt}{\isadigit{2}}{\isadigit{8}}{\isacharbrackright}{\kern0pt}{\isacharcomma}{\kern0pt}\isanewline
\ \ {\isacharbrackleft}{\kern0pt}{\isadigit{4}}{\isadigit{0}}{\isacharcomma}{\kern0pt}{\isadigit{1}}{\isadigit{9}}{\isacharcomma}{\kern0pt}{\isadigit{3}}{\isadigit{2}}{\isacharcomma}{\kern0pt}{\isadigit{3}}{\isacharcomma}{\kern0pt}{\isadigit{3}}{\isadigit{6}}{\isacharcomma}{\kern0pt}{\isadigit{1}}{\isadigit{5}}{\isacharbrackright}{\kern0pt}{\isacharcomma}{\kern0pt}\isanewline
\ \ {\isacharbrackleft}{\kern0pt}{\isadigit{2}}{\isadigit{3}}{\isacharcomma}{\kern0pt}{\isadigit{4}}{\isadigit{6}}{\isacharcomma}{\kern0pt}{\isadigit{2}}{\isadigit{1}}{\isacharcomma}{\kern0pt}{\isadigit{6}}{\isacharcomma}{\kern0pt}{\isadigit{2}}{\isadigit{9}}{\isacharcomma}{\kern0pt}{\isadigit{4}}{\isacharbrackright}{\kern0pt}{\isacharcomma}{\kern0pt}\isanewline
\ \ {\isacharbrackleft}{\kern0pt}{\isadigit{1}}{\isadigit{8}}{\isacharcomma}{\kern0pt}{\isadigit{3}}{\isadigit{9}}{\isacharcomma}{\kern0pt}{\isadigit{2}}{\isacharcomma}{\kern0pt}{\isadigit{3}}{\isadigit{1}}{\isacharcomma}{\kern0pt}{\isadigit{1}}{\isadigit{6}}{\isacharcomma}{\kern0pt}{\isadigit{3}}{\isadigit{7}}{\isacharbrackright}{\kern0pt}{\isacharcomma}{\kern0pt}\isanewline
\ \ {\isacharbrackleft}{\kern0pt}{\isadigit{1}}{\isacharcomma}{\kern0pt}{\isadigit{2}}{\isadigit{2}}{\isacharcomma}{\kern0pt}{\isadigit{1}}{\isadigit{7}}{\isacharcomma}{\kern0pt}{\isadigit{3}}{\isadigit{8}}{\isacharcomma}{\kern0pt}{\isadigit{5}}{\isacharcomma}{\kern0pt}{\isadigit{3}}{\isadigit{0}}{\isacharbrackright}{\kern0pt}{\isacharbrackright}{\kern0pt}{\isacharparenright}{\kern0pt}{\isachardoublequoteclose}\isanewline
\isacommand{lemma}\isamarkupfalse%
\ kp{\isacharunderscore}{\kern0pt}{\isadigit{8}}x{\isadigit{6}}{\isacharunderscore}{\kern0pt}ul{\isacharcolon}{\kern0pt}\ {\isachardoublequoteopen}knights{\isacharunderscore}{\kern0pt}path\ b{\isadigit{8}}x{\isadigit{6}}\ kp{\isadigit{8}}x{\isadigit{6}}ul{\isachardoublequoteclose}\isanewline
%
\isadelimproof
\ \ %
\endisadelimproof
%
\isatagproof
\isacommand{by}\isamarkupfalse%
\ {\isacharparenleft}{\kern0pt}simp\ only{\isacharcolon}{\kern0pt}\ knights{\isacharunderscore}{\kern0pt}path{\isacharunderscore}{\kern0pt}exec{\isacharunderscore}{\kern0pt}simp{\isacharparenright}{\kern0pt}\ eval%
\endisatagproof
{\isafoldproof}%
%
\isadelimproof
\isanewline
%
\endisadelimproof
\isanewline
\isacommand{lemma}\isamarkupfalse%
\ kp{\isacharunderscore}{\kern0pt}{\isadigit{8}}x{\isadigit{6}}{\isacharunderscore}{\kern0pt}ul{\isacharunderscore}{\kern0pt}hd{\isacharcolon}{\kern0pt}\ {\isachardoublequoteopen}hd\ kp{\isadigit{8}}x{\isadigit{6}}ul\ {\isacharequal}{\kern0pt}\ {\isacharparenleft}{\kern0pt}{\isadigit{1}}{\isacharcomma}{\kern0pt}{\isadigit{1}}{\isacharparenright}{\kern0pt}{\isachardoublequoteclose}%
\isadelimproof
\ %
\endisadelimproof
%
\isatagproof
\isacommand{by}\isamarkupfalse%
\ eval%
\endisatagproof
{\isafoldproof}%
%
\isadelimproof
%
\endisadelimproof
\isanewline
\isanewline
\isacommand{lemma}\isamarkupfalse%
\ kp{\isacharunderscore}{\kern0pt}{\isadigit{8}}x{\isadigit{6}}{\isacharunderscore}{\kern0pt}ul{\isacharunderscore}{\kern0pt}last{\isacharcolon}{\kern0pt}\ {\isachardoublequoteopen}last\ kp{\isadigit{8}}x{\isadigit{6}}ul\ {\isacharequal}{\kern0pt}\ {\isacharparenleft}{\kern0pt}{\isadigit{7}}{\isacharcomma}{\kern0pt}{\isadigit{2}}{\isacharparenright}{\kern0pt}{\isachardoublequoteclose}%
\isadelimproof
\ %
\endisadelimproof
%
\isatagproof
\isacommand{by}\isamarkupfalse%
\ eval%
\endisatagproof
{\isafoldproof}%
%
\isadelimproof
%
\endisadelimproof
\isanewline
\isanewline
\isacommand{lemma}\isamarkupfalse%
\ kp{\isacharunderscore}{\kern0pt}{\isadigit{8}}x{\isadigit{6}}{\isacharunderscore}{\kern0pt}ul{\isacharunderscore}{\kern0pt}non{\isacharunderscore}{\kern0pt}nil{\isacharcolon}{\kern0pt}\ {\isachardoublequoteopen}kp{\isadigit{8}}x{\isadigit{6}}ul\ {\isasymnoteq}\ {\isacharbrackleft}{\kern0pt}{\isacharbrackright}{\kern0pt}{\isachardoublequoteclose}%
\isadelimproof
\ %
\endisadelimproof
%
\isatagproof
\isacommand{by}\isamarkupfalse%
\ eval%
\endisatagproof
{\isafoldproof}%
%
\isadelimproof
%
\endisadelimproof
%
\begin{isamarkuptext}%
A Knight's circuit for the \isa{{\isacharparenleft}{\kern0pt}{\isadigit{8}}{\isasymtimes}{\isadigit{6}}{\isacharparenright}{\kern0pt}}-board. I have reversed circuit s.t. the circuit steps 
from \isa{{\isacharparenleft}{\kern0pt}{\isadigit{2}}{\isacharcomma}{\kern0pt}{\isadigit{5}}{\isacharparenright}{\kern0pt}} to \isa{{\isacharparenleft}{\kern0pt}{\isadigit{4}}{\isacharcomma}{\kern0pt}{\isadigit{6}}{\isacharparenright}{\kern0pt}} and not the other way around. This makes the proofs easier.
  \begin{table}[H]
    \begin{tabular}{llllll}
       8 & 29 & 24 & 45 & 12 & 37 \\
      25 & 46 &  9 & 38 & 23 & 44 \\
      30 &  7 & 28 & 13 & 36 & 11 \\
      47 & 26 & 39 & 10 & 43 & 22 \\
       6 & 31 &  4 & 27 & 14 & 35 \\
       3 & 48 & 17 & 40 & 21 & 42 \\
      32 &  5 &  2 & 19 & 34 & 15 \\
       1 & 18 & 33 & 16 & 41 & 20
    \end{tabular}
  \end{table}%
\end{isamarkuptext}\isamarkuptrue%
\isacommand{abbreviation}\isamarkupfalse%
\ {\isachardoublequoteopen}kc{\isadigit{8}}x{\isadigit{6}}\ {\isasymequiv}\ the\ {\isacharparenleft}{\kern0pt}to{\isacharunderscore}{\kern0pt}path\ \isanewline
\ \ {\isacharbrackleft}{\kern0pt}{\isacharbrackleft}{\kern0pt}{\isadigit{8}}{\isacharcomma}{\kern0pt}{\isadigit{2}}{\isadigit{9}}{\isacharcomma}{\kern0pt}{\isadigit{2}}{\isadigit{4}}{\isacharcomma}{\kern0pt}{\isadigit{4}}{\isadigit{5}}{\isacharcomma}{\kern0pt}{\isadigit{1}}{\isadigit{2}}{\isacharcomma}{\kern0pt}{\isadigit{3}}{\isadigit{7}}{\isacharbrackright}{\kern0pt}{\isacharcomma}{\kern0pt}\isanewline
\ \ {\isacharbrackleft}{\kern0pt}{\isadigit{2}}{\isadigit{5}}{\isacharcomma}{\kern0pt}{\isadigit{4}}{\isadigit{6}}{\isacharcomma}{\kern0pt}{\isadigit{9}}{\isacharcomma}{\kern0pt}{\isadigit{3}}{\isadigit{8}}{\isacharcomma}{\kern0pt}{\isadigit{2}}{\isadigit{3}}{\isacharcomma}{\kern0pt}{\isadigit{4}}{\isadigit{4}}{\isacharbrackright}{\kern0pt}{\isacharcomma}{\kern0pt}\isanewline
\ \ {\isacharbrackleft}{\kern0pt}{\isadigit{3}}{\isadigit{0}}{\isacharcomma}{\kern0pt}{\isadigit{7}}{\isacharcomma}{\kern0pt}{\isadigit{2}}{\isadigit{8}}{\isacharcomma}{\kern0pt}{\isadigit{1}}{\isadigit{3}}{\isacharcomma}{\kern0pt}{\isadigit{3}}{\isadigit{6}}{\isacharcomma}{\kern0pt}{\isadigit{1}}{\isadigit{1}}{\isacharbrackright}{\kern0pt}{\isacharcomma}{\kern0pt}\isanewline
\ \ {\isacharbrackleft}{\kern0pt}{\isadigit{4}}{\isadigit{7}}{\isacharcomma}{\kern0pt}{\isadigit{2}}{\isadigit{6}}{\isacharcomma}{\kern0pt}{\isadigit{3}}{\isadigit{9}}{\isacharcomma}{\kern0pt}{\isadigit{1}}{\isadigit{0}}{\isacharcomma}{\kern0pt}{\isadigit{4}}{\isadigit{3}}{\isacharcomma}{\kern0pt}{\isadigit{2}}{\isadigit{2}}{\isacharbrackright}{\kern0pt}{\isacharcomma}{\kern0pt}\isanewline
\ \ {\isacharbrackleft}{\kern0pt}{\isadigit{6}}{\isacharcomma}{\kern0pt}{\isadigit{3}}{\isadigit{1}}{\isacharcomma}{\kern0pt}{\isadigit{4}}{\isacharcomma}{\kern0pt}{\isadigit{2}}{\isadigit{7}}{\isacharcomma}{\kern0pt}{\isadigit{1}}{\isadigit{4}}{\isacharcomma}{\kern0pt}{\isadigit{3}}{\isadigit{5}}{\isacharbrackright}{\kern0pt}{\isacharcomma}{\kern0pt}\isanewline
\ \ {\isacharbrackleft}{\kern0pt}{\isadigit{3}}{\isacharcomma}{\kern0pt}{\isadigit{4}}{\isadigit{8}}{\isacharcomma}{\kern0pt}{\isadigit{1}}{\isadigit{7}}{\isacharcomma}{\kern0pt}{\isadigit{4}}{\isadigit{0}}{\isacharcomma}{\kern0pt}{\isadigit{2}}{\isadigit{1}}{\isacharcomma}{\kern0pt}{\isadigit{4}}{\isadigit{2}}{\isacharbrackright}{\kern0pt}{\isacharcomma}{\kern0pt}\isanewline
\ \ {\isacharbrackleft}{\kern0pt}{\isadigit{3}}{\isadigit{2}}{\isacharcomma}{\kern0pt}{\isadigit{5}}{\isacharcomma}{\kern0pt}{\isadigit{2}}{\isacharcomma}{\kern0pt}{\isadigit{1}}{\isadigit{9}}{\isacharcomma}{\kern0pt}{\isadigit{3}}{\isadigit{4}}{\isacharcomma}{\kern0pt}{\isadigit{1}}{\isadigit{5}}{\isacharbrackright}{\kern0pt}{\isacharcomma}{\kern0pt}\isanewline
\ \ {\isacharbrackleft}{\kern0pt}{\isadigit{1}}{\isacharcomma}{\kern0pt}{\isadigit{1}}{\isadigit{8}}{\isacharcomma}{\kern0pt}{\isadigit{3}}{\isadigit{3}}{\isacharcomma}{\kern0pt}{\isadigit{1}}{\isadigit{6}}{\isacharcomma}{\kern0pt}{\isadigit{4}}{\isadigit{1}}{\isacharcomma}{\kern0pt}{\isadigit{2}}{\isadigit{0}}{\isacharbrackright}{\kern0pt}{\isacharbrackright}{\kern0pt}{\isacharparenright}{\kern0pt}{\isachardoublequoteclose}\isanewline
\isacommand{lemma}\isamarkupfalse%
\ kc{\isacharunderscore}{\kern0pt}{\isadigit{8}}x{\isadigit{6}}{\isacharcolon}{\kern0pt}\ {\isachardoublequoteopen}knights{\isacharunderscore}{\kern0pt}circuit\ b{\isadigit{8}}x{\isadigit{6}}\ kc{\isadigit{8}}x{\isadigit{6}}{\isachardoublequoteclose}\isanewline
%
\isadelimproof
\ \ %
\endisadelimproof
%
\isatagproof
\isacommand{by}\isamarkupfalse%
\ {\isacharparenleft}{\kern0pt}simp\ only{\isacharcolon}{\kern0pt}\ knights{\isacharunderscore}{\kern0pt}circuit{\isacharunderscore}{\kern0pt}exec{\isacharunderscore}{\kern0pt}simp{\isacharparenright}{\kern0pt}\ eval%
\endisatagproof
{\isafoldproof}%
%
\isadelimproof
\isanewline
%
\endisadelimproof
\isanewline
\isacommand{lemma}\isamarkupfalse%
\ kc{\isacharunderscore}{\kern0pt}{\isadigit{8}}x{\isadigit{6}}{\isacharunderscore}{\kern0pt}hd{\isacharcolon}{\kern0pt}\ {\isachardoublequoteopen}hd\ kc{\isadigit{8}}x{\isadigit{6}}\ {\isacharequal}{\kern0pt}\ {\isacharparenleft}{\kern0pt}{\isadigit{1}}{\isacharcomma}{\kern0pt}{\isadigit{1}}{\isacharparenright}{\kern0pt}{\isachardoublequoteclose}%
\isadelimproof
\ %
\endisadelimproof
%
\isatagproof
\isacommand{by}\isamarkupfalse%
\ eval%
\endisatagproof
{\isafoldproof}%
%
\isadelimproof
%
\endisadelimproof
\isanewline
\isanewline
\isacommand{lemma}\isamarkupfalse%
\ kc{\isacharunderscore}{\kern0pt}{\isadigit{8}}x{\isadigit{6}}{\isacharunderscore}{\kern0pt}non{\isacharunderscore}{\kern0pt}nil{\isacharcolon}{\kern0pt}\ {\isachardoublequoteopen}kc{\isadigit{8}}x{\isadigit{6}}\ {\isasymnoteq}\ {\isacharbrackleft}{\kern0pt}{\isacharbrackright}{\kern0pt}{\isachardoublequoteclose}%
\isadelimproof
\ %
\endisadelimproof
%
\isatagproof
\isacommand{by}\isamarkupfalse%
\ eval%
\endisatagproof
{\isafoldproof}%
%
\isadelimproof
%
\endisadelimproof
\isanewline
\isanewline
\isacommand{lemma}\isamarkupfalse%
\ kc{\isacharunderscore}{\kern0pt}{\isadigit{8}}x{\isadigit{6}}{\isacharunderscore}{\kern0pt}si{\isacharcolon}{\kern0pt}\ {\isachardoublequoteopen}step{\isacharunderscore}{\kern0pt}in\ kc{\isadigit{8}}x{\isadigit{6}}\ {\isacharparenleft}{\kern0pt}{\isadigit{2}}{\isacharcomma}{\kern0pt}{\isadigit{5}}{\isacharparenright}{\kern0pt}\ {\isacharparenleft}{\kern0pt}{\isadigit{4}}{\isacharcomma}{\kern0pt}{\isadigit{6}}{\isacharparenright}{\kern0pt}{\isachardoublequoteclose}\ {\isacharparenleft}{\kern0pt}\isakeyword{is}\ {\isachardoublequoteopen}step{\isacharunderscore}{\kern0pt}in\ {\isacharquery}{\kern0pt}ps\ {\isacharunderscore}{\kern0pt}\ {\isacharunderscore}{\kern0pt}{\isachardoublequoteclose}{\isacharparenright}{\kern0pt}\isanewline
%
\isadelimproof
%
\endisadelimproof
%
\isatagproof
\isacommand{proof}\isamarkupfalse%
\ {\isacharminus}{\kern0pt}\isanewline
\ \ \isacommand{have}\isamarkupfalse%
\ {\isachardoublequoteopen}{\isadigit{0}}\ {\isacharless}{\kern0pt}\ {\isacharparenleft}{\kern0pt}{\isadigit{3}}{\isadigit{4}}{\isacharcolon}{\kern0pt}{\isacharcolon}{\kern0pt}nat{\isacharparenright}{\kern0pt}{\isachardoublequoteclose}\ {\isachardoublequoteopen}{\isadigit{3}}{\isadigit{4}}\ {\isacharless}{\kern0pt}\ length\ {\isacharquery}{\kern0pt}ps{\isachardoublequoteclose}\ \isanewline
\ \ \ \ \ \ \ \ {\isachardoublequoteopen}last\ {\isacharparenleft}{\kern0pt}take\ {\isadigit{3}}{\isadigit{4}}\ {\isacharquery}{\kern0pt}ps{\isacharparenright}{\kern0pt}\ {\isacharequal}{\kern0pt}\ {\isacharparenleft}{\kern0pt}{\isadigit{2}}{\isacharcomma}{\kern0pt}{\isadigit{5}}{\isacharparenright}{\kern0pt}{\isachardoublequoteclose}\ {\isachardoublequoteopen}hd\ {\isacharparenleft}{\kern0pt}drop\ {\isadigit{3}}{\isadigit{4}}\ {\isacharquery}{\kern0pt}ps{\isacharparenright}{\kern0pt}\ {\isacharequal}{\kern0pt}\ {\isacharparenleft}{\kern0pt}{\isadigit{4}}{\isacharcomma}{\kern0pt}{\isadigit{6}}{\isacharparenright}{\kern0pt}{\isachardoublequoteclose}\ \isacommand{by}\isamarkupfalse%
\ eval{\isacharplus}{\kern0pt}\isanewline
\ \ \isacommand{then}\isamarkupfalse%
\ \isacommand{show}\isamarkupfalse%
\ {\isacharquery}{\kern0pt}thesis\ \isacommand{unfolding}\isamarkupfalse%
\ step{\isacharunderscore}{\kern0pt}in{\isacharunderscore}{\kern0pt}def\ \isacommand{by}\isamarkupfalse%
\ blast\isanewline
\isacommand{qed}\isamarkupfalse%
%
\endisatagproof
{\isafoldproof}%
%
\isadelimproof
\isanewline
%
\endisadelimproof
\isanewline
\isacommand{abbreviation}\isamarkupfalse%
\ {\isachardoublequoteopen}b{\isadigit{8}}x{\isadigit{7}}\ {\isasymequiv}\ board\ {\isadigit{8}}\ {\isadigit{7}}{\isachardoublequoteclose}%
\begin{isamarkuptext}%
A Knight's path for the \isa{{\isacharparenleft}{\kern0pt}{\isadigit{8}}{\isasymtimes}{\isadigit{7}}{\isacharparenright}{\kern0pt}}-board that starts in the lower-left and ends in the 
upper-left.
  \begin{table}[H]
    \begin{tabular}{lllllll}
      38 & 19 &  6 & 55 & 46 & 21 &  8 \\
       5 & 56 & 39 & 20 &  7 & 54 & 45 \\
      18 & 37 &  4 & 47 & 34 &  9 & 22 \\
       3 & 48 & 35 & 40 & 53 & 44 & 33 \\
      36 & 17 & 52 & 49 & 32 & 23 & 10 \\
      51 &  2 & 29 & 14 & 41 & 26 & 43 \\
      16 & 13 & 50 & 31 & 28 & 11 & 24 \\
       1 & 30 & 15 & 12 & 25 & 42 & 27
    \end{tabular}
  \end{table}%
\end{isamarkuptext}\isamarkuptrue%
\isacommand{abbreviation}\isamarkupfalse%
\ {\isachardoublequoteopen}kp{\isadigit{8}}x{\isadigit{7}}ul\ {\isasymequiv}\ the\ {\isacharparenleft}{\kern0pt}to{\isacharunderscore}{\kern0pt}path\ \isanewline
\ \ {\isacharbrackleft}{\kern0pt}{\isacharbrackleft}{\kern0pt}{\isadigit{3}}{\isadigit{8}}{\isacharcomma}{\kern0pt}{\isadigit{1}}{\isadigit{9}}{\isacharcomma}{\kern0pt}{\isadigit{6}}{\isacharcomma}{\kern0pt}{\isadigit{5}}{\isadigit{5}}{\isacharcomma}{\kern0pt}{\isadigit{4}}{\isadigit{6}}{\isacharcomma}{\kern0pt}{\isadigit{2}}{\isadigit{1}}{\isacharcomma}{\kern0pt}{\isadigit{8}}{\isacharbrackright}{\kern0pt}{\isacharcomma}{\kern0pt}\isanewline
\ \ {\isacharbrackleft}{\kern0pt}{\isadigit{5}}{\isacharcomma}{\kern0pt}{\isadigit{5}}{\isadigit{6}}{\isacharcomma}{\kern0pt}{\isadigit{3}}{\isadigit{9}}{\isacharcomma}{\kern0pt}{\isadigit{2}}{\isadigit{0}}{\isacharcomma}{\kern0pt}{\isadigit{7}}{\isacharcomma}{\kern0pt}{\isadigit{5}}{\isadigit{4}}{\isacharcomma}{\kern0pt}{\isadigit{4}}{\isadigit{5}}{\isacharbrackright}{\kern0pt}{\isacharcomma}{\kern0pt}\isanewline
\ \ {\isacharbrackleft}{\kern0pt}{\isadigit{1}}{\isadigit{8}}{\isacharcomma}{\kern0pt}{\isadigit{3}}{\isadigit{7}}{\isacharcomma}{\kern0pt}{\isadigit{4}}{\isacharcomma}{\kern0pt}{\isadigit{4}}{\isadigit{7}}{\isacharcomma}{\kern0pt}{\isadigit{3}}{\isadigit{4}}{\isacharcomma}{\kern0pt}{\isadigit{9}}{\isacharcomma}{\kern0pt}{\isadigit{2}}{\isadigit{2}}{\isacharbrackright}{\kern0pt}{\isacharcomma}{\kern0pt}\isanewline
\ \ {\isacharbrackleft}{\kern0pt}{\isadigit{3}}{\isacharcomma}{\kern0pt}{\isadigit{4}}{\isadigit{8}}{\isacharcomma}{\kern0pt}{\isadigit{3}}{\isadigit{5}}{\isacharcomma}{\kern0pt}{\isadigit{4}}{\isadigit{0}}{\isacharcomma}{\kern0pt}{\isadigit{5}}{\isadigit{3}}{\isacharcomma}{\kern0pt}{\isadigit{4}}{\isadigit{4}}{\isacharcomma}{\kern0pt}{\isadigit{3}}{\isadigit{3}}{\isacharbrackright}{\kern0pt}{\isacharcomma}{\kern0pt}\isanewline
\ \ {\isacharbrackleft}{\kern0pt}{\isadigit{3}}{\isadigit{6}}{\isacharcomma}{\kern0pt}{\isadigit{1}}{\isadigit{7}}{\isacharcomma}{\kern0pt}{\isadigit{5}}{\isadigit{2}}{\isacharcomma}{\kern0pt}{\isadigit{4}}{\isadigit{9}}{\isacharcomma}{\kern0pt}{\isadigit{3}}{\isadigit{2}}{\isacharcomma}{\kern0pt}{\isadigit{2}}{\isadigit{3}}{\isacharcomma}{\kern0pt}{\isadigit{1}}{\isadigit{0}}{\isacharbrackright}{\kern0pt}{\isacharcomma}{\kern0pt}\isanewline
\ \ {\isacharbrackleft}{\kern0pt}{\isadigit{5}}{\isadigit{1}}{\isacharcomma}{\kern0pt}{\isadigit{2}}{\isacharcomma}{\kern0pt}{\isadigit{2}}{\isadigit{9}}{\isacharcomma}{\kern0pt}{\isadigit{1}}{\isadigit{4}}{\isacharcomma}{\kern0pt}{\isadigit{4}}{\isadigit{1}}{\isacharcomma}{\kern0pt}{\isadigit{2}}{\isadigit{6}}{\isacharcomma}{\kern0pt}{\isadigit{4}}{\isadigit{3}}{\isacharbrackright}{\kern0pt}{\isacharcomma}{\kern0pt}\isanewline
\ \ {\isacharbrackleft}{\kern0pt}{\isadigit{1}}{\isadigit{6}}{\isacharcomma}{\kern0pt}{\isadigit{1}}{\isadigit{3}}{\isacharcomma}{\kern0pt}{\isadigit{5}}{\isadigit{0}}{\isacharcomma}{\kern0pt}{\isadigit{3}}{\isadigit{1}}{\isacharcomma}{\kern0pt}{\isadigit{2}}{\isadigit{8}}{\isacharcomma}{\kern0pt}{\isadigit{1}}{\isadigit{1}}{\isacharcomma}{\kern0pt}{\isadigit{2}}{\isadigit{4}}{\isacharbrackright}{\kern0pt}{\isacharcomma}{\kern0pt}\isanewline
\ \ {\isacharbrackleft}{\kern0pt}{\isadigit{1}}{\isacharcomma}{\kern0pt}{\isadigit{3}}{\isadigit{0}}{\isacharcomma}{\kern0pt}{\isadigit{1}}{\isadigit{5}}{\isacharcomma}{\kern0pt}{\isadigit{1}}{\isadigit{2}}{\isacharcomma}{\kern0pt}{\isadigit{2}}{\isadigit{5}}{\isacharcomma}{\kern0pt}{\isadigit{4}}{\isadigit{2}}{\isacharcomma}{\kern0pt}{\isadigit{2}}{\isadigit{7}}{\isacharbrackright}{\kern0pt}{\isacharbrackright}{\kern0pt}{\isacharparenright}{\kern0pt}{\isachardoublequoteclose}\isanewline
\isacommand{lemma}\isamarkupfalse%
\ kp{\isacharunderscore}{\kern0pt}{\isadigit{8}}x{\isadigit{7}}{\isacharunderscore}{\kern0pt}ul{\isacharcolon}{\kern0pt}\ {\isachardoublequoteopen}knights{\isacharunderscore}{\kern0pt}path\ b{\isadigit{8}}x{\isadigit{7}}\ kp{\isadigit{8}}x{\isadigit{7}}ul{\isachardoublequoteclose}\isanewline
%
\isadelimproof
\ \ %
\endisadelimproof
%
\isatagproof
\isacommand{by}\isamarkupfalse%
\ {\isacharparenleft}{\kern0pt}simp\ only{\isacharcolon}{\kern0pt}\ knights{\isacharunderscore}{\kern0pt}path{\isacharunderscore}{\kern0pt}exec{\isacharunderscore}{\kern0pt}simp{\isacharparenright}{\kern0pt}\ eval%
\endisatagproof
{\isafoldproof}%
%
\isadelimproof
\isanewline
%
\endisadelimproof
\isanewline
\isacommand{lemma}\isamarkupfalse%
\ kp{\isacharunderscore}{\kern0pt}{\isadigit{8}}x{\isadigit{7}}{\isacharunderscore}{\kern0pt}ul{\isacharunderscore}{\kern0pt}hd{\isacharcolon}{\kern0pt}\ {\isachardoublequoteopen}hd\ kp{\isadigit{8}}x{\isadigit{7}}ul\ {\isacharequal}{\kern0pt}\ {\isacharparenleft}{\kern0pt}{\isadigit{1}}{\isacharcomma}{\kern0pt}{\isadigit{1}}{\isacharparenright}{\kern0pt}{\isachardoublequoteclose}%
\isadelimproof
\ %
\endisadelimproof
%
\isatagproof
\isacommand{by}\isamarkupfalse%
\ eval%
\endisatagproof
{\isafoldproof}%
%
\isadelimproof
%
\endisadelimproof
\isanewline
\isanewline
\isacommand{lemma}\isamarkupfalse%
\ kp{\isacharunderscore}{\kern0pt}{\isadigit{8}}x{\isadigit{7}}{\isacharunderscore}{\kern0pt}ul{\isacharunderscore}{\kern0pt}last{\isacharcolon}{\kern0pt}\ {\isachardoublequoteopen}last\ kp{\isadigit{8}}x{\isadigit{7}}ul\ {\isacharequal}{\kern0pt}\ {\isacharparenleft}{\kern0pt}{\isadigit{7}}{\isacharcomma}{\kern0pt}{\isadigit{2}}{\isacharparenright}{\kern0pt}{\isachardoublequoteclose}%
\isadelimproof
\ %
\endisadelimproof
%
\isatagproof
\isacommand{by}\isamarkupfalse%
\ eval%
\endisatagproof
{\isafoldproof}%
%
\isadelimproof
%
\endisadelimproof
\isanewline
\isanewline
\isacommand{lemma}\isamarkupfalse%
\ kp{\isacharunderscore}{\kern0pt}{\isadigit{8}}x{\isadigit{7}}{\isacharunderscore}{\kern0pt}ul{\isacharunderscore}{\kern0pt}non{\isacharunderscore}{\kern0pt}nil{\isacharcolon}{\kern0pt}\ {\isachardoublequoteopen}kp{\isadigit{8}}x{\isadigit{7}}ul\ {\isasymnoteq}\ {\isacharbrackleft}{\kern0pt}{\isacharbrackright}{\kern0pt}{\isachardoublequoteclose}%
\isadelimproof
\ %
\endisadelimproof
%
\isatagproof
\isacommand{by}\isamarkupfalse%
\ eval%
\endisatagproof
{\isafoldproof}%
%
\isadelimproof
%
\endisadelimproof
%
\begin{isamarkuptext}%
A Knight's circuit for the \isa{{\isacharparenleft}{\kern0pt}{\isadigit{8}}{\isasymtimes}{\isadigit{7}}{\isacharparenright}{\kern0pt}}-board. I have reversed circuit s.t. the circuit steps 
from \isa{{\isacharparenleft}{\kern0pt}{\isadigit{2}}{\isacharcomma}{\kern0pt}{\isadigit{6}}{\isacharparenright}{\kern0pt}} to \isa{{\isacharparenleft}{\kern0pt}{\isadigit{4}}{\isacharcomma}{\kern0pt}{\isadigit{7}}{\isacharparenright}{\kern0pt}} and not the other way around. This makes the proofs easier.
  \begin{table}[H]
    \begin{tabular}{lllllll}
      36 & 31 & 18 & 53 & 20 & 29 & 44 \\
      17 & 54 & 35 & 30 & 45 & 52 & 21 \\
      32 & 37 & 46 & 19 &  8 & 43 & 28 \\
      55 & 16 &  7 & 34 & 27 & 22 & 51 \\
      38 & 33 & 26 & 47 &  6 &  9 & 42 \\
       3 & 56 & 15 & 12 & 25 & 50 & 23 \\
      14 & 39 &  2 &  5 & 48 & 41 & 10 \\
       1 &  4 & 13 & 40 & 11 & 24 & 49 
    \end{tabular}
  \end{table}%
\end{isamarkuptext}\isamarkuptrue%
\isacommand{abbreviation}\isamarkupfalse%
\ {\isachardoublequoteopen}kc{\isadigit{8}}x{\isadigit{7}}\ {\isasymequiv}\ the\ {\isacharparenleft}{\kern0pt}to{\isacharunderscore}{\kern0pt}path\ \isanewline
\ \ {\isacharbrackleft}{\kern0pt}{\isacharbrackleft}{\kern0pt}{\isadigit{3}}{\isadigit{6}}{\isacharcomma}{\kern0pt}{\isadigit{3}}{\isadigit{1}}{\isacharcomma}{\kern0pt}{\isadigit{1}}{\isadigit{8}}{\isacharcomma}{\kern0pt}{\isadigit{5}}{\isadigit{3}}{\isacharcomma}{\kern0pt}{\isadigit{2}}{\isadigit{0}}{\isacharcomma}{\kern0pt}{\isadigit{2}}{\isadigit{9}}{\isacharcomma}{\kern0pt}{\isadigit{4}}{\isadigit{4}}{\isacharbrackright}{\kern0pt}{\isacharcomma}{\kern0pt}\isanewline
\ \ {\isacharbrackleft}{\kern0pt}{\isadigit{1}}{\isadigit{7}}{\isacharcomma}{\kern0pt}{\isadigit{5}}{\isadigit{4}}{\isacharcomma}{\kern0pt}{\isadigit{3}}{\isadigit{5}}{\isacharcomma}{\kern0pt}{\isadigit{3}}{\isadigit{0}}{\isacharcomma}{\kern0pt}{\isadigit{4}}{\isadigit{5}}{\isacharcomma}{\kern0pt}{\isadigit{5}}{\isadigit{2}}{\isacharcomma}{\kern0pt}{\isadigit{2}}{\isadigit{1}}{\isacharbrackright}{\kern0pt}{\isacharcomma}{\kern0pt}\isanewline
\ \ {\isacharbrackleft}{\kern0pt}{\isadigit{3}}{\isadigit{2}}{\isacharcomma}{\kern0pt}{\isadigit{3}}{\isadigit{7}}{\isacharcomma}{\kern0pt}{\isadigit{4}}{\isadigit{6}}{\isacharcomma}{\kern0pt}{\isadigit{1}}{\isadigit{9}}{\isacharcomma}{\kern0pt}{\isadigit{8}}{\isacharcomma}{\kern0pt}{\isadigit{4}}{\isadigit{3}}{\isacharcomma}{\kern0pt}{\isadigit{2}}{\isadigit{8}}{\isacharbrackright}{\kern0pt}{\isacharcomma}{\kern0pt}\isanewline
\ \ {\isacharbrackleft}{\kern0pt}{\isadigit{5}}{\isadigit{5}}{\isacharcomma}{\kern0pt}{\isadigit{1}}{\isadigit{6}}{\isacharcomma}{\kern0pt}{\isadigit{7}}{\isacharcomma}{\kern0pt}{\isadigit{3}}{\isadigit{4}}{\isacharcomma}{\kern0pt}{\isadigit{2}}{\isadigit{7}}{\isacharcomma}{\kern0pt}{\isadigit{2}}{\isadigit{2}}{\isacharcomma}{\kern0pt}{\isadigit{5}}{\isadigit{1}}{\isacharbrackright}{\kern0pt}{\isacharcomma}{\kern0pt}\isanewline
\ \ {\isacharbrackleft}{\kern0pt}{\isadigit{3}}{\isadigit{8}}{\isacharcomma}{\kern0pt}{\isadigit{3}}{\isadigit{3}}{\isacharcomma}{\kern0pt}{\isadigit{2}}{\isadigit{6}}{\isacharcomma}{\kern0pt}{\isadigit{4}}{\isadigit{7}}{\isacharcomma}{\kern0pt}{\isadigit{6}}{\isacharcomma}{\kern0pt}{\isadigit{9}}{\isacharcomma}{\kern0pt}{\isadigit{4}}{\isadigit{2}}{\isacharbrackright}{\kern0pt}{\isacharcomma}{\kern0pt}\isanewline
\ \ {\isacharbrackleft}{\kern0pt}{\isadigit{3}}{\isacharcomma}{\kern0pt}{\isadigit{5}}{\isadigit{6}}{\isacharcomma}{\kern0pt}{\isadigit{1}}{\isadigit{5}}{\isacharcomma}{\kern0pt}{\isadigit{1}}{\isadigit{2}}{\isacharcomma}{\kern0pt}{\isadigit{2}}{\isadigit{5}}{\isacharcomma}{\kern0pt}{\isadigit{5}}{\isadigit{0}}{\isacharcomma}{\kern0pt}{\isadigit{2}}{\isadigit{3}}{\isacharbrackright}{\kern0pt}{\isacharcomma}{\kern0pt}\isanewline
\ \ {\isacharbrackleft}{\kern0pt}{\isadigit{1}}{\isadigit{4}}{\isacharcomma}{\kern0pt}{\isadigit{3}}{\isadigit{9}}{\isacharcomma}{\kern0pt}{\isadigit{2}}{\isacharcomma}{\kern0pt}{\isadigit{5}}{\isacharcomma}{\kern0pt}{\isadigit{4}}{\isadigit{8}}{\isacharcomma}{\kern0pt}{\isadigit{4}}{\isadigit{1}}{\isacharcomma}{\kern0pt}{\isadigit{1}}{\isadigit{0}}{\isacharbrackright}{\kern0pt}{\isacharcomma}{\kern0pt}\isanewline
\ \ {\isacharbrackleft}{\kern0pt}{\isadigit{1}}{\isacharcomma}{\kern0pt}{\isadigit{4}}{\isacharcomma}{\kern0pt}{\isadigit{1}}{\isadigit{3}}{\isacharcomma}{\kern0pt}{\isadigit{4}}{\isadigit{0}}{\isacharcomma}{\kern0pt}{\isadigit{1}}{\isadigit{1}}{\isacharcomma}{\kern0pt}{\isadigit{2}}{\isadigit{4}}{\isacharcomma}{\kern0pt}{\isadigit{4}}{\isadigit{9}}{\isacharbrackright}{\kern0pt}{\isacharbrackright}{\kern0pt}{\isacharparenright}{\kern0pt}{\isachardoublequoteclose}\isanewline
\isacommand{lemma}\isamarkupfalse%
\ kc{\isacharunderscore}{\kern0pt}{\isadigit{8}}x{\isadigit{7}}{\isacharcolon}{\kern0pt}\ {\isachardoublequoteopen}knights{\isacharunderscore}{\kern0pt}circuit\ b{\isadigit{8}}x{\isadigit{7}}\ kc{\isadigit{8}}x{\isadigit{7}}{\isachardoublequoteclose}\isanewline
%
\isadelimproof
\ \ %
\endisadelimproof
%
\isatagproof
\isacommand{by}\isamarkupfalse%
\ {\isacharparenleft}{\kern0pt}simp\ only{\isacharcolon}{\kern0pt}\ knights{\isacharunderscore}{\kern0pt}circuit{\isacharunderscore}{\kern0pt}exec{\isacharunderscore}{\kern0pt}simp{\isacharparenright}{\kern0pt}\ eval%
\endisatagproof
{\isafoldproof}%
%
\isadelimproof
\isanewline
%
\endisadelimproof
\isanewline
\isacommand{lemma}\isamarkupfalse%
\ kc{\isacharunderscore}{\kern0pt}{\isadigit{8}}x{\isadigit{7}}{\isacharunderscore}{\kern0pt}hd{\isacharcolon}{\kern0pt}\ {\isachardoublequoteopen}hd\ kc{\isadigit{8}}x{\isadigit{7}}\ {\isacharequal}{\kern0pt}\ {\isacharparenleft}{\kern0pt}{\isadigit{1}}{\isacharcomma}{\kern0pt}{\isadigit{1}}{\isacharparenright}{\kern0pt}{\isachardoublequoteclose}%
\isadelimproof
\ %
\endisadelimproof
%
\isatagproof
\isacommand{by}\isamarkupfalse%
\ eval%
\endisatagproof
{\isafoldproof}%
%
\isadelimproof
%
\endisadelimproof
\isanewline
\isanewline
\isacommand{lemma}\isamarkupfalse%
\ kc{\isacharunderscore}{\kern0pt}{\isadigit{8}}x{\isadigit{7}}{\isacharunderscore}{\kern0pt}non{\isacharunderscore}{\kern0pt}nil{\isacharcolon}{\kern0pt}\ {\isachardoublequoteopen}kc{\isadigit{8}}x{\isadigit{7}}\ {\isasymnoteq}\ {\isacharbrackleft}{\kern0pt}{\isacharbrackright}{\kern0pt}{\isachardoublequoteclose}%
\isadelimproof
\ %
\endisadelimproof
%
\isatagproof
\isacommand{by}\isamarkupfalse%
\ eval%
\endisatagproof
{\isafoldproof}%
%
\isadelimproof
%
\endisadelimproof
\isanewline
\isanewline
\isacommand{lemma}\isamarkupfalse%
\ kc{\isacharunderscore}{\kern0pt}{\isadigit{8}}x{\isadigit{7}}{\isacharunderscore}{\kern0pt}si{\isacharcolon}{\kern0pt}\ {\isachardoublequoteopen}step{\isacharunderscore}{\kern0pt}in\ kc{\isadigit{8}}x{\isadigit{7}}\ {\isacharparenleft}{\kern0pt}{\isadigit{2}}{\isacharcomma}{\kern0pt}{\isadigit{6}}{\isacharparenright}{\kern0pt}\ {\isacharparenleft}{\kern0pt}{\isadigit{4}}{\isacharcomma}{\kern0pt}{\isadigit{7}}{\isacharparenright}{\kern0pt}{\isachardoublequoteclose}\ {\isacharparenleft}{\kern0pt}\isakeyword{is}\ {\isachardoublequoteopen}step{\isacharunderscore}{\kern0pt}in\ {\isacharquery}{\kern0pt}ps\ {\isacharunderscore}{\kern0pt}\ {\isacharunderscore}{\kern0pt}{\isachardoublequoteclose}{\isacharparenright}{\kern0pt}\isanewline
%
\isadelimproof
%
\endisadelimproof
%
\isatagproof
\isacommand{proof}\isamarkupfalse%
\ {\isacharminus}{\kern0pt}\isanewline
\ \ \isacommand{have}\isamarkupfalse%
\ {\isachardoublequoteopen}{\isadigit{0}}\ {\isacharless}{\kern0pt}\ {\isacharparenleft}{\kern0pt}{\isadigit{4}}{\isadigit{1}}{\isacharcolon}{\kern0pt}{\isacharcolon}{\kern0pt}nat{\isacharparenright}{\kern0pt}{\isachardoublequoteclose}\ {\isachardoublequoteopen}{\isadigit{4}}{\isadigit{1}}\ {\isacharless}{\kern0pt}\ length\ {\isacharquery}{\kern0pt}ps{\isachardoublequoteclose}\ \isanewline
\ \ \ \ \ \ \ \ {\isachardoublequoteopen}last\ {\isacharparenleft}{\kern0pt}take\ {\isadigit{4}}{\isadigit{1}}\ {\isacharquery}{\kern0pt}ps{\isacharparenright}{\kern0pt}\ {\isacharequal}{\kern0pt}\ {\isacharparenleft}{\kern0pt}{\isadigit{2}}{\isacharcomma}{\kern0pt}{\isadigit{6}}{\isacharparenright}{\kern0pt}{\isachardoublequoteclose}\ {\isachardoublequoteopen}hd\ {\isacharparenleft}{\kern0pt}drop\ {\isadigit{4}}{\isadigit{1}}\ {\isacharquery}{\kern0pt}ps{\isacharparenright}{\kern0pt}\ {\isacharequal}{\kern0pt}\ {\isacharparenleft}{\kern0pt}{\isadigit{4}}{\isacharcomma}{\kern0pt}{\isadigit{7}}{\isacharparenright}{\kern0pt}{\isachardoublequoteclose}\ \isacommand{by}\isamarkupfalse%
\ eval{\isacharplus}{\kern0pt}\isanewline
\ \ \isacommand{then}\isamarkupfalse%
\ \isacommand{show}\isamarkupfalse%
\ {\isacharquery}{\kern0pt}thesis\ \isacommand{unfolding}\isamarkupfalse%
\ step{\isacharunderscore}{\kern0pt}in{\isacharunderscore}{\kern0pt}def\ \isacommand{by}\isamarkupfalse%
\ blast\isanewline
\isacommand{qed}\isamarkupfalse%
%
\endisatagproof
{\isafoldproof}%
%
\isadelimproof
\isanewline
%
\endisadelimproof
\isanewline
\isacommand{abbreviation}\isamarkupfalse%
\ {\isachardoublequoteopen}b{\isadigit{8}}x{\isadigit{8}}\ {\isasymequiv}\ board\ {\isadigit{8}}\ {\isadigit{8}}{\isachardoublequoteclose}%
\begin{isamarkuptext}%
The path given for the \isa{{\isadigit{8}}{\isasymtimes}{\isadigit{8}}}-board that ends in the upper-left is wrong. The Knight cannot 
move from square 27 to square 28.
  \begin{table}[H]
    \begin{tabular}{llllllll}
      24 & 11 & 37 &  9 & 26 & 21 & 39 &  7 \\
      36 & 64 & 24 & 22 & 38 &  8 & \color{red}{27} & 20 \\
      12 & 23 & 10 & 53 & 58 & 49 &  6 & \color{red}{28} \\
      63 & 35 & 61 & 50 & 55 & 52 & 19 & 40 \\
      46 & 13 & 54 & 57 & 48 & 59 & 29 &  5 \\
      34 & 62 & 47 & 60 & 51 & 56 & 41 & 18 \\
      14 & 45 &  2 & 32 & 16 & 43 &  4 & 30 \\
       1 & 33 & 15 & 44 &  3 & 31 & 17 & 42
    \end{tabular}
  \end{table}%
\end{isamarkuptext}\isamarkuptrue%
\isacommand{abbreviation}\isamarkupfalse%
\ {\isachardoublequoteopen}kp{\isadigit{8}}x{\isadigit{8}}ul{\isacharunderscore}{\kern0pt}false\ {\isasymequiv}\ the\ {\isacharparenleft}{\kern0pt}to{\isacharunderscore}{\kern0pt}path\ \isanewline
\ \ {\isacharbrackleft}{\kern0pt}{\isacharbrackleft}{\kern0pt}{\isadigit{2}}{\isadigit{4}}{\isacharcomma}{\kern0pt}{\isadigit{1}}{\isadigit{1}}{\isacharcomma}{\kern0pt}{\isadigit{3}}{\isadigit{7}}{\isacharcomma}{\kern0pt}{\isadigit{9}}{\isacharcomma}{\kern0pt}{\isadigit{2}}{\isadigit{6}}{\isacharcomma}{\kern0pt}{\isadigit{2}}{\isadigit{1}}{\isacharcomma}{\kern0pt}{\isadigit{3}}{\isadigit{9}}{\isacharcomma}{\kern0pt}{\isadigit{7}}{\isacharbrackright}{\kern0pt}{\isacharcomma}{\kern0pt}\isanewline
\ \ {\isacharbrackleft}{\kern0pt}{\isadigit{3}}{\isadigit{6}}{\isacharcomma}{\kern0pt}{\isadigit{6}}{\isadigit{4}}{\isacharcomma}{\kern0pt}{\isadigit{2}}{\isadigit{5}}{\isacharcomma}{\kern0pt}{\isadigit{2}}{\isadigit{2}}{\isacharcomma}{\kern0pt}{\isadigit{3}}{\isadigit{8}}{\isacharcomma}{\kern0pt}{\isadigit{8}}{\isacharcomma}{\kern0pt}{\isadigit{2}}{\isadigit{7}}{\isacharcomma}{\kern0pt}{\isadigit{2}}{\isadigit{0}}{\isacharbrackright}{\kern0pt}{\isacharcomma}{\kern0pt}\isanewline
\ \ {\isacharbrackleft}{\kern0pt}{\isadigit{1}}{\isadigit{2}}{\isacharcomma}{\kern0pt}{\isadigit{2}}{\isadigit{3}}{\isacharcomma}{\kern0pt}{\isadigit{1}}{\isadigit{0}}{\isacharcomma}{\kern0pt}{\isadigit{5}}{\isadigit{3}}{\isacharcomma}{\kern0pt}{\isadigit{5}}{\isadigit{8}}{\isacharcomma}{\kern0pt}{\isadigit{4}}{\isadigit{9}}{\isacharcomma}{\kern0pt}{\isadigit{6}}{\isacharcomma}{\kern0pt}{\isadigit{2}}{\isadigit{8}}{\isacharbrackright}{\kern0pt}{\isacharcomma}{\kern0pt}\isanewline
\ \ {\isacharbrackleft}{\kern0pt}{\isadigit{6}}{\isadigit{3}}{\isacharcomma}{\kern0pt}{\isadigit{3}}{\isadigit{5}}{\isacharcomma}{\kern0pt}{\isadigit{6}}{\isadigit{1}}{\isacharcomma}{\kern0pt}{\isadigit{5}}{\isadigit{0}}{\isacharcomma}{\kern0pt}{\isadigit{5}}{\isadigit{5}}{\isacharcomma}{\kern0pt}{\isadigit{5}}{\isadigit{2}}{\isacharcomma}{\kern0pt}{\isadigit{1}}{\isadigit{9}}{\isacharcomma}{\kern0pt}{\isadigit{4}}{\isadigit{0}}{\isacharbrackright}{\kern0pt}{\isacharcomma}{\kern0pt}\isanewline
\ \ {\isacharbrackleft}{\kern0pt}{\isadigit{4}}{\isadigit{6}}{\isacharcomma}{\kern0pt}{\isadigit{1}}{\isadigit{3}}{\isacharcomma}{\kern0pt}{\isadigit{5}}{\isadigit{4}}{\isacharcomma}{\kern0pt}{\isadigit{5}}{\isadigit{7}}{\isacharcomma}{\kern0pt}{\isadigit{4}}{\isadigit{8}}{\isacharcomma}{\kern0pt}{\isadigit{5}}{\isadigit{9}}{\isacharcomma}{\kern0pt}{\isadigit{2}}{\isadigit{9}}{\isacharcomma}{\kern0pt}{\isadigit{5}}{\isacharbrackright}{\kern0pt}{\isacharcomma}{\kern0pt}\isanewline
\ \ {\isacharbrackleft}{\kern0pt}{\isadigit{3}}{\isadigit{4}}{\isacharcomma}{\kern0pt}{\isadigit{6}}{\isadigit{2}}{\isacharcomma}{\kern0pt}{\isadigit{4}}{\isadigit{7}}{\isacharcomma}{\kern0pt}{\isadigit{6}}{\isadigit{0}}{\isacharcomma}{\kern0pt}{\isadigit{5}}{\isadigit{1}}{\isacharcomma}{\kern0pt}{\isadigit{5}}{\isadigit{6}}{\isacharcomma}{\kern0pt}{\isadigit{4}}{\isadigit{1}}{\isacharcomma}{\kern0pt}{\isadigit{1}}{\isadigit{8}}{\isacharbrackright}{\kern0pt}{\isacharcomma}{\kern0pt}\isanewline
\ \ {\isacharbrackleft}{\kern0pt}{\isadigit{1}}{\isadigit{4}}{\isacharcomma}{\kern0pt}{\isadigit{4}}{\isadigit{5}}{\isacharcomma}{\kern0pt}{\isadigit{2}}{\isacharcomma}{\kern0pt}{\isadigit{3}}{\isadigit{2}}{\isacharcomma}{\kern0pt}{\isadigit{1}}{\isadigit{6}}{\isacharcomma}{\kern0pt}{\isadigit{4}}{\isadigit{3}}{\isacharcomma}{\kern0pt}{\isadigit{4}}{\isacharcomma}{\kern0pt}{\isadigit{3}}{\isadigit{0}}{\isacharbrackright}{\kern0pt}{\isacharcomma}{\kern0pt}\isanewline
\ \ {\isacharbrackleft}{\kern0pt}{\isadigit{1}}{\isacharcomma}{\kern0pt}{\isadigit{3}}{\isadigit{3}}{\isacharcomma}{\kern0pt}{\isadigit{1}}{\isadigit{5}}{\isacharcomma}{\kern0pt}{\isadigit{4}}{\isadigit{4}}{\isacharcomma}{\kern0pt}{\isadigit{3}}{\isacharcomma}{\kern0pt}{\isadigit{3}}{\isadigit{1}}{\isacharcomma}{\kern0pt}{\isadigit{1}}{\isadigit{7}}{\isacharcomma}{\kern0pt}{\isadigit{4}}{\isadigit{2}}{\isacharbrackright}{\kern0pt}{\isacharbrackright}{\kern0pt}{\isacharparenright}{\kern0pt}{\isachardoublequoteclose}\isanewline
\isanewline
\isacommand{lemma}\isamarkupfalse%
\ {\isachardoublequoteopen}{\isasymnot}knights{\isacharunderscore}{\kern0pt}path\ b{\isadigit{8}}x{\isadigit{8}}\ kp{\isadigit{8}}x{\isadigit{8}}ul{\isacharunderscore}{\kern0pt}false{\isachardoublequoteclose}\isanewline
%
\isadelimproof
\ \ %
\endisadelimproof
%
\isatagproof
\isacommand{by}\isamarkupfalse%
\ {\isacharparenleft}{\kern0pt}simp\ only{\isacharcolon}{\kern0pt}\ knights{\isacharunderscore}{\kern0pt}path{\isacharunderscore}{\kern0pt}exec{\isacharunderscore}{\kern0pt}simp{\isacharparenright}{\kern0pt}\ eval%
\endisatagproof
{\isafoldproof}%
%
\isadelimproof
%
\endisadelimproof
%
\begin{isamarkuptext}%
I have computed a correct Knight's path for the \isa{{\isadigit{8}}{\isasymtimes}{\isadigit{8}}}-board that ends in the upper-left.
  \begin{table}[H]
    \begin{tabular}{llllllll}
      38 & 41 & 36 & 27 & 32 & 43 & 20 & 25 \\
      35 & 64 & 39 & 42 & 21 & 26 & 29 & 44 \\
      40 & 37 &  6 & 33 & 28 & 31 & 24 & 19 \\
       5 & 34 & 63 & 14 &  7 & 22 & 45 & 30 \\
      62 & 13 &  4 &  9 & 58 & 49 & 18 & 23 \\
       3 & 10 & 61 & 52 & 15 &  8 & 57 & 46 \\
      12 & 53 &  2 & 59 & 48 & 55 & 50 & 17 \\
       1 & 60 & 11 & 54 & 51 & 16 & 47 & 56
    \end{tabular}
  \end{table}%
\end{isamarkuptext}\isamarkuptrue%
\isacommand{abbreviation}\isamarkupfalse%
\ {\isachardoublequoteopen}kp{\isadigit{8}}x{\isadigit{8}}ul\ {\isasymequiv}\ the\ {\isacharparenleft}{\kern0pt}to{\isacharunderscore}{\kern0pt}path\ \isanewline
\ \ {\isacharbrackleft}{\kern0pt}{\isacharbrackleft}{\kern0pt}{\isadigit{3}}{\isadigit{8}}{\isacharcomma}{\kern0pt}{\isadigit{4}}{\isadigit{1}}{\isacharcomma}{\kern0pt}{\isadigit{3}}{\isadigit{6}}{\isacharcomma}{\kern0pt}{\isadigit{2}}{\isadigit{7}}{\isacharcomma}{\kern0pt}{\isadigit{3}}{\isadigit{2}}{\isacharcomma}{\kern0pt}{\isadigit{4}}{\isadigit{3}}{\isacharcomma}{\kern0pt}{\isadigit{2}}{\isadigit{0}}{\isacharcomma}{\kern0pt}{\isadigit{2}}{\isadigit{5}}{\isacharbrackright}{\kern0pt}{\isacharcomma}{\kern0pt}\isanewline
\ \ {\isacharbrackleft}{\kern0pt}{\isadigit{3}}{\isadigit{5}}{\isacharcomma}{\kern0pt}{\isadigit{6}}{\isadigit{4}}{\isacharcomma}{\kern0pt}{\isadigit{3}}{\isadigit{9}}{\isacharcomma}{\kern0pt}{\isadigit{4}}{\isadigit{2}}{\isacharcomma}{\kern0pt}{\isadigit{2}}{\isadigit{1}}{\isacharcomma}{\kern0pt}{\isadigit{2}}{\isadigit{6}}{\isacharcomma}{\kern0pt}{\isadigit{2}}{\isadigit{9}}{\isacharcomma}{\kern0pt}{\isadigit{4}}{\isadigit{4}}{\isacharbrackright}{\kern0pt}{\isacharcomma}{\kern0pt}\isanewline
\ \ {\isacharbrackleft}{\kern0pt}{\isadigit{4}}{\isadigit{0}}{\isacharcomma}{\kern0pt}{\isadigit{3}}{\isadigit{7}}{\isacharcomma}{\kern0pt}{\isadigit{6}}{\isacharcomma}{\kern0pt}{\isadigit{3}}{\isadigit{3}}{\isacharcomma}{\kern0pt}{\isadigit{2}}{\isadigit{8}}{\isacharcomma}{\kern0pt}{\isadigit{3}}{\isadigit{1}}{\isacharcomma}{\kern0pt}{\isadigit{2}}{\isadigit{4}}{\isacharcomma}{\kern0pt}{\isadigit{1}}{\isadigit{9}}{\isacharbrackright}{\kern0pt}{\isacharcomma}{\kern0pt}\isanewline
\ \ {\isacharbrackleft}{\kern0pt}{\isadigit{5}}{\isacharcomma}{\kern0pt}{\isadigit{3}}{\isadigit{4}}{\isacharcomma}{\kern0pt}{\isadigit{6}}{\isadigit{3}}{\isacharcomma}{\kern0pt}{\isadigit{1}}{\isadigit{4}}{\isacharcomma}{\kern0pt}{\isadigit{7}}{\isacharcomma}{\kern0pt}{\isadigit{2}}{\isadigit{2}}{\isacharcomma}{\kern0pt}{\isadigit{4}}{\isadigit{5}}{\isacharcomma}{\kern0pt}{\isadigit{3}}{\isadigit{0}}{\isacharbrackright}{\kern0pt}{\isacharcomma}{\kern0pt}\isanewline
\ \ {\isacharbrackleft}{\kern0pt}{\isadigit{6}}{\isadigit{2}}{\isacharcomma}{\kern0pt}{\isadigit{1}}{\isadigit{3}}{\isacharcomma}{\kern0pt}{\isadigit{4}}{\isacharcomma}{\kern0pt}{\isadigit{9}}{\isacharcomma}{\kern0pt}{\isadigit{5}}{\isadigit{8}}{\isacharcomma}{\kern0pt}{\isadigit{4}}{\isadigit{9}}{\isacharcomma}{\kern0pt}{\isadigit{1}}{\isadigit{8}}{\isacharcomma}{\kern0pt}{\isadigit{2}}{\isadigit{3}}{\isacharbrackright}{\kern0pt}{\isacharcomma}{\kern0pt}\isanewline
\ \ {\isacharbrackleft}{\kern0pt}{\isadigit{3}}{\isacharcomma}{\kern0pt}{\isadigit{1}}{\isadigit{0}}{\isacharcomma}{\kern0pt}{\isadigit{6}}{\isadigit{1}}{\isacharcomma}{\kern0pt}{\isadigit{5}}{\isadigit{2}}{\isacharcomma}{\kern0pt}{\isadigit{1}}{\isadigit{5}}{\isacharcomma}{\kern0pt}{\isadigit{8}}{\isacharcomma}{\kern0pt}{\isadigit{5}}{\isadigit{7}}{\isacharcomma}{\kern0pt}{\isadigit{4}}{\isadigit{6}}{\isacharbrackright}{\kern0pt}{\isacharcomma}{\kern0pt}\isanewline
\ \ {\isacharbrackleft}{\kern0pt}{\isadigit{1}}{\isadigit{2}}{\isacharcomma}{\kern0pt}{\isadigit{5}}{\isadigit{3}}{\isacharcomma}{\kern0pt}{\isadigit{2}}{\isacharcomma}{\kern0pt}{\isadigit{5}}{\isadigit{9}}{\isacharcomma}{\kern0pt}{\isadigit{4}}{\isadigit{8}}{\isacharcomma}{\kern0pt}{\isadigit{5}}{\isadigit{5}}{\isacharcomma}{\kern0pt}{\isadigit{5}}{\isadigit{0}}{\isacharcomma}{\kern0pt}{\isadigit{1}}{\isadigit{7}}{\isacharbrackright}{\kern0pt}{\isacharcomma}{\kern0pt}\isanewline
\ \ {\isacharbrackleft}{\kern0pt}{\isadigit{1}}{\isacharcomma}{\kern0pt}{\isadigit{6}}{\isadigit{0}}{\isacharcomma}{\kern0pt}{\isadigit{1}}{\isadigit{1}}{\isacharcomma}{\kern0pt}{\isadigit{5}}{\isadigit{4}}{\isacharcomma}{\kern0pt}{\isadigit{5}}{\isadigit{1}}{\isacharcomma}{\kern0pt}{\isadigit{1}}{\isadigit{6}}{\isacharcomma}{\kern0pt}{\isadigit{4}}{\isadigit{7}}{\isacharcomma}{\kern0pt}{\isadigit{5}}{\isadigit{6}}{\isacharbrackright}{\kern0pt}{\isacharbrackright}{\kern0pt}{\isacharparenright}{\kern0pt}{\isachardoublequoteclose}\ \isanewline
\isanewline
\isacommand{lemma}\isamarkupfalse%
\ kp{\isacharunderscore}{\kern0pt}{\isadigit{8}}x{\isadigit{8}}{\isacharunderscore}{\kern0pt}ul{\isacharcolon}{\kern0pt}\ {\isachardoublequoteopen}knights{\isacharunderscore}{\kern0pt}path\ b{\isadigit{8}}x{\isadigit{8}}\ kp{\isadigit{8}}x{\isadigit{8}}ul{\isachardoublequoteclose}\isanewline
%
\isadelimproof
\ \ %
\endisadelimproof
%
\isatagproof
\isacommand{by}\isamarkupfalse%
\ {\isacharparenleft}{\kern0pt}simp\ only{\isacharcolon}{\kern0pt}\ knights{\isacharunderscore}{\kern0pt}path{\isacharunderscore}{\kern0pt}exec{\isacharunderscore}{\kern0pt}simp{\isacharparenright}{\kern0pt}\ eval%
\endisatagproof
{\isafoldproof}%
%
\isadelimproof
\isanewline
%
\endisadelimproof
\isanewline
\isacommand{lemma}\isamarkupfalse%
\ kp{\isacharunderscore}{\kern0pt}{\isadigit{8}}x{\isadigit{8}}{\isacharunderscore}{\kern0pt}ul{\isacharunderscore}{\kern0pt}hd{\isacharcolon}{\kern0pt}\ {\isachardoublequoteopen}hd\ kp{\isadigit{8}}x{\isadigit{8}}ul\ {\isacharequal}{\kern0pt}\ {\isacharparenleft}{\kern0pt}{\isadigit{1}}{\isacharcomma}{\kern0pt}{\isadigit{1}}{\isacharparenright}{\kern0pt}{\isachardoublequoteclose}%
\isadelimproof
\ %
\endisadelimproof
%
\isatagproof
\isacommand{by}\isamarkupfalse%
\ eval%
\endisatagproof
{\isafoldproof}%
%
\isadelimproof
%
\endisadelimproof
\isanewline
\isanewline
\isacommand{lemma}\isamarkupfalse%
\ kp{\isacharunderscore}{\kern0pt}{\isadigit{8}}x{\isadigit{8}}{\isacharunderscore}{\kern0pt}ul{\isacharunderscore}{\kern0pt}last{\isacharcolon}{\kern0pt}\ {\isachardoublequoteopen}last\ kp{\isadigit{8}}x{\isadigit{8}}ul\ {\isacharequal}{\kern0pt}\ {\isacharparenleft}{\kern0pt}{\isadigit{7}}{\isacharcomma}{\kern0pt}{\isadigit{2}}{\isacharparenright}{\kern0pt}{\isachardoublequoteclose}%
\isadelimproof
\ %
\endisadelimproof
%
\isatagproof
\isacommand{by}\isamarkupfalse%
\ eval%
\endisatagproof
{\isafoldproof}%
%
\isadelimproof
%
\endisadelimproof
\isanewline
\isanewline
\isacommand{lemma}\isamarkupfalse%
\ kp{\isacharunderscore}{\kern0pt}{\isadigit{8}}x{\isadigit{8}}{\isacharunderscore}{\kern0pt}ul{\isacharunderscore}{\kern0pt}non{\isacharunderscore}{\kern0pt}nil{\isacharcolon}{\kern0pt}\ {\isachardoublequoteopen}kp{\isadigit{8}}x{\isadigit{8}}ul\ {\isasymnoteq}\ {\isacharbrackleft}{\kern0pt}{\isacharbrackright}{\kern0pt}{\isachardoublequoteclose}%
\isadelimproof
\ %
\endisadelimproof
%
\isatagproof
\isacommand{by}\isamarkupfalse%
\ eval%
\endisatagproof
{\isafoldproof}%
%
\isadelimproof
%
\endisadelimproof
%
\begin{isamarkuptext}%
A Knight's circuit for the \isa{{\isacharparenleft}{\kern0pt}{\isadigit{8}}{\isasymtimes}{\isadigit{8}}{\isacharparenright}{\kern0pt}}-board.
  \begin{table}[H]
    \begin{tabular}{llllllll}
      48 & 13 & 30 &  9 & 56 & 45 & 28 &  7 \\
      31 & 10 & 47 & 50 & 29 &  8 & 57 & 44 \\
      14 & 49 & 12 & 55 & 46 & 59 &  6 & 27 \\
      11 & 32 & 37 & 60 & 51 & 54 & 43 & 58 \\
      36 & 15 & 52 & 63 & 38 & 61 & 26 &  5 \\
      33 & 64 & 35 & 18 & 53 & 40 & 23 & 42 \\
      16 & 19 &  2 & 39 & 62 & 21 &  4 & 25 \\
       1 & 34 & 17 & 20 &  3 & 24 & 41 & 22
    \end{tabular}
  \end{table}%
\end{isamarkuptext}\isamarkuptrue%
\isacommand{abbreviation}\isamarkupfalse%
\ {\isachardoublequoteopen}kc{\isadigit{8}}x{\isadigit{8}}\ {\isasymequiv}\ the\ {\isacharparenleft}{\kern0pt}to{\isacharunderscore}{\kern0pt}path\ \isanewline
\ \ {\isacharbrackleft}{\kern0pt}{\isacharbrackleft}{\kern0pt}{\isadigit{4}}{\isadigit{8}}{\isacharcomma}{\kern0pt}{\isadigit{1}}{\isadigit{3}}{\isacharcomma}{\kern0pt}{\isadigit{3}}{\isadigit{0}}{\isacharcomma}{\kern0pt}{\isadigit{9}}{\isacharcomma}{\kern0pt}{\isadigit{5}}{\isadigit{6}}{\isacharcomma}{\kern0pt}{\isadigit{4}}{\isadigit{5}}{\isacharcomma}{\kern0pt}{\isadigit{2}}{\isadigit{8}}{\isacharcomma}{\kern0pt}{\isadigit{7}}{\isacharbrackright}{\kern0pt}{\isacharcomma}{\kern0pt}\isanewline
\ \ {\isacharbrackleft}{\kern0pt}{\isadigit{3}}{\isadigit{1}}{\isacharcomma}{\kern0pt}{\isadigit{1}}{\isadigit{0}}{\isacharcomma}{\kern0pt}{\isadigit{4}}{\isadigit{7}}{\isacharcomma}{\kern0pt}{\isadigit{5}}{\isadigit{0}}{\isacharcomma}{\kern0pt}{\isadigit{2}}{\isadigit{9}}{\isacharcomma}{\kern0pt}{\isadigit{8}}{\isacharcomma}{\kern0pt}{\isadigit{5}}{\isadigit{7}}{\isacharcomma}{\kern0pt}{\isadigit{4}}{\isadigit{4}}{\isacharbrackright}{\kern0pt}{\isacharcomma}{\kern0pt}\isanewline
\ \ {\isacharbrackleft}{\kern0pt}{\isadigit{1}}{\isadigit{4}}{\isacharcomma}{\kern0pt}{\isadigit{4}}{\isadigit{9}}{\isacharcomma}{\kern0pt}{\isadigit{1}}{\isadigit{2}}{\isacharcomma}{\kern0pt}{\isadigit{5}}{\isadigit{5}}{\isacharcomma}{\kern0pt}{\isadigit{4}}{\isadigit{6}}{\isacharcomma}{\kern0pt}{\isadigit{5}}{\isadigit{9}}{\isacharcomma}{\kern0pt}{\isadigit{6}}{\isacharcomma}{\kern0pt}{\isadigit{2}}{\isadigit{7}}{\isacharbrackright}{\kern0pt}{\isacharcomma}{\kern0pt}\isanewline
\ \ {\isacharbrackleft}{\kern0pt}{\isadigit{1}}{\isadigit{1}}{\isacharcomma}{\kern0pt}{\isadigit{3}}{\isadigit{2}}{\isacharcomma}{\kern0pt}{\isadigit{3}}{\isadigit{7}}{\isacharcomma}{\kern0pt}{\isadigit{6}}{\isadigit{0}}{\isacharcomma}{\kern0pt}{\isadigit{5}}{\isadigit{1}}{\isacharcomma}{\kern0pt}{\isadigit{5}}{\isadigit{4}}{\isacharcomma}{\kern0pt}{\isadigit{4}}{\isadigit{3}}{\isacharcomma}{\kern0pt}{\isadigit{5}}{\isadigit{8}}{\isacharbrackright}{\kern0pt}{\isacharcomma}{\kern0pt}\isanewline
\ \ {\isacharbrackleft}{\kern0pt}{\isadigit{3}}{\isadigit{6}}{\isacharcomma}{\kern0pt}{\isadigit{1}}{\isadigit{5}}{\isacharcomma}{\kern0pt}{\isadigit{5}}{\isadigit{2}}{\isacharcomma}{\kern0pt}{\isadigit{6}}{\isadigit{3}}{\isacharcomma}{\kern0pt}{\isadigit{3}}{\isadigit{8}}{\isacharcomma}{\kern0pt}{\isadigit{6}}{\isadigit{1}}{\isacharcomma}{\kern0pt}{\isadigit{2}}{\isadigit{6}}{\isacharcomma}{\kern0pt}{\isadigit{5}}{\isacharbrackright}{\kern0pt}{\isacharcomma}{\kern0pt}\isanewline
\ \ {\isacharbrackleft}{\kern0pt}{\isadigit{3}}{\isadigit{3}}{\isacharcomma}{\kern0pt}{\isadigit{6}}{\isadigit{4}}{\isacharcomma}{\kern0pt}{\isadigit{3}}{\isadigit{5}}{\isacharcomma}{\kern0pt}{\isadigit{1}}{\isadigit{8}}{\isacharcomma}{\kern0pt}{\isadigit{5}}{\isadigit{3}}{\isacharcomma}{\kern0pt}{\isadigit{4}}{\isadigit{0}}{\isacharcomma}{\kern0pt}{\isadigit{2}}{\isadigit{3}}{\isacharcomma}{\kern0pt}{\isadigit{4}}{\isadigit{2}}{\isacharbrackright}{\kern0pt}{\isacharcomma}{\kern0pt}\isanewline
\ \ {\isacharbrackleft}{\kern0pt}{\isadigit{1}}{\isadigit{6}}{\isacharcomma}{\kern0pt}{\isadigit{1}}{\isadigit{9}}{\isacharcomma}{\kern0pt}{\isadigit{2}}{\isacharcomma}{\kern0pt}{\isadigit{3}}{\isadigit{9}}{\isacharcomma}{\kern0pt}{\isadigit{6}}{\isadigit{2}}{\isacharcomma}{\kern0pt}{\isadigit{2}}{\isadigit{1}}{\isacharcomma}{\kern0pt}{\isadigit{4}}{\isacharcomma}{\kern0pt}{\isadigit{2}}{\isadigit{5}}{\isacharbrackright}{\kern0pt}{\isacharcomma}{\kern0pt}\isanewline
\ \ {\isacharbrackleft}{\kern0pt}{\isadigit{1}}{\isacharcomma}{\kern0pt}{\isadigit{3}}{\isadigit{4}}{\isacharcomma}{\kern0pt}{\isadigit{1}}{\isadigit{7}}{\isacharcomma}{\kern0pt}{\isadigit{2}}{\isadigit{0}}{\isacharcomma}{\kern0pt}{\isadigit{3}}{\isacharcomma}{\kern0pt}{\isadigit{2}}{\isadigit{4}}{\isacharcomma}{\kern0pt}{\isadigit{4}}{\isadigit{1}}{\isacharcomma}{\kern0pt}{\isadigit{2}}{\isadigit{2}}{\isacharbrackright}{\kern0pt}{\isacharbrackright}{\kern0pt}{\isacharparenright}{\kern0pt}{\isachardoublequoteclose}\isanewline
\isacommand{lemma}\isamarkupfalse%
\ kc{\isacharunderscore}{\kern0pt}{\isadigit{8}}x{\isadigit{8}}{\isacharcolon}{\kern0pt}\ {\isachardoublequoteopen}knights{\isacharunderscore}{\kern0pt}circuit\ b{\isadigit{8}}x{\isadigit{8}}\ kc{\isadigit{8}}x{\isadigit{8}}{\isachardoublequoteclose}\isanewline
%
\isadelimproof
\ \ %
\endisadelimproof
%
\isatagproof
\isacommand{by}\isamarkupfalse%
\ {\isacharparenleft}{\kern0pt}simp\ only{\isacharcolon}{\kern0pt}\ knights{\isacharunderscore}{\kern0pt}circuit{\isacharunderscore}{\kern0pt}exec{\isacharunderscore}{\kern0pt}simp{\isacharparenright}{\kern0pt}\ eval%
\endisatagproof
{\isafoldproof}%
%
\isadelimproof
\isanewline
%
\endisadelimproof
\isanewline
\isacommand{lemma}\isamarkupfalse%
\ kc{\isacharunderscore}{\kern0pt}{\isadigit{8}}x{\isadigit{8}}{\isacharunderscore}{\kern0pt}hd{\isacharcolon}{\kern0pt}\ {\isachardoublequoteopen}hd\ kc{\isadigit{8}}x{\isadigit{8}}\ {\isacharequal}{\kern0pt}\ {\isacharparenleft}{\kern0pt}{\isadigit{1}}{\isacharcomma}{\kern0pt}{\isadigit{1}}{\isacharparenright}{\kern0pt}{\isachardoublequoteclose}%
\isadelimproof
\ %
\endisadelimproof
%
\isatagproof
\isacommand{by}\isamarkupfalse%
\ eval%
\endisatagproof
{\isafoldproof}%
%
\isadelimproof
%
\endisadelimproof
\isanewline
\isanewline
\isacommand{lemma}\isamarkupfalse%
\ kc{\isacharunderscore}{\kern0pt}{\isadigit{8}}x{\isadigit{8}}{\isacharunderscore}{\kern0pt}non{\isacharunderscore}{\kern0pt}nil{\isacharcolon}{\kern0pt}\ {\isachardoublequoteopen}kc{\isadigit{8}}x{\isadigit{8}}\ {\isasymnoteq}\ {\isacharbrackleft}{\kern0pt}{\isacharbrackright}{\kern0pt}{\isachardoublequoteclose}%
\isadelimproof
\ %
\endisadelimproof
%
\isatagproof
\isacommand{by}\isamarkupfalse%
\ eval%
\endisatagproof
{\isafoldproof}%
%
\isadelimproof
%
\endisadelimproof
\isanewline
\isanewline
\isacommand{lemma}\isamarkupfalse%
\ kc{\isacharunderscore}{\kern0pt}{\isadigit{8}}x{\isadigit{8}}{\isacharunderscore}{\kern0pt}si{\isacharcolon}{\kern0pt}\ {\isachardoublequoteopen}step{\isacharunderscore}{\kern0pt}in\ kc{\isadigit{8}}x{\isadigit{8}}\ {\isacharparenleft}{\kern0pt}{\isadigit{2}}{\isacharcomma}{\kern0pt}{\isadigit{7}}{\isacharparenright}{\kern0pt}\ {\isacharparenleft}{\kern0pt}{\isadigit{4}}{\isacharcomma}{\kern0pt}{\isadigit{8}}{\isacharparenright}{\kern0pt}{\isachardoublequoteclose}\ {\isacharparenleft}{\kern0pt}\isakeyword{is}\ {\isachardoublequoteopen}step{\isacharunderscore}{\kern0pt}in\ {\isacharquery}{\kern0pt}ps\ {\isacharunderscore}{\kern0pt}\ {\isacharunderscore}{\kern0pt}{\isachardoublequoteclose}{\isacharparenright}{\kern0pt}\isanewline
%
\isadelimproof
%
\endisadelimproof
%
\isatagproof
\isacommand{proof}\isamarkupfalse%
\ {\isacharminus}{\kern0pt}\isanewline
\ \ \isacommand{have}\isamarkupfalse%
\ {\isachardoublequoteopen}{\isadigit{0}}\ {\isacharless}{\kern0pt}\ {\isacharparenleft}{\kern0pt}{\isadigit{4}}{\isacharcolon}{\kern0pt}{\isacharcolon}{\kern0pt}nat{\isacharparenright}{\kern0pt}{\isachardoublequoteclose}\ {\isachardoublequoteopen}{\isadigit{4}}\ {\isacharless}{\kern0pt}\ length\ {\isacharquery}{\kern0pt}ps{\isachardoublequoteclose}\ \isanewline
\ \ \ \ \ \ \ \ {\isachardoublequoteopen}last\ {\isacharparenleft}{\kern0pt}take\ {\isadigit{4}}\ {\isacharquery}{\kern0pt}ps{\isacharparenright}{\kern0pt}\ {\isacharequal}{\kern0pt}\ {\isacharparenleft}{\kern0pt}{\isadigit{2}}{\isacharcomma}{\kern0pt}{\isadigit{7}}{\isacharparenright}{\kern0pt}{\isachardoublequoteclose}\ {\isachardoublequoteopen}hd\ {\isacharparenleft}{\kern0pt}drop\ {\isadigit{4}}\ {\isacharquery}{\kern0pt}ps{\isacharparenright}{\kern0pt}\ {\isacharequal}{\kern0pt}\ {\isacharparenleft}{\kern0pt}{\isadigit{4}}{\isacharcomma}{\kern0pt}{\isadigit{8}}{\isacharparenright}{\kern0pt}{\isachardoublequoteclose}\ \isacommand{by}\isamarkupfalse%
\ eval{\isacharplus}{\kern0pt}\isanewline
\ \ \isacommand{then}\isamarkupfalse%
\ \isacommand{show}\isamarkupfalse%
\ {\isacharquery}{\kern0pt}thesis\ \isacommand{unfolding}\isamarkupfalse%
\ step{\isacharunderscore}{\kern0pt}in{\isacharunderscore}{\kern0pt}def\ \isacommand{by}\isamarkupfalse%
\ blast\isanewline
\isacommand{qed}\isamarkupfalse%
%
\endisatagproof
{\isafoldproof}%
%
\isadelimproof
\isanewline
%
\endisadelimproof
\isanewline
\isacommand{abbreviation}\isamarkupfalse%
\ {\isachardoublequoteopen}b{\isadigit{8}}x{\isadigit{9}}\ {\isasymequiv}\ board\ {\isadigit{8}}\ {\isadigit{9}}{\isachardoublequoteclose}%
\begin{isamarkuptext}%
A Knight's path for the \isa{{\isacharparenleft}{\kern0pt}{\isadigit{8}}{\isasymtimes}{\isadigit{9}}{\isacharparenright}{\kern0pt}}-board that starts in the lower-left and ends in the 
upper-left.
  \begin{table}[H]
    \begin{tabular}{lllllllll}
      32 & 47 &  6 & 71 & 30 & 45 &  8 & 43 & 26 \\
       5 & 72 & 31 & 46 &  7 & 70 & 27 & 22 &  9 \\
      48 & 33 &  4 & 29 & 64 & 23 & 44 & 25 & 42 \\
       3 & 60 & 35 & 62 & 69 & 28 & 41 & 10 & 21 \\
      34 & 49 & 68 & 65 & 36 & 63 & 24 & 55 & 40 \\
      59 &  2 & 61 & 16 & 67 & 56 & 37 & 20 & 11 \\
      50 & 15 & 66 & 57 & 52 & 13 & 18 & 39 & 54 \\
       1 & 58 & 51 & 14 & 17 & 38 & 53 & 12 & 19
    \end{tabular}
  \end{table}%
\end{isamarkuptext}\isamarkuptrue%
\isacommand{abbreviation}\isamarkupfalse%
\ {\isachardoublequoteopen}kp{\isadigit{8}}x{\isadigit{9}}ul\ {\isasymequiv}\ the\ {\isacharparenleft}{\kern0pt}to{\isacharunderscore}{\kern0pt}path\ \isanewline
\ \ {\isacharbrackleft}{\kern0pt}{\isacharbrackleft}{\kern0pt}{\isadigit{3}}{\isadigit{2}}{\isacharcomma}{\kern0pt}{\isadigit{4}}{\isadigit{7}}{\isacharcomma}{\kern0pt}{\isadigit{6}}{\isacharcomma}{\kern0pt}{\isadigit{7}}{\isadigit{1}}{\isacharcomma}{\kern0pt}{\isadigit{3}}{\isadigit{0}}{\isacharcomma}{\kern0pt}{\isadigit{4}}{\isadigit{5}}{\isacharcomma}{\kern0pt}{\isadigit{8}}{\isacharcomma}{\kern0pt}{\isadigit{4}}{\isadigit{3}}{\isacharcomma}{\kern0pt}{\isadigit{2}}{\isadigit{6}}{\isacharbrackright}{\kern0pt}{\isacharcomma}{\kern0pt}\isanewline
\ \ {\isacharbrackleft}{\kern0pt}{\isadigit{5}}{\isacharcomma}{\kern0pt}{\isadigit{7}}{\isadigit{2}}{\isacharcomma}{\kern0pt}{\isadigit{3}}{\isadigit{1}}{\isacharcomma}{\kern0pt}{\isadigit{4}}{\isadigit{6}}{\isacharcomma}{\kern0pt}{\isadigit{7}}{\isacharcomma}{\kern0pt}{\isadigit{7}}{\isadigit{0}}{\isacharcomma}{\kern0pt}{\isadigit{2}}{\isadigit{7}}{\isacharcomma}{\kern0pt}{\isadigit{2}}{\isadigit{2}}{\isacharcomma}{\kern0pt}{\isadigit{9}}{\isacharbrackright}{\kern0pt}{\isacharcomma}{\kern0pt}\isanewline
\ \ {\isacharbrackleft}{\kern0pt}{\isadigit{4}}{\isadigit{8}}{\isacharcomma}{\kern0pt}{\isadigit{3}}{\isadigit{3}}{\isacharcomma}{\kern0pt}{\isadigit{4}}{\isacharcomma}{\kern0pt}{\isadigit{2}}{\isadigit{9}}{\isacharcomma}{\kern0pt}{\isadigit{6}}{\isadigit{4}}{\isacharcomma}{\kern0pt}{\isadigit{2}}{\isadigit{3}}{\isacharcomma}{\kern0pt}{\isadigit{4}}{\isadigit{4}}{\isacharcomma}{\kern0pt}{\isadigit{2}}{\isadigit{5}}{\isacharcomma}{\kern0pt}{\isadigit{4}}{\isadigit{2}}{\isacharbrackright}{\kern0pt}{\isacharcomma}{\kern0pt}\isanewline
\ \ {\isacharbrackleft}{\kern0pt}{\isadigit{3}}{\isacharcomma}{\kern0pt}{\isadigit{6}}{\isadigit{0}}{\isacharcomma}{\kern0pt}{\isadigit{3}}{\isadigit{5}}{\isacharcomma}{\kern0pt}{\isadigit{6}}{\isadigit{2}}{\isacharcomma}{\kern0pt}{\isadigit{6}}{\isadigit{9}}{\isacharcomma}{\kern0pt}{\isadigit{2}}{\isadigit{8}}{\isacharcomma}{\kern0pt}{\isadigit{4}}{\isadigit{1}}{\isacharcomma}{\kern0pt}{\isadigit{1}}{\isadigit{0}}{\isacharcomma}{\kern0pt}{\isadigit{2}}{\isadigit{1}}{\isacharbrackright}{\kern0pt}{\isacharcomma}{\kern0pt}\isanewline
\ \ {\isacharbrackleft}{\kern0pt}{\isadigit{3}}{\isadigit{4}}{\isacharcomma}{\kern0pt}{\isadigit{4}}{\isadigit{9}}{\isacharcomma}{\kern0pt}{\isadigit{6}}{\isadigit{8}}{\isacharcomma}{\kern0pt}{\isadigit{6}}{\isadigit{5}}{\isacharcomma}{\kern0pt}{\isadigit{3}}{\isadigit{6}}{\isacharcomma}{\kern0pt}{\isadigit{6}}{\isadigit{3}}{\isacharcomma}{\kern0pt}{\isadigit{2}}{\isadigit{4}}{\isacharcomma}{\kern0pt}{\isadigit{5}}{\isadigit{5}}{\isacharcomma}{\kern0pt}{\isadigit{4}}{\isadigit{0}}{\isacharbrackright}{\kern0pt}{\isacharcomma}{\kern0pt}\isanewline
\ \ {\isacharbrackleft}{\kern0pt}{\isadigit{5}}{\isadigit{9}}{\isacharcomma}{\kern0pt}{\isadigit{2}}{\isacharcomma}{\kern0pt}{\isadigit{6}}{\isadigit{1}}{\isacharcomma}{\kern0pt}{\isadigit{1}}{\isadigit{6}}{\isacharcomma}{\kern0pt}{\isadigit{6}}{\isadigit{7}}{\isacharcomma}{\kern0pt}{\isadigit{5}}{\isadigit{6}}{\isacharcomma}{\kern0pt}{\isadigit{3}}{\isadigit{7}}{\isacharcomma}{\kern0pt}{\isadigit{2}}{\isadigit{0}}{\isacharcomma}{\kern0pt}{\isadigit{1}}{\isadigit{1}}{\isacharbrackright}{\kern0pt}{\isacharcomma}{\kern0pt}\isanewline
\ \ {\isacharbrackleft}{\kern0pt}{\isadigit{5}}{\isadigit{0}}{\isacharcomma}{\kern0pt}{\isadigit{1}}{\isadigit{5}}{\isacharcomma}{\kern0pt}{\isadigit{6}}{\isadigit{6}}{\isacharcomma}{\kern0pt}{\isadigit{5}}{\isadigit{7}}{\isacharcomma}{\kern0pt}{\isadigit{5}}{\isadigit{2}}{\isacharcomma}{\kern0pt}{\isadigit{1}}{\isadigit{3}}{\isacharcomma}{\kern0pt}{\isadigit{1}}{\isadigit{8}}{\isacharcomma}{\kern0pt}{\isadigit{3}}{\isadigit{9}}{\isacharcomma}{\kern0pt}{\isadigit{5}}{\isadigit{4}}{\isacharbrackright}{\kern0pt}{\isacharcomma}{\kern0pt}\isanewline
\ \ {\isacharbrackleft}{\kern0pt}{\isadigit{1}}{\isacharcomma}{\kern0pt}{\isadigit{5}}{\isadigit{8}}{\isacharcomma}{\kern0pt}{\isadigit{5}}{\isadigit{1}}{\isacharcomma}{\kern0pt}{\isadigit{1}}{\isadigit{4}}{\isacharcomma}{\kern0pt}{\isadigit{1}}{\isadigit{7}}{\isacharcomma}{\kern0pt}{\isadigit{3}}{\isadigit{8}}{\isacharcomma}{\kern0pt}{\isadigit{5}}{\isadigit{3}}{\isacharcomma}{\kern0pt}{\isadigit{1}}{\isadigit{2}}{\isacharcomma}{\kern0pt}{\isadigit{1}}{\isadigit{9}}{\isacharbrackright}{\kern0pt}{\isacharbrackright}{\kern0pt}{\isacharparenright}{\kern0pt}{\isachardoublequoteclose}\isanewline
\isacommand{lemma}\isamarkupfalse%
\ kp{\isacharunderscore}{\kern0pt}{\isadigit{8}}x{\isadigit{9}}{\isacharunderscore}{\kern0pt}ul{\isacharcolon}{\kern0pt}\ {\isachardoublequoteopen}knights{\isacharunderscore}{\kern0pt}path\ b{\isadigit{8}}x{\isadigit{9}}\ kp{\isadigit{8}}x{\isadigit{9}}ul{\isachardoublequoteclose}\isanewline
%
\isadelimproof
\ \ %
\endisadelimproof
%
\isatagproof
\isacommand{by}\isamarkupfalse%
\ {\isacharparenleft}{\kern0pt}simp\ only{\isacharcolon}{\kern0pt}\ knights{\isacharunderscore}{\kern0pt}path{\isacharunderscore}{\kern0pt}exec{\isacharunderscore}{\kern0pt}simp{\isacharparenright}{\kern0pt}\ eval%
\endisatagproof
{\isafoldproof}%
%
\isadelimproof
\isanewline
%
\endisadelimproof
\isanewline
\isacommand{lemma}\isamarkupfalse%
\ kp{\isacharunderscore}{\kern0pt}{\isadigit{8}}x{\isadigit{9}}{\isacharunderscore}{\kern0pt}ul{\isacharunderscore}{\kern0pt}hd{\isacharcolon}{\kern0pt}\ {\isachardoublequoteopen}hd\ kp{\isadigit{8}}x{\isadigit{9}}ul\ {\isacharequal}{\kern0pt}\ {\isacharparenleft}{\kern0pt}{\isadigit{1}}{\isacharcomma}{\kern0pt}{\isadigit{1}}{\isacharparenright}{\kern0pt}{\isachardoublequoteclose}%
\isadelimproof
\ %
\endisadelimproof
%
\isatagproof
\isacommand{by}\isamarkupfalse%
\ eval%
\endisatagproof
{\isafoldproof}%
%
\isadelimproof
%
\endisadelimproof
\isanewline
\isanewline
\isacommand{lemma}\isamarkupfalse%
\ kp{\isacharunderscore}{\kern0pt}{\isadigit{8}}x{\isadigit{9}}{\isacharunderscore}{\kern0pt}ul{\isacharunderscore}{\kern0pt}last{\isacharcolon}{\kern0pt}\ {\isachardoublequoteopen}last\ kp{\isadigit{8}}x{\isadigit{9}}ul\ {\isacharequal}{\kern0pt}\ {\isacharparenleft}{\kern0pt}{\isadigit{7}}{\isacharcomma}{\kern0pt}{\isadigit{2}}{\isacharparenright}{\kern0pt}{\isachardoublequoteclose}%
\isadelimproof
\ %
\endisadelimproof
%
\isatagproof
\isacommand{by}\isamarkupfalse%
\ eval%
\endisatagproof
{\isafoldproof}%
%
\isadelimproof
%
\endisadelimproof
\isanewline
\isanewline
\isacommand{lemma}\isamarkupfalse%
\ kp{\isacharunderscore}{\kern0pt}{\isadigit{8}}x{\isadigit{9}}{\isacharunderscore}{\kern0pt}ul{\isacharunderscore}{\kern0pt}non{\isacharunderscore}{\kern0pt}nil{\isacharcolon}{\kern0pt}\ {\isachardoublequoteopen}kp{\isadigit{8}}x{\isadigit{9}}ul\ {\isasymnoteq}\ {\isacharbrackleft}{\kern0pt}{\isacharbrackright}{\kern0pt}{\isachardoublequoteclose}%
\isadelimproof
\ %
\endisadelimproof
%
\isatagproof
\isacommand{by}\isamarkupfalse%
\ eval%
\endisatagproof
{\isafoldproof}%
%
\isadelimproof
%
\endisadelimproof
%
\begin{isamarkuptext}%
A Knight's circuit for the \isa{{\isacharparenleft}{\kern0pt}{\isadigit{8}}{\isasymtimes}{\isadigit{9}}{\isacharparenright}{\kern0pt}}-board.
  \begin{table}[H]
    \begin{tabular}{lllllllll}
      42 & 19 & 38 &  5 & 36 & 21 & 34 &  7 & 60 \\
      39 &  4 & 41 & 20 & 63 &  6 & 59 & 22 & 33 \\
      18 & 43 & 70 & 37 & 58 & 35 & 68 & 61 &  8 \\
       3 & 40 & 49 & 64 & 69 & 62 & 57 & 32 & 23 \\
      50 & 17 & 44 & 71 & 48 & 67 & 54 &  9 & 56 \\
      45 &  2 & 65 & 14 & 27 & 12 & 29 & 24 & 31 \\
      16 & 51 & 72 & 47 & 66 & 53 & 26 & 55 & 10 \\
       1 & 46 & 15 & 52 & 13 & 28 & 11 & 30 & 25
    \end{tabular}
  \end{table}%
\end{isamarkuptext}\isamarkuptrue%
\isacommand{abbreviation}\isamarkupfalse%
\ {\isachardoublequoteopen}kc{\isadigit{8}}x{\isadigit{9}}\ {\isasymequiv}\ the\ {\isacharparenleft}{\kern0pt}to{\isacharunderscore}{\kern0pt}path\ \isanewline
\ \ {\isacharbrackleft}{\kern0pt}{\isacharbrackleft}{\kern0pt}{\isadigit{4}}{\isadigit{2}}{\isacharcomma}{\kern0pt}{\isadigit{1}}{\isadigit{9}}{\isacharcomma}{\kern0pt}{\isadigit{3}}{\isadigit{8}}{\isacharcomma}{\kern0pt}{\isadigit{5}}{\isacharcomma}{\kern0pt}{\isadigit{3}}{\isadigit{6}}{\isacharcomma}{\kern0pt}{\isadigit{2}}{\isadigit{1}}{\isacharcomma}{\kern0pt}{\isadigit{3}}{\isadigit{4}}{\isacharcomma}{\kern0pt}{\isadigit{7}}{\isacharcomma}{\kern0pt}{\isadigit{6}}{\isadigit{0}}{\isacharbrackright}{\kern0pt}{\isacharcomma}{\kern0pt}\isanewline
\ \ {\isacharbrackleft}{\kern0pt}{\isadigit{3}}{\isadigit{9}}{\isacharcomma}{\kern0pt}{\isadigit{4}}{\isacharcomma}{\kern0pt}{\isadigit{4}}{\isadigit{1}}{\isacharcomma}{\kern0pt}{\isadigit{2}}{\isadigit{0}}{\isacharcomma}{\kern0pt}{\isadigit{6}}{\isadigit{3}}{\isacharcomma}{\kern0pt}{\isadigit{6}}{\isacharcomma}{\kern0pt}{\isadigit{5}}{\isadigit{9}}{\isacharcomma}{\kern0pt}{\isadigit{2}}{\isadigit{2}}{\isacharcomma}{\kern0pt}{\isadigit{3}}{\isadigit{3}}{\isacharbrackright}{\kern0pt}{\isacharcomma}{\kern0pt}\isanewline
\ \ {\isacharbrackleft}{\kern0pt}{\isadigit{1}}{\isadigit{8}}{\isacharcomma}{\kern0pt}{\isadigit{4}}{\isadigit{3}}{\isacharcomma}{\kern0pt}{\isadigit{7}}{\isadigit{0}}{\isacharcomma}{\kern0pt}{\isadigit{3}}{\isadigit{7}}{\isacharcomma}{\kern0pt}{\isadigit{5}}{\isadigit{8}}{\isacharcomma}{\kern0pt}{\isadigit{3}}{\isadigit{5}}{\isacharcomma}{\kern0pt}{\isadigit{6}}{\isadigit{8}}{\isacharcomma}{\kern0pt}{\isadigit{6}}{\isadigit{1}}{\isacharcomma}{\kern0pt}{\isadigit{8}}{\isacharbrackright}{\kern0pt}{\isacharcomma}{\kern0pt}\isanewline
\ \ {\isacharbrackleft}{\kern0pt}{\isadigit{3}}{\isacharcomma}{\kern0pt}{\isadigit{4}}{\isadigit{0}}{\isacharcomma}{\kern0pt}{\isadigit{4}}{\isadigit{9}}{\isacharcomma}{\kern0pt}{\isadigit{6}}{\isadigit{4}}{\isacharcomma}{\kern0pt}{\isadigit{6}}{\isadigit{9}}{\isacharcomma}{\kern0pt}{\isadigit{6}}{\isadigit{2}}{\isacharcomma}{\kern0pt}{\isadigit{5}}{\isadigit{7}}{\isacharcomma}{\kern0pt}{\isadigit{3}}{\isadigit{2}}{\isacharcomma}{\kern0pt}{\isadigit{2}}{\isadigit{3}}{\isacharbrackright}{\kern0pt}{\isacharcomma}{\kern0pt}\isanewline
\ \ {\isacharbrackleft}{\kern0pt}{\isadigit{5}}{\isadigit{0}}{\isacharcomma}{\kern0pt}{\isadigit{1}}{\isadigit{7}}{\isacharcomma}{\kern0pt}{\isadigit{4}}{\isadigit{4}}{\isacharcomma}{\kern0pt}{\isadigit{7}}{\isadigit{1}}{\isacharcomma}{\kern0pt}{\isadigit{4}}{\isadigit{8}}{\isacharcomma}{\kern0pt}{\isadigit{6}}{\isadigit{7}}{\isacharcomma}{\kern0pt}{\isadigit{5}}{\isadigit{4}}{\isacharcomma}{\kern0pt}{\isadigit{9}}{\isacharcomma}{\kern0pt}{\isadigit{5}}{\isadigit{6}}{\isacharbrackright}{\kern0pt}{\isacharcomma}{\kern0pt}\isanewline
\ \ {\isacharbrackleft}{\kern0pt}{\isadigit{4}}{\isadigit{5}}{\isacharcomma}{\kern0pt}{\isadigit{2}}{\isacharcomma}{\kern0pt}{\isadigit{6}}{\isadigit{5}}{\isacharcomma}{\kern0pt}{\isadigit{1}}{\isadigit{4}}{\isacharcomma}{\kern0pt}{\isadigit{2}}{\isadigit{7}}{\isacharcomma}{\kern0pt}{\isadigit{1}}{\isadigit{2}}{\isacharcomma}{\kern0pt}{\isadigit{2}}{\isadigit{9}}{\isacharcomma}{\kern0pt}{\isadigit{2}}{\isadigit{4}}{\isacharcomma}{\kern0pt}{\isadigit{3}}{\isadigit{1}}{\isacharbrackright}{\kern0pt}{\isacharcomma}{\kern0pt}\isanewline
\ \ {\isacharbrackleft}{\kern0pt}{\isadigit{1}}{\isadigit{6}}{\isacharcomma}{\kern0pt}{\isadigit{5}}{\isadigit{1}}{\isacharcomma}{\kern0pt}{\isadigit{7}}{\isadigit{2}}{\isacharcomma}{\kern0pt}{\isadigit{4}}{\isadigit{7}}{\isacharcomma}{\kern0pt}{\isadigit{6}}{\isadigit{6}}{\isacharcomma}{\kern0pt}{\isadigit{5}}{\isadigit{3}}{\isacharcomma}{\kern0pt}{\isadigit{2}}{\isadigit{6}}{\isacharcomma}{\kern0pt}{\isadigit{5}}{\isadigit{5}}{\isacharcomma}{\kern0pt}{\isadigit{1}}{\isadigit{0}}{\isacharbrackright}{\kern0pt}{\isacharcomma}{\kern0pt}\isanewline
\ \ {\isacharbrackleft}{\kern0pt}{\isadigit{1}}{\isacharcomma}{\kern0pt}{\isadigit{4}}{\isadigit{6}}{\isacharcomma}{\kern0pt}{\isadigit{1}}{\isadigit{5}}{\isacharcomma}{\kern0pt}{\isadigit{5}}{\isadigit{2}}{\isacharcomma}{\kern0pt}{\isadigit{1}}{\isadigit{3}}{\isacharcomma}{\kern0pt}{\isadigit{2}}{\isadigit{8}}{\isacharcomma}{\kern0pt}{\isadigit{1}}{\isadigit{1}}{\isacharcomma}{\kern0pt}{\isadigit{3}}{\isadigit{0}}{\isacharcomma}{\kern0pt}{\isadigit{2}}{\isadigit{5}}{\isacharbrackright}{\kern0pt}{\isacharbrackright}{\kern0pt}{\isacharparenright}{\kern0pt}{\isachardoublequoteclose}\isanewline
\isacommand{lemma}\isamarkupfalse%
\ kc{\isacharunderscore}{\kern0pt}{\isadigit{8}}x{\isadigit{9}}{\isacharcolon}{\kern0pt}\ {\isachardoublequoteopen}knights{\isacharunderscore}{\kern0pt}circuit\ b{\isadigit{8}}x{\isadigit{9}}\ kc{\isadigit{8}}x{\isadigit{9}}{\isachardoublequoteclose}\isanewline
%
\isadelimproof
\ \ %
\endisadelimproof
%
\isatagproof
\isacommand{by}\isamarkupfalse%
\ {\isacharparenleft}{\kern0pt}simp\ only{\isacharcolon}{\kern0pt}\ knights{\isacharunderscore}{\kern0pt}circuit{\isacharunderscore}{\kern0pt}exec{\isacharunderscore}{\kern0pt}simp{\isacharparenright}{\kern0pt}\ eval%
\endisatagproof
{\isafoldproof}%
%
\isadelimproof
\isanewline
%
\endisadelimproof
\isanewline
\isacommand{lemma}\isamarkupfalse%
\ kc{\isacharunderscore}{\kern0pt}{\isadigit{8}}x{\isadigit{9}}{\isacharunderscore}{\kern0pt}hd{\isacharcolon}{\kern0pt}\ {\isachardoublequoteopen}hd\ kc{\isadigit{8}}x{\isadigit{9}}\ {\isacharequal}{\kern0pt}\ {\isacharparenleft}{\kern0pt}{\isadigit{1}}{\isacharcomma}{\kern0pt}{\isadigit{1}}{\isacharparenright}{\kern0pt}{\isachardoublequoteclose}%
\isadelimproof
\ %
\endisadelimproof
%
\isatagproof
\isacommand{by}\isamarkupfalse%
\ eval%
\endisatagproof
{\isafoldproof}%
%
\isadelimproof
%
\endisadelimproof
\isanewline
\isanewline
\isacommand{lemma}\isamarkupfalse%
\ kc{\isacharunderscore}{\kern0pt}{\isadigit{8}}x{\isadigit{9}}{\isacharunderscore}{\kern0pt}non{\isacharunderscore}{\kern0pt}nil{\isacharcolon}{\kern0pt}\ {\isachardoublequoteopen}kc{\isadigit{8}}x{\isadigit{9}}\ {\isasymnoteq}\ {\isacharbrackleft}{\kern0pt}{\isacharbrackright}{\kern0pt}{\isachardoublequoteclose}%
\isadelimproof
\ %
\endisadelimproof
%
\isatagproof
\isacommand{by}\isamarkupfalse%
\ eval%
\endisatagproof
{\isafoldproof}%
%
\isadelimproof
%
\endisadelimproof
\isanewline
\isanewline
\isacommand{lemma}\isamarkupfalse%
\ kc{\isacharunderscore}{\kern0pt}{\isadigit{8}}x{\isadigit{9}}{\isacharunderscore}{\kern0pt}si{\isacharcolon}{\kern0pt}\ {\isachardoublequoteopen}step{\isacharunderscore}{\kern0pt}in\ kc{\isadigit{8}}x{\isadigit{9}}\ {\isacharparenleft}{\kern0pt}{\isadigit{2}}{\isacharcomma}{\kern0pt}{\isadigit{8}}{\isacharparenright}{\kern0pt}\ {\isacharparenleft}{\kern0pt}{\isadigit{4}}{\isacharcomma}{\kern0pt}{\isadigit{9}}{\isacharparenright}{\kern0pt}{\isachardoublequoteclose}\ {\isacharparenleft}{\kern0pt}\isakeyword{is}\ {\isachardoublequoteopen}step{\isacharunderscore}{\kern0pt}in\ {\isacharquery}{\kern0pt}ps\ {\isacharunderscore}{\kern0pt}\ {\isacharunderscore}{\kern0pt}{\isachardoublequoteclose}{\isacharparenright}{\kern0pt}\isanewline
%
\isadelimproof
%
\endisadelimproof
%
\isatagproof
\isacommand{proof}\isamarkupfalse%
\ {\isacharminus}{\kern0pt}\isanewline
\ \ \isacommand{have}\isamarkupfalse%
\ {\isachardoublequoteopen}{\isadigit{0}}\ {\isacharless}{\kern0pt}\ {\isacharparenleft}{\kern0pt}{\isadigit{5}}{\isadigit{5}}{\isacharcolon}{\kern0pt}{\isacharcolon}{\kern0pt}nat{\isacharparenright}{\kern0pt}{\isachardoublequoteclose}\ {\isachardoublequoteopen}{\isadigit{5}}{\isadigit{5}}\ {\isacharless}{\kern0pt}\ length\ {\isacharquery}{\kern0pt}ps{\isachardoublequoteclose}\ \isanewline
\ \ \ \ \ \ \ \ {\isachardoublequoteopen}last\ {\isacharparenleft}{\kern0pt}take\ {\isadigit{5}}{\isadigit{5}}\ {\isacharquery}{\kern0pt}ps{\isacharparenright}{\kern0pt}\ {\isacharequal}{\kern0pt}\ {\isacharparenleft}{\kern0pt}{\isadigit{2}}{\isacharcomma}{\kern0pt}{\isadigit{8}}{\isacharparenright}{\kern0pt}{\isachardoublequoteclose}\ {\isachardoublequoteopen}hd\ {\isacharparenleft}{\kern0pt}drop\ {\isadigit{5}}{\isadigit{5}}\ {\isacharquery}{\kern0pt}ps{\isacharparenright}{\kern0pt}\ {\isacharequal}{\kern0pt}\ {\isacharparenleft}{\kern0pt}{\isadigit{4}}{\isacharcomma}{\kern0pt}{\isadigit{9}}{\isacharparenright}{\kern0pt}{\isachardoublequoteclose}\ \isacommand{by}\isamarkupfalse%
\ eval{\isacharplus}{\kern0pt}\isanewline
\ \ \isacommand{then}\isamarkupfalse%
\ \isacommand{show}\isamarkupfalse%
\ {\isacharquery}{\kern0pt}thesis\ \isacommand{unfolding}\isamarkupfalse%
\ step{\isacharunderscore}{\kern0pt}in{\isacharunderscore}{\kern0pt}def\ \isacommand{by}\isamarkupfalse%
\ blast\isanewline
\isacommand{qed}\isamarkupfalse%
%
\endisatagproof
{\isafoldproof}%
%
\isadelimproof
\isanewline
%
\endisadelimproof
\isanewline
\isacommand{lemmas}\isamarkupfalse%
\ kp{\isacharunderscore}{\kern0pt}{\isadigit{8}}xm{\isacharunderscore}{\kern0pt}ul\ {\isacharequal}{\kern0pt}\ \isanewline
\ \ kp{\isacharunderscore}{\kern0pt}{\isadigit{8}}x{\isadigit{5}}{\isacharunderscore}{\kern0pt}ul\ kp{\isacharunderscore}{\kern0pt}{\isadigit{8}}x{\isadigit{5}}{\isacharunderscore}{\kern0pt}ul{\isacharunderscore}{\kern0pt}hd\ kp{\isacharunderscore}{\kern0pt}{\isadigit{8}}x{\isadigit{5}}{\isacharunderscore}{\kern0pt}ul{\isacharunderscore}{\kern0pt}last\ kp{\isacharunderscore}{\kern0pt}{\isadigit{8}}x{\isadigit{5}}{\isacharunderscore}{\kern0pt}ul{\isacharunderscore}{\kern0pt}non{\isacharunderscore}{\kern0pt}nil\isanewline
\ \ kp{\isacharunderscore}{\kern0pt}{\isadigit{8}}x{\isadigit{6}}{\isacharunderscore}{\kern0pt}ul\ kp{\isacharunderscore}{\kern0pt}{\isadigit{8}}x{\isadigit{6}}{\isacharunderscore}{\kern0pt}ul{\isacharunderscore}{\kern0pt}hd\ kp{\isacharunderscore}{\kern0pt}{\isadigit{8}}x{\isadigit{6}}{\isacharunderscore}{\kern0pt}ul{\isacharunderscore}{\kern0pt}last\ kp{\isacharunderscore}{\kern0pt}{\isadigit{8}}x{\isadigit{6}}{\isacharunderscore}{\kern0pt}ul{\isacharunderscore}{\kern0pt}non{\isacharunderscore}{\kern0pt}nil\isanewline
\ \ kp{\isacharunderscore}{\kern0pt}{\isadigit{8}}x{\isadigit{7}}{\isacharunderscore}{\kern0pt}ul\ kp{\isacharunderscore}{\kern0pt}{\isadigit{8}}x{\isadigit{7}}{\isacharunderscore}{\kern0pt}ul{\isacharunderscore}{\kern0pt}hd\ kp{\isacharunderscore}{\kern0pt}{\isadigit{8}}x{\isadigit{7}}{\isacharunderscore}{\kern0pt}ul{\isacharunderscore}{\kern0pt}last\ kp{\isacharunderscore}{\kern0pt}{\isadigit{8}}x{\isadigit{7}}{\isacharunderscore}{\kern0pt}ul{\isacharunderscore}{\kern0pt}non{\isacharunderscore}{\kern0pt}nil\isanewline
\ \ kp{\isacharunderscore}{\kern0pt}{\isadigit{8}}x{\isadigit{8}}{\isacharunderscore}{\kern0pt}ul\ kp{\isacharunderscore}{\kern0pt}{\isadigit{8}}x{\isadigit{8}}{\isacharunderscore}{\kern0pt}ul{\isacharunderscore}{\kern0pt}hd\ kp{\isacharunderscore}{\kern0pt}{\isadigit{8}}x{\isadigit{8}}{\isacharunderscore}{\kern0pt}ul{\isacharunderscore}{\kern0pt}last\ kp{\isacharunderscore}{\kern0pt}{\isadigit{8}}x{\isadigit{8}}{\isacharunderscore}{\kern0pt}ul{\isacharunderscore}{\kern0pt}non{\isacharunderscore}{\kern0pt}nil\isanewline
\ \ kp{\isacharunderscore}{\kern0pt}{\isadigit{8}}x{\isadigit{9}}{\isacharunderscore}{\kern0pt}ul\ kp{\isacharunderscore}{\kern0pt}{\isadigit{8}}x{\isadigit{9}}{\isacharunderscore}{\kern0pt}ul{\isacharunderscore}{\kern0pt}hd\ kp{\isacharunderscore}{\kern0pt}{\isadigit{8}}x{\isadigit{9}}{\isacharunderscore}{\kern0pt}ul{\isacharunderscore}{\kern0pt}last\ kp{\isacharunderscore}{\kern0pt}{\isadigit{8}}x{\isadigit{9}}{\isacharunderscore}{\kern0pt}ul{\isacharunderscore}{\kern0pt}non{\isacharunderscore}{\kern0pt}nil\isanewline
\isanewline
\isacommand{lemmas}\isamarkupfalse%
\ kc{\isacharunderscore}{\kern0pt}{\isadigit{8}}xm\ {\isacharequal}{\kern0pt}\ \isanewline
\ \ kc{\isacharunderscore}{\kern0pt}{\isadigit{8}}x{\isadigit{5}}\ kc{\isacharunderscore}{\kern0pt}{\isadigit{8}}x{\isadigit{5}}{\isacharunderscore}{\kern0pt}hd\ kc{\isacharunderscore}{\kern0pt}{\isadigit{8}}x{\isadigit{5}}{\isacharunderscore}{\kern0pt}last\ kc{\isacharunderscore}{\kern0pt}{\isadigit{8}}x{\isadigit{5}}{\isacharunderscore}{\kern0pt}non{\isacharunderscore}{\kern0pt}nil\ kc{\isacharunderscore}{\kern0pt}{\isadigit{8}}x{\isadigit{5}}{\isacharunderscore}{\kern0pt}si\isanewline
\ \ kc{\isacharunderscore}{\kern0pt}{\isadigit{8}}x{\isadigit{6}}\ kc{\isacharunderscore}{\kern0pt}{\isadigit{8}}x{\isadigit{6}}{\isacharunderscore}{\kern0pt}hd\ kc{\isacharunderscore}{\kern0pt}{\isadigit{8}}x{\isadigit{6}}{\isacharunderscore}{\kern0pt}non{\isacharunderscore}{\kern0pt}nil\ kc{\isacharunderscore}{\kern0pt}{\isadigit{8}}x{\isadigit{6}}{\isacharunderscore}{\kern0pt}si\isanewline
\ \ kc{\isacharunderscore}{\kern0pt}{\isadigit{8}}x{\isadigit{7}}\ kc{\isacharunderscore}{\kern0pt}{\isadigit{8}}x{\isadigit{7}}{\isacharunderscore}{\kern0pt}hd\ kc{\isacharunderscore}{\kern0pt}{\isadigit{8}}x{\isadigit{7}}{\isacharunderscore}{\kern0pt}non{\isacharunderscore}{\kern0pt}nil\ kc{\isacharunderscore}{\kern0pt}{\isadigit{8}}x{\isadigit{7}}{\isacharunderscore}{\kern0pt}si\isanewline
\ \ kc{\isacharunderscore}{\kern0pt}{\isadigit{8}}x{\isadigit{8}}\ kc{\isacharunderscore}{\kern0pt}{\isadigit{8}}x{\isadigit{8}}{\isacharunderscore}{\kern0pt}hd\ kc{\isacharunderscore}{\kern0pt}{\isadigit{8}}x{\isadigit{8}}{\isacharunderscore}{\kern0pt}non{\isacharunderscore}{\kern0pt}nil\ kc{\isacharunderscore}{\kern0pt}{\isadigit{8}}x{\isadigit{8}}{\isacharunderscore}{\kern0pt}si\isanewline
\ \ kc{\isacharunderscore}{\kern0pt}{\isadigit{8}}x{\isadigit{9}}\ kc{\isacharunderscore}{\kern0pt}{\isadigit{8}}x{\isadigit{9}}{\isacharunderscore}{\kern0pt}hd\ kc{\isacharunderscore}{\kern0pt}{\isadigit{8}}x{\isadigit{9}}{\isacharunderscore}{\kern0pt}non{\isacharunderscore}{\kern0pt}nil\ kc{\isacharunderscore}{\kern0pt}{\isadigit{8}}x{\isadigit{9}}{\isacharunderscore}{\kern0pt}si%
\begin{isamarkuptext}%
For every \isa{{\isadigit{8}}{\isasymtimes}m}-board with \isa{m\ {\isasymge}\ {\isadigit{5}}} there exists a knight's circuit.%
\end{isamarkuptext}\isamarkuptrue%
\isacommand{lemma}\isamarkupfalse%
\ knights{\isacharunderscore}{\kern0pt}circuit{\isacharunderscore}{\kern0pt}{\isadigit{8}}xm{\isacharunderscore}{\kern0pt}exists{\isacharcolon}{\kern0pt}\ \isanewline
\ \ \isakeyword{assumes}\ {\isachardoublequoteopen}m\ {\isasymge}\ {\isadigit{5}}{\isachardoublequoteclose}\ \isanewline
\ \ \isakeyword{shows}\ {\isachardoublequoteopen}{\isasymexists}ps{\isachardot}{\kern0pt}\ knights{\isacharunderscore}{\kern0pt}circuit\ {\isacharparenleft}{\kern0pt}board\ {\isadigit{8}}\ m{\isacharparenright}{\kern0pt}\ ps\ {\isasymand}\ step{\isacharunderscore}{\kern0pt}in\ ps\ {\isacharparenleft}{\kern0pt}{\isadigit{2}}{\isacharcomma}{\kern0pt}int\ m{\isacharminus}{\kern0pt}{\isadigit{1}}{\isacharparenright}{\kern0pt}\ {\isacharparenleft}{\kern0pt}{\isadigit{4}}{\isacharcomma}{\kern0pt}int\ m{\isacharparenright}{\kern0pt}{\isachardoublequoteclose}\isanewline
%
\isadelimproof
\ \ %
\endisadelimproof
%
\isatagproof
\isacommand{using}\isamarkupfalse%
\ assms\isanewline
\isacommand{proof}\isamarkupfalse%
\ {\isacharparenleft}{\kern0pt}induction\ m\ rule{\isacharcolon}{\kern0pt}\ less{\isacharunderscore}{\kern0pt}induct{\isacharparenright}{\kern0pt}\isanewline
\ \ \isacommand{case}\isamarkupfalse%
\ {\isacharparenleft}{\kern0pt}less\ m{\isacharparenright}{\kern0pt}\isanewline
\ \ \isacommand{then}\isamarkupfalse%
\ \isacommand{have}\isamarkupfalse%
\ {\isachardoublequoteopen}m\ {\isasymin}\ {\isacharbraceleft}{\kern0pt}{\isadigit{5}}{\isacharcomma}{\kern0pt}{\isadigit{6}}{\isacharcomma}{\kern0pt}{\isadigit{7}}{\isacharcomma}{\kern0pt}{\isadigit{8}}{\isacharcomma}{\kern0pt}{\isadigit{9}}{\isacharbraceright}{\kern0pt}\ {\isasymor}\ {\isadigit{5}}\ {\isasymle}\ m{\isacharminus}{\kern0pt}{\isadigit{5}}{\isachardoublequoteclose}\ \isacommand{by}\isamarkupfalse%
\ auto\isanewline
\ \ \isacommand{then}\isamarkupfalse%
\ \isacommand{show}\isamarkupfalse%
\ {\isacharquery}{\kern0pt}case\isanewline
\ \ \isacommand{proof}\isamarkupfalse%
\ {\isacharparenleft}{\kern0pt}elim\ disjE{\isacharparenright}{\kern0pt}\isanewline
\ \ \ \ \isacommand{assume}\isamarkupfalse%
\ {\isachardoublequoteopen}m\ {\isasymin}\ {\isacharbraceleft}{\kern0pt}{\isadigit{5}}{\isacharcomma}{\kern0pt}{\isadigit{6}}{\isacharcomma}{\kern0pt}{\isadigit{7}}{\isacharcomma}{\kern0pt}{\isadigit{8}}{\isacharcomma}{\kern0pt}{\isadigit{9}}{\isacharbraceright}{\kern0pt}{\isachardoublequoteclose}\isanewline
\ \ \ \ \isacommand{then}\isamarkupfalse%
\ \isacommand{show}\isamarkupfalse%
\ {\isacharquery}{\kern0pt}thesis\ \isacommand{using}\isamarkupfalse%
\ kc{\isacharunderscore}{\kern0pt}{\isadigit{8}}xm\ \isacommand{by}\isamarkupfalse%
\ fastforce\isanewline
\ \ \isacommand{next}\isamarkupfalse%
\isanewline
\ \ \ \ \isacommand{let}\isamarkupfalse%
\ {\isacharquery}{\kern0pt}ps\isactrlsub {\isadigit{2}}{\isacharequal}{\kern0pt}{\isachardoublequoteopen}kc{\isadigit{8}}x{\isadigit{5}}{\isachardoublequoteclose}\isanewline
\ \ \ \ \isacommand{let}\isamarkupfalse%
\ {\isacharquery}{\kern0pt}b\isactrlsub {\isadigit{2}}{\isacharequal}{\kern0pt}{\isachardoublequoteopen}board\ {\isadigit{8}}\ {\isadigit{5}}{\isachardoublequoteclose}\isanewline
\ \ \ \ \isacommand{have}\isamarkupfalse%
\ ps\isactrlsub {\isadigit{2}}{\isacharunderscore}{\kern0pt}prems{\isacharcolon}{\kern0pt}\ {\isachardoublequoteopen}knights{\isacharunderscore}{\kern0pt}circuit\ {\isacharquery}{\kern0pt}b\isactrlsub {\isadigit{2}}\ {\isacharquery}{\kern0pt}ps\isactrlsub {\isadigit{2}}{\isachardoublequoteclose}\ {\isachardoublequoteopen}hd\ {\isacharquery}{\kern0pt}ps\isactrlsub {\isadigit{2}}\ {\isacharequal}{\kern0pt}\ {\isacharparenleft}{\kern0pt}{\isadigit{1}}{\isacharcomma}{\kern0pt}{\isadigit{1}}{\isacharparenright}{\kern0pt}{\isachardoublequoteclose}\ {\isachardoublequoteopen}last\ {\isacharquery}{\kern0pt}ps\isactrlsub {\isadigit{2}}\ {\isacharequal}{\kern0pt}\ {\isacharparenleft}{\kern0pt}{\isadigit{3}}{\isacharcomma}{\kern0pt}{\isadigit{2}}{\isacharparenright}{\kern0pt}{\isachardoublequoteclose}\isanewline
\ \ \ \ \ \ \isacommand{using}\isamarkupfalse%
\ kc{\isacharunderscore}{\kern0pt}{\isadigit{8}}xm\ \isacommand{by}\isamarkupfalse%
\ auto\isanewline
\ \ \ \ \isacommand{have}\isamarkupfalse%
\ {\isachardoublequoteopen}{\isadigit{2}}{\isadigit{1}}\ {\isacharless}{\kern0pt}\ length\ {\isacharquery}{\kern0pt}ps\isactrlsub {\isadigit{2}}{\isachardoublequoteclose}\ {\isachardoublequoteopen}last\ {\isacharparenleft}{\kern0pt}take\ {\isadigit{2}}{\isadigit{1}}\ {\isacharquery}{\kern0pt}ps\isactrlsub {\isadigit{2}}{\isacharparenright}{\kern0pt}\ {\isacharequal}{\kern0pt}\ {\isacharparenleft}{\kern0pt}{\isadigit{2}}{\isacharcomma}{\kern0pt}int\ {\isadigit{5}}{\isacharminus}{\kern0pt}{\isadigit{1}}{\isacharparenright}{\kern0pt}{\isachardoublequoteclose}\ {\isachardoublequoteopen}hd\ {\isacharparenleft}{\kern0pt}drop\ {\isadigit{2}}{\isadigit{1}}\ {\isacharquery}{\kern0pt}ps\isactrlsub {\isadigit{2}}{\isacharparenright}{\kern0pt}\ {\isacharequal}{\kern0pt}\ {\isacharparenleft}{\kern0pt}{\isadigit{4}}{\isacharcomma}{\kern0pt}int\ {\isadigit{5}}{\isacharparenright}{\kern0pt}{\isachardoublequoteclose}\ \isanewline
\ \ \ \ \ \ \isacommand{by}\isamarkupfalse%
\ eval{\isacharplus}{\kern0pt}\isanewline
\ \ \ \ \isacommand{then}\isamarkupfalse%
\ \isacommand{have}\isamarkupfalse%
\ si{\isacharcolon}{\kern0pt}\ {\isachardoublequoteopen}step{\isacharunderscore}{\kern0pt}in\ {\isacharquery}{\kern0pt}ps\isactrlsub {\isadigit{2}}\ {\isacharparenleft}{\kern0pt}{\isadigit{2}}{\isacharcomma}{\kern0pt}int\ {\isadigit{5}}{\isacharminus}{\kern0pt}{\isadigit{1}}{\isacharparenright}{\kern0pt}\ {\isacharparenleft}{\kern0pt}{\isadigit{4}}{\isacharcomma}{\kern0pt}int\ {\isadigit{5}}{\isacharparenright}{\kern0pt}{\isachardoublequoteclose}\isanewline
\ \ \ \ \ \ \isacommand{unfolding}\isamarkupfalse%
\ step{\isacharunderscore}{\kern0pt}in{\isacharunderscore}{\kern0pt}def\ \isacommand{using}\isamarkupfalse%
\ zero{\isacharunderscore}{\kern0pt}less{\isacharunderscore}{\kern0pt}numeral\ \isacommand{by}\isamarkupfalse%
\ blast\isanewline
\ \ \ \ \isacommand{assume}\isamarkupfalse%
\ m{\isacharunderscore}{\kern0pt}ge{\isacharcolon}{\kern0pt}\ {\isachardoublequoteopen}{\isadigit{5}}\ {\isasymle}\ m{\isacharminus}{\kern0pt}{\isadigit{5}}{\isachardoublequoteclose}\ \isanewline
\ \ \ \ \isacommand{then}\isamarkupfalse%
\ \isacommand{obtain}\isamarkupfalse%
\ ps\isactrlsub {\isadigit{1}}\ \isakeyword{where}\ ps\isactrlsub {\isadigit{1}}{\isacharunderscore}{\kern0pt}IH{\isacharcolon}{\kern0pt}\ {\isachardoublequoteopen}knights{\isacharunderscore}{\kern0pt}circuit\ {\isacharparenleft}{\kern0pt}board\ {\isadigit{8}}\ {\isacharparenleft}{\kern0pt}m{\isacharminus}{\kern0pt}{\isadigit{5}}{\isacharparenright}{\kern0pt}{\isacharparenright}{\kern0pt}\ ps\isactrlsub {\isadigit{1}}{\isachardoublequoteclose}\isanewline
\ \ \ \ \ \ \ \ \ \ \ \ \ \ \ \ \ \ \ \ \ \ \ \ \ \ \ \ \ \ \ \ {\isachardoublequoteopen}step{\isacharunderscore}{\kern0pt}in\ ps\isactrlsub {\isadigit{1}}\ {\isacharparenleft}{\kern0pt}{\isadigit{2}}{\isacharcomma}{\kern0pt}int\ {\isacharparenleft}{\kern0pt}m{\isacharminus}{\kern0pt}{\isadigit{5}}{\isacharparenright}{\kern0pt}{\isacharminus}{\kern0pt}{\isadigit{1}}{\isacharparenright}{\kern0pt}\ {\isacharparenleft}{\kern0pt}{\isadigit{4}}{\isacharcomma}{\kern0pt}int\ {\isacharparenleft}{\kern0pt}m{\isacharminus}{\kern0pt}{\isadigit{5}}{\isacharparenright}{\kern0pt}{\isacharparenright}{\kern0pt}{\isachardoublequoteclose}\isanewline
\ \ \ \ \ \ \isacommand{using}\isamarkupfalse%
\ less{\isachardot}{\kern0pt}IH{\isacharbrackleft}{\kern0pt}of\ {\isachardoublequoteopen}m{\isacharminus}{\kern0pt}{\isadigit{5}}{\isachardoublequoteclose}{\isacharbrackright}{\kern0pt}\ knights{\isacharunderscore}{\kern0pt}path{\isacharunderscore}{\kern0pt}non{\isacharunderscore}{\kern0pt}nil\ \isacommand{by}\isamarkupfalse%
\ auto\isanewline
\ \ \ \ \isacommand{then}\isamarkupfalse%
\ \isacommand{show}\isamarkupfalse%
\ {\isacharquery}{\kern0pt}thesis\isanewline
\ \ \ \ \ \ \isacommand{using}\isamarkupfalse%
\ m{\isacharunderscore}{\kern0pt}ge\ ps\isactrlsub {\isadigit{2}}{\isacharunderscore}{\kern0pt}prems\ si\ knights{\isacharunderscore}{\kern0pt}circuit{\isacharunderscore}{\kern0pt}lr{\isacharunderscore}{\kern0pt}concat{\isacharbrackleft}{\kern0pt}of\ {\isadigit{8}}\ {\isachardoublequoteopen}m{\isacharminus}{\kern0pt}{\isadigit{5}}{\isachardoublequoteclose}\ ps\isactrlsub {\isadigit{1}}\ {\isadigit{5}}\ {\isacharquery}{\kern0pt}ps\isactrlsub {\isadigit{2}}{\isacharbrackright}{\kern0pt}\ \isacommand{by}\isamarkupfalse%
\ auto\isanewline
\ \ \isacommand{qed}\isamarkupfalse%
\isanewline
\isacommand{qed}\isamarkupfalse%
%
\endisatagproof
{\isafoldproof}%
%
\isadelimproof
%
\endisadelimproof
%
\begin{isamarkuptext}%
For every \isa{{\isadigit{8}}{\isasymtimes}m}-board with \isa{m\ {\isasymge}\ {\isadigit{5}}} there exists a knight's path that starts in 
\isa{{\isacharparenleft}{\kern0pt}{\isadigit{1}}{\isacharcomma}{\kern0pt}{\isadigit{1}}{\isacharparenright}{\kern0pt}} (bottom-left) and ends in \isa{{\isacharparenleft}{\kern0pt}{\isadigit{7}}{\isacharcomma}{\kern0pt}{\isadigit{2}}{\isacharparenright}{\kern0pt}} (top-left).%
\end{isamarkuptext}\isamarkuptrue%
\isacommand{lemma}\isamarkupfalse%
\ knights{\isacharunderscore}{\kern0pt}path{\isacharunderscore}{\kern0pt}{\isadigit{8}}xm{\isacharunderscore}{\kern0pt}ul{\isacharunderscore}{\kern0pt}exists{\isacharcolon}{\kern0pt}\ \isanewline
\ \ \isakeyword{assumes}\ {\isachardoublequoteopen}m\ {\isasymge}\ {\isadigit{5}}{\isachardoublequoteclose}\ \isanewline
\ \ \isakeyword{shows}\ {\isachardoublequoteopen}{\isasymexists}ps{\isachardot}{\kern0pt}\ knights{\isacharunderscore}{\kern0pt}path\ {\isacharparenleft}{\kern0pt}board\ {\isadigit{8}}\ m{\isacharparenright}{\kern0pt}\ ps\ {\isasymand}\ hd\ ps\ {\isacharequal}{\kern0pt}\ {\isacharparenleft}{\kern0pt}{\isadigit{1}}{\isacharcomma}{\kern0pt}{\isadigit{1}}{\isacharparenright}{\kern0pt}\ {\isasymand}\ last\ ps\ {\isacharequal}{\kern0pt}\ {\isacharparenleft}{\kern0pt}{\isadigit{7}}{\isacharcomma}{\kern0pt}{\isadigit{2}}{\isacharparenright}{\kern0pt}{\isachardoublequoteclose}\isanewline
%
\isadelimproof
\ \ %
\endisadelimproof
%
\isatagproof
\isacommand{using}\isamarkupfalse%
\ assms\isanewline
\isacommand{proof}\isamarkupfalse%
\ {\isacharminus}{\kern0pt}\isanewline
\ \ \isacommand{have}\isamarkupfalse%
\ {\isachardoublequoteopen}m\ {\isasymin}\ {\isacharbraceleft}{\kern0pt}{\isadigit{5}}{\isacharcomma}{\kern0pt}{\isadigit{6}}{\isacharcomma}{\kern0pt}{\isadigit{7}}{\isacharcomma}{\kern0pt}{\isadigit{8}}{\isacharcomma}{\kern0pt}{\isadigit{9}}{\isacharbraceright}{\kern0pt}\ {\isasymor}\ {\isadigit{5}}\ {\isasymle}\ m{\isacharminus}{\kern0pt}{\isadigit{5}}{\isachardoublequoteclose}\ \isacommand{using}\isamarkupfalse%
\ assms\ \isacommand{by}\isamarkupfalse%
\ auto\isanewline
\ \ \isacommand{then}\isamarkupfalse%
\ \isacommand{show}\isamarkupfalse%
\ {\isacharquery}{\kern0pt}thesis\isanewline
\ \ \isacommand{proof}\isamarkupfalse%
\ {\isacharparenleft}{\kern0pt}elim\ disjE{\isacharparenright}{\kern0pt}\isanewline
\ \ \ \ \isacommand{assume}\isamarkupfalse%
\ {\isachardoublequoteopen}m\ {\isasymin}\ {\isacharbraceleft}{\kern0pt}{\isadigit{5}}{\isacharcomma}{\kern0pt}{\isadigit{6}}{\isacharcomma}{\kern0pt}{\isadigit{7}}{\isacharcomma}{\kern0pt}{\isadigit{8}}{\isacharcomma}{\kern0pt}{\isadigit{9}}{\isacharbraceright}{\kern0pt}{\isachardoublequoteclose}\isanewline
\ \ \ \ \isacommand{then}\isamarkupfalse%
\ \isacommand{show}\isamarkupfalse%
\ {\isacharquery}{\kern0pt}thesis\ \isacommand{using}\isamarkupfalse%
\ kp{\isacharunderscore}{\kern0pt}{\isadigit{8}}xm{\isacharunderscore}{\kern0pt}ul\ \isacommand{by}\isamarkupfalse%
\ fastforce\isanewline
\ \ \isacommand{next}\isamarkupfalse%
\isanewline
\ \ \ \ \isacommand{let}\isamarkupfalse%
\ {\isacharquery}{\kern0pt}ps\isactrlsub {\isadigit{1}}{\isacharequal}{\kern0pt}{\isachardoublequoteopen}kp{\isadigit{8}}x{\isadigit{5}}ul{\isachardoublequoteclose}\isanewline
\ \ \ \ \isacommand{have}\isamarkupfalse%
\ ps\isactrlsub {\isadigit{1}}{\isacharunderscore}{\kern0pt}prems{\isacharcolon}{\kern0pt}\ {\isachardoublequoteopen}knights{\isacharunderscore}{\kern0pt}path\ b{\isadigit{8}}x{\isadigit{5}}\ {\isacharquery}{\kern0pt}ps\isactrlsub {\isadigit{1}}{\isachardoublequoteclose}\ {\isachardoublequoteopen}hd\ {\isacharquery}{\kern0pt}ps\isactrlsub {\isadigit{1}}\ {\isacharequal}{\kern0pt}\ {\isacharparenleft}{\kern0pt}{\isadigit{1}}{\isacharcomma}{\kern0pt}{\isadigit{1}}{\isacharparenright}{\kern0pt}{\isachardoublequoteclose}\ {\isachardoublequoteopen}last\ {\isacharquery}{\kern0pt}ps\isactrlsub {\isadigit{1}}\ {\isacharequal}{\kern0pt}\ {\isacharparenleft}{\kern0pt}{\isadigit{7}}{\isacharcomma}{\kern0pt}{\isadigit{2}}{\isacharparenright}{\kern0pt}{\isachardoublequoteclose}\isanewline
\ \ \ \ \ \ \isacommand{using}\isamarkupfalse%
\ kp{\isacharunderscore}{\kern0pt}{\isadigit{8}}xm{\isacharunderscore}{\kern0pt}ul\ \isacommand{by}\isamarkupfalse%
\ auto\isanewline
\ \ \ \ \isacommand{assume}\isamarkupfalse%
\ m{\isacharunderscore}{\kern0pt}ge{\isacharcolon}{\kern0pt}\ {\isachardoublequoteopen}{\isadigit{5}}\ {\isasymle}\ m{\isacharminus}{\kern0pt}{\isadigit{5}}{\isachardoublequoteclose}\ \isanewline
\ \ \ \ \isacommand{then}\isamarkupfalse%
\ \isacommand{have}\isamarkupfalse%
\ b{\isacharunderscore}{\kern0pt}prems{\isacharcolon}{\kern0pt}\ {\isachardoublequoteopen}{\isadigit{5}}\ {\isasymle}\ min\ {\isadigit{8}}\ {\isacharparenleft}{\kern0pt}m{\isacharminus}{\kern0pt}{\isadigit{5}}{\isacharparenright}{\kern0pt}{\isachardoublequoteclose}\isanewline
\ \ \ \ \ \ \isacommand{unfolding}\isamarkupfalse%
\ board{\isacharunderscore}{\kern0pt}def\ \isacommand{by}\isamarkupfalse%
\ auto\isanewline
\isanewline
\ \ \ \ \isacommand{obtain}\isamarkupfalse%
\ ps\isactrlsub {\isadigit{2}}\ \isakeyword{where}\ {\isachardoublequoteopen}knights{\isacharunderscore}{\kern0pt}circuit\ {\isacharparenleft}{\kern0pt}board\ {\isadigit{8}}\ {\isacharparenleft}{\kern0pt}m{\isacharminus}{\kern0pt}{\isadigit{5}}{\isacharparenright}{\kern0pt}{\isacharparenright}{\kern0pt}\ ps\isactrlsub {\isadigit{2}}{\isachardoublequoteclose}\isanewline
\ \ \ \ \ \ \isacommand{using}\isamarkupfalse%
\ m{\isacharunderscore}{\kern0pt}ge\ knights{\isacharunderscore}{\kern0pt}circuit{\isacharunderscore}{\kern0pt}{\isadigit{8}}xm{\isacharunderscore}{\kern0pt}exists{\isacharbrackleft}{\kern0pt}of\ {\isachardoublequoteopen}{\isacharparenleft}{\kern0pt}m{\isacharminus}{\kern0pt}{\isadigit{5}}{\isacharparenright}{\kern0pt}{\isachardoublequoteclose}{\isacharbrackright}{\kern0pt}\ knights{\isacharunderscore}{\kern0pt}path{\isacharunderscore}{\kern0pt}non{\isacharunderscore}{\kern0pt}nil\ \isacommand{by}\isamarkupfalse%
\ auto\isanewline
\ \ \ \ \isacommand{then}\isamarkupfalse%
\ \isacommand{obtain}\isamarkupfalse%
\ ps\isactrlsub {\isadigit{2}}{\isacharprime}{\kern0pt}\ \isakeyword{where}\ ps\isactrlsub {\isadigit{2}}{\isacharprime}{\kern0pt}{\isacharunderscore}{\kern0pt}prems{\isacharprime}{\kern0pt}{\isacharcolon}{\kern0pt}\ {\isachardoublequoteopen}knights{\isacharunderscore}{\kern0pt}circuit\ {\isacharparenleft}{\kern0pt}board\ {\isadigit{8}}\ {\isacharparenleft}{\kern0pt}m{\isacharminus}{\kern0pt}{\isadigit{5}}{\isacharparenright}{\kern0pt}{\isacharparenright}{\kern0pt}\ ps\isactrlsub {\isadigit{2}}{\isacharprime}{\kern0pt}{\isachardoublequoteclose}\ \isanewline
\ \ \ \ \ \ \ \ {\isachardoublequoteopen}hd\ ps\isactrlsub {\isadigit{2}}{\isacharprime}{\kern0pt}\ {\isacharequal}{\kern0pt}\ {\isacharparenleft}{\kern0pt}{\isadigit{1}}{\isacharcomma}{\kern0pt}{\isadigit{1}}{\isacharparenright}{\kern0pt}{\isachardoublequoteclose}\ {\isachardoublequoteopen}last\ ps\isactrlsub {\isadigit{2}}{\isacharprime}{\kern0pt}\ {\isacharequal}{\kern0pt}\ {\isacharparenleft}{\kern0pt}{\isadigit{3}}{\isacharcomma}{\kern0pt}{\isadigit{2}}{\isacharparenright}{\kern0pt}{\isachardoublequoteclose}\isanewline
\ \ \ \ \ \ \isacommand{using}\isamarkupfalse%
\ b{\isacharunderscore}{\kern0pt}prems\ {\isacartoucheopen}{\isadigit{5}}\ {\isasymle}\ min\ {\isadigit{8}}\ {\isacharparenleft}{\kern0pt}m{\isacharminus}{\kern0pt}{\isadigit{5}}{\isacharparenright}{\kern0pt}{\isacartoucheclose}\ rotate{\isacharunderscore}{\kern0pt}knights{\isacharunderscore}{\kern0pt}circuit\ \isacommand{by}\isamarkupfalse%
\ blast\isanewline
\ \ \ \ \isacommand{then}\isamarkupfalse%
\ \isacommand{have}\isamarkupfalse%
\ ps\isactrlsub {\isadigit{2}}{\isacharprime}{\kern0pt}{\isacharunderscore}{\kern0pt}path{\isacharcolon}{\kern0pt}\ {\isachardoublequoteopen}knights{\isacharunderscore}{\kern0pt}path\ {\isacharparenleft}{\kern0pt}board\ {\isadigit{8}}\ {\isacharparenleft}{\kern0pt}m{\isacharminus}{\kern0pt}{\isadigit{5}}{\isacharparenright}{\kern0pt}{\isacharparenright}{\kern0pt}\ {\isacharparenleft}{\kern0pt}rev\ ps\isactrlsub {\isadigit{2}}{\isacharprime}{\kern0pt}{\isacharparenright}{\kern0pt}{\isachardoublequoteclose}\ \isanewline
\ \ \ \ \ \ {\isachardoublequoteopen}valid{\isacharunderscore}{\kern0pt}step\ {\isacharparenleft}{\kern0pt}last\ ps\isactrlsub {\isadigit{2}}{\isacharprime}{\kern0pt}{\isacharparenright}{\kern0pt}\ {\isacharparenleft}{\kern0pt}hd\ ps\isactrlsub {\isadigit{2}}{\isacharprime}{\kern0pt}{\isacharparenright}{\kern0pt}{\isachardoublequoteclose}\ {\isachardoublequoteopen}hd\ {\isacharparenleft}{\kern0pt}rev\ ps\isactrlsub {\isadigit{2}}{\isacharprime}{\kern0pt}{\isacharparenright}{\kern0pt}\ {\isacharequal}{\kern0pt}\ {\isacharparenleft}{\kern0pt}{\isadigit{3}}{\isacharcomma}{\kern0pt}{\isadigit{2}}{\isacharparenright}{\kern0pt}{\isachardoublequoteclose}\ {\isachardoublequoteopen}last\ {\isacharparenleft}{\kern0pt}rev\ ps\isactrlsub {\isadigit{2}}{\isacharprime}{\kern0pt}{\isacharparenright}{\kern0pt}\ {\isacharequal}{\kern0pt}\ {\isacharparenleft}{\kern0pt}{\isadigit{1}}{\isacharcomma}{\kern0pt}{\isadigit{1}}{\isacharparenright}{\kern0pt}{\isachardoublequoteclose}\isanewline
\ \ \ \ \ \ \isacommand{unfolding}\isamarkupfalse%
\ knights{\isacharunderscore}{\kern0pt}circuit{\isacharunderscore}{\kern0pt}def\ \isacommand{using}\isamarkupfalse%
\ knights{\isacharunderscore}{\kern0pt}path{\isacharunderscore}{\kern0pt}rev\ \isacommand{by}\isamarkupfalse%
\ {\isacharparenleft}{\kern0pt}auto\ simp{\isacharcolon}{\kern0pt}\ hd{\isacharunderscore}{\kern0pt}rev\ last{\isacharunderscore}{\kern0pt}rev{\isacharparenright}{\kern0pt}\isanewline
\isanewline
\ \ \ \ \isacommand{have}\isamarkupfalse%
\ {\isachardoublequoteopen}{\isadigit{3}}{\isadigit{4}}\ {\isacharless}{\kern0pt}\ length\ {\isacharquery}{\kern0pt}ps\isactrlsub {\isadigit{1}}{\isachardoublequoteclose}\ {\isachardoublequoteopen}last\ {\isacharparenleft}{\kern0pt}take\ {\isadigit{3}}{\isadigit{4}}\ {\isacharquery}{\kern0pt}ps\isactrlsub {\isadigit{1}}{\isacharparenright}{\kern0pt}\ {\isacharequal}{\kern0pt}\ {\isacharparenleft}{\kern0pt}{\isadigit{4}}{\isacharcomma}{\kern0pt}{\isadigit{5}}{\isacharparenright}{\kern0pt}{\isachardoublequoteclose}\ {\isachardoublequoteopen}hd\ {\isacharparenleft}{\kern0pt}drop\ {\isadigit{3}}{\isadigit{4}}\ {\isacharquery}{\kern0pt}ps\isactrlsub {\isadigit{1}}{\isacharparenright}{\kern0pt}\ {\isacharequal}{\kern0pt}\ {\isacharparenleft}{\kern0pt}{\isadigit{2}}{\isacharcomma}{\kern0pt}{\isadigit{4}}{\isacharparenright}{\kern0pt}{\isachardoublequoteclose}\ \isacommand{by}\isamarkupfalse%
\ eval{\isacharplus}{\kern0pt}\isanewline
\ \ \ \ \isacommand{then}\isamarkupfalse%
\ \isacommand{have}\isamarkupfalse%
\ {\isachardoublequoteopen}step{\isacharunderscore}{\kern0pt}in\ {\isacharquery}{\kern0pt}ps\isactrlsub {\isadigit{1}}\ {\isacharparenleft}{\kern0pt}{\isadigit{4}}{\isacharcomma}{\kern0pt}{\isadigit{5}}{\isacharparenright}{\kern0pt}\ {\isacharparenleft}{\kern0pt}{\isadigit{2}}{\isacharcomma}{\kern0pt}{\isadigit{4}}{\isacharparenright}{\kern0pt}{\isachardoublequoteclose}\isanewline
\ \ \ \ \ \ \isacommand{unfolding}\isamarkupfalse%
\ step{\isacharunderscore}{\kern0pt}in{\isacharunderscore}{\kern0pt}def\ \isacommand{using}\isamarkupfalse%
\ zero{\isacharunderscore}{\kern0pt}less{\isacharunderscore}{\kern0pt}numeral\ \isacommand{by}\isamarkupfalse%
\ blast\isanewline
\ \ \ \ \isacommand{then}\isamarkupfalse%
\ \isacommand{have}\isamarkupfalse%
\ {\isachardoublequoteopen}step{\isacharunderscore}{\kern0pt}in\ {\isacharquery}{\kern0pt}ps\isactrlsub {\isadigit{1}}\ {\isacharparenleft}{\kern0pt}{\isadigit{4}}{\isacharcomma}{\kern0pt}{\isadigit{5}}{\isacharparenright}{\kern0pt}\ {\isacharparenleft}{\kern0pt}{\isadigit{2}}{\isacharcomma}{\kern0pt}{\isadigit{4}}{\isacharparenright}{\kern0pt}{\isachardoublequoteclose}\ \isanewline
\ \ \ \ \ \ \ \ \ \ \ \ \ \ {\isachardoublequoteopen}valid{\isacharunderscore}{\kern0pt}step\ {\isacharparenleft}{\kern0pt}{\isadigit{4}}{\isacharcomma}{\kern0pt}{\isadigit{5}}{\isacharparenright}{\kern0pt}\ {\isacharparenleft}{\kern0pt}{\isadigit{3}}{\isacharcomma}{\kern0pt}int\ {\isadigit{5}}{\isacharplus}{\kern0pt}{\isadigit{2}}{\isacharparenright}{\kern0pt}{\isachardoublequoteclose}\ \isanewline
\ \ \ \ \ \ \ \ \ \ \ \ \ \ {\isachardoublequoteopen}valid{\isacharunderscore}{\kern0pt}step\ {\isacharparenleft}{\kern0pt}{\isadigit{1}}{\isacharcomma}{\kern0pt}int\ {\isadigit{5}}{\isacharplus}{\kern0pt}{\isadigit{1}}{\isacharparenright}{\kern0pt}\ {\isacharparenleft}{\kern0pt}{\isadigit{2}}{\isacharcomma}{\kern0pt}{\isadigit{4}}{\isacharparenright}{\kern0pt}{\isachardoublequoteclose}\isanewline
\ \ \ \ \ \ \isacommand{unfolding}\isamarkupfalse%
\ valid{\isacharunderscore}{\kern0pt}step{\isacharunderscore}{\kern0pt}def\ \isacommand{by}\isamarkupfalse%
\ auto\isanewline
\ \ \ \ \isacommand{then}\isamarkupfalse%
\ \isacommand{have}\isamarkupfalse%
\ {\isachardoublequoteopen}{\isasymexists}ps{\isachardot}{\kern0pt}\ knights{\isacharunderscore}{\kern0pt}path\ {\isacharparenleft}{\kern0pt}board\ {\isadigit{8}}\ m{\isacharparenright}{\kern0pt}\ ps\ {\isasymand}\ hd\ ps\ {\isacharequal}{\kern0pt}\ hd\ {\isacharquery}{\kern0pt}ps\isactrlsub {\isadigit{1}}\ {\isasymand}\ last\ ps\ {\isacharequal}{\kern0pt}\ last\ {\isacharquery}{\kern0pt}ps\isactrlsub {\isadigit{1}}{\isachardoublequoteclose}\ \ \ \ \ \ \ \ \ \ \ \ \ \ \isanewline
\ \ \ \ \ \ \isacommand{using}\isamarkupfalse%
\ m{\isacharunderscore}{\kern0pt}ge\ ps\isactrlsub {\isadigit{1}}{\isacharunderscore}{\kern0pt}prems\ ps\isactrlsub {\isadigit{2}}{\isacharprime}{\kern0pt}{\isacharunderscore}{\kern0pt}prems{\isacharprime}{\kern0pt}\ ps\isactrlsub {\isadigit{2}}{\isacharprime}{\kern0pt}{\isacharunderscore}{\kern0pt}path\ \isanewline
\ \ \ \ \ \ \ \ \ \ \ \ knights{\isacharunderscore}{\kern0pt}path{\isacharunderscore}{\kern0pt}split{\isacharunderscore}{\kern0pt}concat{\isacharbrackleft}{\kern0pt}of\ {\isadigit{8}}\ {\isadigit{5}}\ {\isacharquery}{\kern0pt}ps\isactrlsub {\isadigit{1}}\ {\isachardoublequoteopen}m{\isacharminus}{\kern0pt}{\isadigit{5}}{\isachardoublequoteclose}\ {\isachardoublequoteopen}rev\ ps\isactrlsub {\isadigit{2}}{\isacharprime}{\kern0pt}{\isachardoublequoteclose}{\isacharbrackright}{\kern0pt}\ \isacommand{by}\isamarkupfalse%
\ auto\isanewline
\ \ \ \ \isacommand{then}\isamarkupfalse%
\ \isacommand{show}\isamarkupfalse%
\ {\isacharquery}{\kern0pt}thesis\ \isacommand{using}\isamarkupfalse%
\ ps\isactrlsub {\isadigit{1}}{\isacharunderscore}{\kern0pt}prems\ \isacommand{by}\isamarkupfalse%
\ auto\isanewline
\ \ \isacommand{qed}\isamarkupfalse%
\isanewline
\isacommand{qed}\isamarkupfalse%
%
\endisatagproof
{\isafoldproof}%
%
\isadelimproof
%
\endisadelimproof
%
\begin{isamarkuptext}%
\isa{{\isadigit{5}}\ {\isasymle}\ {\isacharquery}{\kern0pt}m\ {\isasymLongrightarrow}\ {\isasymexists}ps{\isachardot}{\kern0pt}\ knights{\isacharunderscore}{\kern0pt}circuit\ {\isacharparenleft}{\kern0pt}board\ {\isadigit{8}}\ {\isacharquery}{\kern0pt}m{\isacharparenright}{\kern0pt}\ ps\ {\isasymand}\ step{\isacharunderscore}{\kern0pt}in\ ps\ {\isacharparenleft}{\kern0pt}{\isadigit{2}}{\isacharcomma}{\kern0pt}\ int\ {\isacharquery}{\kern0pt}m\ {\isacharminus}{\kern0pt}\ {\isadigit{1}}{\isacharparenright}{\kern0pt}\ {\isacharparenleft}{\kern0pt}{\isadigit{4}}{\isacharcomma}{\kern0pt}\ int\ {\isacharquery}{\kern0pt}m{\isacharparenright}{\kern0pt}} and \isa{{\isadigit{5}}\ {\isasymle}\ {\isacharquery}{\kern0pt}m\ {\isasymLongrightarrow}\ {\isasymexists}ps{\isachardot}{\kern0pt}\ knights{\isacharunderscore}{\kern0pt}path\ {\isacharparenleft}{\kern0pt}board\ {\isadigit{8}}\ {\isacharquery}{\kern0pt}m{\isacharparenright}{\kern0pt}\ ps\ {\isasymand}\ hd\ ps\ {\isacharequal}{\kern0pt}\ {\isacharparenleft}{\kern0pt}{\isadigit{1}}{\isacharcomma}{\kern0pt}\ {\isadigit{1}}{\isacharparenright}{\kern0pt}\ {\isasymand}\ last\ ps\ {\isacharequal}{\kern0pt}\ {\isacharparenleft}{\kern0pt}{\isadigit{7}}{\isacharcomma}{\kern0pt}\ {\isadigit{2}}{\isacharparenright}{\kern0pt}} formalize Lemma 3 
from \cite{cull_decurtins_1987}.%
\end{isamarkuptext}\isamarkuptrue%
\isacommand{lemmas}\isamarkupfalse%
\ knights{\isacharunderscore}{\kern0pt}path{\isacharunderscore}{\kern0pt}{\isadigit{8}}xm{\isacharunderscore}{\kern0pt}exists\ {\isacharequal}{\kern0pt}\ knights{\isacharunderscore}{\kern0pt}circuit{\isacharunderscore}{\kern0pt}{\isadigit{8}}xm{\isacharunderscore}{\kern0pt}exists\ knights{\isacharunderscore}{\kern0pt}path{\isacharunderscore}{\kern0pt}{\isadigit{8}}xm{\isacharunderscore}{\kern0pt}ul{\isacharunderscore}{\kern0pt}exists%
\isadelimdocument
%
\endisadelimdocument
%
\isatagdocument
%
\isamarkupsection{Knight's Paths and Circuits for \isa{n{\isasymtimes}m}-Boards%
}
\isamarkuptrue%
%
\endisatagdocument
{\isafolddocument}%
%
\isadelimdocument
%
\endisadelimdocument
%
\begin{isamarkuptext}%
In this section the desired theorems are proved. The proof uses the previous lemmas to 
construct paths and circuits for arbitrary \isa{n{\isasymtimes}m}-boards.%
\end{isamarkuptext}\isamarkuptrue%
%
\begin{isamarkuptext}%
A Knight's path for the \isa{{\isacharparenleft}{\kern0pt}{\isadigit{5}}{\isasymtimes}{\isadigit{5}}{\isacharparenright}{\kern0pt}}-board that starts in the lower-left and ends in the 
upper-left.
  \begin{table}[H]
    \begin{tabular}{lllll}
       7 & 20 &  9 & 14 &  5 \\
      10 & 25 &  6 & 21 & 16 \\
      19 &  8 & 15 &  4 & 13 \\
      24 & 11 &  2 & 17 & 22 \\
       1 & 18 & 23 & 12 &  3
    \end{tabular}
  \end{table}%
\end{isamarkuptext}\isamarkuptrue%
\isacommand{abbreviation}\isamarkupfalse%
\ {\isachardoublequoteopen}kp{\isadigit{5}}x{\isadigit{5}}ul\ {\isasymequiv}\ the\ {\isacharparenleft}{\kern0pt}to{\isacharunderscore}{\kern0pt}path\ \isanewline
\ \ {\isacharbrackleft}{\kern0pt}{\isacharbrackleft}{\kern0pt}{\isadigit{7}}{\isacharcomma}{\kern0pt}{\isadigit{2}}{\isadigit{0}}{\isacharcomma}{\kern0pt}{\isadigit{9}}{\isacharcomma}{\kern0pt}{\isadigit{1}}{\isadigit{4}}{\isacharcomma}{\kern0pt}{\isadigit{5}}{\isacharbrackright}{\kern0pt}{\isacharcomma}{\kern0pt}\isanewline
\ \ {\isacharbrackleft}{\kern0pt}{\isadigit{1}}{\isadigit{0}}{\isacharcomma}{\kern0pt}{\isadigit{2}}{\isadigit{5}}{\isacharcomma}{\kern0pt}{\isadigit{6}}{\isacharcomma}{\kern0pt}{\isadigit{2}}{\isadigit{1}}{\isacharcomma}{\kern0pt}{\isadigit{1}}{\isadigit{6}}{\isacharbrackright}{\kern0pt}{\isacharcomma}{\kern0pt}\isanewline
\ \ {\isacharbrackleft}{\kern0pt}{\isadigit{1}}{\isadigit{9}}{\isacharcomma}{\kern0pt}{\isadigit{8}}{\isacharcomma}{\kern0pt}{\isadigit{1}}{\isadigit{5}}{\isacharcomma}{\kern0pt}{\isadigit{4}}{\isacharcomma}{\kern0pt}{\isadigit{1}}{\isadigit{3}}{\isacharbrackright}{\kern0pt}{\isacharcomma}{\kern0pt}\isanewline
\ \ {\isacharbrackleft}{\kern0pt}{\isadigit{2}}{\isadigit{4}}{\isacharcomma}{\kern0pt}{\isadigit{1}}{\isadigit{1}}{\isacharcomma}{\kern0pt}{\isadigit{2}}{\isacharcomma}{\kern0pt}{\isadigit{1}}{\isadigit{7}}{\isacharcomma}{\kern0pt}{\isadigit{2}}{\isadigit{2}}{\isacharbrackright}{\kern0pt}{\isacharcomma}{\kern0pt}\isanewline
\ \ {\isacharbrackleft}{\kern0pt}{\isadigit{1}}{\isacharcomma}{\kern0pt}{\isadigit{1}}{\isadigit{8}}{\isacharcomma}{\kern0pt}{\isadigit{2}}{\isadigit{3}}{\isacharcomma}{\kern0pt}{\isadigit{1}}{\isadigit{2}}{\isacharcomma}{\kern0pt}{\isadigit{3}}{\isacharbrackright}{\kern0pt}{\isacharbrackright}{\kern0pt}{\isacharparenright}{\kern0pt}{\isachardoublequoteclose}\isanewline
\isacommand{lemma}\isamarkupfalse%
\ kp{\isacharunderscore}{\kern0pt}{\isadigit{5}}x{\isadigit{5}}{\isacharunderscore}{\kern0pt}ul{\isacharcolon}{\kern0pt}\ {\isachardoublequoteopen}knights{\isacharunderscore}{\kern0pt}path\ b{\isadigit{5}}x{\isadigit{5}}\ kp{\isadigit{5}}x{\isadigit{5}}ul{\isachardoublequoteclose}\isanewline
%
\isadelimproof
\ \ %
\endisadelimproof
%
\isatagproof
\isacommand{by}\isamarkupfalse%
\ {\isacharparenleft}{\kern0pt}simp\ only{\isacharcolon}{\kern0pt}\ knights{\isacharunderscore}{\kern0pt}path{\isacharunderscore}{\kern0pt}exec{\isacharunderscore}{\kern0pt}simp{\isacharparenright}{\kern0pt}\ eval%
\endisatagproof
{\isafoldproof}%
%
\isadelimproof
%
\endisadelimproof
%
\begin{isamarkuptext}%
A Knight's path for the \isa{{\isacharparenleft}{\kern0pt}{\isadigit{5}}{\isasymtimes}{\isadigit{7}}{\isacharparenright}{\kern0pt}}-board that starts in the lower-left and ends in the 
upper-left.
  \begin{table}[H]
    \begin{tabular}{lllllll}
      17 & 14 & 25 &  6 & 19 &  8 & 29 \\
      26 & 35 & 18 & 15 & 28 &  5 & 20 \\
      13 & 16 & 27 & 24 &  7 & 30 &  9 \\
      34 & 23 &  2 & 11 & 32 & 21 &  4 \\
       1 & 12 & 33 & 22 &  3 & 10 & 31
    \end{tabular}
  \end{table}%
\end{isamarkuptext}\isamarkuptrue%
\isacommand{abbreviation}\isamarkupfalse%
\ {\isachardoublequoteopen}kp{\isadigit{5}}x{\isadigit{7}}ul\ {\isasymequiv}\ the\ {\isacharparenleft}{\kern0pt}to{\isacharunderscore}{\kern0pt}path\ \isanewline
\ \ {\isacharbrackleft}{\kern0pt}{\isacharbrackleft}{\kern0pt}{\isadigit{1}}{\isadigit{7}}{\isacharcomma}{\kern0pt}{\isadigit{1}}{\isadigit{4}}{\isacharcomma}{\kern0pt}{\isadigit{2}}{\isadigit{5}}{\isacharcomma}{\kern0pt}{\isadigit{6}}{\isacharcomma}{\kern0pt}{\isadigit{1}}{\isadigit{9}}{\isacharcomma}{\kern0pt}{\isadigit{8}}{\isacharcomma}{\kern0pt}{\isadigit{2}}{\isadigit{9}}{\isacharbrackright}{\kern0pt}{\isacharcomma}{\kern0pt}\isanewline
\ \ {\isacharbrackleft}{\kern0pt}{\isadigit{2}}{\isadigit{6}}{\isacharcomma}{\kern0pt}{\isadigit{3}}{\isadigit{5}}{\isacharcomma}{\kern0pt}{\isadigit{1}}{\isadigit{8}}{\isacharcomma}{\kern0pt}{\isadigit{1}}{\isadigit{5}}{\isacharcomma}{\kern0pt}{\isadigit{2}}{\isadigit{8}}{\isacharcomma}{\kern0pt}{\isadigit{5}}{\isacharcomma}{\kern0pt}{\isadigit{2}}{\isadigit{0}}{\isacharbrackright}{\kern0pt}{\isacharcomma}{\kern0pt}\isanewline
\ \ {\isacharbrackleft}{\kern0pt}{\isadigit{1}}{\isadigit{3}}{\isacharcomma}{\kern0pt}{\isadigit{1}}{\isadigit{6}}{\isacharcomma}{\kern0pt}{\isadigit{2}}{\isadigit{7}}{\isacharcomma}{\kern0pt}{\isadigit{2}}{\isadigit{4}}{\isacharcomma}{\kern0pt}{\isadigit{7}}{\isacharcomma}{\kern0pt}{\isadigit{3}}{\isadigit{0}}{\isacharcomma}{\kern0pt}{\isadigit{9}}{\isacharbrackright}{\kern0pt}{\isacharcomma}{\kern0pt}\isanewline
\ \ {\isacharbrackleft}{\kern0pt}{\isadigit{3}}{\isadigit{4}}{\isacharcomma}{\kern0pt}{\isadigit{2}}{\isadigit{3}}{\isacharcomma}{\kern0pt}{\isadigit{2}}{\isacharcomma}{\kern0pt}{\isadigit{1}}{\isadigit{1}}{\isacharcomma}{\kern0pt}{\isadigit{3}}{\isadigit{2}}{\isacharcomma}{\kern0pt}{\isadigit{2}}{\isadigit{1}}{\isacharcomma}{\kern0pt}{\isadigit{4}}{\isacharbrackright}{\kern0pt}{\isacharcomma}{\kern0pt}\isanewline
\ \ {\isacharbrackleft}{\kern0pt}{\isadigit{1}}{\isacharcomma}{\kern0pt}{\isadigit{1}}{\isadigit{2}}{\isacharcomma}{\kern0pt}{\isadigit{3}}{\isadigit{3}}{\isacharcomma}{\kern0pt}{\isadigit{2}}{\isadigit{2}}{\isacharcomma}{\kern0pt}{\isadigit{3}}{\isacharcomma}{\kern0pt}{\isadigit{1}}{\isadigit{0}}{\isacharcomma}{\kern0pt}{\isadigit{3}}{\isadigit{1}}{\isacharbrackright}{\kern0pt}{\isacharbrackright}{\kern0pt}{\isacharparenright}{\kern0pt}{\isachardoublequoteclose}\isanewline
\isacommand{lemma}\isamarkupfalse%
\ kp{\isacharunderscore}{\kern0pt}{\isadigit{5}}x{\isadigit{7}}{\isacharunderscore}{\kern0pt}ul{\isacharcolon}{\kern0pt}\ {\isachardoublequoteopen}knights{\isacharunderscore}{\kern0pt}path\ b{\isadigit{5}}x{\isadigit{7}}\ kp{\isadigit{5}}x{\isadigit{7}}ul{\isachardoublequoteclose}\isanewline
%
\isadelimproof
\ \ %
\endisadelimproof
%
\isatagproof
\isacommand{by}\isamarkupfalse%
\ {\isacharparenleft}{\kern0pt}simp\ only{\isacharcolon}{\kern0pt}\ knights{\isacharunderscore}{\kern0pt}path{\isacharunderscore}{\kern0pt}exec{\isacharunderscore}{\kern0pt}simp{\isacharparenright}{\kern0pt}\ eval%
\endisatagproof
{\isafoldproof}%
%
\isadelimproof
%
\endisadelimproof
%
\begin{isamarkuptext}%
A Knight's path for the \isa{{\isacharparenleft}{\kern0pt}{\isadigit{5}}{\isasymtimes}{\isadigit{9}}{\isacharparenright}{\kern0pt}}-board that starts in the lower-left and ends in the 
upper-left.
  \begin{table}[H]
    \begin{tabular}{lllllllll}
       7 & 12 & 37 & 42 &  5 & 18 & 23 & 32 & 27 \\
      38 & 45 &  6 & 11 & 36 & 31 & 26 & 19 & 24 \\
      13 &  8 & 43 &  4 & 41 & 22 & 17 & 28 & 33 \\
      44 & 39 &  2 & 15 & 10 & 35 & 30 & 25 & 20 \\
       1 & 14 &  9 & 40 &  3 & 16 & 21 & 34 & 29
    \end{tabular}
  \end{table}%
\end{isamarkuptext}\isamarkuptrue%
\isacommand{abbreviation}\isamarkupfalse%
\ {\isachardoublequoteopen}kp{\isadigit{5}}x{\isadigit{9}}ul\ {\isasymequiv}\ the\ {\isacharparenleft}{\kern0pt}to{\isacharunderscore}{\kern0pt}path\ \isanewline
\ \ {\isacharbrackleft}{\kern0pt}{\isacharbrackleft}{\kern0pt}{\isadigit{7}}{\isacharcomma}{\kern0pt}{\isadigit{1}}{\isadigit{2}}{\isacharcomma}{\kern0pt}{\isadigit{3}}{\isadigit{7}}{\isacharcomma}{\kern0pt}{\isadigit{4}}{\isadigit{2}}{\isacharcomma}{\kern0pt}{\isadigit{5}}{\isacharcomma}{\kern0pt}{\isadigit{1}}{\isadigit{8}}{\isacharcomma}{\kern0pt}{\isadigit{2}}{\isadigit{3}}{\isacharcomma}{\kern0pt}{\isadigit{3}}{\isadigit{2}}{\isacharcomma}{\kern0pt}{\isadigit{2}}{\isadigit{7}}{\isacharbrackright}{\kern0pt}{\isacharcomma}{\kern0pt}\isanewline
\ \ {\isacharbrackleft}{\kern0pt}{\isadigit{3}}{\isadigit{8}}{\isacharcomma}{\kern0pt}{\isadigit{4}}{\isadigit{5}}{\isacharcomma}{\kern0pt}{\isadigit{6}}{\isacharcomma}{\kern0pt}{\isadigit{1}}{\isadigit{1}}{\isacharcomma}{\kern0pt}{\isadigit{3}}{\isadigit{6}}{\isacharcomma}{\kern0pt}{\isadigit{3}}{\isadigit{1}}{\isacharcomma}{\kern0pt}{\isadigit{2}}{\isadigit{6}}{\isacharcomma}{\kern0pt}{\isadigit{1}}{\isadigit{9}}{\isacharcomma}{\kern0pt}{\isadigit{2}}{\isadigit{4}}{\isacharbrackright}{\kern0pt}{\isacharcomma}{\kern0pt}\isanewline
\ \ {\isacharbrackleft}{\kern0pt}{\isadigit{1}}{\isadigit{3}}{\isacharcomma}{\kern0pt}{\isadigit{8}}{\isacharcomma}{\kern0pt}{\isadigit{4}}{\isadigit{3}}{\isacharcomma}{\kern0pt}{\isadigit{4}}{\isacharcomma}{\kern0pt}{\isadigit{4}}{\isadigit{1}}{\isacharcomma}{\kern0pt}{\isadigit{2}}{\isadigit{2}}{\isacharcomma}{\kern0pt}{\isadigit{1}}{\isadigit{7}}{\isacharcomma}{\kern0pt}{\isadigit{2}}{\isadigit{8}}{\isacharcomma}{\kern0pt}{\isadigit{3}}{\isadigit{3}}{\isacharbrackright}{\kern0pt}{\isacharcomma}{\kern0pt}\isanewline
\ \ {\isacharbrackleft}{\kern0pt}{\isadigit{4}}{\isadigit{4}}{\isacharcomma}{\kern0pt}{\isadigit{3}}{\isadigit{9}}{\isacharcomma}{\kern0pt}{\isadigit{2}}{\isacharcomma}{\kern0pt}{\isadigit{1}}{\isadigit{5}}{\isacharcomma}{\kern0pt}{\isadigit{1}}{\isadigit{0}}{\isacharcomma}{\kern0pt}{\isadigit{3}}{\isadigit{5}}{\isacharcomma}{\kern0pt}{\isadigit{3}}{\isadigit{0}}{\isacharcomma}{\kern0pt}{\isadigit{2}}{\isadigit{5}}{\isacharcomma}{\kern0pt}{\isadigit{2}}{\isadigit{0}}{\isacharbrackright}{\kern0pt}{\isacharcomma}{\kern0pt}\isanewline
\ \ {\isacharbrackleft}{\kern0pt}{\isadigit{1}}{\isacharcomma}{\kern0pt}{\isadigit{1}}{\isadigit{4}}{\isacharcomma}{\kern0pt}{\isadigit{9}}{\isacharcomma}{\kern0pt}{\isadigit{4}}{\isadigit{0}}{\isacharcomma}{\kern0pt}{\isadigit{3}}{\isacharcomma}{\kern0pt}{\isadigit{1}}{\isadigit{6}}{\isacharcomma}{\kern0pt}{\isadigit{2}}{\isadigit{1}}{\isacharcomma}{\kern0pt}{\isadigit{3}}{\isadigit{4}}{\isacharcomma}{\kern0pt}{\isadigit{2}}{\isadigit{9}}{\isacharbrackright}{\kern0pt}{\isacharbrackright}{\kern0pt}{\isacharparenright}{\kern0pt}{\isachardoublequoteclose}\isanewline
\isacommand{lemma}\isamarkupfalse%
\ kp{\isacharunderscore}{\kern0pt}{\isadigit{5}}x{\isadigit{9}}{\isacharunderscore}{\kern0pt}ul{\isacharcolon}{\kern0pt}\ {\isachardoublequoteopen}knights{\isacharunderscore}{\kern0pt}path\ b{\isadigit{5}}x{\isadigit{9}}\ kp{\isadigit{5}}x{\isadigit{9}}ul{\isachardoublequoteclose}\isanewline
%
\isadelimproof
\ \ %
\endisadelimproof
%
\isatagproof
\isacommand{by}\isamarkupfalse%
\ {\isacharparenleft}{\kern0pt}simp\ only{\isacharcolon}{\kern0pt}\ knights{\isacharunderscore}{\kern0pt}path{\isacharunderscore}{\kern0pt}exec{\isacharunderscore}{\kern0pt}simp{\isacharparenright}{\kern0pt}\ eval%
\endisatagproof
{\isafoldproof}%
%
\isadelimproof
\isanewline
%
\endisadelimproof
\isanewline
\isacommand{abbreviation}\isamarkupfalse%
\ {\isachardoublequoteopen}b{\isadigit{7}}x{\isadigit{7}}\ {\isasymequiv}\ board\ {\isadigit{7}}\ {\isadigit{7}}{\isachardoublequoteclose}%
\begin{isamarkuptext}%
A Knight's path for the \isa{{\isacharparenleft}{\kern0pt}{\isadigit{7}}{\isasymtimes}{\isadigit{7}}{\isacharparenright}{\kern0pt}}-board that starts in the lower-left and ends in the 
upper-left.
  \begin{table}[H]
    \begin{tabular}{lllllll}
       9 & 30 & 19 & 42 &  7 & 32 & 17 \\
      20 & 49 &  8 & 31 & 18 & 43 &  6 \\
      29 & 10 & 41 & 36 & 39 & 16 & 33 \\
      48 & 21 & 38 & 27 & 34 &  5 & 44 \\
      11 & 28 & 35 & 40 & 37 & 26 & 15 \\
      22 & 47 &  2 & 13 & 24 & 45 &  4 \\
       1 & 12 & 23 & 46 &  3 & 14 & 25
    \end{tabular}
  \end{table}%
\end{isamarkuptext}\isamarkuptrue%
\isacommand{abbreviation}\isamarkupfalse%
\ {\isachardoublequoteopen}kp{\isadigit{7}}x{\isadigit{7}}ul\ {\isasymequiv}\ the\ {\isacharparenleft}{\kern0pt}to{\isacharunderscore}{\kern0pt}path\ \isanewline
\ \ {\isacharbrackleft}{\kern0pt}{\isacharbrackleft}{\kern0pt}{\isadigit{9}}{\isacharcomma}{\kern0pt}{\isadigit{3}}{\isadigit{0}}{\isacharcomma}{\kern0pt}{\isadigit{1}}{\isadigit{9}}{\isacharcomma}{\kern0pt}{\isadigit{4}}{\isadigit{2}}{\isacharcomma}{\kern0pt}{\isadigit{7}}{\isacharcomma}{\kern0pt}{\isadigit{3}}{\isadigit{2}}{\isacharcomma}{\kern0pt}{\isadigit{1}}{\isadigit{7}}{\isacharbrackright}{\kern0pt}{\isacharcomma}{\kern0pt}\isanewline
\ \ {\isacharbrackleft}{\kern0pt}{\isadigit{2}}{\isadigit{0}}{\isacharcomma}{\kern0pt}{\isadigit{4}}{\isadigit{9}}{\isacharcomma}{\kern0pt}{\isadigit{8}}{\isacharcomma}{\kern0pt}{\isadigit{3}}{\isadigit{1}}{\isacharcomma}{\kern0pt}{\isadigit{1}}{\isadigit{8}}{\isacharcomma}{\kern0pt}{\isadigit{4}}{\isadigit{3}}{\isacharcomma}{\kern0pt}{\isadigit{6}}{\isacharbrackright}{\kern0pt}{\isacharcomma}{\kern0pt}\isanewline
\ \ {\isacharbrackleft}{\kern0pt}{\isadigit{2}}{\isadigit{9}}{\isacharcomma}{\kern0pt}{\isadigit{1}}{\isadigit{0}}{\isacharcomma}{\kern0pt}{\isadigit{4}}{\isadigit{1}}{\isacharcomma}{\kern0pt}{\isadigit{3}}{\isadigit{6}}{\isacharcomma}{\kern0pt}{\isadigit{3}}{\isadigit{9}}{\isacharcomma}{\kern0pt}{\isadigit{1}}{\isadigit{6}}{\isacharcomma}{\kern0pt}{\isadigit{3}}{\isadigit{3}}{\isacharbrackright}{\kern0pt}{\isacharcomma}{\kern0pt}\isanewline
\ \ {\isacharbrackleft}{\kern0pt}{\isadigit{4}}{\isadigit{8}}{\isacharcomma}{\kern0pt}{\isadigit{2}}{\isadigit{1}}{\isacharcomma}{\kern0pt}{\isadigit{3}}{\isadigit{8}}{\isacharcomma}{\kern0pt}{\isadigit{2}}{\isadigit{7}}{\isacharcomma}{\kern0pt}{\isadigit{3}}{\isadigit{4}}{\isacharcomma}{\kern0pt}{\isadigit{5}}{\isacharcomma}{\kern0pt}{\isadigit{4}}{\isadigit{4}}{\isacharbrackright}{\kern0pt}{\isacharcomma}{\kern0pt}\isanewline
\ \ {\isacharbrackleft}{\kern0pt}{\isadigit{1}}{\isadigit{1}}{\isacharcomma}{\kern0pt}{\isadigit{2}}{\isadigit{8}}{\isacharcomma}{\kern0pt}{\isadigit{3}}{\isadigit{5}}{\isacharcomma}{\kern0pt}{\isadigit{4}}{\isadigit{0}}{\isacharcomma}{\kern0pt}{\isadigit{3}}{\isadigit{7}}{\isacharcomma}{\kern0pt}{\isadigit{2}}{\isadigit{6}}{\isacharcomma}{\kern0pt}{\isadigit{1}}{\isadigit{5}}{\isacharbrackright}{\kern0pt}{\isacharcomma}{\kern0pt}\isanewline
\ \ {\isacharbrackleft}{\kern0pt}{\isadigit{2}}{\isadigit{2}}{\isacharcomma}{\kern0pt}{\isadigit{4}}{\isadigit{7}}{\isacharcomma}{\kern0pt}{\isadigit{2}}{\isacharcomma}{\kern0pt}{\isadigit{1}}{\isadigit{3}}{\isacharcomma}{\kern0pt}{\isadigit{2}}{\isadigit{4}}{\isacharcomma}{\kern0pt}{\isadigit{4}}{\isadigit{5}}{\isacharcomma}{\kern0pt}{\isadigit{4}}{\isacharbrackright}{\kern0pt}{\isacharcomma}{\kern0pt}\isanewline
\ \ {\isacharbrackleft}{\kern0pt}{\isadigit{1}}{\isacharcomma}{\kern0pt}{\isadigit{1}}{\isadigit{2}}{\isacharcomma}{\kern0pt}{\isadigit{2}}{\isadigit{3}}{\isacharcomma}{\kern0pt}{\isadigit{4}}{\isadigit{6}}{\isacharcomma}{\kern0pt}{\isadigit{3}}{\isacharcomma}{\kern0pt}{\isadigit{1}}{\isadigit{4}}{\isacharcomma}{\kern0pt}{\isadigit{2}}{\isadigit{5}}{\isacharbrackright}{\kern0pt}{\isacharbrackright}{\kern0pt}{\isacharparenright}{\kern0pt}{\isachardoublequoteclose}\isanewline
\isacommand{lemma}\isamarkupfalse%
\ kp{\isacharunderscore}{\kern0pt}{\isadigit{7}}x{\isadigit{7}}{\isacharunderscore}{\kern0pt}ul{\isacharcolon}{\kern0pt}\ {\isachardoublequoteopen}knights{\isacharunderscore}{\kern0pt}path\ b{\isadigit{7}}x{\isadigit{7}}\ kp{\isadigit{7}}x{\isadigit{7}}ul{\isachardoublequoteclose}\isanewline
%
\isadelimproof
\ \ %
\endisadelimproof
%
\isatagproof
\isacommand{by}\isamarkupfalse%
\ {\isacharparenleft}{\kern0pt}simp\ only{\isacharcolon}{\kern0pt}\ knights{\isacharunderscore}{\kern0pt}path{\isacharunderscore}{\kern0pt}exec{\isacharunderscore}{\kern0pt}simp{\isacharparenright}{\kern0pt}\ eval%
\endisatagproof
{\isafoldproof}%
%
\isadelimproof
\isanewline
%
\endisadelimproof
\isanewline
\isacommand{abbreviation}\isamarkupfalse%
\ {\isachardoublequoteopen}b{\isadigit{7}}x{\isadigit{9}}\ {\isasymequiv}\ board\ {\isadigit{7}}\ {\isadigit{9}}{\isachardoublequoteclose}%
\begin{isamarkuptext}%
A Knight's path for the \isa{{\isacharparenleft}{\kern0pt}{\isadigit{7}}{\isasymtimes}{\isadigit{9}}{\isacharparenright}{\kern0pt}}-board that starts in the lower-left and ends in the 
upper-left.
  \begin{table}[H]
    \begin{tabular}{lllllllll}
      59 &  4 & 17 & 50 & 37 &  6 & 19 & 30 & 39 \\
      16 & 63 & 58 &  5 & 18 & 51 & 38 &  7 & 20 \\
       3 & 60 & 49 & 36 & 57 & 42 & 29 & 40 & 31 \\
      48 & 15 & 62 & 43 & 52 & 35 & 56 & 21 &  8 \\
      61 &  2 & 13 & 26 & 45 & 28 & 41 & 32 & 55 \\
      14 & 47 & 44 & 11 & 24 & 53 & 34 &  9 & 22 \\
       1 & 12 & 25 & 46 & 27 & 10 & 23 & 54 & 33
    \end{tabular}
  \end{table}%
\end{isamarkuptext}\isamarkuptrue%
\isacommand{abbreviation}\isamarkupfalse%
\ {\isachardoublequoteopen}kp{\isadigit{7}}x{\isadigit{9}}ul\ {\isasymequiv}\ the\ {\isacharparenleft}{\kern0pt}to{\isacharunderscore}{\kern0pt}path\ \isanewline
\ \ {\isacharbrackleft}{\kern0pt}{\isacharbrackleft}{\kern0pt}{\isadigit{5}}{\isadigit{9}}{\isacharcomma}{\kern0pt}{\isadigit{4}}{\isacharcomma}{\kern0pt}{\isadigit{1}}{\isadigit{7}}{\isacharcomma}{\kern0pt}{\isadigit{5}}{\isadigit{0}}{\isacharcomma}{\kern0pt}{\isadigit{3}}{\isadigit{7}}{\isacharcomma}{\kern0pt}{\isadigit{6}}{\isacharcomma}{\kern0pt}{\isadigit{1}}{\isadigit{9}}{\isacharcomma}{\kern0pt}{\isadigit{3}}{\isadigit{0}}{\isacharcomma}{\kern0pt}{\isadigit{3}}{\isadigit{9}}{\isacharbrackright}{\kern0pt}{\isacharcomma}{\kern0pt}\isanewline
\ \ {\isacharbrackleft}{\kern0pt}{\isadigit{1}}{\isadigit{6}}{\isacharcomma}{\kern0pt}{\isadigit{6}}{\isadigit{3}}{\isacharcomma}{\kern0pt}{\isadigit{5}}{\isadigit{8}}{\isacharcomma}{\kern0pt}{\isadigit{5}}{\isacharcomma}{\kern0pt}{\isadigit{1}}{\isadigit{8}}{\isacharcomma}{\kern0pt}{\isadigit{5}}{\isadigit{1}}{\isacharcomma}{\kern0pt}{\isadigit{3}}{\isadigit{8}}{\isacharcomma}{\kern0pt}{\isadigit{7}}{\isacharcomma}{\kern0pt}{\isadigit{2}}{\isadigit{0}}{\isacharbrackright}{\kern0pt}{\isacharcomma}{\kern0pt}\isanewline
\ \ {\isacharbrackleft}{\kern0pt}{\isadigit{3}}{\isacharcomma}{\kern0pt}{\isadigit{6}}{\isadigit{0}}{\isacharcomma}{\kern0pt}{\isadigit{4}}{\isadigit{9}}{\isacharcomma}{\kern0pt}{\isadigit{3}}{\isadigit{6}}{\isacharcomma}{\kern0pt}{\isadigit{5}}{\isadigit{7}}{\isacharcomma}{\kern0pt}{\isadigit{4}}{\isadigit{2}}{\isacharcomma}{\kern0pt}{\isadigit{2}}{\isadigit{9}}{\isacharcomma}{\kern0pt}{\isadigit{4}}{\isadigit{0}}{\isacharcomma}{\kern0pt}{\isadigit{3}}{\isadigit{1}}{\isacharbrackright}{\kern0pt}{\isacharcomma}{\kern0pt}\isanewline
\ \ {\isacharbrackleft}{\kern0pt}{\isadigit{4}}{\isadigit{8}}{\isacharcomma}{\kern0pt}{\isadigit{1}}{\isadigit{5}}{\isacharcomma}{\kern0pt}{\isadigit{6}}{\isadigit{2}}{\isacharcomma}{\kern0pt}{\isadigit{4}}{\isadigit{3}}{\isacharcomma}{\kern0pt}{\isadigit{5}}{\isadigit{2}}{\isacharcomma}{\kern0pt}{\isadigit{3}}{\isadigit{5}}{\isacharcomma}{\kern0pt}{\isadigit{5}}{\isadigit{6}}{\isacharcomma}{\kern0pt}{\isadigit{2}}{\isadigit{1}}{\isacharcomma}{\kern0pt}{\isadigit{8}}{\isacharbrackright}{\kern0pt}{\isacharcomma}{\kern0pt}\isanewline
\ \ {\isacharbrackleft}{\kern0pt}{\isadigit{6}}{\isadigit{1}}{\isacharcomma}{\kern0pt}{\isadigit{2}}{\isacharcomma}{\kern0pt}{\isadigit{1}}{\isadigit{3}}{\isacharcomma}{\kern0pt}{\isadigit{2}}{\isadigit{6}}{\isacharcomma}{\kern0pt}{\isadigit{4}}{\isadigit{5}}{\isacharcomma}{\kern0pt}{\isadigit{2}}{\isadigit{8}}{\isacharcomma}{\kern0pt}{\isadigit{4}}{\isadigit{1}}{\isacharcomma}{\kern0pt}{\isadigit{3}}{\isadigit{2}}{\isacharcomma}{\kern0pt}{\isadigit{5}}{\isadigit{5}}{\isacharbrackright}{\kern0pt}{\isacharcomma}{\kern0pt}\isanewline
\ \ {\isacharbrackleft}{\kern0pt}{\isadigit{1}}{\isadigit{4}}{\isacharcomma}{\kern0pt}{\isadigit{4}}{\isadigit{7}}{\isacharcomma}{\kern0pt}{\isadigit{4}}{\isadigit{4}}{\isacharcomma}{\kern0pt}{\isadigit{1}}{\isadigit{1}}{\isacharcomma}{\kern0pt}{\isadigit{2}}{\isadigit{4}}{\isacharcomma}{\kern0pt}{\isadigit{5}}{\isadigit{3}}{\isacharcomma}{\kern0pt}{\isadigit{3}}{\isadigit{4}}{\isacharcomma}{\kern0pt}{\isadigit{9}}{\isacharcomma}{\kern0pt}{\isadigit{2}}{\isadigit{2}}{\isacharbrackright}{\kern0pt}{\isacharcomma}{\kern0pt}\isanewline
\ \ {\isacharbrackleft}{\kern0pt}{\isadigit{1}}{\isacharcomma}{\kern0pt}{\isadigit{1}}{\isadigit{2}}{\isacharcomma}{\kern0pt}{\isadigit{2}}{\isadigit{5}}{\isacharcomma}{\kern0pt}{\isadigit{4}}{\isadigit{6}}{\isacharcomma}{\kern0pt}{\isadigit{2}}{\isadigit{7}}{\isacharcomma}{\kern0pt}{\isadigit{1}}{\isadigit{0}}{\isacharcomma}{\kern0pt}{\isadigit{2}}{\isadigit{3}}{\isacharcomma}{\kern0pt}{\isadigit{5}}{\isadigit{4}}{\isacharcomma}{\kern0pt}{\isadigit{3}}{\isadigit{3}}{\isacharbrackright}{\kern0pt}{\isacharbrackright}{\kern0pt}{\isacharparenright}{\kern0pt}{\isachardoublequoteclose}\isanewline
\isacommand{lemma}\isamarkupfalse%
\ kp{\isacharunderscore}{\kern0pt}{\isadigit{7}}x{\isadigit{9}}{\isacharunderscore}{\kern0pt}ul{\isacharcolon}{\kern0pt}\ {\isachardoublequoteopen}knights{\isacharunderscore}{\kern0pt}path\ b{\isadigit{7}}x{\isadigit{9}}\ kp{\isadigit{7}}x{\isadigit{9}}ul{\isachardoublequoteclose}\isanewline
%
\isadelimproof
\ \ %
\endisadelimproof
%
\isatagproof
\isacommand{by}\isamarkupfalse%
\ {\isacharparenleft}{\kern0pt}simp\ only{\isacharcolon}{\kern0pt}\ knights{\isacharunderscore}{\kern0pt}path{\isacharunderscore}{\kern0pt}exec{\isacharunderscore}{\kern0pt}simp{\isacharparenright}{\kern0pt}\ eval%
\endisatagproof
{\isafoldproof}%
%
\isadelimproof
\isanewline
%
\endisadelimproof
\isanewline
\isacommand{abbreviation}\isamarkupfalse%
\ {\isachardoublequoteopen}b{\isadigit{9}}x{\isadigit{7}}\ {\isasymequiv}\ board\ {\isadigit{9}}\ {\isadigit{7}}{\isachardoublequoteclose}%
\begin{isamarkuptext}%
A Knight's path for the \isa{{\isacharparenleft}{\kern0pt}{\isadigit{9}}{\isasymtimes}{\isadigit{7}}{\isacharparenright}{\kern0pt}}-board that starts in the lower-left and ends in the 
upper-left.
  \begin{table}[H]
    \begin{tabular}{lllllll}
       5 & 20 & 53 & 48 &  7 & 22 & 31 \\
      52 & 63 &  6 & 21 & 32 & 55 &  8 \\
      19 &  4 & 49 & 54 & 47 & 30 & 23 \\
      62 & 51 & 46 & 33 & 56 &  9 & 58 \\
       3 & 18 & 61 & 50 & 59 & 24 & 29 \\
      14 & 43 & 34 & 45 & 28 & 57 & 10 \\
      17 &  2 & 15 & 60 & 35 & 38 & 25 \\
      42 & 13 & 44 & 27 & 40 & 11 & 36 \\
       1 & 16 & 41 & 12 & 37 & 26 & 39
    \end{tabular}
  \end{table}%
\end{isamarkuptext}\isamarkuptrue%
\isacommand{abbreviation}\isamarkupfalse%
\ {\isachardoublequoteopen}kp{\isadigit{9}}x{\isadigit{7}}ul\ {\isasymequiv}\ the\ {\isacharparenleft}{\kern0pt}to{\isacharunderscore}{\kern0pt}path\ \isanewline
\ \ {\isacharbrackleft}{\kern0pt}{\isacharbrackleft}{\kern0pt}{\isadigit{5}}{\isacharcomma}{\kern0pt}{\isadigit{2}}{\isadigit{0}}{\isacharcomma}{\kern0pt}{\isadigit{5}}{\isadigit{3}}{\isacharcomma}{\kern0pt}{\isadigit{4}}{\isadigit{8}}{\isacharcomma}{\kern0pt}{\isadigit{7}}{\isacharcomma}{\kern0pt}{\isadigit{2}}{\isadigit{2}}{\isacharcomma}{\kern0pt}{\isadigit{3}}{\isadigit{1}}{\isacharbrackright}{\kern0pt}{\isacharcomma}{\kern0pt}\isanewline
\ \ {\isacharbrackleft}{\kern0pt}{\isadigit{5}}{\isadigit{2}}{\isacharcomma}{\kern0pt}{\isadigit{6}}{\isadigit{3}}{\isacharcomma}{\kern0pt}{\isadigit{6}}{\isacharcomma}{\kern0pt}{\isadigit{2}}{\isadigit{1}}{\isacharcomma}{\kern0pt}{\isadigit{3}}{\isadigit{2}}{\isacharcomma}{\kern0pt}{\isadigit{5}}{\isadigit{5}}{\isacharcomma}{\kern0pt}{\isadigit{8}}{\isacharbrackright}{\kern0pt}{\isacharcomma}{\kern0pt}\isanewline
\ \ {\isacharbrackleft}{\kern0pt}{\isadigit{1}}{\isadigit{9}}{\isacharcomma}{\kern0pt}{\isadigit{4}}{\isacharcomma}{\kern0pt}{\isadigit{4}}{\isadigit{9}}{\isacharcomma}{\kern0pt}{\isadigit{5}}{\isadigit{4}}{\isacharcomma}{\kern0pt}{\isadigit{4}}{\isadigit{7}}{\isacharcomma}{\kern0pt}{\isadigit{3}}{\isadigit{0}}{\isacharcomma}{\kern0pt}{\isadigit{2}}{\isadigit{3}}{\isacharbrackright}{\kern0pt}{\isacharcomma}{\kern0pt}\isanewline
\ \ {\isacharbrackleft}{\kern0pt}{\isadigit{6}}{\isadigit{2}}{\isacharcomma}{\kern0pt}{\isadigit{5}}{\isadigit{1}}{\isacharcomma}{\kern0pt}{\isadigit{4}}{\isadigit{6}}{\isacharcomma}{\kern0pt}{\isadigit{3}}{\isadigit{3}}{\isacharcomma}{\kern0pt}{\isadigit{5}}{\isadigit{6}}{\isacharcomma}{\kern0pt}{\isadigit{9}}{\isacharcomma}{\kern0pt}{\isadigit{5}}{\isadigit{8}}{\isacharbrackright}{\kern0pt}{\isacharcomma}{\kern0pt}\isanewline
\ \ {\isacharbrackleft}{\kern0pt}{\isadigit{3}}{\isacharcomma}{\kern0pt}{\isadigit{1}}{\isadigit{8}}{\isacharcomma}{\kern0pt}{\isadigit{6}}{\isadigit{1}}{\isacharcomma}{\kern0pt}{\isadigit{5}}{\isadigit{0}}{\isacharcomma}{\kern0pt}{\isadigit{5}}{\isadigit{9}}{\isacharcomma}{\kern0pt}{\isadigit{2}}{\isadigit{4}}{\isacharcomma}{\kern0pt}{\isadigit{2}}{\isadigit{9}}{\isacharbrackright}{\kern0pt}{\isacharcomma}{\kern0pt}\isanewline
\ \ {\isacharbrackleft}{\kern0pt}{\isadigit{1}}{\isadigit{4}}{\isacharcomma}{\kern0pt}{\isadigit{4}}{\isadigit{3}}{\isacharcomma}{\kern0pt}{\isadigit{3}}{\isadigit{4}}{\isacharcomma}{\kern0pt}{\isadigit{4}}{\isadigit{5}}{\isacharcomma}{\kern0pt}{\isadigit{2}}{\isadigit{8}}{\isacharcomma}{\kern0pt}{\isadigit{5}}{\isadigit{7}}{\isacharcomma}{\kern0pt}{\isadigit{1}}{\isadigit{0}}{\isacharbrackright}{\kern0pt}{\isacharcomma}{\kern0pt}\isanewline
\ \ {\isacharbrackleft}{\kern0pt}{\isadigit{1}}{\isadigit{7}}{\isacharcomma}{\kern0pt}{\isadigit{2}}{\isacharcomma}{\kern0pt}{\isadigit{1}}{\isadigit{5}}{\isacharcomma}{\kern0pt}{\isadigit{6}}{\isadigit{0}}{\isacharcomma}{\kern0pt}{\isadigit{3}}{\isadigit{5}}{\isacharcomma}{\kern0pt}{\isadigit{3}}{\isadigit{8}}{\isacharcomma}{\kern0pt}{\isadigit{2}}{\isadigit{5}}{\isacharbrackright}{\kern0pt}{\isacharcomma}{\kern0pt}\isanewline
\ \ {\isacharbrackleft}{\kern0pt}{\isadigit{4}}{\isadigit{2}}{\isacharcomma}{\kern0pt}{\isadigit{1}}{\isadigit{3}}{\isacharcomma}{\kern0pt}{\isadigit{4}}{\isadigit{4}}{\isacharcomma}{\kern0pt}{\isadigit{2}}{\isadigit{7}}{\isacharcomma}{\kern0pt}{\isadigit{4}}{\isadigit{0}}{\isacharcomma}{\kern0pt}{\isadigit{1}}{\isadigit{1}}{\isacharcomma}{\kern0pt}{\isadigit{3}}{\isadigit{6}}{\isacharbrackright}{\kern0pt}{\isacharcomma}{\kern0pt}\isanewline
\ \ {\isacharbrackleft}{\kern0pt}{\isadigit{1}}{\isacharcomma}{\kern0pt}{\isadigit{1}}{\isadigit{6}}{\isacharcomma}{\kern0pt}{\isadigit{4}}{\isadigit{1}}{\isacharcomma}{\kern0pt}{\isadigit{1}}{\isadigit{2}}{\isacharcomma}{\kern0pt}{\isadigit{3}}{\isadigit{7}}{\isacharcomma}{\kern0pt}{\isadigit{2}}{\isadigit{6}}{\isacharcomma}{\kern0pt}{\isadigit{3}}{\isadigit{9}}{\isacharbrackright}{\kern0pt}{\isacharbrackright}{\kern0pt}{\isacharparenright}{\kern0pt}{\isachardoublequoteclose}\isanewline
\isacommand{lemma}\isamarkupfalse%
\ kp{\isacharunderscore}{\kern0pt}{\isadigit{9}}x{\isadigit{7}}{\isacharunderscore}{\kern0pt}ul{\isacharcolon}{\kern0pt}\ {\isachardoublequoteopen}knights{\isacharunderscore}{\kern0pt}path\ b{\isadigit{9}}x{\isadigit{7}}\ kp{\isadigit{9}}x{\isadigit{7}}ul{\isachardoublequoteclose}\isanewline
%
\isadelimproof
\ \ %
\endisadelimproof
%
\isatagproof
\isacommand{by}\isamarkupfalse%
\ {\isacharparenleft}{\kern0pt}simp\ only{\isacharcolon}{\kern0pt}\ knights{\isacharunderscore}{\kern0pt}path{\isacharunderscore}{\kern0pt}exec{\isacharunderscore}{\kern0pt}simp{\isacharparenright}{\kern0pt}\ eval%
\endisatagproof
{\isafoldproof}%
%
\isadelimproof
\isanewline
%
\endisadelimproof
\isanewline
\isacommand{abbreviation}\isamarkupfalse%
\ {\isachardoublequoteopen}b{\isadigit{9}}x{\isadigit{9}}\ {\isasymequiv}\ board\ {\isadigit{9}}\ {\isadigit{9}}{\isachardoublequoteclose}%
\begin{isamarkuptext}%
A Knight's path for the \isa{{\isacharparenleft}{\kern0pt}{\isadigit{9}}{\isasymtimes}{\isadigit{9}}{\isacharparenright}{\kern0pt}}-board that starts in the lower-left and ends in the 
upper-left.
  \begin{table}[H]
    \begin{tabular}{lllllllll}
      13 & 26 & 39 & 52 & 11 & 24 & 37 & 50 &  9 \\
      40 & 81 & 12 & 25 & 38 & 51 & 10 & 23 & 36 \\
      27 & 14 & 53 & 58 & 63 & 68 & 73 &  8 & 49 \\
      80 & 41 & 64 & 67 & 72 & 57 & 62 & 35 & 22 \\
      15 & 28 & 59 & 54 & 65 & 74 & 69 & 48 &  7 \\
      42 & 79 & 66 & 71 & 76 & 61 & 56 & 21 & 34 \\
      29 & 16 & 77 & 60 & 55 & 70 & 75 &  6 & 47 \\
      78 & 43 &  2 & 31 & 18 & 45 &  4 & 33 & 20 \\
       1 & 30 & 17 & 44 &  3 & 32 & 19 & 46 &  5
    \end{tabular}
  \end{table}%
\end{isamarkuptext}\isamarkuptrue%
\isacommand{abbreviation}\isamarkupfalse%
\ {\isachardoublequoteopen}kp{\isadigit{9}}x{\isadigit{9}}ul\ {\isasymequiv}\ the\ {\isacharparenleft}{\kern0pt}to{\isacharunderscore}{\kern0pt}path\ \isanewline
\ \ {\isacharbrackleft}{\kern0pt}{\isacharbrackleft}{\kern0pt}{\isadigit{1}}{\isadigit{3}}{\isacharcomma}{\kern0pt}{\isadigit{2}}{\isadigit{6}}{\isacharcomma}{\kern0pt}{\isadigit{3}}{\isadigit{9}}{\isacharcomma}{\kern0pt}{\isadigit{5}}{\isadigit{2}}{\isacharcomma}{\kern0pt}{\isadigit{1}}{\isadigit{1}}{\isacharcomma}{\kern0pt}{\isadigit{2}}{\isadigit{4}}{\isacharcomma}{\kern0pt}{\isadigit{3}}{\isadigit{7}}{\isacharcomma}{\kern0pt}{\isadigit{5}}{\isadigit{0}}{\isacharcomma}{\kern0pt}{\isadigit{9}}{\isacharbrackright}{\kern0pt}{\isacharcomma}{\kern0pt}\isanewline
\ \ {\isacharbrackleft}{\kern0pt}{\isadigit{4}}{\isadigit{0}}{\isacharcomma}{\kern0pt}{\isadigit{8}}{\isadigit{1}}{\isacharcomma}{\kern0pt}{\isadigit{1}}{\isadigit{2}}{\isacharcomma}{\kern0pt}{\isadigit{2}}{\isadigit{5}}{\isacharcomma}{\kern0pt}{\isadigit{3}}{\isadigit{8}}{\isacharcomma}{\kern0pt}{\isadigit{5}}{\isadigit{1}}{\isacharcomma}{\kern0pt}{\isadigit{1}}{\isadigit{0}}{\isacharcomma}{\kern0pt}{\isadigit{2}}{\isadigit{3}}{\isacharcomma}{\kern0pt}{\isadigit{3}}{\isadigit{6}}{\isacharbrackright}{\kern0pt}{\isacharcomma}{\kern0pt}\isanewline
\ \ {\isacharbrackleft}{\kern0pt}{\isadigit{2}}{\isadigit{7}}{\isacharcomma}{\kern0pt}{\isadigit{1}}{\isadigit{4}}{\isacharcomma}{\kern0pt}{\isadigit{5}}{\isadigit{3}}{\isacharcomma}{\kern0pt}{\isadigit{5}}{\isadigit{8}}{\isacharcomma}{\kern0pt}{\isadigit{6}}{\isadigit{3}}{\isacharcomma}{\kern0pt}{\isadigit{6}}{\isadigit{8}}{\isacharcomma}{\kern0pt}{\isadigit{7}}{\isadigit{3}}{\isacharcomma}{\kern0pt}{\isadigit{8}}{\isacharcomma}{\kern0pt}{\isadigit{4}}{\isadigit{9}}{\isacharbrackright}{\kern0pt}{\isacharcomma}{\kern0pt}\isanewline
\ \ {\isacharbrackleft}{\kern0pt}{\isadigit{8}}{\isadigit{0}}{\isacharcomma}{\kern0pt}{\isadigit{4}}{\isadigit{1}}{\isacharcomma}{\kern0pt}{\isadigit{6}}{\isadigit{4}}{\isacharcomma}{\kern0pt}{\isadigit{6}}{\isadigit{7}}{\isacharcomma}{\kern0pt}{\isadigit{7}}{\isadigit{2}}{\isacharcomma}{\kern0pt}{\isadigit{5}}{\isadigit{7}}{\isacharcomma}{\kern0pt}{\isadigit{6}}{\isadigit{2}}{\isacharcomma}{\kern0pt}{\isadigit{3}}{\isadigit{5}}{\isacharcomma}{\kern0pt}{\isadigit{2}}{\isadigit{2}}{\isacharbrackright}{\kern0pt}{\isacharcomma}{\kern0pt}\isanewline
\ \ {\isacharbrackleft}{\kern0pt}{\isadigit{1}}{\isadigit{5}}{\isacharcomma}{\kern0pt}{\isadigit{2}}{\isadigit{8}}{\isacharcomma}{\kern0pt}{\isadigit{5}}{\isadigit{9}}{\isacharcomma}{\kern0pt}{\isadigit{5}}{\isadigit{4}}{\isacharcomma}{\kern0pt}{\isadigit{6}}{\isadigit{5}}{\isacharcomma}{\kern0pt}{\isadigit{7}}{\isadigit{4}}{\isacharcomma}{\kern0pt}{\isadigit{6}}{\isadigit{9}}{\isacharcomma}{\kern0pt}{\isadigit{4}}{\isadigit{8}}{\isacharcomma}{\kern0pt}{\isadigit{7}}{\isacharbrackright}{\kern0pt}{\isacharcomma}{\kern0pt}\isanewline
\ \ {\isacharbrackleft}{\kern0pt}{\isadigit{4}}{\isadigit{2}}{\isacharcomma}{\kern0pt}{\isadigit{7}}{\isadigit{9}}{\isacharcomma}{\kern0pt}{\isadigit{6}}{\isadigit{6}}{\isacharcomma}{\kern0pt}{\isadigit{7}}{\isadigit{1}}{\isacharcomma}{\kern0pt}{\isadigit{7}}{\isadigit{6}}{\isacharcomma}{\kern0pt}{\isadigit{6}}{\isadigit{1}}{\isacharcomma}{\kern0pt}{\isadigit{5}}{\isadigit{6}}{\isacharcomma}{\kern0pt}{\isadigit{2}}{\isadigit{1}}{\isacharcomma}{\kern0pt}{\isadigit{3}}{\isadigit{4}}{\isacharbrackright}{\kern0pt}{\isacharcomma}{\kern0pt}\isanewline
\ \ {\isacharbrackleft}{\kern0pt}{\isadigit{2}}{\isadigit{9}}{\isacharcomma}{\kern0pt}{\isadigit{1}}{\isadigit{6}}{\isacharcomma}{\kern0pt}{\isadigit{7}}{\isadigit{7}}{\isacharcomma}{\kern0pt}{\isadigit{6}}{\isadigit{0}}{\isacharcomma}{\kern0pt}{\isadigit{5}}{\isadigit{5}}{\isacharcomma}{\kern0pt}{\isadigit{7}}{\isadigit{0}}{\isacharcomma}{\kern0pt}{\isadigit{7}}{\isadigit{5}}{\isacharcomma}{\kern0pt}{\isadigit{6}}{\isacharcomma}{\kern0pt}{\isadigit{4}}{\isadigit{7}}{\isacharbrackright}{\kern0pt}{\isacharcomma}{\kern0pt}\isanewline
\ \ {\isacharbrackleft}{\kern0pt}{\isadigit{7}}{\isadigit{8}}{\isacharcomma}{\kern0pt}{\isadigit{4}}{\isadigit{3}}{\isacharcomma}{\kern0pt}{\isadigit{2}}{\isacharcomma}{\kern0pt}{\isadigit{3}}{\isadigit{1}}{\isacharcomma}{\kern0pt}{\isadigit{1}}{\isadigit{8}}{\isacharcomma}{\kern0pt}{\isadigit{4}}{\isadigit{5}}{\isacharcomma}{\kern0pt}{\isadigit{4}}{\isacharcomma}{\kern0pt}{\isadigit{3}}{\isadigit{3}}{\isacharcomma}{\kern0pt}{\isadigit{2}}{\isadigit{0}}{\isacharbrackright}{\kern0pt}{\isacharcomma}{\kern0pt}\isanewline
\ \ {\isacharbrackleft}{\kern0pt}{\isadigit{1}}{\isacharcomma}{\kern0pt}{\isadigit{3}}{\isadigit{0}}{\isacharcomma}{\kern0pt}{\isadigit{1}}{\isadigit{7}}{\isacharcomma}{\kern0pt}{\isadigit{4}}{\isadigit{4}}{\isacharcomma}{\kern0pt}{\isadigit{3}}{\isacharcomma}{\kern0pt}{\isadigit{3}}{\isadigit{2}}{\isacharcomma}{\kern0pt}{\isadigit{1}}{\isadigit{9}}{\isacharcomma}{\kern0pt}{\isadigit{4}}{\isadigit{6}}{\isacharcomma}{\kern0pt}{\isadigit{5}}{\isacharbrackright}{\kern0pt}{\isacharbrackright}{\kern0pt}{\isacharparenright}{\kern0pt}{\isachardoublequoteclose}\isanewline
\isacommand{lemma}\isamarkupfalse%
\ kp{\isacharunderscore}{\kern0pt}{\isadigit{9}}x{\isadigit{9}}{\isacharunderscore}{\kern0pt}ul{\isacharcolon}{\kern0pt}\ {\isachardoublequoteopen}knights{\isacharunderscore}{\kern0pt}path\ b{\isadigit{9}}x{\isadigit{9}}\ kp{\isadigit{9}}x{\isadigit{9}}ul{\isachardoublequoteclose}\isanewline
%
\isadelimproof
\ \ %
\endisadelimproof
%
\isatagproof
\isacommand{by}\isamarkupfalse%
\ {\isacharparenleft}{\kern0pt}simp\ only{\isacharcolon}{\kern0pt}\ knights{\isacharunderscore}{\kern0pt}path{\isacharunderscore}{\kern0pt}exec{\isacharunderscore}{\kern0pt}simp{\isacharparenright}{\kern0pt}\ eval%
\endisatagproof
{\isafoldproof}%
%
\isadelimproof
%
\endisadelimproof
%
\begin{isamarkuptext}%
The following lemma is a sub-proof used in Lemma 4 in \cite{cull_decurtins_1987}. 
I moved the sub-proof out to a separate lemma.%
\end{isamarkuptext}\isamarkuptrue%
\isacommand{lemma}\isamarkupfalse%
\ knights{\isacharunderscore}{\kern0pt}circuit{\isacharunderscore}{\kern0pt}exists{\isacharunderscore}{\kern0pt}even{\isacharunderscore}{\kern0pt}n{\isacharunderscore}{\kern0pt}gr{\isadigit{1}}{\isadigit{0}}{\isacharcolon}{\kern0pt}\isanewline
\ \ \isakeyword{assumes}\ {\isachardoublequoteopen}even\ n{\isachardoublequoteclose}\ {\isachardoublequoteopen}n\ {\isasymge}\ {\isadigit{1}}{\isadigit{0}}{\isachardoublequoteclose}\ {\isachardoublequoteopen}m\ {\isasymge}\ {\isadigit{5}}{\isachardoublequoteclose}\isanewline
\ \ \ \ \ \ \ \ \ \ {\isachardoublequoteopen}{\isasymexists}ps{\isachardot}{\kern0pt}\ knights{\isacharunderscore}{\kern0pt}path\ {\isacharparenleft}{\kern0pt}board\ {\isacharparenleft}{\kern0pt}n{\isacharminus}{\kern0pt}{\isadigit{5}}{\isacharparenright}{\kern0pt}\ m{\isacharparenright}{\kern0pt}\ ps\ {\isasymand}\ hd\ ps\ {\isacharequal}{\kern0pt}\ {\isacharparenleft}{\kern0pt}int\ {\isacharparenleft}{\kern0pt}n{\isacharminus}{\kern0pt}{\isadigit{5}}{\isacharparenright}{\kern0pt}{\isacharcomma}{\kern0pt}{\isadigit{1}}{\isacharparenright}{\kern0pt}\ \isanewline
\ \ \ \ \ \ \ \ \ \ \ \ {\isasymand}\ last\ ps\ {\isacharequal}{\kern0pt}\ {\isacharparenleft}{\kern0pt}int\ {\isacharparenleft}{\kern0pt}n{\isacharminus}{\kern0pt}{\isadigit{5}}{\isacharparenright}{\kern0pt}{\isacharminus}{\kern0pt}{\isadigit{1}}{\isacharcomma}{\kern0pt}int\ m{\isacharminus}{\kern0pt}{\isadigit{1}}{\isacharparenright}{\kern0pt}{\isachardoublequoteclose}\isanewline
\ \ \isakeyword{shows}\ {\isachardoublequoteopen}{\isasymexists}ps{\isachardot}{\kern0pt}\ knights{\isacharunderscore}{\kern0pt}circuit\ {\isacharparenleft}{\kern0pt}board\ m\ n{\isacharparenright}{\kern0pt}\ ps{\isachardoublequoteclose}\isanewline
%
\isadelimproof
\ \ %
\endisadelimproof
%
\isatagproof
\isacommand{using}\isamarkupfalse%
\ assms\isanewline
\isacommand{proof}\isamarkupfalse%
\ {\isacharminus}{\kern0pt}\isanewline
\ \ \isacommand{let}\isamarkupfalse%
\ {\isacharquery}{\kern0pt}b\isactrlsub {\isadigit{2}}{\isacharequal}{\kern0pt}{\isachardoublequoteopen}board\ {\isacharparenleft}{\kern0pt}n{\isacharminus}{\kern0pt}{\isadigit{5}}{\isacharparenright}{\kern0pt}\ m{\isachardoublequoteclose}\isanewline
\ \ \isacommand{assume}\isamarkupfalse%
\ {\isachardoublequoteopen}n\ {\isasymge}\ {\isadigit{1}}{\isadigit{0}}{\isachardoublequoteclose}\isanewline
\ \ \isacommand{then}\isamarkupfalse%
\ \isacommand{obtain}\isamarkupfalse%
\ ps\isactrlsub {\isadigit{2}}\ \isakeyword{where}\ ps\isactrlsub {\isadigit{2}}{\isacharunderscore}{\kern0pt}prems{\isacharcolon}{\kern0pt}\ {\isachardoublequoteopen}knights{\isacharunderscore}{\kern0pt}path\ {\isacharquery}{\kern0pt}b\isactrlsub {\isadigit{2}}\ ps\isactrlsub {\isadigit{2}}{\isachardoublequoteclose}\ {\isachardoublequoteopen}hd\ ps\isactrlsub {\isadigit{2}}\ {\isacharequal}{\kern0pt}\ {\isacharparenleft}{\kern0pt}int\ {\isacharparenleft}{\kern0pt}n{\isacharminus}{\kern0pt}{\isadigit{5}}{\isacharparenright}{\kern0pt}{\isacharcomma}{\kern0pt}{\isadigit{1}}{\isacharparenright}{\kern0pt}{\isachardoublequoteclose}\ \isanewline
\ \ \ \ \ \ {\isachardoublequoteopen}last\ ps\isactrlsub {\isadigit{2}}\ {\isacharequal}{\kern0pt}\ {\isacharparenleft}{\kern0pt}int\ {\isacharparenleft}{\kern0pt}n{\isacharminus}{\kern0pt}{\isadigit{5}}{\isacharparenright}{\kern0pt}{\isacharminus}{\kern0pt}{\isadigit{1}}{\isacharcomma}{\kern0pt}int\ m{\isacharminus}{\kern0pt}{\isadigit{1}}{\isacharparenright}{\kern0pt}{\isachardoublequoteclose}\isanewline
\ \ \ \ \isacommand{using}\isamarkupfalse%
\ assms\ \isacommand{by}\isamarkupfalse%
\ auto\isanewline
\ \ \isacommand{let}\isamarkupfalse%
\ {\isacharquery}{\kern0pt}ps\isactrlsub {\isadigit{2}}{\isacharunderscore}{\kern0pt}m{\isadigit{2}}{\isacharequal}{\kern0pt}{\isachardoublequoteopen}mirror{\isadigit{2}}\ ps\isactrlsub {\isadigit{2}}{\isachardoublequoteclose}\isanewline
\ \ \isacommand{have}\isamarkupfalse%
\ ps\isactrlsub {\isadigit{2}}{\isacharunderscore}{\kern0pt}m{\isadigit{2}}{\isacharunderscore}{\kern0pt}prems{\isacharcolon}{\kern0pt}\ {\isachardoublequoteopen}knights{\isacharunderscore}{\kern0pt}path\ {\isacharquery}{\kern0pt}b\isactrlsub {\isadigit{2}}\ {\isacharquery}{\kern0pt}ps\isactrlsub {\isadigit{2}}{\isacharunderscore}{\kern0pt}m{\isadigit{2}}{\isachardoublequoteclose}\ {\isachardoublequoteopen}hd\ {\isacharquery}{\kern0pt}ps\isactrlsub {\isadigit{2}}{\isacharunderscore}{\kern0pt}m{\isadigit{2}}\ {\isacharequal}{\kern0pt}\ {\isacharparenleft}{\kern0pt}int\ {\isacharparenleft}{\kern0pt}n{\isacharminus}{\kern0pt}{\isadigit{5}}{\isacharparenright}{\kern0pt}{\isacharcomma}{\kern0pt}int\ m{\isacharparenright}{\kern0pt}{\isachardoublequoteclose}\ \isanewline
\ \ \ \ \ \ {\isachardoublequoteopen}last\ {\isacharquery}{\kern0pt}ps\isactrlsub {\isadigit{2}}{\isacharunderscore}{\kern0pt}m{\isadigit{2}}\ {\isacharequal}{\kern0pt}\ {\isacharparenleft}{\kern0pt}int\ {\isacharparenleft}{\kern0pt}n{\isacharminus}{\kern0pt}{\isadigit{5}}{\isacharparenright}{\kern0pt}{\isacharminus}{\kern0pt}{\isadigit{1}}{\isacharcomma}{\kern0pt}{\isadigit{2}}{\isacharparenright}{\kern0pt}{\isachardoublequoteclose}\isanewline
\ \ \ \ \isacommand{using}\isamarkupfalse%
\ ps\isactrlsub {\isadigit{2}}{\isacharunderscore}{\kern0pt}prems\ mirror{\isadigit{2}}{\isacharunderscore}{\kern0pt}knights{\isacharunderscore}{\kern0pt}path\ hd{\isacharunderscore}{\kern0pt}mirror{\isadigit{2}}\ last{\isacharunderscore}{\kern0pt}mirror{\isadigit{2}}\ \isacommand{by}\isamarkupfalse%
\ auto\isanewline
\isanewline
\ \ \isacommand{obtain}\isamarkupfalse%
\ ps\isactrlsub {\isadigit{1}}\ \isakeyword{where}\ ps\isactrlsub {\isadigit{1}}{\isacharunderscore}{\kern0pt}prems{\isacharcolon}{\kern0pt}\ {\isachardoublequoteopen}knights{\isacharunderscore}{\kern0pt}path\ {\isacharparenleft}{\kern0pt}board\ {\isadigit{5}}\ m{\isacharparenright}{\kern0pt}\ ps\isactrlsub {\isadigit{1}}{\isachardoublequoteclose}\ {\isachardoublequoteopen}hd\ ps\isactrlsub {\isadigit{1}}\ {\isacharequal}{\kern0pt}\ {\isacharparenleft}{\kern0pt}{\isadigit{1}}{\isacharcomma}{\kern0pt}{\isadigit{1}}{\isacharparenright}{\kern0pt}{\isachardoublequoteclose}{\isachardoublequoteopen}last\ ps\isactrlsub {\isadigit{1}}\ {\isacharequal}{\kern0pt}\ {\isacharparenleft}{\kern0pt}{\isadigit{2}}{\isacharcomma}{\kern0pt}int\ m{\isacharminus}{\kern0pt}{\isadigit{1}}{\isacharparenright}{\kern0pt}{\isachardoublequoteclose}\isanewline
\ \ \ \ \isacommand{using}\isamarkupfalse%
\ assms\ knights{\isacharunderscore}{\kern0pt}path{\isacharunderscore}{\kern0pt}{\isadigit{5}}xm{\isacharunderscore}{\kern0pt}exists\ \isacommand{by}\isamarkupfalse%
\ auto\isanewline
\ \ \isacommand{let}\isamarkupfalse%
\ {\isacharquery}{\kern0pt}ps\isactrlsub {\isadigit{1}}{\isacharprime}{\kern0pt}{\isacharequal}{\kern0pt}{\isachardoublequoteopen}trans{\isacharunderscore}{\kern0pt}path\ {\isacharparenleft}{\kern0pt}int\ {\isacharparenleft}{\kern0pt}n{\isacharminus}{\kern0pt}{\isadigit{5}}{\isacharparenright}{\kern0pt}{\isacharcomma}{\kern0pt}{\isadigit{0}}{\isacharparenright}{\kern0pt}\ ps\isactrlsub {\isadigit{1}}{\isachardoublequoteclose}\isanewline
\ \ \isacommand{let}\isamarkupfalse%
\ {\isacharquery}{\kern0pt}b\isactrlsub {\isadigit{1}}{\isacharprime}{\kern0pt}{\isacharequal}{\kern0pt}{\isachardoublequoteopen}trans{\isacharunderscore}{\kern0pt}board\ {\isacharparenleft}{\kern0pt}int\ {\isacharparenleft}{\kern0pt}n{\isacharminus}{\kern0pt}{\isadigit{5}}{\isacharparenright}{\kern0pt}{\isacharcomma}{\kern0pt}{\isadigit{0}}{\isacharparenright}{\kern0pt}\ {\isacharparenleft}{\kern0pt}board\ {\isadigit{5}}\ m{\isacharparenright}{\kern0pt}{\isachardoublequoteclose}\isanewline
\ \ \isacommand{have}\isamarkupfalse%
\ ps\isactrlsub {\isadigit{1}}{\isacharprime}{\kern0pt}{\isacharunderscore}{\kern0pt}prems{\isacharcolon}{\kern0pt}\ {\isachardoublequoteopen}knights{\isacharunderscore}{\kern0pt}path\ {\isacharquery}{\kern0pt}b\isactrlsub {\isadigit{1}}{\isacharprime}{\kern0pt}\ {\isacharquery}{\kern0pt}ps\isactrlsub {\isadigit{1}}{\isacharprime}{\kern0pt}{\isachardoublequoteclose}\ {\isachardoublequoteopen}hd\ {\isacharquery}{\kern0pt}ps\isactrlsub {\isadigit{1}}{\isacharprime}{\kern0pt}\ {\isacharequal}{\kern0pt}\ {\isacharparenleft}{\kern0pt}int\ {\isacharparenleft}{\kern0pt}n{\isacharminus}{\kern0pt}{\isadigit{5}}{\isacharparenright}{\kern0pt}{\isacharplus}{\kern0pt}{\isadigit{1}}{\isacharcomma}{\kern0pt}{\isadigit{1}}{\isacharparenright}{\kern0pt}{\isachardoublequoteclose}\ \isanewline
\ \ \ \ \ \ {\isachardoublequoteopen}last\ {\isacharquery}{\kern0pt}ps\isactrlsub {\isadigit{1}}{\isacharprime}{\kern0pt}\ {\isacharequal}{\kern0pt}\ {\isacharparenleft}{\kern0pt}int\ {\isacharparenleft}{\kern0pt}n{\isacharminus}{\kern0pt}{\isadigit{5}}{\isacharparenright}{\kern0pt}{\isacharplus}{\kern0pt}{\isadigit{2}}{\isacharcomma}{\kern0pt}int\ m{\isacharminus}{\kern0pt}{\isadigit{1}}{\isacharparenright}{\kern0pt}{\isachardoublequoteclose}\isanewline
\ \ \ \ \isacommand{using}\isamarkupfalse%
\ ps\isactrlsub {\isadigit{1}}{\isacharunderscore}{\kern0pt}prems\ trans{\isacharunderscore}{\kern0pt}knights{\isacharunderscore}{\kern0pt}path\ knights{\isacharunderscore}{\kern0pt}path{\isacharunderscore}{\kern0pt}non{\isacharunderscore}{\kern0pt}nil\ hd{\isacharunderscore}{\kern0pt}trans{\isacharunderscore}{\kern0pt}path\ last{\isacharunderscore}{\kern0pt}trans{\isacharunderscore}{\kern0pt}path\ \isacommand{by}\isamarkupfalse%
\ auto\isanewline
\isanewline
\ \ \isacommand{let}\isamarkupfalse%
\ {\isacharquery}{\kern0pt}ps{\isacharequal}{\kern0pt}{\isachardoublequoteopen}{\isacharquery}{\kern0pt}ps\isactrlsub {\isadigit{1}}{\isacharprime}{\kern0pt}{\isacharat}{\kern0pt}{\isacharquery}{\kern0pt}ps\isactrlsub {\isadigit{2}}{\isacharunderscore}{\kern0pt}m{\isadigit{2}}{\isachardoublequoteclose}\isanewline
\ \ \isacommand{let}\isamarkupfalse%
\ {\isacharquery}{\kern0pt}psT{\isacharequal}{\kern0pt}{\isachardoublequoteopen}transpose\ {\isacharquery}{\kern0pt}ps{\isachardoublequoteclose}\isanewline
\isanewline
\ \ \isacommand{have}\isamarkupfalse%
\ {\isachardoublequoteopen}n{\isacharminus}{\kern0pt}{\isadigit{5}}\ {\isasymge}\ {\isadigit{5}}{\isachardoublequoteclose}\ \isacommand{using}\isamarkupfalse%
\ {\isacartoucheopen}n\ {\isasymge}\ {\isadigit{1}}{\isadigit{0}}{\isacartoucheclose}\ \isacommand{by}\isamarkupfalse%
\ auto\isanewline
\ \ \isacommand{have}\isamarkupfalse%
\ inter{\isacharcolon}{\kern0pt}\ {\isachardoublequoteopen}{\isacharquery}{\kern0pt}b\isactrlsub {\isadigit{1}}{\isacharprime}{\kern0pt}\ {\isasyminter}\ {\isacharquery}{\kern0pt}b\isactrlsub {\isadigit{2}}\ {\isacharequal}{\kern0pt}\ {\isacharbraceleft}{\kern0pt}{\isacharbraceright}{\kern0pt}{\isachardoublequoteclose}\isanewline
\ \ \ \ \isacommand{unfolding}\isamarkupfalse%
\ trans{\isacharunderscore}{\kern0pt}board{\isacharunderscore}{\kern0pt}def\ board{\isacharunderscore}{\kern0pt}def\ \isacommand{using}\isamarkupfalse%
\ {\isacartoucheopen}n{\isacharminus}{\kern0pt}{\isadigit{5}}\ {\isasymge}\ {\isadigit{5}}{\isacartoucheclose}\ \isacommand{by}\isamarkupfalse%
\ auto\isanewline
\ \ \isacommand{have}\isamarkupfalse%
\ union{\isacharcolon}{\kern0pt}\ {\isachardoublequoteopen}{\isacharquery}{\kern0pt}b\isactrlsub {\isadigit{1}}{\isacharprime}{\kern0pt}\ {\isasymunion}\ {\isacharquery}{\kern0pt}b\isactrlsub {\isadigit{2}}\ {\isacharequal}{\kern0pt}\ board\ n\ m{\isachardoublequoteclose}\isanewline
\ \ \ \ \isacommand{using}\isamarkupfalse%
\ {\isacartoucheopen}n{\isacharminus}{\kern0pt}{\isadigit{5}}\ {\isasymge}\ {\isadigit{5}}{\isacartoucheclose}\ board{\isacharunderscore}{\kern0pt}concatT{\isacharbrackleft}{\kern0pt}of\ {\isachardoublequoteopen}n{\isacharminus}{\kern0pt}{\isadigit{5}}{\isachardoublequoteclose}\ m\ {\isadigit{5}}{\isacharbrackright}{\kern0pt}\ \isacommand{by}\isamarkupfalse%
\ auto\isanewline
\isanewline
\ \ \isacommand{have}\isamarkupfalse%
\ vs{\isacharcolon}{\kern0pt}\ {\isachardoublequoteopen}valid{\isacharunderscore}{\kern0pt}step\ {\isacharparenleft}{\kern0pt}last\ {\isacharquery}{\kern0pt}ps\isactrlsub {\isadigit{1}}{\isacharprime}{\kern0pt}{\isacharparenright}{\kern0pt}\ {\isacharparenleft}{\kern0pt}hd\ {\isacharquery}{\kern0pt}ps\isactrlsub {\isadigit{2}}{\isacharunderscore}{\kern0pt}m{\isadigit{2}}{\isacharparenright}{\kern0pt}{\isachardoublequoteclose}\ \isakeyword{and}\ {\isachardoublequoteopen}valid{\isacharunderscore}{\kern0pt}step\ {\isacharparenleft}{\kern0pt}last\ {\isacharquery}{\kern0pt}ps\isactrlsub {\isadigit{2}}{\isacharunderscore}{\kern0pt}m{\isadigit{2}}{\isacharparenright}{\kern0pt}\ {\isacharparenleft}{\kern0pt}hd\ {\isacharquery}{\kern0pt}ps\isactrlsub {\isadigit{1}}{\isacharprime}{\kern0pt}{\isacharparenright}{\kern0pt}{\isachardoublequoteclose}\isanewline
\ \ \ \ \isacommand{unfolding}\isamarkupfalse%
\ valid{\isacharunderscore}{\kern0pt}step{\isacharunderscore}{\kern0pt}def\ \isacommand{using}\isamarkupfalse%
\ ps\isactrlsub {\isadigit{1}}{\isacharprime}{\kern0pt}{\isacharunderscore}{\kern0pt}prems\ ps\isactrlsub {\isadigit{2}}{\isacharunderscore}{\kern0pt}m{\isadigit{2}}{\isacharunderscore}{\kern0pt}prems\ \isacommand{by}\isamarkupfalse%
\ auto\isanewline
\ \ \isacommand{then}\isamarkupfalse%
\ \isacommand{have}\isamarkupfalse%
\ vs{\isacharunderscore}{\kern0pt}c{\isacharcolon}{\kern0pt}\ {\isachardoublequoteopen}valid{\isacharunderscore}{\kern0pt}step\ {\isacharparenleft}{\kern0pt}last\ {\isacharquery}{\kern0pt}ps{\isacharparenright}{\kern0pt}\ {\isacharparenleft}{\kern0pt}hd\ {\isacharquery}{\kern0pt}ps{\isacharparenright}{\kern0pt}{\isachardoublequoteclose}\isanewline
\ \ \ \ \isacommand{using}\isamarkupfalse%
\ ps\isactrlsub {\isadigit{1}}{\isacharprime}{\kern0pt}{\isacharunderscore}{\kern0pt}prems\ ps\isactrlsub {\isadigit{2}}{\isacharunderscore}{\kern0pt}m{\isadigit{2}}{\isacharunderscore}{\kern0pt}prems\ knights{\isacharunderscore}{\kern0pt}path{\isacharunderscore}{\kern0pt}non{\isacharunderscore}{\kern0pt}nil\ \isacommand{by}\isamarkupfalse%
\ auto\isanewline
\isanewline
\ \ \isacommand{have}\isamarkupfalse%
\ {\isachardoublequoteopen}knights{\isacharunderscore}{\kern0pt}path\ {\isacharparenleft}{\kern0pt}board\ n\ m{\isacharparenright}{\kern0pt}\ {\isacharquery}{\kern0pt}ps{\isachardoublequoteclose}\isanewline
\ \ \ \ \isacommand{using}\isamarkupfalse%
\ ps\isactrlsub {\isadigit{1}}{\isacharprime}{\kern0pt}{\isacharunderscore}{\kern0pt}prems\ ps\isactrlsub {\isadigit{2}}{\isacharunderscore}{\kern0pt}m{\isadigit{2}}{\isacharunderscore}{\kern0pt}prems\ inter\ vs\ union\ knights{\isacharunderscore}{\kern0pt}path{\isacharunderscore}{\kern0pt}append{\isacharbrackleft}{\kern0pt}of\ {\isacharquery}{\kern0pt}b\isactrlsub {\isadigit{1}}{\isacharprime}{\kern0pt}\ {\isacharquery}{\kern0pt}ps\isactrlsub {\isadigit{1}}{\isacharprime}{\kern0pt}\ {\isacharquery}{\kern0pt}b\isactrlsub {\isadigit{2}}\ {\isacharquery}{\kern0pt}ps\isactrlsub {\isadigit{2}}{\isacharunderscore}{\kern0pt}m{\isadigit{2}}{\isacharbrackright}{\kern0pt}\ \isanewline
\ \ \ \ \isacommand{by}\isamarkupfalse%
\ auto\isanewline
\ \ \isacommand{then}\isamarkupfalse%
\ \isacommand{have}\isamarkupfalse%
\ {\isachardoublequoteopen}knights{\isacharunderscore}{\kern0pt}circuit\ {\isacharparenleft}{\kern0pt}board\ n\ m{\isacharparenright}{\kern0pt}\ {\isacharquery}{\kern0pt}ps{\isachardoublequoteclose}\isanewline
\ \ \ \ \isacommand{unfolding}\isamarkupfalse%
\ knights{\isacharunderscore}{\kern0pt}circuit{\isacharunderscore}{\kern0pt}def\ \isacommand{using}\isamarkupfalse%
\ vs{\isacharunderscore}{\kern0pt}c\ \isacommand{by}\isamarkupfalse%
\ auto\isanewline
\ \ \isacommand{then}\isamarkupfalse%
\ \isacommand{show}\isamarkupfalse%
\ {\isacharquery}{\kern0pt}thesis\ \isacommand{using}\isamarkupfalse%
\ transpose{\isacharunderscore}{\kern0pt}knights{\isacharunderscore}{\kern0pt}circuit\ \isacommand{by}\isamarkupfalse%
\ auto\isanewline
\isacommand{qed}\isamarkupfalse%
%
\endisatagproof
{\isafoldproof}%
%
\isadelimproof
%
\endisadelimproof
%
\begin{isamarkuptext}%
For every \isa{n{\isasymtimes}m}-board with \isa{min\ n\ m\ {\isasymge}\ {\isadigit{5}}} and odd \isa{n} there exists a Knight's path that 
starts in \isa{{\isacharparenleft}{\kern0pt}n{\isacharcomma}{\kern0pt}{\isadigit{1}}{\isacharparenright}{\kern0pt}} (top-left) and ends in \isa{{\isacharparenleft}{\kern0pt}n{\isacharminus}{\kern0pt}{\isadigit{1}}{\isacharcomma}{\kern0pt}m{\isacharminus}{\kern0pt}{\isadigit{1}}{\isacharparenright}{\kern0pt}} (top-right).%
\end{isamarkuptext}\isamarkuptrue%
%
\begin{isamarkuptext}%
This lemma formalizes Lemma 4 from \cite{cull_decurtins_1987}. Formalizing the proof of 
this lemma was quite challenging as a lot of details on how to exactly combine the boards are 
left out in the original proof in \cite{cull_decurtins_1987}.%
\end{isamarkuptext}\isamarkuptrue%
\isacommand{lemma}\isamarkupfalse%
\ knights{\isacharunderscore}{\kern0pt}path{\isacharunderscore}{\kern0pt}odd{\isacharunderscore}{\kern0pt}n{\isacharunderscore}{\kern0pt}exists{\isacharcolon}{\kern0pt}\isanewline
\ \ \isakeyword{assumes}\ {\isachardoublequoteopen}odd\ n{\isachardoublequoteclose}\ {\isachardoublequoteopen}min\ n\ m\ {\isasymge}\ {\isadigit{5}}{\isachardoublequoteclose}\isanewline
\ \ \isakeyword{shows}\ {\isachardoublequoteopen}{\isasymexists}ps{\isachardot}{\kern0pt}\ knights{\isacharunderscore}{\kern0pt}path\ {\isacharparenleft}{\kern0pt}board\ n\ m{\isacharparenright}{\kern0pt}\ ps\ {\isasymand}\ hd\ ps\ {\isacharequal}{\kern0pt}\ {\isacharparenleft}{\kern0pt}int\ n{\isacharcomma}{\kern0pt}{\isadigit{1}}{\isacharparenright}{\kern0pt}\ {\isasymand}\ last\ ps\ {\isacharequal}{\kern0pt}\ {\isacharparenleft}{\kern0pt}int\ n{\isacharminus}{\kern0pt}{\isadigit{1}}{\isacharcomma}{\kern0pt}int\ m{\isacharminus}{\kern0pt}{\isadigit{1}}{\isacharparenright}{\kern0pt}{\isachardoublequoteclose}\isanewline
%
\isadelimproof
\ \ %
\endisadelimproof
%
\isatagproof
\isacommand{using}\isamarkupfalse%
\ assms\isanewline
\isacommand{proof}\isamarkupfalse%
\ {\isacharminus}{\kern0pt}\isanewline
\ \ \isacommand{obtain}\isamarkupfalse%
\ x\ \isakeyword{where}\ {\isachardoublequoteopen}x\ {\isacharequal}{\kern0pt}\ n\ {\isacharplus}{\kern0pt}\ m{\isachardoublequoteclose}\ \isacommand{by}\isamarkupfalse%
\ auto\isanewline
\ \ \isacommand{then}\isamarkupfalse%
\ \isacommand{show}\isamarkupfalse%
\ {\isacharquery}{\kern0pt}thesis\isanewline
\ \ \ \ \isacommand{using}\isamarkupfalse%
\ assms\isanewline
\ \ \isacommand{proof}\isamarkupfalse%
\ {\isacharparenleft}{\kern0pt}induction\ x\ arbitrary{\isacharcolon}{\kern0pt}\ n\ m\ rule{\isacharcolon}{\kern0pt}\ less{\isacharunderscore}{\kern0pt}induct{\isacharparenright}{\kern0pt}\isanewline
\ \ \ \ \isacommand{case}\isamarkupfalse%
\ {\isacharparenleft}{\kern0pt}less\ x{\isacharparenright}{\kern0pt}\isanewline
\ \ \ \ \isacommand{then}\isamarkupfalse%
\ \isacommand{have}\isamarkupfalse%
\ {\isachardoublequoteopen}m\ {\isacharequal}{\kern0pt}\ {\isadigit{5}}\ {\isasymor}\ m\ {\isacharequal}{\kern0pt}\ {\isadigit{6}}\ {\isasymor}\ m\ {\isacharequal}{\kern0pt}\ {\isadigit{7}}\ {\isasymor}\ m\ {\isacharequal}{\kern0pt}\ {\isadigit{8}}\ {\isasymor}\ m\ {\isacharequal}{\kern0pt}\ {\isadigit{9}}\ {\isasymor}\ m\ {\isasymge}\ {\isadigit{1}}{\isadigit{0}}{\isachardoublequoteclose}\ \isacommand{by}\isamarkupfalse%
\ auto\isanewline
\ \ \ \ \isacommand{then}\isamarkupfalse%
\ \isacommand{show}\isamarkupfalse%
\ {\isacharquery}{\kern0pt}case\isanewline
\ \ \ \ \isacommand{proof}\isamarkupfalse%
\ {\isacharparenleft}{\kern0pt}elim\ disjE{\isacharparenright}{\kern0pt}\isanewline
\ \ \ \ \ \ \isacommand{assume}\isamarkupfalse%
\ {\isacharbrackleft}{\kern0pt}simp{\isacharbrackright}{\kern0pt}{\isacharcolon}{\kern0pt}\ {\isachardoublequoteopen}m\ {\isacharequal}{\kern0pt}\ {\isadigit{5}}{\isachardoublequoteclose}\isanewline
\ \ \ \ \ \ \isacommand{have}\isamarkupfalse%
\ {\isachardoublequoteopen}odd\ n{\isachardoublequoteclose}\ {\isachardoublequoteopen}n\ {\isasymge}\ {\isadigit{5}}{\isachardoublequoteclose}\ \isacommand{using}\isamarkupfalse%
\ less\ \isacommand{by}\isamarkupfalse%
\ auto\isanewline
\ \ \ \ \ \ \isacommand{then}\isamarkupfalse%
\ \isacommand{have}\isamarkupfalse%
\ {\isachardoublequoteopen}n\ {\isacharequal}{\kern0pt}\ {\isadigit{5}}\ {\isasymor}\ n\ {\isacharequal}{\kern0pt}\ {\isadigit{7}}\ {\isasymor}\ n\ {\isacharequal}{\kern0pt}\ {\isadigit{9}}\ {\isasymor}\ n{\isacharminus}{\kern0pt}{\isadigit{5}}\ {\isasymge}\ {\isadigit{5}}{\isachardoublequoteclose}\ \isacommand{by}\isamarkupfalse%
\ presburger\isanewline
\ \ \ \ \ \ \isacommand{then}\isamarkupfalse%
\ \isacommand{show}\isamarkupfalse%
\ {\isacharquery}{\kern0pt}thesis\isanewline
\ \ \ \ \ \ \isacommand{proof}\isamarkupfalse%
\ {\isacharparenleft}{\kern0pt}elim\ disjE{\isacharparenright}{\kern0pt}\isanewline
\ \ \ \ \ \ \ \ \isacommand{assume}\isamarkupfalse%
\ {\isacharbrackleft}{\kern0pt}simp{\isacharbrackright}{\kern0pt}{\isacharcolon}{\kern0pt}\ {\isachardoublequoteopen}n\ {\isacharequal}{\kern0pt}\ {\isadigit{5}}{\isachardoublequoteclose}\isanewline
\ \ \ \ \ \ \ \ \isacommand{let}\isamarkupfalse%
\ {\isacharquery}{\kern0pt}ps{\isacharequal}{\kern0pt}{\isachardoublequoteopen}mirror{\isadigit{1}}\ {\isacharparenleft}{\kern0pt}transpose\ kp{\isadigit{5}}x{\isadigit{5}}ul{\isacharparenright}{\kern0pt}{\isachardoublequoteclose}\isanewline
\ \ \ \ \ \ \ \ \isacommand{have}\isamarkupfalse%
\ kp{\isacharcolon}{\kern0pt}\ {\isachardoublequoteopen}knights{\isacharunderscore}{\kern0pt}path\ {\isacharparenleft}{\kern0pt}board\ n\ m{\isacharparenright}{\kern0pt}\ {\isacharquery}{\kern0pt}ps{\isachardoublequoteclose}\isanewline
\ \ \ \ \ \ \ \ \ \ \isacommand{using}\isamarkupfalse%
\ kp{\isacharunderscore}{\kern0pt}{\isadigit{5}}x{\isadigit{5}}{\isacharunderscore}{\kern0pt}ul\ rot{\isadigit{9}}{\isadigit{0}}{\isacharunderscore}{\kern0pt}knights{\isacharunderscore}{\kern0pt}path\ \isacommand{by}\isamarkupfalse%
\ auto\isanewline
\ \ \ \ \ \ \ \ \isacommand{have}\isamarkupfalse%
\ {\isachardoublequoteopen}hd\ {\isacharquery}{\kern0pt}ps\ {\isacharequal}{\kern0pt}\ {\isacharparenleft}{\kern0pt}int\ n{\isacharcomma}{\kern0pt}{\isadigit{1}}{\isacharparenright}{\kern0pt}{\isachardoublequoteclose}\ {\isachardoublequoteopen}last\ {\isacharquery}{\kern0pt}ps\ {\isacharequal}{\kern0pt}\ {\isacharparenleft}{\kern0pt}int\ n{\isacharminus}{\kern0pt}{\isadigit{1}}{\isacharcomma}{\kern0pt}int\ m{\isacharminus}{\kern0pt}{\isadigit{1}}{\isacharparenright}{\kern0pt}{\isachardoublequoteclose}\isanewline
\ \ \ \ \ \ \ \ \ \ \isacommand{by}\isamarkupfalse%
\ {\isacharparenleft}{\kern0pt}simp\ only{\isacharcolon}{\kern0pt}\ {\isacartoucheopen}m\ {\isacharequal}{\kern0pt}\ {\isadigit{5}}{\isacartoucheclose}\ {\isacartoucheopen}n\ {\isacharequal}{\kern0pt}\ {\isadigit{5}}{\isacartoucheclose}\ {\isacharbar}{\kern0pt}\ eval{\isacharparenright}{\kern0pt}{\isacharplus}{\kern0pt}\isanewline
\ \ \ \ \ \ \ \ \isacommand{then}\isamarkupfalse%
\ \isacommand{show}\isamarkupfalse%
\ {\isacharquery}{\kern0pt}thesis\ \isacommand{using}\isamarkupfalse%
\ kp\ \isacommand{by}\isamarkupfalse%
\ auto\isanewline
\ \ \ \ \ \ \isacommand{next}\isamarkupfalse%
\isanewline
\ \ \ \ \ \ \ \ \isacommand{assume}\isamarkupfalse%
\ {\isacharbrackleft}{\kern0pt}simp{\isacharbrackright}{\kern0pt}{\isacharcolon}{\kern0pt}\ {\isachardoublequoteopen}n\ {\isacharequal}{\kern0pt}\ {\isadigit{7}}{\isachardoublequoteclose}\isanewline
\ \ \ \ \ \ \ \ \isacommand{let}\isamarkupfalse%
\ {\isacharquery}{\kern0pt}ps{\isacharequal}{\kern0pt}{\isachardoublequoteopen}mirror{\isadigit{1}}\ {\isacharparenleft}{\kern0pt}transpose\ kp{\isadigit{5}}x{\isadigit{7}}ul{\isacharparenright}{\kern0pt}{\isachardoublequoteclose}\isanewline
\ \ \ \ \ \ \ \ \isacommand{have}\isamarkupfalse%
\ kp{\isacharcolon}{\kern0pt}\ {\isachardoublequoteopen}knights{\isacharunderscore}{\kern0pt}path\ {\isacharparenleft}{\kern0pt}board\ n\ m{\isacharparenright}{\kern0pt}\ {\isacharquery}{\kern0pt}ps{\isachardoublequoteclose}\isanewline
\ \ \ \ \ \ \ \ \ \ \isacommand{using}\isamarkupfalse%
\ kp{\isacharunderscore}{\kern0pt}{\isadigit{5}}x{\isadigit{7}}{\isacharunderscore}{\kern0pt}ul\ rot{\isadigit{9}}{\isadigit{0}}{\isacharunderscore}{\kern0pt}knights{\isacharunderscore}{\kern0pt}path\ \isacommand{by}\isamarkupfalse%
\ auto\isanewline
\ \ \ \ \ \ \ \ \isacommand{have}\isamarkupfalse%
\ {\isachardoublequoteopen}hd\ {\isacharquery}{\kern0pt}ps\ {\isacharequal}{\kern0pt}\ {\isacharparenleft}{\kern0pt}int\ n{\isacharcomma}{\kern0pt}{\isadigit{1}}{\isacharparenright}{\kern0pt}{\isachardoublequoteclose}\ {\isachardoublequoteopen}last\ {\isacharquery}{\kern0pt}ps\ {\isacharequal}{\kern0pt}\ {\isacharparenleft}{\kern0pt}int\ n{\isacharminus}{\kern0pt}{\isadigit{1}}{\isacharcomma}{\kern0pt}int\ m{\isacharminus}{\kern0pt}{\isadigit{1}}{\isacharparenright}{\kern0pt}{\isachardoublequoteclose}\isanewline
\ \ \ \ \ \ \ \ \ \ \isacommand{by}\isamarkupfalse%
\ {\isacharparenleft}{\kern0pt}simp\ only{\isacharcolon}{\kern0pt}\ {\isacartoucheopen}m\ {\isacharequal}{\kern0pt}\ {\isadigit{5}}{\isacartoucheclose}\ {\isacartoucheopen}n\ {\isacharequal}{\kern0pt}\ {\isadigit{7}}{\isacartoucheclose}\ {\isacharbar}{\kern0pt}\ eval{\isacharparenright}{\kern0pt}{\isacharplus}{\kern0pt}\isanewline
\ \ \ \ \ \ \ \ \isacommand{then}\isamarkupfalse%
\ \isacommand{show}\isamarkupfalse%
\ {\isacharquery}{\kern0pt}thesis\ \isacommand{using}\isamarkupfalse%
\ kp\ \isacommand{by}\isamarkupfalse%
\ auto\isanewline
\ \ \ \ \ \ \isacommand{next}\isamarkupfalse%
\isanewline
\ \ \ \ \ \ \ \ \isacommand{assume}\isamarkupfalse%
\ {\isacharbrackleft}{\kern0pt}simp{\isacharbrackright}{\kern0pt}{\isacharcolon}{\kern0pt}\ {\isachardoublequoteopen}n\ {\isacharequal}{\kern0pt}\ {\isadigit{9}}{\isachardoublequoteclose}\isanewline
\ \ \ \ \ \ \ \ \isacommand{let}\isamarkupfalse%
\ {\isacharquery}{\kern0pt}ps{\isacharequal}{\kern0pt}{\isachardoublequoteopen}mirror{\isadigit{1}}\ {\isacharparenleft}{\kern0pt}transpose\ kp{\isadigit{5}}x{\isadigit{9}}ul{\isacharparenright}{\kern0pt}{\isachardoublequoteclose}\isanewline
\ \ \ \ \ \ \ \ \isacommand{have}\isamarkupfalse%
\ kp{\isacharcolon}{\kern0pt}\ {\isachardoublequoteopen}knights{\isacharunderscore}{\kern0pt}path\ {\isacharparenleft}{\kern0pt}board\ n\ m{\isacharparenright}{\kern0pt}\ {\isacharquery}{\kern0pt}ps{\isachardoublequoteclose}\isanewline
\ \ \ \ \ \ \ \ \ \ \isacommand{using}\isamarkupfalse%
\ kp{\isacharunderscore}{\kern0pt}{\isadigit{5}}x{\isadigit{9}}{\isacharunderscore}{\kern0pt}ul\ rot{\isadigit{9}}{\isadigit{0}}{\isacharunderscore}{\kern0pt}knights{\isacharunderscore}{\kern0pt}path\ \isacommand{by}\isamarkupfalse%
\ auto\isanewline
\ \ \ \ \ \ \ \ \isacommand{have}\isamarkupfalse%
\ {\isachardoublequoteopen}hd\ {\isacharquery}{\kern0pt}ps\ {\isacharequal}{\kern0pt}\ {\isacharparenleft}{\kern0pt}int\ n{\isacharcomma}{\kern0pt}{\isadigit{1}}{\isacharparenright}{\kern0pt}{\isachardoublequoteclose}\ {\isachardoublequoteopen}last\ {\isacharquery}{\kern0pt}ps\ {\isacharequal}{\kern0pt}\ {\isacharparenleft}{\kern0pt}int\ n{\isacharminus}{\kern0pt}{\isadigit{1}}{\isacharcomma}{\kern0pt}int\ m{\isacharminus}{\kern0pt}{\isadigit{1}}{\isacharparenright}{\kern0pt}{\isachardoublequoteclose}\isanewline
\ \ \ \ \ \ \ \ \ \ \isacommand{by}\isamarkupfalse%
\ {\isacharparenleft}{\kern0pt}simp\ only{\isacharcolon}{\kern0pt}\ {\isacartoucheopen}m\ {\isacharequal}{\kern0pt}\ {\isadigit{5}}{\isacartoucheclose}\ {\isacartoucheopen}n\ {\isacharequal}{\kern0pt}\ {\isadigit{9}}{\isacartoucheclose}\ {\isacharbar}{\kern0pt}\ eval{\isacharparenright}{\kern0pt}{\isacharplus}{\kern0pt}\isanewline
\ \ \ \ \ \ \ \ \isacommand{then}\isamarkupfalse%
\ \isacommand{show}\isamarkupfalse%
\ {\isacharquery}{\kern0pt}thesis\ \isacommand{using}\isamarkupfalse%
\ kp\ \isacommand{by}\isamarkupfalse%
\ auto\isanewline
\ \ \ \ \ \ \isacommand{next}\isamarkupfalse%
\isanewline
\ \ \ \ \ \ \ \ \isacommand{let}\isamarkupfalse%
\ {\isacharquery}{\kern0pt}b\isactrlsub {\isadigit{2}}{\isacharequal}{\kern0pt}{\isachardoublequoteopen}board\ m\ {\isacharparenleft}{\kern0pt}n{\isacharminus}{\kern0pt}{\isadigit{5}}{\isacharparenright}{\kern0pt}{\isachardoublequoteclose}\isanewline
\ \ \ \ \ \ \ \ \isacommand{assume}\isamarkupfalse%
\ {\isachardoublequoteopen}n{\isacharminus}{\kern0pt}{\isadigit{5}}\ {\isasymge}\ {\isadigit{5}}{\isachardoublequoteclose}\isanewline
\ \ \ \ \ \ \ \ \isacommand{then}\isamarkupfalse%
\ \isacommand{have}\isamarkupfalse%
\ {\isachardoublequoteopen}{\isasymexists}ps{\isachardot}{\kern0pt}\ knights{\isacharunderscore}{\kern0pt}circuit\ {\isacharquery}{\kern0pt}b\isactrlsub {\isadigit{2}}\ ps{\isachardoublequoteclose}\isanewline
\ \ \ \ \ \ \ \ \isacommand{proof}\isamarkupfalse%
\ {\isacharminus}{\kern0pt}\isanewline
\ \ \ \ \ \ \ \ \ \ \isacommand{have}\isamarkupfalse%
\ {\isachardoublequoteopen}n{\isacharminus}{\kern0pt}{\isadigit{5}}\ {\isacharequal}{\kern0pt}\ {\isadigit{6}}\ {\isasymor}\ n{\isacharminus}{\kern0pt}{\isadigit{5}}\ {\isacharequal}{\kern0pt}\ {\isadigit{8}}\ {\isasymor}\ n{\isacharminus}{\kern0pt}{\isadigit{5}}\ {\isasymge}\ {\isadigit{1}}{\isadigit{0}}{\isachardoublequoteclose}\ \isanewline
\ \ \ \ \ \ \ \ \ \ \ \ \isacommand{using}\isamarkupfalse%
\ {\isacartoucheopen}n{\isacharminus}{\kern0pt}{\isadigit{5}}\ {\isasymge}\ {\isadigit{5}}{\isacartoucheclose}\ less\ \isacommand{by}\isamarkupfalse%
\ presburger\isanewline
\ \ \ \ \ \ \ \ \ \ \isacommand{then}\isamarkupfalse%
\ \isacommand{show}\isamarkupfalse%
\ {\isacharquery}{\kern0pt}thesis\isanewline
\ \ \ \ \ \ \ \ \ \ \isacommand{proof}\isamarkupfalse%
\ {\isacharparenleft}{\kern0pt}elim\ disjE{\isacharparenright}{\kern0pt}\isanewline
\ \ \ \ \ \ \ \ \ \ \ \ \isacommand{assume}\isamarkupfalse%
\ {\isachardoublequoteopen}n{\isacharminus}{\kern0pt}{\isadigit{5}}\ {\isacharequal}{\kern0pt}\ {\isadigit{6}}{\isachardoublequoteclose}\isanewline
\ \ \ \ \ \ \ \ \ \ \ \ \isacommand{then}\isamarkupfalse%
\ \isacommand{obtain}\isamarkupfalse%
\ ps\ \isakeyword{where}\ {\isachardoublequoteopen}knights{\isacharunderscore}{\kern0pt}circuit\ {\isacharparenleft}{\kern0pt}board\ {\isacharparenleft}{\kern0pt}n{\isacharminus}{\kern0pt}{\isadigit{5}}{\isacharparenright}{\kern0pt}\ m{\isacharparenright}{\kern0pt}\ ps{\isachardoublequoteclose}\isanewline
\ \ \ \ \ \ \ \ \ \ \ \ \ \ \isacommand{using}\isamarkupfalse%
\ knights{\isacharunderscore}{\kern0pt}path{\isacharunderscore}{\kern0pt}{\isadigit{6}}xm{\isacharunderscore}{\kern0pt}exists{\isacharbrackleft}{\kern0pt}of\ m{\isacharbrackright}{\kern0pt}\ \isacommand{by}\isamarkupfalse%
\ auto\isanewline
\ \ \ \ \ \ \ \ \ \ \ \ \isacommand{then}\isamarkupfalse%
\ \isacommand{show}\isamarkupfalse%
\ {\isacharquery}{\kern0pt}thesis\ \isanewline
\ \ \ \ \ \ \ \ \ \ \ \ \ \ \isacommand{using}\isamarkupfalse%
\ transpose{\isacharunderscore}{\kern0pt}knights{\isacharunderscore}{\kern0pt}circuit\ \isacommand{by}\isamarkupfalse%
\ auto\isanewline
\ \ \ \ \ \ \ \ \ \ \isacommand{next}\isamarkupfalse%
\isanewline
\ \ \ \ \ \ \ \ \ \ \ \ \isacommand{assume}\isamarkupfalse%
\ {\isachardoublequoteopen}n{\isacharminus}{\kern0pt}{\isadigit{5}}\ {\isacharequal}{\kern0pt}\ {\isadigit{8}}{\isachardoublequoteclose}\isanewline
\ \ \ \ \ \ \ \ \ \ \ \ \isacommand{then}\isamarkupfalse%
\ \isacommand{obtain}\isamarkupfalse%
\ ps\ \isakeyword{where}\ {\isachardoublequoteopen}knights{\isacharunderscore}{\kern0pt}circuit\ {\isacharparenleft}{\kern0pt}board\ {\isacharparenleft}{\kern0pt}n{\isacharminus}{\kern0pt}{\isadigit{5}}{\isacharparenright}{\kern0pt}\ m{\isacharparenright}{\kern0pt}\ ps{\isachardoublequoteclose}\isanewline
\ \ \ \ \ \ \ \ \ \ \ \ \ \ \isacommand{using}\isamarkupfalse%
\ knights{\isacharunderscore}{\kern0pt}path{\isacharunderscore}{\kern0pt}{\isadigit{8}}xm{\isacharunderscore}{\kern0pt}exists{\isacharbrackleft}{\kern0pt}of\ m{\isacharbrackright}{\kern0pt}\ \isacommand{by}\isamarkupfalse%
\ auto\isanewline
\ \ \ \ \ \ \ \ \ \ \ \ \isacommand{then}\isamarkupfalse%
\ \isacommand{show}\isamarkupfalse%
\ {\isacharquery}{\kern0pt}thesis\ \isanewline
\ \ \ \ \ \ \ \ \ \ \ \ \ \ \isacommand{using}\isamarkupfalse%
\ transpose{\isacharunderscore}{\kern0pt}knights{\isacharunderscore}{\kern0pt}circuit\ \isacommand{by}\isamarkupfalse%
\ auto\isanewline
\ \ \ \ \ \ \ \ \ \ \isacommand{next}\isamarkupfalse%
\isanewline
\ \ \ \ \ \ \ \ \ \ \ \ \isacommand{assume}\isamarkupfalse%
\ {\isachardoublequoteopen}n{\isacharminus}{\kern0pt}{\isadigit{5}}\ {\isasymge}\ {\isadigit{1}}{\isadigit{0}}{\isachardoublequoteclose}\isanewline
\ \ \ \ \ \ \ \ \ \ \ \ \isacommand{then}\isamarkupfalse%
\ \isacommand{show}\isamarkupfalse%
\ {\isacharquery}{\kern0pt}thesis\ \isanewline
\ \ \ \ \ \ \ \ \ \ \ \ \ \ \isacommand{using}\isamarkupfalse%
\ less\ less{\isachardot}{\kern0pt}IH{\isacharbrackleft}{\kern0pt}of\ {\isachardoublequoteopen}n{\isacharminus}{\kern0pt}{\isadigit{1}}{\isadigit{0}}{\isacharplus}{\kern0pt}m{\isachardoublequoteclose}\ {\isachardoublequoteopen}n{\isacharminus}{\kern0pt}{\isadigit{1}}{\isadigit{0}}{\isachardoublequoteclose}\ m{\isacharbrackright}{\kern0pt}\isanewline
\ \ \ \ \ \ \ \ \ \ \ \ \ \ \ \ \ \ \ \ knights{\isacharunderscore}{\kern0pt}circuit{\isacharunderscore}{\kern0pt}exists{\isacharunderscore}{\kern0pt}even{\isacharunderscore}{\kern0pt}n{\isacharunderscore}{\kern0pt}gr{\isadigit{1}}{\isadigit{0}}{\isacharbrackleft}{\kern0pt}of\ {\isachardoublequoteopen}n{\isacharminus}{\kern0pt}{\isadigit{5}}{\isachardoublequoteclose}\ m{\isacharbrackright}{\kern0pt}\ \isacommand{by}\isamarkupfalse%
\ auto\isanewline
\ \ \ \ \ \ \ \ \ \ \isacommand{qed}\isamarkupfalse%
\isanewline
\ \ \ \ \ \ \ \ \isacommand{qed}\isamarkupfalse%
\isanewline
\ \ \ \ \ \ \ \ \isacommand{then}\isamarkupfalse%
\ \isacommand{obtain}\isamarkupfalse%
\ ps\isactrlsub {\isadigit{2}}\ \isakeyword{where}\ {\isachardoublequoteopen}knights{\isacharunderscore}{\kern0pt}circuit\ {\isacharquery}{\kern0pt}b\isactrlsub {\isadigit{2}}\ ps\isactrlsub {\isadigit{2}}{\isachardoublequoteclose}\ {\isachardoublequoteopen}hd\ ps\isactrlsub {\isadigit{2}}\ {\isacharequal}{\kern0pt}\ {\isacharparenleft}{\kern0pt}{\isadigit{1}}{\isacharcomma}{\kern0pt}{\isadigit{1}}{\isacharparenright}{\kern0pt}{\isachardoublequoteclose}\ {\isachardoublequoteopen}last\ ps\isactrlsub {\isadigit{2}}\ {\isacharequal}{\kern0pt}\ {\isacharparenleft}{\kern0pt}{\isadigit{3}}{\isacharcomma}{\kern0pt}{\isadigit{2}}{\isacharparenright}{\kern0pt}{\isachardoublequoteclose}\isanewline
\ \ \ \ \ \ \ \ \ \ \isacommand{using}\isamarkupfalse%
\ {\isacartoucheopen}n{\isacharminus}{\kern0pt}{\isadigit{5}}\ {\isasymge}\ {\isadigit{5}}{\isacartoucheclose}\ rotate{\isacharunderscore}{\kern0pt}knights{\isacharunderscore}{\kern0pt}circuit{\isacharbrackleft}{\kern0pt}of\ m\ {\isachardoublequoteopen}n{\isacharminus}{\kern0pt}{\isadigit{5}}{\isachardoublequoteclose}{\isacharbrackright}{\kern0pt}\ \isacommand{by}\isamarkupfalse%
\ auto\isanewline
\ \ \ \ \ \ \ \ \isacommand{then}\isamarkupfalse%
\ \isacommand{have}\isamarkupfalse%
\ rev{\isacharunderscore}{\kern0pt}ps\isactrlsub {\isadigit{2}}{\isacharunderscore}{\kern0pt}prems{\isacharcolon}{\kern0pt}\ {\isachardoublequoteopen}knights{\isacharunderscore}{\kern0pt}path\ {\isacharquery}{\kern0pt}b\isactrlsub {\isadigit{2}}\ {\isacharparenleft}{\kern0pt}rev\ ps\isactrlsub {\isadigit{2}}{\isacharparenright}{\kern0pt}{\isachardoublequoteclose}\ {\isachardoublequoteopen}valid{\isacharunderscore}{\kern0pt}step\ {\isacharparenleft}{\kern0pt}last\ ps\isactrlsub {\isadigit{2}}{\isacharparenright}{\kern0pt}\ {\isacharparenleft}{\kern0pt}hd\ ps\isactrlsub {\isadigit{2}}{\isacharparenright}{\kern0pt}{\isachardoublequoteclose}\isanewline
\ \ \ \ \ \ \ \ \ \ \ \ {\isachardoublequoteopen}hd\ {\isacharparenleft}{\kern0pt}rev\ ps\isactrlsub {\isadigit{2}}{\isacharparenright}{\kern0pt}\ {\isacharequal}{\kern0pt}\ {\isacharparenleft}{\kern0pt}{\isadigit{3}}{\isacharcomma}{\kern0pt}{\isadigit{2}}{\isacharparenright}{\kern0pt}{\isachardoublequoteclose}\ {\isachardoublequoteopen}last\ {\isacharparenleft}{\kern0pt}rev\ ps\isactrlsub {\isadigit{2}}{\isacharparenright}{\kern0pt}\ {\isacharequal}{\kern0pt}\ {\isacharparenleft}{\kern0pt}{\isadigit{1}}{\isacharcomma}{\kern0pt}{\isadigit{1}}{\isacharparenright}{\kern0pt}{\isachardoublequoteclose}\isanewline
\ \ \ \ \ \ \ \ \ \ \isacommand{unfolding}\isamarkupfalse%
\ knights{\isacharunderscore}{\kern0pt}circuit{\isacharunderscore}{\kern0pt}def\ \isacommand{using}\isamarkupfalse%
\ knights{\isacharunderscore}{\kern0pt}path{\isacharunderscore}{\kern0pt}rev\ \isacommand{by}\isamarkupfalse%
\ {\isacharparenleft}{\kern0pt}auto\ simp{\isacharcolon}{\kern0pt}\ hd{\isacharunderscore}{\kern0pt}rev\ last{\isacharunderscore}{\kern0pt}rev{\isacharparenright}{\kern0pt}\isanewline
\isanewline
\ \ \ \ \ \ \ \ \isacommand{let}\isamarkupfalse%
\ {\isacharquery}{\kern0pt}ps\isactrlsub {\isadigit{1}}{\isacharequal}{\kern0pt}{\isachardoublequoteopen}kp{\isadigit{5}}x{\isadigit{5}}ul{\isachardoublequoteclose}\isanewline
\ \ \ \ \ \ \ \ \isacommand{have}\isamarkupfalse%
\ ps\isactrlsub {\isadigit{1}}{\isacharunderscore}{\kern0pt}prems{\isacharcolon}{\kern0pt}\ {\isachardoublequoteopen}knights{\isacharunderscore}{\kern0pt}path\ {\isacharparenleft}{\kern0pt}board\ {\isadigit{5}}\ {\isadigit{5}}{\isacharparenright}{\kern0pt}\ {\isacharquery}{\kern0pt}ps\isactrlsub {\isadigit{1}}{\isachardoublequoteclose}\ {\isachardoublequoteopen}hd\ {\isacharquery}{\kern0pt}ps\isactrlsub {\isadigit{1}}\ {\isacharequal}{\kern0pt}\ {\isacharparenleft}{\kern0pt}{\isadigit{1}}{\isacharcomma}{\kern0pt}{\isadigit{1}}{\isacharparenright}{\kern0pt}{\isachardoublequoteclose}\ {\isachardoublequoteopen}last\ {\isacharquery}{\kern0pt}ps\isactrlsub {\isadigit{1}}\ {\isacharequal}{\kern0pt}\ {\isacharparenleft}{\kern0pt}{\isadigit{4}}{\isacharcomma}{\kern0pt}{\isadigit{2}}{\isacharparenright}{\kern0pt}{\isachardoublequoteclose}\isanewline
\ \ \ \ \ \ \ \ \ \ \isacommand{using}\isamarkupfalse%
\ kp{\isacharunderscore}{\kern0pt}{\isadigit{5}}x{\isadigit{5}}{\isacharunderscore}{\kern0pt}ul\ \isacommand{by}\isamarkupfalse%
\ simp\ eval{\isacharplus}{\kern0pt}\isanewline
\isanewline
\ \ \ \ \ \ \ \ \isacommand{have}\isamarkupfalse%
\ {\isachardoublequoteopen}{\isadigit{1}}{\isadigit{6}}\ {\isacharless}{\kern0pt}\ length\ {\isacharquery}{\kern0pt}ps\isactrlsub {\isadigit{1}}{\isachardoublequoteclose}\ {\isachardoublequoteopen}last\ {\isacharparenleft}{\kern0pt}take\ {\isadigit{1}}{\isadigit{6}}\ {\isacharquery}{\kern0pt}ps\isactrlsub {\isadigit{1}}{\isacharparenright}{\kern0pt}\ {\isacharequal}{\kern0pt}\ {\isacharparenleft}{\kern0pt}{\isadigit{4}}{\isacharcomma}{\kern0pt}{\isadigit{5}}{\isacharparenright}{\kern0pt}{\isachardoublequoteclose}\ {\isachardoublequoteopen}hd\ {\isacharparenleft}{\kern0pt}drop\ {\isadigit{1}}{\isadigit{6}}\ {\isacharquery}{\kern0pt}ps\isactrlsub {\isadigit{1}}{\isacharparenright}{\kern0pt}\ {\isacharequal}{\kern0pt}\ {\isacharparenleft}{\kern0pt}{\isadigit{2}}{\isacharcomma}{\kern0pt}{\isadigit{4}}{\isacharparenright}{\kern0pt}{\isachardoublequoteclose}\ \isacommand{by}\isamarkupfalse%
\ eval{\isacharplus}{\kern0pt}\isanewline
\ \ \ \ \ \ \ \ \isacommand{then}\isamarkupfalse%
\ \isacommand{have}\isamarkupfalse%
\ si{\isacharcolon}{\kern0pt}\ {\isachardoublequoteopen}step{\isacharunderscore}{\kern0pt}in\ {\isacharquery}{\kern0pt}ps\isactrlsub {\isadigit{1}}\ {\isacharparenleft}{\kern0pt}{\isadigit{4}}{\isacharcomma}{\kern0pt}{\isadigit{5}}{\isacharparenright}{\kern0pt}\ {\isacharparenleft}{\kern0pt}{\isadigit{2}}{\isacharcomma}{\kern0pt}{\isadigit{4}}{\isacharparenright}{\kern0pt}{\isachardoublequoteclose}\isanewline
\ \ \ \ \ \ \ \ \ \ \isacommand{unfolding}\isamarkupfalse%
\ step{\isacharunderscore}{\kern0pt}in{\isacharunderscore}{\kern0pt}def\ \isacommand{using}\isamarkupfalse%
\ zero{\isacharunderscore}{\kern0pt}less{\isacharunderscore}{\kern0pt}numeral\ \isacommand{by}\isamarkupfalse%
\ blast\isanewline
\isanewline
\ \ \ \ \ \ \ \ \isacommand{have}\isamarkupfalse%
\ vs{\isacharcolon}{\kern0pt}\ {\isachardoublequoteopen}valid{\isacharunderscore}{\kern0pt}step\ {\isacharparenleft}{\kern0pt}{\isadigit{4}}{\isacharcomma}{\kern0pt}{\isadigit{5}}{\isacharparenright}{\kern0pt}\ {\isacharparenleft}{\kern0pt}{\isadigit{3}}{\isacharcomma}{\kern0pt}int\ {\isadigit{5}}{\isacharplus}{\kern0pt}{\isadigit{2}}{\isacharparenright}{\kern0pt}{\isachardoublequoteclose}\ {\isachardoublequoteopen}valid{\isacharunderscore}{\kern0pt}step\ {\isacharparenleft}{\kern0pt}{\isadigit{1}}{\isacharcomma}{\kern0pt}int\ {\isadigit{5}}{\isacharplus}{\kern0pt}{\isadigit{1}}{\isacharparenright}{\kern0pt}\ {\isacharparenleft}{\kern0pt}{\isadigit{2}}{\isacharcomma}{\kern0pt}{\isadigit{4}}{\isacharparenright}{\kern0pt}{\isachardoublequoteclose}\isanewline
\ \ \ \ \ \ \ \ \ \ \isacommand{unfolding}\isamarkupfalse%
\ valid{\isacharunderscore}{\kern0pt}step{\isacharunderscore}{\kern0pt}def\ \isacommand{by}\isamarkupfalse%
\ auto\isanewline
\isanewline
\ \ \ \ \ \ \ \ \isacommand{obtain}\isamarkupfalse%
\ ps\ \isakeyword{where}\ {\isachardoublequoteopen}knights{\isacharunderscore}{\kern0pt}path\ {\isacharparenleft}{\kern0pt}board\ m\ n{\isacharparenright}{\kern0pt}\ ps{\isachardoublequoteclose}\ {\isachardoublequoteopen}hd\ ps\ {\isacharequal}{\kern0pt}\ {\isacharparenleft}{\kern0pt}{\isadigit{1}}{\isacharcomma}{\kern0pt}{\isadigit{1}}{\isacharparenright}{\kern0pt}{\isachardoublequoteclose}\ {\isachardoublequoteopen}last\ ps\ {\isacharequal}{\kern0pt}\ {\isacharparenleft}{\kern0pt}{\isadigit{4}}{\isacharcomma}{\kern0pt}{\isadigit{2}}{\isacharparenright}{\kern0pt}{\isachardoublequoteclose}\isanewline
\ \ \ \ \ \ \ \ \ \ \isacommand{using}\isamarkupfalse%
\ {\isacartoucheopen}n{\isacharminus}{\kern0pt}{\isadigit{5}}\ {\isasymge}\ {\isadigit{5}}{\isacartoucheclose}\ ps\isactrlsub {\isadigit{1}}{\isacharunderscore}{\kern0pt}prems\ rev{\isacharunderscore}{\kern0pt}ps\isactrlsub {\isadigit{2}}{\isacharunderscore}{\kern0pt}prems\ si\ vs\isanewline
\ \ \ \ \ \ \ \ \ \ \ \ \ \ knights{\isacharunderscore}{\kern0pt}path{\isacharunderscore}{\kern0pt}split{\isacharunderscore}{\kern0pt}concat{\isacharbrackleft}{\kern0pt}of\ {\isadigit{5}}\ {\isadigit{5}}\ {\isacharquery}{\kern0pt}ps\isactrlsub {\isadigit{1}}\ {\isachardoublequoteopen}n{\isacharminus}{\kern0pt}{\isadigit{5}}{\isachardoublequoteclose}\ {\isachardoublequoteopen}rev\ ps\isactrlsub {\isadigit{2}}{\isachardoublequoteclose}\ {\isachardoublequoteopen}{\isacharparenleft}{\kern0pt}{\isadigit{4}}{\isacharcomma}{\kern0pt}{\isadigit{5}}{\isacharparenright}{\kern0pt}{\isachardoublequoteclose}\ {\isachardoublequoteopen}{\isacharparenleft}{\kern0pt}{\isadigit{2}}{\isacharcomma}{\kern0pt}{\isadigit{4}}{\isacharparenright}{\kern0pt}{\isachardoublequoteclose}{\isacharbrackright}{\kern0pt}\ \isacommand{by}\isamarkupfalse%
\ auto\isanewline
\ \ \ \ \ \ \ \ \isacommand{then}\isamarkupfalse%
\ \isacommand{show}\isamarkupfalse%
\ {\isacharquery}{\kern0pt}thesis\isanewline
\ \ \ \ \ \ \ \ \ \ \isacommand{using}\isamarkupfalse%
\ rot{\isadigit{9}}{\isadigit{0}}{\isacharunderscore}{\kern0pt}knights{\isacharunderscore}{\kern0pt}path\ hd{\isacharunderscore}{\kern0pt}rot{\isadigit{9}}{\isadigit{0}}{\isacharunderscore}{\kern0pt}knights{\isacharunderscore}{\kern0pt}path\ last{\isacharunderscore}{\kern0pt}rot{\isadigit{9}}{\isadigit{0}}{\isacharunderscore}{\kern0pt}knights{\isacharunderscore}{\kern0pt}path\ \isacommand{by}\isamarkupfalse%
\ fastforce\isanewline
\ \ \ \ \ \ \isacommand{qed}\isamarkupfalse%
\isanewline
\ \ \ \ \isacommand{next}\isamarkupfalse%
\isanewline
\ \ \ \ \ \ \isacommand{assume}\isamarkupfalse%
\ {\isacharbrackleft}{\kern0pt}simp{\isacharbrackright}{\kern0pt}{\isacharcolon}{\kern0pt}\ {\isachardoublequoteopen}m\ {\isacharequal}{\kern0pt}\ {\isadigit{6}}{\isachardoublequoteclose}\isanewline
\ \ \ \ \ \ \isacommand{then}\isamarkupfalse%
\ \isacommand{obtain}\isamarkupfalse%
\ ps\ \isakeyword{where}\ \isanewline
\ \ \ \ \ \ \ \ \ \ ps{\isacharunderscore}{\kern0pt}prems{\isacharcolon}{\kern0pt}\ {\isachardoublequoteopen}knights{\isacharunderscore}{\kern0pt}path\ {\isacharparenleft}{\kern0pt}board\ m\ n{\isacharparenright}{\kern0pt}\ ps{\isachardoublequoteclose}\ {\isachardoublequoteopen}hd\ ps\ {\isacharequal}{\kern0pt}\ {\isacharparenleft}{\kern0pt}{\isadigit{1}}{\isacharcomma}{\kern0pt}{\isadigit{1}}{\isacharparenright}{\kern0pt}{\isachardoublequoteclose}\ {\isachardoublequoteopen}last\ ps\ {\isacharequal}{\kern0pt}\ {\isacharparenleft}{\kern0pt}int\ m{\isacharminus}{\kern0pt}{\isadigit{1}}{\isacharcomma}{\kern0pt}{\isadigit{2}}{\isacharparenright}{\kern0pt}{\isachardoublequoteclose}\isanewline
\ \ \ \ \ \ \ \ \isacommand{using}\isamarkupfalse%
\ less\ knights{\isacharunderscore}{\kern0pt}path{\isacharunderscore}{\kern0pt}{\isadigit{6}}xm{\isacharunderscore}{\kern0pt}exists{\isacharbrackleft}{\kern0pt}of\ n{\isacharbrackright}{\kern0pt}\ \isacommand{by}\isamarkupfalse%
\ auto\isanewline
\ \ \ \ \ \ \isacommand{let}\isamarkupfalse%
\ {\isacharquery}{\kern0pt}ps{\isacharprime}{\kern0pt}{\isacharequal}{\kern0pt}{\isachardoublequoteopen}mirror{\isadigit{1}}\ {\isacharparenleft}{\kern0pt}transpose\ ps{\isacharparenright}{\kern0pt}{\isachardoublequoteclose}\isanewline
\ \ \ \ \ \ \isacommand{have}\isamarkupfalse%
\ {\isachardoublequoteopen}knights{\isacharunderscore}{\kern0pt}path\ {\isacharparenleft}{\kern0pt}board\ n\ m{\isacharparenright}{\kern0pt}\ {\isacharquery}{\kern0pt}ps{\isacharprime}{\kern0pt}{\isachardoublequoteclose}\ {\isachardoublequoteopen}hd\ {\isacharquery}{\kern0pt}ps{\isacharprime}{\kern0pt}\ {\isacharequal}{\kern0pt}\ {\isacharparenleft}{\kern0pt}int\ n{\isacharcomma}{\kern0pt}{\isadigit{1}}{\isacharparenright}{\kern0pt}{\isachardoublequoteclose}\ {\isachardoublequoteopen}last\ {\isacharquery}{\kern0pt}ps{\isacharprime}{\kern0pt}\ {\isacharequal}{\kern0pt}\ {\isacharparenleft}{\kern0pt}int\ n{\isacharminus}{\kern0pt}{\isadigit{1}}{\isacharcomma}{\kern0pt}int\ m{\isacharminus}{\kern0pt}{\isadigit{1}}{\isacharparenright}{\kern0pt}{\isachardoublequoteclose}\isanewline
\ \ \ \ \ \ \ \ \isacommand{using}\isamarkupfalse%
\ ps{\isacharunderscore}{\kern0pt}prems\ rot{\isadigit{9}}{\isadigit{0}}{\isacharunderscore}{\kern0pt}knights{\isacharunderscore}{\kern0pt}path\ hd{\isacharunderscore}{\kern0pt}rot{\isadigit{9}}{\isadigit{0}}{\isacharunderscore}{\kern0pt}knights{\isacharunderscore}{\kern0pt}path\ last{\isacharunderscore}{\kern0pt}rot{\isadigit{9}}{\isadigit{0}}{\isacharunderscore}{\kern0pt}knights{\isacharunderscore}{\kern0pt}path\ \isacommand{by}\isamarkupfalse%
\ auto\isanewline
\ \ \ \ \ \ \isacommand{then}\isamarkupfalse%
\ \isacommand{show}\isamarkupfalse%
\ {\isacharquery}{\kern0pt}thesis\ \isacommand{by}\isamarkupfalse%
\ auto\isanewline
\ \ \ \ \isacommand{next}\isamarkupfalse%
\isanewline
\ \ \ \ \ \ \isacommand{assume}\isamarkupfalse%
\ {\isacharbrackleft}{\kern0pt}simp{\isacharbrackright}{\kern0pt}{\isacharcolon}{\kern0pt}\ {\isachardoublequoteopen}m\ {\isacharequal}{\kern0pt}\ {\isadigit{7}}{\isachardoublequoteclose}\isanewline
\ \ \ \ \ \ \isacommand{have}\isamarkupfalse%
\ {\isachardoublequoteopen}odd\ n{\isachardoublequoteclose}\ {\isachardoublequoteopen}n\ {\isasymge}\ {\isadigit{5}}{\isachardoublequoteclose}\ \isacommand{using}\isamarkupfalse%
\ less\ \isacommand{by}\isamarkupfalse%
\ auto\isanewline
\ \ \ \ \ \ \isacommand{then}\isamarkupfalse%
\ \isacommand{have}\isamarkupfalse%
\ {\isachardoublequoteopen}n\ {\isacharequal}{\kern0pt}\ {\isadigit{5}}\ {\isasymor}\ n\ {\isacharequal}{\kern0pt}\ {\isadigit{7}}\ {\isasymor}\ n\ {\isacharequal}{\kern0pt}\ {\isadigit{9}}\ {\isasymor}\ n{\isacharminus}{\kern0pt}{\isadigit{5}}\ {\isasymge}\ {\isadigit{5}}{\isachardoublequoteclose}\ \isacommand{by}\isamarkupfalse%
\ presburger\isanewline
\ \ \ \ \ \ \isacommand{then}\isamarkupfalse%
\ \isacommand{show}\isamarkupfalse%
\ {\isacharquery}{\kern0pt}thesis\isanewline
\ \ \ \ \ \ \isacommand{proof}\isamarkupfalse%
\ {\isacharparenleft}{\kern0pt}elim\ disjE{\isacharparenright}{\kern0pt}\isanewline
\ \ \ \ \ \ \ \ \isacommand{assume}\isamarkupfalse%
\ {\isacharbrackleft}{\kern0pt}simp{\isacharbrackright}{\kern0pt}{\isacharcolon}{\kern0pt}\ {\isachardoublequoteopen}n\ {\isacharequal}{\kern0pt}\ {\isadigit{5}}{\isachardoublequoteclose}\isanewline
\ \ \ \ \ \ \ \ \isacommand{let}\isamarkupfalse%
\ {\isacharquery}{\kern0pt}ps{\isacharequal}{\kern0pt}{\isachardoublequoteopen}mirror{\isadigit{1}}\ kp{\isadigit{5}}x{\isadigit{7}}lr{\isachardoublequoteclose}\isanewline
\ \ \ \ \ \ \ \ \isacommand{have}\isamarkupfalse%
\ kp{\isacharcolon}{\kern0pt}\ {\isachardoublequoteopen}knights{\isacharunderscore}{\kern0pt}path\ {\isacharparenleft}{\kern0pt}board\ n\ m{\isacharparenright}{\kern0pt}\ {\isacharquery}{\kern0pt}ps{\isachardoublequoteclose}\isanewline
\ \ \ \ \ \ \ \ \ \ \isacommand{using}\isamarkupfalse%
\ kp{\isacharunderscore}{\kern0pt}{\isadigit{5}}x{\isadigit{7}}{\isacharunderscore}{\kern0pt}lr\ mirror{\isadigit{1}}{\isacharunderscore}{\kern0pt}knights{\isacharunderscore}{\kern0pt}path\ \isacommand{by}\isamarkupfalse%
\ auto\isanewline
\ \ \ \ \ \ \ \ \isacommand{have}\isamarkupfalse%
\ {\isachardoublequoteopen}hd\ {\isacharquery}{\kern0pt}ps\ {\isacharequal}{\kern0pt}\ {\isacharparenleft}{\kern0pt}int\ n{\isacharcomma}{\kern0pt}{\isadigit{1}}{\isacharparenright}{\kern0pt}{\isachardoublequoteclose}\ {\isachardoublequoteopen}last\ {\isacharquery}{\kern0pt}ps\ {\isacharequal}{\kern0pt}\ {\isacharparenleft}{\kern0pt}int\ n{\isacharminus}{\kern0pt}{\isadigit{1}}{\isacharcomma}{\kern0pt}int\ m{\isacharminus}{\kern0pt}{\isadigit{1}}{\isacharparenright}{\kern0pt}{\isachardoublequoteclose}\isanewline
\ \ \ \ \ \ \ \ \ \ \isacommand{by}\isamarkupfalse%
\ {\isacharparenleft}{\kern0pt}simp\ only{\isacharcolon}{\kern0pt}\ {\isacartoucheopen}m\ {\isacharequal}{\kern0pt}\ {\isadigit{7}}{\isacartoucheclose}\ {\isacartoucheopen}n\ {\isacharequal}{\kern0pt}\ {\isadigit{5}}{\isacartoucheclose}\ {\isacharbar}{\kern0pt}\ eval{\isacharparenright}{\kern0pt}{\isacharplus}{\kern0pt}\isanewline
\ \ \ \ \ \ \ \ \isacommand{then}\isamarkupfalse%
\ \isacommand{show}\isamarkupfalse%
\ {\isacharquery}{\kern0pt}thesis\ \isacommand{using}\isamarkupfalse%
\ kp\ \isacommand{by}\isamarkupfalse%
\ auto\isanewline
\ \ \ \ \ \ \isacommand{next}\isamarkupfalse%
\isanewline
\ \ \ \ \ \ \ \ \isacommand{assume}\isamarkupfalse%
\ {\isacharbrackleft}{\kern0pt}simp{\isacharbrackright}{\kern0pt}{\isacharcolon}{\kern0pt}\ {\isachardoublequoteopen}n\ {\isacharequal}{\kern0pt}\ {\isadigit{7}}{\isachardoublequoteclose}\isanewline
\ \ \ \ \ \ \ \ \isacommand{let}\isamarkupfalse%
\ {\isacharquery}{\kern0pt}ps{\isacharequal}{\kern0pt}{\isachardoublequoteopen}mirror{\isadigit{1}}\ {\isacharparenleft}{\kern0pt}transpose\ kp{\isadigit{7}}x{\isadigit{7}}ul{\isacharparenright}{\kern0pt}{\isachardoublequoteclose}\isanewline
\ \ \ \ \ \ \ \ \isacommand{have}\isamarkupfalse%
\ kp{\isacharcolon}{\kern0pt}\ {\isachardoublequoteopen}knights{\isacharunderscore}{\kern0pt}path\ {\isacharparenleft}{\kern0pt}board\ n\ m{\isacharparenright}{\kern0pt}\ {\isacharquery}{\kern0pt}ps{\isachardoublequoteclose}\isanewline
\ \ \ \ \ \ \ \ \ \ \isacommand{using}\isamarkupfalse%
\ kp{\isacharunderscore}{\kern0pt}{\isadigit{7}}x{\isadigit{7}}{\isacharunderscore}{\kern0pt}ul\ rot{\isadigit{9}}{\isadigit{0}}{\isacharunderscore}{\kern0pt}knights{\isacharunderscore}{\kern0pt}path\ \isacommand{by}\isamarkupfalse%
\ auto\isanewline
\ \ \ \ \ \ \ \ \isacommand{have}\isamarkupfalse%
\ {\isachardoublequoteopen}hd\ {\isacharquery}{\kern0pt}ps\ {\isacharequal}{\kern0pt}\ {\isacharparenleft}{\kern0pt}int\ n{\isacharcomma}{\kern0pt}{\isadigit{1}}{\isacharparenright}{\kern0pt}{\isachardoublequoteclose}\ {\isachardoublequoteopen}last\ {\isacharquery}{\kern0pt}ps\ {\isacharequal}{\kern0pt}\ {\isacharparenleft}{\kern0pt}int\ n{\isacharminus}{\kern0pt}{\isadigit{1}}{\isacharcomma}{\kern0pt}int\ m{\isacharminus}{\kern0pt}{\isadigit{1}}{\isacharparenright}{\kern0pt}{\isachardoublequoteclose}\isanewline
\ \ \ \ \ \ \ \ \ \ \isacommand{by}\isamarkupfalse%
\ {\isacharparenleft}{\kern0pt}simp\ only{\isacharcolon}{\kern0pt}\ {\isacartoucheopen}m\ {\isacharequal}{\kern0pt}\ {\isadigit{7}}{\isacartoucheclose}\ {\isacartoucheopen}n\ {\isacharequal}{\kern0pt}\ {\isadigit{7}}{\isacartoucheclose}\ {\isacharbar}{\kern0pt}\ eval{\isacharparenright}{\kern0pt}{\isacharplus}{\kern0pt}\isanewline
\ \ \ \ \ \ \ \ \isacommand{then}\isamarkupfalse%
\ \isacommand{show}\isamarkupfalse%
\ {\isacharquery}{\kern0pt}thesis\ \isacommand{using}\isamarkupfalse%
\ kp\ \isacommand{by}\isamarkupfalse%
\ auto\isanewline
\ \ \ \ \ \ \isacommand{next}\isamarkupfalse%
\isanewline
\ \ \ \ \ \ \ \ \isacommand{assume}\isamarkupfalse%
\ {\isacharbrackleft}{\kern0pt}simp{\isacharbrackright}{\kern0pt}{\isacharcolon}{\kern0pt}\ {\isachardoublequoteopen}n\ {\isacharequal}{\kern0pt}\ {\isadigit{9}}{\isachardoublequoteclose}\isanewline
\ \ \ \ \ \ \ \ \isacommand{let}\isamarkupfalse%
\ {\isacharquery}{\kern0pt}ps{\isacharequal}{\kern0pt}{\isachardoublequoteopen}mirror{\isadigit{1}}\ {\isacharparenleft}{\kern0pt}transpose\ kp{\isadigit{7}}x{\isadigit{9}}ul{\isacharparenright}{\kern0pt}{\isachardoublequoteclose}\isanewline
\ \ \ \ \ \ \ \ \isacommand{have}\isamarkupfalse%
\ kp{\isacharcolon}{\kern0pt}\ {\isachardoublequoteopen}knights{\isacharunderscore}{\kern0pt}path\ {\isacharparenleft}{\kern0pt}board\ n\ m{\isacharparenright}{\kern0pt}\ {\isacharquery}{\kern0pt}ps{\isachardoublequoteclose}\isanewline
\ \ \ \ \ \ \ \ \ \ \isacommand{using}\isamarkupfalse%
\ kp{\isacharunderscore}{\kern0pt}{\isadigit{7}}x{\isadigit{9}}{\isacharunderscore}{\kern0pt}ul\ rot{\isadigit{9}}{\isadigit{0}}{\isacharunderscore}{\kern0pt}knights{\isacharunderscore}{\kern0pt}path\ \isacommand{by}\isamarkupfalse%
\ auto\isanewline
\ \ \ \ \ \ \ \ \isacommand{have}\isamarkupfalse%
\ {\isachardoublequoteopen}hd\ {\isacharquery}{\kern0pt}ps\ {\isacharequal}{\kern0pt}\ {\isacharparenleft}{\kern0pt}int\ n{\isacharcomma}{\kern0pt}{\isadigit{1}}{\isacharparenright}{\kern0pt}{\isachardoublequoteclose}\ {\isachardoublequoteopen}last\ {\isacharquery}{\kern0pt}ps\ {\isacharequal}{\kern0pt}\ {\isacharparenleft}{\kern0pt}int\ n{\isacharminus}{\kern0pt}{\isadigit{1}}{\isacharcomma}{\kern0pt}int\ m{\isacharminus}{\kern0pt}{\isadigit{1}}{\isacharparenright}{\kern0pt}{\isachardoublequoteclose}\isanewline
\ \ \ \ \ \ \ \ \ \ \isacommand{by}\isamarkupfalse%
\ {\isacharparenleft}{\kern0pt}simp\ only{\isacharcolon}{\kern0pt}\ {\isacartoucheopen}m\ {\isacharequal}{\kern0pt}\ {\isadigit{7}}{\isacartoucheclose}\ {\isacartoucheopen}n\ {\isacharequal}{\kern0pt}\ {\isadigit{9}}{\isacartoucheclose}\ {\isacharbar}{\kern0pt}\ eval{\isacharparenright}{\kern0pt}{\isacharplus}{\kern0pt}\isanewline
\ \ \ \ \ \ \ \ \isacommand{then}\isamarkupfalse%
\ \isacommand{show}\isamarkupfalse%
\ {\isacharquery}{\kern0pt}thesis\ \isacommand{using}\isamarkupfalse%
\ kp\ \isacommand{by}\isamarkupfalse%
\ auto\isanewline
\ \ \ \ \ \ \isacommand{next}\isamarkupfalse%
\isanewline
\ \ \ \ \ \ \ \ \isacommand{let}\isamarkupfalse%
\ {\isacharquery}{\kern0pt}b\isactrlsub {\isadigit{2}}{\isacharequal}{\kern0pt}{\isachardoublequoteopen}board\ m\ {\isacharparenleft}{\kern0pt}n{\isacharminus}{\kern0pt}{\isadigit{5}}{\isacharparenright}{\kern0pt}{\isachardoublequoteclose}\isanewline
\ \ \ \ \ \ \ \ \isacommand{let}\isamarkupfalse%
\ {\isacharquery}{\kern0pt}b\isactrlsub {\isadigit{2}}T{\isacharequal}{\kern0pt}{\isachardoublequoteopen}board\ {\isacharparenleft}{\kern0pt}n{\isacharminus}{\kern0pt}{\isadigit{5}}{\isacharparenright}{\kern0pt}\ m{\isachardoublequoteclose}\isanewline
\ \ \ \ \ \ \ \ \isacommand{assume}\isamarkupfalse%
\ {\isachardoublequoteopen}n{\isacharminus}{\kern0pt}{\isadigit{5}}\ {\isasymge}\ {\isadigit{5}}{\isachardoublequoteclose}\isanewline
\ \ \ \ \ \ \ \ \isacommand{then}\isamarkupfalse%
\ \isacommand{have}\isamarkupfalse%
\ {\isachardoublequoteopen}{\isasymexists}ps{\isachardot}{\kern0pt}\ knights{\isacharunderscore}{\kern0pt}circuit\ {\isacharquery}{\kern0pt}b\isactrlsub {\isadigit{2}}\ ps{\isachardoublequoteclose}\isanewline
\ \ \ \ \ \ \ \ \isacommand{proof}\isamarkupfalse%
\ {\isacharminus}{\kern0pt}\isanewline
\ \ \ \ \ \ \ \ \ \ \isacommand{have}\isamarkupfalse%
\ {\isachardoublequoteopen}n{\isacharminus}{\kern0pt}{\isadigit{5}}\ {\isacharequal}{\kern0pt}\ {\isadigit{6}}\ {\isasymor}\ n{\isacharminus}{\kern0pt}{\isadigit{5}}\ {\isacharequal}{\kern0pt}\ {\isadigit{8}}\ {\isasymor}\ n{\isacharminus}{\kern0pt}{\isadigit{5}}\ {\isasymge}\ {\isadigit{1}}{\isadigit{0}}{\isachardoublequoteclose}\ \isanewline
\ \ \ \ \ \ \ \ \ \ \ \ \isacommand{using}\isamarkupfalse%
\ {\isacartoucheopen}n{\isacharminus}{\kern0pt}{\isadigit{5}}\ {\isasymge}\ {\isadigit{5}}{\isacartoucheclose}\ less\ \isacommand{by}\isamarkupfalse%
\ presburger\isanewline
\ \ \ \ \ \ \ \ \ \ \isacommand{then}\isamarkupfalse%
\ \isacommand{show}\isamarkupfalse%
\ {\isacharquery}{\kern0pt}thesis\isanewline
\ \ \ \ \ \ \ \ \ \ \isacommand{proof}\isamarkupfalse%
\ {\isacharparenleft}{\kern0pt}elim\ disjE{\isacharparenright}{\kern0pt}\isanewline
\ \ \ \ \ \ \ \ \ \ \ \ \isacommand{assume}\isamarkupfalse%
\ {\isachardoublequoteopen}n{\isacharminus}{\kern0pt}{\isadigit{5}}\ {\isacharequal}{\kern0pt}\ {\isadigit{6}}{\isachardoublequoteclose}\isanewline
\ \ \ \ \ \ \ \ \ \ \ \ \isacommand{then}\isamarkupfalse%
\ \isacommand{obtain}\isamarkupfalse%
\ ps\ \isakeyword{where}\ {\isachardoublequoteopen}knights{\isacharunderscore}{\kern0pt}circuit\ {\isacharparenleft}{\kern0pt}board\ {\isacharparenleft}{\kern0pt}n{\isacharminus}{\kern0pt}{\isadigit{5}}{\isacharparenright}{\kern0pt}\ m{\isacharparenright}{\kern0pt}\ ps{\isachardoublequoteclose}\isanewline
\ \ \ \ \ \ \ \ \ \ \ \ \ \ \isacommand{using}\isamarkupfalse%
\ knights{\isacharunderscore}{\kern0pt}path{\isacharunderscore}{\kern0pt}{\isadigit{6}}xm{\isacharunderscore}{\kern0pt}exists{\isacharbrackleft}{\kern0pt}of\ m{\isacharbrackright}{\kern0pt}\ \isacommand{by}\isamarkupfalse%
\ auto\isanewline
\ \ \ \ \ \ \ \ \ \ \ \ \isacommand{then}\isamarkupfalse%
\ \isacommand{show}\isamarkupfalse%
\ {\isacharquery}{\kern0pt}thesis\ \isanewline
\ \ \ \ \ \ \ \ \ \ \ \ \ \ \isacommand{using}\isamarkupfalse%
\ transpose{\isacharunderscore}{\kern0pt}knights{\isacharunderscore}{\kern0pt}circuit\ \isacommand{by}\isamarkupfalse%
\ auto\isanewline
\ \ \ \ \ \ \ \ \ \ \isacommand{next}\isamarkupfalse%
\isanewline
\ \ \ \ \ \ \ \ \ \ \ \ \isacommand{assume}\isamarkupfalse%
\ {\isachardoublequoteopen}n{\isacharminus}{\kern0pt}{\isadigit{5}}\ {\isacharequal}{\kern0pt}\ {\isadigit{8}}{\isachardoublequoteclose}\isanewline
\ \ \ \ \ \ \ \ \ \ \ \ \isacommand{then}\isamarkupfalse%
\ \isacommand{obtain}\isamarkupfalse%
\ ps\ \isakeyword{where}\ {\isachardoublequoteopen}knights{\isacharunderscore}{\kern0pt}circuit\ {\isacharparenleft}{\kern0pt}board\ {\isacharparenleft}{\kern0pt}n{\isacharminus}{\kern0pt}{\isadigit{5}}{\isacharparenright}{\kern0pt}\ m{\isacharparenright}{\kern0pt}\ ps{\isachardoublequoteclose}\isanewline
\ \ \ \ \ \ \ \ \ \ \ \ \ \ \isacommand{using}\isamarkupfalse%
\ knights{\isacharunderscore}{\kern0pt}path{\isacharunderscore}{\kern0pt}{\isadigit{8}}xm{\isacharunderscore}{\kern0pt}exists{\isacharbrackleft}{\kern0pt}of\ m{\isacharbrackright}{\kern0pt}\ \isacommand{by}\isamarkupfalse%
\ auto\isanewline
\ \ \ \ \ \ \ \ \ \ \ \ \isacommand{then}\isamarkupfalse%
\ \isacommand{show}\isamarkupfalse%
\ {\isacharquery}{\kern0pt}thesis\ \isanewline
\ \ \ \ \ \ \ \ \ \ \ \ \ \ \isacommand{using}\isamarkupfalse%
\ transpose{\isacharunderscore}{\kern0pt}knights{\isacharunderscore}{\kern0pt}circuit\ \isacommand{by}\isamarkupfalse%
\ auto\isanewline
\ \ \ \ \ \ \ \ \ \ \isacommand{next}\isamarkupfalse%
\isanewline
\ \ \ \ \ \ \ \ \ \ \ \ \isacommand{assume}\isamarkupfalse%
\ {\isachardoublequoteopen}n{\isacharminus}{\kern0pt}{\isadigit{5}}\ {\isasymge}\ {\isadigit{1}}{\isadigit{0}}{\isachardoublequoteclose}\isanewline
\ \ \ \ \ \ \ \ \ \ \ \ \isacommand{then}\isamarkupfalse%
\ \isacommand{show}\isamarkupfalse%
\ {\isacharquery}{\kern0pt}thesis\ \isanewline
\ \ \ \ \ \ \ \ \ \ \ \ \ \ \isacommand{using}\isamarkupfalse%
\ less\ less{\isachardot}{\kern0pt}IH{\isacharbrackleft}{\kern0pt}of\ {\isachardoublequoteopen}n{\isacharminus}{\kern0pt}{\isadigit{1}}{\isadigit{0}}{\isacharplus}{\kern0pt}m{\isachardoublequoteclose}\ {\isachardoublequoteopen}n{\isacharminus}{\kern0pt}{\isadigit{1}}{\isadigit{0}}{\isachardoublequoteclose}\ m{\isacharbrackright}{\kern0pt}\isanewline
\ \ \ \ \ \ \ \ \ \ \ \ \ \ \ \ \ \ \ \ knights{\isacharunderscore}{\kern0pt}circuit{\isacharunderscore}{\kern0pt}exists{\isacharunderscore}{\kern0pt}even{\isacharunderscore}{\kern0pt}n{\isacharunderscore}{\kern0pt}gr{\isadigit{1}}{\isadigit{0}}{\isacharbrackleft}{\kern0pt}of\ {\isachardoublequoteopen}n{\isacharminus}{\kern0pt}{\isadigit{5}}{\isachardoublequoteclose}\ m{\isacharbrackright}{\kern0pt}\ \isacommand{by}\isamarkupfalse%
\ auto\isanewline
\ \ \ \ \ \ \ \ \ \ \isacommand{qed}\isamarkupfalse%
\isanewline
\ \ \ \ \ \ \ \ \isacommand{qed}\isamarkupfalse%
\isanewline
\ \ \ \ \ \ \ \ \isacommand{then}\isamarkupfalse%
\ \isacommand{obtain}\isamarkupfalse%
\ ps\isactrlsub {\isadigit{2}}\ \isakeyword{where}\ ps\isactrlsub {\isadigit{2}}{\isacharunderscore}{\kern0pt}prems{\isacharcolon}{\kern0pt}\ {\isachardoublequoteopen}knights{\isacharunderscore}{\kern0pt}circuit\ {\isacharquery}{\kern0pt}b\isactrlsub {\isadigit{2}}\ ps\isactrlsub {\isadigit{2}}{\isachardoublequoteclose}\ {\isachardoublequoteopen}hd\ ps\isactrlsub {\isadigit{2}}\ {\isacharequal}{\kern0pt}\ {\isacharparenleft}{\kern0pt}{\isadigit{1}}{\isacharcomma}{\kern0pt}{\isadigit{1}}{\isacharparenright}{\kern0pt}{\isachardoublequoteclose}\ \isanewline
\ \ \ \ \ \ \ \ \ \ \ \ {\isachardoublequoteopen}last\ ps\isactrlsub {\isadigit{2}}\ {\isacharequal}{\kern0pt}\ {\isacharparenleft}{\kern0pt}{\isadigit{3}}{\isacharcomma}{\kern0pt}{\isadigit{2}}{\isacharparenright}{\kern0pt}{\isachardoublequoteclose}\isanewline
\ \ \ \ \ \ \ \ \ \ \isacommand{using}\isamarkupfalse%
\ {\isacartoucheopen}n{\isacharminus}{\kern0pt}{\isadigit{5}}\ {\isasymge}\ {\isadigit{5}}{\isacartoucheclose}\ rotate{\isacharunderscore}{\kern0pt}knights{\isacharunderscore}{\kern0pt}circuit{\isacharbrackleft}{\kern0pt}of\ m\ {\isachardoublequoteopen}n{\isacharminus}{\kern0pt}{\isadigit{5}}{\isachardoublequoteclose}{\isacharbrackright}{\kern0pt}\ \isacommand{by}\isamarkupfalse%
\ auto\isanewline
\ \ \ \ \ \ \ \ \isacommand{let}\isamarkupfalse%
\ {\isacharquery}{\kern0pt}ps\isactrlsub {\isadigit{2}}T{\isacharequal}{\kern0pt}{\isachardoublequoteopen}transpose\ ps\isactrlsub {\isadigit{2}}{\isachardoublequoteclose}\isanewline
\ \ \ \ \ \ \ \ \isacommand{have}\isamarkupfalse%
\ ps\isactrlsub {\isadigit{2}}T{\isacharunderscore}{\kern0pt}prems{\isacharcolon}{\kern0pt}\ {\isachardoublequoteopen}knights{\isacharunderscore}{\kern0pt}path\ {\isacharquery}{\kern0pt}b\isactrlsub {\isadigit{2}}T\ {\isacharquery}{\kern0pt}ps\isactrlsub {\isadigit{2}}T{\isachardoublequoteclose}\ {\isachardoublequoteopen}hd\ {\isacharquery}{\kern0pt}ps\isactrlsub {\isadigit{2}}T\ {\isacharequal}{\kern0pt}\ {\isacharparenleft}{\kern0pt}{\isadigit{1}}{\isacharcomma}{\kern0pt}{\isadigit{1}}{\isacharparenright}{\kern0pt}{\isachardoublequoteclose}\ {\isachardoublequoteopen}last\ {\isacharquery}{\kern0pt}ps\isactrlsub {\isadigit{2}}T\ {\isacharequal}{\kern0pt}\ {\isacharparenleft}{\kern0pt}{\isadigit{2}}{\isacharcomma}{\kern0pt}{\isadigit{3}}{\isacharparenright}{\kern0pt}{\isachardoublequoteclose}\isanewline
\ \ \ \ \ \ \ \ \ \ \isacommand{using}\isamarkupfalse%
\ ps\isactrlsub {\isadigit{2}}{\isacharunderscore}{\kern0pt}prems\ transpose{\isacharunderscore}{\kern0pt}knights{\isacharunderscore}{\kern0pt}path\ knights{\isacharunderscore}{\kern0pt}path{\isacharunderscore}{\kern0pt}non{\isacharunderscore}{\kern0pt}nil\ hd{\isacharunderscore}{\kern0pt}transpose\ last{\isacharunderscore}{\kern0pt}transpose\ \isanewline
\ \ \ \ \ \ \ \ \ \ \isacommand{unfolding}\isamarkupfalse%
\ knights{\isacharunderscore}{\kern0pt}circuit{\isacharunderscore}{\kern0pt}def\ transpose{\isacharunderscore}{\kern0pt}square{\isacharunderscore}{\kern0pt}def\ \isacommand{by}\isamarkupfalse%
\ auto\isanewline
\isanewline
\ \ \ \ \ \ \ \ \isacommand{let}\isamarkupfalse%
\ {\isacharquery}{\kern0pt}ps\isactrlsub {\isadigit{1}}{\isacharequal}{\kern0pt}{\isachardoublequoteopen}kp{\isadigit{5}}x{\isadigit{7}}lr{\isachardoublequoteclose}\isanewline
\ \ \ \ \ \ \ \ \isacommand{have}\isamarkupfalse%
\ ps\isactrlsub {\isadigit{1}}{\isacharunderscore}{\kern0pt}prems{\isacharcolon}{\kern0pt}\ {\isachardoublequoteopen}knights{\isacharunderscore}{\kern0pt}path\ b{\isadigit{5}}x{\isadigit{7}}\ {\isacharquery}{\kern0pt}ps\isactrlsub {\isadigit{1}}{\isachardoublequoteclose}\ {\isachardoublequoteopen}hd\ {\isacharquery}{\kern0pt}ps\isactrlsub {\isadigit{1}}\ {\isacharequal}{\kern0pt}\ {\isacharparenleft}{\kern0pt}{\isadigit{1}}{\isacharcomma}{\kern0pt}{\isadigit{1}}{\isacharparenright}{\kern0pt}{\isachardoublequoteclose}\ {\isachardoublequoteopen}last\ {\isacharquery}{\kern0pt}ps\isactrlsub {\isadigit{1}}\ {\isacharequal}{\kern0pt}\ {\isacharparenleft}{\kern0pt}{\isadigit{2}}{\isacharcomma}{\kern0pt}{\isadigit{6}}{\isacharparenright}{\kern0pt}{\isachardoublequoteclose}\isanewline
\ \ \ \ \ \ \ \ \ \ \isacommand{using}\isamarkupfalse%
\ kp{\isacharunderscore}{\kern0pt}{\isadigit{5}}x{\isadigit{7}}{\isacharunderscore}{\kern0pt}lr\ \isacommand{by}\isamarkupfalse%
\ simp\ eval{\isacharplus}{\kern0pt}\isanewline
\isanewline
\ \ \ \ \ \ \ \ \isacommand{have}\isamarkupfalse%
\ {\isachardoublequoteopen}{\isadigit{2}}{\isadigit{9}}\ {\isacharless}{\kern0pt}\ length\ {\isacharquery}{\kern0pt}ps\isactrlsub {\isadigit{1}}{\isachardoublequoteclose}\ {\isachardoublequoteopen}last\ {\isacharparenleft}{\kern0pt}take\ {\isadigit{2}}{\isadigit{9}}\ {\isacharquery}{\kern0pt}ps\isactrlsub {\isadigit{1}}{\isacharparenright}{\kern0pt}\ {\isacharequal}{\kern0pt}\ {\isacharparenleft}{\kern0pt}{\isadigit{4}}{\isacharcomma}{\kern0pt}{\isadigit{2}}{\isacharparenright}{\kern0pt}{\isachardoublequoteclose}\ {\isachardoublequoteopen}hd\ {\isacharparenleft}{\kern0pt}drop\ {\isadigit{2}}{\isadigit{9}}\ {\isacharquery}{\kern0pt}ps\isactrlsub {\isadigit{1}}{\isacharparenright}{\kern0pt}\ {\isacharequal}{\kern0pt}\ {\isacharparenleft}{\kern0pt}{\isadigit{5}}{\isacharcomma}{\kern0pt}{\isadigit{4}}{\isacharparenright}{\kern0pt}{\isachardoublequoteclose}\ \isacommand{by}\isamarkupfalse%
\ eval{\isacharplus}{\kern0pt}\isanewline
\ \ \ \ \ \ \ \ \isacommand{then}\isamarkupfalse%
\ \isacommand{have}\isamarkupfalse%
\ si{\isacharcolon}{\kern0pt}\ {\isachardoublequoteopen}step{\isacharunderscore}{\kern0pt}in\ {\isacharquery}{\kern0pt}ps\isactrlsub {\isadigit{1}}\ {\isacharparenleft}{\kern0pt}{\isadigit{4}}{\isacharcomma}{\kern0pt}{\isadigit{2}}{\isacharparenright}{\kern0pt}\ {\isacharparenleft}{\kern0pt}{\isadigit{5}}{\isacharcomma}{\kern0pt}{\isadigit{4}}{\isacharparenright}{\kern0pt}{\isachardoublequoteclose}\isanewline
\ \ \ \ \ \ \ \ \ \ \isacommand{unfolding}\isamarkupfalse%
\ step{\isacharunderscore}{\kern0pt}in{\isacharunderscore}{\kern0pt}def\ \isacommand{using}\isamarkupfalse%
\ zero{\isacharunderscore}{\kern0pt}less{\isacharunderscore}{\kern0pt}numeral\ \isacommand{by}\isamarkupfalse%
\ blast\isanewline
\isanewline
\ \ \ \ \ \ \ \ \isacommand{have}\isamarkupfalse%
\ vs{\isacharcolon}{\kern0pt}\ {\isachardoublequoteopen}valid{\isacharunderscore}{\kern0pt}step\ {\isacharparenleft}{\kern0pt}{\isadigit{4}}{\isacharcomma}{\kern0pt}{\isadigit{2}}{\isacharparenright}{\kern0pt}\ {\isacharparenleft}{\kern0pt}int\ {\isadigit{5}}{\isacharplus}{\kern0pt}{\isadigit{1}}{\isacharcomma}{\kern0pt}{\isadigit{1}}{\isacharparenright}{\kern0pt}{\isachardoublequoteclose}\ {\isachardoublequoteopen}valid{\isacharunderscore}{\kern0pt}step\ {\isacharparenleft}{\kern0pt}int\ {\isadigit{5}}{\isacharplus}{\kern0pt}{\isadigit{2}}{\isacharcomma}{\kern0pt}{\isadigit{3}}{\isacharparenright}{\kern0pt}\ {\isacharparenleft}{\kern0pt}{\isadigit{5}}{\isacharcomma}{\kern0pt}{\isadigit{4}}{\isacharparenright}{\kern0pt}{\isachardoublequoteclose}\isanewline
\ \ \ \ \ \ \ \ \ \ \isacommand{unfolding}\isamarkupfalse%
\ valid{\isacharunderscore}{\kern0pt}step{\isacharunderscore}{\kern0pt}def\ \isacommand{by}\isamarkupfalse%
\ auto\isanewline
\isanewline
\ \ \ \ \ \ \ \ \isacommand{obtain}\isamarkupfalse%
\ ps\ \isakeyword{where}\ {\isachardoublequoteopen}knights{\isacharunderscore}{\kern0pt}path\ {\isacharparenleft}{\kern0pt}board\ n\ m{\isacharparenright}{\kern0pt}\ ps{\isachardoublequoteclose}\ {\isachardoublequoteopen}hd\ ps\ {\isacharequal}{\kern0pt}\ {\isacharparenleft}{\kern0pt}{\isadigit{1}}{\isacharcomma}{\kern0pt}{\isadigit{1}}{\isacharparenright}{\kern0pt}{\isachardoublequoteclose}\ {\isachardoublequoteopen}last\ ps\ {\isacharequal}{\kern0pt}\ {\isacharparenleft}{\kern0pt}{\isadigit{2}}{\isacharcomma}{\kern0pt}{\isadigit{6}}{\isacharparenright}{\kern0pt}{\isachardoublequoteclose}\isanewline
\ \ \ \ \ \ \ \ \ \ \isacommand{using}\isamarkupfalse%
\ {\isacartoucheopen}n{\isacharminus}{\kern0pt}{\isadigit{5}}\ {\isasymge}\ {\isadigit{5}}{\isacartoucheclose}\ ps\isactrlsub {\isadigit{1}}{\isacharunderscore}{\kern0pt}prems\ ps\isactrlsub {\isadigit{2}}T{\isacharunderscore}{\kern0pt}prems\ si\ vs\isanewline
\ \ \ \ \ \ \ \ \ \ \ \ \ \ knights{\isacharunderscore}{\kern0pt}path{\isacharunderscore}{\kern0pt}split{\isacharunderscore}{\kern0pt}concatT{\isacharbrackleft}{\kern0pt}of\ {\isadigit{5}}\ m\ {\isacharquery}{\kern0pt}ps\isactrlsub {\isadigit{1}}\ {\isachardoublequoteopen}n{\isacharminus}{\kern0pt}{\isadigit{5}}{\isachardoublequoteclose}\ {\isacharquery}{\kern0pt}ps\isactrlsub {\isadigit{2}}T\ {\isachardoublequoteopen}{\isacharparenleft}{\kern0pt}{\isadigit{4}}{\isacharcomma}{\kern0pt}{\isadigit{2}}{\isacharparenright}{\kern0pt}{\isachardoublequoteclose}\ {\isachardoublequoteopen}{\isacharparenleft}{\kern0pt}{\isadigit{5}}{\isacharcomma}{\kern0pt}{\isadigit{4}}{\isacharparenright}{\kern0pt}{\isachardoublequoteclose}{\isacharbrackright}{\kern0pt}\ \isacommand{by}\isamarkupfalse%
\ auto\isanewline
\ \ \ \ \ \ \ \ \isacommand{then}\isamarkupfalse%
\ \isacommand{show}\isamarkupfalse%
\ {\isacharquery}{\kern0pt}thesis\isanewline
\ \ \ \ \ \ \ \ \ \ \isacommand{using}\isamarkupfalse%
\ mirror{\isadigit{1}}{\isacharunderscore}{\kern0pt}knights{\isacharunderscore}{\kern0pt}path\ hd{\isacharunderscore}{\kern0pt}mirror{\isadigit{1}}\ last{\isacharunderscore}{\kern0pt}mirror{\isadigit{1}}\ \isacommand{by}\isamarkupfalse%
\ fastforce\isanewline
\ \ \ \ \ \ \isacommand{qed}\isamarkupfalse%
\isanewline
\ \ \ \ \isacommand{next}\isamarkupfalse%
\isanewline
\ \ \ \ \ \ \isacommand{assume}\isamarkupfalse%
\ {\isacharbrackleft}{\kern0pt}simp{\isacharbrackright}{\kern0pt}{\isacharcolon}{\kern0pt}\ {\isachardoublequoteopen}m\ {\isacharequal}{\kern0pt}\ {\isadigit{8}}{\isachardoublequoteclose}\isanewline
\ \ \ \ \ \ \isacommand{then}\isamarkupfalse%
\ \isacommand{obtain}\isamarkupfalse%
\ ps\ \isakeyword{where}\ ps{\isacharunderscore}{\kern0pt}prems{\isacharcolon}{\kern0pt}\ {\isachardoublequoteopen}knights{\isacharunderscore}{\kern0pt}path\ {\isacharparenleft}{\kern0pt}board\ m\ n{\isacharparenright}{\kern0pt}\ ps{\isachardoublequoteclose}\ {\isachardoublequoteopen}hd\ ps\ {\isacharequal}{\kern0pt}\ {\isacharparenleft}{\kern0pt}{\isadigit{1}}{\isacharcomma}{\kern0pt}{\isadigit{1}}{\isacharparenright}{\kern0pt}{\isachardoublequoteclose}\ \isanewline
\ \ \ \ \ \ \ \ \ \ {\isachardoublequoteopen}last\ ps\ {\isacharequal}{\kern0pt}\ {\isacharparenleft}{\kern0pt}int\ m{\isacharminus}{\kern0pt}{\isadigit{1}}{\isacharcomma}{\kern0pt}{\isadigit{2}}{\isacharparenright}{\kern0pt}{\isachardoublequoteclose}\isanewline
\ \ \ \ \ \ \ \ \isacommand{using}\isamarkupfalse%
\ less\ knights{\isacharunderscore}{\kern0pt}path{\isacharunderscore}{\kern0pt}{\isadigit{8}}xm{\isacharunderscore}{\kern0pt}exists{\isacharbrackleft}{\kern0pt}of\ n{\isacharbrackright}{\kern0pt}\ \isacommand{by}\isamarkupfalse%
\ auto\isanewline
\ \ \ \ \ \ \isacommand{let}\isamarkupfalse%
\ {\isacharquery}{\kern0pt}ps{\isacharprime}{\kern0pt}{\isacharequal}{\kern0pt}{\isachardoublequoteopen}mirror{\isadigit{1}}\ {\isacharparenleft}{\kern0pt}transpose\ ps{\isacharparenright}{\kern0pt}{\isachardoublequoteclose}\isanewline
\ \ \ \ \ \ \isacommand{have}\isamarkupfalse%
\ {\isachardoublequoteopen}knights{\isacharunderscore}{\kern0pt}path\ {\isacharparenleft}{\kern0pt}board\ n\ m{\isacharparenright}{\kern0pt}\ {\isacharquery}{\kern0pt}ps{\isacharprime}{\kern0pt}{\isachardoublequoteclose}\ {\isachardoublequoteopen}hd\ {\isacharquery}{\kern0pt}ps{\isacharprime}{\kern0pt}\ {\isacharequal}{\kern0pt}\ {\isacharparenleft}{\kern0pt}int\ n{\isacharcomma}{\kern0pt}{\isadigit{1}}{\isacharparenright}{\kern0pt}{\isachardoublequoteclose}\ {\isachardoublequoteopen}last\ {\isacharquery}{\kern0pt}ps{\isacharprime}{\kern0pt}\ {\isacharequal}{\kern0pt}\ {\isacharparenleft}{\kern0pt}int\ n{\isacharminus}{\kern0pt}{\isadigit{1}}{\isacharcomma}{\kern0pt}int\ m{\isacharminus}{\kern0pt}{\isadigit{1}}{\isacharparenright}{\kern0pt}{\isachardoublequoteclose}\isanewline
\ \ \ \ \ \ \ \ \isacommand{using}\isamarkupfalse%
\ ps{\isacharunderscore}{\kern0pt}prems\ rot{\isadigit{9}}{\isadigit{0}}{\isacharunderscore}{\kern0pt}knights{\isacharunderscore}{\kern0pt}path\ hd{\isacharunderscore}{\kern0pt}rot{\isadigit{9}}{\isadigit{0}}{\isacharunderscore}{\kern0pt}knights{\isacharunderscore}{\kern0pt}path\ last{\isacharunderscore}{\kern0pt}rot{\isadigit{9}}{\isadigit{0}}{\isacharunderscore}{\kern0pt}knights{\isacharunderscore}{\kern0pt}path\ \isacommand{by}\isamarkupfalse%
\ auto\isanewline
\ \ \ \ \ \ \isacommand{then}\isamarkupfalse%
\ \isacommand{show}\isamarkupfalse%
\ {\isacharquery}{\kern0pt}thesis\ \isacommand{by}\isamarkupfalse%
\ auto\isanewline
\ \ \ \ \isacommand{next}\isamarkupfalse%
\isanewline
\ \ \ \ \ \ \isacommand{assume}\isamarkupfalse%
\ {\isacharbrackleft}{\kern0pt}simp{\isacharbrackright}{\kern0pt}{\isacharcolon}{\kern0pt}\ {\isachardoublequoteopen}m\ {\isacharequal}{\kern0pt}\ {\isadigit{9}}{\isachardoublequoteclose}\isanewline
\ \ \ \ \ \ \isacommand{have}\isamarkupfalse%
\ {\isachardoublequoteopen}odd\ n{\isachardoublequoteclose}\ {\isachardoublequoteopen}n\ {\isasymge}\ {\isadigit{5}}{\isachardoublequoteclose}\ \isacommand{using}\isamarkupfalse%
\ less\ \isacommand{by}\isamarkupfalse%
\ auto\isanewline
\ \ \ \ \ \ \isacommand{then}\isamarkupfalse%
\ \isacommand{have}\isamarkupfalse%
\ {\isachardoublequoteopen}n\ {\isacharequal}{\kern0pt}\ {\isadigit{5}}\ {\isasymor}\ n\ {\isacharequal}{\kern0pt}\ {\isadigit{7}}\ {\isasymor}\ n\ {\isacharequal}{\kern0pt}\ {\isadigit{9}}\ {\isasymor}\ n{\isacharminus}{\kern0pt}{\isadigit{5}}\ {\isasymge}\ {\isadigit{5}}{\isachardoublequoteclose}\ \isacommand{by}\isamarkupfalse%
\ presburger\isanewline
\ \ \ \ \ \ \isacommand{then}\isamarkupfalse%
\ \isacommand{show}\isamarkupfalse%
\ {\isacharquery}{\kern0pt}thesis\isanewline
\ \ \ \ \ \ \isacommand{proof}\isamarkupfalse%
\ {\isacharparenleft}{\kern0pt}elim\ disjE{\isacharparenright}{\kern0pt}\isanewline
\ \ \ \ \ \ \ \ \isacommand{assume}\isamarkupfalse%
\ {\isacharbrackleft}{\kern0pt}simp{\isacharbrackright}{\kern0pt}{\isacharcolon}{\kern0pt}\ {\isachardoublequoteopen}n\ {\isacharequal}{\kern0pt}\ {\isadigit{5}}{\isachardoublequoteclose}\isanewline
\ \ \ \ \ \ \ \ \isacommand{let}\isamarkupfalse%
\ {\isacharquery}{\kern0pt}ps{\isacharequal}{\kern0pt}{\isachardoublequoteopen}mirror{\isadigit{1}}\ kp{\isadigit{5}}x{\isadigit{9}}lr{\isachardoublequoteclose}\isanewline
\ \ \ \ \ \ \ \ \isacommand{have}\isamarkupfalse%
\ kp{\isacharcolon}{\kern0pt}\ {\isachardoublequoteopen}knights{\isacharunderscore}{\kern0pt}path\ {\isacharparenleft}{\kern0pt}board\ n\ m{\isacharparenright}{\kern0pt}\ {\isacharquery}{\kern0pt}ps{\isachardoublequoteclose}\isanewline
\ \ \ \ \ \ \ \ \ \ \isacommand{using}\isamarkupfalse%
\ kp{\isacharunderscore}{\kern0pt}{\isadigit{5}}x{\isadigit{9}}{\isacharunderscore}{\kern0pt}lr\ mirror{\isadigit{1}}{\isacharunderscore}{\kern0pt}knights{\isacharunderscore}{\kern0pt}path\ \isacommand{by}\isamarkupfalse%
\ auto\isanewline
\ \ \ \ \ \ \ \ \isacommand{have}\isamarkupfalse%
\ {\isachardoublequoteopen}hd\ {\isacharquery}{\kern0pt}ps\ {\isacharequal}{\kern0pt}\ {\isacharparenleft}{\kern0pt}int\ n{\isacharcomma}{\kern0pt}{\isadigit{1}}{\isacharparenright}{\kern0pt}{\isachardoublequoteclose}\ {\isachardoublequoteopen}last\ {\isacharquery}{\kern0pt}ps\ {\isacharequal}{\kern0pt}\ {\isacharparenleft}{\kern0pt}int\ n{\isacharminus}{\kern0pt}{\isadigit{1}}{\isacharcomma}{\kern0pt}int\ m{\isacharminus}{\kern0pt}{\isadigit{1}}{\isacharparenright}{\kern0pt}{\isachardoublequoteclose}\isanewline
\ \ \ \ \ \ \ \ \ \ \isacommand{by}\isamarkupfalse%
\ {\isacharparenleft}{\kern0pt}simp\ only{\isacharcolon}{\kern0pt}\ {\isacartoucheopen}m\ {\isacharequal}{\kern0pt}\ {\isadigit{9}}{\isacartoucheclose}\ {\isacartoucheopen}n\ {\isacharequal}{\kern0pt}\ {\isadigit{5}}{\isacartoucheclose}\ {\isacharbar}{\kern0pt}\ eval{\isacharparenright}{\kern0pt}{\isacharplus}{\kern0pt}\isanewline
\ \ \ \ \ \ \ \ \isacommand{then}\isamarkupfalse%
\ \isacommand{show}\isamarkupfalse%
\ {\isacharquery}{\kern0pt}thesis\ \isacommand{using}\isamarkupfalse%
\ kp\ \isacommand{by}\isamarkupfalse%
\ auto\isanewline
\ \ \ \ \ \ \isacommand{next}\isamarkupfalse%
\isanewline
\ \ \ \ \ \ \ \ \isacommand{assume}\isamarkupfalse%
\ {\isacharbrackleft}{\kern0pt}simp{\isacharbrackright}{\kern0pt}{\isacharcolon}{\kern0pt}\ {\isachardoublequoteopen}n\ {\isacharequal}{\kern0pt}\ {\isadigit{7}}{\isachardoublequoteclose}\isanewline
\ \ \ \ \ \ \ \ \isacommand{let}\isamarkupfalse%
\ {\isacharquery}{\kern0pt}ps{\isacharequal}{\kern0pt}{\isachardoublequoteopen}mirror{\isadigit{1}}\ {\isacharparenleft}{\kern0pt}transpose\ kp{\isadigit{9}}x{\isadigit{7}}ul{\isacharparenright}{\kern0pt}{\isachardoublequoteclose}\isanewline
\ \ \ \ \ \ \ \ \isacommand{have}\isamarkupfalse%
\ kp{\isacharcolon}{\kern0pt}\ {\isachardoublequoteopen}knights{\isacharunderscore}{\kern0pt}path\ {\isacharparenleft}{\kern0pt}board\ n\ m{\isacharparenright}{\kern0pt}\ {\isacharquery}{\kern0pt}ps{\isachardoublequoteclose}\isanewline
\ \ \ \ \ \ \ \ \ \ \isacommand{using}\isamarkupfalse%
\ kp{\isacharunderscore}{\kern0pt}{\isadigit{9}}x{\isadigit{7}}{\isacharunderscore}{\kern0pt}ul\ rot{\isadigit{9}}{\isadigit{0}}{\isacharunderscore}{\kern0pt}knights{\isacharunderscore}{\kern0pt}path\ \isacommand{by}\isamarkupfalse%
\ auto\isanewline
\ \ \ \ \ \ \ \ \isacommand{have}\isamarkupfalse%
\ {\isachardoublequoteopen}hd\ {\isacharquery}{\kern0pt}ps\ {\isacharequal}{\kern0pt}\ {\isacharparenleft}{\kern0pt}int\ n{\isacharcomma}{\kern0pt}{\isadigit{1}}{\isacharparenright}{\kern0pt}{\isachardoublequoteclose}\ {\isachardoublequoteopen}last\ {\isacharquery}{\kern0pt}ps\ {\isacharequal}{\kern0pt}\ {\isacharparenleft}{\kern0pt}int\ n{\isacharminus}{\kern0pt}{\isadigit{1}}{\isacharcomma}{\kern0pt}int\ m{\isacharminus}{\kern0pt}{\isadigit{1}}{\isacharparenright}{\kern0pt}{\isachardoublequoteclose}\isanewline
\ \ \ \ \ \ \ \ \ \ \isacommand{by}\isamarkupfalse%
\ {\isacharparenleft}{\kern0pt}simp\ only{\isacharcolon}{\kern0pt}\ {\isacartoucheopen}m\ {\isacharequal}{\kern0pt}\ {\isadigit{9}}{\isacartoucheclose}\ {\isacartoucheopen}n\ {\isacharequal}{\kern0pt}\ {\isadigit{7}}{\isacartoucheclose}\ {\isacharbar}{\kern0pt}\ eval{\isacharparenright}{\kern0pt}{\isacharplus}{\kern0pt}\isanewline
\ \ \ \ \ \ \ \ \isacommand{then}\isamarkupfalse%
\ \isacommand{show}\isamarkupfalse%
\ {\isacharquery}{\kern0pt}thesis\ \isacommand{using}\isamarkupfalse%
\ kp\ \isacommand{by}\isamarkupfalse%
\ auto\isanewline
\ \ \ \ \ \ \isacommand{next}\isamarkupfalse%
\isanewline
\ \ \ \ \ \ \ \ \isacommand{assume}\isamarkupfalse%
\ {\isacharbrackleft}{\kern0pt}simp{\isacharbrackright}{\kern0pt}{\isacharcolon}{\kern0pt}\ {\isachardoublequoteopen}n\ {\isacharequal}{\kern0pt}\ {\isadigit{9}}{\isachardoublequoteclose}\isanewline
\ \ \ \ \ \ \ \ \isacommand{let}\isamarkupfalse%
\ {\isacharquery}{\kern0pt}ps{\isacharequal}{\kern0pt}{\isachardoublequoteopen}mirror{\isadigit{1}}\ {\isacharparenleft}{\kern0pt}transpose\ kp{\isadigit{9}}x{\isadigit{9}}ul{\isacharparenright}{\kern0pt}{\isachardoublequoteclose}\isanewline
\ \ \ \ \ \ \ \ \isacommand{have}\isamarkupfalse%
\ kp{\isacharcolon}{\kern0pt}\ {\isachardoublequoteopen}knights{\isacharunderscore}{\kern0pt}path\ {\isacharparenleft}{\kern0pt}board\ n\ m{\isacharparenright}{\kern0pt}\ {\isacharquery}{\kern0pt}ps{\isachardoublequoteclose}\isanewline
\ \ \ \ \ \ \ \ \ \ \isacommand{using}\isamarkupfalse%
\ kp{\isacharunderscore}{\kern0pt}{\isadigit{9}}x{\isadigit{9}}{\isacharunderscore}{\kern0pt}ul\ rot{\isadigit{9}}{\isadigit{0}}{\isacharunderscore}{\kern0pt}knights{\isacharunderscore}{\kern0pt}path\ \isacommand{by}\isamarkupfalse%
\ auto\isanewline
\ \ \ \ \ \ \ \ \isacommand{have}\isamarkupfalse%
\ {\isachardoublequoteopen}hd\ {\isacharquery}{\kern0pt}ps\ {\isacharequal}{\kern0pt}\ {\isacharparenleft}{\kern0pt}int\ n{\isacharcomma}{\kern0pt}{\isadigit{1}}{\isacharparenright}{\kern0pt}{\isachardoublequoteclose}\ {\isachardoublequoteopen}last\ {\isacharquery}{\kern0pt}ps\ {\isacharequal}{\kern0pt}\ {\isacharparenleft}{\kern0pt}int\ n{\isacharminus}{\kern0pt}{\isadigit{1}}{\isacharcomma}{\kern0pt}int\ m{\isacharminus}{\kern0pt}{\isadigit{1}}{\isacharparenright}{\kern0pt}{\isachardoublequoteclose}\isanewline
\ \ \ \ \ \ \ \ \ \ \isacommand{by}\isamarkupfalse%
\ {\isacharparenleft}{\kern0pt}simp\ only{\isacharcolon}{\kern0pt}\ {\isacartoucheopen}m\ {\isacharequal}{\kern0pt}\ {\isadigit{9}}{\isacartoucheclose}\ {\isacartoucheopen}n\ {\isacharequal}{\kern0pt}\ {\isadigit{9}}{\isacartoucheclose}\ {\isacharbar}{\kern0pt}\ eval{\isacharparenright}{\kern0pt}{\isacharplus}{\kern0pt}\isanewline
\ \ \ \ \ \ \ \ \isacommand{then}\isamarkupfalse%
\ \isacommand{show}\isamarkupfalse%
\ {\isacharquery}{\kern0pt}thesis\ \isacommand{using}\isamarkupfalse%
\ kp\ \isacommand{by}\isamarkupfalse%
\ auto\isanewline
\ \ \ \ \ \ \isacommand{next}\isamarkupfalse%
\isanewline
\ \ \ \ \ \ \ \ \isacommand{let}\isamarkupfalse%
\ {\isacharquery}{\kern0pt}b\isactrlsub {\isadigit{2}}{\isacharequal}{\kern0pt}{\isachardoublequoteopen}board\ m\ {\isacharparenleft}{\kern0pt}n{\isacharminus}{\kern0pt}{\isadigit{5}}{\isacharparenright}{\kern0pt}{\isachardoublequoteclose}\isanewline
\ \ \ \ \ \ \ \ \isacommand{let}\isamarkupfalse%
\ {\isacharquery}{\kern0pt}b\isactrlsub {\isadigit{2}}T{\isacharequal}{\kern0pt}{\isachardoublequoteopen}board\ {\isacharparenleft}{\kern0pt}n{\isacharminus}{\kern0pt}{\isadigit{5}}{\isacharparenright}{\kern0pt}\ m{\isachardoublequoteclose}\isanewline
\ \ \ \ \ \ \ \ \isacommand{assume}\isamarkupfalse%
\ {\isachardoublequoteopen}n{\isacharminus}{\kern0pt}{\isadigit{5}}\ {\isasymge}\ {\isadigit{5}}{\isachardoublequoteclose}\isanewline
\ \ \ \ \ \ \ \ \isacommand{then}\isamarkupfalse%
\ \isacommand{have}\isamarkupfalse%
\ {\isachardoublequoteopen}{\isasymexists}ps{\isachardot}{\kern0pt}\ knights{\isacharunderscore}{\kern0pt}circuit\ {\isacharquery}{\kern0pt}b\isactrlsub {\isadigit{2}}\ ps{\isachardoublequoteclose}\isanewline
\ \ \ \ \ \ \ \ \isacommand{proof}\isamarkupfalse%
\ {\isacharminus}{\kern0pt}\isanewline
\ \ \ \ \ \ \ \ \ \ \isacommand{have}\isamarkupfalse%
\ {\isachardoublequoteopen}n{\isacharminus}{\kern0pt}{\isadigit{5}}\ {\isacharequal}{\kern0pt}\ {\isadigit{6}}\ {\isasymor}\ n{\isacharminus}{\kern0pt}{\isadigit{5}}\ {\isacharequal}{\kern0pt}\ {\isadigit{8}}\ {\isasymor}\ n{\isacharminus}{\kern0pt}{\isadigit{5}}\ {\isasymge}\ {\isadigit{1}}{\isadigit{0}}{\isachardoublequoteclose}\ \isanewline
\ \ \ \ \ \ \ \ \ \ \ \ \isacommand{using}\isamarkupfalse%
\ {\isacartoucheopen}n{\isacharminus}{\kern0pt}{\isadigit{5}}\ {\isasymge}\ {\isadigit{5}}{\isacartoucheclose}\ less\ \isacommand{by}\isamarkupfalse%
\ presburger\isanewline
\ \ \ \ \ \ \ \ \ \ \isacommand{then}\isamarkupfalse%
\ \isacommand{show}\isamarkupfalse%
\ {\isacharquery}{\kern0pt}thesis\isanewline
\ \ \ \ \ \ \ \ \ \ \isacommand{proof}\isamarkupfalse%
\ {\isacharparenleft}{\kern0pt}elim\ disjE{\isacharparenright}{\kern0pt}\isanewline
\ \ \ \ \ \ \ \ \ \ \ \ \isacommand{assume}\isamarkupfalse%
\ {\isachardoublequoteopen}n{\isacharminus}{\kern0pt}{\isadigit{5}}\ {\isacharequal}{\kern0pt}\ {\isadigit{6}}{\isachardoublequoteclose}\isanewline
\ \ \ \ \ \ \ \ \ \ \ \ \isacommand{then}\isamarkupfalse%
\ \isacommand{obtain}\isamarkupfalse%
\ ps\ \isakeyword{where}\ {\isachardoublequoteopen}knights{\isacharunderscore}{\kern0pt}circuit\ {\isacharparenleft}{\kern0pt}board\ {\isacharparenleft}{\kern0pt}n{\isacharminus}{\kern0pt}{\isadigit{5}}{\isacharparenright}{\kern0pt}\ m{\isacharparenright}{\kern0pt}\ ps{\isachardoublequoteclose}\isanewline
\ \ \ \ \ \ \ \ \ \ \ \ \ \ \isacommand{using}\isamarkupfalse%
\ knights{\isacharunderscore}{\kern0pt}path{\isacharunderscore}{\kern0pt}{\isadigit{6}}xm{\isacharunderscore}{\kern0pt}exists{\isacharbrackleft}{\kern0pt}of\ m{\isacharbrackright}{\kern0pt}\ \isacommand{by}\isamarkupfalse%
\ auto\isanewline
\ \ \ \ \ \ \ \ \ \ \ \ \isacommand{then}\isamarkupfalse%
\ \isacommand{show}\isamarkupfalse%
\ {\isacharquery}{\kern0pt}thesis\ \isanewline
\ \ \ \ \ \ \ \ \ \ \ \ \ \ \isacommand{using}\isamarkupfalse%
\ transpose{\isacharunderscore}{\kern0pt}knights{\isacharunderscore}{\kern0pt}circuit\ \isacommand{by}\isamarkupfalse%
\ auto\isanewline
\ \ \ \ \ \ \ \ \ \ \isacommand{next}\isamarkupfalse%
\isanewline
\ \ \ \ \ \ \ \ \ \ \ \ \isacommand{assume}\isamarkupfalse%
\ {\isachardoublequoteopen}n{\isacharminus}{\kern0pt}{\isadigit{5}}\ {\isacharequal}{\kern0pt}\ {\isadigit{8}}{\isachardoublequoteclose}\isanewline
\ \ \ \ \ \ \ \ \ \ \ \ \isacommand{then}\isamarkupfalse%
\ \isacommand{obtain}\isamarkupfalse%
\ ps\ \isakeyword{where}\ {\isachardoublequoteopen}knights{\isacharunderscore}{\kern0pt}circuit\ {\isacharparenleft}{\kern0pt}board\ {\isacharparenleft}{\kern0pt}n{\isacharminus}{\kern0pt}{\isadigit{5}}{\isacharparenright}{\kern0pt}\ m{\isacharparenright}{\kern0pt}\ ps{\isachardoublequoteclose}\isanewline
\ \ \ \ \ \ \ \ \ \ \ \ \ \ \isacommand{using}\isamarkupfalse%
\ knights{\isacharunderscore}{\kern0pt}path{\isacharunderscore}{\kern0pt}{\isadigit{8}}xm{\isacharunderscore}{\kern0pt}exists{\isacharbrackleft}{\kern0pt}of\ m{\isacharbrackright}{\kern0pt}\ \isacommand{by}\isamarkupfalse%
\ auto\isanewline
\ \ \ \ \ \ \ \ \ \ \ \ \isacommand{then}\isamarkupfalse%
\ \isacommand{show}\isamarkupfalse%
\ {\isacharquery}{\kern0pt}thesis\ \isanewline
\ \ \ \ \ \ \ \ \ \ \ \ \ \ \isacommand{using}\isamarkupfalse%
\ transpose{\isacharunderscore}{\kern0pt}knights{\isacharunderscore}{\kern0pt}circuit\ \isacommand{by}\isamarkupfalse%
\ auto\isanewline
\ \ \ \ \ \ \ \ \ \ \isacommand{next}\isamarkupfalse%
\isanewline
\ \ \ \ \ \ \ \ \ \ \ \ \isacommand{assume}\isamarkupfalse%
\ {\isachardoublequoteopen}n{\isacharminus}{\kern0pt}{\isadigit{5}}\ {\isasymge}\ {\isadigit{1}}{\isadigit{0}}{\isachardoublequoteclose}\isanewline
\ \ \ \ \ \ \ \ \ \ \ \ \isacommand{then}\isamarkupfalse%
\ \isacommand{show}\isamarkupfalse%
\ {\isacharquery}{\kern0pt}thesis\ \isanewline
\ \ \ \ \ \ \ \ \ \ \ \ \ \ \isacommand{using}\isamarkupfalse%
\ less\ less{\isachardot}{\kern0pt}IH{\isacharbrackleft}{\kern0pt}of\ {\isachardoublequoteopen}n{\isacharminus}{\kern0pt}{\isadigit{1}}{\isadigit{0}}{\isacharplus}{\kern0pt}m{\isachardoublequoteclose}\ {\isachardoublequoteopen}n{\isacharminus}{\kern0pt}{\isadigit{1}}{\isadigit{0}}{\isachardoublequoteclose}\ m{\isacharbrackright}{\kern0pt}\isanewline
\ \ \ \ \ \ \ \ \ \ \ \ \ \ \ \ \ \ \ \ knights{\isacharunderscore}{\kern0pt}circuit{\isacharunderscore}{\kern0pt}exists{\isacharunderscore}{\kern0pt}even{\isacharunderscore}{\kern0pt}n{\isacharunderscore}{\kern0pt}gr{\isadigit{1}}{\isadigit{0}}{\isacharbrackleft}{\kern0pt}of\ {\isachardoublequoteopen}n{\isacharminus}{\kern0pt}{\isadigit{5}}{\isachardoublequoteclose}\ m{\isacharbrackright}{\kern0pt}\ \isacommand{by}\isamarkupfalse%
\ auto\isanewline
\ \ \ \ \ \ \ \ \ \ \isacommand{qed}\isamarkupfalse%
\isanewline
\ \ \ \ \ \ \ \ \isacommand{qed}\isamarkupfalse%
\isanewline
\ \ \ \ \ \ \ \ \isacommand{then}\isamarkupfalse%
\ \isacommand{obtain}\isamarkupfalse%
\ ps\isactrlsub {\isadigit{2}}\ \isakeyword{where}\ ps\isactrlsub {\isadigit{2}}{\isacharunderscore}{\kern0pt}prems{\isacharcolon}{\kern0pt}\ {\isachardoublequoteopen}knights{\isacharunderscore}{\kern0pt}circuit\ {\isacharquery}{\kern0pt}b\isactrlsub {\isadigit{2}}\ ps\isactrlsub {\isadigit{2}}{\isachardoublequoteclose}\ {\isachardoublequoteopen}hd\ ps\isactrlsub {\isadigit{2}}\ {\isacharequal}{\kern0pt}\ {\isacharparenleft}{\kern0pt}{\isadigit{1}}{\isacharcomma}{\kern0pt}{\isadigit{1}}{\isacharparenright}{\kern0pt}{\isachardoublequoteclose}\ \isanewline
\ \ \ \ \ \ \ \ \ \ \ \ {\isachardoublequoteopen}last\ ps\isactrlsub {\isadigit{2}}\ {\isacharequal}{\kern0pt}\ {\isacharparenleft}{\kern0pt}{\isadigit{3}}{\isacharcomma}{\kern0pt}{\isadigit{2}}{\isacharparenright}{\kern0pt}{\isachardoublequoteclose}\isanewline
\ \ \ \ \ \ \ \ \ \ \isacommand{using}\isamarkupfalse%
\ {\isacartoucheopen}n{\isacharminus}{\kern0pt}{\isadigit{5}}\ {\isasymge}\ {\isadigit{5}}{\isacartoucheclose}\ rotate{\isacharunderscore}{\kern0pt}knights{\isacharunderscore}{\kern0pt}circuit{\isacharbrackleft}{\kern0pt}of\ m\ {\isachardoublequoteopen}n{\isacharminus}{\kern0pt}{\isadigit{5}}{\isachardoublequoteclose}{\isacharbrackright}{\kern0pt}\ \isacommand{by}\isamarkupfalse%
\ auto\isanewline
\ \ \ \ \ \ \ \ \isacommand{let}\isamarkupfalse%
\ {\isacharquery}{\kern0pt}ps\isactrlsub {\isadigit{2}}T{\isacharequal}{\kern0pt}{\isachardoublequoteopen}transpose\ {\isacharparenleft}{\kern0pt}rev\ ps\isactrlsub {\isadigit{2}}{\isacharparenright}{\kern0pt}{\isachardoublequoteclose}\isanewline
\ \ \ \ \ \ \ \ \isacommand{have}\isamarkupfalse%
\ ps\isactrlsub {\isadigit{2}}T{\isacharunderscore}{\kern0pt}prems{\isacharcolon}{\kern0pt}\ {\isachardoublequoteopen}knights{\isacharunderscore}{\kern0pt}path\ {\isacharquery}{\kern0pt}b\isactrlsub {\isadigit{2}}T\ {\isacharquery}{\kern0pt}ps\isactrlsub {\isadigit{2}}T{\isachardoublequoteclose}\ {\isachardoublequoteopen}hd\ {\isacharquery}{\kern0pt}ps\isactrlsub {\isadigit{2}}T\ {\isacharequal}{\kern0pt}\ {\isacharparenleft}{\kern0pt}{\isadigit{2}}{\isacharcomma}{\kern0pt}{\isadigit{3}}{\isacharparenright}{\kern0pt}{\isachardoublequoteclose}\ {\isachardoublequoteopen}last\ {\isacharquery}{\kern0pt}ps\isactrlsub {\isadigit{2}}T\ {\isacharequal}{\kern0pt}\ {\isacharparenleft}{\kern0pt}{\isadigit{1}}{\isacharcomma}{\kern0pt}{\isadigit{1}}{\isacharparenright}{\kern0pt}{\isachardoublequoteclose}\isanewline
\ \ \ \ \ \ \ \ \ \ \isacommand{using}\isamarkupfalse%
\ ps\isactrlsub {\isadigit{2}}{\isacharunderscore}{\kern0pt}prems\ knights{\isacharunderscore}{\kern0pt}path{\isacharunderscore}{\kern0pt}rev\ transpose{\isacharunderscore}{\kern0pt}knights{\isacharunderscore}{\kern0pt}path\ knights{\isacharunderscore}{\kern0pt}path{\isacharunderscore}{\kern0pt}non{\isacharunderscore}{\kern0pt}nil\ \isanewline
\ \ \ \ \ \ \ \ \ \ \ \ \ \ \ \ hd{\isacharunderscore}{\kern0pt}transpose\ last{\isacharunderscore}{\kern0pt}transpose\ \isanewline
\ \ \ \ \ \ \ \ \ \ \isacommand{unfolding}\isamarkupfalse%
\ knights{\isacharunderscore}{\kern0pt}circuit{\isacharunderscore}{\kern0pt}def\ transpose{\isacharunderscore}{\kern0pt}square{\isacharunderscore}{\kern0pt}def\ \isacommand{by}\isamarkupfalse%
\ {\isacharparenleft}{\kern0pt}auto\ simp{\isacharcolon}{\kern0pt}\ hd{\isacharunderscore}{\kern0pt}rev\ last{\isacharunderscore}{\kern0pt}rev{\isacharparenright}{\kern0pt}\isanewline
\isanewline
\ \ \ \ \ \ \ \ \isacommand{let}\isamarkupfalse%
\ {\isacharquery}{\kern0pt}ps\isactrlsub {\isadigit{1}}{\isacharequal}{\kern0pt}{\isachardoublequoteopen}kp{\isadigit{5}}x{\isadigit{9}}lr{\isachardoublequoteclose}\isanewline
\ \ \ \ \ \ \ \ \isacommand{have}\isamarkupfalse%
\ ps\isactrlsub {\isadigit{1}}{\isacharunderscore}{\kern0pt}prems{\isacharcolon}{\kern0pt}\ {\isachardoublequoteopen}knights{\isacharunderscore}{\kern0pt}path\ b{\isadigit{5}}x{\isadigit{9}}\ {\isacharquery}{\kern0pt}ps\isactrlsub {\isadigit{1}}{\isachardoublequoteclose}\ {\isachardoublequoteopen}hd\ {\isacharquery}{\kern0pt}ps\isactrlsub {\isadigit{1}}\ {\isacharequal}{\kern0pt}\ {\isacharparenleft}{\kern0pt}{\isadigit{1}}{\isacharcomma}{\kern0pt}{\isadigit{1}}{\isacharparenright}{\kern0pt}{\isachardoublequoteclose}\ {\isachardoublequoteopen}last\ {\isacharquery}{\kern0pt}ps\isactrlsub {\isadigit{1}}\ {\isacharequal}{\kern0pt}\ {\isacharparenleft}{\kern0pt}{\isadigit{2}}{\isacharcomma}{\kern0pt}{\isadigit{8}}{\isacharparenright}{\kern0pt}{\isachardoublequoteclose}\isanewline
\ \ \ \ \ \ \ \ \ \ \isacommand{using}\isamarkupfalse%
\ kp{\isacharunderscore}{\kern0pt}{\isadigit{5}}x{\isadigit{9}}{\isacharunderscore}{\kern0pt}lr\ \isacommand{by}\isamarkupfalse%
\ simp\ eval{\isacharplus}{\kern0pt}\isanewline
\isanewline
\ \ \ \ \ \ \ \ \isacommand{have}\isamarkupfalse%
\ {\isachardoublequoteopen}{\isadigit{1}}{\isadigit{6}}\ {\isacharless}{\kern0pt}\ length\ {\isacharquery}{\kern0pt}ps\isactrlsub {\isadigit{1}}{\isachardoublequoteclose}\ {\isachardoublequoteopen}last\ {\isacharparenleft}{\kern0pt}take\ {\isadigit{1}}{\isadigit{6}}\ {\isacharquery}{\kern0pt}ps\isactrlsub {\isadigit{1}}{\isacharparenright}{\kern0pt}\ {\isacharequal}{\kern0pt}\ {\isacharparenleft}{\kern0pt}{\isadigit{5}}{\isacharcomma}{\kern0pt}{\isadigit{4}}{\isacharparenright}{\kern0pt}{\isachardoublequoteclose}\ {\isachardoublequoteopen}hd\ {\isacharparenleft}{\kern0pt}drop\ {\isadigit{1}}{\isadigit{6}}\ {\isacharquery}{\kern0pt}ps\isactrlsub {\isadigit{1}}{\isacharparenright}{\kern0pt}\ {\isacharequal}{\kern0pt}\ {\isacharparenleft}{\kern0pt}{\isadigit{4}}{\isacharcomma}{\kern0pt}{\isadigit{2}}{\isacharparenright}{\kern0pt}{\isachardoublequoteclose}\ \isacommand{by}\isamarkupfalse%
\ eval{\isacharplus}{\kern0pt}\isanewline
\ \ \ \ \ \ \ \ \isacommand{then}\isamarkupfalse%
\ \isacommand{have}\isamarkupfalse%
\ si{\isacharcolon}{\kern0pt}\ {\isachardoublequoteopen}step{\isacharunderscore}{\kern0pt}in\ {\isacharquery}{\kern0pt}ps\isactrlsub {\isadigit{1}}\ {\isacharparenleft}{\kern0pt}{\isadigit{5}}{\isacharcomma}{\kern0pt}{\isadigit{4}}{\isacharparenright}{\kern0pt}\ {\isacharparenleft}{\kern0pt}{\isadigit{4}}{\isacharcomma}{\kern0pt}{\isadigit{2}}{\isacharparenright}{\kern0pt}{\isachardoublequoteclose}\isanewline
\ \ \ \ \ \ \ \ \ \ \isacommand{unfolding}\isamarkupfalse%
\ step{\isacharunderscore}{\kern0pt}in{\isacharunderscore}{\kern0pt}def\ \isacommand{using}\isamarkupfalse%
\ zero{\isacharunderscore}{\kern0pt}less{\isacharunderscore}{\kern0pt}numeral\ \isacommand{by}\isamarkupfalse%
\ blast\isanewline
\isanewline
\ \ \ \ \ \ \ \ \isacommand{have}\isamarkupfalse%
\ vs{\isacharcolon}{\kern0pt}\ {\isachardoublequoteopen}valid{\isacharunderscore}{\kern0pt}step\ {\isacharparenleft}{\kern0pt}{\isadigit{5}}{\isacharcomma}{\kern0pt}{\isadigit{4}}{\isacharparenright}{\kern0pt}\ {\isacharparenleft}{\kern0pt}int\ {\isadigit{5}}{\isacharplus}{\kern0pt}{\isadigit{2}}{\isacharcomma}{\kern0pt}{\isadigit{3}}{\isacharparenright}{\kern0pt}{\isachardoublequoteclose}\ {\isachardoublequoteopen}valid{\isacharunderscore}{\kern0pt}step\ {\isacharparenleft}{\kern0pt}int\ {\isadigit{5}}{\isacharplus}{\kern0pt}{\isadigit{1}}{\isacharcomma}{\kern0pt}{\isadigit{1}}{\isacharparenright}{\kern0pt}\ {\isacharparenleft}{\kern0pt}{\isadigit{4}}{\isacharcomma}{\kern0pt}{\isadigit{2}}{\isacharparenright}{\kern0pt}{\isachardoublequoteclose}\isanewline
\ \ \ \ \ \ \ \ \ \ \isacommand{unfolding}\isamarkupfalse%
\ valid{\isacharunderscore}{\kern0pt}step{\isacharunderscore}{\kern0pt}def\ \isacommand{by}\isamarkupfalse%
\ auto\isanewline
\isanewline
\ \ \ \ \ \ \ \ \isacommand{obtain}\isamarkupfalse%
\ ps\ \isakeyword{where}\ {\isachardoublequoteopen}knights{\isacharunderscore}{\kern0pt}path\ {\isacharparenleft}{\kern0pt}board\ n\ m{\isacharparenright}{\kern0pt}\ ps{\isachardoublequoteclose}\ {\isachardoublequoteopen}hd\ ps\ {\isacharequal}{\kern0pt}\ {\isacharparenleft}{\kern0pt}{\isadigit{1}}{\isacharcomma}{\kern0pt}{\isadigit{1}}{\isacharparenright}{\kern0pt}{\isachardoublequoteclose}\ {\isachardoublequoteopen}last\ ps\ {\isacharequal}{\kern0pt}\ {\isacharparenleft}{\kern0pt}{\isadigit{2}}{\isacharcomma}{\kern0pt}{\isadigit{8}}{\isacharparenright}{\kern0pt}{\isachardoublequoteclose}\isanewline
\ \ \ \ \ \ \ \ \ \ \isacommand{using}\isamarkupfalse%
\ {\isacartoucheopen}n{\isacharminus}{\kern0pt}{\isadigit{5}}\ {\isasymge}\ {\isadigit{5}}{\isacartoucheclose}\ ps\isactrlsub {\isadigit{1}}{\isacharunderscore}{\kern0pt}prems\ ps\isactrlsub {\isadigit{2}}T{\isacharunderscore}{\kern0pt}prems\ si\ vs\isanewline
\ \ \ \ \ \ \ \ \ \ \ \ \ \ knights{\isacharunderscore}{\kern0pt}path{\isacharunderscore}{\kern0pt}split{\isacharunderscore}{\kern0pt}concatT{\isacharbrackleft}{\kern0pt}of\ {\isadigit{5}}\ m\ {\isacharquery}{\kern0pt}ps\isactrlsub {\isadigit{1}}\ {\isachardoublequoteopen}n{\isacharminus}{\kern0pt}{\isadigit{5}}{\isachardoublequoteclose}\ {\isacharquery}{\kern0pt}ps\isactrlsub {\isadigit{2}}T\ {\isachardoublequoteopen}{\isacharparenleft}{\kern0pt}{\isadigit{5}}{\isacharcomma}{\kern0pt}{\isadigit{4}}{\isacharparenright}{\kern0pt}{\isachardoublequoteclose}\ {\isachardoublequoteopen}{\isacharparenleft}{\kern0pt}{\isadigit{4}}{\isacharcomma}{\kern0pt}{\isadigit{2}}{\isacharparenright}{\kern0pt}{\isachardoublequoteclose}{\isacharbrackright}{\kern0pt}\ \isacommand{by}\isamarkupfalse%
\ auto\isanewline
\ \ \ \ \ \ \ \ \isacommand{then}\isamarkupfalse%
\ \isacommand{show}\isamarkupfalse%
\ {\isacharquery}{\kern0pt}thesis\isanewline
\ \ \ \ \ \ \ \ \ \ \isacommand{using}\isamarkupfalse%
\ mirror{\isadigit{1}}{\isacharunderscore}{\kern0pt}knights{\isacharunderscore}{\kern0pt}path\ hd{\isacharunderscore}{\kern0pt}mirror{\isadigit{1}}\ last{\isacharunderscore}{\kern0pt}mirror{\isadigit{1}}\ \isacommand{by}\isamarkupfalse%
\ fastforce\isanewline
\ \ \ \ \ \ \isacommand{qed}\isamarkupfalse%
\isanewline
\ \ \ \ \isacommand{next}\isamarkupfalse%
\isanewline
\ \ \ \ \ \ \isacommand{let}\isamarkupfalse%
\ {\isacharquery}{\kern0pt}b\isactrlsub {\isadigit{1}}{\isacharequal}{\kern0pt}{\isachardoublequoteopen}board\ n\ {\isadigit{5}}{\isachardoublequoteclose}\isanewline
\ \ \ \ \ \ \isacommand{let}\isamarkupfalse%
\ {\isacharquery}{\kern0pt}b\isactrlsub {\isadigit{2}}{\isacharequal}{\kern0pt}{\isachardoublequoteopen}board\ n\ {\isacharparenleft}{\kern0pt}m{\isacharminus}{\kern0pt}{\isadigit{5}}{\isacharparenright}{\kern0pt}{\isachardoublequoteclose}\isanewline
\ \ \ \ \ \ \isacommand{assume}\isamarkupfalse%
\ {\isachardoublequoteopen}m\ {\isasymge}\ {\isadigit{1}}{\isadigit{0}}{\isachardoublequoteclose}\isanewline
\ \ \ \ \ \ \isacommand{then}\isamarkupfalse%
\ \isacommand{have}\isamarkupfalse%
\ {\isachardoublequoteopen}n{\isacharplus}{\kern0pt}{\isadigit{5}}\ {\isacharless}{\kern0pt}\ x{\isachardoublequoteclose}\ {\isachardoublequoteopen}{\isadigit{5}}\ {\isasymle}\ min\ n\ {\isadigit{5}}{\isachardoublequoteclose}\ {\isachardoublequoteopen}n{\isacharplus}{\kern0pt}{\isacharparenleft}{\kern0pt}m{\isacharminus}{\kern0pt}{\isadigit{5}}{\isacharparenright}{\kern0pt}\ {\isacharless}{\kern0pt}\ x{\isachardoublequoteclose}\ {\isachardoublequoteopen}{\isadigit{5}}\ {\isasymle}\ min\ n\ {\isacharparenleft}{\kern0pt}m{\isacharminus}{\kern0pt}{\isadigit{5}}{\isacharparenright}{\kern0pt}{\isachardoublequoteclose}\ \isanewline
\ \ \ \ \ \ \ \ \isacommand{using}\isamarkupfalse%
\ less\ \isacommand{by}\isamarkupfalse%
\ auto\isanewline
\ \ \ \ \ \ \isacommand{then}\isamarkupfalse%
\ \isacommand{obtain}\isamarkupfalse%
\ ps\isactrlsub {\isadigit{1}}\ ps\isactrlsub {\isadigit{2}}\ \isakeyword{where}\ kp{\isacharunderscore}{\kern0pt}prems{\isacharcolon}{\kern0pt}\ \isanewline
\ \ \ \ \ \ \ \ \ \ {\isachardoublequoteopen}knights{\isacharunderscore}{\kern0pt}path\ {\isacharquery}{\kern0pt}b\isactrlsub {\isadigit{1}}\ ps\isactrlsub {\isadigit{1}}{\isachardoublequoteclose}\ {\isachardoublequoteopen}hd\ ps\isactrlsub {\isadigit{1}}\ {\isacharequal}{\kern0pt}\ {\isacharparenleft}{\kern0pt}int\ n{\isacharcomma}{\kern0pt}{\isadigit{1}}{\isacharparenright}{\kern0pt}{\isachardoublequoteclose}\ {\isachardoublequoteopen}last\ ps\isactrlsub {\isadigit{1}}\ {\isacharequal}{\kern0pt}\ {\isacharparenleft}{\kern0pt}int\ n{\isacharminus}{\kern0pt}{\isadigit{1}}{\isacharcomma}{\kern0pt}{\isadigit{4}}{\isacharparenright}{\kern0pt}{\isachardoublequoteclose}\isanewline
\ \ \ \ \ \ \ \ \ \ {\isachardoublequoteopen}knights{\isacharunderscore}{\kern0pt}path\ {\isacharparenleft}{\kern0pt}board\ n\ {\isacharparenleft}{\kern0pt}m{\isacharminus}{\kern0pt}{\isadigit{5}}{\isacharparenright}{\kern0pt}{\isacharparenright}{\kern0pt}\ ps\isactrlsub {\isadigit{2}}{\isachardoublequoteclose}\ {\isachardoublequoteopen}hd\ ps\isactrlsub {\isadigit{2}}\ {\isacharequal}{\kern0pt}\ {\isacharparenleft}{\kern0pt}int\ n{\isacharcomma}{\kern0pt}{\isadigit{1}}{\isacharparenright}{\kern0pt}{\isachardoublequoteclose}\ {\isachardoublequoteopen}last\ ps\isactrlsub {\isadigit{2}}\ {\isacharequal}{\kern0pt}\ {\isacharparenleft}{\kern0pt}int\ n{\isacharminus}{\kern0pt}{\isadigit{1}}{\isacharcomma}{\kern0pt}int\ {\isacharparenleft}{\kern0pt}m{\isacharminus}{\kern0pt}{\isadigit{5}}{\isacharparenright}{\kern0pt}{\isacharminus}{\kern0pt}{\isadigit{1}}{\isacharparenright}{\kern0pt}{\isachardoublequoteclose}\isanewline
\ \ \ \ \ \ \ \ \isacommand{using}\isamarkupfalse%
\ less{\isachardot}{\kern0pt}prems\ less{\isachardot}{\kern0pt}IH{\isacharbrackleft}{\kern0pt}of\ {\isachardoublequoteopen}n{\isacharplus}{\kern0pt}{\isadigit{5}}{\isachardoublequoteclose}\ n\ {\isachardoublequoteopen}{\isadigit{5}}{\isachardoublequoteclose}{\isacharbrackright}{\kern0pt}\ less{\isachardot}{\kern0pt}IH{\isacharbrackleft}{\kern0pt}of\ {\isachardoublequoteopen}n{\isacharplus}{\kern0pt}{\isacharparenleft}{\kern0pt}m{\isacharminus}{\kern0pt}{\isadigit{5}}{\isacharparenright}{\kern0pt}{\isachardoublequoteclose}\ n\ {\isachardoublequoteopen}m{\isacharminus}{\kern0pt}{\isadigit{5}}{\isachardoublequoteclose}{\isacharbrackright}{\kern0pt}\ \isacommand{by}\isamarkupfalse%
\ auto\isanewline
\ \ \ \ \ \ \isacommand{let}\isamarkupfalse%
\ {\isacharquery}{\kern0pt}ps{\isacharequal}{\kern0pt}{\isachardoublequoteopen}ps\isactrlsub {\isadigit{1}}{\isacharat}{\kern0pt}trans{\isacharunderscore}{\kern0pt}path\ {\isacharparenleft}{\kern0pt}{\isadigit{0}}{\isacharcomma}{\kern0pt}int\ {\isadigit{5}}{\isacharparenright}{\kern0pt}\ ps\isactrlsub {\isadigit{2}}{\isachardoublequoteclose}\isanewline
\ \ \ \ \ \ \isacommand{have}\isamarkupfalse%
\ {\isachardoublequoteopen}valid{\isacharunderscore}{\kern0pt}step\ {\isacharparenleft}{\kern0pt}last\ ps\isactrlsub {\isadigit{1}}{\isacharparenright}{\kern0pt}\ {\isacharparenleft}{\kern0pt}int\ n{\isacharcomma}{\kern0pt}int\ {\isadigit{5}}{\isacharplus}{\kern0pt}{\isadigit{1}}{\isacharparenright}{\kern0pt}{\isachardoublequoteclose}\ \isanewline
\ \ \ \ \ \ \ \ \isacommand{unfolding}\isamarkupfalse%
\ valid{\isacharunderscore}{\kern0pt}step{\isacharunderscore}{\kern0pt}def\ \isacommand{using}\isamarkupfalse%
\ kp{\isacharunderscore}{\kern0pt}prems\ \isacommand{by}\isamarkupfalse%
\ auto\isanewline
\ \ \ \ \ \ \isacommand{then}\isamarkupfalse%
\ \isacommand{have}\isamarkupfalse%
\ {\isachardoublequoteopen}knights{\isacharunderscore}{\kern0pt}path\ {\isacharparenleft}{\kern0pt}board\ n\ m{\isacharparenright}{\kern0pt}\ {\isacharquery}{\kern0pt}ps{\isachardoublequoteclose}\ {\isachardoublequoteopen}hd\ {\isacharquery}{\kern0pt}ps\ {\isacharequal}{\kern0pt}\ {\isacharparenleft}{\kern0pt}int\ n{\isacharcomma}{\kern0pt}{\isadigit{1}}{\isacharparenright}{\kern0pt}{\isachardoublequoteclose}\ {\isachardoublequoteopen}last\ {\isacharquery}{\kern0pt}ps\ {\isacharequal}{\kern0pt}\ {\isacharparenleft}{\kern0pt}int\ n{\isacharminus}{\kern0pt}{\isadigit{1}}{\isacharcomma}{\kern0pt}int\ m{\isacharminus}{\kern0pt}{\isadigit{1}}{\isacharparenright}{\kern0pt}{\isachardoublequoteclose}\isanewline
\ \ \ \ \ \ \ \ \isacommand{using}\isamarkupfalse%
\ {\isacartoucheopen}m\ {\isasymge}\ {\isadigit{1}}{\isadigit{0}}{\isacartoucheclose}\ kp{\isacharunderscore}{\kern0pt}prems\ knights{\isacharunderscore}{\kern0pt}path{\isacharunderscore}{\kern0pt}concat{\isacharbrackleft}{\kern0pt}of\ n\ {\isadigit{5}}\ ps\isactrlsub {\isadigit{1}}\ {\isachardoublequoteopen}m{\isacharminus}{\kern0pt}{\isadigit{5}}{\isachardoublequoteclose}\ ps\isactrlsub {\isadigit{2}}{\isacharbrackright}{\kern0pt}\ \isanewline
\ \ \ \ \ \ \ \ \ \ \ \ \ \ knights{\isacharunderscore}{\kern0pt}path{\isacharunderscore}{\kern0pt}non{\isacharunderscore}{\kern0pt}nil\ trans{\isacharunderscore}{\kern0pt}path{\isacharunderscore}{\kern0pt}non{\isacharunderscore}{\kern0pt}nil\ last{\isacharunderscore}{\kern0pt}trans{\isacharunderscore}{\kern0pt}path\ \isacommand{by}\isamarkupfalse%
\ auto\isanewline
\ \ \ \ \ \ \isacommand{then}\isamarkupfalse%
\ \isacommand{show}\isamarkupfalse%
\ {\isacharquery}{\kern0pt}thesis\ \isacommand{by}\isamarkupfalse%
\ auto\isanewline
\ \ \ \ \isacommand{qed}\isamarkupfalse%
\isanewline
\ \ \isacommand{qed}\isamarkupfalse%
\isanewline
\isacommand{qed}\isamarkupfalse%
%
\endisatagproof
{\isafoldproof}%
%
\isadelimproof
%
\endisadelimproof
%
\begin{isamarkuptext}%
Auxiliary lemma that constructs a Knight's circuit if \isa{m\ {\isasymge}\ {\isadigit{5}}} and \isa{n\ {\isasymge}\ {\isadigit{1}}{\isadigit{0}}\ {\isasymand}\ even\ n}.%
\end{isamarkuptext}\isamarkuptrue%
\isacommand{lemma}\isamarkupfalse%
\ knights{\isacharunderscore}{\kern0pt}circuit{\isacharunderscore}{\kern0pt}exists{\isacharunderscore}{\kern0pt}n{\isacharunderscore}{\kern0pt}even{\isacharunderscore}{\kern0pt}gr{\isacharunderscore}{\kern0pt}{\isadigit{1}}{\isadigit{0}}{\isacharcolon}{\kern0pt}\ \isanewline
\ \ \isakeyword{assumes}\ {\isachardoublequoteopen}n\ {\isasymge}\ {\isadigit{1}}{\isadigit{0}}\ {\isasymand}\ even\ n{\isachardoublequoteclose}\ {\isachardoublequoteopen}m\ {\isasymge}\ {\isadigit{5}}{\isachardoublequoteclose}\isanewline
\ \ \isakeyword{shows}\ {\isachardoublequoteopen}{\isasymexists}ps{\isachardot}{\kern0pt}\ knights{\isacharunderscore}{\kern0pt}circuit\ {\isacharparenleft}{\kern0pt}board\ n\ m{\isacharparenright}{\kern0pt}\ ps{\isachardoublequoteclose}\isanewline
%
\isadelimproof
\ \ %
\endisadelimproof
%
\isatagproof
\isacommand{using}\isamarkupfalse%
\ assms\isanewline
\isacommand{proof}\isamarkupfalse%
\ {\isacharminus}{\kern0pt}\isanewline
\ \ \isacommand{obtain}\isamarkupfalse%
\ ps\isactrlsub {\isadigit{1}}\ \isakeyword{where}\ ps\isactrlsub {\isadigit{1}}{\isacharunderscore}{\kern0pt}prems{\isacharcolon}{\kern0pt}\ {\isachardoublequoteopen}knights{\isacharunderscore}{\kern0pt}path\ {\isacharparenleft}{\kern0pt}board\ {\isadigit{5}}\ m{\isacharparenright}{\kern0pt}\ ps\isactrlsub {\isadigit{1}}{\isachardoublequoteclose}\ {\isachardoublequoteopen}hd\ ps\isactrlsub {\isadigit{1}}\ {\isacharequal}{\kern0pt}\ {\isacharparenleft}{\kern0pt}{\isadigit{1}}{\isacharcomma}{\kern0pt}{\isadigit{1}}{\isacharparenright}{\kern0pt}{\isachardoublequoteclose}\ \isanewline
\ \ \ \ \ \ {\isachardoublequoteopen}last\ ps\isactrlsub {\isadigit{1}}\ {\isacharequal}{\kern0pt}\ {\isacharparenleft}{\kern0pt}{\isadigit{2}}{\isacharcomma}{\kern0pt}int\ m{\isacharminus}{\kern0pt}{\isadigit{1}}{\isacharparenright}{\kern0pt}{\isachardoublequoteclose}\isanewline
\ \ \ \ \isacommand{using}\isamarkupfalse%
\ assms\ knights{\isacharunderscore}{\kern0pt}path{\isacharunderscore}{\kern0pt}{\isadigit{5}}xm{\isacharunderscore}{\kern0pt}exists\ \isacommand{by}\isamarkupfalse%
\ auto\isanewline
\ \ \isacommand{let}\isamarkupfalse%
\ {\isacharquery}{\kern0pt}ps\isactrlsub {\isadigit{1}}{\isacharprime}{\kern0pt}{\isacharequal}{\kern0pt}{\isachardoublequoteopen}trans{\isacharunderscore}{\kern0pt}path\ {\isacharparenleft}{\kern0pt}int\ {\isacharparenleft}{\kern0pt}n{\isacharminus}{\kern0pt}{\isadigit{5}}{\isacharparenright}{\kern0pt}{\isacharcomma}{\kern0pt}{\isadigit{0}}{\isacharparenright}{\kern0pt}\ ps\isactrlsub {\isadigit{1}}{\isachardoublequoteclose}\isanewline
\ \ \isacommand{let}\isamarkupfalse%
\ {\isacharquery}{\kern0pt}b{\isadigit{5}}xm{\isacharprime}{\kern0pt}{\isacharequal}{\kern0pt}{\isachardoublequoteopen}trans{\isacharunderscore}{\kern0pt}board\ {\isacharparenleft}{\kern0pt}int\ {\isacharparenleft}{\kern0pt}n{\isacharminus}{\kern0pt}{\isadigit{5}}{\isacharparenright}{\kern0pt}{\isacharcomma}{\kern0pt}{\isadigit{0}}{\isacharparenright}{\kern0pt}\ {\isacharparenleft}{\kern0pt}board\ {\isadigit{5}}\ m{\isacharparenright}{\kern0pt}{\isachardoublequoteclose}\isanewline
\ \ \isacommand{have}\isamarkupfalse%
\ ps\isactrlsub {\isadigit{1}}{\isacharprime}{\kern0pt}{\isacharunderscore}{\kern0pt}prems{\isacharcolon}{\kern0pt}\ {\isachardoublequoteopen}knights{\isacharunderscore}{\kern0pt}path\ {\isacharquery}{\kern0pt}b{\isadigit{5}}xm{\isacharprime}{\kern0pt}\ {\isacharquery}{\kern0pt}ps\isactrlsub {\isadigit{1}}{\isacharprime}{\kern0pt}{\isachardoublequoteclose}\ {\isachardoublequoteopen}hd\ {\isacharquery}{\kern0pt}ps\isactrlsub {\isadigit{1}}{\isacharprime}{\kern0pt}\ {\isacharequal}{\kern0pt}\ {\isacharparenleft}{\kern0pt}int\ {\isacharparenleft}{\kern0pt}n{\isacharminus}{\kern0pt}{\isadigit{5}}{\isacharparenright}{\kern0pt}{\isacharplus}{\kern0pt}{\isadigit{1}}{\isacharcomma}{\kern0pt}{\isadigit{1}}{\isacharparenright}{\kern0pt}{\isachardoublequoteclose}\ \isanewline
\ \ \ \ \ \ {\isachardoublequoteopen}last\ {\isacharquery}{\kern0pt}ps\isactrlsub {\isadigit{1}}{\isacharprime}{\kern0pt}\ {\isacharequal}{\kern0pt}\ {\isacharparenleft}{\kern0pt}int\ {\isacharparenleft}{\kern0pt}n{\isacharminus}{\kern0pt}{\isadigit{5}}{\isacharparenright}{\kern0pt}{\isacharplus}{\kern0pt}{\isadigit{2}}{\isacharcomma}{\kern0pt}int\ m{\isacharminus}{\kern0pt}{\isadigit{1}}{\isacharparenright}{\kern0pt}{\isachardoublequoteclose}\isanewline
\ \ \ \ \isacommand{using}\isamarkupfalse%
\ ps\isactrlsub {\isadigit{1}}{\isacharunderscore}{\kern0pt}prems\ trans{\isacharunderscore}{\kern0pt}knights{\isacharunderscore}{\kern0pt}path\ knights{\isacharunderscore}{\kern0pt}path{\isacharunderscore}{\kern0pt}non{\isacharunderscore}{\kern0pt}nil\ hd{\isacharunderscore}{\kern0pt}trans{\isacharunderscore}{\kern0pt}path\ last{\isacharunderscore}{\kern0pt}trans{\isacharunderscore}{\kern0pt}path\ \isacommand{by}\isamarkupfalse%
\ auto\isanewline
\ \ \ \ \isanewline
\ \ \isacommand{assume}\isamarkupfalse%
\ {\isachardoublequoteopen}n\ {\isasymge}\ {\isadigit{1}}{\isadigit{0}}\ {\isasymand}\ even\ n{\isachardoublequoteclose}\isanewline
\ \ \isacommand{then}\isamarkupfalse%
\ \isacommand{have}\isamarkupfalse%
\ {\isachardoublequoteopen}odd\ {\isacharparenleft}{\kern0pt}n{\isacharminus}{\kern0pt}{\isadigit{5}}{\isacharparenright}{\kern0pt}{\isachardoublequoteclose}\ {\isachardoublequoteopen}min\ {\isacharparenleft}{\kern0pt}n{\isacharminus}{\kern0pt}{\isadigit{5}}{\isacharparenright}{\kern0pt}\ m\ {\isasymge}\ {\isadigit{5}}{\isachardoublequoteclose}\ \isacommand{using}\isamarkupfalse%
\ assms\ \isacommand{by}\isamarkupfalse%
\ auto\isanewline
\ \ \isacommand{then}\isamarkupfalse%
\ \isacommand{obtain}\isamarkupfalse%
\ ps\isactrlsub {\isadigit{2}}\ \isakeyword{where}\ ps\isactrlsub {\isadigit{2}}{\isacharunderscore}{\kern0pt}prems{\isacharcolon}{\kern0pt}\ {\isachardoublequoteopen}knights{\isacharunderscore}{\kern0pt}path\ {\isacharparenleft}{\kern0pt}board\ {\isacharparenleft}{\kern0pt}n{\isacharminus}{\kern0pt}{\isadigit{5}}{\isacharparenright}{\kern0pt}\ m{\isacharparenright}{\kern0pt}\ ps\isactrlsub {\isadigit{2}}{\isachardoublequoteclose}\ {\isachardoublequoteopen}hd\ ps\isactrlsub {\isadigit{2}}\ {\isacharequal}{\kern0pt}\ {\isacharparenleft}{\kern0pt}int\ {\isacharparenleft}{\kern0pt}n{\isacharminus}{\kern0pt}{\isadigit{5}}{\isacharparenright}{\kern0pt}{\isacharcomma}{\kern0pt}{\isadigit{1}}{\isacharparenright}{\kern0pt}{\isachardoublequoteclose}\ \isanewline
\ \ \ \ \ \ {\isachardoublequoteopen}last\ ps\isactrlsub {\isadigit{2}}\ {\isacharequal}{\kern0pt}\ {\isacharparenleft}{\kern0pt}int\ {\isacharparenleft}{\kern0pt}n{\isacharminus}{\kern0pt}{\isadigit{5}}{\isacharparenright}{\kern0pt}{\isacharminus}{\kern0pt}{\isadigit{1}}{\isacharcomma}{\kern0pt}int\ m{\isacharminus}{\kern0pt}{\isadigit{1}}{\isacharparenright}{\kern0pt}{\isachardoublequoteclose}\isanewline
\ \ \ \ \isacommand{using}\isamarkupfalse%
\ knights{\isacharunderscore}{\kern0pt}path{\isacharunderscore}{\kern0pt}odd{\isacharunderscore}{\kern0pt}n{\isacharunderscore}{\kern0pt}exists{\isacharbrackleft}{\kern0pt}of\ {\isachardoublequoteopen}n{\isacharminus}{\kern0pt}{\isadigit{5}}{\isachardoublequoteclose}\ m{\isacharbrackright}{\kern0pt}\ \isacommand{by}\isamarkupfalse%
\ auto\isanewline
\ \ \isacommand{let}\isamarkupfalse%
\ {\isacharquery}{\kern0pt}ps\isactrlsub {\isadigit{2}}{\isacharprime}{\kern0pt}{\isacharequal}{\kern0pt}{\isachardoublequoteopen}mirror{\isadigit{2}}\ ps\isactrlsub {\isadigit{2}}{\isachardoublequoteclose}\isanewline
\ \ \isacommand{have}\isamarkupfalse%
\ ps\isactrlsub {\isadigit{2}}{\isacharprime}{\kern0pt}{\isacharunderscore}{\kern0pt}prems{\isacharcolon}{\kern0pt}\ {\isachardoublequoteopen}knights{\isacharunderscore}{\kern0pt}path\ {\isacharparenleft}{\kern0pt}board\ {\isacharparenleft}{\kern0pt}n{\isacharminus}{\kern0pt}{\isadigit{5}}{\isacharparenright}{\kern0pt}\ m{\isacharparenright}{\kern0pt}\ {\isacharquery}{\kern0pt}ps\isactrlsub {\isadigit{2}}{\isacharprime}{\kern0pt}{\isachardoublequoteclose}\ {\isachardoublequoteopen}hd\ {\isacharquery}{\kern0pt}ps\isactrlsub {\isadigit{2}}{\isacharprime}{\kern0pt}\ {\isacharequal}{\kern0pt}\ {\isacharparenleft}{\kern0pt}int\ {\isacharparenleft}{\kern0pt}n{\isacharminus}{\kern0pt}{\isadigit{5}}{\isacharparenright}{\kern0pt}{\isacharcomma}{\kern0pt}int\ m{\isacharparenright}{\kern0pt}{\isachardoublequoteclose}\ \isanewline
\ \ \ \ \ \ {\isachardoublequoteopen}last\ {\isacharquery}{\kern0pt}ps\isactrlsub {\isadigit{2}}{\isacharprime}{\kern0pt}\ {\isacharequal}{\kern0pt}\ {\isacharparenleft}{\kern0pt}int\ {\isacharparenleft}{\kern0pt}n{\isacharminus}{\kern0pt}{\isadigit{5}}{\isacharparenright}{\kern0pt}{\isacharminus}{\kern0pt}{\isadigit{1}}{\isacharcomma}{\kern0pt}{\isadigit{2}}{\isacharparenright}{\kern0pt}{\isachardoublequoteclose}\isanewline
\ \ \ \ \isacommand{using}\isamarkupfalse%
\ ps\isactrlsub {\isadigit{2}}{\isacharunderscore}{\kern0pt}prems\ mirror{\isadigit{2}}{\isacharunderscore}{\kern0pt}knights{\isacharunderscore}{\kern0pt}path\ hd{\isacharunderscore}{\kern0pt}mirror{\isadigit{2}}\ last{\isacharunderscore}{\kern0pt}mirror{\isadigit{2}}\ \isacommand{by}\isamarkupfalse%
\ auto\isanewline
\isanewline
\ \ \isacommand{have}\isamarkupfalse%
\ inter{\isacharcolon}{\kern0pt}\ {\isachardoublequoteopen}{\isacharquery}{\kern0pt}b{\isadigit{5}}xm{\isacharprime}{\kern0pt}\ {\isasyminter}\ board\ {\isacharparenleft}{\kern0pt}n{\isacharminus}{\kern0pt}{\isadigit{5}}{\isacharparenright}{\kern0pt}\ m\ {\isacharequal}{\kern0pt}\ {\isacharbraceleft}{\kern0pt}{\isacharbraceright}{\kern0pt}{\isachardoublequoteclose}\ \isanewline
\ \ \ \ \isacommand{unfolding}\isamarkupfalse%
\ trans{\isacharunderscore}{\kern0pt}board{\isacharunderscore}{\kern0pt}def\ board{\isacharunderscore}{\kern0pt}def\ \isacommand{by}\isamarkupfalse%
\ auto\ \isanewline
\isanewline
\ \ \isacommand{have}\isamarkupfalse%
\ union{\isacharcolon}{\kern0pt}\ {\isachardoublequoteopen}board\ n\ m\ {\isacharequal}{\kern0pt}\ {\isacharquery}{\kern0pt}b{\isadigit{5}}xm{\isacharprime}{\kern0pt}\ {\isasymunion}\ board\ {\isacharparenleft}{\kern0pt}n{\isacharminus}{\kern0pt}{\isadigit{5}}{\isacharparenright}{\kern0pt}\ m{\isachardoublequoteclose}\isanewline
\ \ \ \ \isacommand{using}\isamarkupfalse%
\ {\isacartoucheopen}n\ {\isasymge}\ {\isadigit{1}}{\isadigit{0}}\ {\isasymand}\ even\ n{\isacartoucheclose}\ board{\isacharunderscore}{\kern0pt}concatT{\isacharbrackleft}{\kern0pt}of\ {\isachardoublequoteopen}n{\isacharminus}{\kern0pt}{\isadigit{5}}{\isachardoublequoteclose}\ m\ {\isadigit{5}}{\isacharbrackright}{\kern0pt}\ \isacommand{by}\isamarkupfalse%
\ auto\isanewline
\isanewline
\ \ \isacommand{have}\isamarkupfalse%
\ vs{\isacharcolon}{\kern0pt}\ {\isachardoublequoteopen}valid{\isacharunderscore}{\kern0pt}step\ {\isacharparenleft}{\kern0pt}last\ {\isacharquery}{\kern0pt}ps\isactrlsub {\isadigit{1}}{\isacharprime}{\kern0pt}{\isacharparenright}{\kern0pt}\ {\isacharparenleft}{\kern0pt}hd\ {\isacharquery}{\kern0pt}ps\isactrlsub {\isadigit{2}}{\isacharprime}{\kern0pt}{\isacharparenright}{\kern0pt}{\isachardoublequoteclose}\ {\isachardoublequoteopen}valid{\isacharunderscore}{\kern0pt}step\ {\isacharparenleft}{\kern0pt}last\ {\isacharquery}{\kern0pt}ps\isactrlsub {\isadigit{2}}{\isacharprime}{\kern0pt}{\isacharparenright}{\kern0pt}\ {\isacharparenleft}{\kern0pt}hd\ {\isacharquery}{\kern0pt}ps\isactrlsub {\isadigit{1}}{\isacharprime}{\kern0pt}{\isacharparenright}{\kern0pt}{\isachardoublequoteclose}\isanewline
\ \ \ \ \isacommand{using}\isamarkupfalse%
\ ps\isactrlsub {\isadigit{1}}{\isacharprime}{\kern0pt}{\isacharunderscore}{\kern0pt}prems\ ps\isactrlsub {\isadigit{2}}{\isacharprime}{\kern0pt}{\isacharunderscore}{\kern0pt}prems\ \isacommand{unfolding}\isamarkupfalse%
\ valid{\isacharunderscore}{\kern0pt}step{\isacharunderscore}{\kern0pt}def\ \isacommand{by}\isamarkupfalse%
\ auto\isanewline
\isanewline
\ \ \isacommand{let}\isamarkupfalse%
\ {\isacharquery}{\kern0pt}ps{\isacharequal}{\kern0pt}{\isachardoublequoteopen}{\isacharquery}{\kern0pt}ps\isactrlsub {\isadigit{1}}{\isacharprime}{\kern0pt}\ {\isacharat}{\kern0pt}\ {\isacharquery}{\kern0pt}ps\isactrlsub {\isadigit{2}}{\isacharprime}{\kern0pt}{\isachardoublequoteclose}\isanewline
\ \ \isacommand{have}\isamarkupfalse%
\ {\isachardoublequoteopen}last\ {\isacharquery}{\kern0pt}ps\ {\isacharequal}{\kern0pt}\ last\ {\isacharquery}{\kern0pt}ps\isactrlsub {\isadigit{2}}{\isacharprime}{\kern0pt}{\isachardoublequoteclose}\ {\isachardoublequoteopen}hd\ {\isacharquery}{\kern0pt}ps\ {\isacharequal}{\kern0pt}\ hd\ {\isacharquery}{\kern0pt}ps\isactrlsub {\isadigit{1}}{\isacharprime}{\kern0pt}{\isachardoublequoteclose}\isanewline
\ \ \ \ \isacommand{using}\isamarkupfalse%
\ ps\isactrlsub {\isadigit{1}}{\isacharprime}{\kern0pt}{\isacharunderscore}{\kern0pt}prems\ ps\isactrlsub {\isadigit{2}}{\isacharprime}{\kern0pt}{\isacharunderscore}{\kern0pt}prems\ knights{\isacharunderscore}{\kern0pt}path{\isacharunderscore}{\kern0pt}non{\isacharunderscore}{\kern0pt}nil\ \isacommand{by}\isamarkupfalse%
\ auto\isanewline
\ \ \isacommand{then}\isamarkupfalse%
\ \isacommand{have}\isamarkupfalse%
\ vs{\isacharunderscore}{\kern0pt}c{\isacharcolon}{\kern0pt}\ {\isachardoublequoteopen}valid{\isacharunderscore}{\kern0pt}step\ {\isacharparenleft}{\kern0pt}last\ {\isacharquery}{\kern0pt}ps{\isacharparenright}{\kern0pt}\ {\isacharparenleft}{\kern0pt}hd\ {\isacharquery}{\kern0pt}ps{\isacharparenright}{\kern0pt}{\isachardoublequoteclose}\isanewline
\ \ \ \ \isacommand{using}\isamarkupfalse%
\ vs\ \isacommand{by}\isamarkupfalse%
\ auto\isanewline
\isanewline
\ \ \isacommand{have}\isamarkupfalse%
\ {\isachardoublequoteopen}knights{\isacharunderscore}{\kern0pt}path\ {\isacharparenleft}{\kern0pt}board\ n\ m{\isacharparenright}{\kern0pt}\ {\isacharquery}{\kern0pt}ps{\isachardoublequoteclose}\isanewline
\ \ \ \ \isacommand{using}\isamarkupfalse%
\ ps\isactrlsub {\isadigit{1}}{\isacharprime}{\kern0pt}{\isacharunderscore}{\kern0pt}prems\ ps\isactrlsub {\isadigit{2}}{\isacharprime}{\kern0pt}{\isacharunderscore}{\kern0pt}prems\ inter\ union\ vs\ knights{\isacharunderscore}{\kern0pt}path{\isacharunderscore}{\kern0pt}append\ \isacommand{by}\isamarkupfalse%
\ auto\isanewline
\ \ \isacommand{then}\isamarkupfalse%
\ \isacommand{show}\isamarkupfalse%
\ {\isacharquery}{\kern0pt}thesis\isanewline
\ \ \ \ \isacommand{using}\isamarkupfalse%
\ vs{\isacharunderscore}{\kern0pt}c\ \isacommand{unfolding}\isamarkupfalse%
\ knights{\isacharunderscore}{\kern0pt}circuit{\isacharunderscore}{\kern0pt}def\ \isacommand{by}\isamarkupfalse%
\ blast\isanewline
\isacommand{qed}\isamarkupfalse%
%
\endisatagproof
{\isafoldproof}%
%
\isadelimproof
%
\endisadelimproof
%
\begin{isamarkuptext}%
Final Theorem 1: For every \isa{n{\isasymtimes}m}-board with \isa{min\ n\ m\ {\isasymge}\ {\isadigit{5}}} and \isa{n{\isacharasterisk}{\kern0pt}m} even there exists a 
Knight's circuit.%
\end{isamarkuptext}\isamarkuptrue%
\isacommand{theorem}\isamarkupfalse%
\ knights{\isacharunderscore}{\kern0pt}circuit{\isacharunderscore}{\kern0pt}exists{\isacharcolon}{\kern0pt}\ \isanewline
\ \ \isakeyword{assumes}\ {\isachardoublequoteopen}min\ n\ m\ {\isasymge}\ {\isadigit{5}}{\isachardoublequoteclose}\ {\isachardoublequoteopen}even\ {\isacharparenleft}{\kern0pt}n{\isacharasterisk}{\kern0pt}m{\isacharparenright}{\kern0pt}{\isachardoublequoteclose}\isanewline
\ \ \isakeyword{shows}\ {\isachardoublequoteopen}{\isasymexists}ps{\isachardot}{\kern0pt}\ knights{\isacharunderscore}{\kern0pt}circuit\ {\isacharparenleft}{\kern0pt}board\ n\ m{\isacharparenright}{\kern0pt}\ ps{\isachardoublequoteclose}\isanewline
%
\isadelimproof
\ \ %
\endisadelimproof
%
\isatagproof
\isacommand{using}\isamarkupfalse%
\ assms\isanewline
\isacommand{proof}\isamarkupfalse%
\ {\isacharminus}{\kern0pt}\isanewline
\ \ \isacommand{have}\isamarkupfalse%
\ {\isachardoublequoteopen}n\ {\isacharequal}{\kern0pt}\ {\isadigit{6}}\ {\isasymor}\ m\ {\isacharequal}{\kern0pt}\ {\isadigit{6}}\ {\isasymor}\ n\ {\isacharequal}{\kern0pt}\ {\isadigit{8}}\ {\isasymor}\ m\ {\isacharequal}{\kern0pt}\ {\isadigit{8}}\ {\isasymor}\ {\isacharparenleft}{\kern0pt}n\ {\isasymge}\ {\isadigit{1}}{\isadigit{0}}\ {\isasymand}\ even\ n{\isacharparenright}{\kern0pt}\ {\isasymor}\ {\isacharparenleft}{\kern0pt}m\ {\isasymge}\ {\isadigit{1}}{\isadigit{0}}\ {\isasymand}\ even\ m{\isacharparenright}{\kern0pt}{\isachardoublequoteclose}\isanewline
\ \ \ \ \isacommand{using}\isamarkupfalse%
\ assms\ \isacommand{by}\isamarkupfalse%
\ auto\isanewline
\ \ \isacommand{then}\isamarkupfalse%
\ \isacommand{show}\isamarkupfalse%
\ {\isacharquery}{\kern0pt}thesis\isanewline
\ \ \isacommand{proof}\isamarkupfalse%
\ {\isacharparenleft}{\kern0pt}elim\ disjE{\isacharparenright}{\kern0pt}\isanewline
\ \ \ \ \isacommand{assume}\isamarkupfalse%
\ {\isachardoublequoteopen}n\ {\isacharequal}{\kern0pt}\ {\isadigit{6}}{\isachardoublequoteclose}\isanewline
\ \ \ \ \isacommand{then}\isamarkupfalse%
\ \isacommand{show}\isamarkupfalse%
\ {\isacharquery}{\kern0pt}thesis\isanewline
\ \ \ \ \ \ \isacommand{using}\isamarkupfalse%
\ assms\ knights{\isacharunderscore}{\kern0pt}path{\isacharunderscore}{\kern0pt}{\isadigit{6}}xm{\isacharunderscore}{\kern0pt}exists\ \isacommand{by}\isamarkupfalse%
\ auto\isanewline
\ \ \isacommand{next}\isamarkupfalse%
\isanewline
\ \ \ \ \isacommand{assume}\isamarkupfalse%
\ {\isachardoublequoteopen}m\ {\isacharequal}{\kern0pt}\ {\isadigit{6}}{\isachardoublequoteclose}\isanewline
\ \ \ \ \isacommand{then}\isamarkupfalse%
\ \isacommand{obtain}\isamarkupfalse%
\ ps\ \isakeyword{where}\ {\isachardoublequoteopen}knights{\isacharunderscore}{\kern0pt}circuit\ {\isacharparenleft}{\kern0pt}board\ m\ n{\isacharparenright}{\kern0pt}\ ps{\isachardoublequoteclose}\isanewline
\ \ \ \ \ \ \isacommand{using}\isamarkupfalse%
\ assms\ knights{\isacharunderscore}{\kern0pt}path{\isacharunderscore}{\kern0pt}{\isadigit{6}}xm{\isacharunderscore}{\kern0pt}exists\ \isacommand{by}\isamarkupfalse%
\ auto\isanewline
\ \ \ \ \isacommand{then}\isamarkupfalse%
\ \isacommand{show}\isamarkupfalse%
\ {\isacharquery}{\kern0pt}thesis\isanewline
\ \ \ \ \ \ \isacommand{using}\isamarkupfalse%
\ transpose{\isacharunderscore}{\kern0pt}knights{\isacharunderscore}{\kern0pt}circuit\ \isacommand{by}\isamarkupfalse%
\ auto\isanewline
\ \ \isacommand{next}\isamarkupfalse%
\isanewline
\ \ \ \ \isacommand{assume}\isamarkupfalse%
\ {\isachardoublequoteopen}n\ {\isacharequal}{\kern0pt}\ {\isadigit{8}}{\isachardoublequoteclose}\isanewline
\ \ \ \ \isacommand{then}\isamarkupfalse%
\ \isacommand{show}\isamarkupfalse%
\ {\isacharquery}{\kern0pt}thesis\isanewline
\ \ \ \ \ \ \isacommand{using}\isamarkupfalse%
\ assms\ knights{\isacharunderscore}{\kern0pt}path{\isacharunderscore}{\kern0pt}{\isadigit{8}}xm{\isacharunderscore}{\kern0pt}exists\ \isacommand{by}\isamarkupfalse%
\ auto\isanewline
\ \ \isacommand{next}\isamarkupfalse%
\isanewline
\ \ \ \ \isacommand{assume}\isamarkupfalse%
\ {\isachardoublequoteopen}m\ {\isacharequal}{\kern0pt}\ {\isadigit{8}}{\isachardoublequoteclose}\isanewline
\ \ \ \ \isacommand{then}\isamarkupfalse%
\ \isacommand{obtain}\isamarkupfalse%
\ ps\ \isakeyword{where}\ {\isachardoublequoteopen}knights{\isacharunderscore}{\kern0pt}circuit\ {\isacharparenleft}{\kern0pt}board\ m\ n{\isacharparenright}{\kern0pt}\ ps{\isachardoublequoteclose}\isanewline
\ \ \ \ \ \ \isacommand{using}\isamarkupfalse%
\ assms\ knights{\isacharunderscore}{\kern0pt}path{\isacharunderscore}{\kern0pt}{\isadigit{8}}xm{\isacharunderscore}{\kern0pt}exists\ \isacommand{by}\isamarkupfalse%
\ auto\isanewline
\ \ \ \ \isacommand{then}\isamarkupfalse%
\ \isacommand{show}\isamarkupfalse%
\ {\isacharquery}{\kern0pt}thesis\isanewline
\ \ \ \ \ \ \isacommand{using}\isamarkupfalse%
\ transpose{\isacharunderscore}{\kern0pt}knights{\isacharunderscore}{\kern0pt}circuit\ \isacommand{by}\isamarkupfalse%
\ auto\isanewline
\ \ \isacommand{next}\isamarkupfalse%
\isanewline
\ \ \ \ \isacommand{assume}\isamarkupfalse%
\ {\isachardoublequoteopen}n\ {\isasymge}\ {\isadigit{1}}{\isadigit{0}}\ {\isasymand}\ even\ n{\isachardoublequoteclose}\isanewline
\ \ \ \ \isacommand{then}\isamarkupfalse%
\ \isacommand{show}\isamarkupfalse%
\ {\isacharquery}{\kern0pt}thesis\isanewline
\ \ \ \ \ \ \isacommand{using}\isamarkupfalse%
\ assms\ knights{\isacharunderscore}{\kern0pt}circuit{\isacharunderscore}{\kern0pt}exists{\isacharunderscore}{\kern0pt}n{\isacharunderscore}{\kern0pt}even{\isacharunderscore}{\kern0pt}gr{\isacharunderscore}{\kern0pt}{\isadigit{1}}{\isadigit{0}}\ \isacommand{by}\isamarkupfalse%
\ auto\isanewline
\ \ \isacommand{next}\isamarkupfalse%
\isanewline
\ \ \ \ \isacommand{assume}\isamarkupfalse%
\ {\isachardoublequoteopen}m\ {\isasymge}\ {\isadigit{1}}{\isadigit{0}}\ {\isasymand}\ even\ m{\isachardoublequoteclose}\isanewline
\ \ \ \ \isacommand{then}\isamarkupfalse%
\ \isacommand{obtain}\isamarkupfalse%
\ ps\ \isakeyword{where}\ {\isachardoublequoteopen}knights{\isacharunderscore}{\kern0pt}circuit\ {\isacharparenleft}{\kern0pt}board\ m\ n{\isacharparenright}{\kern0pt}\ ps{\isachardoublequoteclose}\isanewline
\ \ \ \ \ \ \isacommand{using}\isamarkupfalse%
\ assms\ knights{\isacharunderscore}{\kern0pt}circuit{\isacharunderscore}{\kern0pt}exists{\isacharunderscore}{\kern0pt}n{\isacharunderscore}{\kern0pt}even{\isacharunderscore}{\kern0pt}gr{\isacharunderscore}{\kern0pt}{\isadigit{1}}{\isadigit{0}}\ \isacommand{by}\isamarkupfalse%
\ auto\isanewline
\ \ \ \ \isacommand{then}\isamarkupfalse%
\ \isacommand{show}\isamarkupfalse%
\ {\isacharquery}{\kern0pt}thesis\isanewline
\ \ \ \ \ \ \isacommand{using}\isamarkupfalse%
\ transpose{\isacharunderscore}{\kern0pt}knights{\isacharunderscore}{\kern0pt}circuit\ \isacommand{by}\isamarkupfalse%
\ auto\isanewline
\ \ \isacommand{qed}\isamarkupfalse%
\isanewline
\isacommand{qed}\isamarkupfalse%
%
\endisatagproof
{\isafoldproof}%
%
\isadelimproof
%
\endisadelimproof
%
\begin{isamarkuptext}%
Final Theorem 2: for every \isa{n{\isasymtimes}m}-board with \isa{min\ n\ m\ {\isasymge}\ {\isadigit{5}}} there exists a Knight's path.%
\end{isamarkuptext}\isamarkuptrue%
\isacommand{theorem}\isamarkupfalse%
\ knights{\isacharunderscore}{\kern0pt}path{\isacharunderscore}{\kern0pt}exists{\isacharcolon}{\kern0pt}\ \isanewline
\ \ \isakeyword{assumes}\ {\isachardoublequoteopen}min\ n\ m\ {\isasymge}\ {\isadigit{5}}{\isachardoublequoteclose}\isanewline
\ \ \isakeyword{shows}\ {\isachardoublequoteopen}{\isasymexists}ps{\isachardot}{\kern0pt}\ knights{\isacharunderscore}{\kern0pt}path\ {\isacharparenleft}{\kern0pt}board\ n\ m{\isacharparenright}{\kern0pt}\ ps{\isachardoublequoteclose}\isanewline
%
\isadelimproof
\ \ %
\endisadelimproof
%
\isatagproof
\isacommand{using}\isamarkupfalse%
\ assms\isanewline
\isacommand{proof}\isamarkupfalse%
\ {\isacharminus}{\kern0pt}\isanewline
\ \ \isacommand{have}\isamarkupfalse%
\ {\isachardoublequoteopen}odd\ n\ {\isasymor}\ odd\ m\ {\isasymor}\ even\ {\isacharparenleft}{\kern0pt}n{\isacharasterisk}{\kern0pt}m{\isacharparenright}{\kern0pt}{\isachardoublequoteclose}\ \isacommand{by}\isamarkupfalse%
\ simp\isanewline
\ \ \isacommand{then}\isamarkupfalse%
\ \isacommand{show}\isamarkupfalse%
\ {\isacharquery}{\kern0pt}thesis\isanewline
\ \ \isacommand{proof}\isamarkupfalse%
\ {\isacharparenleft}{\kern0pt}elim\ disjE{\isacharparenright}{\kern0pt}\isanewline
\ \ \ \ \isacommand{assume}\isamarkupfalse%
\ {\isachardoublequoteopen}odd\ n{\isachardoublequoteclose}\isanewline
\ \ \ \ \isacommand{then}\isamarkupfalse%
\ \isacommand{show}\isamarkupfalse%
\ {\isacharquery}{\kern0pt}thesis\isanewline
\ \ \ \ \ \ \isacommand{using}\isamarkupfalse%
\ assms\ knights{\isacharunderscore}{\kern0pt}path{\isacharunderscore}{\kern0pt}odd{\isacharunderscore}{\kern0pt}n{\isacharunderscore}{\kern0pt}exists\ \isacommand{by}\isamarkupfalse%
\ auto\isanewline
\ \ \isacommand{next}\isamarkupfalse%
\isanewline
\ \ \ \ \isacommand{assume}\isamarkupfalse%
\ {\isachardoublequoteopen}odd\ m{\isachardoublequoteclose}\isanewline
\ \ \ \ \isacommand{then}\isamarkupfalse%
\ \isacommand{obtain}\isamarkupfalse%
\ ps\ \isakeyword{where}\ {\isachardoublequoteopen}knights{\isacharunderscore}{\kern0pt}path\ {\isacharparenleft}{\kern0pt}board\ m\ n{\isacharparenright}{\kern0pt}\ ps{\isachardoublequoteclose}\isanewline
\ \ \ \ \ \ \isacommand{using}\isamarkupfalse%
\ assms\ knights{\isacharunderscore}{\kern0pt}path{\isacharunderscore}{\kern0pt}odd{\isacharunderscore}{\kern0pt}n{\isacharunderscore}{\kern0pt}exists\ \isacommand{by}\isamarkupfalse%
\ auto\isanewline
\ \ \ \ \isacommand{then}\isamarkupfalse%
\ \isacommand{show}\isamarkupfalse%
\ {\isacharquery}{\kern0pt}thesis\isanewline
\ \ \ \ \ \ \isacommand{using}\isamarkupfalse%
\ transpose{\isacharunderscore}{\kern0pt}knights{\isacharunderscore}{\kern0pt}path\ \isacommand{by}\isamarkupfalse%
\ auto\isanewline
\ \ \isacommand{next}\isamarkupfalse%
\isanewline
\ \ \ \ \isacommand{assume}\isamarkupfalse%
\ {\isachardoublequoteopen}even\ {\isacharparenleft}{\kern0pt}n{\isacharasterisk}{\kern0pt}m{\isacharparenright}{\kern0pt}{\isachardoublequoteclose}\isanewline
\ \ \ \ \isacommand{then}\isamarkupfalse%
\ \isacommand{show}\isamarkupfalse%
\ {\isacharquery}{\kern0pt}thesis\isanewline
\ \ \ \ \ \ \isacommand{using}\isamarkupfalse%
\ assms\ knights{\isacharunderscore}{\kern0pt}circuit{\isacharunderscore}{\kern0pt}exists\ \isacommand{by}\isamarkupfalse%
\ {\isacharparenleft}{\kern0pt}auto\ simp{\isacharcolon}{\kern0pt}\ knights{\isacharunderscore}{\kern0pt}circuit{\isacharunderscore}{\kern0pt}def{\isacharparenright}{\kern0pt}\isanewline
\ \ \isacommand{qed}\isamarkupfalse%
\isanewline
\isacommand{qed}\isamarkupfalse%
%
\endisatagproof
{\isafoldproof}%
%
\isadelimproof
%
\endisadelimproof
%
\begin{isamarkuptext}%
THE END%
\end{isamarkuptext}\isamarkuptrue%
%
\isadelimtheory
%
\endisadelimtheory
%
\isatagtheory
\isacommand{end}\isamarkupfalse%
%
\endisatagtheory
{\isafoldtheory}%
%
\isadelimtheory
%
\endisadelimtheory
%
\end{isabellebody}%
\endinput
%:%file=KnightsTour.tex%:%
%:%6=1%:%
%:%7=2%:%
%:%12=3%:%
%:%13=3%:%
%:%14=4%:%
%:%15=5%:%
%:%29=7%:%
%:%41=9%:%
%:%42=10%:%
%:%43=11%:%
%:%44=12%:%
%:%45=13%:%
%:%46=14%:%
%:%47=15%:%
%:%48=16%:%
%:%49=17%:%
%:%50=18%:%
%:%51=19%:%
%:%52=20%:%
%:%53=21%:%
%:%54=22%:%
%:%55=23%:%
%:%56=24%:%
%:%57=25%:%
%:%58=26%:%
%:%59=27%:%
%:%63=29%:%
%:%65=30%:%
%:%66=30%:%
%:%68=31%:%
%:%69=32%:%
%:%71=33%:%
%:%72=33%:%
%:%74=35%:%
%:%75=36%:%
%:%77=37%:%
%:%78=37%:%
%:%79=38%:%
%:%81=40%:%
%:%82=41%:%
%:%84=42%:%
%:%85=42%:%
%:%87=44%:%
%:%88=45%:%
%:%89=46%:%
%:%91=47%:%
%:%92=47%:%
%:%93=48%:%
%:%96=51%:%
%:%97=52%:%
%:%98=53%:%
%:%100=54%:%
%:%101=54%:%
%:%102=55%:%
%:%103=56%:%
%:%104=57%:%
%:%105=58%:%
%:%106=58%:%
%:%108=58%:%
%:%112=58%:%
%:%122=60%:%
%:%123=61%:%
%:%125=62%:%
%:%126=62%:%
%:%133=64%:%
%:%145=66%:%
%:%146=67%:%
%:%155=69%:%
%:%165=71%:%
%:%166=71%:%
%:%167=72%:%
%:%168=73%:%
%:%169=74%:%
%:%170=75%:%
%:%171=75%:%
%:%172=76%:%
%:%173=77%:%
%:%175=79%:%
%:%177=80%:%
%:%178=80%:%
%:%179=81%:%
%:%180=82%:%
%:%181=83%:%
%:%182=83%:%
%:%183=84%:%
%:%185=86%:%
%:%186=87%:%
%:%187=88%:%
%:%188=88%:%
%:%189=89%:%
%:%190=90%:%
%:%191=91%:%
%:%192=92%:%
%:%193=93%:%
%:%194=93%:%
%:%195=94%:%
%:%202=96%:%
%:%212=98%:%
%:%213=98%:%
%:%216=99%:%
%:%220=99%:%
%:%221=99%:%
%:%226=99%:%
%:%229=100%:%
%:%230=101%:%
%:%231=101%:%
%:%234=102%:%
%:%238=102%:%
%:%239=102%:%
%:%244=102%:%
%:%247=103%:%
%:%248=104%:%
%:%249=104%:%
%:%252=105%:%
%:%256=105%:%
%:%257=105%:%
%:%258=105%:%
%:%263=105%:%
%:%266=106%:%
%:%267=107%:%
%:%268=107%:%
%:%271=108%:%
%:%275=108%:%
%:%276=108%:%
%:%277=108%:%
%:%278=108%:%
%:%283=108%:%
%:%286=109%:%
%:%287=110%:%
%:%288=110%:%
%:%295=111%:%
%:%296=111%:%
%:%297=112%:%
%:%298=112%:%
%:%299=113%:%
%:%300=113%:%
%:%301=113%:%
%:%302=114%:%
%:%303=114%:%
%:%304=115%:%
%:%305=115%:%
%:%306=116%:%
%:%307=116%:%
%:%308=117%:%
%:%309=117%:%
%:%310=118%:%
%:%311=118%:%
%:%312=119%:%
%:%313=119%:%
%:%314=120%:%
%:%315=120%:%
%:%316=121%:%
%:%317=121%:%
%:%318=121%:%
%:%319=122%:%
%:%320=122%:%
%:%321=122%:%
%:%322=123%:%
%:%328=123%:%
%:%331=124%:%
%:%332=125%:%
%:%333=125%:%
%:%336=126%:%
%:%340=126%:%
%:%341=126%:%
%:%342=127%:%
%:%343=127%:%
%:%348=127%:%
%:%351=128%:%
%:%352=129%:%
%:%353=129%:%
%:%354=130%:%
%:%355=131%:%
%:%358=132%:%
%:%362=132%:%
%:%363=132%:%
%:%364=133%:%
%:%365=133%:%
%:%366=134%:%
%:%367=134%:%
%:%368=135%:%
%:%369=135%:%
%:%370=135%:%
%:%371=136%:%
%:%372=136%:%
%:%373=137%:%
%:%374=137%:%
%:%375=137%:%
%:%376=138%:%
%:%377=138%:%
%:%378=138%:%
%:%379=139%:%
%:%389=141%:%
%:%391=142%:%
%:%392=142%:%
%:%399=143%:%
%:%400=143%:%
%:%401=144%:%
%:%402=144%:%
%:%403=145%:%
%:%404=145%:%
%:%405=145%:%
%:%406=146%:%
%:%407=146%:%
%:%408=147%:%
%:%409=147%:%
%:%410=148%:%
%:%411=148%:%
%:%412=148%:%
%:%413=148%:%
%:%414=148%:%
%:%415=149%:%
%:%416=149%:%
%:%417=150%:%
%:%418=150%:%
%:%419=151%:%
%:%420=151%:%
%:%421=152%:%
%:%422=152%:%
%:%423=152%:%
%:%424=153%:%
%:%425=153%:%
%:%426=154%:%
%:%427=154%:%
%:%428=155%:%
%:%434=155%:%
%:%437=156%:%
%:%438=157%:%
%:%439=157%:%
%:%442=158%:%
%:%446=158%:%
%:%447=158%:%
%:%448=158%:%
%:%453=158%:%
%:%456=159%:%
%:%457=160%:%
%:%458=160%:%
%:%461=161%:%
%:%465=161%:%
%:%466=161%:%
%:%467=161%:%
%:%468=161%:%
%:%473=161%:%
%:%476=162%:%
%:%477=163%:%
%:%478=163%:%
%:%479=164%:%
%:%482=165%:%
%:%486=165%:%
%:%487=165%:%
%:%488=165%:%
%:%502=167%:%
%:%512=169%:%
%:%513=169%:%
%:%516=170%:%
%:%520=170%:%
%:%521=170%:%
%:%522=170%:%
%:%527=170%:%
%:%530=171%:%
%:%531=172%:%
%:%532=172%:%
%:%535=173%:%
%:%539=173%:%
%:%540=173%:%
%:%545=173%:%
%:%548=174%:%
%:%549=175%:%
%:%550=175%:%
%:%553=176%:%
%:%557=176%:%
%:%558=176%:%
%:%563=176%:%
%:%566=177%:%
%:%567=178%:%
%:%568=178%:%
%:%571=179%:%
%:%575=179%:%
%:%576=179%:%
%:%577=179%:%
%:%582=179%:%
%:%585=180%:%
%:%586=181%:%
%:%587=181%:%
%:%594=182%:%
%:%595=182%:%
%:%596=183%:%
%:%597=183%:%
%:%598=184%:%
%:%599=184%:%
%:%600=185%:%
%:%601=185%:%
%:%602=185%:%
%:%603=186%:%
%:%604=186%:%
%:%605=186%:%
%:%606=187%:%
%:%607=187%:%
%:%608=187%:%
%:%609=188%:%
%:%610=188%:%
%:%611=188%:%
%:%612=189%:%
%:%613=189%:%
%:%614=190%:%
%:%615=190%:%
%:%616=190%:%
%:%617=190%:%
%:%618=190%:%
%:%619=191%:%
%:%620=191%:%
%:%625=191%:%
%:%628=192%:%
%:%629=193%:%
%:%630=193%:%
%:%631=194%:%
%:%638=195%:%
%:%639=195%:%
%:%640=196%:%
%:%641=196%:%
%:%642=197%:%
%:%643=197%:%
%:%644=198%:%
%:%645=198%:%
%:%646=198%:%
%:%647=199%:%
%:%648=199%:%
%:%649=199%:%
%:%650=199%:%
%:%651=200%:%
%:%652=200%:%
%:%653=200%:%
%:%654=200%:%
%:%655=201%:%
%:%656=201%:%
%:%657=201%:%
%:%658=201%:%
%:%659=202%:%
%:%660=202%:%
%:%661=203%:%
%:%662=203%:%
%:%663=204%:%
%:%664=204%:%
%:%665=205%:%
%:%666=205%:%
%:%667=206%:%
%:%668=206%:%
%:%669=206%:%
%:%670=207%:%
%:%671=207%:%
%:%672=207%:%
%:%673=207%:%
%:%674=208%:%
%:%675=208%:%
%:%676=208%:%
%:%677=208%:%
%:%678=209%:%
%:%679=209%:%
%:%680=209%:%
%:%681=209%:%
%:%682=210%:%
%:%683=210%:%
%:%684=211%:%
%:%690=211%:%
%:%693=212%:%
%:%694=213%:%
%:%695=213%:%
%:%702=214%:%
%:%703=214%:%
%:%704=215%:%
%:%705=215%:%
%:%706=216%:%
%:%707=216%:%
%:%708=216%:%
%:%709=216%:%
%:%710=216%:%
%:%711=217%:%
%:%712=217%:%
%:%717=217%:%
%:%720=218%:%
%:%721=219%:%
%:%722=219%:%
%:%729=220%:%
%:%730=220%:%
%:%731=221%:%
%:%732=221%:%
%:%733=222%:%
%:%734=222%:%
%:%735=223%:%
%:%736=223%:%
%:%737=224%:%
%:%738=225%:%
%:%739=225%:%
%:%740=226%:%
%:%741=226%:%
%:%742=226%:%
%:%743=227%:%
%:%744=227%:%
%:%745=227%:%
%:%746=228%:%
%:%747=228%:%
%:%748=228%:%
%:%749=229%:%
%:%750=230%:%
%:%751=230%:%
%:%752=231%:%
%:%753=231%:%
%:%754=231%:%
%:%755=232%:%
%:%756=233%:%
%:%757=233%:%
%:%758=234%:%
%:%759=234%:%
%:%760=234%:%
%:%761=235%:%
%:%762=235%:%
%:%763=235%:%
%:%764=236%:%
%:%765=236%:%
%:%766=236%:%
%:%767=237%:%
%:%768=237%:%
%:%769=237%:%
%:%770=238%:%
%:%771=238%:%
%:%772=239%:%
%:%773=239%:%
%:%774=239%:%
%:%775=239%:%
%:%776=240%:%
%:%777=240%:%
%:%782=240%:%
%:%785=241%:%
%:%786=242%:%
%:%787=242%:%
%:%794=243%:%
%:%795=243%:%
%:%796=244%:%
%:%797=244%:%
%:%798=245%:%
%:%799=245%:%
%:%800=245%:%
%:%801=246%:%
%:%802=246%:%
%:%803=246%:%
%:%804=247%:%
%:%805=247%:%
%:%806=247%:%
%:%807=248%:%
%:%808=248%:%
%:%809=248%:%
%:%810=248%:%
%:%811=249%:%
%:%817=249%:%
%:%820=250%:%
%:%821=251%:%
%:%822=251%:%
%:%825=252%:%
%:%829=252%:%
%:%830=252%:%
%:%835=252%:%
%:%838=253%:%
%:%839=254%:%
%:%840=254%:%
%:%843=255%:%
%:%847=255%:%
%:%848=255%:%
%:%849=255%:%
%:%850=255%:%
%:%855=255%:%
%:%858=256%:%
%:%859=257%:%
%:%860=257%:%
%:%863=258%:%
%:%867=258%:%
%:%868=258%:%
%:%873=258%:%
%:%876=259%:%
%:%877=260%:%
%:%878=260%:%
%:%881=261%:%
%:%885=261%:%
%:%886=261%:%
%:%891=261%:%
%:%894=262%:%
%:%895=263%:%
%:%896=263%:%
%:%897=264%:%
%:%900=265%:%
%:%904=265%:%
%:%905=265%:%
%:%906=265%:%
%:%911=265%:%
%:%914=266%:%
%:%915=267%:%
%:%916=267%:%
%:%919=268%:%
%:%923=268%:%
%:%924=268%:%
%:%925=268%:%
%:%930=268%:%
%:%933=269%:%
%:%934=270%:%
%:%935=270%:%
%:%938=271%:%
%:%942=271%:%
%:%943=271%:%
%:%944=271%:%
%:%949=271%:%
%:%952=272%:%
%:%953=273%:%
%:%954=273%:%
%:%961=274%:%
%:%962=274%:%
%:%963=275%:%
%:%964=275%:%
%:%965=276%:%
%:%966=276%:%
%:%967=276%:%
%:%968=277%:%
%:%969=277%:%
%:%970=277%:%
%:%971=278%:%
%:%972=278%:%
%:%973=278%:%
%:%974=279%:%
%:%975=279%:%
%:%976=279%:%
%:%977=280%:%
%:%978=280%:%
%:%979=281%:%
%:%980=281%:%
%:%981=282%:%
%:%982=282%:%
%:%983=283%:%
%:%984=283%:%
%:%985=284%:%
%:%991=284%:%
%:%994=285%:%
%:%995=286%:%
%:%996=286%:%
%:%1003=287%:%
%:%1004=287%:%
%:%1005=288%:%
%:%1006=288%:%
%:%1007=289%:%
%:%1008=289%:%
%:%1009=289%:%
%:%1010=290%:%
%:%1011=290%:%
%:%1012=290%:%
%:%1013=291%:%
%:%1014=291%:%
%:%1015=291%:%
%:%1016=291%:%
%:%1017=291%:%
%:%1018=292%:%
%:%1019=292%:%
%:%1024=292%:%
%:%1027=293%:%
%:%1028=294%:%
%:%1029=294%:%
%:%1032=295%:%
%:%1036=295%:%
%:%1037=295%:%
%:%1038=295%:%
%:%1043=295%:%
%:%1046=296%:%
%:%1047=297%:%
%:%1048=297%:%
%:%1049=298%:%
%:%1050=299%:%
%:%1053=300%:%
%:%1057=300%:%
%:%1058=300%:%
%:%1059=301%:%
%:%1060=301%:%
%:%1061=302%:%
%:%1062=302%:%
%:%1063=303%:%
%:%1064=303%:%
%:%1065=303%:%
%:%1066=303%:%
%:%1067=304%:%
%:%1068=304%:%
%:%1069=304%:%
%:%1070=305%:%
%:%1071=305%:%
%:%1072=306%:%
%:%1073=306%:%
%:%1074=307%:%
%:%1075=307%:%
%:%1076=308%:%
%:%1077=308%:%
%:%1078=308%:%
%:%1079=308%:%
%:%1080=308%:%
%:%1081=309%:%
%:%1082=309%:%
%:%1083=309%:%
%:%1084=309%:%
%:%1085=309%:%
%:%1086=310%:%
%:%1087=310%:%
%:%1088=311%:%
%:%1089=311%:%
%:%1090=312%:%
%:%1091=312%:%
%:%1092=312%:%
%:%1093=313%:%
%:%1094=314%:%
%:%1095=315%:%
%:%1096=315%:%
%:%1097=315%:%
%:%1098=316%:%
%:%1099=316%:%
%:%1100=316%:%
%:%1101=317%:%
%:%1102=317%:%
%:%1103=317%:%
%:%1104=318%:%
%:%1105=318%:%
%:%1106=318%:%
%:%1107=319%:%
%:%1108=319%:%
%:%1109=319%:%
%:%1110=320%:%
%:%1111=321%:%
%:%1112=321%:%
%:%1113=322%:%
%:%1114=322%:%
%:%1115=322%:%
%:%1116=323%:%
%:%1117=323%:%
%:%1118=324%:%
%:%1119=324%:%
%:%1120=324%:%
%:%1121=325%:%
%:%1122=325%:%
%:%1123=325%:%
%:%1124=325%:%
%:%1125=325%:%
%:%1126=326%:%
%:%1127=326%:%
%:%1128=327%:%
%:%1129=327%:%
%:%1134=327%:%
%:%1137=328%:%
%:%1138=329%:%
%:%1139=329%:%
%:%1140=330%:%
%:%1141=331%:%
%:%1144=332%:%
%:%1148=332%:%
%:%1149=332%:%
%:%1150=333%:%
%:%1151=333%:%
%:%1152=334%:%
%:%1153=334%:%
%:%1154=335%:%
%:%1155=335%:%
%:%1156=335%:%
%:%1157=335%:%
%:%1158=336%:%
%:%1159=336%:%
%:%1160=336%:%
%:%1161=337%:%
%:%1162=337%:%
%:%1163=338%:%
%:%1164=338%:%
%:%1165=339%:%
%:%1166=339%:%
%:%1167=340%:%
%:%1168=340%:%
%:%1169=340%:%
%:%1170=340%:%
%:%1171=340%:%
%:%1172=341%:%
%:%1173=341%:%
%:%1174=342%:%
%:%1175=342%:%
%:%1176=343%:%
%:%1177=343%:%
%:%1178=343%:%
%:%1179=344%:%
%:%1180=344%:%
%:%1181=344%:%
%:%1182=345%:%
%:%1183=345%:%
%:%1184=345%:%
%:%1185=346%:%
%:%1186=346%:%
%:%1187=346%:%
%:%1188=347%:%
%:%1189=347%:%
%:%1190=348%:%
%:%1191=348%:%
%:%1200=350%:%
%:%1202=351%:%
%:%1203=351%:%
%:%1204=352%:%
%:%1205=353%:%
%:%1206=354%:%
%:%1209=355%:%
%:%1213=355%:%
%:%1214=355%:%
%:%1215=356%:%
%:%1216=356%:%
%:%1217=357%:%
%:%1218=357%:%
%:%1219=358%:%
%:%1220=358%:%
%:%1221=359%:%
%:%1222=359%:%
%:%1223=360%:%
%:%1224=360%:%
%:%1225=360%:%
%:%1226=361%:%
%:%1227=361%:%
%:%1228=362%:%
%:%1229=362%:%
%:%1230=362%:%
%:%1231=363%:%
%:%1232=363%:%
%:%1233=364%:%
%:%1234=364%:%
%:%1235=364%:%
%:%1236=365%:%
%:%1237=365%:%
%:%1238=365%:%
%:%1239=365%:%
%:%1240=366%:%
%:%1250=368%:%
%:%1252=369%:%
%:%1253=369%:%
%:%1254=370%:%
%:%1255=371%:%
%:%1258=372%:%
%:%1262=372%:%
%:%1263=372%:%
%:%1264=373%:%
%:%1265=373%:%
%:%1266=374%:%
%:%1267=374%:%
%:%1268=375%:%
%:%1269=375%:%
%:%1270=375%:%
%:%1271=376%:%
%:%1272=376%:%
%:%1273=376%:%
%:%1274=377%:%
%:%1275=377%:%
%:%1276=377%:%
%:%1277=378%:%
%:%1278=378%:%
%:%1279=378%:%
%:%1280=379%:%
%:%1281=379%:%
%:%1282=379%:%
%:%1283=379%:%
%:%1284=379%:%
%:%1285=380%:%
%:%1286=380%:%
%:%1287=381%:%
%:%1288=381%:%
%:%1289=382%:%
%:%1290=382%:%
%:%1291=382%:%
%:%1292=382%:%
%:%1293=383%:%
%:%1294=383%:%
%:%1295=383%:%
%:%1296=384%:%
%:%1297=384%:%
%:%1298=384%:%
%:%1299=385%:%
%:%1300=385%:%
%:%1301=385%:%
%:%1302=385%:%
%:%1303=386%:%
%:%1309=386%:%
%:%1312=387%:%
%:%1313=388%:%
%:%1314=388%:%
%:%1317=389%:%
%:%1321=389%:%
%:%1322=389%:%
%:%1323=389%:%
%:%1332=391%:%
%:%1334=392%:%
%:%1335=392%:%
%:%1336=393%:%
%:%1337=394%:%
%:%1340=395%:%
%:%1344=395%:%
%:%1345=395%:%
%:%1346=396%:%
%:%1347=396%:%
%:%1348=397%:%
%:%1349=397%:%
%:%1350=398%:%
%:%1351=398%:%
%:%1352=398%:%
%:%1353=399%:%
%:%1354=399%:%
%:%1355=399%:%
%:%1356=400%:%
%:%1357=400%:%
%:%1358=400%:%
%:%1359=401%:%
%:%1360=401%:%
%:%1361=401%:%
%:%1362=402%:%
%:%1363=402%:%
%:%1364=402%:%
%:%1365=402%:%
%:%1366=403%:%
%:%1367=403%:%
%:%1376=405%:%
%:%1378=406%:%
%:%1379=406%:%
%:%1380=407%:%
%:%1381=408%:%
%:%1384=409%:%
%:%1388=409%:%
%:%1389=409%:%
%:%1390=410%:%
%:%1391=410%:%
%:%1392=410%:%
%:%1397=410%:%
%:%1400=411%:%
%:%1401=412%:%
%:%1402=420%:%
%:%1403=421%:%
%:%1404=422%:%
%:%1405=422%:%
%:%1406=423%:%
%:%1407=424%:%
%:%1414=425%:%
%:%1415=425%:%
%:%1416=426%:%
%:%1417=426%:%
%:%1418=427%:%
%:%1419=427%:%
%:%1420=427%:%
%:%1421=427%:%
%:%1422=427%:%
%:%1423=428%:%
%:%1424=428%:%
%:%1425=429%:%
%:%1426=429%:%
%:%1427=430%:%
%:%1428=430%:%
%:%1429=431%:%
%:%1430=431%:%
%:%1431=431%:%
%:%1432=431%:%
%:%1433=432%:%
%:%1434=432%:%
%:%1435=432%:%
%:%1436=433%:%
%:%1437=433%:%
%:%1438=433%:%
%:%1439=434%:%
%:%1440=434%:%
%:%1441=434%:%
%:%1442=435%:%
%:%1443=435%:%
%:%1444=435%:%
%:%1445=436%:%
%:%1446=437%:%
%:%1447=437%:%
%:%1448=438%:%
%:%1449=438%:%
%:%1450=439%:%
%:%1451=440%:%
%:%1452=440%:%
%:%1453=441%:%
%:%1454=441%:%
%:%1455=441%:%
%:%1456=442%:%
%:%1457=443%:%
%:%1458=443%:%
%:%1459=444%:%
%:%1460=444%:%
%:%1461=444%:%
%:%1462=445%:%
%:%1463=445%:%
%:%1464=445%:%
%:%1465=446%:%
%:%1466=446%:%
%:%1467=447%:%
%:%1468=447%:%
%:%1469=447%:%
%:%1470=448%:%
%:%1480=450%:%
%:%1482=451%:%
%:%1483=451%:%
%:%1484=452%:%
%:%1485=453%:%
%:%1488=454%:%
%:%1492=454%:%
%:%1493=454%:%
%:%1494=455%:%
%:%1495=455%:%
%:%1496=456%:%
%:%1497=456%:%
%:%1498=457%:%
%:%1499=457%:%
%:%1500=458%:%
%:%1501=458%:%
%:%1502=459%:%
%:%1503=459%:%
%:%1504=460%:%
%:%1505=460%:%
%:%1506=461%:%
%:%1507=461%:%
%:%1508=462%:%
%:%1509=462%:%
%:%1510=462%:%
%:%1511=462%:%
%:%1512=462%:%
%:%1513=463%:%
%:%1514=463%:%
%:%1515=464%:%
%:%1516=464%:%
%:%1517=465%:%
%:%1518=465%:%
%:%1519=465%:%
%:%1520=466%:%
%:%1521=466%:%
%:%1522=466%:%
%:%1523=467%:%
%:%1524=467%:%
%:%1525=467%:%
%:%1526=468%:%
%:%1527=468%:%
%:%1528=468%:%
%:%1529=469%:%
%:%1530=469%:%
%:%1531=469%:%
%:%1532=470%:%
%:%1533=470%:%
%:%1534=470%:%
%:%1535=471%:%
%:%1536=471%:%
%:%1537=471%:%
%:%1538=471%:%
%:%1539=471%:%
%:%1540=472%:%
%:%1541=472%:%
%:%1542=473%:%
%:%1543=473%:%
%:%1552=475%:%
%:%1554=476%:%
%:%1555=476%:%
%:%1556=477%:%
%:%1557=478%:%
%:%1560=479%:%
%:%1564=479%:%
%:%1565=479%:%
%:%1566=479%:%
%:%1567=479%:%
%:%1572=479%:%
%:%1575=480%:%
%:%1576=481%:%
%:%1577=481%:%
%:%1584=482%:%
%:%1585=482%:%
%:%1586=483%:%
%:%1587=483%:%
%:%1588=484%:%
%:%1589=484%:%
%:%1590=484%:%
%:%1591=484%:%
%:%1592=485%:%
%:%1593=485%:%
%:%1594=485%:%
%:%1595=485%:%
%:%1596=486%:%
%:%1597=486%:%
%:%1598=486%:%
%:%1599=486%:%
%:%1600=487%:%
%:%1601=487%:%
%:%1602=487%:%
%:%1603=488%:%
%:%1604=488%:%
%:%1605=489%:%
%:%1606=489%:%
%:%1607=490%:%
%:%1608=490%:%
%:%1609=490%:%
%:%1610=491%:%
%:%1611=491%:%
%:%1612=491%:%
%:%1613=492%:%
%:%1614=492%:%
%:%1615=492%:%
%:%1616=492%:%
%:%1617=493%:%
%:%1618=493%:%
%:%1619=494%:%
%:%1620=494%:%
%:%1621=495%:%
%:%1622=495%:%
%:%1623=495%:%
%:%1624=495%:%
%:%1625=495%:%
%:%1626=496%:%
%:%1627=496%:%
%:%1628=497%:%
%:%1629=497%:%
%:%1634=497%:%
%:%1637=498%:%
%:%1638=499%:%
%:%1639=499%:%
%:%1646=500%:%
%:%1647=500%:%
%:%1648=501%:%
%:%1649=501%:%
%:%1650=502%:%
%:%1651=502%:%
%:%1652=502%:%
%:%1653=502%:%
%:%1654=503%:%
%:%1655=503%:%
%:%1656=503%:%
%:%1657=504%:%
%:%1658=504%:%
%:%1659=505%:%
%:%1660=505%:%
%:%1661=506%:%
%:%1662=506%:%
%:%1663=506%:%
%:%1664=507%:%
%:%1665=507%:%
%:%1666=507%:%
%:%1667=508%:%
%:%1668=508%:%
%:%1669=508%:%
%:%1670=509%:%
%:%1671=509%:%
%:%1672=509%:%
%:%1673=510%:%
%:%1674=510%:%
%:%1675=510%:%
%:%1676=511%:%
%:%1677=511%:%
%:%1678=511%:%
%:%1679=512%:%
%:%1680=512%:%
%:%1681=512%:%
%:%1682=512%:%
%:%1683=513%:%
%:%1684=513%:%
%:%1685=514%:%
%:%1686=514%:%
%:%1687=515%:%
%:%1688=515%:%
%:%1689=515%:%
%:%1690=515%:%
%:%1691=515%:%
%:%1692=516%:%
%:%1693=516%:%
%:%1694=517%:%
%:%1695=517%:%
%:%1704=519%:%
%:%1705=520%:%
%:%1707=521%:%
%:%1708=521%:%
%:%1709=522%:%
%:%1710=523%:%
%:%1713=524%:%
%:%1717=524%:%
%:%1718=524%:%
%:%1719=525%:%
%:%1720=525%:%
%:%1721=526%:%
%:%1722=526%:%
%:%1723=527%:%
%:%1724=527%:%
%:%1725=528%:%
%:%1726=528%:%
%:%1727=528%:%
%:%1728=528%:%
%:%1729=529%:%
%:%1730=529%:%
%:%1731=529%:%
%:%1732=530%:%
%:%1733=530%:%
%:%1734=530%:%
%:%1735=531%:%
%:%1736=531%:%
%:%1737=531%:%
%:%1738=532%:%
%:%1739=532%:%
%:%1740=533%:%
%:%1741=533%:%
%:%1742=533%:%
%:%1743=534%:%
%:%1744=534%:%
%:%1745=534%:%
%:%1746=535%:%
%:%1747=535%:%
%:%1748=535%:%
%:%1749=536%:%
%:%1750=536%:%
%:%1751=536%:%
%:%1752=537%:%
%:%1753=538%:%
%:%1754=538%:%
%:%1755=539%:%
%:%1756=539%:%
%:%1757=539%:%
%:%1758=539%:%
%:%1759=540%:%
%:%1760=540%:%
%:%1761=540%:%
%:%1762=541%:%
%:%1763=541%:%
%:%1764=541%:%
%:%1765=542%:%
%:%1766=543%:%
%:%1767=543%:%
%:%1768=544%:%
%:%1769=544%:%
%:%1770=545%:%
%:%1771=545%:%
%:%1772=546%:%
%:%1773=546%:%
%:%1774=546%:%
%:%1775=547%:%
%:%1776=547%:%
%:%1777=547%:%
%:%1778=548%:%
%:%1779=548%:%
%:%1780=548%:%
%:%1781=549%:%
%:%1782=549%:%
%:%1783=549%:%
%:%1784=550%:%
%:%1785=550%:%
%:%1786=551%:%
%:%1787=551%:%
%:%1788=551%:%
%:%1789=552%:%
%:%1790=553%:%
%:%1791=553%:%
%:%1792=554%:%
%:%1793=554%:%
%:%1794=554%:%
%:%1795=555%:%
%:%1796=556%:%
%:%1797=556%:%
%:%1798=557%:%
%:%1799=557%:%
%:%1800=557%:%
%:%1801=558%:%
%:%1802=559%:%
%:%1803=559%:%
%:%1804=559%:%
%:%1805=560%:%
%:%1806=560%:%
%:%1807=560%:%
%:%1808=561%:%
%:%1809=561%:%
%:%1810=561%:%
%:%1811=562%:%
%:%1812=562%:%
%:%1813=562%:%
%:%1814=563%:%
%:%1815=563%:%
%:%1816=563%:%
%:%1817=564%:%
%:%1818=564%:%
%:%1819=564%:%
%:%1820=565%:%
%:%1821=565%:%
%:%1822=565%:%
%:%1823=566%:%
%:%1824=566%:%
%:%1825=566%:%
%:%1826=567%:%
%:%1827=567%:%
%:%1828=567%:%
%:%1829=568%:%
%:%1830=568%:%
%:%1831=568%:%
%:%1832=569%:%
%:%1833=569%:%
%:%1834=570%:%
%:%1835=570%:%
%:%1836=571%:%
%:%1837=571%:%
%:%1838=571%:%
%:%1839=571%:%
%:%1840=572%:%
%:%1841=572%:%
%:%1842=572%:%
%:%1843=572%:%
%:%1844=572%:%
%:%1845=573%:%
%:%1846=573%:%
%:%1847=574%:%
%:%1848=574%:%
%:%1849=575%:%
%:%1850=575%:%
%:%1851=575%:%
%:%1852=576%:%
%:%1853=576%:%
%:%1854=576%:%
%:%1855=577%:%
%:%1856=578%:%
%:%1857=578%:%
%:%1858=578%:%
%:%1859=579%:%
%:%1860=580%:%
%:%1861=580%:%
%:%1862=581%:%
%:%1863=581%:%
%:%1864=581%:%
%:%1865=582%:%
%:%1866=582%:%
%:%1867=582%:%
%:%1868=583%:%
%:%1869=584%:%
%:%1870=584%:%
%:%1871=584%:%
%:%1872=585%:%
%:%1873=585%:%
%:%1874=585%:%
%:%1875=586%:%
%:%1876=586%:%
%:%1877=586%:%
%:%1878=587%:%
%:%1879=587%:%
%:%1880=587%:%
%:%1881=588%:%
%:%1882=588%:%
%:%1883=588%:%
%:%1884=589%:%
%:%1885=589%:%
%:%1886=589%:%
%:%1887=590%:%
%:%1888=590%:%
%:%1889=590%:%
%:%1890=590%:%
%:%1891=591%:%
%:%1892=591%:%
%:%1893=591%:%
%:%1894=592%:%
%:%1895=592%:%
%:%1896=592%:%
%:%1897=593%:%
%:%1898=593%:%
%:%1899=594%:%
%:%1914=596%:%
%:%1918=598%:%
%:%1928=600%:%
%:%1929=600%:%
%:%1930=601%:%
%:%1931=602%:%
%:%1932=602%:%
%:%1933=603%:%
%:%1934=604%:%
%:%1935=605%:%
%:%1936=606%:%
%:%1937=606%:%
%:%1938=607%:%
%:%1945=609%:%
%:%1955=611%:%
%:%1956=611%:%
%:%1959=612%:%
%:%1963=612%:%
%:%1964=612%:%
%:%1965=612%:%
%:%1970=612%:%
%:%1973=613%:%
%:%1974=614%:%
%:%1975=614%:%
%:%1978=615%:%
%:%1982=615%:%
%:%1983=615%:%
%:%1984=615%:%
%:%1989=615%:%
%:%1992=616%:%
%:%1993=617%:%
%:%1994=617%:%
%:%1997=618%:%
%:%2001=618%:%
%:%2002=618%:%
%:%2007=618%:%
%:%2010=619%:%
%:%2011=620%:%
%:%2012=620%:%
%:%2015=621%:%
%:%2019=621%:%
%:%2020=621%:%
%:%2025=621%:%
%:%2028=622%:%
%:%2029=623%:%
%:%2030=623%:%
%:%2037=624%:%
%:%2038=624%:%
%:%2039=625%:%
%:%2040=625%:%
%:%2041=626%:%
%:%2042=626%:%
%:%2043=626%:%
%:%2044=627%:%
%:%2045=627%:%
%:%2046=628%:%
%:%2047=628%:%
%:%2048=629%:%
%:%2049=629%:%
%:%2050=629%:%
%:%2051=629%:%
%:%2052=629%:%
%:%2053=630%:%
%:%2054=630%:%
%:%2055=631%:%
%:%2056=631%:%
%:%2057=632%:%
%:%2058=632%:%
%:%2059=632%:%
%:%2060=632%:%
%:%2061=632%:%
%:%2062=633%:%
%:%2063=633%:%
%:%2064=634%:%
%:%2065=634%:%
%:%2070=634%:%
%:%2073=635%:%
%:%2074=636%:%
%:%2075=636%:%
%:%2076=637%:%
%:%2083=638%:%
%:%2084=638%:%
%:%2085=639%:%
%:%2086=639%:%
%:%2087=640%:%
%:%2088=640%:%
%:%2089=640%:%
%:%2090=640%:%
%:%2091=641%:%
%:%2092=641%:%
%:%2093=642%:%
%:%2094=642%:%
%:%2095=643%:%
%:%2096=643%:%
%:%2097=643%:%
%:%2098=643%:%
%:%2099=644%:%
%:%2100=644%:%
%:%2101=644%:%
%:%2102=644%:%
%:%2103=645%:%
%:%2104=645%:%
%:%2105=645%:%
%:%2106=646%:%
%:%2107=646%:%
%:%2108=647%:%
%:%2109=647%:%
%:%2110=648%:%
%:%2111=648%:%
%:%2112=648%:%
%:%2113=648%:%
%:%2114=648%:%
%:%2115=649%:%
%:%2116=649%:%
%:%2117=650%:%
%:%2123=650%:%
%:%2126=651%:%
%:%2127=652%:%
%:%2128=652%:%
%:%2129=653%:%
%:%2136=654%:%
%:%2137=654%:%
%:%2138=655%:%
%:%2139=655%:%
%:%2140=656%:%
%:%2141=656%:%
%:%2142=656%:%
%:%2143=656%:%
%:%2144=657%:%
%:%2145=657%:%
%:%2146=658%:%
%:%2147=658%:%
%:%2148=659%:%
%:%2149=659%:%
%:%2150=659%:%
%:%2151=659%:%
%:%2152=660%:%
%:%2153=660%:%
%:%2154=660%:%
%:%2155=660%:%
%:%2156=661%:%
%:%2157=661%:%
%:%2158=661%:%
%:%2159=662%:%
%:%2160=662%:%
%:%2161=663%:%
%:%2162=663%:%
%:%2163=664%:%
%:%2164=664%:%
%:%2165=664%:%
%:%2166=664%:%
%:%2167=664%:%
%:%2168=665%:%
%:%2169=665%:%
%:%2170=666%:%
%:%2176=666%:%
%:%2179=667%:%
%:%2180=668%:%
%:%2181=668%:%
%:%2184=669%:%
%:%2188=669%:%
%:%2189=669%:%
%:%2190=669%:%
%:%2195=669%:%
%:%2198=670%:%
%:%2199=671%:%
%:%2200=671%:%
%:%2203=672%:%
%:%2207=672%:%
%:%2208=672%:%
%:%2209=672%:%
%:%2210=672%:%
%:%2215=672%:%
%:%2218=673%:%
%:%2219=674%:%
%:%2220=674%:%
%:%2221=675%:%
%:%2224=676%:%
%:%2228=676%:%
%:%2229=676%:%
%:%2230=676%:%
%:%2235=676%:%
%:%2238=677%:%
%:%2239=678%:%
%:%2240=678%:%
%:%2243=679%:%
%:%2247=679%:%
%:%2248=679%:%
%:%2249=679%:%
%:%2254=679%:%
%:%2257=680%:%
%:%2258=681%:%
%:%2259=681%:%
%:%2262=682%:%
%:%2266=682%:%
%:%2267=682%:%
%:%2268=682%:%
%:%2273=682%:%
%:%2276=683%:%
%:%2277=684%:%
%:%2278=684%:%
%:%2279=685%:%
%:%2282=686%:%
%:%2286=686%:%
%:%2287=686%:%
%:%2288=686%:%
%:%2293=686%:%
%:%2296=687%:%
%:%2297=688%:%
%:%2298=688%:%
%:%2299=689%:%
%:%2300=690%:%
%:%2303=691%:%
%:%2307=691%:%
%:%2308=691%:%
%:%2309=692%:%
%:%2310=692%:%
%:%2311=693%:%
%:%2312=693%:%
%:%2313=694%:%
%:%2314=694%:%
%:%2315=694%:%
%:%2316=695%:%
%:%2317=695%:%
%:%2318=695%:%
%:%2319=696%:%
%:%2320=696%:%
%:%2321=696%:%
%:%2322=696%:%
%:%2323=697%:%
%:%2324=697%:%
%:%2325=698%:%
%:%2326=698%:%
%:%2327=699%:%
%:%2328=699%:%
%:%2329=699%:%
%:%2330=700%:%
%:%2331=701%:%
%:%2332=702%:%
%:%2333=702%:%
%:%2334=702%:%
%:%2335=703%:%
%:%2336=703%:%
%:%2337=703%:%
%:%2338=704%:%
%:%2339=704%:%
%:%2340=704%:%
%:%2341=705%:%
%:%2347=705%:%
%:%2350=706%:%
%:%2351=707%:%
%:%2352=707%:%
%:%2353=708%:%
%:%2354=709%:%
%:%2357=710%:%
%:%2361=710%:%
%:%2362=710%:%
%:%2363=710%:%
%:%2368=710%:%
%:%2371=711%:%
%:%2372=712%:%
%:%2373=712%:%
%:%2374=713%:%
%:%2375=714%:%
%:%2378=715%:%
%:%2382=715%:%
%:%2383=715%:%
%:%2384=716%:%
%:%2385=716%:%
%:%2386=717%:%
%:%2387=717%:%
%:%2388=718%:%
%:%2389=718%:%
%:%2390=718%:%
%:%2391=718%:%
%:%2392=719%:%
%:%2393=719%:%
%:%2394=719%:%
%:%2395=720%:%
%:%2396=720%:%
%:%2397=720%:%
%:%2398=721%:%
%:%2399=721%:%
%:%2400=721%:%
%:%2401=722%:%
%:%2402=722%:%
%:%2403=722%:%
%:%2404=723%:%
%:%2405=723%:%
%:%2406=723%:%
%:%2407=723%:%
%:%2408=723%:%
%:%2409=724%:%
%:%2424=726%:%
%:%2428=728%:%
%:%2438=730%:%
%:%2439=730%:%
%:%2440=731%:%
%:%2441=731%:%
%:%2442=732%:%
%:%2443=732%:%
%:%2444=733%:%
%:%2445=733%:%
%:%2446=734%:%
%:%2447=735%:%
%:%2448=735%:%
%:%2449=736%:%
%:%2450=737%:%
%:%2451=738%:%
%:%2452=738%:%
%:%2453=739%:%
%:%2454=740%:%
%:%2455=741%:%
%:%2456=742%:%
%:%2457=742%:%
%:%2458=743%:%
%:%2459=744%:%
%:%2460=744%:%
%:%2461=745%:%
%:%2462=746%:%
%:%2463=747%:%
%:%2464=747%:%
%:%2465=748%:%
%:%2466=749%:%
%:%2467=750%:%
%:%2468=750%:%
%:%2469=751%:%
%:%2470=752%:%
%:%2471=753%:%
%:%2472=754%:%
%:%2473=754%:%
%:%2474=755%:%
%:%2475=756%:%
%:%2476=756%:%
%:%2477=757%:%
%:%2484=759%:%
%:%2494=761%:%
%:%2495=761%:%
%:%2502=762%:%
%:%2503=762%:%
%:%2504=763%:%
%:%2505=763%:%
%:%2506=764%:%
%:%2507=764%:%
%:%2508=765%:%
%:%2509=765%:%
%:%2510=766%:%
%:%2511=766%:%
%:%2512=767%:%
%:%2513=767%:%
%:%2514=767%:%
%:%2515=767%:%
%:%2516=768%:%
%:%2517=768%:%
%:%2518=768%:%
%:%2519=769%:%
%:%2520=769%:%
%:%2521=769%:%
%:%2522=770%:%
%:%2523=770%:%
%:%2524=770%:%
%:%2525=771%:%
%:%2526=771%:%
%:%2527=771%:%
%:%2528=772%:%
%:%2529=772%:%
%:%2530=772%:%
%:%2531=773%:%
%:%2532=773%:%
%:%2533=773%:%
%:%2534=774%:%
%:%2535=774%:%
%:%2536=774%:%
%:%2537=775%:%
%:%2538=775%:%
%:%2539=776%:%
%:%2540=776%:%
%:%2541=776%:%
%:%2542=777%:%
%:%2543=777%:%
%:%2544=777%:%
%:%2545=778%:%
%:%2546=778%:%
%:%2547=779%:%
%:%2548=779%:%
%:%2549=780%:%
%:%2550=780%:%
%:%2551=781%:%
%:%2552=781%:%
%:%2553=782%:%
%:%2554=782%:%
%:%2555=783%:%
%:%2556=783%:%
%:%2557=784%:%
%:%2558=784%:%
%:%2559=784%:%
%:%2560=784%:%
%:%2561=785%:%
%:%2562=785%:%
%:%2563=785%:%
%:%2564=786%:%
%:%2565=786%:%
%:%2566=786%:%
%:%2567=787%:%
%:%2568=787%:%
%:%2569=787%:%
%:%2570=788%:%
%:%2571=788%:%
%:%2572=788%:%
%:%2573=789%:%
%:%2574=789%:%
%:%2575=789%:%
%:%2576=789%:%
%:%2577=790%:%
%:%2578=790%:%
%:%2579=790%:%
%:%2580=791%:%
%:%2581=791%:%
%:%2582=792%:%
%:%2583=792%:%
%:%2584=792%:%
%:%2585=793%:%
%:%2586=793%:%
%:%2587=793%:%
%:%2588=794%:%
%:%2589=794%:%
%:%2590=794%:%
%:%2591=795%:%
%:%2592=795%:%
%:%2593=796%:%
%:%2599=796%:%
%:%2602=797%:%
%:%2603=798%:%
%:%2604=798%:%
%:%2611=799%:%
%:%2612=799%:%
%:%2613=800%:%
%:%2614=800%:%
%:%2615=801%:%
%:%2616=801%:%
%:%2617=802%:%
%:%2618=802%:%
%:%2619=803%:%
%:%2620=803%:%
%:%2621=804%:%
%:%2622=804%:%
%:%2623=804%:%
%:%2624=804%:%
%:%2625=805%:%
%:%2626=805%:%
%:%2627=805%:%
%:%2628=806%:%
%:%2629=806%:%
%:%2630=806%:%
%:%2631=807%:%
%:%2632=807%:%
%:%2633=807%:%
%:%2634=808%:%
%:%2635=808%:%
%:%2636=808%:%
%:%2637=809%:%
%:%2638=809%:%
%:%2639=809%:%
%:%2640=810%:%
%:%2641=810%:%
%:%2642=810%:%
%:%2643=811%:%
%:%2644=811%:%
%:%2645=811%:%
%:%2646=812%:%
%:%2647=812%:%
%:%2648=813%:%
%:%2649=813%:%
%:%2650=813%:%
%:%2651=814%:%
%:%2652=814%:%
%:%2653=814%:%
%:%2654=815%:%
%:%2655=815%:%
%:%2656=816%:%
%:%2657=816%:%
%:%2658=817%:%
%:%2659=817%:%
%:%2660=818%:%
%:%2661=818%:%
%:%2662=819%:%
%:%2663=819%:%
%:%2664=820%:%
%:%2665=820%:%
%:%2666=821%:%
%:%2667=821%:%
%:%2668=821%:%
%:%2669=821%:%
%:%2670=822%:%
%:%2671=822%:%
%:%2672=822%:%
%:%2673=823%:%
%:%2674=823%:%
%:%2675=823%:%
%:%2676=824%:%
%:%2677=824%:%
%:%2678=824%:%
%:%2679=825%:%
%:%2680=825%:%
%:%2681=825%:%
%:%2682=826%:%
%:%2683=826%:%
%:%2684=826%:%
%:%2685=826%:%
%:%2686=827%:%
%:%2687=827%:%
%:%2688=827%:%
%:%2689=828%:%
%:%2690=828%:%
%:%2691=829%:%
%:%2692=829%:%
%:%2693=829%:%
%:%2694=830%:%
%:%2695=830%:%
%:%2696=830%:%
%:%2697=831%:%
%:%2698=831%:%
%:%2699=831%:%
%:%2700=832%:%
%:%2701=832%:%
%:%2702=833%:%
%:%2708=833%:%
%:%2711=834%:%
%:%2712=835%:%
%:%2713=835%:%
%:%2720=836%:%
%:%2721=836%:%
%:%2722=837%:%
%:%2723=837%:%
%:%2724=838%:%
%:%2725=838%:%
%:%2726=838%:%
%:%2727=839%:%
%:%2728=839%:%
%:%2729=839%:%
%:%2730=840%:%
%:%2731=840%:%
%:%2732=840%:%
%:%2733=841%:%
%:%2734=841%:%
%:%2735=841%:%
%:%2736=842%:%
%:%2737=842%:%
%:%2738=842%:%
%:%2739=843%:%
%:%2740=844%:%
%:%2741=845%:%
%:%2742=845%:%
%:%2743=845%:%
%:%2744=846%:%
%:%2745=846%:%
%:%2746=846%:%
%:%2747=847%:%
%:%2748=847%:%
%:%2749=847%:%
%:%2750=848%:%
%:%2751=848%:%
%:%2752=848%:%
%:%2753=849%:%
%:%2754=849%:%
%:%2755=849%:%
%:%2756=850%:%
%:%2762=850%:%
%:%2765=851%:%
%:%2766=852%:%
%:%2767=852%:%
%:%2774=853%:%
%:%2775=853%:%
%:%2776=854%:%
%:%2777=854%:%
%:%2778=855%:%
%:%2779=855%:%
%:%2780=855%:%
%:%2781=856%:%
%:%2782=856%:%
%:%2783=856%:%
%:%2784=857%:%
%:%2785=857%:%
%:%2786=857%:%
%:%2787=858%:%
%:%2788=858%:%
%:%2789=858%:%
%:%2790=859%:%
%:%2791=859%:%
%:%2792=859%:%
%:%2793=860%:%
%:%2794=861%:%
%:%2795=862%:%
%:%2796=862%:%
%:%2797=862%:%
%:%2798=863%:%
%:%2799=863%:%
%:%2800=863%:%
%:%2801=864%:%
%:%2802=864%:%
%:%2803=864%:%
%:%2804=865%:%
%:%2805=865%:%
%:%2806=865%:%
%:%2807=866%:%
%:%2808=866%:%
%:%2809=866%:%
%:%2810=867%:%
%:%2816=867%:%
%:%2819=868%:%
%:%2820=869%:%
%:%2821=869%:%
%:%2828=870%:%
%:%2829=870%:%
%:%2830=871%:%
%:%2831=871%:%
%:%2832=872%:%
%:%2833=872%:%
%:%2834=872%:%
%:%2835=873%:%
%:%2836=873%:%
%:%2837=873%:%
%:%2838=874%:%
%:%2839=874%:%
%:%2840=874%:%
%:%2841=875%:%
%:%2842=875%:%
%:%2843=875%:%
%:%2844=876%:%
%:%2845=876%:%
%:%2846=876%:%
%:%2847=877%:%
%:%2848=878%:%
%:%2849=879%:%
%:%2850=879%:%
%:%2851=879%:%
%:%2852=880%:%
%:%2853=880%:%
%:%2854=880%:%
%:%2855=881%:%
%:%2856=881%:%
%:%2857=881%:%
%:%2858=882%:%
%:%2859=882%:%
%:%2860=882%:%
%:%2861=883%:%
%:%2862=883%:%
%:%2863=883%:%
%:%2864=884%:%
%:%2870=884%:%
%:%2873=885%:%
%:%2874=886%:%
%:%2875=886%:%
%:%2882=887%:%
%:%2883=887%:%
%:%2884=888%:%
%:%2885=888%:%
%:%2886=889%:%
%:%2887=889%:%
%:%2888=889%:%
%:%2889=890%:%
%:%2890=890%:%
%:%2891=890%:%
%:%2892=891%:%
%:%2893=891%:%
%:%2894=891%:%
%:%2895=892%:%
%:%2896=892%:%
%:%2897=892%:%
%:%2898=893%:%
%:%2899=893%:%
%:%2900=893%:%
%:%2901=894%:%
%:%2902=895%:%
%:%2903=896%:%
%:%2904=896%:%
%:%2905=896%:%
%:%2906=897%:%
%:%2907=897%:%
%:%2908=897%:%
%:%2909=898%:%
%:%2910=898%:%
%:%2911=898%:%
%:%2912=899%:%
%:%2913=899%:%
%:%2914=899%:%
%:%2915=900%:%
%:%2916=900%:%
%:%2917=900%:%
%:%2918=901%:%
%:%2924=901%:%
%:%2927=902%:%
%:%2928=903%:%
%:%2929=903%:%
%:%2932=904%:%
%:%2936=904%:%
%:%2937=904%:%
%:%2938=904%:%
%:%2943=904%:%
%:%2946=905%:%
%:%2947=906%:%
%:%2948=906%:%
%:%2951=907%:%
%:%2955=907%:%
%:%2956=907%:%
%:%2957=907%:%
%:%2958=907%:%
%:%2963=907%:%
%:%2966=908%:%
%:%2967=909%:%
%:%2968=909%:%
%:%2971=910%:%
%:%2975=910%:%
%:%2976=910%:%
%:%2977=910%:%
%:%2982=910%:%
%:%2985=911%:%
%:%2986=912%:%
%:%2987=912%:%
%:%2990=913%:%
%:%2994=913%:%
%:%2995=913%:%
%:%2996=913%:%
%:%2997=913%:%
%:%3002=913%:%
%:%3005=914%:%
%:%3006=915%:%
%:%3007=915%:%
%:%3010=916%:%
%:%3014=916%:%
%:%3015=916%:%
%:%3020=916%:%
%:%3023=917%:%
%:%3024=918%:%
%:%3025=918%:%
%:%3028=919%:%
%:%3032=919%:%
%:%3033=919%:%
%:%3034=919%:%
%:%3035=919%:%
%:%3040=919%:%
%:%3043=920%:%
%:%3044=921%:%
%:%3045=921%:%
%:%3048=922%:%
%:%3052=922%:%
%:%3053=922%:%
%:%3058=922%:%
%:%3061=923%:%
%:%3062=924%:%
%:%3063=924%:%
%:%3066=925%:%
%:%3070=925%:%
%:%3071=925%:%
%:%3072=925%:%
%:%3073=925%:%
%:%3078=925%:%
%:%3081=926%:%
%:%3082=927%:%
%:%3083=927%:%
%:%3086=928%:%
%:%3090=928%:%
%:%3091=928%:%
%:%3092=928%:%
%:%3097=928%:%
%:%3100=929%:%
%:%3101=930%:%
%:%3102=930%:%
%:%3105=931%:%
%:%3109=931%:%
%:%3110=931%:%
%:%3111=931%:%
%:%3116=931%:%
%:%3119=932%:%
%:%3120=933%:%
%:%3121=933%:%
%:%3122=934%:%
%:%3125=935%:%
%:%3129=935%:%
%:%3130=935%:%
%:%3131=935%:%
%:%3136=935%:%
%:%3139=936%:%
%:%3140=937%:%
%:%3141=937%:%
%:%3142=938%:%
%:%3145=939%:%
%:%3149=939%:%
%:%3150=939%:%
%:%3151=939%:%
%:%3156=939%:%
%:%3159=940%:%
%:%3160=941%:%
%:%3161=941%:%
%:%3164=942%:%
%:%3168=942%:%
%:%3169=942%:%
%:%3174=942%:%
%:%3177=943%:%
%:%3178=944%:%
%:%3179=944%:%
%:%3180=945%:%
%:%3187=946%:%
%:%3188=946%:%
%:%3189=947%:%
%:%3190=947%:%
%:%3191=948%:%
%:%3192=948%:%
%:%3193=948%:%
%:%3194=949%:%
%:%3195=949%:%
%:%3196=949%:%
%:%3197=950%:%
%:%3198=950%:%
%:%3199=950%:%
%:%3200=951%:%
%:%3201=951%:%
%:%3202=952%:%
%:%3203=952%:%
%:%3204=953%:%
%:%3205=953%:%
%:%3206=954%:%
%:%3207=954%:%
%:%3208=955%:%
%:%3209=955%:%
%:%3210=955%:%
%:%3211=956%:%
%:%3212=956%:%
%:%3213=956%:%
%:%3214=957%:%
%:%3215=957%:%
%:%3216=958%:%
%:%3217=958%:%
%:%3218=959%:%
%:%3219=959%:%
%:%3220=959%:%
%:%3221=960%:%
%:%3222=960%:%
%:%3223=960%:%
%:%3224=961%:%
%:%3225=961%:%
%:%3226=961%:%
%:%3227=962%:%
%:%3228=962%:%
%:%3229=963%:%
%:%3230=963%:%
%:%3231=964%:%
%:%3232=964%:%
%:%3233=965%:%
%:%3234=965%:%
%:%3235=966%:%
%:%3236=966%:%
%:%3237=966%:%
%:%3238=967%:%
%:%3239=967%:%
%:%3240=967%:%
%:%3241=968%:%
%:%3247=968%:%
%:%3250=969%:%
%:%3251=970%:%
%:%3252=970%:%
%:%3253=971%:%
%:%3260=972%:%
%:%3261=972%:%
%:%3262=973%:%
%:%3263=973%:%
%:%3264=974%:%
%:%3265=974%:%
%:%3266=974%:%
%:%3267=975%:%
%:%3268=975%:%
%:%3269=975%:%
%:%3270=976%:%
%:%3271=976%:%
%:%3272=976%:%
%:%3273=977%:%
%:%3274=977%:%
%:%3275=978%:%
%:%3276=978%:%
%:%3277=979%:%
%:%3278=979%:%
%:%3279=980%:%
%:%3280=980%:%
%:%3281=981%:%
%:%3282=981%:%
%:%3283=981%:%
%:%3284=982%:%
%:%3285=982%:%
%:%3286=982%:%
%:%3287=983%:%
%:%3288=983%:%
%:%3289=984%:%
%:%3290=984%:%
%:%3291=985%:%
%:%3292=985%:%
%:%3293=985%:%
%:%3294=986%:%
%:%3295=986%:%
%:%3296=986%:%
%:%3297=987%:%
%:%3298=987%:%
%:%3299=987%:%
%:%3300=988%:%
%:%3301=988%:%
%:%3302=989%:%
%:%3303=989%:%
%:%3304=990%:%
%:%3305=990%:%
%:%3306=991%:%
%:%3307=991%:%
%:%3308=992%:%
%:%3309=992%:%
%:%3310=992%:%
%:%3311=993%:%
%:%3312=993%:%
%:%3313=993%:%
%:%3314=994%:%
%:%3320=994%:%
%:%3323=995%:%
%:%3324=996%:%
%:%3325=996%:%
%:%3326=997%:%
%:%3327=998%:%
%:%3330=999%:%
%:%3334=999%:%
%:%3335=999%:%
%:%3336=1000%:%
%:%3337=1000%:%
%:%3338=1001%:%
%:%3339=1001%:%
%:%3340=1002%:%
%:%3341=1002%:%
%:%3342=1002%:%
%:%3343=1002%:%
%:%3344=1003%:%
%:%3345=1003%:%
%:%3346=1003%:%
%:%3347=1004%:%
%:%3348=1004%:%
%:%3349=1004%:%
%:%3350=1005%:%
%:%3351=1005%:%
%:%3352=1005%:%
%:%3353=1006%:%
%:%3354=1006%:%
%:%3355=1006%:%
%:%3356=1007%:%
%:%3357=1007%:%
%:%3358=1007%:%
%:%3359=1007%:%
%:%3360=1008%:%
%:%3366=1008%:%
%:%3369=1009%:%
%:%3370=1010%:%
%:%3371=1010%:%
%:%3372=1011%:%
%:%3373=1012%:%
%:%3376=1013%:%
%:%3380=1013%:%
%:%3381=1013%:%
%:%3382=1014%:%
%:%3383=1014%:%
%:%3384=1015%:%
%:%3385=1015%:%
%:%3386=1016%:%
%:%3387=1016%:%
%:%3388=1016%:%
%:%3389=1017%:%
%:%3390=1017%:%
%:%3391=1017%:%
%:%3392=1018%:%
%:%3393=1018%:%
%:%3398=1018%:%
%:%3401=1019%:%
%:%3402=1020%:%
%:%3403=1020%:%
%:%3404=1021%:%
%:%3405=1022%:%
%:%3408=1023%:%
%:%3412=1023%:%
%:%3413=1023%:%
%:%3414=1023%:%
%:%3415=1024%:%
%:%3416=1024%:%
%:%3421=1024%:%
%:%3424=1025%:%
%:%3425=1026%:%
%:%3426=1026%:%
%:%3427=1027%:%
%:%3428=1028%:%
%:%3431=1029%:%
%:%3435=1029%:%
%:%3436=1029%:%
%:%3437=1030%:%
%:%3438=1030%:%
%:%3439=1031%:%
%:%3440=1031%:%
%:%3441=1032%:%
%:%3442=1032%:%
%:%3443=1032%:%
%:%3444=1032%:%
%:%3445=1033%:%
%:%3446=1033%:%
%:%3447=1033%:%
%:%3448=1034%:%
%:%3449=1034%:%
%:%3450=1034%:%
%:%3451=1035%:%
%:%3452=1035%:%
%:%3453=1035%:%
%:%3454=1036%:%
%:%3455=1036%:%
%:%3456=1036%:%
%:%3457=1037%:%
%:%3458=1037%:%
%:%3459=1037%:%
%:%3460=1037%:%
%:%3461=1038%:%
%:%3467=1038%:%
%:%3470=1039%:%
%:%3471=1040%:%
%:%3472=1040%:%
%:%3473=1041%:%
%:%3474=1042%:%
%:%3477=1043%:%
%:%3481=1043%:%
%:%3482=1043%:%
%:%3483=1044%:%
%:%3484=1044%:%
%:%3485=1045%:%
%:%3486=1045%:%
%:%3487=1046%:%
%:%3488=1046%:%
%:%3489=1046%:%
%:%3490=1047%:%
%:%3491=1047%:%
%:%3492=1047%:%
%:%3493=1048%:%
%:%3494=1048%:%
%:%3499=1048%:%
%:%3502=1049%:%
%:%3503=1050%:%
%:%3504=1050%:%
%:%3505=1051%:%
%:%3506=1052%:%
%:%3509=1053%:%
%:%3513=1053%:%
%:%3514=1053%:%
%:%3515=1053%:%
%:%3516=1054%:%
%:%3517=1054%:%
%:%3522=1054%:%
%:%3525=1055%:%
%:%3526=1056%:%
%:%3527=1056%:%
%:%3528=1057%:%
%:%3529=1058%:%
%:%3532=1059%:%
%:%3536=1059%:%
%:%3537=1059%:%
%:%3538=1060%:%
%:%3539=1060%:%
%:%3540=1061%:%
%:%3541=1061%:%
%:%3542=1062%:%
%:%3543=1062%:%
%:%3544=1062%:%
%:%3545=1063%:%
%:%3546=1063%:%
%:%3547=1063%:%
%:%3548=1064%:%
%:%3549=1064%:%
%:%3550=1064%:%
%:%3551=1064%:%
%:%3552=1065%:%
%:%3553=1065%:%
%:%3554=1066%:%
%:%3555=1066%:%
%:%3556=1067%:%
%:%3557=1067%:%
%:%3558=1067%:%
%:%3559=1068%:%
%:%3560=1069%:%
%:%3561=1070%:%
%:%3562=1070%:%
%:%3563=1070%:%
%:%3564=1071%:%
%:%3565=1071%:%
%:%3566=1071%:%
%:%3567=1072%:%
%:%3568=1072%:%
%:%3569=1072%:%
%:%3570=1073%:%
%:%3576=1073%:%
%:%3579=1074%:%
%:%3580=1075%:%
%:%3581=1075%:%
%:%3582=1076%:%
%:%3583=1077%:%
%:%3586=1078%:%
%:%3590=1078%:%
%:%3591=1078%:%
%:%3592=1079%:%
%:%3593=1079%:%
%:%3594=1080%:%
%:%3595=1080%:%
%:%3596=1081%:%
%:%3597=1081%:%
%:%3598=1081%:%
%:%3599=1082%:%
%:%3600=1082%:%
%:%3601=1082%:%
%:%3602=1083%:%
%:%3603=1083%:%
%:%3604=1083%:%
%:%3605=1084%:%
%:%3606=1084%:%
%:%3607=1084%:%
%:%3608=1085%:%
%:%3609=1085%:%
%:%3610=1085%:%
%:%3611=1086%:%
%:%3612=1086%:%
%:%3613=1086%:%
%:%3614=1086%:%
%:%3615=1086%:%
%:%3616=1087%:%
%:%3622=1087%:%
%:%3625=1088%:%
%:%3626=1089%:%
%:%3627=1089%:%
%:%3628=1090%:%
%:%3629=1091%:%
%:%3632=1092%:%
%:%3636=1092%:%
%:%3637=1092%:%
%:%3638=1093%:%
%:%3639=1093%:%
%:%3640=1094%:%
%:%3641=1094%:%
%:%3642=1095%:%
%:%3643=1095%:%
%:%3644=1095%:%
%:%3645=1096%:%
%:%3646=1096%:%
%:%3647=1096%:%
%:%3648=1097%:%
%:%3649=1097%:%
%:%3650=1097%:%
%:%3651=1097%:%
%:%3652=1098%:%
%:%3653=1098%:%
%:%3654=1099%:%
%:%3655=1099%:%
%:%3656=1100%:%
%:%3657=1100%:%
%:%3658=1100%:%
%:%3659=1101%:%
%:%3660=1102%:%
%:%3661=1103%:%
%:%3662=1103%:%
%:%3663=1103%:%
%:%3664=1104%:%
%:%3665=1104%:%
%:%3666=1104%:%
%:%3667=1105%:%
%:%3668=1105%:%
%:%3669=1105%:%
%:%3670=1106%:%
%:%3676=1106%:%
%:%3679=1107%:%
%:%3680=1108%:%
%:%3681=1108%:%
%:%3682=1109%:%
%:%3683=1110%:%
%:%3690=1111%:%
%:%3691=1111%:%
%:%3692=1112%:%
%:%3693=1112%:%
%:%3694=1113%:%
%:%3695=1113%:%
%:%3696=1113%:%
%:%3697=1114%:%
%:%3698=1114%:%
%:%3699=1114%:%
%:%3700=1115%:%
%:%3701=1115%:%
%:%3702=1115%:%
%:%3703=1116%:%
%:%3704=1116%:%
%:%3705=1116%:%
%:%3706=1117%:%
%:%3707=1117%:%
%:%3708=1117%:%
%:%3709=1118%:%
%:%3710=1118%:%
%:%3711=1118%:%
%:%3712=1118%:%
%:%3713=1118%:%
%:%3714=1119%:%
%:%3729=1121%:%
%:%3741=1123%:%
%:%3742=1124%:%
%:%3744=1125%:%
%:%3745=1125%:%
%:%3746=1126%:%
%:%3747=1127%:%
%:%3750=1128%:%
%:%3754=1128%:%
%:%3755=1128%:%
%:%3756=1128%:%
%:%3761=1128%:%
%:%3764=1129%:%
%:%3765=1130%:%
%:%3766=1130%:%
%:%3767=1131%:%
%:%3768=1132%:%
%:%3771=1133%:%
%:%3775=1133%:%
%:%3776=1133%:%
%:%3777=1134%:%
%:%3778=1134%:%
%:%3779=1135%:%
%:%3780=1135%:%
%:%3781=1136%:%
%:%3782=1136%:%
%:%3783=1137%:%
%:%3784=1137%:%
%:%3785=1138%:%
%:%3786=1138%:%
%:%3787=1138%:%
%:%3788=1138%:%
%:%3789=1138%:%
%:%3790=1139%:%
%:%3796=1139%:%
%:%3799=1140%:%
%:%3800=1141%:%
%:%3801=1141%:%
%:%3802=1142%:%
%:%3803=1143%:%
%:%3806=1144%:%
%:%3810=1144%:%
%:%3811=1144%:%
%:%3812=1145%:%
%:%3813=1145%:%
%:%3814=1146%:%
%:%3815=1146%:%
%:%3816=1147%:%
%:%3817=1147%:%
%:%3818=1148%:%
%:%3819=1148%:%
%:%3820=1149%:%
%:%3821=1149%:%
%:%3822=1149%:%
%:%3823=1149%:%
%:%3824=1149%:%
%:%3825=1150%:%
%:%3840=1152%:%
%:%3852=1154%:%
%:%3853=1155%:%
%:%3854=1156%:%
%:%3863=1158%:%
%:%3875=1160%:%
%:%3877=1161%:%
%:%3878=1161%:%
%:%3879=1162%:%
%:%3880=1163%:%
%:%3882=1165%:%
%:%3884=1166%:%
%:%3885=1166%:%
%:%3886=1167%:%
%:%3893=1169%:%
%:%3903=1171%:%
%:%3904=1171%:%
%:%3907=1172%:%
%:%3911=1172%:%
%:%3912=1172%:%
%:%3917=1172%:%
%:%3920=1173%:%
%:%3921=1174%:%
%:%3922=1174%:%
%:%3925=1175%:%
%:%3929=1175%:%
%:%3930=1175%:%
%:%3935=1175%:%
%:%3938=1176%:%
%:%3939=1177%:%
%:%3940=1177%:%
%:%3947=1178%:%
%:%3948=1178%:%
%:%3949=1179%:%
%:%3950=1179%:%
%:%3951=1180%:%
%:%3952=1180%:%
%:%3953=1180%:%
%:%3954=1180%:%
%:%3955=1181%:%
%:%3956=1181%:%
%:%3961=1181%:%
%:%3964=1182%:%
%:%3965=1183%:%
%:%3966=1183%:%
%:%3967=1184%:%
%:%3970=1185%:%
%:%3974=1185%:%
%:%3975=1185%:%
%:%3976=1185%:%
%:%3981=1185%:%
%:%3984=1186%:%
%:%3985=1187%:%
%:%3986=1187%:%
%:%3987=1188%:%
%:%3988=1189%:%
%:%3991=1190%:%
%:%3995=1190%:%
%:%3996=1190%:%
%:%3997=1190%:%
%:%4002=1190%:%
%:%4005=1191%:%
%:%4006=1192%:%
%:%4007=1192%:%
%:%4008=1193%:%
%:%4009=1194%:%
%:%4012=1195%:%
%:%4016=1195%:%
%:%4017=1195%:%
%:%4018=1196%:%
%:%4019=1196%:%
%:%4020=1197%:%
%:%4021=1197%:%
%:%4022=1198%:%
%:%4023=1198%:%
%:%4024=1198%:%
%:%4025=1199%:%
%:%4026=1199%:%
%:%4027=1200%:%
%:%4028=1200%:%
%:%4029=1201%:%
%:%4030=1201%:%
%:%4031=1202%:%
%:%4032=1202%:%
%:%4033=1203%:%
%:%4034=1203%:%
%:%4035=1204%:%
%:%4036=1204%:%
%:%4041=1204%:%
%:%4044=1205%:%
%:%4045=1206%:%
%:%4046=1206%:%
%:%4047=1207%:%
%:%4054=1208%:%
%:%4055=1208%:%
%:%4056=1209%:%
%:%4057=1209%:%
%:%4058=1210%:%
%:%4059=1210%:%
%:%4060=1210%:%
%:%4061=1210%:%
%:%4062=1211%:%
%:%4063=1211%:%
%:%4064=1212%:%
%:%4065=1212%:%
%:%4066=1213%:%
%:%4067=1213%:%
%:%4068=1213%:%
%:%4069=1213%:%
%:%4070=1214%:%
%:%4071=1214%:%
%:%4072=1214%:%
%:%4073=1214%:%
%:%4074=1215%:%
%:%4075=1215%:%
%:%4076=1215%:%
%:%4077=1216%:%
%:%4078=1216%:%
%:%4079=1217%:%
%:%4080=1217%:%
%:%4081=1218%:%
%:%4082=1218%:%
%:%4083=1218%:%
%:%4084=1218%:%
%:%4085=1218%:%
%:%4086=1219%:%
%:%4087=1219%:%
%:%4088=1220%:%
%:%4094=1220%:%
%:%4097=1221%:%
%:%4098=1222%:%
%:%4099=1222%:%
%:%4100=1223%:%
%:%4107=1224%:%
%:%4108=1224%:%
%:%4109=1225%:%
%:%4110=1225%:%
%:%4111=1226%:%
%:%4112=1226%:%
%:%4113=1226%:%
%:%4114=1226%:%
%:%4115=1227%:%
%:%4116=1227%:%
%:%4117=1228%:%
%:%4118=1228%:%
%:%4119=1229%:%
%:%4120=1229%:%
%:%4121=1229%:%
%:%4122=1229%:%
%:%4123=1230%:%
%:%4124=1230%:%
%:%4125=1230%:%
%:%4126=1230%:%
%:%4127=1231%:%
%:%4128=1231%:%
%:%4129=1231%:%
%:%4130=1232%:%
%:%4131=1232%:%
%:%4132=1233%:%
%:%4133=1233%:%
%:%4134=1234%:%
%:%4135=1234%:%
%:%4136=1234%:%
%:%4137=1234%:%
%:%4138=1234%:%
%:%4139=1235%:%
%:%4140=1235%:%
%:%4141=1236%:%
%:%4147=1236%:%
%:%4150=1237%:%
%:%4151=1238%:%
%:%4152=1238%:%
%:%4155=1239%:%
%:%4159=1239%:%
%:%4160=1239%:%
%:%4161=1239%:%
%:%4166=1239%:%
%:%4169=1240%:%
%:%4170=1241%:%
%:%4171=1241%:%
%:%4174=1242%:%
%:%4178=1242%:%
%:%4179=1242%:%
%:%4180=1242%:%
%:%4189=1244%:%
%:%4191=1245%:%
%:%4192=1245%:%
%:%4193=1246%:%
%:%4200=1247%:%
%:%4201=1247%:%
%:%4202=1248%:%
%:%4203=1248%:%
%:%4204=1248%:%
%:%4205=1248%:%
%:%4206=1249%:%
%:%4207=1249%:%
%:%4208=1250%:%
%:%4209=1250%:%
%:%4210=1251%:%
%:%4211=1251%:%
%:%4212=1252%:%
%:%4213=1252%:%
%:%4214=1253%:%
%:%4215=1253%:%
%:%4216=1254%:%
%:%4217=1254%:%
%:%4218=1254%:%
%:%4219=1255%:%
%:%4220=1255%:%
%:%4221=1255%:%
%:%4222=1256%:%
%:%4223=1256%:%
%:%4224=1256%:%
%:%4225=1256%:%
%:%4226=1257%:%
%:%4227=1257%:%
%:%4228=1257%:%
%:%4229=1258%:%
%:%4230=1258%:%
%:%4231=1259%:%
%:%4232=1259%:%
%:%4233=1260%:%
%:%4234=1260%:%
%:%4235=1260%:%
%:%4236=1260%:%
%:%4237=1260%:%
%:%4238=1260%:%
%:%4239=1261%:%
%:%4240=1261%:%
%:%4241=1262%:%
%:%4242=1262%:%
%:%4243=1263%:%
%:%4244=1263%:%
%:%4245=1263%:%
%:%4246=1263%:%
%:%4247=1263%:%
%:%4248=1263%:%
%:%4249=1264%:%
%:%4250=1264%:%
%:%4251=1264%:%
%:%4252=1265%:%
%:%4253=1265%:%
%:%4254=1265%:%
%:%4255=1266%:%
%:%4256=1266%:%
%:%4257=1267%:%
%:%4258=1267%:%
%:%4259=1268%:%
%:%4265=1268%:%
%:%4268=1269%:%
%:%4269=1270%:%
%:%4270=1270%:%
%:%4271=1271%:%
%:%4274=1272%:%
%:%4278=1272%:%
%:%4279=1272%:%
%:%4280=1272%:%
%:%4285=1272%:%
%:%4288=1273%:%
%:%4289=1274%:%
%:%4290=1274%:%
%:%4291=1275%:%
%:%4298=1276%:%
%:%4299=1276%:%
%:%4300=1277%:%
%:%4301=1277%:%
%:%4302=1278%:%
%:%4303=1278%:%
%:%4304=1279%:%
%:%4305=1279%:%
%:%4306=1280%:%
%:%4307=1280%:%
%:%4308=1280%:%
%:%4309=1281%:%
%:%4310=1281%:%
%:%4311=1281%:%
%:%4312=1282%:%
%:%4313=1282%:%
%:%4314=1282%:%
%:%4315=1283%:%
%:%4316=1283%:%
%:%4317=1283%:%
%:%4318=1284%:%
%:%4319=1284%:%
%:%4320=1284%:%
%:%4321=1285%:%
%:%4322=1285%:%
%:%4323=1285%:%
%:%4324=1285%:%
%:%4325=1286%:%
%:%4331=1286%:%
%:%4334=1287%:%
%:%4335=1288%:%
%:%4336=1288%:%
%:%4337=1289%:%
%:%4340=1290%:%
%:%4344=1290%:%
%:%4345=1290%:%
%:%4346=1290%:%
%:%4355=1292%:%
%:%4357=1293%:%
%:%4358=1293%:%
%:%4359=1294%:%
%:%4360=1295%:%
%:%4363=1296%:%
%:%4367=1296%:%
%:%4368=1296%:%
%:%4369=1297%:%
%:%4370=1297%:%
%:%4371=1298%:%
%:%4372=1298%:%
%:%4373=1299%:%
%:%4374=1299%:%
%:%4375=1299%:%
%:%4376=1299%:%
%:%4377=1300%:%
%:%4378=1300%:%
%:%4379=1301%:%
%:%4380=1301%:%
%:%4381=1302%:%
%:%4382=1302%:%
%:%4383=1303%:%
%:%4384=1303%:%
%:%4385=1304%:%
%:%4386=1304%:%
%:%4387=1305%:%
%:%4388=1306%:%
%:%4389=1306%:%
%:%4390=1306%:%
%:%4391=1306%:%
%:%4392=1307%:%
%:%4393=1307%:%
%:%4394=1308%:%
%:%4395=1308%:%
%:%4396=1308%:%
%:%4397=1309%:%
%:%4398=1309%:%
%:%4399=1309%:%
%:%4400=1310%:%
%:%4401=1310%:%
%:%4402=1310%:%
%:%4403=1311%:%
%:%4404=1311%:%
%:%4405=1311%:%
%:%4406=1311%:%
%:%4407=1311%:%
%:%4408=1312%:%
%:%4409=1312%:%
%:%4418=1314%:%
%:%4420=1315%:%
%:%4421=1315%:%
%:%4422=1316%:%
%:%4423=1317%:%
%:%4424=1318%:%
%:%4425=1318%:%
%:%4432=1319%:%
%:%4433=1319%:%
%:%4434=1320%:%
%:%4435=1320%:%
%:%4436=1321%:%
%:%4437=1321%:%
%:%4438=1321%:%
%:%4439=1322%:%
%:%4440=1322%:%
%:%4441=1322%:%
%:%4442=1323%:%
%:%4443=1323%:%
%:%4444=1323%:%
%:%4445=1324%:%
%:%4446=1325%:%
%:%4447=1325%:%
%:%4448=1326%:%
%:%4449=1326%:%
%:%4450=1326%:%
%:%4451=1327%:%
%:%4452=1327%:%
%:%4453=1328%:%
%:%4459=1328%:%
%:%4462=1329%:%
%:%4463=1330%:%
%:%4464=1330%:%
%:%4471=1331%:%
%:%4472=1331%:%
%:%4473=1332%:%
%:%4474=1332%:%
%:%4475=1333%:%
%:%4476=1333%:%
%:%4477=1333%:%
%:%4478=1334%:%
%:%4479=1334%:%
%:%4480=1334%:%
%:%4481=1335%:%
%:%4482=1335%:%
%:%4483=1335%:%
%:%4484=1336%:%
%:%4485=1337%:%
%:%4486=1337%:%
%:%4487=1338%:%
%:%4488=1338%:%
%:%4489=1338%:%
%:%4490=1339%:%
%:%4491=1339%:%
%:%4492=1340%:%
%:%4498=1340%:%
%:%4501=1341%:%
%:%4502=1342%:%
%:%4503=1342%:%
%:%4506=1343%:%
%:%4510=1343%:%
%:%4511=1343%:%
%:%4512=1343%:%
%:%4517=1343%:%
%:%4520=1344%:%
%:%4521=1345%:%
%:%4522=1345%:%
%:%4529=1346%:%
%:%4530=1346%:%
%:%4531=1347%:%
%:%4532=1347%:%
%:%4533=1348%:%
%:%4534=1348%:%
%:%4535=1348%:%
%:%4536=1349%:%
%:%4537=1349%:%
%:%4538=1349%:%
%:%4539=1350%:%
%:%4540=1350%:%
%:%4541=1350%:%
%:%4542=1350%:%
%:%4543=1351%:%
%:%4544=1351%:%
%:%4545=1351%:%
%:%4546=1352%:%
%:%4547=1352%:%
%:%4548=1353%:%
%:%4549=1353%:%
%:%4550=1354%:%
%:%4551=1354%:%
%:%4552=1354%:%
%:%4553=1355%:%
%:%4554=1355%:%
%:%4555=1355%:%
%:%4556=1356%:%
%:%4557=1356%:%
%:%4558=1356%:%
%:%4559=1357%:%
%:%4560=1357%:%
%:%4561=1357%:%
%:%4562=1358%:%
%:%4563=1358%:%
%:%4564=1359%:%
%:%4565=1359%:%
%:%4566=1360%:%
%:%4567=1360%:%
%:%4568=1360%:%
%:%4569=1361%:%
%:%4570=1361%:%
%:%4571=1361%:%
%:%4572=1362%:%
%:%4573=1362%:%
%:%4574=1362%:%
%:%4575=1363%:%
%:%4576=1363%:%
%:%4577=1363%:%
%:%4578=1364%:%
%:%4579=1365%:%
%:%4580=1365%:%
%:%4581=1366%:%
%:%4582=1366%:%
%:%4583=1367%:%
%:%4584=1367%:%
%:%4585=1368%:%
%:%4586=1368%:%
%:%4587=1368%:%
%:%4588=1369%:%
%:%4589=1369%:%
%:%4590=1369%:%
%:%4591=1370%:%
%:%4592=1370%:%
%:%4593=1371%:%
%:%4599=1371%:%
%:%4602=1372%:%
%:%4603=1373%:%
%:%4604=1373%:%
%:%4605=1374%:%
%:%4612=1375%:%
%:%4613=1375%:%
%:%4614=1376%:%
%:%4615=1376%:%
%:%4616=1377%:%
%:%4617=1377%:%
%:%4618=1378%:%
%:%4619=1378%:%
%:%4620=1378%:%
%:%4621=1379%:%
%:%4622=1379%:%
%:%4623=1379%:%
%:%4624=1380%:%
%:%4625=1380%:%
%:%4626=1380%:%
%:%4627=1380%:%
%:%4628=1381%:%
%:%4629=1381%:%
%:%4630=1381%:%
%:%4631=1382%:%
%:%4632=1383%:%
%:%4633=1383%:%
%:%4634=1384%:%
%:%4635=1385%:%
%:%4636=1386%:%
%:%4637=1386%:%
%:%4638=1387%:%
%:%4639=1387%:%
%:%4640=1387%:%
%:%4641=1388%:%
%:%4642=1388%:%
%:%4643=1389%:%
%:%4649=1389%:%
%:%4652=1390%:%
%:%4653=1391%:%
%:%4654=1391%:%
%:%4655=1392%:%
%:%4656=1393%:%
%:%4663=1394%:%
%:%4664=1394%:%
%:%4665=1395%:%
%:%4666=1395%:%
%:%4667=1396%:%
%:%4668=1396%:%
%:%4669=1396%:%
%:%4670=1397%:%
%:%4671=1398%:%
%:%4672=1398%:%
%:%4673=1398%:%
%:%4674=1399%:%
%:%4675=1399%:%
%:%4676=1399%:%
%:%4677=1399%:%
%:%4678=1400%:%
%:%4679=1400%:%
%:%4680=1401%:%
%:%4681=1401%:%
%:%4682=1401%:%
%:%4683=1402%:%
%:%4684=1402%:%
%:%4685=1402%:%
%:%4686=1403%:%
%:%4687=1403%:%
%:%4688=1403%:%
%:%4689=1404%:%
%:%4690=1404%:%
%:%4691=1404%:%
%:%4692=1405%:%
%:%4693=1405%:%
%:%4694=1405%:%
%:%4695=1405%:%
%:%4696=1406%:%
%:%4702=1406%:%
%:%4705=1407%:%
%:%4706=1408%:%
%:%4707=1408%:%
%:%4710=1409%:%
%:%4714=1409%:%
%:%4715=1409%:%
%:%4720=1409%:%
%:%4723=1410%:%
%:%4724=1411%:%
%:%4725=1411%:%
%:%4728=1412%:%
%:%4732=1412%:%
%:%4733=1412%:%
%:%4747=1414%:%
%:%4759=1416%:%
%:%4760=1417%:%
%:%4761=1418%:%
%:%4763=1419%:%
%:%4764=1419%:%
%:%4765=1420%:%
%:%4766=1421%:%
%:%4767=1422%:%
%:%4768=1423%:%
%:%4769=1424%:%
%:%4772=1425%:%
%:%4776=1425%:%
%:%4777=1425%:%
%:%4778=1426%:%
%:%4779=1426%:%
%:%4780=1427%:%
%:%4781=1427%:%
%:%4782=1428%:%
%:%4783=1428%:%
%:%4784=1429%:%
%:%4785=1429%:%
%:%4786=1430%:%
%:%4787=1430%:%
%:%4788=1431%:%
%:%4789=1431%:%
%:%4790=1431%:%
%:%4791=1431%:%
%:%4792=1432%:%
%:%4793=1432%:%
%:%4794=1432%:%
%:%4795=1432%:%
%:%4796=1432%:%
%:%4797=1433%:%
%:%4798=1434%:%
%:%4799=1434%:%
%:%4800=1435%:%
%:%4801=1436%:%
%:%4802=1436%:%
%:%4803=1436%:%
%:%4804=1436%:%
%:%4805=1437%:%
%:%4806=1437%:%
%:%4807=1438%:%
%:%4808=1438%:%
%:%4809=1439%:%
%:%4810=1440%:%
%:%4811=1440%:%
%:%4812=1440%:%
%:%4813=1441%:%
%:%4814=1442%:%
%:%4815=1442%:%
%:%4816=1443%:%
%:%4817=1443%:%
%:%4818=1443%:%
%:%4819=1444%:%
%:%4820=1444%:%
%:%4821=1444%:%
%:%4822=1445%:%
%:%4823=1445%:%
%:%4824=1445%:%
%:%4825=1446%:%
%:%4826=1446%:%
%:%4827=1446%:%
%:%4828=1447%:%
%:%4829=1448%:%
%:%4830=1448%:%
%:%4831=1448%:%
%:%4832=1449%:%
%:%4833=1450%:%
%:%4834=1450%:%
%:%4835=1450%:%
%:%4836=1450%:%
%:%4837=1451%:%
%:%4838=1451%:%
%:%4839=1451%:%
%:%4840=1451%:%
%:%4841=1451%:%
%:%4842=1452%:%
%:%4843=1452%:%
%:%4844=1452%:%
%:%4845=1452%:%
%:%4846=1453%:%
%:%4847=1454%:%
%:%4848=1454%:%
%:%4849=1455%:%
%:%4850=1455%:%
%:%4851=1455%:%
%:%4852=1456%:%
%:%4853=1456%:%
%:%4854=1456%:%
%:%4855=1457%:%
%:%4856=1457%:%
%:%4857=1457%:%
%:%4858=1458%:%
%:%4859=1458%:%
%:%4860=1458%:%
%:%4861=1459%:%
%:%4862=1459%:%
%:%4863=1459%:%
%:%4864=1460%:%
%:%4865=1460%:%
%:%4866=1460%:%
%:%4867=1461%:%
%:%4868=1461%:%
%:%4869=1461%:%
%:%4870=1462%:%
%:%4871=1463%:%
%:%4872=1463%:%
%:%4873=1464%:%
%:%4874=1464%:%
%:%4875=1464%:%
%:%4876=1465%:%
%:%4877=1466%:%
%:%4878=1466%:%
%:%4879=1467%:%
%:%4880=1467%:%
%:%4881=1467%:%
%:%4882=1468%:%
%:%4883=1469%:%
%:%4884=1469%:%
%:%4885=1469%:%
%:%4886=1470%:%
%:%4887=1470%:%
%:%4888=1471%:%
%:%4889=1471%:%
%:%4890=1472%:%
%:%4891=1473%:%
%:%4892=1473%:%
%:%4893=1474%:%
%:%4894=1475%:%
%:%4895=1475%:%
%:%4896=1475%:%
%:%4897=1475%:%
%:%4898=1476%:%
%:%4904=1476%:%
%:%4907=1477%:%
%:%4908=1478%:%
%:%4909=1478%:%
%:%4912=1479%:%
%:%4916=1479%:%
%:%4917=1479%:%
%:%4926=1481%:%
%:%4928=1482%:%
%:%4929=1482%:%
%:%4930=1483%:%
%:%4931=1484%:%
%:%4932=1485%:%
%:%4933=1486%:%
%:%4940=1487%:%
%:%4941=1487%:%
%:%4942=1488%:%
%:%4943=1488%:%
%:%4944=1489%:%
%:%4945=1489%:%
%:%4946=1489%:%
%:%4947=1490%:%
%:%4948=1490%:%
%:%4949=1490%:%
%:%4950=1491%:%
%:%4951=1491%:%
%:%4952=1492%:%
%:%4953=1492%:%
%:%4954=1493%:%
%:%4955=1493%:%
%:%4956=1494%:%
%:%4957=1495%:%
%:%4958=1495%:%
%:%4959=1495%:%
%:%4960=1496%:%
%:%4961=1496%:%
%:%4962=1496%:%
%:%4963=1497%:%
%:%4964=1497%:%
%:%4965=1497%:%
%:%4966=1498%:%
%:%4967=1498%:%
%:%4968=1498%:%
%:%4969=1499%:%
%:%4970=1499%:%
%:%4971=1499%:%
%:%4972=1500%:%
%:%4973=1500%:%
%:%4974=1500%:%
%:%4975=1501%:%
%:%4976=1501%:%
%:%4977=1502%:%
%:%4978=1502%:%
%:%4979=1503%:%
%:%4980=1503%:%
%:%4981=1503%:%
%:%4982=1504%:%
%:%4983=1504%:%
%:%4984=1504%:%
%:%4985=1505%:%
%:%4986=1505%:%
%:%4987=1505%:%
%:%4988=1505%:%
%:%4989=1506%:%
%:%4990=1506%:%
%:%4991=1506%:%
%:%4992=1506%:%
%:%4993=1506%:%
%:%4994=1507%:%
%:%4995=1507%:%
%:%4996=1507%:%
%:%4997=1508%:%
%:%4998=1508%:%
%:%4999=1508%:%
%:%5000=1509%:%
%:%5001=1509%:%
%:%5002=1510%:%
%:%5012=1512%:%
%:%5014=1513%:%
%:%5015=1513%:%
%:%5016=1514%:%
%:%5017=1515%:%
%:%5018=1516%:%
%:%5019=1517%:%
%:%5022=1518%:%
%:%5026=1518%:%
%:%5027=1518%:%
%:%5028=1519%:%
%:%5029=1519%:%
%:%5030=1520%:%
%:%5031=1520%:%
%:%5032=1521%:%
%:%5033=1521%:%
%:%5034=1522%:%
%:%5035=1522%:%
%:%5036=1523%:%
%:%5037=1523%:%
%:%5038=1523%:%
%:%5039=1524%:%
%:%5040=1525%:%
%:%5041=1525%:%
%:%5042=1526%:%
%:%5043=1526%:%
%:%5044=1527%:%
%:%5045=1527%:%
%:%5046=1528%:%
%:%5047=1528%:%
%:%5048=1528%:%
%:%5049=1529%:%
%:%5050=1530%:%
%:%5051=1530%:%
%:%5052=1531%:%
%:%5053=1531%:%
%:%5054=1531%:%
%:%5055=1532%:%
%:%5056=1532%:%
%:%5057=1532%:%
%:%5058=1533%:%
%:%5059=1533%:%
%:%5060=1533%:%
%:%5061=1534%:%
%:%5062=1535%:%
%:%5063=1535%:%
%:%5064=1536%:%
%:%5065=1536%:%
%:%5066=1536%:%
%:%5067=1537%:%
%:%5068=1538%:%
%:%5069=1538%:%
%:%5070=1538%:%
%:%5071=1539%:%
%:%5072=1540%:%
%:%5073=1540%:%
%:%5074=1540%:%
%:%5075=1541%:%
%:%5076=1541%:%
%:%5077=1541%:%
%:%5078=1541%:%
%:%5079=1541%:%
%:%5080=1542%:%
%:%5081=1542%:%
%:%5082=1543%:%
%:%5083=1543%:%
%:%5084=1544%:%
%:%5085=1544%:%
%:%5086=1545%:%
%:%5087=1545%:%
%:%5088=1546%:%
%:%5089=1546%:%
%:%5090=1546%:%
%:%5091=1546%:%
%:%5092=1547%:%
%:%5102=1549%:%
%:%5103=1550%:%
%:%5105=1551%:%
%:%5106=1551%:%
%:%5107=1552%:%
%:%5108=1553%:%
%:%5109=1554%:%
%:%5116=1555%:%
%:%5117=1555%:%
%:%5118=1556%:%
%:%5119=1556%:%
%:%5120=1557%:%
%:%5121=1557%:%
%:%5122=1558%:%
%:%5123=1558%:%
%:%5124=1559%:%
%:%5125=1559%:%
%:%5126=1559%:%
%:%5127=1560%:%
%:%5128=1560%:%
%:%5129=1561%:%
%:%5130=1561%:%
%:%5131=1561%:%
%:%5132=1562%:%
%:%5133=1562%:%
%:%5134=1562%:%
%:%5135=1563%:%
%:%5136=1564%:%
%:%5137=1564%:%
%:%5138=1564%:%
%:%5139=1565%:%
%:%5140=1565%:%
%:%5141=1565%:%
%:%5142=1566%:%
%:%5143=1566%:%
%:%5144=1566%:%
%:%5145=1567%:%
%:%5155=1569%:%
%:%5156=1570%:%
%:%5158=1571%:%
%:%5159=1571%:%
%:%5160=1572%:%
%:%5161=1573%:%
%:%5162=1574%:%
%:%5169=1575%:%
%:%5170=1575%:%
%:%5171=1576%:%
%:%5172=1576%:%
%:%5173=1577%:%
%:%5174=1577%:%
%:%5175=1577%:%
%:%5176=1577%:%
%:%5177=1578%:%
%:%5178=1578%:%
%:%5179=1578%:%
%:%5180=1579%:%
%:%5181=1579%:%
%:%5182=1579%:%
%:%5183=1580%:%
%:%5193=1582%:%
%:%5194=1583%:%
%:%5195=1584%:%
%:%5197=1585%:%
%:%5198=1585%:%
%:%5199=1586%:%
%:%5200=1587%:%
%:%5201=1588%:%
%:%5202=1589%:%
%:%5209=1590%:%
%:%5210=1590%:%
%:%5211=1591%:%
%:%5212=1591%:%
%:%5213=1592%:%
%:%5214=1593%:%
%:%5215=1593%:%
%:%5216=1593%:%
%:%5217=1593%:%
%:%5218=1594%:%
%:%5219=1595%:%
%:%5220=1595%:%
%:%5221=1595%:%
%:%5222=1596%:%
%:%5223=1597%:%
%:%5224=1597%:%
%:%5225=1598%:%
%:%5226=1598%:%
%:%5227=1598%:%
%:%5228=1599%:%
%:%5229=1599%:%
%:%5230=1599%:%
%:%5231=1600%:%
%:%5232=1601%:%
%:%5233=1601%:%
%:%5234=1602%:%
%:%5235=1603%:%
%:%5236=1604%:%
%:%5237=1604%:%
%:%5238=1605%:%
%:%5239=1605%:%
%:%5240=1605%:%
%:%5241=1606%:%
%:%5242=1606%:%
%:%5243=1606%:%
%:%5244=1607%:%
%:%5245=1607%:%
%:%5246=1607%:%
%:%5247=1608%:%
%:%5248=1608%:%
%:%5249=1608%:%
%:%5250=1609%:%
%:%5265=1611%:%
%:%5277=1613%:%
%:%5278=1614%:%
%:%5279=1615%:%
%:%5283=1617%:%
%:%5285=1618%:%
%:%5286=1618%:%
%:%5287=1619%:%
%:%5288=1620%:%
%:%5289=1621%:%
%:%5290=1622%:%
%:%5291=1622%:%
%:%5292=1623%:%
%:%5293=1624%:%
%:%5294=1625%:%
%:%5295=1626%:%
%:%5296=1626%:%
%:%5297=1627%:%
%:%5298=1628%:%
%:%5299=1629%:%
%:%5300=1630%:%
%:%5301=1630%:%
%:%5302=1631%:%
%:%5303=1632%:%
%:%5304=1633%:%
%:%5305=1634%:%
%:%5306=1634%:%
%:%5307=1635%:%
%:%5308=1636%:%
%:%5312=1640%:%
%:%5314=1641%:%
%:%5315=1641%:%
%:%5316=1642%:%
%:%5317=1643%:%
%:%5319=1645%:%
%:%5320=1646%:%
%:%5321=1647%:%
%:%5322=1647%:%
%:%5323=1648%:%
%:%5325=1650%:%
%:%5326=1651%:%
%:%5328=1652%:%
%:%5329=1652%:%
%:%5331=1654%:%
%:%5333=1655%:%
%:%5334=1655%:%
%:%5346=1662%:%
%:%5358=1664%:%
%:%5359=1665%:%
%:%5360=1666%:%
%:%5361=1667%:%
%:%5363=1669%:%
%:%5364=1669%:%
%:%5366=1671%:%
%:%5367=1672%:%
%:%5368=1673%:%
%:%5369=1674%:%
%:%5370=1675%:%
%:%5371=1676%:%
%:%5372=1677%:%
%:%5373=1678%:%
%:%5374=1679%:%
%:%5375=1680%:%
%:%5376=1681%:%
%:%5378=1682%:%
%:%5379=1682%:%
%:%5384=1687%:%
%:%5385=1688%:%
%:%5386=1688%:%
%:%5389=1689%:%
%:%5393=1689%:%
%:%5394=1689%:%
%:%5399=1689%:%
%:%5402=1690%:%
%:%5403=1691%:%
%:%5404=1691%:%
%:%5406=1691%:%
%:%5410=1691%:%
%:%5411=1691%:%
%:%5418=1691%:%
%:%5419=1692%:%
%:%5420=1693%:%
%:%5421=1693%:%
%:%5423=1693%:%
%:%5427=1693%:%
%:%5428=1693%:%
%:%5435=1693%:%
%:%5436=1694%:%
%:%5437=1695%:%
%:%5438=1695%:%
%:%5440=1695%:%
%:%5444=1695%:%
%:%5445=1695%:%
%:%5454=1697%:%
%:%5455=1698%:%
%:%5456=1699%:%
%:%5457=1700%:%
%:%5458=1701%:%
%:%5459=1702%:%
%:%5460=1703%:%
%:%5461=1704%:%
%:%5462=1705%:%
%:%5463=1706%:%
%:%5464=1707%:%
%:%5466=1708%:%
%:%5467=1708%:%
%:%5472=1713%:%
%:%5473=1714%:%
%:%5474=1714%:%
%:%5477=1715%:%
%:%5481=1715%:%
%:%5482=1715%:%
%:%5487=1715%:%
%:%5490=1716%:%
%:%5491=1717%:%
%:%5492=1717%:%
%:%5494=1717%:%
%:%5498=1717%:%
%:%5499=1717%:%
%:%5506=1717%:%
%:%5507=1718%:%
%:%5508=1719%:%
%:%5509=1719%:%
%:%5511=1719%:%
%:%5515=1719%:%
%:%5516=1719%:%
%:%5523=1719%:%
%:%5524=1720%:%
%:%5525=1721%:%
%:%5526=1721%:%
%:%5528=1721%:%
%:%5532=1721%:%
%:%5533=1721%:%
%:%5540=1721%:%
%:%5541=1722%:%
%:%5542=1723%:%
%:%5543=1723%:%
%:%5545=1725%:%
%:%5546=1726%:%
%:%5547=1727%:%
%:%5548=1728%:%
%:%5549=1729%:%
%:%5550=1730%:%
%:%5551=1731%:%
%:%5552=1732%:%
%:%5553=1733%:%
%:%5554=1734%:%
%:%5555=1735%:%
%:%5557=1736%:%
%:%5558=1736%:%
%:%5563=1741%:%
%:%5564=1742%:%
%:%5565=1742%:%
%:%5568=1743%:%
%:%5572=1743%:%
%:%5573=1743%:%
%:%5578=1743%:%
%:%5581=1744%:%
%:%5582=1745%:%
%:%5583=1745%:%
%:%5585=1745%:%
%:%5589=1745%:%
%:%5590=1745%:%
%:%5597=1745%:%
%:%5598=1746%:%
%:%5599=1747%:%
%:%5600=1747%:%
%:%5602=1747%:%
%:%5606=1747%:%
%:%5607=1747%:%
%:%5614=1747%:%
%:%5615=1748%:%
%:%5616=1749%:%
%:%5617=1749%:%
%:%5619=1749%:%
%:%5623=1749%:%
%:%5624=1749%:%
%:%5633=1751%:%
%:%5634=1752%:%
%:%5635=1753%:%
%:%5636=1754%:%
%:%5637=1755%:%
%:%5638=1756%:%
%:%5639=1757%:%
%:%5640=1758%:%
%:%5641=1759%:%
%:%5642=1760%:%
%:%5643=1761%:%
%:%5645=1762%:%
%:%5646=1762%:%
%:%5651=1767%:%
%:%5652=1768%:%
%:%5653=1768%:%
%:%5656=1769%:%
%:%5660=1769%:%
%:%5661=1769%:%
%:%5666=1769%:%
%:%5669=1770%:%
%:%5670=1771%:%
%:%5671=1771%:%
%:%5673=1771%:%
%:%5677=1771%:%
%:%5678=1771%:%
%:%5685=1771%:%
%:%5686=1772%:%
%:%5687=1773%:%
%:%5688=1773%:%
%:%5690=1773%:%
%:%5694=1773%:%
%:%5695=1773%:%
%:%5702=1773%:%
%:%5703=1774%:%
%:%5704=1775%:%
%:%5705=1775%:%
%:%5707=1775%:%
%:%5711=1775%:%
%:%5712=1775%:%
%:%5719=1775%:%
%:%5720=1776%:%
%:%5721=1777%:%
%:%5722=1777%:%
%:%5724=1779%:%
%:%5725=1780%:%
%:%5726=1781%:%
%:%5727=1782%:%
%:%5728=1783%:%
%:%5729=1784%:%
%:%5730=1785%:%
%:%5731=1786%:%
%:%5732=1787%:%
%:%5733=1788%:%
%:%5734=1789%:%
%:%5736=1790%:%
%:%5737=1790%:%
%:%5742=1795%:%
%:%5743=1796%:%
%:%5744=1796%:%
%:%5747=1797%:%
%:%5751=1797%:%
%:%5752=1797%:%
%:%5757=1797%:%
%:%5760=1798%:%
%:%5761=1799%:%
%:%5762=1799%:%
%:%5764=1799%:%
%:%5768=1799%:%
%:%5769=1799%:%
%:%5776=1799%:%
%:%5777=1800%:%
%:%5778=1801%:%
%:%5779=1801%:%
%:%5781=1801%:%
%:%5785=1801%:%
%:%5786=1801%:%
%:%5793=1801%:%
%:%5794=1802%:%
%:%5795=1803%:%
%:%5796=1803%:%
%:%5798=1803%:%
%:%5802=1803%:%
%:%5803=1803%:%
%:%5812=1805%:%
%:%5813=1806%:%
%:%5814=1807%:%
%:%5815=1808%:%
%:%5816=1809%:%
%:%5817=1810%:%
%:%5818=1811%:%
%:%5819=1812%:%
%:%5820=1813%:%
%:%5821=1814%:%
%:%5822=1815%:%
%:%5824=1816%:%
%:%5825=1816%:%
%:%5830=1821%:%
%:%5831=1822%:%
%:%5832=1822%:%
%:%5835=1823%:%
%:%5839=1823%:%
%:%5840=1823%:%
%:%5845=1823%:%
%:%5848=1824%:%
%:%5849=1825%:%
%:%5850=1825%:%
%:%5852=1825%:%
%:%5856=1825%:%
%:%5857=1825%:%
%:%5864=1825%:%
%:%5865=1826%:%
%:%5866=1827%:%
%:%5867=1827%:%
%:%5869=1827%:%
%:%5873=1827%:%
%:%5874=1827%:%
%:%5881=1827%:%
%:%5882=1828%:%
%:%5883=1829%:%
%:%5884=1829%:%
%:%5886=1829%:%
%:%5890=1829%:%
%:%5891=1829%:%
%:%5898=1829%:%
%:%5899=1830%:%
%:%5900=1831%:%
%:%5901=1831%:%
%:%5903=1833%:%
%:%5904=1834%:%
%:%5905=1835%:%
%:%5906=1836%:%
%:%5907=1837%:%
%:%5908=1838%:%
%:%5909=1839%:%
%:%5910=1840%:%
%:%5911=1841%:%
%:%5912=1842%:%
%:%5913=1843%:%
%:%5915=1844%:%
%:%5916=1844%:%
%:%5921=1849%:%
%:%5922=1850%:%
%:%5923=1850%:%
%:%5926=1851%:%
%:%5930=1851%:%
%:%5931=1851%:%
%:%5936=1851%:%
%:%5939=1852%:%
%:%5940=1853%:%
%:%5941=1853%:%
%:%5943=1853%:%
%:%5947=1853%:%
%:%5948=1853%:%
%:%5955=1853%:%
%:%5956=1854%:%
%:%5957=1855%:%
%:%5958=1855%:%
%:%5960=1855%:%
%:%5964=1855%:%
%:%5965=1855%:%
%:%5972=1855%:%
%:%5973=1856%:%
%:%5974=1857%:%
%:%5975=1857%:%
%:%5977=1857%:%
%:%5981=1857%:%
%:%5982=1857%:%
%:%5991=1859%:%
%:%5992=1860%:%
%:%5993=1861%:%
%:%5994=1862%:%
%:%5995=1863%:%
%:%5996=1864%:%
%:%5997=1865%:%
%:%5998=1866%:%
%:%5999=1867%:%
%:%6000=1868%:%
%:%6001=1869%:%
%:%6003=1870%:%
%:%6004=1870%:%
%:%6009=1875%:%
%:%6010=1876%:%
%:%6011=1876%:%
%:%6014=1877%:%
%:%6018=1877%:%
%:%6019=1877%:%
%:%6024=1877%:%
%:%6027=1878%:%
%:%6028=1879%:%
%:%6029=1879%:%
%:%6031=1879%:%
%:%6035=1879%:%
%:%6036=1879%:%
%:%6043=1879%:%
%:%6044=1880%:%
%:%6045=1881%:%
%:%6046=1881%:%
%:%6048=1881%:%
%:%6052=1881%:%
%:%6053=1881%:%
%:%6060=1881%:%
%:%6061=1882%:%
%:%6062=1883%:%
%:%6063=1883%:%
%:%6065=1883%:%
%:%6069=1883%:%
%:%6070=1883%:%
%:%6077=1883%:%
%:%6078=1884%:%
%:%6079=1885%:%
%:%6080=1885%:%
%:%6082=1888%:%
%:%6083=1889%:%
%:%6084=1890%:%
%:%6085=1891%:%
%:%6086=1892%:%
%:%6087=1893%:%
%:%6088=1894%:%
%:%6089=1895%:%
%:%6090=1896%:%
%:%6091=1897%:%
%:%6093=1898%:%
%:%6094=1898%:%
%:%6099=1903%:%
%:%6100=1904%:%
%:%6101=1904%:%
%:%6104=1905%:%
%:%6108=1905%:%
%:%6109=1905%:%
%:%6114=1905%:%
%:%6117=1906%:%
%:%6118=1907%:%
%:%6119=1907%:%
%:%6121=1907%:%
%:%6125=1907%:%
%:%6126=1907%:%
%:%6133=1907%:%
%:%6134=1908%:%
%:%6135=1909%:%
%:%6136=1909%:%
%:%6138=1909%:%
%:%6142=1909%:%
%:%6143=1909%:%
%:%6150=1909%:%
%:%6151=1910%:%
%:%6152=1911%:%
%:%6153=1911%:%
%:%6155=1911%:%
%:%6159=1911%:%
%:%6160=1911%:%
%:%6169=1914%:%
%:%6170=1915%:%
%:%6171=1916%:%
%:%6172=1917%:%
%:%6173=1918%:%
%:%6174=1919%:%
%:%6175=1920%:%
%:%6176=1921%:%
%:%6177=1922%:%
%:%6178=1923%:%
%:%6180=1924%:%
%:%6181=1924%:%
%:%6186=1929%:%
%:%6187=1930%:%
%:%6188=1930%:%
%:%6191=1931%:%
%:%6195=1931%:%
%:%6196=1931%:%
%:%6201=1931%:%
%:%6204=1932%:%
%:%6205=1933%:%
%:%6206=1933%:%
%:%6208=1933%:%
%:%6212=1933%:%
%:%6213=1933%:%
%:%6220=1933%:%
%:%6221=1934%:%
%:%6222=1935%:%
%:%6223=1935%:%
%:%6225=1935%:%
%:%6229=1935%:%
%:%6230=1935%:%
%:%6237=1935%:%
%:%6238=1936%:%
%:%6239=1937%:%
%:%6240=1937%:%
%:%6242=1937%:%
%:%6246=1937%:%
%:%6247=1937%:%
%:%6254=1937%:%
%:%6255=1938%:%
%:%6256=1939%:%
%:%6257=1939%:%
%:%6258=1940%:%
%:%6259=1941%:%
%:%6260=1942%:%
%:%6261=1943%:%
%:%6262=1944%:%
%:%6263=1945%:%
%:%6264=1946%:%
%:%6265=1946%:%
%:%6266=1947%:%
%:%6267=1948%:%
%:%6268=1949%:%
%:%6269=1950%:%
%:%6270=1951%:%
%:%6272=1953%:%
%:%6273=1954%:%
%:%6275=1955%:%
%:%6276=1955%:%
%:%6277=1956%:%
%:%6278=1957%:%
%:%6281=1958%:%
%:%6285=1958%:%
%:%6286=1958%:%
%:%6287=1959%:%
%:%6288=1959%:%
%:%6289=1960%:%
%:%6290=1960%:%
%:%6291=1961%:%
%:%6292=1961%:%
%:%6293=1961%:%
%:%6294=1961%:%
%:%6295=1962%:%
%:%6296=1962%:%
%:%6297=1962%:%
%:%6298=1963%:%
%:%6299=1963%:%
%:%6300=1964%:%
%:%6301=1964%:%
%:%6302=1965%:%
%:%6303=1965%:%
%:%6304=1965%:%
%:%6305=1965%:%
%:%6306=1965%:%
%:%6307=1966%:%
%:%6308=1966%:%
%:%6309=1967%:%
%:%6310=1967%:%
%:%6311=1968%:%
%:%6312=1968%:%
%:%6313=1968%:%
%:%6314=1969%:%
%:%6315=1970%:%
%:%6316=1970%:%
%:%6317=1970%:%
%:%6318=1971%:%
%:%6319=1972%:%
%:%6320=1972%:%
%:%6321=1973%:%
%:%6322=1973%:%
%:%6323=1974%:%
%:%6324=1974%:%
%:%6325=1975%:%
%:%6326=1975%:%
%:%6327=1975%:%
%:%6328=1976%:%
%:%6329=1976%:%
%:%6330=1976%:%
%:%6331=1977%:%
%:%6332=1977%:%
%:%6333=1977%:%
%:%6334=1978%:%
%:%6335=1979%:%
%:%6336=1979%:%
%:%6337=1979%:%
%:%6338=1979%:%
%:%6339=1980%:%
%:%6340=1981%:%
%:%6341=1981%:%
%:%6342=1982%:%
%:%6343=1982%:%
%:%6344=1982%:%
%:%6345=1983%:%
%:%6346=1983%:%
%:%6347=1983%:%
%:%6348=1984%:%
%:%6349=1984%:%
%:%6350=1984%:%
%:%6351=1985%:%
%:%6352=1986%:%
%:%6353=1986%:%
%:%6354=1986%:%
%:%6355=1986%:%
%:%6356=1987%:%
%:%6357=1987%:%
%:%6358=1988%:%
%:%6368=1990%:%
%:%6369=1991%:%
%:%6371=1992%:%
%:%6372=1992%:%
%:%6373=1993%:%
%:%6374=1994%:%
%:%6377=1995%:%
%:%6381=1995%:%
%:%6382=1995%:%
%:%6383=1996%:%
%:%6384=1996%:%
%:%6385=1997%:%
%:%6386=1997%:%
%:%6387=1997%:%
%:%6388=1997%:%
%:%6389=1998%:%
%:%6390=1998%:%
%:%6391=1998%:%
%:%6392=1999%:%
%:%6393=1999%:%
%:%6394=2000%:%
%:%6395=2000%:%
%:%6396=2001%:%
%:%6397=2001%:%
%:%6398=2001%:%
%:%6399=2001%:%
%:%6400=2001%:%
%:%6401=2002%:%
%:%6402=2002%:%
%:%6403=2003%:%
%:%6404=2003%:%
%:%6405=2004%:%
%:%6406=2004%:%
%:%6407=2004%:%
%:%6408=2005%:%
%:%6409=2006%:%
%:%6410=2006%:%
%:%6411=2006%:%
%:%6412=2007%:%
%:%6413=2007%:%
%:%6414=2008%:%
%:%6415=2008%:%
%:%6416=2009%:%
%:%6417=2009%:%
%:%6418=2010%:%
%:%6419=2011%:%
%:%6420=2011%:%
%:%6421=2011%:%
%:%6422=2012%:%
%:%6423=2012%:%
%:%6424=2012%:%
%:%6425=2013%:%
%:%6426=2013%:%
%:%6427=2013%:%
%:%6428=2014%:%
%:%6429=2015%:%
%:%6430=2015%:%
%:%6431=2015%:%
%:%6432=2015%:%
%:%6433=2016%:%
%:%6434=2017%:%
%:%6435=2017%:%
%:%6436=2018%:%
%:%6437=2018%:%
%:%6438=2018%:%
%:%6439=2019%:%
%:%6440=2019%:%
%:%6441=2019%:%
%:%6442=2020%:%
%:%6443=2020%:%
%:%6444=2020%:%
%:%6445=2021%:%
%:%6446=2022%:%
%:%6447=2022%:%
%:%6448=2022%:%
%:%6449=2022%:%
%:%6450=2023%:%
%:%6451=2023%:%
%:%6452=2024%:%
%:%6462=2026%:%
%:%6463=2027%:%
%:%6465=2028%:%
%:%6466=2028%:%
%:%6473=2030%:%
%:%6483=2032%:%
%:%6484=2032%:%
%:%6486=2035%:%
%:%6487=2036%:%
%:%6488=2037%:%
%:%6489=2038%:%
%:%6490=2039%:%
%:%6491=2040%:%
%:%6492=2041%:%
%:%6493=2042%:%
%:%6494=2043%:%
%:%6495=2044%:%
%:%6496=2045%:%
%:%6498=2046%:%
%:%6499=2046%:%
%:%6505=2052%:%
%:%6506=2053%:%
%:%6507=2053%:%
%:%6510=2054%:%
%:%6514=2054%:%
%:%6515=2054%:%
%:%6520=2054%:%
%:%6523=2055%:%
%:%6524=2056%:%
%:%6525=2056%:%
%:%6527=2056%:%
%:%6531=2056%:%
%:%6532=2056%:%
%:%6539=2056%:%
%:%6540=2057%:%
%:%6541=2058%:%
%:%6542=2058%:%
%:%6544=2058%:%
%:%6548=2058%:%
%:%6549=2058%:%
%:%6556=2058%:%
%:%6557=2059%:%
%:%6558=2060%:%
%:%6559=2060%:%
%:%6561=2060%:%
%:%6565=2060%:%
%:%6566=2060%:%
%:%6575=2062%:%
%:%6576=2063%:%
%:%6577=2064%:%
%:%6578=2065%:%
%:%6579=2066%:%
%:%6580=2067%:%
%:%6581=2068%:%
%:%6582=2069%:%
%:%6583=2070%:%
%:%6584=2071%:%
%:%6585=2072%:%
%:%6587=2073%:%
%:%6588=2073%:%
%:%6594=2079%:%
%:%6595=2080%:%
%:%6596=2080%:%
%:%6599=2081%:%
%:%6603=2081%:%
%:%6604=2081%:%
%:%6609=2081%:%
%:%6612=2082%:%
%:%6613=2083%:%
%:%6614=2083%:%
%:%6616=2083%:%
%:%6620=2083%:%
%:%6621=2083%:%
%:%6628=2083%:%
%:%6629=2084%:%
%:%6630=2085%:%
%:%6631=2085%:%
%:%6633=2085%:%
%:%6637=2085%:%
%:%6638=2085%:%
%:%6645=2085%:%
%:%6646=2086%:%
%:%6647=2087%:%
%:%6648=2087%:%
%:%6650=2089%:%
%:%6651=2090%:%
%:%6652=2091%:%
%:%6653=2092%:%
%:%6654=2093%:%
%:%6655=2094%:%
%:%6656=2095%:%
%:%6657=2096%:%
%:%6658=2097%:%
%:%6659=2098%:%
%:%6660=2099%:%
%:%6661=2100%:%
%:%6663=2101%:%
%:%6664=2101%:%
%:%6670=2107%:%
%:%6671=2108%:%
%:%6672=2109%:%
%:%6673=2109%:%
%:%6676=2110%:%
%:%6680=2110%:%
%:%6681=2110%:%
%:%6690=2112%:%
%:%6691=2113%:%
%:%6692=2114%:%
%:%6693=2115%:%
%:%6694=2116%:%
%:%6695=2117%:%
%:%6696=2118%:%
%:%6697=2119%:%
%:%6698=2120%:%
%:%6699=2121%:%
%:%6700=2122%:%
%:%6701=2123%:%
%:%6703=2124%:%
%:%6704=2124%:%
%:%6710=2130%:%
%:%6711=2131%:%
%:%6712=2131%:%
%:%6715=2132%:%
%:%6719=2132%:%
%:%6720=2132%:%
%:%6725=2132%:%
%:%6728=2133%:%
%:%6729=2134%:%
%:%6730=2134%:%
%:%6732=2134%:%
%:%6736=2134%:%
%:%6737=2134%:%
%:%6744=2134%:%
%:%6745=2135%:%
%:%6746=2136%:%
%:%6747=2136%:%
%:%6749=2136%:%
%:%6753=2136%:%
%:%6754=2136%:%
%:%6761=2136%:%
%:%6762=2137%:%
%:%6763=2138%:%
%:%6764=2138%:%
%:%6766=2138%:%
%:%6770=2138%:%
%:%6771=2138%:%
%:%6780=2140%:%
%:%6781=2141%:%
%:%6782=2142%:%
%:%6783=2143%:%
%:%6784=2144%:%
%:%6785=2145%:%
%:%6786=2146%:%
%:%6787=2147%:%
%:%6788=2148%:%
%:%6789=2149%:%
%:%6790=2150%:%
%:%6792=2151%:%
%:%6793=2151%:%
%:%6799=2157%:%
%:%6800=2158%:%
%:%6801=2158%:%
%:%6804=2159%:%
%:%6808=2159%:%
%:%6809=2159%:%
%:%6814=2159%:%
%:%6817=2160%:%
%:%6818=2161%:%
%:%6819=2161%:%
%:%6821=2161%:%
%:%6825=2161%:%
%:%6826=2161%:%
%:%6833=2161%:%
%:%6834=2162%:%
%:%6835=2163%:%
%:%6836=2163%:%
%:%6838=2163%:%
%:%6842=2163%:%
%:%6843=2163%:%
%:%6850=2163%:%
%:%6851=2164%:%
%:%6852=2165%:%
%:%6853=2165%:%
%:%6855=2167%:%
%:%6856=2168%:%
%:%6857=2169%:%
%:%6858=2170%:%
%:%6859=2171%:%
%:%6860=2172%:%
%:%6861=2173%:%
%:%6862=2174%:%
%:%6863=2175%:%
%:%6864=2176%:%
%:%6865=2177%:%
%:%6866=2178%:%
%:%6868=2179%:%
%:%6869=2179%:%
%:%6875=2185%:%
%:%6876=2186%:%
%:%6877=2186%:%
%:%6880=2187%:%
%:%6884=2187%:%
%:%6885=2187%:%
%:%6890=2187%:%
%:%6893=2188%:%
%:%6894=2189%:%
%:%6895=2189%:%
%:%6897=2189%:%
%:%6901=2189%:%
%:%6902=2189%:%
%:%6909=2189%:%
%:%6910=2190%:%
%:%6911=2191%:%
%:%6912=2191%:%
%:%6914=2191%:%
%:%6918=2191%:%
%:%6919=2191%:%
%:%6926=2191%:%
%:%6927=2192%:%
%:%6928=2193%:%
%:%6929=2193%:%
%:%6931=2193%:%
%:%6935=2193%:%
%:%6936=2193%:%
%:%6945=2195%:%
%:%6946=2196%:%
%:%6947=2197%:%
%:%6948=2198%:%
%:%6949=2199%:%
%:%6950=2200%:%
%:%6951=2201%:%
%:%6952=2202%:%
%:%6953=2203%:%
%:%6954=2204%:%
%:%6955=2205%:%
%:%6957=2206%:%
%:%6958=2206%:%
%:%6964=2212%:%
%:%6965=2213%:%
%:%6966=2213%:%
%:%6969=2214%:%
%:%6973=2214%:%
%:%6974=2214%:%
%:%6979=2214%:%
%:%6982=2215%:%
%:%6983=2216%:%
%:%6984=2216%:%
%:%6986=2216%:%
%:%6990=2216%:%
%:%6991=2216%:%
%:%6998=2216%:%
%:%6999=2217%:%
%:%7000=2218%:%
%:%7001=2218%:%
%:%7003=2218%:%
%:%7007=2218%:%
%:%7008=2218%:%
%:%7015=2218%:%
%:%7016=2219%:%
%:%7017=2220%:%
%:%7018=2220%:%
%:%7020=2222%:%
%:%7021=2223%:%
%:%7022=2224%:%
%:%7023=2225%:%
%:%7024=2226%:%
%:%7025=2227%:%
%:%7026=2228%:%
%:%7027=2229%:%
%:%7028=2230%:%
%:%7029=2231%:%
%:%7030=2232%:%
%:%7031=2233%:%
%:%7033=2234%:%
%:%7034=2234%:%
%:%7040=2240%:%
%:%7041=2241%:%
%:%7042=2241%:%
%:%7045=2242%:%
%:%7049=2242%:%
%:%7050=2242%:%
%:%7055=2242%:%
%:%7058=2243%:%
%:%7059=2244%:%
%:%7060=2244%:%
%:%7062=2244%:%
%:%7066=2244%:%
%:%7067=2244%:%
%:%7074=2244%:%
%:%7075=2245%:%
%:%7076=2246%:%
%:%7077=2246%:%
%:%7079=2246%:%
%:%7083=2246%:%
%:%7084=2246%:%
%:%7091=2246%:%
%:%7092=2247%:%
%:%7093=2248%:%
%:%7094=2248%:%
%:%7096=2248%:%
%:%7100=2248%:%
%:%7101=2248%:%
%:%7110=2250%:%
%:%7111=2251%:%
%:%7112=2252%:%
%:%7113=2253%:%
%:%7114=2254%:%
%:%7115=2255%:%
%:%7116=2256%:%
%:%7117=2257%:%
%:%7118=2258%:%
%:%7119=2259%:%
%:%7120=2260%:%
%:%7122=2261%:%
%:%7123=2261%:%
%:%7129=2267%:%
%:%7130=2268%:%
%:%7131=2268%:%
%:%7134=2269%:%
%:%7138=2269%:%
%:%7139=2269%:%
%:%7144=2269%:%
%:%7147=2270%:%
%:%7148=2271%:%
%:%7149=2271%:%
%:%7151=2271%:%
%:%7155=2271%:%
%:%7156=2271%:%
%:%7163=2271%:%
%:%7164=2272%:%
%:%7165=2273%:%
%:%7166=2273%:%
%:%7168=2273%:%
%:%7172=2273%:%
%:%7173=2273%:%
%:%7180=2273%:%
%:%7181=2274%:%
%:%7182=2275%:%
%:%7183=2275%:%
%:%7185=2277%:%
%:%7186=2278%:%
%:%7187=2279%:%
%:%7188=2280%:%
%:%7189=2281%:%
%:%7190=2282%:%
%:%7191=2283%:%
%:%7192=2284%:%
%:%7193=2285%:%
%:%7194=2286%:%
%:%7195=2287%:%
%:%7196=2288%:%
%:%7198=2289%:%
%:%7199=2289%:%
%:%7205=2295%:%
%:%7206=2296%:%
%:%7207=2296%:%
%:%7210=2297%:%
%:%7214=2297%:%
%:%7215=2297%:%
%:%7220=2297%:%
%:%7223=2298%:%
%:%7224=2299%:%
%:%7225=2299%:%
%:%7227=2299%:%
%:%7231=2299%:%
%:%7232=2299%:%
%:%7239=2299%:%
%:%7240=2300%:%
%:%7241=2301%:%
%:%7242=2301%:%
%:%7244=2301%:%
%:%7248=2301%:%
%:%7249=2301%:%
%:%7256=2301%:%
%:%7257=2302%:%
%:%7258=2303%:%
%:%7259=2303%:%
%:%7261=2303%:%
%:%7265=2303%:%
%:%7266=2303%:%
%:%7275=2305%:%
%:%7276=2306%:%
%:%7277=2307%:%
%:%7278=2308%:%
%:%7279=2309%:%
%:%7280=2310%:%
%:%7281=2311%:%
%:%7282=2312%:%
%:%7283=2313%:%
%:%7284=2314%:%
%:%7285=2315%:%
%:%7287=2316%:%
%:%7288=2316%:%
%:%7294=2322%:%
%:%7295=2323%:%
%:%7296=2323%:%
%:%7299=2324%:%
%:%7303=2324%:%
%:%7304=2324%:%
%:%7309=2324%:%
%:%7312=2325%:%
%:%7313=2326%:%
%:%7314=2326%:%
%:%7316=2326%:%
%:%7320=2326%:%
%:%7321=2326%:%
%:%7328=2326%:%
%:%7329=2327%:%
%:%7330=2328%:%
%:%7331=2328%:%
%:%7333=2328%:%
%:%7337=2328%:%
%:%7338=2328%:%
%:%7345=2328%:%
%:%7346=2329%:%
%:%7347=2330%:%
%:%7348=2330%:%
%:%7349=2331%:%
%:%7350=2332%:%
%:%7351=2333%:%
%:%7352=2334%:%
%:%7353=2335%:%
%:%7354=2336%:%
%:%7355=2337%:%
%:%7356=2337%:%
%:%7357=2338%:%
%:%7358=2339%:%
%:%7359=2340%:%
%:%7360=2341%:%
%:%7361=2342%:%
%:%7363=2344%:%
%:%7364=2345%:%
%:%7366=2346%:%
%:%7367=2346%:%
%:%7368=2347%:%
%:%7369=2348%:%
%:%7372=2349%:%
%:%7376=2349%:%
%:%7377=2349%:%
%:%7378=2350%:%
%:%7379=2350%:%
%:%7380=2351%:%
%:%7381=2351%:%
%:%7382=2352%:%
%:%7383=2352%:%
%:%7384=2352%:%
%:%7385=2352%:%
%:%7386=2353%:%
%:%7387=2353%:%
%:%7388=2353%:%
%:%7389=2354%:%
%:%7390=2354%:%
%:%7391=2355%:%
%:%7392=2355%:%
%:%7393=2356%:%
%:%7394=2356%:%
%:%7395=2356%:%
%:%7396=2356%:%
%:%7397=2356%:%
%:%7398=2357%:%
%:%7399=2357%:%
%:%7400=2358%:%
%:%7401=2358%:%
%:%7402=2359%:%
%:%7403=2359%:%
%:%7404=2360%:%
%:%7405=2360%:%
%:%7406=2361%:%
%:%7407=2361%:%
%:%7408=2361%:%
%:%7409=2362%:%
%:%7410=2362%:%
%:%7411=2363%:%
%:%7412=2363%:%
%:%7413=2363%:%
%:%7414=2364%:%
%:%7415=2365%:%
%:%7416=2365%:%
%:%7417=2365%:%
%:%7418=2366%:%
%:%7419=2367%:%
%:%7420=2367%:%
%:%7421=2367%:%
%:%7422=2368%:%
%:%7423=2368%:%
%:%7424=2368%:%
%:%7425=2369%:%
%:%7426=2369%:%
%:%7427=2369%:%
%:%7428=2369%:%
%:%7429=2370%:%
%:%7430=2370%:%
%:%7431=2370%:%
%:%7432=2371%:%
%:%7433=2372%:%
%:%7434=2373%:%
%:%7435=2373%:%
%:%7436=2373%:%
%:%7437=2374%:%
%:%7438=2374%:%
%:%7439=2374%:%
%:%7440=2375%:%
%:%7441=2375%:%
%:%7442=2375%:%
%:%7443=2376%:%
%:%7444=2376%:%
%:%7445=2377%:%
%:%7455=2379%:%
%:%7457=2380%:%
%:%7458=2380%:%
%:%7459=2381%:%
%:%7460=2382%:%
%:%7463=2383%:%
%:%7467=2383%:%
%:%7468=2383%:%
%:%7469=2384%:%
%:%7470=2384%:%
%:%7471=2385%:%
%:%7472=2385%:%
%:%7473=2385%:%
%:%7474=2385%:%
%:%7475=2386%:%
%:%7476=2386%:%
%:%7477=2386%:%
%:%7478=2387%:%
%:%7479=2387%:%
%:%7480=2388%:%
%:%7481=2388%:%
%:%7482=2389%:%
%:%7483=2389%:%
%:%7484=2389%:%
%:%7485=2389%:%
%:%7486=2389%:%
%:%7487=2390%:%
%:%7488=2390%:%
%:%7489=2391%:%
%:%7490=2391%:%
%:%7491=2392%:%
%:%7492=2392%:%
%:%7493=2393%:%
%:%7494=2393%:%
%:%7495=2393%:%
%:%7496=2394%:%
%:%7497=2394%:%
%:%7498=2394%:%
%:%7499=2395%:%
%:%7500=2395%:%
%:%7501=2395%:%
%:%7502=2395%:%
%:%7503=2396%:%
%:%7504=2396%:%
%:%7505=2397%:%
%:%7506=2397%:%
%:%7507=2397%:%
%:%7508=2398%:%
%:%7509=2399%:%
%:%7510=2399%:%
%:%7511=2399%:%
%:%7512=2400%:%
%:%7513=2401%:%
%:%7514=2401%:%
%:%7515=2401%:%
%:%7516=2402%:%
%:%7517=2402%:%
%:%7518=2402%:%
%:%7519=2403%:%
%:%7520=2403%:%
%:%7521=2403%:%
%:%7522=2403%:%
%:%7523=2404%:%
%:%7524=2404%:%
%:%7525=2404%:%
%:%7526=2405%:%
%:%7527=2406%:%
%:%7528=2407%:%
%:%7529=2407%:%
%:%7530=2407%:%
%:%7531=2408%:%
%:%7532=2408%:%
%:%7533=2408%:%
%:%7534=2409%:%
%:%7535=2409%:%
%:%7536=2409%:%
%:%7537=2410%:%
%:%7538=2410%:%
%:%7539=2410%:%
%:%7540=2410%:%
%:%7541=2410%:%
%:%7542=2411%:%
%:%7543=2411%:%
%:%7544=2412%:%
%:%7554=2414%:%
%:%7555=2415%:%
%:%7557=2416%:%
%:%7558=2416%:%
%:%7565=2418%:%
%:%7575=2420%:%
%:%7576=2420%:%
%:%7578=2422%:%
%:%7579=2423%:%
%:%7580=2424%:%
%:%7581=2425%:%
%:%7582=2426%:%
%:%7583=2427%:%
%:%7584=2428%:%
%:%7585=2429%:%
%:%7586=2430%:%
%:%7587=2431%:%
%:%7588=2432%:%
%:%7589=2433%:%
%:%7590=2434%:%
%:%7591=2435%:%
%:%7593=2436%:%
%:%7594=2436%:%
%:%7602=2444%:%
%:%7603=2445%:%
%:%7604=2445%:%
%:%7607=2446%:%
%:%7611=2446%:%
%:%7612=2446%:%
%:%7617=2446%:%
%:%7620=2447%:%
%:%7621=2448%:%
%:%7622=2448%:%
%:%7624=2448%:%
%:%7628=2448%:%
%:%7629=2448%:%
%:%7636=2448%:%
%:%7637=2449%:%
%:%7638=2450%:%
%:%7639=2450%:%
%:%7641=2450%:%
%:%7645=2450%:%
%:%7646=2450%:%
%:%7653=2450%:%
%:%7654=2451%:%
%:%7655=2452%:%
%:%7656=2452%:%
%:%7658=2452%:%
%:%7662=2452%:%
%:%7663=2452%:%
%:%7672=2454%:%
%:%7673=2455%:%
%:%7674=2456%:%
%:%7675=2457%:%
%:%7676=2458%:%
%:%7677=2459%:%
%:%7678=2460%:%
%:%7679=2461%:%
%:%7680=2462%:%
%:%7681=2463%:%
%:%7682=2464%:%
%:%7683=2465%:%
%:%7684=2466%:%
%:%7686=2467%:%
%:%7687=2467%:%
%:%7695=2475%:%
%:%7696=2476%:%
%:%7697=2476%:%
%:%7700=2477%:%
%:%7704=2477%:%
%:%7705=2477%:%
%:%7710=2477%:%
%:%7713=2478%:%
%:%7714=2479%:%
%:%7715=2479%:%
%:%7717=2479%:%
%:%7721=2479%:%
%:%7722=2479%:%
%:%7729=2479%:%
%:%7730=2480%:%
%:%7731=2481%:%
%:%7732=2481%:%
%:%7734=2481%:%
%:%7738=2481%:%
%:%7739=2481%:%
%:%7746=2481%:%
%:%7747=2482%:%
%:%7748=2483%:%
%:%7749=2483%:%
%:%7751=2483%:%
%:%7755=2483%:%
%:%7756=2483%:%
%:%7763=2483%:%
%:%7764=2484%:%
%:%7765=2485%:%
%:%7766=2485%:%
%:%7773=2486%:%
%:%7774=2486%:%
%:%7775=2487%:%
%:%7776=2487%:%
%:%7777=2488%:%
%:%7778=2488%:%
%:%7779=2489%:%
%:%7780=2489%:%
%:%7781=2489%:%
%:%7782=2489%:%
%:%7783=2489%:%
%:%7784=2490%:%
%:%7790=2490%:%
%:%7793=2491%:%
%:%7794=2492%:%
%:%7795=2492%:%
%:%7797=2494%:%
%:%7798=2495%:%
%:%7799=2496%:%
%:%7800=2497%:%
%:%7801=2498%:%
%:%7802=2499%:%
%:%7803=2500%:%
%:%7804=2501%:%
%:%7805=2502%:%
%:%7806=2503%:%
%:%7807=2504%:%
%:%7808=2505%:%
%:%7809=2506%:%
%:%7810=2507%:%
%:%7812=2508%:%
%:%7813=2508%:%
%:%7821=2516%:%
%:%7822=2517%:%
%:%7823=2517%:%
%:%7826=2518%:%
%:%7830=2518%:%
%:%7831=2518%:%
%:%7836=2518%:%
%:%7839=2519%:%
%:%7840=2520%:%
%:%7841=2520%:%
%:%7843=2520%:%
%:%7847=2520%:%
%:%7848=2520%:%
%:%7855=2520%:%
%:%7856=2521%:%
%:%7857=2522%:%
%:%7858=2522%:%
%:%7860=2522%:%
%:%7864=2522%:%
%:%7865=2522%:%
%:%7872=2522%:%
%:%7873=2523%:%
%:%7874=2524%:%
%:%7875=2524%:%
%:%7877=2524%:%
%:%7881=2524%:%
%:%7882=2524%:%
%:%7891=2526%:%
%:%7892=2527%:%
%:%7893=2528%:%
%:%7894=2529%:%
%:%7895=2530%:%
%:%7896=2531%:%
%:%7897=2532%:%
%:%7898=2533%:%
%:%7899=2534%:%
%:%7900=2535%:%
%:%7901=2536%:%
%:%7902=2537%:%
%:%7903=2538%:%
%:%7904=2539%:%
%:%7906=2540%:%
%:%7907=2540%:%
%:%7915=2548%:%
%:%7916=2549%:%
%:%7917=2549%:%
%:%7920=2550%:%
%:%7924=2550%:%
%:%7925=2550%:%
%:%7930=2550%:%
%:%7933=2551%:%
%:%7934=2552%:%
%:%7935=2552%:%
%:%7937=2552%:%
%:%7941=2552%:%
%:%7942=2552%:%
%:%7949=2552%:%
%:%7950=2553%:%
%:%7951=2554%:%
%:%7952=2554%:%
%:%7954=2554%:%
%:%7958=2554%:%
%:%7959=2554%:%
%:%7966=2554%:%
%:%7967=2555%:%
%:%7968=2556%:%
%:%7969=2556%:%
%:%7976=2557%:%
%:%7977=2557%:%
%:%7978=2558%:%
%:%7979=2558%:%
%:%7980=2559%:%
%:%7981=2559%:%
%:%7982=2560%:%
%:%7983=2560%:%
%:%7984=2560%:%
%:%7985=2560%:%
%:%7986=2560%:%
%:%7987=2561%:%
%:%7993=2561%:%
%:%7996=2562%:%
%:%7997=2563%:%
%:%7998=2563%:%
%:%8000=2565%:%
%:%8001=2566%:%
%:%8002=2567%:%
%:%8003=2568%:%
%:%8004=2569%:%
%:%8005=2570%:%
%:%8006=2571%:%
%:%8007=2572%:%
%:%8008=2573%:%
%:%8009=2574%:%
%:%8010=2575%:%
%:%8011=2576%:%
%:%8012=2577%:%
%:%8013=2578%:%
%:%8015=2579%:%
%:%8016=2579%:%
%:%8024=2587%:%
%:%8025=2588%:%
%:%8026=2588%:%
%:%8029=2589%:%
%:%8033=2589%:%
%:%8034=2589%:%
%:%8039=2589%:%
%:%8042=2590%:%
%:%8043=2591%:%
%:%8044=2591%:%
%:%8046=2591%:%
%:%8050=2591%:%
%:%8051=2591%:%
%:%8058=2591%:%
%:%8059=2592%:%
%:%8060=2593%:%
%:%8061=2593%:%
%:%8063=2593%:%
%:%8067=2593%:%
%:%8068=2593%:%
%:%8075=2593%:%
%:%8076=2594%:%
%:%8077=2595%:%
%:%8078=2595%:%
%:%8080=2595%:%
%:%8084=2595%:%
%:%8085=2595%:%
%:%8094=2597%:%
%:%8095=2598%:%
%:%8096=2599%:%
%:%8097=2600%:%
%:%8098=2601%:%
%:%8099=2602%:%
%:%8100=2603%:%
%:%8101=2604%:%
%:%8102=2605%:%
%:%8103=2606%:%
%:%8104=2607%:%
%:%8105=2608%:%
%:%8106=2609%:%
%:%8107=2610%:%
%:%8109=2611%:%
%:%8110=2611%:%
%:%8118=2619%:%
%:%8119=2620%:%
%:%8120=2620%:%
%:%8123=2621%:%
%:%8127=2621%:%
%:%8128=2621%:%
%:%8133=2621%:%
%:%8136=2622%:%
%:%8137=2623%:%
%:%8138=2623%:%
%:%8140=2623%:%
%:%8144=2623%:%
%:%8145=2623%:%
%:%8152=2623%:%
%:%8153=2624%:%
%:%8154=2625%:%
%:%8155=2625%:%
%:%8157=2625%:%
%:%8161=2625%:%
%:%8162=2625%:%
%:%8169=2625%:%
%:%8170=2626%:%
%:%8171=2627%:%
%:%8172=2627%:%
%:%8179=2628%:%
%:%8180=2628%:%
%:%8181=2629%:%
%:%8182=2629%:%
%:%8183=2630%:%
%:%8184=2630%:%
%:%8185=2631%:%
%:%8186=2631%:%
%:%8187=2631%:%
%:%8188=2631%:%
%:%8189=2631%:%
%:%8190=2632%:%
%:%8196=2632%:%
%:%8199=2633%:%
%:%8200=2634%:%
%:%8201=2634%:%
%:%8203=2636%:%
%:%8204=2637%:%
%:%8205=2638%:%
%:%8206=2639%:%
%:%8207=2640%:%
%:%8208=2641%:%
%:%8209=2642%:%
%:%8210=2643%:%
%:%8211=2644%:%
%:%8212=2645%:%
%:%8213=2646%:%
%:%8214=2647%:%
%:%8215=2648%:%
%:%8216=2649%:%
%:%8218=2650%:%
%:%8219=2650%:%
%:%8227=2658%:%
%:%8228=2659%:%
%:%8229=2660%:%
%:%8230=2660%:%
%:%8233=2661%:%
%:%8237=2661%:%
%:%8238=2661%:%
%:%8247=2663%:%
%:%8248=2664%:%
%:%8249=2665%:%
%:%8250=2666%:%
%:%8251=2667%:%
%:%8252=2668%:%
%:%8253=2669%:%
%:%8254=2670%:%
%:%8255=2671%:%
%:%8256=2672%:%
%:%8257=2673%:%
%:%8258=2674%:%
%:%8259=2675%:%
%:%8261=2676%:%
%:%8262=2676%:%
%:%8270=2684%:%
%:%8271=2685%:%
%:%8272=2686%:%
%:%8273=2686%:%
%:%8276=2687%:%
%:%8280=2687%:%
%:%8281=2687%:%
%:%8286=2687%:%
%:%8289=2688%:%
%:%8290=2689%:%
%:%8291=2689%:%
%:%8293=2689%:%
%:%8297=2689%:%
%:%8298=2689%:%
%:%8305=2689%:%
%:%8306=2690%:%
%:%8307=2691%:%
%:%8308=2691%:%
%:%8310=2691%:%
%:%8314=2691%:%
%:%8315=2691%:%
%:%8322=2691%:%
%:%8323=2692%:%
%:%8324=2693%:%
%:%8325=2693%:%
%:%8327=2693%:%
%:%8331=2693%:%
%:%8332=2693%:%
%:%8341=2695%:%
%:%8342=2696%:%
%:%8343=2697%:%
%:%8344=2698%:%
%:%8345=2699%:%
%:%8346=2700%:%
%:%8347=2701%:%
%:%8348=2702%:%
%:%8349=2703%:%
%:%8350=2704%:%
%:%8351=2705%:%
%:%8352=2706%:%
%:%8353=2707%:%
%:%8355=2708%:%
%:%8356=2708%:%
%:%8364=2716%:%
%:%8365=2717%:%
%:%8366=2717%:%
%:%8369=2718%:%
%:%8373=2718%:%
%:%8374=2718%:%
%:%8379=2718%:%
%:%8382=2719%:%
%:%8383=2720%:%
%:%8384=2720%:%
%:%8386=2720%:%
%:%8390=2720%:%
%:%8391=2720%:%
%:%8398=2720%:%
%:%8399=2721%:%
%:%8400=2722%:%
%:%8401=2722%:%
%:%8403=2722%:%
%:%8407=2722%:%
%:%8408=2722%:%
%:%8415=2722%:%
%:%8416=2723%:%
%:%8417=2724%:%
%:%8418=2724%:%
%:%8425=2725%:%
%:%8426=2725%:%
%:%8427=2726%:%
%:%8428=2726%:%
%:%8429=2727%:%
%:%8430=2727%:%
%:%8431=2728%:%
%:%8432=2728%:%
%:%8433=2728%:%
%:%8434=2728%:%
%:%8435=2728%:%
%:%8436=2729%:%
%:%8442=2729%:%
%:%8445=2730%:%
%:%8446=2731%:%
%:%8447=2731%:%
%:%8449=2733%:%
%:%8450=2734%:%
%:%8451=2735%:%
%:%8452=2736%:%
%:%8453=2737%:%
%:%8454=2738%:%
%:%8455=2739%:%
%:%8456=2740%:%
%:%8457=2741%:%
%:%8458=2742%:%
%:%8459=2743%:%
%:%8460=2744%:%
%:%8461=2745%:%
%:%8462=2746%:%
%:%8464=2747%:%
%:%8465=2747%:%
%:%8473=2755%:%
%:%8474=2756%:%
%:%8475=2756%:%
%:%8478=2757%:%
%:%8482=2757%:%
%:%8483=2757%:%
%:%8488=2757%:%
%:%8491=2758%:%
%:%8492=2759%:%
%:%8493=2759%:%
%:%8495=2759%:%
%:%8499=2759%:%
%:%8500=2759%:%
%:%8507=2759%:%
%:%8508=2760%:%
%:%8509=2761%:%
%:%8510=2761%:%
%:%8512=2761%:%
%:%8516=2761%:%
%:%8517=2761%:%
%:%8524=2761%:%
%:%8525=2762%:%
%:%8526=2763%:%
%:%8527=2763%:%
%:%8529=2763%:%
%:%8533=2763%:%
%:%8534=2763%:%
%:%8543=2765%:%
%:%8544=2766%:%
%:%8545=2767%:%
%:%8546=2768%:%
%:%8547=2769%:%
%:%8548=2770%:%
%:%8549=2771%:%
%:%8550=2772%:%
%:%8551=2773%:%
%:%8552=2774%:%
%:%8553=2775%:%
%:%8554=2776%:%
%:%8555=2777%:%
%:%8557=2778%:%
%:%8558=2778%:%
%:%8566=2786%:%
%:%8567=2787%:%
%:%8568=2787%:%
%:%8571=2788%:%
%:%8575=2788%:%
%:%8576=2788%:%
%:%8581=2788%:%
%:%8584=2789%:%
%:%8585=2790%:%
%:%8586=2790%:%
%:%8588=2790%:%
%:%8592=2790%:%
%:%8593=2790%:%
%:%8600=2790%:%
%:%8601=2791%:%
%:%8602=2792%:%
%:%8603=2792%:%
%:%8605=2792%:%
%:%8609=2792%:%
%:%8610=2792%:%
%:%8617=2792%:%
%:%8618=2793%:%
%:%8619=2794%:%
%:%8620=2794%:%
%:%8627=2795%:%
%:%8628=2795%:%
%:%8629=2796%:%
%:%8630=2796%:%
%:%8631=2797%:%
%:%8632=2797%:%
%:%8633=2798%:%
%:%8634=2798%:%
%:%8635=2798%:%
%:%8636=2798%:%
%:%8637=2798%:%
%:%8638=2799%:%
%:%8644=2799%:%
%:%8647=2800%:%
%:%8648=2801%:%
%:%8649=2801%:%
%:%8650=2802%:%
%:%8651=2803%:%
%:%8652=2804%:%
%:%8653=2805%:%
%:%8654=2806%:%
%:%8655=2807%:%
%:%8656=2808%:%
%:%8657=2808%:%
%:%8658=2809%:%
%:%8659=2810%:%
%:%8660=2811%:%
%:%8661=2812%:%
%:%8662=2813%:%
%:%8664=2815%:%
%:%8666=2816%:%
%:%8667=2816%:%
%:%8668=2817%:%
%:%8669=2818%:%
%:%8672=2819%:%
%:%8676=2819%:%
%:%8677=2819%:%
%:%8678=2820%:%
%:%8679=2820%:%
%:%8680=2821%:%
%:%8681=2821%:%
%:%8682=2822%:%
%:%8683=2822%:%
%:%8684=2822%:%
%:%8685=2822%:%
%:%8686=2823%:%
%:%8687=2823%:%
%:%8688=2823%:%
%:%8689=2824%:%
%:%8690=2824%:%
%:%8691=2825%:%
%:%8692=2825%:%
%:%8693=2826%:%
%:%8694=2826%:%
%:%8695=2826%:%
%:%8696=2826%:%
%:%8697=2826%:%
%:%8698=2827%:%
%:%8699=2827%:%
%:%8700=2828%:%
%:%8701=2828%:%
%:%8702=2829%:%
%:%8703=2829%:%
%:%8704=2830%:%
%:%8705=2830%:%
%:%8706=2831%:%
%:%8707=2831%:%
%:%8708=2831%:%
%:%8709=2832%:%
%:%8710=2832%:%
%:%8711=2833%:%
%:%8712=2833%:%
%:%8713=2834%:%
%:%8714=2834%:%
%:%8715=2834%:%
%:%8716=2835%:%
%:%8717=2835%:%
%:%8718=2835%:%
%:%8719=2835%:%
%:%8720=2836%:%
%:%8721=2836%:%
%:%8722=2837%:%
%:%8723=2837%:%
%:%8724=2837%:%
%:%8725=2838%:%
%:%8726=2839%:%
%:%8727=2839%:%
%:%8728=2839%:%
%:%8729=2840%:%
%:%8730=2840%:%
%:%8731=2840%:%
%:%8732=2841%:%
%:%8733=2841%:%
%:%8734=2841%:%
%:%8735=2842%:%
%:%8736=2842%:%
%:%8737=2843%:%
%:%8747=2845%:%
%:%8748=2846%:%
%:%8750=2847%:%
%:%8751=2847%:%
%:%8752=2848%:%
%:%8753=2849%:%
%:%8756=2850%:%
%:%8760=2850%:%
%:%8761=2850%:%
%:%8762=2851%:%
%:%8763=2851%:%
%:%8764=2852%:%
%:%8765=2852%:%
%:%8766=2852%:%
%:%8767=2852%:%
%:%8768=2853%:%
%:%8769=2853%:%
%:%8770=2853%:%
%:%8771=2854%:%
%:%8772=2854%:%
%:%8773=2855%:%
%:%8774=2855%:%
%:%8775=2856%:%
%:%8776=2856%:%
%:%8777=2856%:%
%:%8778=2856%:%
%:%8779=2856%:%
%:%8780=2857%:%
%:%8781=2857%:%
%:%8782=2858%:%
%:%8783=2858%:%
%:%8784=2859%:%
%:%8785=2859%:%
%:%8786=2860%:%
%:%8787=2860%:%
%:%8788=2860%:%
%:%8789=2861%:%
%:%8790=2861%:%
%:%8791=2862%:%
%:%8792=2862%:%
%:%8793=2862%:%
%:%8794=2863%:%
%:%8795=2863%:%
%:%8796=2863%:%
%:%8797=2864%:%
%:%8798=2865%:%
%:%8799=2865%:%
%:%8800=2866%:%
%:%8801=2866%:%
%:%8802=2866%:%
%:%8803=2867%:%
%:%8804=2867%:%
%:%8805=2867%:%
%:%8806=2868%:%
%:%8807=2869%:%
%:%8808=2869%:%
%:%8809=2869%:%
%:%8810=2870%:%
%:%8811=2870%:%
%:%8812=2870%:%
%:%8813=2871%:%
%:%8814=2872%:%
%:%8815=2872%:%
%:%8816=2872%:%
%:%8817=2872%:%
%:%8818=2873%:%
%:%8819=2874%:%
%:%8820=2874%:%
%:%8821=2874%:%
%:%8822=2875%:%
%:%8823=2875%:%
%:%8824=2875%:%
%:%8825=2876%:%
%:%8826=2876%:%
%:%8827=2876%:%
%:%8828=2876%:%
%:%8829=2877%:%
%:%8830=2877%:%
%:%8831=2877%:%
%:%8832=2878%:%
%:%8833=2879%:%
%:%8834=2880%:%
%:%8835=2880%:%
%:%8836=2880%:%
%:%8837=2881%:%
%:%8838=2881%:%
%:%8839=2881%:%
%:%8840=2882%:%
%:%8841=2882%:%
%:%8842=2883%:%
%:%8843=2883%:%
%:%8844=2884%:%
%:%8845=2884%:%
%:%8846=2884%:%
%:%8847=2884%:%
%:%8848=2884%:%
%:%8849=2885%:%
%:%8850=2885%:%
%:%8851=2886%:%
%:%8861=2888%:%
%:%8862=2889%:%
%:%8864=2890%:%
%:%8865=2890%:%
%:%8872=2892%:%
%:%8884=2894%:%
%:%8885=2895%:%
%:%8889=2897%:%
%:%8890=2898%:%
%:%8891=2899%:%
%:%8892=2900%:%
%:%8893=2901%:%
%:%8894=2902%:%
%:%8895=2903%:%
%:%8896=2904%:%
%:%8897=2905%:%
%:%8898=2906%:%
%:%8899=2907%:%
%:%8901=2908%:%
%:%8902=2908%:%
%:%8907=2913%:%
%:%8908=2914%:%
%:%8909=2914%:%
%:%8912=2915%:%
%:%8916=2915%:%
%:%8917=2915%:%
%:%8926=2917%:%
%:%8927=2918%:%
%:%8928=2919%:%
%:%8929=2920%:%
%:%8930=2921%:%
%:%8931=2922%:%
%:%8932=2923%:%
%:%8933=2924%:%
%:%8934=2925%:%
%:%8935=2926%:%
%:%8936=2927%:%
%:%8938=2928%:%
%:%8939=2928%:%
%:%8944=2933%:%
%:%8945=2934%:%
%:%8946=2934%:%
%:%8949=2935%:%
%:%8953=2935%:%
%:%8954=2935%:%
%:%8963=2937%:%
%:%8964=2938%:%
%:%8965=2939%:%
%:%8966=2940%:%
%:%8967=2941%:%
%:%8968=2942%:%
%:%8969=2943%:%
%:%8970=2944%:%
%:%8971=2945%:%
%:%8972=2946%:%
%:%8973=2947%:%
%:%8975=2948%:%
%:%8976=2948%:%
%:%8981=2953%:%
%:%8982=2954%:%
%:%8983=2954%:%
%:%8986=2955%:%
%:%8990=2955%:%
%:%8991=2955%:%
%:%8996=2955%:%
%:%8999=2956%:%
%:%9000=2957%:%
%:%9001=2957%:%
%:%9003=2959%:%
%:%9004=2960%:%
%:%9005=2961%:%
%:%9006=2962%:%
%:%9007=2963%:%
%:%9008=2964%:%
%:%9009=2965%:%
%:%9010=2966%:%
%:%9011=2967%:%
%:%9012=2968%:%
%:%9013=2969%:%
%:%9014=2970%:%
%:%9015=2971%:%
%:%9017=2972%:%
%:%9018=2972%:%
%:%9025=2979%:%
%:%9026=2980%:%
%:%9027=2980%:%
%:%9030=2981%:%
%:%9034=2981%:%
%:%9035=2981%:%
%:%9040=2981%:%
%:%9043=2982%:%
%:%9044=2983%:%
%:%9045=2983%:%
%:%9047=2985%:%
%:%9048=2986%:%
%:%9049=2987%:%
%:%9050=2988%:%
%:%9051=2989%:%
%:%9052=2990%:%
%:%9053=2991%:%
%:%9054=2992%:%
%:%9055=2993%:%
%:%9056=2994%:%
%:%9057=2995%:%
%:%9058=2996%:%
%:%9059=2997%:%
%:%9061=2998%:%
%:%9062=2998%:%
%:%9069=3005%:%
%:%9070=3006%:%
%:%9071=3006%:%
%:%9074=3007%:%
%:%9078=3007%:%
%:%9079=3007%:%
%:%9084=3007%:%
%:%9087=3008%:%
%:%9088=3009%:%
%:%9089=3009%:%
%:%9091=3011%:%
%:%9092=3012%:%
%:%9093=3013%:%
%:%9094=3014%:%
%:%9095=3015%:%
%:%9096=3016%:%
%:%9097=3017%:%
%:%9098=3018%:%
%:%9099=3019%:%
%:%9100=3020%:%
%:%9101=3021%:%
%:%9102=3022%:%
%:%9103=3023%:%
%:%9104=3024%:%
%:%9105=3025%:%
%:%9107=3026%:%
%:%9108=3026%:%
%:%9117=3035%:%
%:%9118=3036%:%
%:%9119=3036%:%
%:%9122=3037%:%
%:%9126=3037%:%
%:%9127=3037%:%
%:%9132=3037%:%
%:%9135=3038%:%
%:%9136=3039%:%
%:%9137=3039%:%
%:%9139=3041%:%
%:%9140=3042%:%
%:%9141=3043%:%
%:%9142=3044%:%
%:%9143=3045%:%
%:%9144=3046%:%
%:%9145=3047%:%
%:%9146=3048%:%
%:%9147=3049%:%
%:%9148=3050%:%
%:%9149=3051%:%
%:%9150=3052%:%
%:%9151=3053%:%
%:%9152=3054%:%
%:%9153=3055%:%
%:%9155=3056%:%
%:%9156=3056%:%
%:%9165=3065%:%
%:%9166=3066%:%
%:%9167=3066%:%
%:%9170=3067%:%
%:%9174=3067%:%
%:%9175=3067%:%
%:%9184=3069%:%
%:%9185=3070%:%
%:%9187=3071%:%
%:%9188=3071%:%
%:%9189=3072%:%
%:%9190=3073%:%
%:%9191=3074%:%
%:%9192=3075%:%
%:%9195=3076%:%
%:%9199=3076%:%
%:%9200=3076%:%
%:%9201=3077%:%
%:%9202=3077%:%
%:%9203=3078%:%
%:%9204=3078%:%
%:%9205=3079%:%
%:%9206=3079%:%
%:%9207=3080%:%
%:%9208=3080%:%
%:%9209=3080%:%
%:%9210=3081%:%
%:%9211=3082%:%
%:%9212=3082%:%
%:%9213=3082%:%
%:%9214=3083%:%
%:%9215=3083%:%
%:%9216=3084%:%
%:%9217=3084%:%
%:%9218=3085%:%
%:%9219=3086%:%
%:%9220=3086%:%
%:%9221=3086%:%
%:%9222=3087%:%
%:%9223=3088%:%
%:%9224=3088%:%
%:%9225=3089%:%
%:%9226=3089%:%
%:%9227=3089%:%
%:%9228=3090%:%
%:%9229=3090%:%
%:%9230=3091%:%
%:%9231=3091%:%
%:%9232=3092%:%
%:%9233=3092%:%
%:%9234=3093%:%
%:%9235=3094%:%
%:%9236=3094%:%
%:%9237=3094%:%
%:%9238=3095%:%
%:%9239=3096%:%
%:%9240=3096%:%
%:%9241=3097%:%
%:%9242=3097%:%
%:%9243=3098%:%
%:%9244=3099%:%
%:%9245=3099%:%
%:%9246=3099%:%
%:%9247=3099%:%
%:%9248=3100%:%
%:%9249=3100%:%
%:%9250=3101%:%
%:%9251=3101%:%
%:%9252=3101%:%
%:%9253=3101%:%
%:%9254=3102%:%
%:%9255=3102%:%
%:%9256=3103%:%
%:%9257=3103%:%
%:%9258=3103%:%
%:%9259=3104%:%
%:%9260=3105%:%
%:%9261=3105%:%
%:%9262=3106%:%
%:%9263=3106%:%
%:%9264=3106%:%
%:%9265=3106%:%
%:%9266=3107%:%
%:%9267=3107%:%
%:%9268=3107%:%
%:%9269=3108%:%
%:%9270=3108%:%
%:%9271=3108%:%
%:%9272=3109%:%
%:%9273=3110%:%
%:%9274=3110%:%
%:%9275=3111%:%
%:%9276=3111%:%
%:%9277=3112%:%
%:%9278=3112%:%
%:%9279=3113%:%
%:%9280=3113%:%
%:%9281=3113%:%
%:%9282=3114%:%
%:%9283=3114%:%
%:%9284=3114%:%
%:%9285=3114%:%
%:%9286=3115%:%
%:%9287=3115%:%
%:%9288=3115%:%
%:%9289=3115%:%
%:%9290=3115%:%
%:%9291=3116%:%
%:%9301=3118%:%
%:%9302=3119%:%
%:%9306=3120%:%
%:%9307=3121%:%
%:%9308=3122%:%
%:%9310=3123%:%
%:%9311=3123%:%
%:%9312=3124%:%
%:%9313=3125%:%
%:%9316=3126%:%
%:%9320=3126%:%
%:%9321=3126%:%
%:%9322=3127%:%
%:%9323=3127%:%
%:%9324=3128%:%
%:%9325=3128%:%
%:%9326=3128%:%
%:%9327=3129%:%
%:%9328=3129%:%
%:%9329=3129%:%
%:%9330=3130%:%
%:%9331=3130%:%
%:%9332=3131%:%
%:%9333=3131%:%
%:%9334=3132%:%
%:%9335=3132%:%
%:%9336=3133%:%
%:%9337=3133%:%
%:%9338=3133%:%
%:%9339=3133%:%
%:%9340=3134%:%
%:%9341=3134%:%
%:%9342=3134%:%
%:%9343=3135%:%
%:%9344=3135%:%
%:%9345=3136%:%
%:%9346=3136%:%
%:%9347=3137%:%
%:%9348=3137%:%
%:%9349=3137%:%
%:%9350=3137%:%
%:%9351=3138%:%
%:%9352=3138%:%
%:%9353=3138%:%
%:%9354=3138%:%
%:%9355=3139%:%
%:%9356=3139%:%
%:%9357=3139%:%
%:%9358=3140%:%
%:%9359=3140%:%
%:%9360=3141%:%
%:%9361=3141%:%
%:%9362=3142%:%
%:%9363=3142%:%
%:%9364=3143%:%
%:%9365=3143%:%
%:%9366=3144%:%
%:%9367=3144%:%
%:%9368=3144%:%
%:%9369=3145%:%
%:%9370=3145%:%
%:%9371=3146%:%
%:%9372=3146%:%
%:%9373=3147%:%
%:%9374=3147%:%
%:%9375=3147%:%
%:%9376=3147%:%
%:%9377=3147%:%
%:%9378=3148%:%
%:%9379=3148%:%
%:%9380=3149%:%
%:%9381=3149%:%
%:%9382=3150%:%
%:%9383=3150%:%
%:%9384=3151%:%
%:%9385=3151%:%
%:%9386=3152%:%
%:%9387=3152%:%
%:%9388=3152%:%
%:%9389=3153%:%
%:%9390=3153%:%
%:%9391=3154%:%
%:%9392=3154%:%
%:%9393=3155%:%
%:%9394=3155%:%
%:%9395=3155%:%
%:%9396=3155%:%
%:%9397=3155%:%
%:%9398=3156%:%
%:%9399=3156%:%
%:%9400=3157%:%
%:%9401=3157%:%
%:%9402=3158%:%
%:%9403=3158%:%
%:%9404=3159%:%
%:%9405=3159%:%
%:%9406=3160%:%
%:%9407=3160%:%
%:%9408=3160%:%
%:%9409=3161%:%
%:%9410=3161%:%
%:%9411=3162%:%
%:%9412=3162%:%
%:%9413=3163%:%
%:%9414=3163%:%
%:%9415=3163%:%
%:%9416=3163%:%
%:%9417=3163%:%
%:%9418=3164%:%
%:%9419=3164%:%
%:%9420=3165%:%
%:%9421=3165%:%
%:%9422=3166%:%
%:%9423=3166%:%
%:%9424=3167%:%
%:%9425=3167%:%
%:%9426=3167%:%
%:%9427=3168%:%
%:%9428=3168%:%
%:%9429=3169%:%
%:%9430=3169%:%
%:%9431=3170%:%
%:%9432=3170%:%
%:%9433=3170%:%
%:%9434=3171%:%
%:%9435=3171%:%
%:%9436=3171%:%
%:%9437=3172%:%
%:%9438=3172%:%
%:%9439=3173%:%
%:%9440=3173%:%
%:%9441=3174%:%
%:%9442=3174%:%
%:%9443=3174%:%
%:%9444=3175%:%
%:%9445=3175%:%
%:%9446=3175%:%
%:%9447=3176%:%
%:%9448=3176%:%
%:%9449=3176%:%
%:%9450=3177%:%
%:%9451=3177%:%
%:%9452=3177%:%
%:%9453=3178%:%
%:%9454=3178%:%
%:%9455=3179%:%
%:%9456=3179%:%
%:%9457=3180%:%
%:%9458=3180%:%
%:%9459=3180%:%
%:%9460=3181%:%
%:%9461=3181%:%
%:%9462=3181%:%
%:%9463=3182%:%
%:%9464=3182%:%
%:%9465=3182%:%
%:%9466=3183%:%
%:%9467=3183%:%
%:%9468=3183%:%
%:%9469=3184%:%
%:%9470=3184%:%
%:%9471=3185%:%
%:%9472=3185%:%
%:%9473=3186%:%
%:%9474=3186%:%
%:%9475=3186%:%
%:%9476=3187%:%
%:%9477=3187%:%
%:%9478=3188%:%
%:%9479=3188%:%
%:%9480=3189%:%
%:%9481=3189%:%
%:%9482=3190%:%
%:%9483=3190%:%
%:%9484=3191%:%
%:%9485=3191%:%
%:%9486=3191%:%
%:%9487=3192%:%
%:%9488=3192%:%
%:%9489=3192%:%
%:%9490=3193%:%
%:%9491=3193%:%
%:%9492=3193%:%
%:%9493=3194%:%
%:%9494=3195%:%
%:%9495=3195%:%
%:%9496=3195%:%
%:%9497=3195%:%
%:%9498=3196%:%
%:%9499=3197%:%
%:%9500=3197%:%
%:%9501=3198%:%
%:%9502=3198%:%
%:%9503=3199%:%
%:%9504=3199%:%
%:%9505=3199%:%
%:%9506=3200%:%
%:%9507=3201%:%
%:%9508=3201%:%
%:%9509=3201%:%
%:%9510=3202%:%
%:%9511=3202%:%
%:%9512=3202%:%
%:%9513=3203%:%
%:%9514=3203%:%
%:%9515=3203%:%
%:%9516=3203%:%
%:%9517=3204%:%
%:%9518=3205%:%
%:%9519=3205%:%
%:%9520=3206%:%
%:%9521=3206%:%
%:%9522=3206%:%
%:%9523=3207%:%
%:%9524=3208%:%
%:%9525=3208%:%
%:%9526=3209%:%
%:%9527=3209%:%
%:%9528=3210%:%
%:%9529=3210%:%
%:%9530=3211%:%
%:%9531=3211%:%
%:%9532=3211%:%
%:%9533=3212%:%
%:%9534=3212%:%
%:%9535=3212%:%
%:%9536=3213%:%
%:%9537=3213%:%
%:%9538=3214%:%
%:%9539=3214%:%
%:%9540=3215%:%
%:%9541=3215%:%
%:%9542=3216%:%
%:%9543=3216%:%
%:%9544=3216%:%
%:%9545=3217%:%
%:%9546=3218%:%
%:%9547=3218%:%
%:%9548=3218%:%
%:%9549=3219%:%
%:%9550=3219%:%
%:%9551=3220%:%
%:%9552=3220%:%
%:%9553=3221%:%
%:%9554=3221%:%
%:%9555=3221%:%
%:%9556=3222%:%
%:%9557=3222%:%
%:%9558=3222%:%
%:%9559=3222%:%
%:%9560=3223%:%
%:%9561=3223%:%
%:%9562=3224%:%
%:%9563=3224%:%
%:%9564=3225%:%
%:%9565=3225%:%
%:%9566=3225%:%
%:%9567=3225%:%
%:%9568=3226%:%
%:%9569=3226%:%
%:%9570=3226%:%
%:%9571=3226%:%
%:%9572=3227%:%
%:%9573=3227%:%
%:%9574=3227%:%
%:%9575=3228%:%
%:%9576=3228%:%
%:%9577=3229%:%
%:%9578=3229%:%
%:%9579=3230%:%
%:%9580=3230%:%
%:%9581=3231%:%
%:%9582=3231%:%
%:%9583=3232%:%
%:%9584=3232%:%
%:%9585=3232%:%
%:%9586=3233%:%
%:%9587=3233%:%
%:%9588=3234%:%
%:%9589=3234%:%
%:%9590=3235%:%
%:%9591=3235%:%
%:%9592=3235%:%
%:%9593=3235%:%
%:%9594=3235%:%
%:%9595=3236%:%
%:%9596=3236%:%
%:%9597=3237%:%
%:%9598=3237%:%
%:%9599=3238%:%
%:%9600=3238%:%
%:%9601=3239%:%
%:%9602=3239%:%
%:%9603=3240%:%
%:%9604=3240%:%
%:%9605=3240%:%
%:%9606=3241%:%
%:%9607=3241%:%
%:%9608=3242%:%
%:%9609=3242%:%
%:%9610=3243%:%
%:%9611=3243%:%
%:%9612=3243%:%
%:%9613=3243%:%
%:%9614=3243%:%
%:%9615=3244%:%
%:%9616=3244%:%
%:%9617=3245%:%
%:%9618=3245%:%
%:%9619=3246%:%
%:%9620=3246%:%
%:%9621=3247%:%
%:%9622=3247%:%
%:%9623=3248%:%
%:%9624=3248%:%
%:%9625=3248%:%
%:%9626=3249%:%
%:%9627=3249%:%
%:%9628=3250%:%
%:%9629=3250%:%
%:%9630=3251%:%
%:%9631=3251%:%
%:%9632=3251%:%
%:%9633=3251%:%
%:%9634=3251%:%
%:%9635=3252%:%
%:%9636=3252%:%
%:%9637=3253%:%
%:%9638=3253%:%
%:%9639=3254%:%
%:%9640=3254%:%
%:%9641=3255%:%
%:%9642=3255%:%
%:%9643=3256%:%
%:%9644=3256%:%
%:%9645=3256%:%
%:%9646=3257%:%
%:%9647=3257%:%
%:%9648=3258%:%
%:%9649=3258%:%
%:%9650=3259%:%
%:%9651=3259%:%
%:%9652=3259%:%
%:%9653=3260%:%
%:%9654=3260%:%
%:%9655=3260%:%
%:%9656=3261%:%
%:%9657=3261%:%
%:%9658=3262%:%
%:%9659=3262%:%
%:%9660=3263%:%
%:%9661=3263%:%
%:%9662=3263%:%
%:%9663=3264%:%
%:%9664=3264%:%
%:%9665=3264%:%
%:%9666=3265%:%
%:%9667=3265%:%
%:%9668=3265%:%
%:%9669=3266%:%
%:%9670=3266%:%
%:%9671=3266%:%
%:%9672=3267%:%
%:%9673=3267%:%
%:%9674=3268%:%
%:%9675=3268%:%
%:%9676=3269%:%
%:%9677=3269%:%
%:%9678=3269%:%
%:%9679=3270%:%
%:%9680=3270%:%
%:%9681=3270%:%
%:%9682=3271%:%
%:%9683=3271%:%
%:%9684=3271%:%
%:%9685=3272%:%
%:%9686=3272%:%
%:%9687=3272%:%
%:%9688=3273%:%
%:%9689=3273%:%
%:%9690=3274%:%
%:%9691=3274%:%
%:%9692=3275%:%
%:%9693=3275%:%
%:%9694=3275%:%
%:%9695=3276%:%
%:%9696=3276%:%
%:%9697=3277%:%
%:%9698=3277%:%
%:%9699=3278%:%
%:%9700=3278%:%
%:%9701=3279%:%
%:%9702=3279%:%
%:%9703=3280%:%
%:%9704=3280%:%
%:%9705=3280%:%
%:%9706=3281%:%
%:%9707=3282%:%
%:%9708=3282%:%
%:%9709=3282%:%
%:%9710=3283%:%
%:%9711=3283%:%
%:%9712=3284%:%
%:%9713=3284%:%
%:%9714=3285%:%
%:%9715=3285%:%
%:%9716=3286%:%
%:%9717=3286%:%
%:%9718=3286%:%
%:%9719=3287%:%
%:%9720=3288%:%
%:%9721=3288%:%
%:%9722=3289%:%
%:%9723=3289%:%
%:%9724=3290%:%
%:%9725=3290%:%
%:%9726=3290%:%
%:%9727=3291%:%
%:%9728=3292%:%
%:%9729=3292%:%
%:%9730=3292%:%
%:%9731=3293%:%
%:%9732=3293%:%
%:%9733=3293%:%
%:%9734=3294%:%
%:%9735=3294%:%
%:%9736=3294%:%
%:%9737=3294%:%
%:%9738=3295%:%
%:%9739=3296%:%
%:%9740=3296%:%
%:%9741=3297%:%
%:%9742=3297%:%
%:%9743=3297%:%
%:%9744=3298%:%
%:%9745=3299%:%
%:%9746=3299%:%
%:%9747=3300%:%
%:%9748=3300%:%
%:%9749=3301%:%
%:%9750=3301%:%
%:%9751=3302%:%
%:%9752=3302%:%
%:%9753=3302%:%
%:%9754=3303%:%
%:%9755=3303%:%
%:%9756=3303%:%
%:%9757=3304%:%
%:%9758=3304%:%
%:%9759=3305%:%
%:%9760=3305%:%
%:%9761=3306%:%
%:%9762=3306%:%
%:%9763=3307%:%
%:%9764=3307%:%
%:%9765=3307%:%
%:%9766=3308%:%
%:%9767=3309%:%
%:%9768=3309%:%
%:%9769=3309%:%
%:%9770=3310%:%
%:%9771=3310%:%
%:%9772=3311%:%
%:%9773=3311%:%
%:%9774=3312%:%
%:%9775=3312%:%
%:%9776=3312%:%
%:%9777=3313%:%
%:%9778=3313%:%
%:%9779=3313%:%
%:%9780=3313%:%
%:%9781=3314%:%
%:%9782=3314%:%
%:%9783=3315%:%
%:%9784=3315%:%
%:%9785=3316%:%
%:%9786=3316%:%
%:%9787=3316%:%
%:%9788=3316%:%
%:%9789=3317%:%
%:%9790=3317%:%
%:%9791=3317%:%
%:%9792=3317%:%
%:%9793=3318%:%
%:%9794=3318%:%
%:%9795=3318%:%
%:%9796=3319%:%
%:%9797=3319%:%
%:%9798=3320%:%
%:%9799=3320%:%
%:%9800=3321%:%
%:%9801=3321%:%
%:%9802=3322%:%
%:%9803=3322%:%
%:%9804=3323%:%
%:%9805=3323%:%
%:%9806=3323%:%
%:%9807=3324%:%
%:%9808=3324%:%
%:%9809=3325%:%
%:%9810=3325%:%
%:%9811=3326%:%
%:%9812=3326%:%
%:%9813=3326%:%
%:%9814=3326%:%
%:%9815=3326%:%
%:%9816=3327%:%
%:%9817=3327%:%
%:%9818=3328%:%
%:%9819=3328%:%
%:%9820=3329%:%
%:%9821=3329%:%
%:%9822=3330%:%
%:%9823=3330%:%
%:%9824=3331%:%
%:%9825=3331%:%
%:%9826=3331%:%
%:%9827=3332%:%
%:%9828=3332%:%
%:%9829=3333%:%
%:%9830=3333%:%
%:%9831=3334%:%
%:%9832=3334%:%
%:%9833=3334%:%
%:%9834=3334%:%
%:%9835=3334%:%
%:%9836=3335%:%
%:%9837=3335%:%
%:%9838=3336%:%
%:%9839=3336%:%
%:%9840=3337%:%
%:%9841=3337%:%
%:%9842=3338%:%
%:%9843=3338%:%
%:%9844=3339%:%
%:%9845=3339%:%
%:%9846=3339%:%
%:%9847=3340%:%
%:%9848=3340%:%
%:%9849=3341%:%
%:%9850=3341%:%
%:%9851=3342%:%
%:%9852=3342%:%
%:%9853=3342%:%
%:%9854=3342%:%
%:%9855=3342%:%
%:%9856=3343%:%
%:%9857=3343%:%
%:%9858=3344%:%
%:%9859=3344%:%
%:%9860=3345%:%
%:%9861=3345%:%
%:%9862=3346%:%
%:%9863=3346%:%
%:%9864=3347%:%
%:%9865=3347%:%
%:%9866=3347%:%
%:%9867=3348%:%
%:%9868=3348%:%
%:%9869=3349%:%
%:%9870=3349%:%
%:%9871=3350%:%
%:%9872=3350%:%
%:%9873=3350%:%
%:%9874=3351%:%
%:%9875=3351%:%
%:%9876=3351%:%
%:%9877=3352%:%
%:%9878=3352%:%
%:%9879=3353%:%
%:%9880=3353%:%
%:%9881=3354%:%
%:%9882=3354%:%
%:%9883=3354%:%
%:%9884=3355%:%
%:%9885=3355%:%
%:%9886=3355%:%
%:%9887=3356%:%
%:%9888=3356%:%
%:%9889=3356%:%
%:%9890=3357%:%
%:%9891=3357%:%
%:%9892=3357%:%
%:%9893=3358%:%
%:%9894=3358%:%
%:%9895=3359%:%
%:%9896=3359%:%
%:%9897=3360%:%
%:%9898=3360%:%
%:%9899=3360%:%
%:%9900=3361%:%
%:%9901=3361%:%
%:%9902=3361%:%
%:%9903=3362%:%
%:%9904=3362%:%
%:%9905=3362%:%
%:%9906=3363%:%
%:%9907=3363%:%
%:%9908=3363%:%
%:%9909=3364%:%
%:%9910=3364%:%
%:%9911=3365%:%
%:%9912=3365%:%
%:%9913=3366%:%
%:%9914=3366%:%
%:%9915=3366%:%
%:%9916=3367%:%
%:%9917=3367%:%
%:%9918=3368%:%
%:%9919=3368%:%
%:%9920=3369%:%
%:%9921=3369%:%
%:%9922=3370%:%
%:%9923=3370%:%
%:%9924=3371%:%
%:%9925=3371%:%
%:%9926=3371%:%
%:%9927=3372%:%
%:%9928=3373%:%
%:%9929=3373%:%
%:%9930=3373%:%
%:%9931=3374%:%
%:%9932=3374%:%
%:%9933=3375%:%
%:%9934=3375%:%
%:%9935=3376%:%
%:%9936=3376%:%
%:%9937=3377%:%
%:%9938=3378%:%
%:%9939=3378%:%
%:%9940=3378%:%
%:%9941=3379%:%
%:%9942=3380%:%
%:%9943=3380%:%
%:%9944=3381%:%
%:%9945=3381%:%
%:%9946=3382%:%
%:%9947=3382%:%
%:%9948=3382%:%
%:%9949=3383%:%
%:%9950=3384%:%
%:%9951=3384%:%
%:%9952=3384%:%
%:%9953=3385%:%
%:%9954=3385%:%
%:%9955=3385%:%
%:%9956=3386%:%
%:%9957=3386%:%
%:%9958=3386%:%
%:%9959=3386%:%
%:%9960=3387%:%
%:%9961=3388%:%
%:%9962=3388%:%
%:%9963=3389%:%
%:%9964=3389%:%
%:%9965=3389%:%
%:%9966=3390%:%
%:%9967=3391%:%
%:%9968=3391%:%
%:%9969=3392%:%
%:%9970=3392%:%
%:%9971=3393%:%
%:%9972=3393%:%
%:%9973=3394%:%
%:%9974=3394%:%
%:%9975=3394%:%
%:%9976=3395%:%
%:%9977=3395%:%
%:%9978=3395%:%
%:%9979=3396%:%
%:%9980=3396%:%
%:%9981=3397%:%
%:%9982=3397%:%
%:%9983=3398%:%
%:%9984=3398%:%
%:%9985=3399%:%
%:%9986=3399%:%
%:%9987=3400%:%
%:%9988=3400%:%
%:%9989=3401%:%
%:%9990=3401%:%
%:%9991=3401%:%
%:%9992=3402%:%
%:%9993=3402%:%
%:%9994=3402%:%
%:%9995=3403%:%
%:%9996=3403%:%
%:%9997=3403%:%
%:%9998=3404%:%
%:%9999=3405%:%
%:%10000=3406%:%
%:%10001=3406%:%
%:%10002=3406%:%
%:%10003=3407%:%
%:%10004=3407%:%
%:%10005=3408%:%
%:%10006=3408%:%
%:%10007=3409%:%
%:%10008=3409%:%
%:%10009=3409%:%
%:%10010=3409%:%
%:%10011=3410%:%
%:%10012=3410%:%
%:%10013=3410%:%
%:%10014=3411%:%
%:%10015=3411%:%
%:%10016=3412%:%
%:%10017=3412%:%
%:%10018=3413%:%
%:%10019=3413%:%
%:%10020=3413%:%
%:%10021=3413%:%
%:%10022=3414%:%
%:%10023=3414%:%
%:%10024=3415%:%
%:%10025=3415%:%
%:%10026=3416%:%
%:%10036=3418%:%
%:%10038=3419%:%
%:%10039=3419%:%
%:%10040=3420%:%
%:%10041=3421%:%
%:%10044=3422%:%
%:%10048=3422%:%
%:%10049=3422%:%
%:%10050=3423%:%
%:%10051=3423%:%
%:%10052=3424%:%
%:%10053=3424%:%
%:%10054=3425%:%
%:%10055=3426%:%
%:%10056=3426%:%
%:%10057=3426%:%
%:%10058=3427%:%
%:%10059=3427%:%
%:%10060=3428%:%
%:%10061=3428%:%
%:%10062=3429%:%
%:%10063=3429%:%
%:%10064=3430%:%
%:%10065=3431%:%
%:%10066=3431%:%
%:%10067=3431%:%
%:%10068=3432%:%
%:%10069=3433%:%
%:%10070=3433%:%
%:%10071=3434%:%
%:%10072=3434%:%
%:%10073=3434%:%
%:%10074=3434%:%
%:%10075=3434%:%
%:%10076=3435%:%
%:%10077=3435%:%
%:%10078=3435%:%
%:%10079=3436%:%
%:%10080=3437%:%
%:%10081=3437%:%
%:%10082=3437%:%
%:%10083=3438%:%
%:%10084=3438%:%
%:%10085=3439%:%
%:%10086=3439%:%
%:%10087=3440%:%
%:%10088=3441%:%
%:%10089=3441%:%
%:%10090=3441%:%
%:%10091=3442%:%
%:%10092=3443%:%
%:%10093=3443%:%
%:%10094=3444%:%
%:%10095=3444%:%
%:%10096=3444%:%
%:%10097=3445%:%
%:%10098=3446%:%
%:%10099=3446%:%
%:%10100=3447%:%
%:%10101=3447%:%
%:%10102=3447%:%
%:%10103=3448%:%
%:%10104=3449%:%
%:%10105=3449%:%
%:%10106=3450%:%
%:%10107=3450%:%
%:%10108=3450%:%
%:%10109=3450%:%
%:%10110=3451%:%
%:%10111=3452%:%
%:%10112=3452%:%
%:%10113=3453%:%
%:%10114=3453%:%
%:%10115=3454%:%
%:%10116=3454%:%
%:%10117=3454%:%
%:%10118=3455%:%
%:%10119=3455%:%
%:%10120=3455%:%
%:%10121=3456%:%
%:%10122=3456%:%
%:%10123=3456%:%
%:%10124=3457%:%
%:%10125=3458%:%
%:%10126=3458%:%
%:%10127=3459%:%
%:%10128=3459%:%
%:%10129=3459%:%
%:%10130=3460%:%
%:%10131=3460%:%
%:%10132=3460%:%
%:%10133=3461%:%
%:%10134=3461%:%
%:%10135=3461%:%
%:%10136=3461%:%
%:%10137=3462%:%
%:%10147=3464%:%
%:%10148=3465%:%
%:%10150=3466%:%
%:%10151=3466%:%
%:%10152=3467%:%
%:%10153=3468%:%
%:%10156=3469%:%
%:%10160=3469%:%
%:%10161=3469%:%
%:%10162=3470%:%
%:%10163=3470%:%
%:%10164=3471%:%
%:%10165=3471%:%
%:%10166=3472%:%
%:%10167=3472%:%
%:%10168=3472%:%
%:%10169=3473%:%
%:%10170=3473%:%
%:%10171=3473%:%
%:%10172=3474%:%
%:%10173=3474%:%
%:%10174=3475%:%
%:%10175=3475%:%
%:%10176=3476%:%
%:%10177=3476%:%
%:%10178=3476%:%
%:%10179=3477%:%
%:%10180=3477%:%
%:%10181=3477%:%
%:%10182=3478%:%
%:%10183=3478%:%
%:%10184=3479%:%
%:%10185=3479%:%
%:%10186=3480%:%
%:%10187=3480%:%
%:%10188=3480%:%
%:%10189=3481%:%
%:%10190=3481%:%
%:%10191=3481%:%
%:%10192=3482%:%
%:%10193=3482%:%
%:%10194=3482%:%
%:%10195=3483%:%
%:%10196=3483%:%
%:%10197=3483%:%
%:%10198=3484%:%
%:%10199=3484%:%
%:%10200=3485%:%
%:%10201=3485%:%
%:%10202=3486%:%
%:%10203=3486%:%
%:%10204=3486%:%
%:%10205=3487%:%
%:%10206=3487%:%
%:%10207=3487%:%
%:%10208=3488%:%
%:%10209=3488%:%
%:%10210=3489%:%
%:%10211=3489%:%
%:%10212=3490%:%
%:%10213=3490%:%
%:%10214=3490%:%
%:%10215=3491%:%
%:%10216=3491%:%
%:%10217=3491%:%
%:%10218=3492%:%
%:%10219=3492%:%
%:%10220=3492%:%
%:%10221=3493%:%
%:%10222=3493%:%
%:%10223=3493%:%
%:%10224=3494%:%
%:%10225=3494%:%
%:%10226=3495%:%
%:%10227=3495%:%
%:%10228=3496%:%
%:%10229=3496%:%
%:%10230=3496%:%
%:%10231=3497%:%
%:%10232=3497%:%
%:%10233=3497%:%
%:%10234=3498%:%
%:%10235=3498%:%
%:%10236=3499%:%
%:%10237=3499%:%
%:%10238=3500%:%
%:%10239=3500%:%
%:%10240=3500%:%
%:%10241=3501%:%
%:%10242=3501%:%
%:%10243=3501%:%
%:%10244=3502%:%
%:%10245=3502%:%
%:%10246=3502%:%
%:%10247=3503%:%
%:%10248=3503%:%
%:%10249=3503%:%
%:%10250=3504%:%
%:%10251=3504%:%
%:%10252=3505%:%
%:%10262=3507%:%
%:%10264=3508%:%
%:%10265=3508%:%
%:%10266=3509%:%
%:%10267=3510%:%
%:%10270=3511%:%
%:%10274=3511%:%
%:%10275=3511%:%
%:%10276=3512%:%
%:%10277=3512%:%
%:%10278=3513%:%
%:%10279=3513%:%
%:%10280=3513%:%
%:%10281=3514%:%
%:%10282=3514%:%
%:%10283=3514%:%
%:%10284=3515%:%
%:%10285=3515%:%
%:%10286=3516%:%
%:%10287=3516%:%
%:%10288=3517%:%
%:%10289=3517%:%
%:%10290=3517%:%
%:%10291=3518%:%
%:%10292=3518%:%
%:%10293=3518%:%
%:%10294=3519%:%
%:%10295=3519%:%
%:%10296=3520%:%
%:%10297=3520%:%
%:%10298=3521%:%
%:%10299=3521%:%
%:%10300=3521%:%
%:%10301=3522%:%
%:%10302=3522%:%
%:%10303=3522%:%
%:%10304=3523%:%
%:%10305=3523%:%
%:%10306=3523%:%
%:%10307=3524%:%
%:%10308=3524%:%
%:%10309=3524%:%
%:%10310=3525%:%
%:%10311=3525%:%
%:%10312=3526%:%
%:%10313=3526%:%
%:%10314=3527%:%
%:%10315=3527%:%
%:%10316=3527%:%
%:%10317=3528%:%
%:%10318=3528%:%
%:%10319=3528%:%
%:%10320=3529%:%
%:%10321=3529%:%
%:%10322=3530%:%
%:%10332=3532%:%
%:%10340=3534%:%



% optional bibliography
\bibliographystyle{abbrv}
\bibliography{root}

\end{document}

%%% Local Variables:
%%% mode: latex
%%% TeX-master: t
%%% End:
