\documentclass[11pt,a4paper]{article}
\usepackage[T1]{fontenc}
\usepackage{isabelle,isabellesym}

% further packages required for unusual symbols (see also
% isabellesym.sty), use only when needed

%\usepackage{amssymb}
  %for \<leadsto>, \<box>, \<diamond>, \<sqsupset>, \<mho>, \<Join>,
  %\<lhd>, \<lesssim>, \<greatersim>, \<lessapprox>, \<greaterapprox>,
  %\<triangleq>, \<yen>, \<lozenge>

%\usepackage{eurosym}
  %for \<euro>

%\usepackage[only,bigsqcap,bigparallel,fatsemi,interleave,sslash]{stmaryrd}
  %for \<Sqinter>, \<Parallel>, \<Zsemi>, \<Parallel>, \<sslash>

%\usepackage{eufrak}
  %for \<AA> ... \<ZZ>, \<aa> ... \<zz> (also included in amssymb)

%\usepackage{textcomp}
  %for \<onequarter>, \<onehalf>, \<threequarters>, \<degree>, \<cent>,
  %\<currency>

\usepackage{amsmath}
\usepackage{float}

% this should be the last package used
\usepackage{pdfsetup}

% urls in roman style, theory text in math-similar italics
\urlstyle{rm}
\isabellestyle{it}

% for uniform font size
%\renewcommand{\isastyle}{\isastyleminor}

\begin{document}

\title{Formalization of "Knight's Tour Revisited"}
\author{Lukas Koller}
\maketitle

\begin{abstract}
This is a formalization of \cite{cull_decurtins_1987}. In \cite{cull_decurtins_1987}
the existence of Knight's paths and Knight's circuits are proved for arbitrary $n\times m$-boards with 
$\operatorname{min}(n,m) \geq 5$ and for the Knight's circuit $n\cdot m$ is even.

A Knight's path is a sequence of squares on a chessboard s.t. every step in sequence is a valid move for 
a Knight. A Knight is a chess figure that is only able to move two squares vertically and one square 
horizontally or two squares horizontally and one square vertically. Finding a Knight's path is an 
instance of the Hamiltonian Path Problem. A Knight's circuit is a Knight's path, where additionally 
the Knight can move from the last square to the first square of the path, forming a loop.

The main idea for the proof of the existence of a Knight's path on a $n\times m$-board, is to inductivly 
construct a Knight's paths for the $n\times m$-board from a few pre-computed Knight's paths for small 
boards, i.e. $5\times 5$, $5\times 6$, ..., $8\times 9$. The paths for small boards are transformed 
(i.e. transpose, mirror, translate) and concatenated to create paths for larger boards.

While formalizing the proofs I have noticed two mistakes in the original proof in \cite{cull_decurtins_1987}: 
(i) the pre-computed path for the $6\times 6$-board that ends in the upper-left (in Figure 2) and (ii) 
the pre-computed path for the $8\times 8$-board that ends in the upper-left (in Figure 5) are incorrect. 
I.e. on the $6\times 6$-board the Knight cannot step from square 26 to square 27; in the $8\times 8$-board 
the Knight cannot step from square 27 to square 28. In this formalization I have replaced the two incorrect 
paths with correct paths.
\end{abstract}

\tableofcontents

% sane default for proof documents
\parindent 0pt\parskip 0.5ex

% generated text of all theories
\input{session}

% optional bibliography
\bibliographystyle{abbrv}
\bibliography{root}

\end{document}

%%% Local Variables:
%%% mode: latex
%%% TeX-master: t
%%% End:
